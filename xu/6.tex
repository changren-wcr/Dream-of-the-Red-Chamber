\section*{狱庙相逢}
1、羁候所门外(晚)\par
凄风苦雨。一盏风灯在布满铁钉的门上晃动。门楼上,“羁候所”三个大字隐约可辨。门内不时传出一阵阵哭声。
字幕(叠):第二十六集:狱庙相逢

2、羁候所院内\par
牢头老三和一个狱卒提着风灯快步走过一排排牢房。

3、狱神庙门外\par
老三和狱卒在门前停步。昏暗的灯光里,依稀可见匾额上湿漉漉的三个大字:狱神庙。

4、狱神庙内\par
阴森森的三间庙堂,两边用木槛隔成了临时牢房。一灯如豆,把狱神塑像的影子长长地投在墙上。
“哗啦”一声,庙门大开,老三和狱卒跨进,两边的牢房里一阵骚动。狱卒打开右侧牢门。
左侧牢内,一双双惊恐的眼睛从木槛缝隙里往外窥看。“贾宝玉!”老三一声断喝。右侧牢内,蓬头垢面的贾宝玉往墙角草堆里塞了个什么,缓缓站起。
贾兰、贾菌等坐在靠墙的一溜草荐上,惊惶地看着。“交出来!”老三伸出一只手。宝玉:“什么?”狱卒:“私留的东西!”
老三瞟了一眼蹲在地上的贾环。贾环点了点头。宝玉冷冷地:“不是连汗巾子都收去了吗?”“啪”地一个耳光打在宝玉脸上,老三破口喝骂:“放你娘的屁!交出来!”
狱卒:“你当还在你们府里哪!”贾兰慢慢蹭过来:“你们到底要什么?”狱卒“啪”地一脚踹在贾兰身上:“滚过去!”
老三一字一顿地:“玻璃绣球灯!交出来!”宝玉吃惊地睁大了眼睛。狱卒斜睨着贾环。贾环朝墙角努了努嘴。
狱卒一把拉开宝玉,朝墙角俯下身去。宝玉不顾一切地扑过来,被老三死死拖住。狱卒从草堆里摸出玻璃绣球灯,冷笑一声递给老三。
老三使劲把宝玉搡在地下,转身对贾环:“哎,你!”说着从怀里掏出一个饼子扔在地上。贾环抓起饼子大口啃着。
牢门“咣”地一声带上了。宝玉扑在门上,双手紧紧抓住木槛。“二叔!”贾兰走过来,轻轻啜泣着。

5、檐下\par
雨丝如注。

6、城门内(春)\par
夕阳里,一乘小轿缓缓而来。一个干瘦的中年男子扶着轿杆走在一侧,眼角的余光里微透着几分紧张。

7、城门外闹市\par
小轿穿出城门洞。各种嘈杂的音响里,高亢悠扬的卖花市此伏彼起。轿内,小红被紧紧捆在座上,嘴里塞着东西。
小轿被拥挤的人群阻住。瘦子微微皱了皱眉,顺着人群翘首张望的方向看过去。城墙下搭起一座高台,数名军校荷戈环立高台四周。
台下两侧搭着席棚,棚内隐隐传出哭声。一阵锣响,台上数十名披头散发的丫鬟、婆子被押下去,几个男仆从另一侧被押上高台。
拥挤的人群一阵骚动:“快看快看,林之孝!荣国府的管家!”“哪一个是?”“那个,中间儿站着的!”
“啧啧,常日里只见他们买人卖人,谁承想今儿也尝着让人卖的滋味儿了!”轿内,小红惊异地听着。
小轿在人群中缓缓移动。遮挡轿窗的帘子被小红使劲蹭开了一条缝,两只俊秀的眼睛满含着泪水仓惶地朝着台上窥看。
瘦子下意识地朝轿窗瞥了一眼,急忙伸手拉下窗帘,拼命分开人群,推着小轿朝外挤去。

8、小巷(晚)\par
黑暗中,两个身影一前一后,匆匆走来。

9、马贩子王短腿家院内\par
狭窄的天井里,胡乱堆放着鞍鞯、笼头、缰绳、马鞭等什物。正房檐下,停放着一乘小轿。

10、正房内\par
炕桌上布放着杯盏、酒壶和一碟碟小菜。王短腿、瘦子并两个轿夫围桌而坐。老三焦躁地来回走动着。瘦子笑嘻嘻地看着王短腿:“……哎,怎么样?”
王短腿皱着眉:“抬走!我不找这个麻烦!”瘦子:“啧,怎么也得先看看货!”王短腿:“不看!抬走!”
瘦子:“你不是说话就去西边儿贩马么?顺手捎在马车里,只要离开京城。就没人认得出了……”
王短腿:“你也真敢揽这个生意,也不看看,这是什么人家!”老三停步,朝着王短腿:“怎么还不来?”
王短腿:“急什么?不来咱们就先喝着。”老三:“我后半夜还当值呢!”瘦子看着王短腿:“便宜呀!……咱们不是外人,我只本钱,给一百两银子,这生意让你了!”
王短腿冷笑一声:“便宜?按《逃人法》,这可是杀头的罪!”

11、小巷深处\par
黑暗中,两个身影在一扇门前停步。其中一人上前叩门。

12、王短腿家正房内\par
叩门声传来。老三对王短腿:“来了!”王短腿下炕,急步走出。

13、院内\par
王短腿拔开门闩,笑嘻嘻地一抱拳:“倪二哥!”倪二进门:“怎么这么早就插门?”王短腿瞥了一眼倪二身后的贾芸:“……屋里说吧。”
倪二、贾芸穿过天井。檐下小轿微微晃动了一下。贾芸惊异地拉了拉倪二。倪二停步:“嗯?”王短腿忙跟上来:“请,请,屋里请!”

14、正房内\par
王短腿让着倪二、贾芸进门。老三迎上来:“二哥,怎么这早晚儿才来?”瘦子并两个轿夫立起迎上:“倪二哥!”
倪二:“你怎么也在这儿?”瘦子嘿嘿一笑:“二哥,我手头碰上个生意,你来得正好,给掂掂盘子!”
倪二一摆手:“先不谈你那些缺德生意。来,认识认识!……这就是我常说的芸二爷!”众人急忙抱拳:“芸二爷!芸二爷……”
倪二指着众人一一介绍:“这位朋友是贩马的,人称……”王短腿:“马贩子王短腿!”众笑。贾芸拱手:“幸会!”
倪二:“这位是羁候所的牢头、我的桃园兄弟老三!”贾芸拱手:“幸会!”倪二:“这位是九城闻名的人牙子,人送雅号‘鬼难拿’!”
瘦子笑嘻嘻地指指背后:“这是我的两个兄弟。”贾芸拱手:“幸会幸会!”王短腿:“芸二爷,地方窄扁点儿,请上坐!”
贾芸对倪二:“老二,这……”倪二:“都是朋友,不外的,坐吧!”众人依次落座。王短腿端起酒杯:“来来来!”
倪二:“慢着!先说正事儿。老三!”老三看了看贾芸:“芸二爷,贵本家西府的赦老爷、政老爷、琏二爷和东府的珍大爷还押在刑部牢里,余下的都在羁候所。
前儿有旨意让发卖,这不,白天押到南门外开市,傍晚儿押回羁候所。……价码标得倒是不高,可一看是这两府的人,谁敢买呀?”
王短腿斜了一眼瘦子,瘦子略有些尴尬地笑笑。老三:“听说今儿好不容易卖出去一拨儿,都是些上了年纪的下人。”
王短腿:“主子也卖么?”老三:“看什么样的主子了。象宝二爷、琏二奶奶这号主子,还有身上有些个事儿的,自己想让卖出来怕还没那个福份呢!
……你想想,两府人丁加起来有六、七百号,羁候所笼共只有多大?连狱神庙都关满了!一天到晓,哭的、喊的,上吊的、撞墙的……”
倪二:“我让你查的那个丫头……”老三看着倪二,摇了摇头。贾芸急切地:“都查遍了?”老三端起酒杯,抿了口酒。
贾芸紧盯着老三。老三:“我是挨着个儿扒拉的。”贾芸看了看倪二,轻轻叹了口气。瘦子瞟着倪二:“二哥,我手里倒是碰上个绝色的丫头,听说还是……”
倪二摆摆手,转脸对老三:“老三,你再想法儿查查。……还有件事儿。”老三:“嗯?”倪二:“你今儿后半夜当值?”
老三点点头。倪二往前探了探身子,刚要说什么。门外“啪”的一声,众人一齐朝门口看去。两个轿夫急忙站起来走出房门,瘦子随后跟出。
贾芸从敞开的门缝里疑问地看着檐下的小轿。

15、院内\par
瘦子抓起轿帘往里看了看,轻声骂了一句。瘦子回头朝屋内看看,倪二正小声跟老三说着什么。瘦子并两个轿夫朝房门走来。

16、正房内\par
老三面有难色:“这……”倪二眼一瞪:“什么这,那的!老三,我可是朝芸二爷拍过胸脯了!”老三一拍桌子:“好吧!既然二哥应下了,我不能让二哥没脸!”
贾芸端起酒杯:“三哥快人快语,我敬你一杯!”王短腿:“江湖上讲的就是个义字当先!来!咱们一块儿干了!”
瘦子并两个轿夫入座,随倪二、贾芸、老三端起酒杯。

17、院内\par
夜空里浙浙沥沥的飘起了小雨。细细的雨丝薄在小轿顶上,发出轻微的“扑扑”声。

18、正房内\par
王短腿拉开房门,一愣:“呸!这天儿!老三、二爷,等等!”说着扭头走进里间。老三:“下啦?……好!下雨天儿没人出来!”
贾芸怔怔地看着门外的小轿。王短腿拎着两件蓑衣和两顶竹笠从里间走出:“二爷,委屈了点儿。”贾芸:“哦……”
倪二:“老三,二爷可交给你了!”老三:“二哥放心!”

19、院内\par
老三,贾芸匆匆走出院门。王短腿插上门,转身指着小轿:“快把人解下来!”两个轿夫应声走向小轿,把小红从轿内拖出。

20、正房内\par
轿夫抬着小红进门。王短腿:“呸!捆成这样儿!”说着转身走进里间。瘦子解下捆绑小红的绳子,回头冲着倪二笑笑:“二哥,看看货怎么样?”
小红闭着眼睛,轻轻地哼了一声。瘦子拽出塞在小红嘴里的东西。倪二坐在炕沿上招招手:“抬过来!”两个轿夫把小红抬到坑上。
小红靠墙坐着,慢慢睁开眼睛,呆呆地看着对面的倪二。倪二咂嘴点头:“不错不错!”瘦子:“二哥看值个什么价儿?”
倪二:“看卖给谁了。要是大家子买去做姨娘,怎么也得个几百两!……你没花多少吧?”瘦子嘿嘿一笑:“真佛面前不说假话,八十两!”
倪二惊讶地:“什么?八十两?白给吧!”瘦子:“不过……是个黑户,京里不便出手。”倪二:“哦?”瘦子:“怎么样二哥?这个便宜生意让你做了!”
王短腿端来一碗热面汤,搁在小红面前的炕桌上。倪二乜斜着眼:“慢!……让这个傻丫头先陪爷们儿喝两盅!”
小红“呼”地一下子挺直了身子,翻身下炕,“扑通”跪在倪二面前,泪水扑簌簌流下来。倪二睁大了眼睛:“嗯?”
小红哽咽着:“求求大爷,行行好,把我送到牢里去……”瘦子眉头一皱:“起来起来!”倪二抬起一只手制止瘦子,上下打量着小红:“你……说什么?送你到牢里去?”
小红强忍住哭泣:“家抄了,爹娘、主子都……,求求大爷,能让我见上一面,来世变牛变马报答大爷……”倪二:“家抄了?谁家?!”
王短腿斜了瘦子一眼,冷冷地:“荣国府,贾家!”倪二:“啊?”瘦子急忙走过来:“二哥……”倪儿紧盯着小红:“你是……谁房里的?”
小红:“琏二奶奶。”倪二紧张地:“你姓什么?”小红:“姓林。”倪二倏地站起来:“你是……小红?”小红一惊:“你怎么知道的?”倪二顿着脚:“唉呀!该死!该死!”

21、狱神庙(夜)\par
哭泣声、喝骂声里,凤姐、李纨、平儿、秋桐被一个狱卒从左侧牢里押出,贾环、贾兰、贾菌被另一狱卒从右侧牢里押出。
狱卒:“不许嚎丧,快走!”宝玉呆呆地跟在后面。狱卒转身对宝玉:“你,回去!”凤姐等鱼贯走出庙门,消失在雨夜里。

22、王短腿家正房内(夜)\par
倪二咬着牙,一声不响地在房内来回走动着。小红坐在炕上,困惑地看着倪二。
瘦子的目光跟随着倪二:“二哥,人可不是我捆的,不信你问小红姑娘。那个王仁交给我就是捆着手脚堵着嘴的……”
倪二猛地站住:“算了!”倪二端起酒杯一口喝干,重重地往炕桌上一蹴:“王仁,王八蛋!……短腿!”
王短腿:“二哥?”倪二:“拿纸拿笔来!”王短腿不解地:“要纸笔干什么?”倪二眼一瞪:“我让你拿你就拿去!”
王短腿急忙走进里间。倪二斟满一杯酒。王短腿走出,把纸、笔、砚、墨放在炕桌上。倪二:“研墨!有色儿就行!”
众人不解地看着倪二。倪二冲着小红一抱拳:“我倪二不是人,刚才冒犯了,请多担待!”说着“嗖”地一下从腰里拽出一把解腕尖刀,左手拇指往刀尖上一按,一股鲜血冒了出来。
众人惊呼:“倪二哥!”倪二把刀插回腰间,左手拇指端端正正地在空白纸的左下角按上了一个血指印。王短腿:“二哥,你这是……”
倪二扭头冲着瘦子:“来,兄弟,我不会写字儿,你帮个忙儿。”瘦子拿起笔,不解地看着倪二。倪二一字一顿地:“借到银八十两,立借据人倪二。”
瘦子写好借据,用嘴吹了吹。倪二把借据往瘦子面前一推:“眼下没钱,十天内还你!人,放生了!”瘦子瞠日结舌:“二哥……”
小红“扑通”跪在炕上,呜咽着:“恩人……”倪二:“快起来快起来!……芸二爷……是我的朋友。”小红:“谁?”倪二:“贾芸,芸二爷!找你可找苦了!”小红:“芸……”

23、狱神庙(雨夜)\par
一个狱卒背对着宝玉,从一摞食盒里拿出一盘盘菜肴摆在一张当门而设的破旧的小桌上。宝玉满腹狐疑地看着。
狱卒斟满了两杯酒,转过身来,颤声叫道:“宝叔!”宝玉惊异地:“你……是谁?”狱卒一把摘下帽子:“叔叔,是我!”
宝玉:“芸儿?”狱卒打扮的贾芸“扑通”一下跪在了宝玉面前,哽咽着:“叔叔!”宝玉一把抱住贾芸:“你……怎么到这儿来了?”
贾芸急忙站起来,抹了抹眼睛,把宝玉拉到摆满酒菜的桌旁:“叔叔,请坐下!”宝玉困惑不解地看着贾芸,慢慢坐下。
贾芸擎起酒杯,强笑笑:“……叔叔,先前在家的时候,常想孝敬叔叔,一直没个机缘,今儿……”说着又哽咽起来。宝玉眼圈一红,连忙咬住嘴唇。
二人默然良久。宝玉含着泪水,慢慢端起酒杯:“……自遭家难以来,亲朋故旧,躲之惟恐不及。……我先时的知交,如今一个都……。老太爷、老爷当日提携了多少人!桃李门墙,绛帐春风……,没象雨村那样恩将仇报、落井下石就算很不错了!”
宝玉低下头去,泪水“叭哒叭哒”地滴落在酒杯里。贾芸:“叔叔……”宝玉猛地抬起头:“……难得你一片情意,来!干了!”说着举起酒杯,和泪吞下。
贾芸强忍着泪水,给宝玉斟酒。宝玉慢慢抬起头来。贾芸含着泪,默默地夹起各色菜肴,往宝玉面前的盘子里堆着。
宝玉呆呆地看着贾芸。盘子里已经堆不下了,贾芸还在不停地夹着。宝玉:“芸儿,别……”贾芸看了看盘子:“哦……”
宝玉深情地看着贾芸:“……还记得吗?那回,你送来的白海棠?”贾芸叹了口气。
宝玉:“难为你想着。……那时候,园子里的姐妹们都在……第一次结诗社……海棠诗社……咏白海棠……”
宝玉沉浸在回忆里:“……半卷湘帘半掩门,碾冰为土玉为盆。偷来梨蕊三分白,借得梅花一缕魂……”
贾芸仿佛被感染了,眼睛里闪动着泪光。宝玉:“这是林妹妹的……”贾芸喃喃地:“白海棠……好象就是昨儿的事……”
宝玉突然伏在桌子上抽泣起来。贾芸端起酒杯。一下子倒在嘴里,使劲咽了下去。
门外,夜雨潺潺。突然,传来一阵凄厉的哭喊声。宝玉急忙擦了擦眼泪,惊疑地看着贾芸。
“鸳鸯——”“鸳鸯姐姐——”“鸳鸯姑娘——”哭喊声在雨夜里回荡。宝玉:“什么?鸳鸯?”宝玉和贾芸“唿”地站起来,一齐扑到小木窗上。

24、羁候所院内(夜)\par
雨声、哭喊声、喝骂声、杂沓的脚步声搅成一片。几个狱卒从一间牢房里拖出一具女尸,朝远处走去。

25、狱神庙内(夜)\par
宝玉双手抓着窗栏,头深深地埋在窗台上。贾芸吁了一口长气,转过身来,慢慢走向酒桌。
贾芸停步,抬起头,怔怔地看着狱神塑像。狱神在幽暗的灯影里,显得格外阴森、狰狞。
贾芸打了个冷噤,急忙转身:“叔叔!”宝玉木然不动。贾芸大步走过去,一把拉住宝玉:“叔叔!”
宝玉慢慢抬起头,泪水、泥土沾了满脸。贾芸:“叔叔,这不是你呆的地方!我已经想好了,找几个朋友,救你出去!”
宝玉轻轻摇了摇头。贾芸:“真的!叔叔!”宝玉惨然一笑:“普天之下,莫非王土,我又能跑到那儿去?……再说,父母家人都在难中,我一个人出去又有什么意思?”
贾芸哽咽着:“我实在不忍心看着叔叔……”宝玉抓着贾芸:“你要是真想救我,只有一个办法!”贾芸:“叔叔快说!只要能救你……”
宝玉:“北静王爷任侠尚义,恤弱扶孤,一般落拓才士来到京师,尚且能生馆死殡。咱们家和北静府世世交好,断无不救之理!只是……王爷去年奉旨视边,不知道什么时候才能回来……”
贾芸眼睛一亮:“叔叔!……天明我就走,去找王爷!”宝玉激动地:“你……”贾芸:“叔叔放心!我虽说没读过什么书,可还知道申包胥哭秦庭的故事!见了王爷,我会……”
宝玉一把抓住贾芸的手:“……可……你走了,你母亲一个人……”贾芸黯然:“她已经……过世了。”
宝玉一惊:“什么时候?”贾芸:“叔叔搬出园子不久。”宝玉慢慢垂下头去,又慢慢抬起头来,含着泪:“我对不住你,我……”
贾芸:“不不,叔叔那时候正病着。”宝玉:“你一个人怎么……发送的?”贾芸:“……多亏了琏二婶子房里的小红,把她的体己,全给了我了。”
宝玉:“小红?……哦。她现在……?”贾芸摇摇头,泪水啪啪滴落在宝玉的手背上。

26、羁候所门外(晨)\par
夜雨初霁。笨重的大门缓缓打开,百余名男妇老幼在军校、狱卒的押解下走出大门。一乘小轿远远抬来。

27、狱神庙内\par
庙门洞开,白昼的亮光从门、窗斜射进来。宝玉、凤姐各自站在左右两个牢房里,手扶着木槛,轻声对话。
凤姐:“……他几时走的?”宝玉:“天快亮的时候。……去找北静王爷了。”凤姐:“什么?”宝玉:“找北静王爷!”凤姐含着泪:“真的?”

28、驿道(晨)\par
杨柳新绿。贾芸扬鞭催马,奔驰而过。十四、五岁的板儿牵着一头驴走在道边,骑在驴背上的刘姥姥回过头去,若有所思地看着贾芸远去的背影。

29、荣宁街\par
板儿牵着驴走过荣府正门。刘姥姥含泪看着台阶上十几个挺胸叠肚的男仆。

30、荣国府前角门\par
门口全副执事,一乘大轿自内抬出。若干男仆驱赶着行人,板儿扶着刘姥姥站在路边。大轿窗内,贾雨村含笑端坐。

31、狱神庙外\par
一乘小轿停放在门外。板儿把驴拴在轿旁的木桩上,搀着刘姥姥走向庙门。

32、狱神庙内\par
右侧牢内,宝玉贴着木槛呆呆地站着。左侧牢内,凤姐面色惨白,靠在草荐上。
小红跪在凤姐面前泣不成声:“……就这样,那个黑了心的王仁把我卖给了人牙子,他自己带着巧姑娘跑了。听人牙子说,他要把巧姑娘……”
凤姐微张着嘴,嘴唇抖着,泪水无声地簌簌流下。“姑奶奶!姑奶奶……”随着苍老的哭声,刘姥姥拉着板儿踉跄而入。
小红急忙擦泪回头,惊异地:“刘姥姥?……二奶奶,刘姥姥来了!”
凤姐挣扎欲起:“刘姥姥?……”刘姥姥走过来:“姑奶奶!”凤姐一把抱住刘姥姥,痛哭失声。
刘姥姥老泪纵横:“……才知道信儿……搁下活儿就……”“姥姥!”板儿递过一个大包袱。刘姥姥急忙擦擦泪,接过包袱:“过来,快给姑奶奶磕头!”
板儿规规矩矩地跪下:“给姑奶奶请安!”刘姥姥头一低,泪水又扑簌簌流下来。
凤姐强忍住哭,睁开泪眼。刘姥姥哽咽着:“这是板儿,亏着你们府里照看,这几年没冻着饿着,还念了点子书。这不,眼看长成个人了。还比着你们府里,学了些规矩……”
板儿又趴在地上磕了个头,起身站在一边。刘姥姥:“……不象那年进府里去,就知道傻吃,跟你们巧哥儿争柚子、争佛手……”
小红一下子把手堵在嘴里。刘姥姥:“……哎?巧哥儿呢?”凤姐头靠在墙壁上,眼睛紧紧地闭着,双肩耸动,泪流满面。
刘姥姥惊疑地:“怎么了?巧哥儿……怎么着了?”小红艰难地:“……让她舅舅给……卖了……”
刘姥姥:“什么?”小红呜呜地哭起来。
刘姥姥用轻得几乎听不见的声音,喃喃地说:“……亲舅舅?”右面牢里,宝玉闭目流涕。
左面牢里,刘姥姥一把抓住小红,颤抖着:“卖到……哪儿去了?”小红:“南省……瓜洲……”
刘姥姥急忙用手抹抹泪:“……知道地儿就行,我去找!……拼了我这把老骨头,怎么着也把她找回来!”
凤姐抓着刘姥姥的手,说不出话。刘姥姥强笑笑:“这孩子的名儿还是我起的,‘以毒攻毒,以火攻火’,我还是那句话儿。姑奶奶放心吧,那时候我就说过,一时有不遂心的事,必然是遇难呈样,逢凶化吉……,都从这‘巧’字上来……”说着哽咽起来。
凤姐挣扎着坐起来,身子一软,跪倒在刘姥姥面前。刘姥姥慌忙抱住凤姐:“这是怎么说?这可折死我了!快起来!快……”
凤姐:“不,姥姥,你听我说。……我自小争胜好强,凡百的事,没有让人的。凌弱欺孤,图财害命,我也是干过的。如今,想回头积德积寿也不能够了,我命里无子,只有一个巧儿,我落到这一步,是该着的,不承想又报应到她身上。姥姥若能救出她来,我就是下十八层地狱,也心甘了。她果真有造化,就让她跟着姥姥,日后寻个庄户人家,安份守己,可千万别学我……”说着连连碰地,泣不成声。
刘姥姥死死抱住凤姐:“可别这么着!日头还有个云彩遮住的时候,日子长了,得往远处想想!……就是先时做过什么事,有那几句话,菩萨也饶过了,……再听我说一句,常念念‘救苦救难大慈大悲南无观世音菩萨’。”
凤姐慢慢抬起头来。刘姥姥搀着凤姐坐好。随手打开包袱。包袱里是一套套衣袄裙裤。
刘姥姥:“这几身衣裳,还是那年从府里带回去的,只是过年过节拿出来看看,老也舍不得穿。这回,都带来了,怕姑奶奶们没得换……”
刘姥姥一件件往外拿着:“……这几件,给姑奶奶,这几件,给平姑娘,这几件,给鸳鸯姑娘……”
右侧牢里,宝玉失声痛哭。刘姥姥一惊:“谁?”小红:“宝二爷。”
刘姥姥连忙拿起一套男装,颤巍巍地走出牢门,来到宝玉面前,双眼流泪:“玉哥儿,还……认得我吗?”
宝玉:“刘姥姥……”刘姥姥:“哎……,这套衣服,是我赶着做的,你……换换。……那年,你给我的那个成窑盅子,我还留着呢……”
宝玉隔着木槛,抓住刘姥姥的手,泣不成声。一阵匆忙的脚步声,牢头老三走进狱神庙。
宝玉一惊,下意识地往后退退。老三朝左侧牢房:“小红姑娘!”小红应声走出。
老三:“你爹娘的事,我打听清楚了。昨儿……和平姑娘一块儿,让平安州的一家富商买走了。”
小红“哇”地一声哭起来。老三:“别哭,别哭,这是好事,不用总在这儿担惊受怕了! ”
宝玉惊疑地看着老三。老三:“小红姑娘,你该走了,万一让人认了出来,可就……”
小红:“不不,我不走了!”凤姐、宝玉不由自主地:“小红!”
小红含泪左右看看:“琏二奶奶,宝二爷,我是府里的家生女儿,服侍主子是我的本份。二奶奶待我恩重如山,可我对不起二奶奶,把巧姑娘给……”
凤姐:“小红!这怨不得你……”小红:“不。……这会子主子遭着难,我再只顾自己去了,我可成了什么人了?”
小红捂着脸呜呜地哭着。众人感动地看着小红。小红抬起头,慢慢走到宝玉面前:“……那年我在怡红院,给二爷倒了一次茶,秋纹姐姐、碧痕姐姐骂我‘没脸下流,正经活不干,专等着巧宗儿’;后来给二奶奶送了一次东西,晴雯姐姐又刺打我‘爬上高枝儿去了’,如今她们都不在了,这个‘巧宗儿’,这个‘高枝儿’,就让我都占了吧……”
宝玉、凤姐掩面恸哭。刘姥姥拉着小红的手:“好孩子,好孩子……”
老三:“小红姑娘,我看这么着吧:王短腿说话就去西边贩马,你就住他家,让倪二哥的闺女陪着你。我当值的时候,你就进来伺候主子。等芸二爷回来,再商量个道理。”
刘姥姥掩着泪:“好,好,我替他们主子奴才,谢谢这位爷了!”说着就要往地下跪。
老三慌忙一把搀住:“哎哎老人家,使不得!……他们府上西廊下的芸二爷,和我是朋友,没说的!”宝玉怔怔地看着老三。
(闪回)横眉立目的老三“啪”地一个耳光打过来。刘姥姥唏嘘着:“天底下还是好人多……”宝玉感慨地吁了口气。
刘姥姥:“姑奶奶……玉哥儿……我……走了……”宝玉、凤姐齐唰唰跪在各自的牢里:“刘姥姥……”

33、大路(晴日)\par
田野上一片新绿,路边开满了各色野花。刘姥姥骑驴走来,板儿跟在后面,不时用手中的柳枝轻轻抽打着驴背。

34、崇山峻岭(晴日)\par
贾芸牵着马在山道上攀援。

35、狱神庙(黄昏)\par
左侧牢内,凤姐面朝狱神塑像跪着,双手合十。

36、江中(阴雨)\par
云低水阔,细雨潇潇。一叶扁舟顺流而下,刘姥姥和板儿坐在低矮的舱内,呆呆地看着跳动的水珠。

37、荒漠\par
狂风呼啸,石走沙飞。贾芸牵着马艰难地行进着。

38、狱神庙(清晨)\par
小红推门走入。

39、大路(夏)\par
赤日炎炎。一辆马车奔驰而去。车棚内,刘姥姥给板儿擦汗。

40、山巅(夏)\par
贾芸立马眺望。

41、狱神庙\par
右侧牢内,宝玉无言伫立。

42、街市(夏)\par
各种叫卖声、喧嚣声里,刘姥姥拉着板儿走进一家店铺,比比划划地询问着什么。
柜台后面的伙计们互相看看,一齐摇头。刘姥姥叹了口气,走出店铺。
另一家店铺门前的台阶上,一个癞头和尚向阳扪虱。刘姥姥走近。弯腰询问着什么。
癞头和尚看着刘姥姥嘻嘻笑着。刘姥姥叹了口气,直起身子。

43、石拱桥头\par
一个缁衣女尼搓着小钹向过往行人化缘。刘姥姥扶着扳儿下桥,和女尼擦肩走过。
刘姥姥回头,若有所思地看着女尼的背影。刘姥姥转身,死死地盯着女尼,慢慢走过去。
板儿莫名其妙地跟着刘姥姥。刘姥姥伸手拍了拍女尼的肩膀,女尼回过头来。
刘姥姥惊喜地:“四姑娘!”女尼微微一震,停止了搓动,端详着刘姥姥。
刘姥姥:“四姑娘!你……怎么在这儿?”女尼合目低头,手中的小钹又搓动起来,转身欲去。
刘姥姥一把拉住女尼:“四姑娘!你……不认识我了?……我是……”女尼漠然平视:“施主,你说的什么?”
刘姥姥噙着泪花:“惜春姑娘!你们贾家遭事了,你知道吗?”女尼:“阿弥陀佛,什么假家真家,你……认错人了!”说着转身快步走去。
刘姥姥跟了几步,颤声:“你忘了?那年在园子里,我还找你要过画儿……”女尼消逝在人群里。刘姥姥凄怆地:“惜春姑娘……”

44、瓜州渡口\par
石碑上刻着四个大字:瓜州古渡。刘姥姥和板儿坐在一个茶摊的凉棚下面,呆呆地看着前面的粗瓷茶碗。

45、藏春院花厅\par
丝竹盈耳,歌曲绕梁。几名抚琴弄箫的娼女坐在一侧。教曲师傅坐在另一侧,自掐檀板,闭目品味。
巧姐含着泪,轻移莲步,款摆腰肢,口中娓娓唱曲:“……露滴牡丹开,鱼水得和谐,……檀口揾香腮……但沾着些儿麻上来……”
教曲师傅“啪”地一拍檀板,乐声骤停。教曲师傅瞪着眼睛喝骂:“你是唱曲儿还是嚎丧?……一天不挨打,你就过不去!”
巧姐害怕地往后退退。教曲师傅:“重来!”乐声复起。

46、藏春院穿堂\par
桌上打开的包袱里放着几封银子。刘姥姥含泪陪笑对着鸨母:“……这瓜州地面上,无论问到谁,却说妈妈心眼实诚,为人厚道,又最是个积德行善的,求妈妈……”
鸨母不耐烦地:“行了行了!我告诉你,不看着你这么大年纪,千里迢迢从天子脚下找到这儿来,又说得这么可怜,我才不会答应你赎她!……再说一遍,银子不够是不行的!快凑去吧,晚了可就接客了!”
刘姥姥低声下气地:“是是,我这就回去凑银子。……再求求妈妈,能让我见一面……”鸨母:“不行!这会子正请着先生教曲儿呢!”说着站起身来向内走去。
刘姥姥:“哎,妈妈……”一个龟奴过来挡住:“快走吧!呆会儿翻了脸,想赎也赎不成了!”刘姥姥慢慢转身,收拾好包袱,拉着板儿向门外走去。

47、藏春院门外(黄昏)\par
刘姥姥和板儿走下台阶,回头看了一眼。院门一侧“藏春院”木牌下,挂着一溜小木牌,上面分别写着:“赵飞燕”、“杨玉环”、“王昭君”……
门内不时传出男男女女的嬉闹声、唱曲声。刘姥姥抹抹眼睛,冲着板儿一跺脚:“走!回去!卖房子,卖地!”

48、荒原(晚秋)\par
秋风瑟瑟,衰草连天。几只灰狼嗥叫着追赶一匹奔马。马背上的贾芸不时紧张地回头。
两只灰狼狂奔着抄到贾芸前面,堵住了去路。贾芸急忙勒马。狼群把贾芸围在中央,渐渐逼近。
贾芸的额上沁出了冷汗。“嗖”地一箭飞来,一只狼中箭,嗥叫着倒地。
群狼扑过来,撕咬着伤狼。贾芸惊异地抬头。戎装的北静王水溶挽弓背箭,带着十余骑护卫纵马驰来。
群狼丢下伤狼四散奔逃。水溶等绕着贾芸转了一圈儿,勒马停住。两名护卫下马,把伤狼捆住。
贾芸惊呼:“王爷!”水溶一愣:“嗯?”贾芸滚下马背,“扑通”一下跪在水溶马前,叩头不止,涕泗交流。
水溶下马,惊疑地:“你是……?”贾芸从怀里掏出一个信封,双手高高举过头顶。水溶接过,急忙展读,大惊:“贾宝玉……”

49、狱神庙(晚秋)\par
右侧牢内,宝玉从小窗中呆呆地看着天空里的雁阵。左侧牢内,凤姐躺在草荐上闭目呻吟,小红跪在旁边,用嘴唇试着粗瓷碗里的水温。
一阵急促的脚步声,老三神色惶惶地走进来:“宝二爷!不好了!”宝玉走过来抓住木槛:“怎么?”老三:“贵府大老爷、二老爷、琏二爷、珍大爷秋审已经定谳,单等着钉封一到,就要问斩了!”
宝玉大惊失色:“啊?”左侧牢内,“啪”地一声,粗瓷碗落地。
老三:“还有,原来府上的几个清客,如今又投在贾雨村门下,贾雨村举发宝二爷在一首什么诗里骂了皇上,他们几个都做了证。听说事情要闹大了,宝二爷快想想怎么对口吧!”
左侧牢里传来小红的一声尖叫:“二奶奶!”

50、瓜州·藏春院花厅(冬)\par
火盆里炭火熊熊。桌上解开的包袱里,露出一封封银子和手镯、耳环、戒指等物。
刘姥姥从包袱里拿起两个荷包,抽开系子,掏出两个笔锭如意的锞子,又搁在包袱上。
鸨母眼睛一亮:“哟!”连忙拿过锞子,翻过来正过去地看着,不禁眉开眼笑。
刘姥姥:“妈妈,够了吧?”鸨母连连点头,对旁边的一个龟奴:“叫她出来吧!”
鸨母托着锞子,试探地看着刘姥姥:“你一个庄稼人,哪儿来的这个?”
刘姥姥眼圈儿一红。门帘一响,巧姐哭着跑入:“刘姥姥!”
刘姥姥急忙迎过去,一把抱住巧姐:“巧哥儿!”板儿站在一旁,呆呆地看着抱头痛哭的刘姥姥和巧姐,眼眶里噙满了泪水。
刘姥姥抬起头,慈爱地抚摸着巧姐:“孩子,咱们……回家!”

51、狱神庙(冬)\par
彤云密布,大雪纷飞。老三走过狱神庙门外。“老三!”宝玉隔着小木窗,轻轻喊了一声。
老三停步,左右看看,凑到窗下。宝玉:“我们老爷……有信儿吗?”老三皱了皱眉:“怪事儿,一点儿信儿也得不着!”
宝玉叹了口气。老三:“哎,你那个环儿兄弟这些日子一直吵吵着,有什么事儿只要审他,他就往你身上推!……你提防着他点儿!”
宝玉紧锁着眉头,说不出话。老三:“我走了。”宝玉:“哎,老三!……凤姐姐打入冬起就不会说话了,你看能不能……悄悄请个先生……”
老三看看宝玉:“嗐”了一声,转身离去。

52、狱神庙内\par
雪花从小窗外飘进来,洒落在草荐上。凤姐平躺着,大睁着眼睛,呆滞的目光散视着上方,身上的雪花已经积了薄薄的一层。
小红象泥胎一样坐在旁边,两行清泪已经冻在腮边了。一阵料峭的寒风从窗外吹进,卷起的雪花满屋里飞旋着。
凤姐微微动了一下,嘴唇艰难地翕翕张开,挤出了两个隐约可辨的字:“……小红……”小红象然低头看着凤姐。凤姐:“……小红……”
小红一愣:“二奶奶……说话了?”凤姐:“……小红……”小红象发疯一样跳起来:“二奶奶说话了:二奶奶说话了!……宝二爷,二奶奶说话了!”
右侧牢内,宝玉一下扑到木槛上:“什么?”左侧牢内,小红喊着:“二奶奶……说话了!”宝玉:“说话了?说的……什么?”
小红扑在凤姐身上,哆嗦着:“二奶奶,这就要说什么?”凤姐微弱的,勉强可以听见的声音:“……我要……死了……”
小红咽着泪水:“不不,二奶奶,你能说活了,这就要好了……”凤姐突然“呜”地一下哭出声来:“千万……把我……送回……金……陵……”
小红:“二奶奶……”凤姐挣着命一下子撑起身子,撕裂人心地喊了一声:“巧姐儿……”

53、狱神庙门外\par
喊声钻出雪窗,在茫茫天地间消逝了。

54、驿道\par
一辆马车冒雪奔驰着。车棚内,刘姥姥搂着巧姐儿:“乖乖,咱们回家了。先去看你妈……”巧姐噙着泪花依偎着刘姥姥,甜甜地笑了。

55、狱神庙门外\par
北风怒号。两个狱卒从敞开的庙门内抬出用草荐裹着的凤姐。宝玉神情麻木的脸挤在小窗的木槛上。
两个狱卒用木杠斜错开一前一后抬着草荐,草荐的一头拖在雪地上,划出一道长长的沟痕……歌声起:
机关算尽太聪明,反算了卿卿性命。
生前心已碎,死后性空灵。
家富人宁,终有个家亡人散各奔腾。
枉费了、意悬悬半世心;好一似、荡悠悠三更梦。
忽喇喇似大厦倾,
昏惨惨似灯将尽。
呀!一场欢喜忽悲辛。
叹人世,终难定!

56、雪原\par
歌声起,草荐裹着的凤姐被扔进雪坑。纷纷扬扬的大雪,掩埋着一切……

