\chapter{绣鸳鸯梦兆绛芸轩 \quad 识分定情悟梨香院}
\zhu{分定:命中注定的缘分。
}
\ji{绛芸轩梦兆是金针暗度法,
\zhu{
金针暗度:
度:授与,传授。比喻把高超的技艺偷偷传授给别人。
后也以刺绣譬喻作诗文者别有巧妙,明清小说、戏剧评点家也常用来评点小说、戏剧中的巧妙章法和构思。
从《红楼梦》有关回的正文看,作批者是指曹雪芹表面在写一回事,而实际有更深的、另外的含义,或者是运用巧妙的艺术手法而不露痕迹。
}
夹写月钱是为袭人渐入金屋地步,梨香院是明写大家蓄戏,不免奸淫之陋。
可不慎哉,慎哉!}\par
\qi{造物何尝作主张,任人禀受福修长。
\zhu{
“任人禀受福修长”即“各人各得眼泪”。而“造物何尝作主张”意思是“福”还是由“情缘”决定的,不是命定的。
}
划蔷亦自非容易,解得臣忠子也良。
\zhu{
《长生殿》第一出《传概》中《满江红》曲词有云:“今古情场,问谁个真心到底?……看臣忠子孝,总由情至。”
由儿女情长引导出“忠”和“孝”这种传统社会根本性的意识形态,说它们都是以真“情”为价值根基。
}
}\par
话说贾母自王夫人处回来,见宝玉一日好似一日,心中自是欢喜。
因怕将来贾政又叫他,遂命人将贾政的亲随小厮头儿唤来,吩咐他“以后倘有会人待客诸样的事,你老爷要叫宝玉,你不用上来传话,就回他说我说了:一则打重了,得着实将养几个月才走得;二则他的星宿不利,\zhu{星宿(宿音“秀”)不利:星宿:我国古代对星座的称呼。
旧时星相术士等用生辰八字,按天上星宿的运数,来推算人的禄命和吉凶祸福,凡遇不吉利的事情,认为是相应的星宿不吉利之故,就需祭星消灾。
}
祭了星不见外人,过了八月才许出二门。
”那小厮头儿听了,领命而去。
贾母又命李嬷嬷袭人等来,将此话说与宝玉,使他放心。
那宝玉本就懒与士大夫诸男人接谈,又最厌峨冠礼服贺吊往还等事,今日得了这句话,越发得了意,不但将亲戚朋友一概杜绝了,而且连家庭中晨昏定省亦发都随他的便了,\zhu{定省:定,侍候就寝;省,探望问候。
定省指子女早晚向父母请安问好的礼节。
}日日只在园中游卧,不过每日一清早到贾母王夫人处走走就回来了,却每每甘心为诸丫鬟充役,竟也得十分闲消日月。
或如宝钗辈有时见机导劝,反生起气来,只说:“好好的一个清净洁白女儿,也学的钓名沽誉,入了国贼禄鬼之流。
这总是前人无故生事,立言竖辞,原为导后世的须眉浊物。
不想我生不幸,亦且琼闺绣阁中亦染此风,真真有负天地钟灵毓秀之德!”\zhu{
钟:聚。
毓:音“玉”,养育。
钟灵毓秀:旧时认为杰出有为的人才,是天地间灵秀之气聚集培育出来的。
}\meng{宝玉何等心思,作者何等意见,此文何等笔墨!}因此祸延古人,除四书外,竟将别的书焚了。
众人见他如此疯颠,也都不向他说这些正经话了。
独有林黛玉自幼不曾劝他去立身扬名等语,所以深敬黛玉。
\ping{宝玉的这套看不起功名利禄的清高做派也不新鲜了,想要钓名沽誉当个国贼禄鬼是有门槛的,而享受闺阁清净,当个富贵闲人,只需要投胎投得好就可以了。
只有经历过功名的人,才有资格去谈淡泊名利,看破红尘,没有这个探索尝试的过程,何谈淡泊,大概不过是怯懦与无能的伪装和借口罢了。
}\par
闲言少述。
如今且说王凤姐自见金钏死后,忽见几家仆人常来孝敬他些东西,\meng{为当涂人一笑。
\zhu{当涂:执掌大权,身居要位。
}}又不时的来请安奉承,自己倒生了疑惑,不知何意。
这日又见人来孝敬他东西,因晚间无人时笑问平儿道:“这几家人不大管我的事,为什么忽然这么和我贴近?”平儿冷笑道:“奶奶连这个都想不起来了?我猜他们的女儿都必是太太房里的丫头,如今太太房里有四个大的,一个月一两银子的分例,下剩的都是一个月几百钱。
如今金钏儿死了,必定他们要弄这两银子的巧宗儿呢。
”凤姐听了,笑道:“是了,是了,倒是你提醒了。
我看这些人也太不知足,钱也赚够了,苦事情又侵不着,弄个丫头搪塞着身子也就罢了,又还想这个。
也罢了,他们几家的钱容易也不能花到我跟前,这是他们自寻的,送什么来,我就收什么,横竖我有主意。
”\meng{确见高论!而其心思则不可问矣。
任事者戒之!}凤姐儿安下这个心,所以自管迁延着,等那些人把东西送足了,然后乘空方回王夫人。
\par
这日午间,薛姨妈母女两个与林黛玉等正在王夫人房里,大家吃东西呢,凤姐儿得便回王夫人道:“自从玉钏儿姐姐死了,太太跟前少着一个人。
太太或看准了那个丫头好,就吩咐,下月好发放月钱的。
”王夫人听了,想了一想,道:“依我说,什么是例,必定四个五个的,够使就罢了,竟可以免了罢。
”凤姐笑道:“论理,太太说的也是。
这原是旧例,别人屋里还有两个呢,太太倒不按例了。
况且省下一两银子也有限。
”王夫人听了,又想一想,道:“也罢,这个分例只管关了来,\zhu{关:领取。
}
不用补人,就把这一两银子给他妹妹玉钏儿罢。
他姐姐伏侍了我一场,没个好结果,剩下他妹妹跟着我,吃个双分子也不为过逾了。
”凤姐答应着,回头找玉钏儿,笑道:“大喜,大喜。
”玉钏儿过来磕了头。
\ping{悲乎?喜乎?}\par
王夫人问道:“正要问你,如今赵姨娘周姨娘的月例多少?”凤姐道:“那是定例,每人二两。
赵姨娘有环兄弟的二两,共是四两,另外四串钱。
”王夫人道:“可都按数给他们?”凤姐见问的奇怪,忙道:“怎么不按数给!”王夫人道:“前儿我恍惚听见有人抱怨,说短了一吊钱,是什么原故?”凤姐忙笑道:“姨娘们的丫头,月例原是人各一吊。
从旧年他们外头商议的,\ping{凤姐可能拿别人当作替罪羊、背锅侠,把自己塑造成执行者而非决策者,不承担责任。
}姨娘们每位的丫头分例减半,人各五百钱,每位两个丫头,所以短了一吊钱。
这也抱怨不着我,我倒乐得给他们呢,他们外头又扣着,难道我添上不成。
这个事我不过是接手儿,怎么来,怎么去,由不得我作主。
我倒说了两三回,仍旧添上这两分的。
他们说只有这个项数,叫我也难再说了。
如今我手里每月连日子都不错给他们呢。
先时在外头关,那个月不打饥荒,\zhu{打饥荒:此指纠缠不休,找麻烦。
}何曾顺顺溜溜的得过一遭儿。
”\meng{能事能言。
}\ping{此时王夫人的语气好像在怀疑凤姐克扣月钱,其实凤姐并没有准时发放月例。
第三十九回,袭人问平儿为何不发月钱,平儿悄声告诉袭人,王熙凤已经把月钱支领了出来,但是挪用月钱放贷以赚取利钱。
}王夫人听说,也就罢了,半日又问:“老太太屋里几个一两的?”凤姐道:“八个。
如今只有七个,那一个是袭人。
”王夫人道:“这就是了。
你宝兄弟也并没有一两的丫头,袭人还算是老太太房里的人。
”凤姐笑道:“袭人原是老太太的人,不过给了宝兄弟使。
他这一两银子还在老太太的丫头分例上领。
如今说因为袭人是宝玉的人,裁了这一两银子,断然使不得。
若说再添一个人给老太太,这个还可以裁他的。
若不裁他的,须得环兄弟屋里也添上一个才公道均匀了。
就是晴雯麝月等七个大丫头,每月人各月钱一吊,佳蕙等八个小丫头,每月人各月钱五百,还是老太太的话,别人如何恼得气得呢。
”薛姨娘笑道:“只听凤丫头的嘴,倒像倒了核桃车子的,
\zhu{
倒了核桃车子:车子一倒,核桃滚出来,相互碰撞会发出一片清脆的响声。比喻说话干脆、响亮、接连不断。
}
只听他的帐也清楚,理也公道。
”凤姐笑道:“姑妈,难道我说错了不成?”薛姨妈笑道:“说的何尝错,只是你慢些说岂不省力。
”凤姐才要笑,忙又忍住了,听王夫人示下。
\par
王夫人想了半日,向凤姐儿道:“明儿挑一个好丫头送去老太太使,补袭人,把袭人的一分裁了。
把我每月的月例二十两银子里,拿出二两银子一吊钱来给袭人。
\meng{写尽慈母苦心。
}以后凡事有赵姨娘周姨娘的,也有袭人的,只是袭人的这一分都从我的分例上匀出来,不必动官中的就是了。
”\ping{袭人从贾府解雇,而重新受雇于王夫人。
袭人提前领姨娘薪水,预备姨娘的位置进一步稳固。
}凤姐一一的答应了,笑推薛姨妈道:“姑妈听见了,我素日说的话如何?今儿果然应了我的话。
”薛姨妈道:“早就该如此。
模样儿自然不用说的,他的那一种行事大方,说话见人和气里头带着刚硬要强,这个实在难得。
”王夫人含泪说道:“你们那里知道袭人那孩子的好处?\ji{“孩子”二字愈见亲热,故后文连呼二声“我的儿”。
}比我的宝玉强十倍!\ji{忽加“我的宝玉”四字,愈令人堕泪,加“我的”二字者,是明显袭人是“彼的”。
然彼的何如此好,我的何如此不好?又气又恨,宝玉罪有万重矣。
作者有多少眼泪写此一句,观者又不知有多少眼泪也。
}宝玉果然是有造化的,能够得他长长远远的伏侍他一辈子,也就罢了。
”\ji{真好文字,此批得出者。
\zhu{这句话的意思大概是,批书人自己夸自己能够发现作者文字之美,是作者的知音。
}}
凤姐道:“既这么样,就开了脸,\zhu{开脸:旧俗女子出嫁时用线绞净脸上的汗毛,修齐鬓角,叫作“开脸”。
}明放他在屋里岂不好?”王夫人道:“那就不好了,一则都年轻,二则老爷也不许,三则那宝玉见袭人是个丫头,纵有放纵的事,倒能听他的劝,如今作了跟前人,\zhu{跟前人:这里指被收作妾的丫鬟,义同前面的“房里人”。
}那袭人该劝的也不敢十分劝了。
\meng{苦心!作子弟的,读此等文章,能不坠泪?}如今且浑着,等再过二三年再说。
”\par
说毕半日,凤姐见无话,便转身出来。
刚至廊檐上,只见有几个执事的媳妇子正等他回事呢,见他出来,都笑道:“奶奶今儿回什么事,这半天?可是要热着了。
”凤姐把袖子挽了几挽,跐着那角门的门槛子,\zhu{跐:音“此”,为了支持身体用脚踩。
}
\meng{能事得意之人,如画。
}笑道:“这里过门风倒凉快,吹一吹再走。
”又告诉众人道:“你们说我回了这半日的话,太太把二百年头里的事都想起来问我,难道我不说罢。
”又冷笑道:“我从今以后倒要干几样尅毒事了。
\zhu{尅:同“克”,“克”通“刻”,苛刻。
}抱怨给太太听,我也不怕。
糊涂油蒙了心,烂了舌头,不得好死的下作东西,别作娘的春梦!明儿一裹脑子扣的日子还有呢。
\zhu{一裹脑子:一股脑儿,统统。}
\meng{的真活现。
}如今裁了丫头的钱,就抱怨了咱们。
也不想一想是奴几,\zhu{奴几:奴才辈。
几:指排列、辈分。
}也配使两三个丫头!”一面骂,一面方走了,自去挑人回贾母话去,不在话下。
\par
却说王夫人等这里吃毕西瓜,又说了一回闲话,各自方散去。
宝钗与黛玉等回至园中,宝钗因约黛玉往藕香榭去,黛玉回说立刻要洗澡,便各自散了。
宝钗独自行来,顺路进了怡红院,意欲寻宝玉谈讲以解午倦。
不想一入院来,鸦雀无闻,一并连两只仙鹤在芭蕉下都睡着了。
宝钗便顺着游廊来至房中,只见外间床上横三竖四,都是丫头们睡觉。
转过十锦槅子,\zhu{十锦:即“什锦”,由多种原料制成或多种花样拼成的。
槅子:架子,放置器物的木器。木架上分不同形状的许多层小格,格内可放入各种器皿、用具。也作“格子”。
}来至宝玉的房内。
宝玉在床上睡着了,袭人坐在身旁,手里做针线,旁边放着一柄白犀麈。
\zhu{
麈:音“主”,一种似骆驼的鹿类动物。
白犀麈:一种精致贵重的拂尘。
拂尘:形如马尾,后有持柄,用以拂拭尘土,或驱赶蝇蚊,俗称“蝇甩子”。
古时多用麈兽之尾制成,故又称麈尾。
}宝钗走近前来,悄悄的笑道:“你也过于小心了,这个屋里那里还有苍蝇蚊子,还拿蝇帚子赶什么?”袭人不防,猛抬头见是宝钗,忙放下针线,起身悄悄笑道:“姑娘来了,我倒也不防,唬了一跳。
\meng{闲情闲景,随便拈来,便是佳文佳语。
}姑娘不知道,虽然没有苍蝇蚊子,谁知有一种小虫子,从这纱眼里钻进来,人也看不见,只睡着了,咬一口,就像蚂蚁夹的。
”宝钗道:“怨不得。
这屋子后头又近水,又都是香花儿,这屋子里头又香。
这种虫子都是花心里长的,闻香就扑。
”说着,一面又瞧他手里的针线,原来是个白绫红里的兜肚,上面扎着鸳鸯戏莲的花样,红莲绿叶,五色鸳鸯。
宝钗道:“嗳哟,好鲜亮活计!这是谁的,也值的费这么大工夫?”袭人向床上努嘴儿。
\meng{妙形景。
}宝钗笑道:“这么大了,还带这个?”袭人笑道:“他原是不带,所以特特的做的好了,叫他看见由不得不带。
如今天气热,睡觉都不留神,哄他带上了,便是夜里纵盖不严些儿,也就不怕了。
你说这一个就用了工夫,还没看见他身上现带的那一个呢。
”宝钗笑道:“也亏你奈烦。
”袭人道:“今儿做的工夫大了,脖子低的怪酸的。
”\meng{随便写来,有神有理,生出下文多少故事。
}又笑道:“好姑娘,你略坐一坐,我出去走走就来。
”说着便走了。
宝钗只顾看着活计,便不留心,一蹲身,刚刚的也坐在袭人方才坐的所在,因又见那活计实在可爱,不由的拿起针来,替他代刺。
\par
不想林黛玉因遇见史湘云约他来与袭人道喜,二人来至院中,见静悄悄的,湘云便转身先到厢房里去找袭人。
林黛玉却来至窗外,隔着纱窗往里一看,只见宝玉穿着银红纱衫子,随便睡着在床上,宝钗坐在身旁做针线,旁边放着蝇帚子,林黛玉见了这个景儿,连忙把身子一藏,手握着嘴不敢笑出来,招手儿叫湘云。
\ping{黛玉和宝玉此时已经心意相通,知道彼此的真心,所以看到宝钗坐在宝玉床上刺绣而不生气。
}湘云一见他这般景况,只当有什么新闻,忙也来一看,也要笑时,忽然想起宝钗素日待他厚道,便忙掩住口。
知道林黛玉不让人,怕他言语之中取笑,便忙拉过他来道:“走罢。
我想起袭人来,他说午间要到池子里去洗衣裳,想必去了,咱们那里找他去。
”林黛玉心下明白,冷笑了两声,只得随他走了。
\meng{触眼偏生碍,多心偏是痴。
万魔随事起,何日是完时?}\par
这里宝钗只刚做了两三个花瓣,忽见宝玉在梦中喊骂说:“和尚道士的话如何信得?什么是金玉姻缘,我偏说是木石姻缘!”薛宝钗听了这话,不觉怔了。
\meng{请问:此“怔了”是呓语之故,还是呓语之意不妥之故?猜猜。
}\ping{宝玉心里所想,梦里所说,宝钗由此知道了宝玉对于金玉良缘的真实态度,至此完结回目上句:“绣鸳鸯梦兆绛芸轩”。
}忽见袭人走过来,笑道:“还没有醒呢。
”宝钗摇头。
袭人又笑道:“我才碰见林姑娘史大姑娘,他们可曾进来?”宝钗道:“没见他们进来。
”因向袭人笑道:“他们没告诉你什么话?”袭人笑道:“左不过是他们那些玩话,有什么正经说的。
”宝钗笑道:“他们说的可不是玩话,我正要告诉你呢,你又忙忙的出去了。
”\par
一句话未完,只见凤姐儿打发人来叫袭人。
宝钗笑道:“就是为那话了。
”袭人只得唤起两个丫鬟来,一同宝钗出怡红院,自往凤姐这里来。
果然是告诉他这话,又叫他与王夫人叩头,且不必去见贾母,倒把袭人不好意思的。
\ping{袭人不去见贾母,贾母也会知道这件事。
王夫人在本回的解决办法是,另挑一个好的丫头给贾母使,补袭人的缺。
当贾母看到自己多了一个新的丫头的时候,就会知道这一切。
袭人本来是老太太的丫鬟,是借给宝玉使唤,袭人被王夫人强行从贾府辞退,改为自己私人雇佣,还不让袭人去辞别贾母,这样一来袭人在贾母心中的形象就会变成见利忘义的负心人,贾母再也不能是袭人的靠山了,贾府甚至都不是袭人的雇主,袭人的处境很尴尬,从此只能以王夫人作为自己的靠山,成为王夫人和贾母斗争的一枚棋子。
}见过王夫人急忙回来,宝玉已醒了,问起原故,袭人且含糊答应,至夜间人静,袭人方告诉。
\meng{夜深人静时,不减长生殿风味。
\zhu{长生殿:白居易《长恨歌》“七月七日长生殿,夜半无人私语时”写唐玄宗和杨贵妃的情事。}
何等告法?何等听法?人生不遇此等景况,实辜负此一生!}宝玉喜不自禁,又向他笑道:“我可看你回家去不去了!那一回往家里走了一趟,回来就说你哥哥要赎你,又说在这里没着落,终久算什么,说了那么些无情无义的生分话唬我。
\ji{“唬”字妙!尔果系明决男子,何得畏女子唬哉?}从今以后,我可看谁来敢叫你去。
”袭人听了,便冷笑道:“你倒别这么说。
从此以后我是太太的人了,我要走连你也不必告诉,只回了太太就走。
”宝玉笑道:“就便算我不好,你回了太太竟去了,叫别人听见说我不好,你去了你也没意思。
”袭人笑道:“有什么没意思,难道作了强盗贼,我也跟着罢。
再不然,还有一个死呢。
人活百岁,横竖要死,这一口气不在,听不见看不见就罢了。
”\meng{自古及今,大凡大英雄、大豪杰,忠臣孝子,至其真极,不过一死,呜呼哀哉!}\ping{按理说,袭人成为王夫人的心腹,成为宝玉的预备姨娘,本来应该高兴才对,但是袭人此番讲话却很悲伤。
估计袭人也意识到自己的尴尬处境,自己被王夫人从贾母那里公开地挖到王夫人这边来,自己已经不是贾府的仆人,得罪了贾母,命运寄托在王夫人一个人身上,这样的被动“跳槽”,也使得自己站到了贾母和王夫人斗争的前线,很可能在未来成为婆媳斗争的炮灰。
}宝玉听见这话,便忙握他的嘴,说道:“罢,罢,罢,不用说这些话了。
”袭人深知宝玉性情古怪,听见奉承吉利话又厌虚而不实,听了这些尽情实话又生悲感,便悔自己说冒撞了,连忙笑着用话截开,只拣那宝玉素喜谈者问之。
先问他春风秋月,再谈及粉淡脂莹,然后谈到女儿如何好,又谈到女儿死,袭人忙掩住口。
宝玉谈至浓快时,见他不说了,便笑道:“人谁不死,只要死的好。
那些个须眉浊物,只知道文死谏,武死战,这二死是大丈夫死名死节。
竟何如不死的好!必定有昏君他方谏,他只顾邀名,猛拚一死,\zhu{拚:同“拼”,舍弃,不顾惜一切。
}将来弃君于何地!必定有刀兵他方战,猛拚一死,他只顾图汗马之名,将来弃国于何地!所以这皆非正死。
”袭人道:“忠臣良将,出于不得已他才死。
”宝玉道:“那武将不过仗血气之勇,疏谋少略,他自己无能,送了性命,这难道也是不得已!那文官更不可比武官了,他念两句书汙在心里,\zhu{汙:同“污”,停积不流的水,这里引申为动词,积聚、郁结,不得发泄。
}若朝廷少有疵瑕,他就胡谈乱劝,只顾他邀忠烈之名,浊气一涌,即时拚死,这难道也是不得已!还要知道,那朝廷是受命于天,他不圣不仁,那天地断不把这万几重任与他了。
\zhu{几:同机。
“万机”即万事形容皇帝政务繁多,“日理万几”的意思。
}可知那些死的都是沽名,并不知大义。
\meng{此一段议论文武之死,真真确确,的非凡常可能道者。
}比如我此时若果有造化,该死于此时的,趁你们在,我就死了,再能够你们哭我的眼泪流成大河,把我的尸首漂起来,送到那鸦雀不到的幽僻之处,随风化了,自此再不要托生为人,就是我死的得时了。
”袭人忽见说出这些疯话来,忙说困了,不理他。
那宝玉方合眼睡着,至次日也就丢开了。
\par
一日,宝玉因各处游的烦腻,便想起《牡丹亭》曲来,自己看了两遍,犹不惬怀,\zhu{惬怀:称心如意。
}因闻得梨香院的十二个女孩子中有小旦龄官最是唱的好,因着意出角门来找时,只见宝官玉官都在院内,见宝玉来了,都笑嘻嘻的让坐。
宝玉因问:“龄官独在那里?”众人都告诉他说:“在他房里呢。
”宝玉忙至他房内,只见龄官独自倒在枕上,见他进来,文风不动。
\zhu{文风不动:纹丝不动。}
\meng{另有风味。
}宝玉素习与别的女孩子顽惯了的,只当龄官也同别人一样,因进前来身旁坐下,又陪笑央他起来唱“袅晴丝”一套。
\zhu{“袅晴丝”一套:袅晴丝是《牡丹亭·惊梦》中第一支曲《步步娇》的首三字。
“袅晴丝”一套,代指《惊梦》一出的曲子。
}不想龄官见他坐下,忙抬身起来躲避,正色说道:“嗓子哑了。
前儿娘娘传进我们去,我还没有唱呢。
”\ping{第十八回,元妃省亲,龄官拒绝唱《游园》、《惊梦》,因为这两出原非本角之戏。
从十八回可见龄官是一个很有个性的戏曲演员,这里龄官提及自己前儿不给娘娘唱戏,今天也不给宝玉唱戏,基于十八回的铺垫,也就没那么突兀了。
}宝玉见他坐正了,再一细看,原来就是那日蔷薇花下划“蔷”字那一个。
又见如此景况,从来未经过这番被人弃厌,自己便讪讪的红了脸,只得出来了。
宝官等不解何故,因问其所以。
宝玉便说了,遂出来。
\meng{非龄官不能如此作势,非宝玉不能如此忍[耐]。
其文冷中浓,其意韵而诚,有“富贵不能移,威武不能屈”之意。
\zhu{
富贵不能淫,贫贱不能移,威武不能屈:出自《孟子·滕文公下》,是孟子与弟子谈论“大丈夫”的时候,提到的三句话。
意思是说,真正的大丈夫,富贵不能使他腐化堕落,贫贱不能使他改变志向,武力也不能使他屈服。
}
}宝官便说道:“只略等一等,蔷二爷来了叫他唱,是必唱的。
”宝玉听了,心下纳闷,因问:“蔷哥儿那去了?”宝官道:“才出去了,一定还是龄官要什么,他去变弄去了。
”\par
宝玉听了,以为奇特,少站片时,果见贾蔷从外头来了,手里又提着个雀儿笼子,上面扎着个小戏台,并一个雀儿,兴兴头头的往里走着找龄官。
见了宝玉,只得站住。
宝玉问他:“是个什么雀儿,会衔旗串戏台?”贾蔷笑道:“是个玉顶金豆。
”
\zhu{
玉顶金豆:鸟名。
《燕京岁时记》:“交嘴者,长四五寸,嘴左右交,以别雌雄,有红黄二色。驯而优者能开锁衔旗。”
“祝顶红者,小于家雀而红其顶,技如交嘴而灵巧过之。”
玉顶金豆当即祝顶红之俗名。
}
宝玉道:“多少钱买的?”贾蔷道:“一两八钱银子。
”一面说,一面让宝玉坐,自己往龄官房里来。
宝玉此刻把听曲子的心都没了,且要看他和龄官是怎样。
只见贾蔷进去笑道:“你起来,瞧这个顽意儿。
”龄官起身问是什么,贾蔷道:“买了雀儿你顽,省得天天闷闷的无个开心。
我先顽个你看。
”说着,便拿些谷子哄的那个雀儿在戏台上乱串,衔鬼脸旗帜。
众女孩子都笑道“有趣”,独龄官冷笑了两声,赌气仍睡去了。
贾蔷还只管陪笑,问他好不好。
龄官道:“你们家把好好的人弄了来,关在这牢坑里学这个劳什子还不算,
\zhu{劳什子:指令人讨厌的东西。}
\ping{在古代,戏子地位很低。
}你这会子又弄个雀儿来,也偏生干这个。
你分明是弄了他来打趣形容我们,还问我好不好。
”贾蔷听了,不觉慌起来,连忙赌身立誓。
又道:“今儿我那里的香脂油蒙了心!
\zhu{香脂油蒙了心:形容迷了心窍,做事糊涂。}
费一二两银子买他来,原说解闷,就没有想到这上头。
罢,罢,放了生,\zhu{放了生:释放鸟兽虫鱼等类小生物,佛家视为善举,认为可积阴德。
}免免你的灾病。
”\meng{此一番文章从“划蔷”而来,“蔷”之划为不谬矣。
}说着,果然将雀儿放了,一顿把将笼子拆了。
\zhu{一顿把:这是一个不可分拆的词语,犹言“一下子”、“一鼓作气”。
}\ping{囚禁的小鸟重获自由,可能暗示了龄官的结局。
}龄官还说:“那雀儿虽不如人,他也有个老雀儿在窝里,你拿了他来弄这个劳什子也忍得!今儿我咳嗽出两口血来,太太叫大夫来瞧,不说替我细问问,你且弄这个来取笑。
偏生我这没人管没人理的,又偏病。
”说着又哭起来。
\ping{第三十一回,袭人被踢吐血,想着往日常听人说:“少年吐血,年月不保,纵然命长,终是废人了。
”}贾蔷忙道:“昨儿晚上我问了大夫,他说不相干。
他说吃两剂药,后儿再瞧。
谁知今儿又吐了。
这会子请他去。
”说着,便要请去。
龄官又叫“站住,这会子大毒日头地下,你赌气子去请了来我也不瞧。
”贾蔷听如此说,只得又站住。
宝玉见了这般景况,不觉痴了,这才领会了划“蔷”深意。
\meng{点明。
}自己站不住,也抽身走了。
贾蔷一心都在龄官身上,也不顾送,倒是别的女孩子送了出来。
\ping{龄官的脾气和黛玉很像,贾蔷哄龄官也像宝玉哄黛玉,宝玉可能也在这两人身上看到了自己和黛玉的影子,引导出之后的开悟。
}\par
那宝玉一心裁夺盘算,痴痴的回至怡红院中,正值林黛玉和袭人坐着说话儿呢。
宝玉一进来,就和袭人长叹,说道:“我昨晚上的话竟说错了,怪道老爷说我是‘管窥蠡测’。
昨夜说你们的眼泪单葬我,这就错了。
我竟不能全得了。
从此后只是各人各得眼泪罢了。
”\meng{这样悟了,才是真悟。
}\ping{本回宝玉对袭人说:“趁你们在,我就死了,再能够你们哭我的眼泪流成大河,把我的尸首漂起来。
”从得所有人的眼泪,到得一个人的眼泪,是宝玉从广泛的爱到专一的爱的转折点,回应了本回题目的下句“识分定情悟梨香院”。
}袭人昨夜不过是些顽话,已经忘了,不想宝玉今又提起来,便笑道:“你可真真有些疯了。
”宝玉默默不对,自此深悟人生情缘,各有分定,只是每每暗伤“不知将来葬我洒泪者为谁?”此皆宝玉心中所怀,也不可十分妄拟。
\ping{肯定是黛玉了。
}\par
且说林黛玉当下见了宝玉如此形像,便知是又从那里着了魔来,也不便多问,因向他说道:“我才在舅母跟前听的明儿是薛姨妈的生日,叫我顺便来问你出去不出去。
你打发人前头说一声去。
”宝玉道:“上回连大老爷的生日我也没去,这会子我又去,倘或碰见了人呢?我一概都不去。
\ping{对待亲戚,不好有厚薄之分。
}
这么怪热的,又穿衣裳,我不去姨妈也未必恼。
”袭人忙道:“这是什么话?他比不得大老爷。
这里又住的近,又是亲戚,你不去岂不叫他思量。
你怕热,只清早起到那里磕个头,吃钟茶再来,岂不好看。
”
宝玉未说话,黛玉便先笑道:“你看着人家赶蚊子分上,也该去走走。
”宝玉不解,忙问:“怎么赶蚊子?”袭人便将昨日睡觉无人作伴,宝姑娘坐了一坐的话说了出来。
宝玉听了,忙说:“不该。
我怎么睡着了,亵渎了他。
”一面又说:“明日必去。
”\par
正说着,忽见史湘云穿的齐齐整整的走来辞,说家里打发人来接他。
宝玉林黛玉听说,忙站起来让坐。
史湘云也不坐,宝、林两个只得送他至前面。
那史湘云只是眼泪汪汪的,见有他家人在跟前,又不敢十分委曲。
\ping{豪爽如史湘云,亦有忍辱负重,心事重重不敢说的时候。
}少时薛宝钗赶来,愈觉缱绻难舍。
还是宝钗心内明白,他家人若回去告诉了他婶娘,待他家去又恐受气,因此倒催他走了。
众人送至二门前,宝玉还要往外送,\ji{每逢此时就忘却严父,可知前云“为你们死也情愿”不假。
}倒是湘云拦住了。
一时,回身又叫宝玉到跟前,悄悄的嘱道:“便是老太太想不起我来,你时常提着打发人接我去。
”\ping{湘云回家宛如羊入虎口般不情愿,只想着赶紧被接出来,一方面可能是叔叔婶婶确实对自己不好,另一方面可能是大观园众姐妹实在让人不想离开。
}宝玉连连答应了。
眼看着他上车去了,大家方才进来。
要知端的,且听下回分解。
\ping{湘云暂时离开,使得下回大观园起诗社时而湘云不在,为湘云为了入社补写菊花诗做铺垫。
}\par
\qi{总评:绛芸轩梦兆是金针暗度法,夹写月钱是为袭人渐入金屋地步,梨香院是明写大家蓄戏,不免奸淫之陋。
可慎哉,慎哉!}
\dai{071}{绣鸳鸯梦兆绛芸轩}
\dai{072}{贾蔷龄官情意缠绵}
\sun{p36-1}{白玉钏亲尝莲叶羹,绣鸳鸯梦兆绛芸轩}{图右侧:宝玉要喝莲叶汤,王夫人命玉钏给宝玉送去。
宝玉因金钏之事愧疚于心,对玉钏备加体贴。
傅家来人请安,宝玉只顾说话,不慎将汤碗撞翻,自已被烫了手,只管问玉钏烫了哪里。
图左侧:转日,袭人加了月例,黛玉约了湘云来给袭人贺喜。
黛玉隔着纱窗见到,宝玉睡在床上,宝钗在身旁做针线,旁边还放着蝇帚子,便招手叫湘云快看。
}