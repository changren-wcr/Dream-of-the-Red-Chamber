\section*{探春远嫁}

1、荣宁街口(春)\par
春光明媚。街口的风筝摊、空竹摊、鲜花摊前,围聚着一簇簇的男妇老幼。
随风飘摇的各色风筝、上下抖动嗡嗡作响的空竹、绚丽的鲜花和人们的笑脸交映成为一幅春趣盎然的市井民俗图画。
薛蟠策马急驰而来。人们纷纷惊惶地躲避。
字幕(叠)第二十三集:探春远嫁

2、荣国府前角门\par
薛蟠下马,气喘吁吁地奔入角门。一个男仆带住喷着响鼻儿的马,莫名其妙地望着薛蟠的背影。

3、宝玉外书房门口\par
两颗用骨头做成的弹子飞向两、三丈高的空中,一只“蜡嘴”鸟斜刺里冲过来,轻巧地依次衔住弹子。
茗烟和四、五个小厮欢呼雀跃.茗烟抬起手来,打了个唿哨,“蜡嘴”应声飞来,稳稳地降落在茗烟的手臂上,
张嘴吐出两颗弹子。茗烟又从荷包里掏出一颗弹子,连同先前的两颗,一甩手,一齐向空中抛去。
“蜡嘴”“唰”地一下腾空飞起,优美地转动着身体,依次衔住三颗弹子。
茗烟兴奋地蹦了一个高,又一屁股坐在地下,朝“蜡嘴”一招手,打了个唿哨。
薛蟠从后面一把拎起茗烟。“蜡嘴”失去了目标,盘旋着落在茗烟的头上。
茗烟吃了一惊,忙回过头定神看了看薛蟠:“哦,是薛大爷……”
薛蟠大口喘着气:“快!快找宝玉!”茗烟头上顶着“蜡嘴”:“什么?”
薛蟠急切地:“快把宝玉找来!就说……就说北静王府来人了!”
茗烟面有难色:“……薛大爷,那一回你老人家叫我指着老爷哄宝二爷出来,过后差一点儿……”
薛蟠一瞪眼:“啪”地劈头就是一巴掌:“嚼你娘的蛆!快去!”
“蜡嘴”扑楞楞飞上了房檐。茗烟揉着脑袋朝房檐上瞥了一眼,嗫嚅着:“要是二爷问什么事儿……”
薛蟠抬腿朝茗烟屁股上踢了一脚,吼道:“出大事了!还不快去!”
茗烟连忙捂着屁股转身跑去。小厮们陪笑着踅过来:“薛大爷……”“薛大爷……”
“薛大爷,你老人家玩玩打弹雀么?”
薛蟠一跺脚:“滚!”小厮们一溜烟朝院外跑去。

4、王夫人院\par
两个小丫鬟打起正房的门帘,探春带着侍书,从房内走出。
玉钏儿笑嘻嘻地送出门外:“三姑娘走好!”
探春:“玉钏儿姐姐留步吧,我去看看二哥哥。”
玉钏儿:“宝二爷这会子怕是正用着功呢。”

5、宝玉房内\par
临窗大案上放着一个精巧、洁净的小石臼,案旁硕大的胆瓶里插满了一枝枝盛开的桃花。
宝玉端着一个精致的小竹筐。细心地选摘着一片片红红的花瓣。

6、宝玉房门外\par
探春带着侍书沿着抄手游廊走来。坐在门旁的麝月一眼瞥见,忙站起来。
探春含笑摇了摇手,轻手轻脚地径向房内走去。

7、宝玉房内\par
宝玉摘下一片花瓣,凑在鼻子上闻了闻,然后轻轻放入小筐内。
探春悄悄走近,不解地看着宝玉的举动。宝玉在案旁坐下,移过小石臼,细心地往里面吹了吹,
然后从竹筐里捏出几片花瓣放入臼中,拿起一根光滑的小石杵,慢慢地舂捣起来。
探春惊讶地看着,忍不住发问:“二哥哥做什么呢?”宝玉吃了一惊,忙把石臼和竹筐朝案下藏去。
探春“扑哧”一下笑了。宝玉回头看清是探春,也笑了:“是三妹妹,我还当是……”
探春忍住笑,接过小石臼,朝里面看了看:“这是做什么呢?”
宝玉:“做胭脂呀。”探春疑问地:“用桃花做胭脂?”
宝玉道:“不独桃花,四时花卉,凡是红的,都能做胭脂的。”
探春:“哦?”宝玉来了精神:“先要选好花瓣。你看,一朵花的花瓣也不是一样深浅的,
再把花瓣放在石臼里,慢慢舂成浆汁。再用细纱滤过……”
探春在案旁坐下,双手托起腮,饶有兴致地听着。宝玉拿起一小块蚕丝压成的圆饼:
“再把丝棉放在花汁里浸上五、六天,浸透了,就拿到太阳光下面去晒,晒干就能用了。……今儿是三月初一……”
探春无意中瞥见地下随便扔着的一张纸条,上面恭楷写着“朔望讲书”,不禁微微一笑。
宝玉沉吟着:“……二姐姐说话就该回娘家了。她临出阁的时候,我给她做的玫瑰胭脂,怕该用完了。”
探春笑了笑:“才刚我去太太房里,袭人姐姐正回太太说你用功呢。原来是用这个功呢!”
宝玉摇头晃脑地:“汉光武帝说的:‘吾乐此,不知疲’么!”
探春笑着又瞥了一眼地下的纸条。宝玉:“……这还是晴雯姐姐教给我的呢……”说到“晴雯”,宝玉的目光黯淡了……

8、宝玉外书房门口\par
薛蟠焦急地来回踱步。

9、宝玉房内\par
宝玉突然想起了什么,忙走到靠槅扇摆着的博古架旁,伸手拿下一个小盒,回到探春身后。
探春托腮凝思。宝玉打开小盒盖,拿出一个彩绘泥人儿,悄悄放在探春面前。
探春欣喜地捧起泥人儿。泥人儿是一个小胖丫头,双手托着腮,若有所思地歪着脑袋,
头发上的黑色、面颊上的胭脂红都稍稍错了位。然而,正是这仿佛漫不经心的几笔,透出了一种大拙之美。
宝玉:“三妹妹,象不象你?”探春想起自己刚才的样子,忍俊不禁,“扑哧”一下笑出声来:
“二哥哥,你从哪儿弄来的这个?”宝玉:“城里城外,大廊小庙有的是这个。”
探春:“二哥哥,你再去逛的时候,别忘了再给我带几个回来。还有象你上回买的那柳枝儿编的小篮子,
整竹子根枢的香盒儿,胶泥垛的风炉儿,我喜欢得什么似的,谁知她们都爱上了,都当宝贝似的抢了去了。”
宝玉笑道:“原来要这个。这不值什么,拿五百钱出去给小子们,管拉一车来。”
探春:“小厮们知道什么!你拣那朴而不俗、直而不拙的带些给我,我还象上回的鞋做一双你穿,比那一双还加功夫,好不好呢?”
宝玉笑了笑:“你提起鞋来,我想起个故事,那一回我穿着,可巧遇见了老爷,老爷就不受用,问是谁做的。
我哪儿敢提‘三妹妹’三个字!就回说是前儿过生日,舅母给的。老爷听了,才不好说什么,半日还说:
“何苦来!虚耗人力,作践绫罗,做这样的东西。”
探春看着宝玉绘声绘色地学着贾政,忍不住大笑起来。
宝玉更来了精神:“后来听袭人说,赵姨娘知道你给我做鞋,气得抱怨得了不得,说:‘环儿是她正经兄弟,鞋搭拉袜搭拉的没人看见,倒做这些东西!”
探春听说,登时沉下脸来:“这话糊涂到什么田地!怎么我是该做鞋的么?
环儿难道是没有分例的?没有人的?一般的衣裳是衣裳,鞋袜是鞋袜,丫头婆子一屋子,怎么抱怨这些话!
给谁听呢!我不过是闲着没事儿,做一双半双。爱给哪个哥哥兄弟,随我的心。谁敢管我不成!这也是白气!”
宝玉没想到探春动了气,略有些尴尬地:“……袭人也是听说,也不一定真。”
麝月进门:“二爷,茗烟来回话,说薛大爷有急事要找二爷。”宝玉:“哦?”

10、宝玉外书房门口\par
薛蟠焦急地来回踱步。“当!当!当!”二门上云板响了三下。
薛蟠忙走上甬道朝远处张望。一个小厮慌慌张张地拐过墙角迎面跑来,和薛蟠撞了个满怀。
薛蟠不提防。“扑”地一屁股坐在地上。小厮忙上来搀扶。
宝玉带着茗烟匆匆走来。薛蟠瞪起眼睛刚要骂,扭头看见宝玉。
宝玉:“怎么了?”小厮惶恐地:“……南安王府的太妃忽然来了。门上支使我传话……”
薛蟠一把拉住宝玉,扭头对小厮:“去去去!”
小厮忙转身跑去。薛蟠朝茗烟一扬下巴:“快去备马!”宝玉:“去哪儿?”薛蟠一拍大腿:“出大事儿了!”
宝玉微微一笑:“你又哄我。……今儿南安太妃来,我怕是出不去。”
薛蟠着急地伸出一只手比成“王八”形:“谁要哄你谁是这个!卫府上出大事儿了!”
宝玉:“谁家?”薛蟠:“咳!卫若兰卫大哥府上!”宝玉:“怎么?”
薛蟠:“旧年秋后南安王爷奉旨征西,卫老世伯不是带兵跟着去西海沿子了么?”宝玉:“不是说要奏凯班师了么?”
薛蟠:“什么奏凯!败了!”宝玉惊异地:“什么?”薛蟠:“王爷都让番兵给活捉了去了!”
宝玉:“那卫老世伯呢?”薛蟠悲怆地:“……战死了。”宝玉:“啊?”薛蟠抹了抹眼睛。
宝玉:“走!去看看卫大哥!……茗烟!备马!”茗烟答应一声,慌忙把“蜡嘴”塞进鸟笼子里关好。宝玉:“快去!”

11、荣国府正门\par
兽头大门缓缓开启,南安太妃的轿从进入大门。

12、荣国府前角门\par
宝王、薛蟠、茗烟策马驰出角门。

13、射圃·箭道\par
急促的马蹄声。弓弦响处,一根根直立的竹竿被利箭劈破。空旷的箭靶前,一个戎装挂孝的少年发疯似的往来骑射。
宝玉、薛蟠、茗烟策马而来,见状,一齐带住马,默默地看着。箭囊空了,少年一手挽弓,一手拔出佩剑,
纵马突入竹竿箭靶阵中,狠狠地左劈右砍。断裂的竹竿纷纷倒地。宝玉、薛蟠下马,把缰绳扔给茗烟,朝少年跑去。
少年纵马践踏着倒地的竹竿。宝玉、薛蟠高喊:“卫大哥!卫大哥!”
卫若兰纵马挥剑,在空中乱砍。宝玉:“卫大哥!”薛蟠奔过去,远远拦在马前:“卫大哥!”
卫若兰猛然带马。烈马长嘶,前蹄腾空而起。
薛蟠忙闪过一边。卫若兰泪痕满面,滚鞍下马,“扑通”一下跪在宝玉、薛蟠二人面前。二人急忙将卫若兰抱住扶起。
宝玉:“……我们去过府上了……”

14、大观园中一隅\par
桥头路旁绽开着各种不知名的花,红的、黄的、粉的、紫的……,随着阵阵微风轻轻地颤动着。
两个上了年纪的婆子坐在桥头,呆呆地望着不远处的一片竹林。竹林内,一群粗使仆妇嘻嘻哈哈地到处乱刨竹笋,并不时
随手折断碍事的竹枝。
探夫带着侍书远远走来。一个婆子叹了口气:“唉,糟蹋吧,糟蹋吧,索性我也不管了!”
另一个婆子:“唉,三姑娘管事时候许下的话,谁知道都不算数了。”
探春正要上桥,听见对面有人说到自己,不由地停下脚步。
两个婆子背对着探春,管自絮叨着:“一朝天子一朝臣吧!”
“可不是么!今儿荒腔,明儿走板儿,哪有个准头!”“白忙活了一场……”探春轻轻叹了口气。
“三姑娘!三姑娘!”周瑞家的从后面匆匆追来。两个婆子回头看见探春,忙立起来。
探春:“是周大娘,什么事?”周瑞家的:“南安王府的太妃来了,要见姑娘。”
“哦?”探春抬眼看了看左近的潇湘馆:“那好,我去找林姐姐她们。”
周瑞家的:“不用了,太太说,太妃单要见三姑娘。”探春疑问地看着周瑞家的。

15、射圃\par
卫若兰的雕弓和脱了鞘的佩剑扔在嫩绿如茵的草地上。茗烟牵着四匹马立在一边。卫若兰、宝玉、薛蟠席地而坐,嘿然无语。
残阳如血,子规声声。一阵急促的马蹄声自远而近。茗烟:“冯大爷来了!”
冯紫英翻身下马:“若兰!叫我好找!”宝玉、薛蟠忙站起来:“冯大哥!”
冯紫英:“你们二位也在!”卫若兰一动不动地坐着。冯紫英紧张地:“你们听说了么?朝廷要议和了!”
薛蟠瞪大了眼睛:“什么?”宝玉:“议和?” 

16、荣国府·荣禧堂\par

探春进门,抬眼看去。
南安太妃坐在上首,贾母陪坐在下首,邢夫人、王夫人侍立在贾母身后。
贾母强笑着招呼探春:“三丫头,快过来见过太妃。”探春忙上前拜见:“给太妃请安。”
南安太妃伸手扶起探春,慈爱地:“给孩子个座儿。”探春困惑地看了看贾母,又看了看站着的邢、王二夫人。

17、射圃\par
冯紫英:“……旨意说,着南安郡王的妹子和番……”薛蟠吃惊地:“嫁给番王?”
冯紫英:“……两边息了兵,就把南安王爷送回来。”宝玉着急地:“王爷就一个妹子,南安太妃怎么舍得嫁那么远?”
冯紫英:“听说……太妃要认个义女……”宝玉吸了口气,仿佛有一种不祥的预感。
卫若兰猛地一把抓起佩剑,狠命朝雕弓剁去。弓弦“啪”地一下崩断了。

18、荣国府·荣禧堂\par
惊呆了的探春仿佛失去了知觉,怔怔地坐在南安太妃的身旁。
南安太妃含着泪,期待地看着探春。贾母含泪看着探春。邢夫人微微瞥了一眼王夫人。王夫人含泪看着探春。
探春怔怔地坐着。两边的侍女、丫鬟人人屏声敛气、垂首肃立。空气仿佛凝结了。

19、门外\par
二十名华冠丽服的王府小厮抬着十抬用大红绫子覆盖着的礼品,静静地在阶下候立。

20、荣禧堂内\par
房内死一般地寂静。两边侍立的丫鬟们悄悄地交换着不安的眼神。贾母神色庄重。正面悬挂的“待漏随朝墨龙大画”上,
云雾海潮仿佛在悄悄逝去,巨龙的钩爪锯牙渐渐逼来。贾母苍老、干涩的声音颤抖着:“……三丫头,给太妃磕……磕头……”

21、街道(黄昏)\par
宝玉、茗烟策马急驰而来。

22、荣国府正门\par
兽头大门开启,一对对映着“南安郡王府”字样的灯笼自门内挑出。南安王妃的轿从缓缓出门。

23、荣国府前角门\par
宝玉、茗烟翻身下马。

24、府内甬道\par
宝玉匆匆走来。 

25 大观园前角门\par
宝玉匆匆进门。守园婆子:“二爷快着出来,过会子就要关门了!”宝玉好象没听见一祥,径直朝园内走去。

26、怡红院门外\par
暮霭笼罩着荒落空寂的墙垣。紧紧关闭着的院门外,虫飞草长,花落莺啼。
宝玉不由地放慢了脚步,目光里透出了几分怅惘。

27、潇湘馆·黛玉房内\par
一支新蜡摇曳着烛光,照亮了文案上一厚叠写满了蝇头小楷的老油竹纸。
黛玉手搦湘管,伏案临帖。紫鹃一面用小蜡剪剔着烛芯,一面略带埋怨地看了看林黛玉:
“看不清了,姑娘明儿白天再临吧!”
黛玉头也不抬:“就这几个字了。”紫鹃:“舅老爷上任还不到半年,且回不来查问他的功课呢!
姑娘犯不上这么早就急着替他预备这个,……病了一冬,才大安了几天儿!”
黛玉搁下笔,嫣然一笑:“话多!”紫鹃一面小心地收拾着案上的纸笔砚墨,一面笑嘻嘻地叨唠:
“说真个的,打从宝玉搬出园子,姑娘这一场病,耗了他多少心神!跑东跑西、寻方觅药的不算,回回来了,
赶上姑娘吃药,他总要先偷偷尝尝苦不苦,姑娘漱口,他总要先悄悄试试烫不烫。
回回要走了,总要站在院子里独自叨咕一会子。不知道的,都笑他呆,我留心听了听,才知道他是为姑娘祈祷呢……”
黛玉渐渐收了笑容,怔怔地看着紫鹃。紫鹃也收了笑容,同情地看着黛玉,半晌,缓缓开口道:“姑娘又何尝不是呢?
病着的时候,他来了,就好些;一天不来,就重些……”
黛玉的泪珠扑簌簌滚落下来。紫鹃:“……镇日里,见不着也哭,见着也哭……”黛玉:“别说了……”
紫鹃:“……有一年到头苦自己的,不如两个人干脆捅破了窗户纸,再商量个长久的法子……”
黛玉睁大了泪眼,惊愕地看着紫鹃,说不出话来:“你……”
紫鹃含着泪:“论理,我不该说这个话。可……可不这么着,谁给姑娘做这个主呢?……”说着,忍不住哽咽泪下。
宝玉匆匆进门,猛然看见黛玉和紫鹃相对悲泣,不禁忡然失色:“……怎么……”
紫鹃连忙抹了抹眼睛:“哦,是二爷!……怎么这会子来了?”宝玉惊惶地:“林妹妹!怎么了?”
黛玉站起来,“呜”地一声,掩面走入内室。宝玉一把抓住紫鹃的手臂:“是不是……南安太妃……”紫鹃不解地:“什么?”
宝玉:“林妹妹今儿……出去过?”紫鹃摇摇头。宝玉:“……有谁来过?”紫鹃摇摇头。
宝玉略松了一口气:“怎么好好的又哭?”紫鹃看着宝玉,欲言又止,轻轻地叹了口气。
宝玉急忙走进内室。紫鹃轻手轻脚地走出房外,无力地靠在廊柱上……

28、大观园前角门(晚)\par
袭人掩饰着不安,含笑询问守园婆子:“……宝二爷……进园子了?”守园婆子点了点头:“进去有一个多时辰了。”

29、沁芳桥\par
袭人匆匆过桥。

30、潇湘馆院内\par
紫鹃迎下台阶:“袭人姐姐!”袭人:“宝二爷……来过?”紫鹃点点头:“姐姐快请进来。”

31、房内\par
紫鹃拉着袭人边进门边说:“……慌慌张张地来了,没坐一会子。就叫上林姑娘去秋爽斋了。”
袭人:“哦?”紫鹃疑问地看着袭人:“袭人姐姐,好象……出什么事情了?”
袭人略一沉吟:“……今儿午错时候,南安郡王府的太妃来了,一定要认三姑娘做闺女……”
紫鹃诧异地:“有这么好的事?”袭人微微冷笑:“好事?”紫鹃:“怎么?”
袭人:“南安王爷在西海沿子打了败仗,朝廷要议和,南安太妃认咱们三姑娘,是为的嫁给番王做王妃的!”
紫鹃吃惊地:“啊?”袭人:“已经择定了三月十六上路……”紫鹃呆看着袭人,说不出话来。袭人叹了口气。
紫鹃喃喃地:“……这么说,三姑娘在家里也就是十几天的住头了?”
袭人苦笑笑:“明儿初二还是她的生日……”

32、秋爽斋院门外\par
宝玉提着灯笼笑嘻嘻地对翠墨:“……三妹妹回来你告诉她,我和林妹妹不等她了。”
黛玉:“明儿一早过来给她拜寿。”翠墨点点头。宝玉、黛玉转身离去。

33、园中路上\par
宝玉一手提着灯笼照路,一手搀着黛玉,小心翼翼地走来。
黛玉:“我说不会有什么事吧!三妹妹真让南安太妃找去了,翠墨会不知道?你真是个无事忙!”
宝玉孩子气地凑过去,冲着黛玉微微一皱鼻子,甜甜地笑了。

34、潇湘馆\par
袭人心神不定地:“这么晚了,怎么还不回来?”说着,站了起来。
紫鹃笑嘻嘻地把袭人按在椅子上:“丢不了!”

35、园中路上\par
黛玉突然停住了脚步,紧张地拉了宝玉一把,“宝玉,你看!”宝玉抬眼朝前面看去。远处,五、六只灯笼晃过桥头。
宝玉不假思索地:“许是三妹妹。走,咱们迎迎!”黛玉:“别!……要是查夜的呢?这早晚碰见,又不知道说出什么来!”
灯笼在花木掩映的曲径上时隐时现地闪烁着,迎面而来。宝玉忙拉着黛玉朝路旁的荼靡架下躲去。
灯笼、人群越来越近。荼靡架下,黛玉拉拉宝玉,指了指宝玉手中的灯笼。
宝玉张嘴就要吹,黛玉连忙阻止,抬手解开自己的披风扣子。
宝玉忙把灯笼的提杆插在荼靡架上,腾出手来,给黛王扣好披风,接着解下自己的披风围在灯笼上,把烛光遮住。
二人悄悄回过头去,从荼靡蔓叶的缝隙里向路上窥看。路上,周瑞家的带着几个婆子、媳妇簇拥着毫无表情的探春缓缓走过。
人群中,傻大姐笑嘻嘻地左顾右盼。宝玉、黛玉二人惊魂甫定,相视一笑。

36、潇湘馆\par
袭人不安地立起:“不行,我得去找找!”紫鹃忙掩饰地笑着:“再等等!……别找岔了道。”

37、荼靡架下\par
左近的石罅里泻出一脉清泉,静无声息地缓缓流去。微风轻轻拂过花木,落红委地,宛如阵阵花雨。
偶尔响起的“沙沙”的枝叶声,仿佛是闺中少女伤春的叹息,给暮春的夜晚又增添了几分静谧。
宝玉把围着灯笼的披风慢慢移开。柔和的烛光在黛玉含情脉脉的眸子里微微闪动。宝玉甜甜地看着黛玉。
一阵微风拂来,黛玉的披风皱起了条条波纹。宝玉拿起自己的披风凑近黛玉的双肩。
黛玉含笑摇了摇头,接过披风,轻轻披在宝玉肩上。二人默默无语地对视着。
宝玉借着烛光朝地下看了看,脱下披风,折了两折,铺在一块紧靠着荼靡架的长方石上。
黛玉的眼睛里闪过一丝慌乱。宝玉示意黛玉坐下。黛玉连忙摇摇头,背过脸去。宝玉茫然地看着黛玉。
良久,黛玉缓缓转过脸来,怔怔地看看宝玉,慢慢坐在石上。
宝玉不由自主地跟着坐下。宝玉忘情地拉起黛玉的一只纤手。黛玉微微一震,连忙把手抽了回来。
宝玉惶惑地看着黛玉。黛玉的双肩微微抽动着,泪水涌出眼眶。宝玉不知所措地慢慢缩回双手。
黛玉双目紧闭,泪水簌簌而下。宝玉惊惶不安地看着黛玉。黛玉慢慢睁开泪眼看着宝玉。
宝玉欲言又止。黛玉微微颤抖着伸过一只手来,指尖犹豫地触到宝玉的手,不由地打了一个哆嗦。
宝玉轻轻握住黛玉的手。烛影里,微风摇曳着荼靡枝蔓。
石罅旁,汨汨流逝着粼粼碧波……

38、路上\par
袭人、紫鹃匆匆走来。袭人抬眼看见不远处荼靡架下的灯光,愣了一下,不由地加快了脚步。
紫鹃在后面暗暗着急。袭人蹑手蹑脚地绕到荼靡架近处的一块山石后面,小心翼翼地伸头看去,不禁大吃一惊。

39、荼靡架下\par
宝玉、黛玉坐在石上,手拉着手,眼睛里噙着泪花,甜甜地含笑对视。

40、山石后面\par
袭人睁大了眼睛,不知所措地呆立着。紫鹃不安地偷觑着袭人。突然,对面路上传来傻大姐的一声惊叫:“哎呀!有鬼!”
袭人、紫鹃吃惊地抬头。

41、荼靡架下\par
宝玉、黛玉吃惊地回头。

42、路上\par
傻大姐惊恐地往后退着:“鬼!鬼……”周瑞家的并几个婆子、媳妇一齐站住,手中的灯笼抖个不住。

43、荼靡架下\par
宝玉惊惶地拉起黛玉朝近处的山石跑去。

44、路上\par
周瑞家的颤声呵斥傻大姐:“别……别出声!”
傻大姐不顾一切地惊叫着:“看!一个男鬼一个女鬼!”

45、山石后面\par
宝玉拉着黛玉转过山石,猛然看见面前的袭人、紫鹃,吃惊地“啊”了一声。
袭入连忙捂住宝玉的嘴,哆嗦着:“二爷别出声!”紫鹃一把搂住黛玉:“姑娘,是我!”

46、路上\par
傻大姐趴在地下,把头插在众人的腿缝里,撅着屁股,筛糠似地哆嗦着。
周瑞家的壮壮胆子,颤声高喊:“是谁?快出来!”

47、山石后面\par
紫鹃一只手搂着黛玉,一只手推着宝玉:“二……二爷,快跑!快出去!”
宝玉慌乱地拉着黛玉:“……林妹妹……”紫鹃急切地:“姑娘有我呢,你快跑!”宝玉犹豫着。
紫鹃:“快跑吧!出不去园子可就……”宝玉慌忙朝一条小路上跑去。

48、路上\par
众人七嘴八舌地嚷着:“跑了!”“跑了!”“一个!”“还有一个呢?”

49、山石后面\par
袭人又气又怕地瞪了紫鹃一眼,抬腿要跑。紫鹃低声惊呼:“不好!灯笼!”袭人忙回头朝荼靡架看去。 

50、荼靡架下\par
一只灯笼寂寞地在微风中摇晃着。

51、山石后面\par
袭人咬了咬嘴唇,“唿”地一下从山石后面闪出来,朝荼靡架奔去。

52、路上\par
众人纷纷惊惧地后退。

53、荼靡架下\par
袭人一把扯下灯笼,扭头朝岔路上跑去。

54、山石后面\par
紫鹃屏声敛气,紧紧地搂着黛玉。

55、大观园前角门\par
宝玉匆匆跑出角门。守园婆子惊疑地看着宝玉的背影。

56、荼靡架下\par
五六只灯笼聚在一起,照亮了宝玉、黛玉坐过的石块。周瑞家的俯身捡起落在石上的披风,借着灯光一看,不禁愕然失声:“啊?!”

57、王夫人房内(晨)\par
王夫人倏地立起,颤声惊问:“什么?!”周瑞家的捧着宝玉的披风站在下面,偷觑着王夫人的脸色。
王夫人重重地跌坐在椅子上,两眼直直地盯着披风。外面不远处,鼓乐之声大作。
凤姐满面春风地走进房门,高声笑着:“请太太安!南安太妃已经到了荣禧堂了,北静王妃和各府诰命也都贺喜来了,
老太太诸太太就过去!”王夫人木然看着凤姐。凤姐惊疑地看看王夫人,又看看周瑞家的,目光落在披风上。
周瑞家的躲着凤姐的目光。凤姐指着披风刚要发问。王夫人缓缓立起,声音喑哑:“……走吧……”
凤姐和周瑞家的连忙上前搀扶。

58、廊下\par
凤姐、周瑞家的搀着王夫人走出房门。林之孝家的上前垂手侍立。凤姐:“什么事?”
林之孝家的:“回二奶奶,孙府的人送咱们家二姑娘回来省亲,已经到了。”
凤姐:“你先打发孙家的人回去,安顿二姑娘在西小院歇着,这会子正忙,下半日再见吧。”
林之孝家的“是”了一声,转身离去。

59、荣禧堂\par
正面大紫檀雕螭案上,中间设着一个大大的青铜鼎炉,旁边是一个珊瑚色剔剑环香盒,另一边是一个宜德铜铸筯瓶。
鼎炉里升起缕缕轻烟。贾赦、贾珍、贾琏等跪在下面。大明宫掌宫内相戴权站在上面,庄严地除下明黄缎套,徐徐展开了一幅三尺长的手卷,上面是墨泽光润的回个恭楷大字:“惠我西黎”,后有一行小宇并“万几宸翰之宝”。
贾赦、贾珍、贾琏等喜不自禁,连连叩首。噪耳的鼓乐声从门外传来。

60、宝玉房内\par
“哗啦”一下,笔、墨、纸、砚落了满地。宝玉一拳砸在临窗大案上,泪流满面,恨恨地:“……国家有难,满朝文武都是干什么用的?!竟让一个弱女子去和番……”
袭人慌忙上来捂他的嘴:“别嚷嚷,小祖宗!让人听见可就惹大祸了!”
宝玉扒拉开袭人:“我去看看三妹妹!”袭人一把抱住宝玉:“求求你,二爷!这几天躲躲,可千万不能再进园子去了!”
宝玉挣脱袭人,抹着眼泪跑出房门。袭人哭着追到门口:“二爷!”

61、秋爽斋·探春书房\par
花梨大理石案上放着一方端砚。探春轻轻吹了吹诗稿上的墨迹,把手中的湖笔架在笔山上,无意中,目光落在笔山旁的一只竹雕签筒上。
竹雕签筒里插着一根象牙花名签子,探春伸手掣出。签上画着一枝杏花。探春默念上面镌着的一句唐诗:
(心声)“日边红杏倚云栽。”(闪回)少女们清脆的笑声。
“我还不知道得个什么呢!”探春伸手从竹雕签筒里掣出一根花名签子,上下看了看,红着脸扔在地下:
“这东西不好,不该行这个令。”袭人忙拾起签子念道:“日边红杏倚云栽。”
湘云一把抢过签子念道:“得此签者,必得贵婿,大家恭贺一杯,共同饮一杯!”
众人大笑大叫。湘云高声:“我们家已经有了个王妃,难道你也是个王妃不成?大喜大喜!”
众人大笑大叫。(闪回完)探春长吁了一口气,把花名签子插回签筒。宝玉进门,走到探春身旁:“三妹妹。”
探春:“哦,二哥哥。”宝玉掩饰地拿起案上的诗稿,强笑笑:“……好俊气的蝇头小楷!是……才抄的?”
探春默默点了点头。宝玉翻开诗稿的扉页,一愣,看了探春一眼,默念着(心声):“娣探谨奉二兄文几……”
(闪回)探春的丫头翠墨递给宝玉一副花笺。宝玉展读:“娣探谨奉二兄文几……”
宝玉走进秋爽斋院门。黛玉笑吟吟地:“又来了一个。”李纨,宝钗、迎春、惜春含笑转身。
探春:“我偶然起了个结诗社的念头,写了几个帖子试一试,谁知一招皆到。”
宝玉:“早该起个诗社了。”侍书在探春书房内的花梨大理石案上摆好四份纸笔。
司棋点着一支三寸长的“梦甜香”擎在手里。宝钗端坐凝思。
宝玉背着手在回廊下踱来踱去。黛玉蹲着用手指划地。探春俯案疾书。
“梦甜香”燃剩寸许。探春搁笔。众人围看,宝手念道:“咏白海棠……”(闪回完)
宝玉轻轻合上诗稿:“这不都是咱们诗社中的诗吗?你这是?……”
探春黯然一笑:“带走。”宝玉惊异地看着探春。
探春:“带走。……早晚翻翻,权当……跟这园子里的人又见面了。”
宝玉眼圈儿一红,半晌,哽咽着:“……晴雯死了,司棋死了,香菱死了,柳五儿死了,入画和四儿撵出去了,
芳官儿她们出家了,我和宝姐姐搬走了,二姐姐嫁人了,四妹妹除了诵经打坐百事不问,
邢姐姐、琴妹妹和李家姐妹也都各自去了,湘云妹妹眼看有了人家,也不来了,说话你又要走……”
探春含泪:“……二哥哥,自古以来多少豪门望族,有几个捱过了百年的?灌、绛、王、谢方盛之时,谁又能想到日后的瓦解冰消?‘君子之泽,五世而斩。’不独这个园子,就怕连咱们这个家……也有那一天!”
宝玉伏案饮泣。探春黯然神伤,自言自语地:“……再过几天又是清明了,多想再放一回风筝……再结一次……诗社……”
宝玉双肩耸动,忍不住哭出声来。探春拿起诗稿递过来:“……二哥哥,再……写点儿什么吧!”
宝玉慢慢抬起头来。探春深情地:“二哥哥……”宝玉接过诗稿翻开,提笔蘸了蘸墨,忍住眼泪,在后面空白行格处奋笔疾书:
人间几度清明,一编书是英雄泪!……
歌声起(无字的歌)。歌声里:两盆盛开的白海棠如堆冰砌雪。四张誊清的诗稿依次摞在花梨大理石案上。
紫蟹黄菊。几支湖笔次第勾去用针绾在墙上的诗题:《忆菊》、《问菊》……白雪红梅……寒塘孤鹤……冷月残花……

62、宝玉房内\par
“啪”地一声,王夫人拍案而起。袭人、麝月等丫头、婆子跪了一地。王夫人泪流满面,拍着桌子:
“……我白养着你们干什么用的?!”众人跪在地下,大气也不敢出。
王夫人:“宝玉呢?去哪儿了?”没有人吭声。
王夫人怒不可遏地:“袭人!”袭人打了个哆嗦,簌簌流泪,说不出话。玉钏儿进门,偷偷瞥了袭人一眼,趋上几步:
“太太,琏二奶奶带着二姑娘给太太请安来了。”
王夫人指着众人:“都不许动!”说着,用手帕擦了擦眼睛,扶着玉钏儿出门。

63、贾母院垂花门\par
一群婆子、媳妇、小厮围挤成一堆,听傻大姐比比划划地说着什么。
邢夫人刚要出门,听见门外众人“啧啧”的惊叹声,停住脚步。
傻大姐神秘地压低了声音:“……周大娘不让说是宝二爷……”

64、王夫人房内\par
王夫人心神不定地坐着,眉头紧锁。面容憔悴的迎春正哭哭啼啼地诉说着:“……姓孙的简直不是个人!
一味好色,家里所有的媳妇丫头都……,我略劝一劝,就骂我是……是‘醋汁子老婆拧出来的’。”
凤姐偷觑着王夫人的脸色。迎春:“……成日里指着我的脸骂:‘你别和我充夫人娘子,你老子使了我五千银子,
把你准折卖给我的!’……好不好,就打一顿……撵在下房里睡去……”
王夫人“啪哒啪哒”落下泪来。迎春挽起袖子,露出臂膊上一处处青紫的伤痕。凤姐噙着泪花咬着牙根骂道:
“黑了心的野种!下这么狠的手!”王夫人心疼地抚摸着迎春臂膊上的伤痕,唏嘘着说不出话来。
绣桔立在一旁默默垂泣。玉钏含泪拉起绣桔的一只手。
王夫人:“……碰上了这种不知好歹的人,可怎么办呢?当初你叔叔也劝过大老爷,大老爷不听,
执意要作这门亲。唉!我的儿,这也是你的命!”
迎春抽泣着:“我不信我的命就这么不好!从小没了娘,幸而过婶子这边过了几年心净日子,如今偏又是这么个结果!”
王夫人、凤姐等泪眼相向,无语凝噎。

65、宝玉房内\par
房内一片饮泣之声。袭人、麝月并丫鬟婆子们仍旧跪在原处。袭人的眼角挂着泪珠,怔怔地想着什么。

66、王夫人房内\par
迎春离座含泪裣衽:“……太太,我走了。”王夫人、凤姐起身送到房门口。
凤姐强笑着问迎春:“二妹妹想歇在哪一处呢?”
迎春:“我……还记挂着我的屋子,还能在园里旧房子里住三五天,死也甘心了。……不知道还有没有下次了……”
说着:“呜”地一声,转身出门。绣桔忙跟上搀扶。玉钏儿跟出。王夫人叫住凤姐:“你等等,……有话跟你说。”

67、宝玉房内\par
袭人咬了咬嘴唇,慢慢从地下站了起来。众人惊愕地看着袭人。袭人挪动着跪麻了的双脚,慢慢朝房门走去。
麝月:“袭人姐姐……”袭人回过头来环视着众人,凄惋地笑笑,摇了摇头,泪水“啪哒啪哒”滴落下来。

68、王夫人房内\par
邢夫人神色惶惶地匆匆进门,嘴里叨咕着:“……这是怎么说的!这是怎么说的……”
王夫人和凤姐忙起身让座。邢夫人一面落座,一面没头没脑地说着:“唉呀呀!我从老太太房里出来,
这一路上,到处是三个一堆、五个一伙,……唉!”王夫人和凤姐紧张地对视。
王夫人迟疑着问邢夫人:“……怎么?”邢夫人朝房门外:“进来!”
傻大姐冒冒失失地进门,在门槛上绊了一下,就势趴在地下磕了个头。
王夫人和凤姐吃了一惊。邢夫人指着傻大姐:“你说说,你都胡唚了些什么!”
傻大姐跪直了身子,诧异地争辩:“我没胡唚,这都是真的!”邢夫人:“打嘴!”
傻大姐:“什么?”随即醒悟过来,“噢”了一声,举起手来,在自己脸颊上拍了一下,委屈地:
“都怨老太太,昨儿晚上单叫我去给三姑娘打灯笼。我不怕人,就怕鬼!周大娘骂了我一路,说‘狗眼这么尖!’,
还说要挖了当鱼泡踩,还不许告诉老太太。早知道不是鬼是宝二爷,我就……”
王夫人脸色煞白:“你……看清楚了?”傻大姐摇了摇头:“就看见一个男的一个女的,……过后想想,
那个男的象是宝二爷,那个女的象是……”王夫人颤声:“……是谁?”傻大姐瞟了一眼凤姐。凤姐狠狠地瞪着傻大姐。
傻大姐害怕地:“是……是……”邢夫人一拍桌子:“是谁?”“是我!”袭人突然出现在房门内。邢夫人、王夫人、凤姐愕然膛目。
傻大姐困惑地晃了晃脑袋。袭人慢慢跪下,半晌,抬起头来:
“……是我没廉耻,勾着宝二爷到……荼靡架底下去的。……二爷怕太太不放心,我还强拉着不让回来。如今……带累了二爷的声名品行。
太太要打要杀,我一个人领。不干……二爷的事……”

69、宝玉房内(晚)\par
宝玉进门:“袭人姐姐!”没有人答话。宝玉:“袭人姐姐!”
麝月从内室走出:“二爷。”宝玉:“袭人姐姐呢?”
麝月抽抽答答地:“……让太太给……关起来了。”

70、空屋内(晚)\par
一灯荧荧,把袭人的身影长长地投在地上。袭人手里攥着一条汗巾子,站在屋子当中,两眼直直地盯着房梁,
泪水簌簌流下。门外“哗啦”一声锁响。袭人忙把汗巾子藏在身后。门“吱”地一下开了,玉钏儿一手端着一盏明瓦防风灯台,
一手搀着王夫人走进门来。玉钏儿把灯台放在桌上,房内立刻亮了许多。玉钏儿退出门外,随手带上房门。
袭人看着王夫人,“扑通”跪下。王夫人忙上前搀起袭人,一把搂在怀里,哽咽着:“我的儿,你受委屈了!”
袭人“呜”地一声:“太太……”王夫人摩挲着袭人:“我知道,那种事……断不是你做得出来的,你是为……保全宝玉。……
好孩子,你保全了他,也就是保全了我……”袭人依在王夫人怀里“呜呜”地哭着。王夫人动情地搂着袭人。
袭人强忍住泪,搀着王夫人坐在椅子上,自己立在一边,半晌,抬起头来:“太太,我知道,把二爷洗清白了,
我也就留不住了。有几句话本不该说的,可……这会子不说,就没日子说了……”说着,又滴下泪来。
王夫人:“好孩子,你说吧。”袭人吁了一口气:“第一件,二爷屋里的丫头们都大了,我走了,没个人管着,难保不生事,
求太太早早都放出去吧。只是……好歹留着麝月,这些年,我留心看着,只有她……能替我……长长远远地伺候二爷……”
王夫人含泪点头。袭人:“第二件,二爷和林姑娘也都大了,这一回算是万幸,遮过去了。可咱们府里的那起小人,
太太知道,口里眼里心里都够使的,难说不让他们看出破绽来。再说,太太也不能总看着二爷不进园子里去。
怎么想个法子,让二爷远远地离开家,到别处过一阵子就好了……”
王夫人一把攥住袭人的手:“我的儿,你可算是跟我想到一块儿去了!眼下只有一个法子,探丫头出阁,过几日就要上路,
妹妹和番哥哥送嫁,不说前朝,本朝也有这种故事的。今儿又听北静王妃说,旨意简派了北静王爷做送嫁钦差,路上不怕不照应。
只是……路太远,他又没出过门,……老太太怕也不答应,……他自己再使性子不去……,唉……”
袭人:“……不说……老爷的任所在西边么?……不知道路过不路过?”
王大人眼睛一亮:“你往下说!”袭人:“……我盘算着,就是不路过,也该差不远。太太先捎个信去,求老爷去路上等着看三姑娘一眼,就便把二爷接去……”
王人人含着泪连连点头。袭人:“……老太太是明白人,索性把二爷的事都回明了……”
王夫人含着泪连连点头。袭人:“……跟二爷就说送一程,个把月就回来,也算尽了兄妹的情分……”
王夫人哽咽着:“好孩子,好孩子!平日里只说你笨笨的,谁知道节骨眼儿上……难为你想得这么周全……”
袭人流着泪:“……也只有……这一回了……”
王夫人:“好孩子,你再委屈几日,我这一两天就打发人找你哥哥嫂子来接你……”
袭人“扑通”跪下,放声大哭:“太太……”

71、荣国府正门(晨)\par
鞭炮轰响,鼓乐齐鸣。兽头大门缓缓开启。

72、大观园前角门\par
宝玉把紧闭着的门拍得山响,门内亳无声息。

73、荣禧堂前\par
盛妆的探春缓缓步下台阶。

74、大观园后角门\par
宝玉下马,匆匆走至门前。园门紧闭,显然是新钉上去的一副门环上,挂着一把大铜锁。宝玉叹了口气。

75、荣禧堂前\par
探春慢慢抬头朝天上看去。晴空里,各种各祥的风筝随风飘摇……探春缓缓走到赵姨娘面前,半晌,张嘴轻轻地喊了一声:“娘……”
赵姨娘捂着脸呜呜大哭。

76、荣国府后街门\par
一辆骡车冷冷清清地离开角门,骡车棚内,袭人捂着脸呜呜大哭。袭人的哥哥、嫂子无言呆坐。

77、江中\par
船帆升起,一艘结红挂彩的帆船缓缓离岸。探春伫立船头,江风拂动着她的大红披风。宝玉站在探春身旁,按剑凝望。
歌声起:
一帆风雨路三千,把骨肉家园齐来抛闪。恐哭损残年,告爹娘,休把儿悬念。
自古穷通皆有定,离合岂无缘?从今分两地,各自保平安。奴去也,莫牵连……
断了线的风筝颤颤悠悠地向天边飘去……帆影渐远……
路上,骡车渐远……水天相接处,帆影消失于空茫之中……
