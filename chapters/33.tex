\chapter{手足耽耽小动唇舌 \quad 不肖种种大承笞挞}
\zhu{耽耽:音“丹丹”,注目而视的样子。
笞挞:音“吃踏”,用棍杖篾板打罚。
}
\par
\qi{富贵公子,侯王应袭,容易在红粉场中作罪。
\zhu{“富贵公子”指贾宝玉,“侯王应袭”似乎是指忠顺王和北静王,他们都喜欢蒋玉菡,故曰“容易在红粉场中作罪”。}
风流情性,诗赋文词,偏只为莺花路间留滞。
笑嘻嘻,哭啼啼,总是一般情事。
}\par
却说王夫人唤他母亲上来,拿几件簪环当面赏与,又吩咐请几众僧人念经超度。
他母亲磕头谢了出去。
\ping{女儿被逼死,母亲还要磕头谢恩,着实可怜。
}\par
原来宝玉会过雨村回来听见了,便知金钏儿含羞赌气自尽,心中早又五内摧伤,进来被王夫人数落教训,也无可回说。
见宝钗进来,方得便出来,茫然不知何往,背着手,低头一面感叹,一面慢慢的走着,信步来至厅上。
刚转过屏门,不想对面来了一人正往里走,可巧儿撞了个满怀。
只听那人喝了一声“站住!”宝玉唬了一跳,抬头一看,不是别人,却是他父亲,不觉的倒抽了一口气,只得垂手一旁站了。
贾政道:“好端端的,你垂头丧气嗐些什么?\zhu{嗐:音“害”,表示不满,惋惜或懊悔。
}方才雨村来了要见你,叫你那半天你才出来;既出来了,全无一点慷慨挥洒谈吐,仍是葳葳蕤蕤。
\zhu{葳蕤[wēiruí]:
本指草木茂盛枝叶下垂貌,此指无精打采、萎靡不振。
}我看你脸上一团思欲愁闷气色,这会子又咳声叹气。
\zhu{咳[hāi]:
咳声叹气:因忧愁、烦闷或痛苦而发出叹息声。
}你那些还不足,还不自在?无故这样,却是为何?”宝玉素日虽是口角伶俐,只是此时一心总为金钏儿感伤,恨不得此时也身亡命殒,\meng{真有此情,真有此理。
}跟了金钏儿去。
如今见了他父亲说这些话,究竟不曾听见,只是怔呵呵的站着。
\par
贾政见他惶悚,\zhu{悚:音“送”三声,惊惧,恐惧。
}应对不似往日,原本无气的,这一来倒生了三分气。
方欲说话,忽有回事人来回:“忠顺亲王府里有人来,要见老爷。
”贾政听了,心下疑惑,暗暗思忖道:“素日并不和忠顺府来往,为什么今日打发人来?”一面想,一面令“快请”,急走出来看时,却是忠顺府长史官,\zhu{长史官:总管王府内事务的官吏。
从南朝起始设,其后各代王府都沿设此职。
}忙接进厅上坐了献茶。
未及叙谈,那长史官先就说道:“下官此来,并非擅造潭府,\zhu{造:到,去。潭府:深宅大院。
常用作对他人住宅的尊称。
潭:深邃的样子。
}皆因奉王命而来,有一件事相求。
看王爷面上,敢烦老大人作主,不但王爷知情,且连下官辈亦感谢不尽。
”\ping{直入主题,毫不寒暄,佐证了贾政所说的,贾家和忠顺府并不怎么来往。
}贾政听了这话,抓不住头脑,忙陪笑起身问道:“大人既奉王命而来,不知有何见谕,
\zhu{见谕:吩咐,指示。}
望大人宣明,学生好遵谕承办。
”那长史官便冷笑道:“也不必承办,只用大人一句话就完了。
我们府里有一个做小旦的琪官,一向好好在府里,如今竟三五日不见回去,各处去找,又摸不着他的道路,因此各处访察。
这一城内,十停人倒有八停人都说,\zhu{十停有八停:十分之八。
}他近日和衔玉的那位令郎相与甚厚。
下官辈等听了,尊府不比别家,可以擅入索取,\ping{言外之意是,别家如果被怀疑的话,就可以擅入索取。
}因此启明王爷。
王爷亦云:‘若是别的戏子呢,一百个也罢了;只是这琪官随机应答,谨慎老诚,甚合我老人家的心,竟断断少不得此人。
’\ping{如果王爷仅仅要琪官唱戏的话,琪官也不会逃跑。
可能王爷把琪官当作了娈童玩耍,才逼得他逃走了。
}故此求老大人转谕令郎,请将琪官放回,一则可慰王爷谆谆奉恳,二则下官辈也可免操劳求觅之苦\foot{长史官这段话,列藏本有独特异文:“……我们府里有一个作小旦的琪官,那原是奉旨由内园赐出,只从出来,好好在府里住了不上半年,如今三日五日不见了,各处去找,又摸不着他的道路,因此各处察访。
这一城内,十停人到有八停人都说,他竟日和衔玉的那位令郎相与甚厚。
下官辈听了,尊府不比别家,可以擅来索取,因此启明王爷。
王爷亦云:‘若是别的戏子,一百个也罢了,只是这琪官,乃奉旨所赐,不便转赠令郎。
’若令郎十分爱慕,老大人竟密题一本请旨,岂不两便。
若大人不题奏时,还得转达令郎,请将琪官放出。
一则可免王爷负恩之罪,二则下官辈也可免操劳求觅之苦。
”比别本多出的话,是拉扯上朝廷,称琪官乃“奉旨所赐”,如此上纲上线,宝玉的罪名就大了。
这段异文究竟是作者原稿,还是后人妄改,学界存在不同看法,录以备考。
}。
”说毕,忙打一躬。
\par
贾政听了这话,又惊又气,即命唤宝玉来。
宝玉也不知是何原故,忙赶来时,贾政便问:“该死的奴才!你在家不读书也罢了,怎么又做出这些无法无天的事来!那琪官现是忠顺王爷驾前承奉的人,你是何等草芥,无故引逗他出来,如今祸及于我。
”\ping{“祸及于我”让人感觉薄情。
}宝玉听了唬了一跳,忙回道:“实在不知此事。
究竟连‘琪官’两个字不知为何物,岂更又加‘引逗’二字!”说着便哭了。
贾政未及开言,只见那长史官冷笑道:“公子也不必掩饰。
或隐藏在家,或知其下落,早说了出来,我们也少受些辛苦,岂不念公子之德?”宝玉连说不知,“恐是讹传,也未见得。
”那长史官冷笑道:“现有据证,何必还赖?必定当着老大人说了出来,公子岂不吃亏?既云不知此人,那红汗巾子怎么到了公子腰里?”宝玉听了这话,不觉轰去魂魄,目瞪口呆,心下自思:“这话他如何得知!他既连这样机密事都知道了,大约别的瞒他不过,不如打发他去了,免的再说出别的事来。
”因说道:“大人既知他的底细,如何连他置买房舍这样大事倒不晓得了?听得说他如今在东郊离城二十里有个什么紫檀堡,他在那里置了几亩田地几间房舍。
想是在那里也未可知。
”那长史官听了,笑道:“这样说,一定是在那里。
我且去找一回,若有了便罢,若没有,还要来请教。
”\meng{宝玉其人,爱之有馀,岂可挞者?用此等文章逼之,能不使人肝胆愤烈,以成下文之严酷耶?}说着,便忙忙的走了。
\par
贾政此时气的目瞪口歪,一面送那长史官,一面回头命宝玉“不许动!回来有话问你!”一直送那官员去了。
才回身,忽见贾环带着几个小厮一阵乱跑。
贾政喝令小厮“快打,快打!”贾环见了他父亲,唬的骨软筋酥,忙低头站住。
贾政便问:“你跑什么?带着你的那些人都不管你,不知往那里逛去,由你野马一般!”喝令叫跟上学的人来。
贾环见他父亲盛怒,便乘机说道:“方才原不曾跑,只因从那井边一过,那井里淹死了一个丫头,我看见人头这样大,身子这样粗,泡的实在可怕,所以才赶着跑了过来。
”贾政听了惊疑,问道:“好端端的,谁去跳井?我家从无这样事情,自祖宗以来,皆是宽柔以待下人。
——大约我近年于家务疏懒,自然执事人操克夺之权,\zhu{克夺之权:生杀予夺之权。
}致使生出这暴殄轻生的祸患。
\zhu{暴殄:恣意糟踏。
殄:音“舔”,灭绝。
轻生:不爱惜生命。
}若外人知道,祖宗颜面何在!”\ping{暗写贾府以往对待下人的宽柔。
}喝令快叫贾琏、赖大、来兴\foot{原作“兴来”,除列本作“来兴儿来”、杨本作“来兴”外,诸本均同。
按“兴来”不通,故诸校本多据杨、列本校改作“来兴”,这样就衍生了一个人名出来。
虽然书中偶有这种昙花一现的人物,但此处贾政要找管理家务的人来问话,一个主子贾琏、一个奴才大总管赖大,已经够了,也无须第三人的。
目前没有其他更好的校法,暂仍之。
}。
\par
小厮们答应了一声,方欲叫去,贾环忙上前拉住贾政的袍襟,贴膝跪下道:“父亲不用生气。
此事除太太房里的人,别人一点也不知道。
我听见我母亲说……”说到这里,便回头四顾一看。
\meng{如画。
}贾政知意,将眼一看众小厮,小厮们明白,都往两边后面退去。
贾环便悄悄说道:“我母亲告诉我说,宝玉哥哥前日在太太屋里,拉着太太的丫头金钏儿强奸不遂,\meng{再逼下文,有不得不尽情苦打之势。
}打了一顿。
那金钏儿便赌气投井死了。
”\ping{手足耽耽小动唇舌,贾环促成了矛盾的集中爆发。
}话未说完,把个贾政气的面如金纸,\zhu{面如金纸:脸色像金纸一样毫无血色。
形容极为愤怒或恐惧。
金纸是做冥币用的纸。
}大喝:“快拿宝玉来!”一面说,一面便往里边书房里去,喝令:“今日再有人劝我,我把这冠带家私一应交与他与宝玉过去!\zhu{冠带:帽子和束带,是官服的代称,这里代指官爵。
家私:财产,代指家业。
}我免不得做个罪人,把这几根烦恼鬓毛剃去,寻个干净去处自了,\zhu{鬓毛:即头发,佛家称为“烦恼丝”。
干净:佛家以为人世污浊不净,唯有佛门才能通向清净世界,即所谓净土。
剃去烦恼鬓毛与寻个干净去处,都是出家当和尚的意思。
}也免得上辱先人下生逆子之罪。
”\meng{一激再激,实文实事。
}\ping{宝玉的母亲王夫人,天天吃斋念佛;宝玉的父亲贾政,一着急生气就要出家当和尚。
在这样的环境熏陶下,宝玉“悬崖撒手,弃而为僧”就显得可以理解了。
}众门客仆从见贾政这个形景,便知又是为宝玉了,一个个都是啖指咬舌,
\zhu{啖(音“旦”):吃。}
连忙退出。
那贾政喘吁吁直挺挺坐在椅子上,满面泪痕,\meng{为天下父母一哭。
}一叠声“拿宝玉!拿大棍!拿索子捆上!把各门都关上!有人传信往里头去,立刻打死!”众小厮们只得齐声答应,有几个来找宝玉。
\par
那宝玉听见贾政吩咐他“不许动”,早知多凶少吉,那里承望贾环又添了许多的话。
正在厅上干转,怎得个人来往里头去捎信,偏生没个人,连茗烟也不知在那里。
正盼望时,只见一个老姆姆出来。
宝玉如得了珍宝,便赶上来拉他,说道:“快进去告诉:老爷要打我呢!快去,快去!要紧,要紧!”宝玉一则急了,说话不明白;二则老婆子偏生又聋,竟不曾听见是什么话,把“要紧”二字只听作“跳井”二字,便笑道:“跳井让他跳去,二爷怕什么?”宝玉见是个聋子,便着急道:“你出去叫我的小厮来罢。
”那婆子道:“有什么不了的事?老早的完了。
太太又赏了衣服,又赏了银子,怎么不了事的!”\meng{写老婆子爱说无要紧的话,真如见其人,如闻其声。
}\ping{老婆子属于做久了奴才的人,对于同为奴才的金钏的生命都很漠视。
}\par
宝玉急的跺脚,正没抓寻处,只见贾政的小厮走来,逼着他出去了。
贾政一见,眼都红紫了,也不暇问他在外流荡优伶,表赠私物,在家荒疏学业,淫辱母婢等语,\meng{了结得灵活。
}只喝令:“堵起嘴来,着实打死!”小厮们不敢违拗,只得将宝玉按在凳上,举起大板打了十来下。
贾政犹嫌打轻了,一脚踢开掌板的,自己夺过来,咬着牙狠命盖了三四十下。
\ping{老主子命令打小主子,小厮太难了。
}众门客见打的不祥了,忙上前夺劝。
贾政那里肯听,说道:“你们问问他干的勾当可饶不可饶!素日皆是你们这些人把他酿坏了,到这步田地还来解劝。
明日酿到他弑君杀父,你们才不劝不成!”\par
众人听这话不好听,知道气急了,忙又退出,只得觅人进去给信。
王夫人不敢先回贾母,只得忙穿衣出来,也不顾有人没人,忙忙赶往书房中来,\meng{为天下慈母一哭。
}慌的众门客小厮等避之不及。
王夫人一进房来,贾政更如火上浇油一般,那板子越发下去的又狠又快。
按宝玉的两个小厮忙松了手走开,宝玉早已动弹不得了。
贾政还欲打时,早被王夫人抱住板子。
贾政道:“罢了,罢了!今日必定要气死我才罢!”王夫人哭道:“宝玉虽然该打,老爷也要自重。
况且炎天暑日的,老太太身上也不大好,打死宝玉事小,倘或老太太一时不自在了,岂不事大!”\meng{父母之心,昊天罔极。
\zhu{罔:音“网”,通「无」,没有。昊天罔极:苍天广阔无穷尽。比喻父母恩德如苍天广大,无以回报。}
贾政、王夫人易地则皆然。
\zhu{易地:互换所处的位置。
}}\ping{在古代,作为父亲的贾政,对于自己的儿子贾宝玉和自己的妻子王夫人都有绝对的权威,王夫人只好搬出贾政的母亲劝丈夫,其实还是借助于父权,只是由于贾政的父亲去世早,贾政的母亲贾母就成了父权的代表。
}贾政冷笑道:“倒休提这话。
我养了这不肖的孽障,已经不孝;教训他一番,又有众人护持;不如趁今日一发勒死了,以绝将来之患!”说着,便要绳索来勒死。
王夫人连忙抱住哭道:“老爷虽然应当管教儿子,\ping{贾政这里可能是怨恨王夫人对于贾宝玉疏于管教,当自己管教的时候,王夫人又出面阻挡。
}也要看夫妻分上。
我如今已将五十岁的人,只有这个孽障,必定苦苦的以他为法,\zhu{法:这里可能是依靠的意思。
}我也不敢深劝。
今日越发要他死,岂不是有意绝我。
既要勒死他,快拿绳子来先勒死我,再勒死他。
我们娘儿们不敢含怨,到底在阴司里得个依靠。
\ji{未丧母者来细玩,既丧母者来痛哭。
} \meng{使人读之,声哽咽而泪如雨下。
}”说毕,爬在宝玉身上大哭起来。
贾政听了此话,不觉长叹一声,向椅上坐了,泪如雨下。
王夫人抱着宝玉,只见他面白气弱,底下穿着一条绿纱小衣皆是血渍。
\zhu{小衣:裤子的俗称。}
禁不住解下汗巾看,由臀至胫,
\zhu{胫[jìng]:小腿。}
或青或紫,或整或破,竟无一点好处,不觉失声大哭起来,“苦命的儿吓!”\zhu{吓:音“赫”,叹词,表示不满意,认为不该如此。
}因哭出“苦命儿”来,忽又想起贾珠来,便叫着贾珠哭道:“若有你活着,便死一百个我也不管了。
”\ping{王夫人对于贾宝玉很失望,更喜欢读书上进考科举的贾珠。
这时候王夫人搬出贾珠也是要提醒贾政自己已经死了一个儿子,不能再废一个了。
在儒家伦理里,对于王夫人来说,最重要的是她得有个儿子。这才能保证自己的身份和地位。
}此时里面的人闻得王夫人出来,那李宫裁、王熙凤与迎春姊妹早已出来了。
王夫人哭着贾珠的名字,\meng{慈母如画。
}别人还可,惟有宫裁禁不住也放声哭了。
贾政听了,那泪珠更似滚瓜一般滚了下来。
\par
正没开交处,忽听丫鬟来说:“老太太来了。
”一句话未了,只听窗外颤巍巍的声气说道:\meng{老人家神影活现。
}“先打死我,再打死他,岂不干净了!”贾政见他母亲来了,又急又痛,连忙迎接出来,只见贾母扶着丫头,喘吁吁的走来。
\par
贾政上前躬身陪笑道:“大暑热天,母亲有何生气亲自走来?有话只该叫了儿子进去吩咐。
”贾母听说,便止住步喘息一回,\meng{大家规模,一丝不乱。
}厉声说道:“你原来是和我说话!我倒有话吩咐,只是可怜我一生没养个好儿子,却教我和谁说去!”贾政听这话不像,
\zhu{不像:指言行超越常轨,不成话。}
忙跪下含泪说道:“为儿的教训儿子,也为的是光宗耀祖。
母亲这话,我做儿的如何禁得起?”贾母听说,便啐了一口,说道:“我说一句话,你就禁不起,你那样下死手的板子,难道宝玉就禁得起了?\meng{偏有是理。
}你说教训儿子是光宗耀祖,当初你父亲怎么教训你来!\meng{如此碍犯文字,随景生情,毫无牵滞。
}”说着,不觉就滚下泪来。
\par
贾政又陪笑道:“母亲也不必伤感,皆是作儿的一时性起,从此以后再不打他了。
”贾母便冷笑道:“你也不必和我使性子赌气的。
你的儿子,我也不该管你打不打。
我猜着你也厌烦我们娘儿们。
不如我们赶早儿离了你,大家干净!”说着便令人去看轿马,“我和你太太宝玉立刻回南京去!”家下人只得干答应着。
贾母又叫王夫人道:“你也不必哭了。
如今宝玉年纪小,你疼他,他将来长大成人,为官作宰的,也未必想着你是他母亲了。
你如今倒不要疼他,只怕将来还少生一口气呢。
”\ping{贾母指桑骂槐,明说宝玉,暗指贾政长大成人,为官作宰后,不想着自己作为母亲疼他的恩情,反而惹自己生气。
}
贾政听说,忙叩头哭道:“母亲如此说,贾政无立足之地。
”贾母冷笑道:“你分明使我无立足之地,你反说起你来!只是我们回去了,你心里干净,看有谁来许你打。
”一面说,一面只令快打点行李车轿回去。
贾政苦苦叩求认罪。
\ping{贾政此生科举无望,全家没几个出息的,有出息的儿子死了,从他接收到的信息来说,小儿子如此不成器,确实需要管教,可是又被母亲指责,贾政也太难了。
}\par
贾母一面说话,一面又记挂宝玉,忙进来看时,只见今日这顿打不比往日,又是心疼,又是生气,也抱着哭个不了。
王夫人与凤姐等解劝了一会,方渐渐的止住。
早有丫鬟媳妇等上来,要搀宝玉,凤姐便骂道:\meng{能事者自不凡。
}“糊涂东西,也不睁开眼瞧瞧!打的这么个样儿,还要搀着走!还不快进去把那藤屉子春凳抬出来呢。
”\zhu{春凳:一种面较宽的可坐可卧的长凳。
藤屉子:凳面用藤皮编成。
}众人听说连忙进去,果然抬出春凳来,将宝玉抬放凳上,随着贾母王夫人等进去,送至贾母房中。
\par
彼时贾政见贾母气未全消,不敢自便,也跟了进去。
看看宝玉,果然打重了。
再看看王夫人,“儿”一声,“肉”一声,“你替珠儿早死了,留着珠儿,免你父亲生气,我也不白操这半世的心了。
这会子你倘或有个好歹,丢下我,叫我靠那一个!”\ping{王夫人对于贾宝玉真是失望透顶了,甚至要用宝玉的死换回贾珠的活,只是因为别无选择,自己只剩下这一个儿子,只能依靠他。
}数落一场,又哭“不争气的儿”。
贾政听了,也就灰心,\meng{天下作父兄者,教子弟时亦当留意。
}自悔不该下毒手打到如此地步。
先劝贾母,贾母含泪说道:“你不出去,还在这里做什么!难道于心不足,还要眼看着他死了才去不成!\meng{遣之有法。
}”贾政听说,方退了出来。
\par
此时薛姨妈同宝钗、香菱、袭人、史湘云也都在这里。
袭人满心委屈,只不好十分使出来,见众人围着,灌水的灌水,打扇的打扇,自己插不下手去,便越性走出来到二门前,令小厮们找了茗烟来细问:\meng{各自有各自一番作用。
}“方才好端端的,为什么打起来?你也不早来透个信儿!”茗烟急的说:“偏生我没在跟前,打到半中间我才听见了。
忙打听原故,却是为琪官金钏姐姐的事。
”袭人道:“老爷怎么得知道的?”茗烟道:“那琪官的事,多半是薛大爷素日吃醋,没法儿出气,不知在外头唆挑了谁来,在老爷跟前下的火。
\zhu{下的火:使坏进谗的意思。
}那金钏儿的事是三爷说的,我也是听见老爷的人说的。
”袭人听了这两件事都对景,\zhu{对景:对得上号;情况符合。
}心中也就信了八九分。
然后回来,只见众人都替宝玉疗治。
调停完备,
\zhu{调停:照料,安排(多见于早期白话)。}
贾母令“好生抬到他房内去”。
众人答应,七手八脚,忙把宝玉送入怡红院内自己床上卧好。
又乱了半日,众人渐渐散去,袭人方进前来经心扶侍,\zhu{经心:留意,留心。
}问他端的。
且听下回分解。
\par
\qi{总评:严酷其刑以教子,不情中十分用情;\zhu{此句指贾政打宝玉。
贾政的“严酷其刑”看似无情,但目的是让宝玉“走正路”,还是出于父子之情。
}牵连不断以思婢,有恩处一等无恩。
\zhu{此句指王夫人因金钏儿自杀而内疚不安。
王夫人厚葬金钏,似乎“有恩”,但实际上正是王夫人造成了金钏的悲剧,所以说“有恩处一等无恩”。
}严父慈母一般爱子,亲优溺婢总是乖淫。
\zhu{
优指蒋玉菡,婢指金钏。
乖:背离,不正常。
}蒙头花柳,谁解春光,
\zhu{这两句的意思是,宝玉“亲优溺婢”,其中自有痴情真情,只是难为一般人所理解,就像蒙着头看花红柳绿,难见其美艳。}
跳出樊笼,\zhu{樊笼:鸟笼,比喻受约束或不自由的境地。
}一场笑话。
\zhu{这两句的意思是说如果从人生如梦之价值虚无的立场观照,无论宝玉、贾政、王夫人的行为,都是可笑的,
故云“跳出樊笼,一场笑话”。陶渊明《归园田居》中有“久在樊笼里,复得返自然”,陶诗中樊笼指仕途,这里泛指整个人生。
}
}
\dai{065}{不肖种种大承笞挞}
\dai{066}{贾母训斥贾政}
\sun{p33-1}{手足耽耽小动唇舌,不肖种种大承笞挞}{图上侧中部:贾雨村拜访贾政,并要见贾宝玉。图右侧:接着,贾政又听贾环谗言,说金钏投井实因宝玉强奸未遂。
图左侧:贾政气极,喝令下人:“堵起嘴来,着实打死!”又嫌下人打得轻,自己夺过大板,狠命地打。
图中部:王夫人及贾母闻讯赶来。
}