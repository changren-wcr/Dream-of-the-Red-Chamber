\chapter{憨湘云醉眠芍药裀 \quad 呆香菱情解石榴裙}
\zhu{
裀:音“音”,褥子、垫子、毯子等的通称。
芍药裀:用芍药的落花当褥子。
}
\qi{众姊妹一番赠贶,\zhu{贶:音“矿”,赠、赐与;也称别人所赠的东西或恩惠。
}诸僧尼一番祷祝,\zhu{祷祝:祷告祝福。
}确是宝玉生辰。
园中行礼,不亢不卑,席上设筵,不丰不啬,确是宝玉分地。
\zhu{分:身分。
地:地位。
}\hang
探春围棋理事,气象严厉;香菱斗草善谑,姿态俊逸。
湘云喜饮酒,何等疏爽;黛玉怕吃茶,何等妩媚。
\zhu{在本回中,袭人送茶,黛玉笑道:“你知道我这病,大夫不许我多吃茶,这半钟尽够了,难为你想的到。
”}晴雯刺芳官,语极尖利;袭人给裙子,意极醇良。
字字曲到。
\zhu{曲:周遍;多方面;详尽。
}}\par
话说平儿出来吩咐林之孝家的道:“大事化为小事,小事化为没事,方是兴旺之家。
若得不了一点子小事,便扬铃打鼓的乱折腾起来,不成道理。
如今将他母女带回,照旧去当差。
将秦显家的仍旧退回。
再不必提此事。
只是每日小心巡察要紧。
”说毕,起身走了。
柳家的母女忙向上磕头,林家的带回园中,回了李纨探春,二人皆说:“知道了,能可无事,很好。
”\par
司棋等人空兴头了一阵。
那秦显家的好容易等了这个空子钻了来,\zhu{好容易:好不容易。
}
只兴头上半天。
在厨房内正乱着接收家伙、米粮、煤炭等物,又查出许多亏空来,说:“粳米短了两石,
\zhu{
粳米:粳稻碾出的米。
粳稻:水稻的一种。
分蘖力弱,秆硬不易倒伏,较耐肥,米质黏性较籼稻强,胀性小。
}
常用米又多支了一个月的,炭也欠着额数。
”一面又打点送林之孝家的礼,悄悄的备了一篓炭,五百斤木柴,一担粳米,
在外边就遣了子侄送入林家去了;又打点送帐房的礼;又预备几样菜蔬请几位同事的人,说:“我来了,全仗列位扶持。
自今以后都是一家人了。
我有照顾不到的,好歹大家照顾些。
”\ping{
管理厨房肯定是一个有油水的肥差,否则不会这么多人都盯着企图上位。
秦显家的来了就查出来许多亏空,不知道这是上一任柳家的贪污挪用假公济私导致的,还是秦显家的把自己送礼打点购买岗位的开销用公款支付了,才造成亏空。
失势的柳家的正好成为了替罪羊,被秦显家的栽赃诬陷。
}\par
正乱着,忽有人来说与他:“看过这早饭就出去罢。
柳嫂儿原无事,如今还交与他管了。
”秦显家的听了,轰去魂魄,垂头丧气,登时掩旗息鼓,卷包而出。
送人之物白丢了许多,自己倒要折变了赔补亏空。
\zhu{折变:变卖。
}连司棋都气了个倒仰,无计挽回,只得罢了。
\par
赵姨娘正因彩云私赠了许多东西,被玉钏儿吵出,生恐查诘出来,\zhu{查诘:检查盘问。
}
每日捏一把汗打听信儿。
忽见彩云来告诉说:“都是宝玉应了,从此无事。
”赵姨娘方把心放下来。
谁知贾环听如此说,便起了疑心,将彩云凡私赠之物都拿了出来,照着彩云的脸摔了去,说:“这两面三刀的东西!我不稀罕。
你不和宝玉好,他如何肯替你应。
你既有担当给了我,原该不与一个人知道。
如今你既然告诉他,如今我再要这个,也没趣儿。
”彩云见如此,急的发身赌誓,\zhu{发身:发誓时以自己身体、性命为赌咒对象。
}至于哭了,百般解说,贾环执意不信,说:“不看你素日之情,去告诉二嫂子,就说你偷来给我,我不敢要。
你细想去。
”
说毕,摔手出去了。
\ping{贾环迅速撇清自己,没有担当,扬长而去,同样地宝玉在自己和金钏调笑被王夫人看到后,也是赶紧跑了出去,同样不够有担当。
}急的赵姨娘骂:“没造化的种子,蛆心孽障。
”气的彩云哭个泪干肠断。
赵姨娘百般的安慰他:“好孩子,他辜负了你的心,我看的真。
让我收起来,过两日他自然回转过来了。
”说着,便要收东西。
彩云赌气一顿包起来,乘人不见时,\zhu{乘:趁着。
}来至园中,都撇在河内,顺水沉的沉漂的漂了。
自己气的夜间在被内暗哭。
\par
当下又值宝玉生日已到,原来宝琴也是这日,二人相同。
因王夫人不在家,也不曾像往年闹热。
只有张道士送了四样礼,换的寄名符儿;
\zhu{
寄名:一种旧时的习惯。父母为求小孩顺利成长,而将其托名在菩萨或尼姑、道士处做干儿子或干女儿,称为「寄名」。也称为「寄籍」。
}
还有几处僧尼庙的和尚姑子送了供尖儿,\zhu{供尖儿:指供品的顶端部分,以其馈人,以示祝福;
另一说,即蜜供,面粉所做小条,油炸后拌蜜,堆成塔状,用来供神佛。
}并寿星纸马疏头,\zhu{寿星:星名,即南极老人星。
古人迷信,作为长寿老人的象征。
民间常将其画成或塑成长须老人样,头部长而隆起。
俗称“寿星老儿”。
旧俗常用作对被祝寿人的称呼或对年长者的敬称。
纸马:旧俗用于祭祀时供焚化的纸糊的人、车、马等造型,也指供焚化的印有神像的纸片。
疏头:旧时称分条陈述事情的文字及僧道拜忏所焚化的祝文等叫“疏”,也称“疏头”。
}并本命星官值年太岁周年换的锁儿。
\zhu{星官:我国古代把若干颗恒星组成一组,以地上的一种事物命名,即称一个星官。
星官又为星的总称,因古人以为星座有尊卑,如人之官曹列位,故称。
本命星:指与人生日干支相值的星。
太岁:中国古代传说中的值岁神,被视作凶煞,不可触犯。
锁:指寄名锁,旧时怕幼儿夭亡,给寺院或道观一定财物,让幼儿当“寄名”弟子,并在幼儿的项下系一小金锁,名“寄名锁”。
“锁”也有“锁住时间”的意思,包含着永葆青春不会随便老去的意义。
换:第二十九回,凤姐请求张道士换女儿巧姐的寄名符,换寄名锁应该类似。
}家中常走的女先儿来上寿。
王子腾那边,仍是一套衣服,一双鞋袜,一百寿桃,一百束上用银丝挂面。
薛姨娘处减一等。
其馀家中人,尤氏仍是一双鞋袜;凤姐儿是一个宫制四面和合荷包,\zhu{
指荷包制作的工艺,用四块锦缎料子联缀而成,取字面上的吉祥意。
}
里面装一个金寿星,一件波斯国所制玩器。
\zhu{波斯国:古国名,即今伊朗。
}各庙中遣人去放堂舍钱。
\zhu{放堂:旧时施主在寺庙中普遍布施僧众以期消灾得福,叫放堂。
}又另有宝琴之礼,不能备述。
姐妹中皆随便,或有一扇的,或有一字的,或有一画的,或有一诗的,聊复应景而已。
\zhu{聊:姑且,暂且。
}\par
这日宝玉清晨起来,梳洗已毕,冠带出来。
至前厅院中,已有李贵等四五个人在那里设下天地香烛,\zhu{天地:即“天地桌”,拜祭天地时陈设香烛、供品的桌子。
}宝玉炷了香。
行毕礼,奠茶焚纸后,便至宁府中宗祠祖先堂两处行毕礼,出至月台上,
\zhu{月台:正殿或正房前面凸出的平台。}
又朝上遥拜过贾母、贾政、王夫人等。
一顺到尤氏上房,行过礼,坐了一回,方回荣府。
先至薛姨妈处,薛姨妈再三拉着,然后又遇见薛蝌,让一回,方进园来。
晴雯麝月二人跟随,小丫头夹着毡子,从李氏起,一一挨着所长的房中到过。
复出二门,至李、赵、张、王四个奶妈家让了一回,方进来。
虽众人要行礼,也不曾受。
回至房中,袭人等只都来说一声就是了。
王夫人有言,不令年轻人受礼,恐折了福寿,故皆不磕头。
\par
歇一时,贾环贾兰等来了,袭人连忙拉住,坐了一坐,便去了。
宝玉笑说走乏了,便歪在床上。
方吃了半盏茶,只听外面咭咭呱呱,一群丫头笑进来,原来是翠墨、小螺、翠缕、入画,邢岫烟的丫头篆儿,并奶子抱巧姐儿,彩鸾、绣鸾八九个人,都抱着红毡笑着走来,说:“拜寿的挤破了门了,快拿面来我们吃。
”刚进来时,探春、湘云、宝琴、岫烟、惜春也都来了。
宝玉忙迎出来,笑说:“不敢起动,快预备好茶。
”进入房中,不免推让一回,大家归坐。
袭人等捧过茶来,才吃了一口,平儿也打扮的花枝招展的来了。
宝玉忙迎出来,笑说:“我方才到凤姐姐门上,回了进去,不能见,我又打发人进去让姐姐的。
”\zhu{让:邀请。
}平儿笑道:“我正打发你姐姐梳头,不得出来回你。
后来听见又说让我,我那里禁当的起,所以特赶来磕头。
”宝玉笑道:“我也禁当不起。
”袭人早在外间安了坐,让他坐。
平儿便福下去,宝玉作揖不迭。
平儿便跪下去,宝玉也忙还跪下,袭人连忙搀起来。
又下了福,宝玉又还了一揖。
袭人笑推宝玉:“你再作揖。
”宝玉道:“已经完了,怎么又作揖?”袭人笑道:“这是他来给你拜寿。
今儿也是他的生日,你也该给他拜寿。
”宝玉听了,喜的忙作下揖去,说:“原来今儿也是姐姐的芳诞。
”平儿还万福不迭。
湘云拉宝琴岫烟说:“你们四个人对拜寿,直拜一天才是。
”探春忙问:“原来邢妹妹也是今儿?我怎么就忘了。
”忙命丫头:“去告诉二奶奶,赶着补了一分礼,与琴姑娘的一样,送到二姑娘屋里去。
”丫头答应着去了。
岫烟见湘云直口说出来,少不得要到各房去让让。
\ping{囊中羞涩不方便过生日,闹出来大家知道前来庆贺,还需要回礼招待。}
\par
探春笑道:“倒有些意思,一年十二个月,月月有几个生日。
人多了,便这等巧,也有三个一日、两个一日的。
大年初一日也不白过,大姐姐占了去。
怨不得他福大,生日比别人就占先。
又是太祖太爷的生日。
过了灯节,就是老太太和宝姐姐,他们娘儿两个遇的巧。
三月初一日是太太,初九日是琏二哥哥。
二月没人。
”袭人道:“二月十二是林姑娘,怎么没人?就只不是咱家的人。
”
\ping{
第二十二回:“就在贾母上房排了几席家宴酒席,并无一个外客,只有薛姨妈、史湘云、宝钗是客,馀者皆是自己人。”
第三十五回贾母道:“提起姊妹,不是我当着姨太太的面奉承,千真万真,从我们家四个女孩儿算起,全不如宝丫头。”
在贾母的语境中,黛玉是自己家的人;在袭人的语境中,黛玉是外人。
}
探春笑道:“我这个记性是怎么了!”
宝玉笑指袭人道:“他和林妹妹是一日,所以他记的。
”探春笑道:“原来你两个倒是一日。
每年连头也不给我们磕一个。
平儿的生日我们也不知道,这也是才知道。
”平儿笑道:“我们是那牌儿名上的人,生日也没拜寿的福,又没受礼职份,可吵闹什么,可不悄悄的过去。
今儿他又偏吵出来了,等姑娘们回房,我再行礼去罢。
”探春笑道:“也不敢惊动。
只是今儿倒要替你过个生日,我心才过得去。
”宝玉湘云等一齐都说:“很是。
”探春便吩咐了丫头:“去告诉他奶奶,就说我们大家说了,今儿一日不放平儿出去,我们也大家凑了分子过生日呢。
”丫头笑着去了,半日,回来说:“二奶奶说了,多谢姑娘们给他脸。
不知过生日给他些什么吃,只别忘了二奶奶,就不来絮聒他了。
”\zhu{
聒[guō]:(声音)嘈杂扰人。
聒絮:唠叨。
}众人都笑了。
\par
探春因说道:“可巧今儿里头厨房不预备饭,一应下面弄菜都是外头收拾。
咱们就凑了钱叫柳家的来揽了去,只在咱们里头收拾倒好。
”众人都说是极。
探春一面遣人去问李纨、宝钗、黛玉,一面遣人去传柳家的进来,吩咐他内厨房中快收拾两桌酒席。
柳家的不知何意,因说外厨房都预备了。
探春笑道:“你原来不知道,今儿是平姑娘的华诞。
\zhu{华诞:生日的美称。
}外头预备的是上头的,这如今我们私下又凑了分子,单为平姑娘预备两桌请他。
你只管拣新巧的菜蔬预备了来,开了帐和我那里领钱。
”柳家的笑道:“原来今日也是平姑娘的千秋,我竟不知道。
”说着,便向平儿磕下头去,慌的平儿拉起他来。
柳家的忙去预备酒席。
\par
这里探春又邀了宝玉,同到厅上去吃面,等到李纨宝钗一齐来全,又遣人去请薛姨妈与黛玉。
因天气和暖,黛玉之疾渐愈,故也来了。
花团锦簇,挤了一厅的人。
\par
谁知薛蝌又送了巾扇香帛四色寿礼与宝玉,宝玉于是过去陪他吃面。
两家皆治了寿酒,互相酬送,彼此同领。
至午间,宝玉又陪薛蝌吃了两杯酒。
宝钗带了宝琴过来与薛蝌行礼,把盏毕,宝钗因嘱薛蝌:“家里的酒也不用送过那边去,这虚套竟可收了。
你只请伙计们吃罢。
我们和宝兄弟进去还要待人去呢,也不能陪你了。
”薛蝌忙说:“姐姐兄弟只管请,只怕伙计们也就好来了。
”宝玉忙又告过罪,方同他姊妹回来。
\par
一进角门,宝钗便命婆子将门锁上,把钥匙要了自己拿着。
宝玉忙说:“这一道门何必关,又没多的人走。
况且姨娘、姐姐、妹妹都在里头,倘或家去取什么,岂不费事。
”宝钗笑道:“小心没过逾的。
\zhu{过逾:过分。
}你瞧你们那边,这几日七事八事,竟没有我们这边的人,可知是这门关的有功效了。
若是开着,保不住那起人图顺脚,抄近路从这里走,拦谁的是?不如锁了,连妈和我也禁着些,大家别走。
纵有了事,就赖不着这边的人了。
”宝玉笑道:“原来姐姐也知道我们那边近日丢了东西?”宝钗笑道:“你只知道玫瑰露和茯苓霜两件,乃因人而及物。
若非因人,你连这两件还不知道呢。
殊不知还有几件比这两件大的呢。
若以后叨登不出来,是大家的造化;若叨登出来,不知里头连累多少人呢。
\ping{
伏下何事?可能是夜里下人吃酒聚赌事,或者该事在书遗失的后半部分。
另一种可能是作者挂一漏万,不可能面面俱到,从一个切面透视其他没提到的事情。
}
你也是不管事的人,我才告诉你。
平儿是个明白人,我前儿也告诉了他,皆因他奶奶不在外头,所以使他明白了。
若不出来,大家乐得丢开手。
若犯出来,他心里已有稿子,自有头绪,就冤屈不着平人了。
\zhu{平人:无罪之人。
}你只听我说,以后留神小心就是了,这话也不可对第二个人讲。
”\par
说着,来到沁芳亭边,只见袭人、香菱、待书、素云、晴雯、麝月、芳官、蕊官、藕官等十来个人都在那里看鱼作耍。
见他们来了,都说:“芍药栏里预备下了,快去上席罢。
”宝钗等遂携了他们同到了芍药栏中红香圃三间小敞厅内。
连尤氏已请过来了,诸人都在那里,只没平儿。
\par
原来平儿出去,有赖、林诸家送了礼来,连三接四,上中下三等家人来拜寿送礼的不少,平儿忙着打发赏钱道谢,一面又色色的回明凤姐儿,不过留下几样,也有不收的,也有收下即刻赏与人的。
忙了一回,又直待凤姐儿吃过面,方换了衣裳往园里来。
\par
刚进了园,就有几个丫鬟来找他,一同到了红香圃中。
只见筵开玳瑁,褥设芙蓉。
\zhu{
玳瑁:龟类动物,甲壳可作酒器或装饰品,甚名贵。
筵开玳瑁,褥设芙蓉:犹言开玳瑁之筵,设芙蓉之褥。
形容筵席的珍贵和铺设的华丽。
}众人都笑:“寿星全了。
”上面四座定要让他四个人坐,四人皆不肯。
薛姨妈说:“我老天拔地,又不合你们的群儿,我倒觉拘的慌,不如我到厅上随便躺躺去倒好。
我又吃不下什么去,又不大吃酒,这里让他们倒便宜。
”尤氏等执意不从。
宝钗道:“这也罢了,倒是让妈在厅上歪着自如些,有爱吃的送些过去,倒自在了。
且前头没人在那里,又可照看了。
”探春等笑道:“既这样,恭敬不如从命。
”因大家送了他到议事厅上,眼看着命丫头们铺了一个锦褥并靠背引枕之类,又嘱咐:“好生给姨妈捶腿,要茶要水别推三扯四的。
回来送了东西来,姨妈吃了就赏你们吃。
只别离了这里出去。
”小丫头们都答应了。
\par
探春等方回来。
终久让宝琴、岫烟二人在上,平儿面西坐,宝玉面东坐。
探春又接了鸳鸯来,二人并肩对面相陪。
西边一桌,宝钗、黛玉、湘云、迎春、惜春,一面又拉了香菱、玉钏儿二人打横。
\zhu{打横:坐在桌子侧面的位置。
}三桌上,尤氏、李纨,又拉了袭人、彩云陪坐。
四桌上便是紫鹃、莺儿、晴雯、小螺、司棋等人围坐。
当下探春等还要把盏,宝琴等四人都说:“这一闹,一日都坐不成了。
”方才罢了。
两个女先儿要弹词上寿,众人都说:“我们没人要听那些野话,你厅上去说给姨太太解闷儿去罢。
”一面又将各色吃食拣了,命人送与薛姨妈去。
\par
宝玉便说:“雅坐无趣,须要行令才好。
”众人有的说行这个令好,那个又说行那个令好。
黛玉道:“依我说,拿了笔砚将各色全都写了,拈成阄儿,咱们抓出那个来,就是那个。
”众人都道妙。
即拿了一副笔砚花笺。
香菱近日学了诗,又天天学写字,见了笔砚便图不得,
\zhu{图不得:忍耐不住;支持不住。}
连忙起座说:“我写。
”大家想了一回,共得了十来个,念着,香菱一一的写了,搓成阄儿,掷在一个瓶中间。
探春便命平儿拣,平儿向内搅了一搅,用箸拈了一个出来,打开看,上写着“射覆”二字。
\zhu{射:猜。
覆:遮盖;隐藏。
射覆:原为古时的一种猜谜游戏,用碗盆等把某物遮盖起来,猜中者胜。
后来也作为酒令的一种,如这里覆者先用诗文、成语、典故等隐寓某一事物,射者猜度,用也隐寓该事物的另一诗文、成语、典故等揭谜底,若射者猜不出或猜错以及覆者误判射者的猜度时,都要罚酒。
}宝钗笑道:“把个酒令的祖宗拈出来。
‘射覆’从古有的,如今失了传,这是后人纂的,比一切的令都难。
这里头倒有一半是不会的,不如毁了,另拈一个雅俗共赏的。
”探春笑道:“既拈了出来,如何又毁。
如今再拈一个,若是雅俗共赏的,便叫他们行去。
咱们行这个。
”说着又着袭人拈了一个,却是“拇战”。
\zhu{拇战:行酒令的一种,也叫豁拳,划拳。
}史湘云笑着说:“这个简断爽利,合了我的脾气。
我不行这个‘射覆’,没的垂头丧气闷人,我只划拳去了。
”探春道:“惟有他乱令,宝姐姐快罚他一钟。
”宝钗不容分说,便灌湘云一杯。
\par
探春道:“我吃一杯,我是令官,也不用宣,只听我分派。
”命取了令骰令盆来,“从琴妹掷起,挨下掷去,对了点的二人射覆。
”宝琴一掷,是个三,岫烟宝玉等皆掷的不对,直到香菱方掷了个三。
宝琴笑道:“只好室内生春,\zhu{室内生春:这里指所射覆的谜底只限于本室的事物。
生春:喻想得新巧,妙趣横生,又含有吉利的意思。
}若说到外头去,可太没头绪了。
”探春道:“自然。
三次不中者罚一杯。
你覆,他射。
”宝琴想了一想,说了个“老”字。
香菱原生于这令,一时想不到,满室满席都不见有与“老”字相连的成语。
湘云先听了,便也乱看,忽见门斗上贴着“红香圃”三个字,便知宝琴覆的是“吾不如老圃”的“圃”字。
\zhu{吾不如老圃:见《论语·子路》。
圃:音“普”,种植莱蔬花果的园地。
老圃:老菜农。
}见香菱射不着,众人击鼓又催,便悄悄的拉香菱,教他说“药”字。
\zhu{“药”字:可能是指包括“红香圃”三间小敞厅在内的“芍药栏”。
}黛玉偏看见了,说“快罚他,又在那里私相传递呢。
”哄的众人都知道了,忙又罚了一杯,恨的湘云拿筷子敲黛玉的手。
于是罚了香菱一杯。
下则宝钗和探春对了点子。
探春便覆了一个“人”字。
宝钗笑道:“这个‘人’字泛的很。
”探春笑道:“添一字,两覆一射也不泛了。
”说着,便又说了一个“窗”字。
宝钗一想,因见席上有鸡,便射着他是用“鸡窗”“鸡人”二典了,\zhu{鸡窗:指书室。
传说晋代兖州刺史宋处宗得一长鸣鸡,经常放在窗边,鸡忽然会说人话,同处宗终日交谈,处宗因而学问大进,后人遂用鸡窗代称书室。
见南朝宋刘义庆《幽明录》。
鸡人:古代宫中头戴“绛帻”(帻:音“则”,头巾。
绛帻:红布头巾,象征雄鸡鸡冠)专职司晨报晓的卫士。
}因射了一个“埘”字。
探春知他射着,用了“鸡栖于埘”的典,\zhu{鸡栖于埘:见《诗·王风·君子于役》。
埘:音“时”,凿在墙壁上的鸡窝。
}二人一笑,各饮一口门杯。
\par
湘云等不得,早和宝玉“三”“五”乱叫,划起拳来。
那边尤氏和鸳鸯隔着席也“七”“八”乱叫划起来。
平儿袭人也作了一对划拳,叮叮当当只听得腕上的镯子响。
一时湘云赢了宝玉,鸳鸯赢了尤氏,袭人赢了平儿,三个人限酒底酒面,\zhu{酒底:每行完一个酒令时,饮干一杯酒,叫“酒底”。
酒面:斟满一杯酒,不饮,先行酒令,叫酒面。
“酒面”的本义是满杯的样子。
}湘云便说:“酒面要一句古文,一句旧诗,一句骨牌名,一句曲牌名,还要一句时宪书上的话,\zhu{时宪书:即历书,记载年、月、日、时、节气等可供查考的书。
也称为“历本”。
}共总凑成一句话。
酒底要关人事的果菜名。
”众人听了,都笑说:“惟有他的令也比人唠叨,倒也有意思。
”便催宝玉快说。
宝玉笑道:“谁说过这个,也等想一想儿。
”黛玉便道:“你多喝一钟,我替你说。
”宝玉真个喝了酒,听黛玉说道:\par
\hop
落霞与孤鹜齐飞,风急江天过雁哀,却是一只折足雁,叫的人九回肠,这是鸿雁来宾。
\par
\zhu{“落霞”句:见唐代王勃《滕王阁序》。
鹜:音“务”,野鸭。
“风急”句:或系对宋代陆游《寒夕》诗中的“风急江天无过雁”句的误记。
“折足雁”:骨牌副儿名。
这副牌由“长三”、“一二”、“长三”组成。
九回肠:曲牌名。
“鸿雁来宾”:《礼记·月令》:“季秋之月……鸿雁来宾”。
旧时历书引此语作为秋季的标志。
}\par
\hop
说的大家笑了,说:“这一串子倒有些意思。
”黛玉又拈了一个榛穰,\zhu{
榛(音“真”)子:榛树的果实,瓤可食用或榨油。
穰:瓜、果内部可食的部分。
通“瓤”。
}说酒底道:\par
\hop
榛子非关隔院砧,何来万户捣衣声。
\zhu{
砧:音“真”,捣衣石。
隔院砧:邻家的捣衣石。
“万户捣衣声”:见唐代李白《子夜吴歌·秋歌》“长安一片月,万户捣衣声”。
这两句话利用“榛”与“砧”同音异义的特点,既巧妙地符合了酒令的规定,也说明了榛子与捣衣声无关的事实。
}\par
\ping{
黛玉哀怨,黛玉的身世遭遇恰如其酒令中的折足孤雁,失伴哀鸣。
}
\par
\hop
令完,鸳鸯袭人等皆说的是一句俗语,都带一个“寿”字的,不能多赘。
\par
大家轮流乱划了一阵,这上面湘云又和宝琴对了手,李纨和岫烟对了点子。
李纨便覆了一个“瓢”字,岫烟便射了一个“绿”字,二人会意,各饮一口。
\zhu{李纨同岫烟射覆一段:李纨覆“瓢”字,是看到席上有樽(酒杯),故用既有“瓢”字又有“樽”字的诗句“瓢樽空挂壁”(见宋代苏辙《九日三首》之一)来隐寓“樽”字。
岫烟射“绿”字,是用既有“绿”字又有“樽”字的诗句“愁向绿樽生”(见唐代刘希夷《送友人之新丰》),来猜李纨所覆的“樽”字。
}湘云的拳却输了,请酒面酒底。
宝琴笑道:“请君入瓮。
”\zhu{请君入瓮:唐天授二年,武则天命来俊臣审周兴。
来先问周:怎样才能使犯人招供?周答:把犯人放进四面围火的大瓮中,他就不会不招。
来即如法置瓮并对周讲:奉命审你,请你入瓮。
周当即伏罪。
见《资治通鉴·唐纪》。
}
大家笑起来,说:“这个典用的当。
”湘云便说道:\par
\hop
奔腾而砰湃,\zhu{砰湃:象声词。
形容水流汹涌、暴雨等声。
}江间波浪兼天涌,须要铁锁缆孤舟,既遇着一江风,不宜出行。
\par
\zhu{“奔腾”句:见宋代欧阳修《秋声赋》“初淅沥以萧飒,忽奔腾而砰湃”。
“江间”句:见唐代杜甫《秋兴八首》“江间波浪兼天涌,塞上风云接地阴”。
兼天:连天。
“铁锁”句:骨牌副儿名。
这副牌由“长三”、“三六”、“长三”组成。
一江风:曲牌名。
“不宜出行”:旧时历书上每个日子的下面都载“宜”什么或“忌”什么的迷信话。
造历者按干支五行推算,如认为某日外出不吉利就写上“不宜出行”,如认为某日会亲友吉利就写上“宜会亲友”,等等。
}\par
\hop
说的众人都笑了,说:“好个诌断了肠子的。
怪道他出这个令,故意惹人笑。
”又听他说酒底。
湘云吃了酒,拣了一块鸭肉呷口,忽见碗内有半个鸭头,遂拣了出来吃脑子。
众人催他:“别只顾吃,到底快说了。
”湘云便用箸子举着说道:\par
\hop
这鸭头不是那丫头,头上那讨桂花油。
\par
\ping{
湘云放达,湘云幼小时父母早丧,家业凋零,后来又夫妻离散,青春孤居,其生活历程也正像江上孤舟,数经风涛。
}
\par
\hop
众人越发笑起来,引的晴雯、小螺、莺儿等一干人都走过来说:“云姑娘会开心儿,拿着我们取笑儿,快罚一杯才罢。
怎见得我们就该擦桂花油的?倒得每人给一瓶子桂花油擦擦。
”黛玉笑道:“他倒有心给你们一瓶子油,又怕挂误着打盗窃的官司。
”\zhu{挂误:贻误,连累。
}众人不理论,宝玉却明白,忙低了头。
彩云有心病,不觉的红了脸。
宝钗忙暗暗的瞅了黛玉一眼。
黛玉自悔失言,原是趣宝玉的,\zhu{趣:取笑,打趣。
}就忘了趣着彩云。
自悔不及,忙一顿行令划拳岔开了。
\par
底下宝玉可巧和宝钗对了点子。
宝钗覆了一个“宝”字,宝玉想了一想,便知是宝钗作戏指自己所佩通灵玉而言,便笑道:“姐姐拿我作雅谑,我却射着了。
说出来姐姐别恼,就是姐姐的讳‘钗’字就是了。
”众人道:“怎么解?”宝玉道:“他说‘宝’,底下自然是‘玉’了。
我射‘钗’字,旧诗曾有‘敲断玉钗红烛冷’,\zhu{敲断玉钗红烛冷:见南宋郑会《题邸间壁》诗。
玉钗:代指灯花。
}岂不射着了。
”湘云说道:“这用时事却使不得,两个人都该罚。
”香菱忙道:“不止时事,这也有出处。
”湘云道:“‘宝玉’二字并无出处,不过是春联上或有之,诗书纪载并无,算不得。
”香菱道:“前日我读岑嘉州五言律,\zhu{岑嘉州:即唐代岑参,因曾任嘉州刺史,故称。
}现有一句说‘此乡多宝玉’,怎么你倒忘了?后来又读李义山七言绝句,又有一句‘宝钗无日不生尘’,\zhu{宝钗无日不生尘:见唐代李义山(商隐)《残花》诗。
“无”原诗作“何”。
宝钗生尘:形容女子懒于梳妆。
}我还笑说他两个名字都原来在唐诗上呢。
”
\ping{
用射覆所引诗文来作人物将来遭遇的暗示。“此乡多宝玉,慎勿厌清贫。”
宝玉后来清贫有“贫穷难耐凄凉”《西江月·嘲贾宝玉》、“寒冬噎酸虀,雪夜围破毡”(第十九回脂评)等语可证。
“若但掩关劳独梦,宝钗何日不生尘?”宝钗后来寡居独处,喻之为金钗“雪里埋”、“金无彩”、“生尘”都无不可,所以闭门自重,“掩关”二字与其《咏白海棠》诗句“珍重芳姿昼掩门”的意思一样。
“劳独梦”三字与其《忆菊》“空篱旧圃秋无迹,瘦月清霜梦有知”、《春灯谜》“晓筹不用鸡人报,五夜无烦侍女添”等诗参看,就知更非偶然了。
宝玉所引唐诗“敲断玉钗红烛冷,计程应说到常山。”。“玉钗”在这里该是隐寓黛玉和宝钗的。
宝玉后来离家出走,“断”绝了与她们之间的往来音讯。心事虚化,美梦成空,此所谓“红烛冷”也。
大观园姊妹们记挂着他的出走,时时叨念着他,正切合“计程应说到常山”情景。
此外,宝钗射“埘”字以隐《诗经·君子于役》一诗,不正是妻子思念丈夫久出不归吗?
}
众人笑说:“这可问住了,快罚一杯。
”湘云无语,只得饮了。
大家又该对点的对点,划拳的划拳。
这些人因贾母王夫人不在家,没了管束,便任意取乐,呼三喝四,喊七叫八。
满厅中红飞翠舞,玉动珠摇,真是十分热闹。
顽了一回,大家方起席散了一散,倏然不见了湘云,只当他外头自便就来,谁知越等越没了影响,\zhu{影响:影子和声响,引申为踪迹;音信,消息。
}使人各处去找,那里找得着。
\par
接着林之孝家的同着几个老婆子来,生恐有正事呼唤,二者恐丫鬟们年青,乘王夫人不在家不服探春等约束,\zhu{乘:趁着。
}恣意痛饮,失了体统,故来请问有事无事。
探春见他们来了,便知其意,忙笑道:“你们又不放心,来查我们来了。
我们没有多吃酒,不过是大家顽笑,将酒作个引子,妈妈们别耽心。
”李纨尤氏都也笑说:“你们歇着去罢,我们也不敢叫他们多吃了。
”林之孝家的等人笑说:“我们知道,连老太太叫姑娘吃酒姑娘们还不肯吃,何况太太们不在家,自然顽罢了。
我们怕有事,来打听打听。
二则天长了,姑娘们顽一回子还该点补些小食儿。
\zhu{点补:谓进食少量食品。
}素日又不大吃杂东西,如今吃一两杯酒,若不多吃些东西,怕受伤。
”探春笑道:“妈妈们说的是,我们也正要吃呢。
”因回头命取点心来。
两旁丫鬟们答应了,忙去传点心。
探春又笑让:“你们歇着去罢,或是姨妈那里说话儿去。
我们即刻打发人送酒你们吃去。
”林之孝家的等人笑回:“不敢领了。
”又站了一回,方退了出来。
平儿摸着脸笑道:“我的脸都热了,也不好意思见他们。
依我说竟收了罢,别惹他们再来,倒没意思了。
”探春笑道:“不相干,横竖咱们不认真喝酒就罢了。
”\par
正说着,只见一个小丫头笑嘻嘻的走来:“姑娘们快瞧云姑娘去,吃醉了图凉快,在山子后头一块青板石凳上睡着了。
”众人听说,都笑道:“快别吵嚷。
”说着,都走来看时,果见湘云卧于山石僻处一个石凳子上,业经香梦沉酣。
\zhu{业经:已经。
}四面芍药花飞了一身,满头、脸、衣襟上皆是红香散乱,手中的扇子在地下,也半被落花埋了,一群蜂蝶闹穰穰的围着他,\zhu{闹穰穰:即“闹嚷嚷”。
}又用鲛帕包了一包芍药花瓣枕着。
\zhu{鲛:音“交”,这里指鲛绡(绡音“消”)纱。
传说南海中有鲛人,即人鱼,能织绡,后用以泛称薄纱。
“醉眠芍药裀”是曹雪芹为史湘云憨态狂放不羁写真的精彩之笔。情节构思当从唐代卢纶《春词》“醉眠芳树下,半被落花埋”化出。
}众人看了,又是爱,又是笑,忙上来推唤挽扶。
湘云口内犹作睡语说酒令,唧唧嘟嘟说:\par
\hop
泉香而酒冽,玉盏盛来琥珀光,直饮到梅梢月上,醉扶归,却为宜会亲友。
\par
\zhu{“泉香”句:见宋代欧阳修《醉翁亭记》。
冽:音“列”,清凉。
“玉盏”句:见唐代李白《客中行》“兰陵美酒郁金香,玉碗盛来琥珀光”。
“梅梢月上”:骨牌副儿名。
这副牌由“长五”、“幺五”、“长五”组成。
醉扶归:曲牌名。
}\par
\hop
\chai{xiangyun}{湘云眠芍}
众人笑推他,说道:“快醒醒儿吃饭去,这潮凳上还睡出病来呢。
”湘云慢启秋波,见了众人,低头看了一看自己,方知是醉了。
原是来纳凉避静的,不觉的因多罚了两杯酒,娇嫋不胜,\zhu{嫋:同“袅”。
}便睡着了,心中反觉自愧。
连忙起身扎挣着同人来至红香圃中,\zhu{扎挣:勉强支持。
}用过水,又吃了两盏酽茶。
\zhu{酽:音“雁”,酒、茶等味厚。
}探春忙命将醒酒石拿来给他衔在口内,\zhu{醒酒石:相传是一种能够解酒的石头。
}一时又命他喝了一些酸汤,方才觉得好了些。
\par
当下又选了几样果菜与凤姐送去,凤姐儿也送了几样来。
宝钗等吃过点心,大家也有坐的,也有立的,也有在外观花的,也有扶栏观鱼的,各自取便说笑不一。
探春便和宝琴下棋,宝钗岫烟观局。
林黛玉和宝玉在一簇花下唧唧哝哝不知说些什么。
只见林之孝家的和一群女人带了一个媳妇进来。
那媳妇愁眉苦脸,也不敢进厅,只到了阶下,便朝上跪下了,碰头有声。
探春因一块棋受了敌,算来算去总得了两个眼,便折了官着,\zhu{
围棋规则:围棋盘纵横 19 条线形成了 361 个交叉点,黑先白后依次落子在点上,最后看哪一方占的地盘多(指交叉点数)哪一方为胜。
上下左右相邻的同色棋子为一个整体,称为“连”。
棋子(连)上下左右空地为气。 无“气”的棋子被“吃”(从棋盘拿出),无“气”处不能落子。
眼:走棋时一方棋域中所留的空隙,叫“眼”。
在这空隙中,对方不能下子,否则会被吃掉。
如果只有一个眼,如果对手在包围这片棋子的前提下填满内部的这个眼,则能吃掉这片棋子;但是如果有两个眼,则对手即使填满其中的一个眼,由于还有另一个眼,这片棋子不会被吃掉,还能反过来吃掉对手填在这个眼里的棋子。如此以来,有两个眼,相连的一片子才能活。
折了官着:即收官时下错子而失败。
“收官”也叫“收束”、“收官子”、“收官着”,指围棋下到最后阶段,双方所占地域大体已定,尚馀周围及边角空白,可以轮次填子,填满为止。
这时所下的子叫做官子,也称官着。
}两眼只瞅着棋枰,一只手却伸在盒内,只管抓弄棋子作想,林之孝家的站了半天,因回头要茶时才看见,问:“什么事?”林之孝家的便指那媳妇说:“这是四姑娘屋里的小丫头彩儿的娘,现是园内伺候的人。
嘴很不好,才是我听见了问着他,他说的话也不敢回姑娘,竟要撵出去才是。
”探春道:“怎么不回大奶奶?”林之孝家的道:“方才大奶奶都往厅上姨太太处去了,顶头看见,我已回明白了,叫回姑娘来。
”探春道:“怎么不回二奶奶?”平儿道:“不回去也罢,我回去说一声就是了。
”探春点点头,道:“既这么着,就撵出他去,等太太来了,再回定夺。
”说毕仍又下棋。
这林之孝家的带了那人去不提。
\par
黛玉和宝玉二人站在花下,遥遥知意。
黛玉便说道:“你家三丫头倒是个乖人。
虽然叫他管些事,倒也一步儿不肯多走。
差不多的人就早作起威福来了。
”宝玉道:“你不知道呢。
你病着时,他干了好几件事。
这园子也分了人管,如今多掐一草也不能了。
又蠲了几件事,单拿我和凤姐姐作筏子禁别人。
最是心里有算计的人,岂只乖而已。
”黛玉道:“要这样才好,咱们家里也太花费了。
我虽不管事,心里每常闲了,替你们一算计,出的多进的少,如今若不省俭,必致后手不接。
”宝玉笑道:“凭他怎么后手不接,也短不了咱们两个人的。
”黛玉听了,转身就往厅上寻宝钗说笑去了。
\ping{黛玉也看不上宝玉只求自己眼前享乐,不管家族长远未来。
}\par
宝玉正欲走时,只见袭人走来,手内捧着一个小连环洋漆茶盘,
\zhu{连环:两个套连的圆环。}
里面可式放着两钟新茶,\zhu{可式:正合适。
}因问:“他往那去了?我见你两个半日没吃茶,巴巴的倒了两钟来,他又走了。
”宝玉道:“那不是他,你给他送去。
”说着自拿了一钟。
袭人便送了那钟去,偏和宝钗在一处,只得一钟茶,便说:“那位渴了那位先接了,我再倒去。
”宝钗笑道:“我却不渴,只要一口漱一漱就够了。
”说着先拿起来喝了一口,剩下半杯递在黛玉手内。
袭人笑说:“我再倒去。
”黛玉笑道:“你知道我这病,大夫不许我多吃茶,这半钟尽够了,难为你想的到。
”说毕,饮干,将杯放下。
袭人又来接宝玉的。
宝玉因问:“这半日没见芳官,他在那里呢?”袭人四顾一瞧说:“才在这里几个人斗草的,这会子不见了。
”\par
宝玉听说,便忙回至房中,果见芳官面向里睡在床上。
宝玉推他说道:“快别睡觉,咱们外头顽去,一回儿好吃饭的。
”芳官道:“你们吃酒不理我,教我闷了半日,可不来睡觉罢了。
”宝玉拉了他起来,笑道:“咱们晚上家里再吃,回来我叫袭人姐姐带了你桌上吃饭,何如?”芳官道:“藕官蕊官都不上去,单我在那里也不好。
我也不惯吃那个面条子,早起也没好生吃。
才刚饿了,我已告诉了柳嫂子,先给我做一碗汤盛半碗粳米饭送来,我这里吃了就完事。
若是晚上吃酒,不许教人管着我,我要尽力吃够了才罢。
我先在家里,吃二三斤好惠泉酒呢。
\zhu{
惠泉酒:惠泉水所酿的酒。
惠泉在江苏无锡惠山第一峰下,号称“天下第二泉”。
}
如今学了这劳什子,他们说怕坏嗓子,这几年也没闻见。
乘今儿我是要开斋了。
”\zhu{乘:趁着。
开斋:因斋戒而吃素的人解除斋戒,恢复吃荤。
}宝玉道:“这个容易。
”\par
说着,只见柳家的果遣了人送了一个盒子来。
小燕接着揭开,里面是一碗虾丸鸡皮汤,又是一碗酒酿清蒸鸭子,一碟腌的胭脂鹅脯,还有一碟四个奶油松瓤卷酥,\zhu{松瓤:松仁,松子里面的仁,芳香可食。
}并一大碗热腾腾碧荧荧蒸的绿畦香稻粳米饭。
小燕放在案上,走去拿了小菜并碗箸过来,拨了一碗饭。
芳官便说:“油腻腻的,谁吃这些东西。
”\ping{第四十一回,丫鬟端来螃蟹馅的小饺儿,贾母得知后皱眉说:“这油腻腻的,谁吃这个!”第五十四回,凤姐要送来的鸭子肉粥和枣儿熬的粳米粥,贾母笑道:“不是油腻腻的就是甜的。
”芳官也像贾母一样吃腻了精致美食开始挑食,俨然一副主子气象。
}只将汤泡饭吃了一碗,拣了两块腌鹅就不吃了。
宝玉闻着,倒觉比往常之味有胜些似的,遂吃了一个卷酥,又命小燕也拨了半碗饭,泡汤一吃,十分香甜可口。
小燕和芳官都笑了。
吃毕,小燕便将剩的要交回。
宝玉道:“你吃了罢,若不够再要些来。
”小燕道:“不用要,这就够了。
方才麝月姐姐拿了两盘子点心给我们吃了,我再吃了这个,尽不用再吃了。
”说着,便站在桌旁一顿吃了,又留下两个卷酥,说:“这个留着给我妈吃。
晚上要吃酒,给我两碗酒吃就是了。
”宝玉笑道:“你也爱吃酒?等着咱们晚上痛喝一阵。
你袭人姐姐和晴雯姐姐量也好,也要喝,只是每日不好意思。
今儿大家开斋。
还有一件事,想着嘱咐你,我竟忘了,此刻才想起来。
以后芳官全要你照看他,他或有不到的去处,你提他,袭人照顾不过这些人来。
”小燕道:“我都知道,都不用操心。
但只这五儿怎么样?”宝玉道:“你和柳家的说去,明儿直叫他进来罢,等我告诉他们一声就完了。
”芳官听了,笑道:“这倒是正经。
”小燕又叫两个小丫头进来,伏侍洗手倒茶,自己收了家伙,交与婆子,也洗了手,便去找柳家的,不在话下。
\par
宝玉便出来,仍往红香圃寻众姐妹,芳官在后拿着巾扇。
刚出了院门,只见袭人晴雯二人携手回来。
宝玉问:“你们做什么?”袭人道:“摆下饭了,等你吃饭呢。
”宝玉便笑着将方才吃的饭一节告诉了他两个。
袭人笑道:“我说你是猫儿食,闻见了香就好,隔锅饭儿香。
虽然如此,也该上去陪他们多少应个景儿。
”晴雯用手指戳在芳官额上,说道:“你就是个狐媚子,什么空儿跑了去吃饭,两个人怎么就约下了,也不告诉我们一声儿。
”袭人笑道:“不过是误打误撞的遇见了,说约下了可是没有的事。
”晴雯道:“既这么着,要我们无用。
明儿我们都走了,让芳官一个人就够使了。
”袭人笑道:“我们都去了使得,你却去不得。
”晴雯道:“惟有我是第一个要去,又懒又笨,性子又不好,又没用。
”\ping{谶语,伏下晴雯被撵出。
}袭人笑道:“倘或那孔雀褂子再烧个窟窿,你去了谁可会补呢。
你倒别和我拿三撇四的,\zhu{拿三撇四:装腔作势,故意刁难,以提高自己的身价。
}我烦你做个什么,把你懒的横针不拈,竖线不动。
一般也不是我的私活烦你,横竖都是他的,你就都不肯做。
怎么我去了几天,你病的七死八活,一夜连命也不顾给他做了出来,这又是什么原故?你到底说话,别只佯憨,和我笑,也当不了什么。
”大家说着,来至厅上。
薛姨妈也来了。
大家依序坐下吃饭。
宝玉只用茶泡了半碗饭,应景而已。
一时吃毕,大家吃茶闲话,又随便顽笑。
\par
外面小螺和香菱、芳官、蕊官、藕官、荳官等四五个人,都满园中顽了一回,大家采了些花草来兜着,坐在花草堆中斗草。
这一个说:“我有观音柳。
”那一个说:“我有罗汉松。
”那一个又说:“我有君子竹。
”这一个又说:“我有美人蕉。
”这个又说:“我有星星翠。
”那个又说:“我有月月红。
”这个又说:“我有《牡丹亭》上的牡丹花。
”那个又说:“我有《琵琶记》里的枇杷果。
”
\zhu{
枇杷[pípá]:常绿小乔木,叶子椭圆形,开白色小花,有芳香。果实也叫枇杷,球形,橙黄色,是常见水果。
}
荳官便说:“我有姐妹花。
”众人没了,香菱便说:“我有夫妻蕙。
”荳官说:“从没听见有个夫妻蕙。
”香菱道:“一箭一花为兰,一箭数花为蕙。
凡蕙有两枝,上下结花者为兄弟蕙,有并头结花者为夫妻蕙。
我这枝并头的,怎么不是。
”荳官没的说了,便起身笑道:“依你说,若是这两枝一大一小,就是老子儿子蕙了。
若两枝背面开的,就是仇人蕙了。
\ping{从“夫妻蕙”到“仇人蕙”,作者是在暗示“夫妻”将成为“仇人”。}
你汉子去了大半年,你想夫妻了?便扯上蕙也有夫妻,好不害羞!”香菱听了,红了脸,忙要起身拧他,笑骂道:“我把你这个烂了嘴的小蹄子!\zhu{小蹄子:骂年轻女孩或婢女的话。
}满嘴里汗\bie 的胡说了。
\zhu{汗\bie :\bie 音“憋”,生热病者,汗多难出,心中烦躁,神志不清,往往胡言乱语,称为“汗\bie ”。
这里借以骂人家“胡说”。
}等我起来打不死你这小蹄子!”荳官见他要勾来,怎容他起来,便忙连身将他压倒。
回头笑着央告蕊官等:“你们来,帮着我拧他这诌嘴。
”\zhu{诌:音“周”,信口胡说,编瞎话。
}两个人滚在草地下。
众人拍手笑说:“了不得了,那是一洼子水,可惜污了他的新裙子了。
”荳官回头看了一看,果见旁边有一汪积雨,香菱的半扇裙子都污湿了,自己不好意思,忙夺了手跑了。
众人笑个不住,怕香菱拿他们出气,也都哄笑一散。
\par
香菱起身低头一瞧,那裙上犹滴滴点点流下绿水来。
正恨骂不绝,可巧宝玉见他们斗草,也寻了些花草来凑戏,忽见众人跑了,只剩了香菱一个低头弄裙,因问:“怎么散了?”香菱便说:“我有一枝夫妻蕙,他们不知道,反说我诌,因此闹起来,把我的新裙子也脏了。
”宝玉笑道:“你有夫妻蕙,我这里倒有一枝并蒂菱。
”口内说,手内却真个拈着一枝并蒂菱花,又拈了那枝夫妻蕙在手内。
香菱道:“什么夫妻不夫妻,并蒂不并蒂,你瞧瞧这裙子。
”\ping{宝玉手中的并蒂菱花和香菱手里的夫妻蕙,可能暗示他俩人之间未来的姻缘?
}宝玉方低头一瞧,便嗳呀了一声,说:“怎么就拖在泥里了?可惜这石榴红绫最不经染。
”香菱道:“这是前儿琴姑娘带了来的。
姑娘做了一条,我做了一条,今儿才上身。
”宝玉跌脚叹道:“若你们家,一日遭踏这一百件也不值什么。
只是头一件既系琴姑娘带来的,你和宝姐姐每人才一件,他的尚好,你的先脏了,岂不辜负他的心。
二则姨妈老人家嘴碎,饶这么样,我还听见常说你们不知过日子,只会遭踏东西,不知惜福呢。
这叫姨妈看见了,又说一个不清。
”香菱听了这话,却碰在心坎儿上,反倒喜欢起来了,因笑道:“就是这话了。
我虽有几条新裙子,都不和这一样,若有一样的,赶着换了,也就好了。
过后再说。
”宝玉道:“你快休动,只站着方好,不然连小衣儿膝裤鞋面都要拖脏。
\zhu{膝裤:裹扎于双脚膝下部位的套裤,其制类似无底之袜。}
我有个主意:袭人上月做了一条和这个一模一样的,他因有孝,如今也不穿。
竟送了你换下这个来,如何?”香菱笑着摇头说:“不好。
他们倘或听见了倒不好。
”宝玉道:“这怕什么。
等他们孝满了,他爱什么难道不许你送他别的不成。
你若这样,还是你素日为人了!况且不是瞒人的事,只管告诉宝姐姐也可,只不过怕姨妈老人家生气罢了。
”香菱想了一想有理,便点头笑道:“就是这样罢了,别辜负了你的心。
我等着你,千万叫他亲自送来才好。
”
\ping{
香菱避嫌的一种表现,避的就是与宝玉的私相授受,所以必定要拉袭人作见证来冲淡这份暧昧。
}
\par
宝玉听了,喜欢非常,答应了忙忙的回来,一壁里低头心下暗算:“可惜这么一个人,没父母,连自己本姓都忘了,被人拐出来,偏又卖与了这个霸王。
”因又想起上日平儿也是意外想不到的,
\ping{
平儿之事指第四十四回中平儿受委屈后,宝玉让平儿来到怡红院,劝其理妆,温存慰藉。
第四十四回“平儿理妆”:“宝玉又将盆内的一枝并蒂秋蕙用竹剪刀撷了下来,与他簪在鬓上”;
本回“情解石榴裙”:“香菱见宝玉蹲在地下,将方才的夫妻蕙与并蒂菱用树枝儿抠了一个坑,先抓些落花来铺垫了,将这菱蕙安放好,又将些落花来掩了,方撮土掩埋平服”。
这两个情节应是作者有意设置的相似之处。
这些花草用“并蒂”、“夫妻”来命名,体现出宝玉和平儿、香菱的独特情愫。
}
今日更是意外之意外的事了。
一壁胡思乱想,\ji{又下此四字。
}来至房中,拉了袭人,细细告诉了他原故。
香菱之为人,无人不怜爱的。
袭人又本是个手中撒漫的,\zhu{撒漫:这里是大手大脚、不吝惜财物的意思。
}况与香菱素相交好,一闻此信,忙就开箱取了出来折好,随了宝玉来寻着香菱,他还站在那里等呢。
袭人笑道:“我说你太淘气了,足的淘出个故事来才罢。
”香菱红了脸,笑说:“多谢姐姐了,谁知那起促狭鬼使黑心。
”说着,接了裙子,展开一看,果然同自己的一样。
又命宝玉背过脸去,自己叉手向内解下来,将这条系上。
袭人道:“把这脏了的交与我拿回去,收拾了再给你送来。
你若拿回去,看见了也是要问的。
”香菱道:“好姐姐,你拿去不拘给那个妹妹罢。
我有了这个,不要他了。
”
\ping{
对比“平儿理妆”一节,从宝玉将平儿的衣服熨了叠好和之前与平儿的对话可以看出,
平儿换下来的衣服是要还回去的,这就使得宝玉借给平儿衣服这件事变得极其普通合理。
可是香菱却不同,宝玉是将衣服给了香菱的,而香菱也没有再要回自己的衣服,
这在香菱一方来看实际上是隐蔽地完成了一次她与宝玉之间的私相授受,是一种隐秘的女儿心思。
}
袭人道:“你倒大方的好。
”香菱忙又万福道谢,袭人拿了脏裙便走。
\par
香菱见宝玉蹲在地下,将方才的夫妻蕙与并蒂菱用树枝儿抠了一个坑,先抓些落花来铺垫了,将这菱蕙安放好,又将些落花来掩了,方撮土掩埋平服。
香菱拉他的手,笑道:“这又叫做什么?怪道人人说你惯会鬼鬼祟祟使人肉麻的事。
你瞧瞧,你这手弄的泥乌苔滑的,还不快洗去。
”\ping{宝玉把香菱的夫妻蕙和自己的并蒂菱埋在一起,甚至香菱都觉得肉麻不太合适,可能暗示两人之间未来的姻缘。
}宝玉笑着,方起身走了去洗手,香菱也自走开。
二人已走远了数步,香菱复转身回来叫住宝玉。
宝玉不知有何话,扎着两只泥手,笑嘻嘻的转来问:“什么?”香菱只顾笑。
因那边他的小丫头臻儿走来说:“二姑娘等你说话呢。
”香菱方向宝玉道:“裙子的事可别向你哥哥说才好。
”说毕,即转身走了。
宝玉笑道:“可不我疯了,往虎口里探头儿去呢。
”\ping{薛蟠好吃醋。
第二十五回:(薛蟠)又恐薛姨妈被人挤倒,又恐薛宝钗被人瞧见,又恐香菱被人臊皮——知道贾珍等是在女人身上做工夫的。
}说着,也回去洗手去了。
不知端详,且听下回分解。
\par
\qi{总评:写寻闹是贾母不在家景况,写设筵亦是贾母不在家景况。
如此说来,如彼说来,真有笔歌墨舞之乐。
\hang
看湘云醉卧青石,满身花影,宛若百十名姝,抱云笙月鼓而簇拥太真者。
\zhu{太真:即杨玉环,道号太真,受宠于唐玄宗,封为贵妃。
}}
\dai{123}{憨湘云醉眠芍药裀}
\dai{124}{呆香菱情解石榴裙}
\sun{p62-1}{庆生日行令射覆,憨湘云醉眠芍药裀}{图右侧:宝玉、宝琴、岫烟、平儿四人同一天生日,众人围坐庆生,两个女先儿要弹词上寿被拒绝,雅坐无趣,须要行令,于是射覆,划拳。
图左侧:湘云屡次被罚,不胜酒力,不觉醉卧芍药丛石板之上。
}
\sun{p62-2}{观鱼下棋叙闲话}{图右侧:大家有在外观花的,也有扶栏观鱼的。
探春和宝琴下棋,宝钗岫烟观局。
图左侧:林黛玉和宝玉在一簇花下不知说些什么。
}