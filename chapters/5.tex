\chapter{开生面梦演红楼梦\quad 立新场情传幻境情}
\qi{万种豪华原是幻,何尝造孽,何是风流?曲终人散有谁留,为甚营求?只爱蝇头!一番遭遇几多愁,点水根由,泉涌难酬!}\par
题曰:\par 
春困葳蕤拥绣衾,\zhu{葳蕤:葳音“威”,蕤音“锐”二声。
本指草木茂盛枝叶下垂貌,此指无精打采、萎靡不振。}恍随仙子别红尘。
\par
问谁幻入华胥境?千古风流造孽人\foot{甲戌本无此回前诗,己本(贴条)、戚本、蒙本、杨本、舒本等有,据补。
}。
\zhu{华胥:胥音“须”,华胥是中国上古时期华胥国的女首领,她是伏羲和女娲的母亲。
《列子·黄帝》:“(黄帝)昼寝而梦,游于华胥氏之国。
……其国无师长,自然而已。
其民无嗜欲,自然而已。
不知乐生,不知恶死,故无夭殇;不知亲己,不知踈物,故无爱憎;不知背逆,不知向顺,故无利害;都无所爱惜,都无所畏忌。
入水不溺,入火不热。
斫挞无伤痛,指掷无痟痒。
乘空如履实,寝虚若处床。
云雾不硋其视,雷霆不乱其听,美恶不滑其心,山谷不踬其步,神行而已。
黄帝既寤,怡然自得”后用以指理想的安乐和平之境,或作梦境的代称。
}\par
\hop
却说薛家母子在荣府中寄居等事略已表明,此回则暂不能写矣。
\jia{此等处实又非别部小说之熟套起法。
}\par
如今且说林黛玉\jia{不叙宝钗,反仍叙黛玉。
盖前回只不过欲出宝钗,非实写之文耳,此回若仍续写,则将二玉高搁矣,故急转笔仍归至黛玉,使荣府正文方不至于冷落也。
今写黛玉,神妙之至,何也?因写黛玉实是写宝钗,非真有意去写黛玉,几乎又被作者瞒过。
\ping{这个评论者似乎是钗粉黛黑,写黛玉不是真的写黛玉,而是为了写宝钗。}
}自在荣府以来,贾母万般怜爱,寝食起居,一如宝玉,\jia{妙极!所谓一击两鸣法,宝玉身份可知。
}迎春、探春、惜春三个亲孙女倒且靠后。
\jia{此句写贾母。
}便是宝玉和黛玉二人之亲密友爱,亦自较别个不同,\jia{此句妙,细思有多少文章。
}日则同行同坐,夜则同息同止,真是言和意顺,略无参商。
\zhu{略无参(参:音“申”)商:指彼此感情融洽,没有一点隔阂、矛盾。
“参”和“商”都是星宿名,属二十八宿。
因两星此出彼没,故常用来比喻两人分离不得见面。
参商也用以比喻人与人之间感情不和。
}不想如今忽然来了一个薛宝钗,\jia{总是奇峻之笔,写来健拔,似新出之一人耳。
}年岁虽大不多,然品格端方,容貌丰美,\jia{此处如此写宝钗,前回中略不一写,可知前回迥非十二钗之正文也。
}\jia{欲出宝钗,便不肯从宝钗身上写来,却先款款叙出二玉,陡然转出宝钗,三人方可鼎立。
行文之法又一变体。
}人多谓黛玉所不及。
\jia{此句定评,想世人目中各有所取也。
按黛玉宝钗二人,一如姣花,一如纤柳,各极其妙者,然世人性分甘苦不同之故耳。
}而且宝钗行为豁达,随分从时,\zhu{随分从时:安于本分、顺应环境。
}不比黛玉孤高自许,目无下尘。
\jia{将两个行止摄总一写,实是难写,亦实系千部小说中未敢写者。
}故比黛玉大得下人之心。
便是那些小丫头子们,亦多喜与宝钗去顽笑。
因此黛玉心中便有些悒郁不忿之意,\jia{此一句是今古才人通病,如人人皆如我黛玉之为人,方许他妒。
}\jia{此是黛玉缺处。
}宝钗却浑然不觉。
\jia{这还是天性,后文中则是又加学力了。
\zhu{又加学力:意思大概是后天的努力,与前面的“天性”相对。}
}那宝玉亦在孩提之间,况自天性所禀来的一片愚拙偏僻,\jia{四字是极不好,却是极妙。
只不要被作者瞒过。
}视姊妹弟兄皆出一体,并无亲疏远近之别。
\jia{如此反谓“愚痴”,正从世人意中写也。
}其中因与黛玉同随贾母一处坐卧,故略比别个姊妹熟惯些。
既熟惯,则更觉亲密,既亲密,则不免一时有求全之毁、不虞之隙。
\zhu{求全之毁,不虞之隙:因要求完美而常有责难;因相处亲密而常有料不到的矛盾。
毁:诋毁,责难。
不虞:没料到。
隙:嫌隙、裂痕。
从《孟子·离娄上》“有不虞之誉,有求全之毁”句衍化而来,含义亦有所不同。
}\jia{八字定评,有趣。
不独黛玉、宝玉二人,亦可为古今天下亲密人当头一喝。
}\jia{八字为二玉一生文字之纲。
}\ping{亲密容易导致追求完美,追求完美导致放大了原本小的瑕疵。
}这日不知为何,他二人言语有些不合起来,黛玉又\jia{“又”字妙极!补出近日无限垂泪之事矣,此仍淡淡写来,使后文来得不突然。
}气的独在房中垂泪,宝玉又\jia{“又”字妙极!凡用二“又”字,如双峰对峙,总补二玉正文。
}自悔语言冒撞,前去俯就,那黛玉方渐渐的回转来。
\par
因东边宁府中花园内梅花盛开,\jia{元春消息动矣。
\zhu{元妃省亲的先兆。}
}贾珍之妻尤氏乃治酒请贾母、邢夫人、王夫人等赏花。
是日,先携了贾蓉之妻二人来面请。
贾母等于早饭后过来,就在会芳园\jia{随笔带出,妙!字义可思。
\zhu{会芳园:可能是作者和评论者在现实生活中经历过的一处园林。}
}游玩。
先茶后酒,不过皆是宁荣二府女眷家宴小集,并无别样新文趣事可记。
\jia{这是第一家宴,偏如此草草写。
此如晋人倒食甘蔗,渐入佳境一样。
\zhu{
倒食甘蔗:即“倒吃甘蔗“,歇后语,意为渐入佳境。
甘蔗愈近根部甜度愈高,故愈吃愈甜。
语本南朝宋·刘义庆《世说新语·排调篇》:
「顾长康啖甘蔗,先食尾。问所以,云:『渐至佳境。』」
}
}\par
一时宝玉倦怠,欲睡中觉,贾母命人好生哄着,歇息一回再来。
贾蓉之妻秦氏便忙笑回道:“我们这里有给宝叔收拾下的屋子,老祖宗放心,只管交与我就是了。
”又向宝玉的奶娘丫鬟等道:“嬷嬷姐姐们,请宝叔随我这里来。
”贾母素知秦氏是个极妥当的人,生得袅娜纤巧,行事又温柔和平,乃重孙媳中第一个得意之人,\jia{借贾母心中定评,又夹写出秦氏来。
}见他去安置宝玉,自是安稳的。
\par
当下秦氏引了一簇人来至上房内间。
宝玉抬头看见一幅画贴在上面,画的人物固好,其故事乃是《燃藜图》,\zhu{
藜:音“黎”,一年生草本植物,茎高数尺,老可为杖;燃烧时光亮耐久,可以当烛。
《燃藜图》:这是劝人勤学苦读的画。
题材来自六朝无名氏《三辅黄图·
阁部》所载故事:“刘向于成帝之末,校书天禄阁,专精覃思(覃:深)。
夜有老人著黄衣,柱藜杖,叩阁面进,见向暗中独坐诵书,老人乃吹杖端烟然(燃),因以见面。
授‘五行洪范’之文……至曙而去。
请问姓名,云,我是太乙之精。
”
}也不看系何人所画,心中便有些不快。
又有一幅对联,写的是:\par
\hop
世事洞明皆学问,人情练达即文章。
\jia{看此联极俗,用于此则极妙。
盖作者正为古今王孙公子,劈头先下金针。
}\jia{如此画联,焉能入梦?}\par
\hop
既看了这两句,纵然室宇精美,铺陈华丽,亦断断不肯在这里了,忙说:“快出去!快出去!”\ping{厌学,厌世俗,纵使睡觉也要远离。
}秦氏听了笑道:“这里还不好,可往那里去呢?不然往我屋里去吧。
”宝玉点头微笑。
有一个嬷嬷说道:“那里有个叔叔往侄儿的房里睡觉的礼?”秦氏笑道:“嗳哟哟!不怕他恼。
他能多大了,就忌讳这些个!上月你没看见我那个兄弟来了,\jia{伏下秦钟,妙!}
虽然和宝叔同年,两个人若站在一处,只怕那一个还高些呢。
”\jia{又伏下一人,随笔便出,得隙便入,精细之极。
}宝玉道:“我怎么没见过?你带他来我瞧瞧。
”\jia{侯门少年纨绔活跳下来。
}众人笑道:“隔着二三十里,那里带去?见的日子有呢。
”说着,大家来至秦氏房中。
刚至房门,便有一股细细的甜香\jia{此香名“引梦香”。
}袭了人来。
宝玉便愈觉得眼饧骨软,\zhu{
饧:音“行”,眼睛半睁半闭或呆滞无神。
眼饧:眼皮滞涩、朦胧欲睡。
}\jia{刻骨吸髓之情景,如何想得来,又如何写得来?}连说:“好香!”\chen{进房如梦境。
}入房向壁上看时,有唐伯虎画的《海棠春睡图》,\zhu{海棠春睡:喻杨贵妃醉态。
《明皇杂录》:“上尝登沉香亭,召妃子。
妃子时卯酒未醒(卯酒:在晨间喝的酒),高力士从侍儿扶掖而至。
上皇笑曰:岂是妃子醉耶?海棠睡未足耳。
”此图是否实有,未能确知。
}\jia{妙图。
}两边有宋学士秦太虚写的一副对联,\zhu{宋学士秦太虚:北宋词人秦观,一字太虚,乃苏(轼)门四学士之一。
词风婉约媚丽,多写男女情爱。
这副对联不见于其《淮海集》。
}其联云:\par
\hop
嫩寒锁梦因春冷,\jia{艳极,淫极!}芳气笼人是酒香。
\jia{已入梦境矣。
}\zhu{嫩寒:春天的微寒。
锁梦:不成梦,不能入睡。
}\ping{按照人民文学出版社的注释,上句的意思是春天的微寒导致不能入睡。
另一种可能的解释是春天还有一点微寒的时候,盖着被子睡觉,不愿意从春梦中醒来。
}\par
\hop

\chai{keqing}{可卿春困}
案上设着武则天当日镜室中设的宝镜,\zhu{武则天:唐高宗的皇后,后登极称帝,改国号为周。
史载她的宫闱生活十分秽乱。
据说在高宗时她曾造了一座镜殿,四壁都安着镜子(见清朱鹤龄注李商隐《镜槛》诗)。
作者在这里用了一系列与古代香艳故事的风流韵事有关的器物,渲染秦氏房中陈设的华丽秾艳。
}\jia{设譬调侃耳,若真以为然,则又被作者瞒过。
}一边摆着飞燕立着舞过的金盘,\zhu{赵飞燕:汉成帝的皇后,身轻善舞。
据乐史《杨太真外传》引《汉成帝内传》:“汉成帝获飞燕,身轻欲不胜风,恐其飘翥(翥:音“助”,鸟向上飞),帝为造水晶盘,令宫人掌之而歌舞。
”}盘内盛着安禄山掷过伤了太真乳的木瓜。
\zhu{太真:即杨玉环,道号太真,受宠于唐玄宗,封为贵妃。
安史之乱前,玄宗宠信安禄山,杨贵妃曾认安禄山为养子,关系暧昧。
木瓜伤乳事,可能从《诗经·卫风·木瓜》“投我以木瓜”句联想而来。
又据宋代高承《事物纪原》“诃子”条(诃:音“喝“):“贵妃私安禄山,指爪伤胸乳之间,遂作诃子饰之。
”掷瓜伤乳,因“掷”、“指”音同,“瓜”、“爪”形近,或即由此讹转附会而来。
}上面设着寿\sout{昌}[阳]公主于含章殿下卧的榻,\zhu{寿昌公主应是寿阳公主之误。
寿阳公主:南朝宋武帝刘裕的女儿。
据《太平御览·时序部》引《杂五行书》:“宋武帝女寿阳公主,人日(旧历正月初七)卧于含章殿檐下,梅花落于公主额上,成五出花(五出花:可能是五瓣花),拂之不去,皇后留之,……宫女奇其异,竞效之,今梅花妆是也。
”}悬的是同昌公主制的联珠帐。
\zhu{同昌公主制的联珠帐:苏鹗《杜阳杂编》:唐懿宗“咸通九年,同昌公主出降(出降:公主下嫁),宅于广化里,……堂中设连珠之帐,却寒之帘。
……连珠帐,续真珠以成也(真珠:珍珠)。
”(见《旧小说》七)}\ping{镜子、盘子、木瓜、榻、帐本是卧室陈设之物,而从宝玉的视角来看,却如此香艳,为下文他的春梦做了铺垫。
}
宝玉含笑连说:“这里好!”\chen{摆设就合着他的意。
}
秦氏笑道:“我这屋子,大约神仙也可以住得了。
”说着,亲自展开了西子浣过的纱衾,\zhu{浣:音“换”,洗。西子:即西施。
衾:被子。
传说中有西子浣纱的故事,明代梁辰鱼著传奇《浣纱记》即本此。
}移了红娘抱过的鸳枕。
\zhu{红娘抱过的鸳枕:红娘:崔莺莺的丫鬟。
这里是指莺莺到西厢与张生幽会时,红娘送衾枕事。
见《西厢记》第四本第一折。
}\jia{一路设譬之文,迥非《石头记》大笔所屑,别有他属,余所不知。
}\ping{对秦可卿卧室的淫侈描写,可能暗示了她“淫丧天香楼”的悲惨结局。
}于是众奶母伏侍宝玉卧好,款款散去,只留下袭人、\jia{一个再见。
}媚人、\jia{二新出。
}晴雯、\jia{三新出,名妙而文。
}
麝月\jia{四新出,尤妙。
}\jia{看此四婢之名,则知历来小说难与并肩。
}四个丫鬟为伴。
\jia{文至此不知从何处想来。
}秦氏便分咐小丫鬟们,好生在廊檐下看着猫儿狗儿打架。
\zhu{
第七十三回,呆大姐捡到了绣着两个人赤条条的盘踞相抱的香囊,猜测是两个妖精打架。
由此可见“打架”一词的特殊含义。
}
\jia{细极。
}\par
那宝玉刚合上眼,便惚惚睡去,犹似秦氏在前,遂悠悠荡荡,随了秦氏,至一所在。
\jia{此梦文情固佳,然必用秦氏引梦,又用秦氏出梦,竟不知立意何属?}\jia{惟批书人知之。
}但见朱栏白石,绿树清溪,真是人迹希逢,飞尘不到。
\jia{一篇《蓬莱赋》。
}宝玉在梦中欢喜,想道:“这个去处有趣!我就在这里过一生,纵然失了家也愿意,强如天天被父母、师傅打去。
”\jia{一句忙里点出小儿心性。
}正胡思之间,忽听山后有人作歌曰:\par
\hop
春梦随云散,\jia{开口拿“春”字,最紧要!}飞花逐水流。
\jia{二句比也。
\zhu{比:比喻。}
}\par
寄言众儿女,何必觅闲愁。
\jia{将通部人一喝。
}\ping{以梦散花落喻女怨男痴的儿女之情终将被打灭,难以顺遂心愿,空遗愁苦在心中。
}\par
\hop
宝玉听了,是女子的声音。
\jia{写出终日与女儿厮混最熟。
}歌声未息,早见那边走出一个人来,蹁跹袅娜,端的与人不同。
\zhu{端的:果然、真的。
}有赋为证:\zhu{赋:文体名。
起于战国,盛于两汉。
赋有骈体的,也有散体的。
}\par
\hop
方离柳坞,\zhu{柳坞:植柳以为屏障。
坞:音“误”,原指作为屏障的土堡。
}乍出花房。
但行处,鸟惊庭树;\zhu{鸟惊庭树:极言仙姑之美。
《庄子·齐物论》:“毛嫱、丽姬.人之所美也;鱼见之深入,鸟见之高飞,糜鹿见之决骤(决骤:快速奔走):四者孰知天下之正色哉?”后因以“鱼入鸟惊”形容女子的美貌,与“沉鱼落雁”义同。
}将到时,影度回廊。
\zhu{影度回廊:身影在回廊上移动。
这里似从《吴郡志》所载吴王夫差闻西施着木鞋步回廊之声的传说化出,形容仙姑身姿之美。
回廊:曲折回环的走廊。
}
仙袂乍飘兮,\zhu{袂:衣袖。
}闻麝兰之馥郁;\zhu{麝兰:麝香和兰草,为古代贵族妇女常佩之香料。
亦用以代指香气。
}荷衣欲动兮,\zhu{荷衣:用荷花、荷叶制成的衣裳,神仙的一种服饰。
屈原《九歌·少司命》:“荷衣兮蕙带。
”}听环佩之铿锵。
靥笑春桃兮,云堆翠髻;唇绽樱颗兮,\zhu{唇绽樱颗兮:形容双唇似刚成熟的樱桃那样鲜红饱满。
张宽《太真明皇并笛图》:“露湿樱唇金缕衣。
”}榴齿含香。
\zhu{榴齿:形容牙齿整齐如一排石榴子。
《博物论》:“石榴子似人齿,带淡红色,光皎若琥珀。
”}纤腰之楚楚兮,\zhu{楚楚:原义为鲜明整洁的样子,这里作纤细秀美解。
}回风舞雪;\zhu{回风舞雪:形容仙子体态轻盈飘忽。
曹植《洛神赋》:“仿佛兮若轻烟之蔽月,飘飘兮若流风之回雪。
”}珠翠之辉辉兮,满额鹅黄。
\zhu{满额鹅黄:妇女在额上涂嫩黄色作妆饰的习俗。
或谓始于汉代,或谓始于六朝。
李商隐《蝶》:“寿阳公主嫁时妆,八字宫眉捧额黄。
”鹅黄:嫩黄,黄色之娇美者,如幼鹅的毛色。
}出没花间兮,宜嗔宜喜;\zhu{宜嗔宜喜:无论是生气还是高兴,都使人感到美。
《西厢记》第一折:“我见他宜嗔宜喜春风面。
”}徘徊池上兮,若飞若扬。
蛾眉颦笑兮,将言而未语;莲步乍移兮,\zhu{莲步:旧时对美女脚步的称谓。
语本《南史·齐东昏侯纪》:“凿金为莲华(花)以帖地(帖:通“贴”),令潘妃行其上,曰:‘此步步生莲华(花)也。
’”}欲止而欲行。
羡彼之良质兮,冰清玉润;慕彼之华服兮,熌灼文章;\zhu{熌:同“闪”。
熌灼文章:花纹灿烂。
文章:花纹错杂相间。
}爱彼之貌容兮,香培玉琢;\zhu{香培玉琢:用香料造就,用美玉雕成。
}美彼之态度兮,凤翥龙翔。
\zhu{
翥:音“助”,鸟向上飞。
凤翥龙翔:意即龙飞凤舞,形容仙子体态风度的飘逸。
}其素若何?春梅绽雪。
其洁若何?秋菊披霜。
其静若何?松生空谷。
其艳若何?霞映澄塘。
其文若何?\zhu{文:华美,有文采,与“质”(朴实,缺乏文采)相对。
}龙游曲沼。
其神若何?月射寒江。
应惭西子,实愧王嫱。
\zhu{王嫱:即王昭君,汉元帝时宫人,貌美。
}吁!奇矣哉,生于孰地,来自何方?信矣乎,瑶池不二,紫府无双。
\zhu{瑶池、紫府:均古代传说中的仙境。
瑶池在昆仑山上,西王母所居之处。
见《史记·大宛列传赞》、《穆天子传》卷三。
紫府在青丘凤山,天真仙女曾游此地。
见《海内十洲记·长洲》。
}果何人哉?如斯之美也!\jia{按此书凡例,本无赞赋闲文,前有宝玉二词,今复见此一赋,何也?盖此二人乃通部大纲,不得不用此套。
前词却是作者别有深意,故见其妙。
此赋则不见长,然亦不可无者也。
}\par
\hop
宝玉见是一个仙姑,喜的忙上来作揖,笑问道:“神仙姐姐,\jia{千古未闻之奇称,写来竟成千古未闻之奇语。
故是千古未有之奇文。
}不知从那里来,如今要往那里去?我也不知这里是何处,望乞携带携带。
”那仙姑笑道:“吾居离恨天之上,灌愁海之中,乃放春山遣香洞太虚幻境警幻仙姑是也。
\jia{与首回中甄士隐梦境一照。
}\ping{警幻,可能是警告你人生是一个空幻的意思。
}司人间之风情月债,掌尘世之女怨男痴。
因近来风流冤孽,\jia{四字可畏。
}缠绵于此处,是以前来访察机会,布散相思。
今忽与尔相逢,亦非偶然。
此离吾境不远,别无他物,仅有自采仙茗一盏,亲酿美酒一瓮,素练魔舞歌姬数人,\zhu{素:没染色,白色。
练:绢。
魔舞:即天魔舞。
本为唐代一种宫廷舞乐,王建《宫词》:“十六天魔舞袖长。
”元顺帝至正十四年制天魔舞,系宫廷大型队舞,以宫女十六人,盛妆扮成菩萨相,有多种乐器伴奏,应节而舞。
}新填《红楼梦》\jia{点题。
盖作者自云所历不过红楼一梦耳。
}仙曲十二支,试随吾一游否?”宝玉听了,喜跃非常,便忘了秦氏在何处,\jia{细极。
}竟随了仙姑,至一所在,\chen{士隐曾见此匾对,而僧道不能领入,留此回警幻邀宝玉后文。
}有石牌横建,上书“太虚幻境”四个大字,两边一副对联,乃是:\par
\hop
假作真时真亦假,无为有处有还无。
\jia{正恐观者忘却首回,故特将甄士隐梦景重一滃染。
\zhu{滃:音“翁”,三声,云气腾涌,水大的样子。
滃染:烘染。
}}\par
\hop
转过牌坊,便是一座宫门,也横书四个大字,道是“孽海情天”。
又有一副对联,大书云:\par
\hop
厚地高天,堪叹古今情不尽;\par
痴男怨女,可怜风月债难偿。
\par
\hop
宝玉看了,\jia{菩萨天尊皆因僧道而有,以点俗人,独不许幻造太虚幻境以警情者乎?观者恶其荒唐,余则喜其新鲜。
}
\jia{
有修庙造塔祈福者,余今意欲起太虚幻境,\sout{以}[似]较修七十二司更有功德。
}心下自思道:“原来如此。
但不知何为‘古今之情’,又何为‘风月之债’?从今倒要领略领略。
”宝玉只顾如此一想,不料早把些邪魔招入膏肓了。
\jia{奇极,妙文!}\zhu{
膏:心之下。
肓:音“荒”,横膈膜。
膏肓:古代中医称心脏与横隔膜之间的部位叫膏肓。
《左传》成公十年:晋景公患重病,求医于秦国,秦桓公派名医缓前往医治,缓未到,晋景公梦见他的病化作两个童子藏到膏之下,肓之上。
缓诊断后说:“疾不可为也。
在肓之上,膏之下,攻之不可,达之不得,药不至焉,不可为也。
”后遂称病重垂危、不可救药叫病入膏肓。
}\ping{邪魔为何?古今之情?风月之债?无这一遭灵隐之旅,木石联盟可能还差点痴。
}当下随了仙姑进入二层门内,只见两边配殿,皆有匾额、对联,一时看不尽许多,惟见有几处写的是:“痴情司”、“结怨司”、“朝啼司”、“夜哭司”、“春感司”、“秋悲司”。
\jia{虚陪六个。
}看了,因向仙姑道:“敢烦仙姑引我到那各司中游玩游玩,不知可使得?”仙姑道:“此各司中皆贮的是普天之下所有的女子过去未来的簿册。
尔凡眼尘躯,未便先知的。
”宝玉听了,那里肯依,复央之再四。
仙姑无奈,说:“也罢,就在此司内略随喜随喜罢了。
”\zhu{随喜:佛教术语。
谓见人作善事而随之生欢喜心。
后游览参观寺院,亦称随喜。
}宝玉喜不自胜,抬头看这司的匾上,乃是“薄命司”\jia{正文。
}
三字,两边对联写道是:\par
\hop
春恨秋悲皆自惹,花容月貌为谁妍?\par
\hop
宝玉看了,便知\jia{“便知”二字是字法,最为紧要之至。
}感叹。
\par
进入门来,只见有十数个大厨,
\zhu{厨:储藏物品的柜子,通「橱」。}
皆用封条封着。
看那封条上,皆是各省地名。
宝玉一心只拣自己的家乡封条看,遂无心看别省的了。
只见那边厨上封条上大书七字云:“金陵十二钗正册”。
\jia{正文题。
}宝玉因问:“何为‘金陵十二钗正册’?”警幻道:“即贵省中十二冠首女子之册,故为‘正册’。
”宝玉道:“常听\jia{“常听”二字,神理极妙。
}人说,金陵极大,怎么只十二个女子?如今单我们家里,上上下下,就有几百女孩儿呢。
”\jia{贵公子口声。
}警幻冷笑道:“贵省女子固多,不过择其紧要者录之。
下边二厨则又次之。
馀者庸常之辈,则无册可录矣。
”宝玉听说,再看下首二厨上,果然一个写着“金陵十二钗副册”,又一个写着“金陵十二钗又副册”。
宝玉便伸手先将“又副册”厨门开了,拿出一本册来,揭开一看,只见这首页上画着一幅画,又非人物,亦非山水,不过是水墨滃染的满纸乌云浊雾而已。
\zhu{滃:音“翁”,三声,云气腾涌,水大的样子。
滃染:烘染。
}后有几行字迹,写道是:\par
\hop
霁月难逢,彩云易散。
\zhu{霁月难逢:雨过天晴时的明月叫“霁月”,点“晴”字,喻晴雯人品高尚,然而遭遇艰难。
《宣和遗事·元集》:“大概光风霁月之时少。
”彩云易散:隐指晴雯的横遭摧残而寿夭。
“彩云”,寓“雯”字(雯,即彩云)。
白居易《简简吟》:“大都好物不坚牢,彩云易散琉璃脆。
”}\par
心比天高,身为下贱。
\zhu{身为下贱:指晴雯身为女奴,地位十分低下。
}\par
风流灵巧招人怨。
\par
寿夭多因毁谤生,多情公子空牵念。
\jia{恰极之至!“病补雀金裘”回中与此合看。
}\par
\zhu{这首是晴雯判词。
画面喻晴雯处境的污浊与险恶。
多情公子:指贾宝玉。
}\par
\hop
宝玉看了,又见后面画着一簇鲜花,一床破席。
\zhu{画面寓“花气袭(谐音“席”)人”四字,隐花袭人姓名。
}也有几句言词,写道是:\par
\hop
枉自温柔和顺,空云似桂如兰。
\zhu{枉、空:徒然,白白地。
}\par
堪羡优伶有福,谁知公子无缘。
\jia{骂死宝玉,却是自悔。
}\par
\zhu{这首是袭人判词。
优伶:旧时对歌舞戏剧艺人的称谓,这里指蒋玉函。
公子:指贾宝玉。
根据脂批,袭人出嫁先于宝玉出家,故有末二句判词。
}\par
\hop
宝玉看了不解。
遂掷下这个,又去开了“副册”厨门,拿起一本册来,揭开看时,只见画着一株桂花,\zhu{画面“一株桂花”暗指“夏金桂”。
}下面有一池沼,其中水涸泥干,莲枯藕败。
后面书云:\par
\hop
根并荷花一茎香,\jia{却是咏菱妙句。
}
\zhu{
根并荷花:指菱根挨着莲根。
隐寓香菱就是原来的英莲。
}
平生遭际实堪伤。
\zhu{
遭际:遭遇。
}\par
自从两地生孤木,\jia{拆字法。
}致使香魂返故乡。
\par
\zhu{这首是香菱判词。
“莲枯藕败”隐指英莲及其结局。
两地生孤木:拆字法,两个“土”(地)字,加一个“木”字,指“桂”,寓夏金桂。
照画面与后二句判词,香菱的结局当被夏金桂虐待致死。
}\par
\hop
宝玉看了仍不解。
便又掷下,再去取“正册”看。
只见头一页上便画着两株枯木,木上悬着一围玉带;又有一堆雪,雪下一股金簪。
也有四句言词,道是:\par
\hop
可叹停机德,\jia{此句薛。
}\qi{乐羊子妻事。
}
\zhu{
停机德:指符合封建道德规范要求的一种妇德。
东汉乐羊子远出求学,中道而归,其妻以停下织机割断经线为喻,劝其不要中断学业,以期求取功名。
见《后汉书·列女传》,这里指薛宝钗。
}
堪怜咏絮才,\jia{此句林。
}\zhu{
咏絮才:指女子敏捷的才思。
晋人谢道韫,聪明有才辩,某天大雪,韫叔谢安问:“白雪纷纷何所似?”韫堂兄谢朗答道:“撒盐空中差可拟。
”道韫曰:“未若柳絮因风起。
”谢安赞赏不已。
见《世说新语·言语》。
这里指林黛玉。
}\par
玉带林中挂,金簪雪里埋。
\jia{寓意深远,皆生非其地之意。
}\par
\zhu{这首是薛宝钗和林黛玉判词。
玉带林中挂:前三字倒读谐“林黛玉”三字,又暗示贾宝玉对林黛玉的牵挂。
金簪雪里埋:金簪,喻“宝钗”,雪,谐音“薛”,句意暗寓其结局之冷落与凄苦。
}\par
\ping{人民文学出版社的注释对“挂”的解释是“牵挂”,但是“林中挂”是一个整体的有现实意义的意象——挂于林中,联系“玉带”倒读为“黛玉”,可能更合理的解释是黛玉挂于林中,即为自缢身亡。
人民文学出版社的注释对“金簪雪里埋”的解释是暗寓薛宝钗结局之冷落与凄苦,但是“雪里埋”也是一个整体的有现实意义的意象——埋在雪里,联系“金簪”喻宝钗,可能更合理的解释是宝钗埋在雪里,可能是宝钗流落户外冻死在雪天,也可能是死后无法体面的下葬,而在雪天被抛尸野外。
判词结尾都是暗示主人公的结局下场,并不避讳死亡的结局,如晴雯的判词“寿夭多因毁谤生”,香菱的判词“致使香魂返故乡”,元春的判词“虎兔相逢大梦归”,迎春的判词“一载赴黄粱”。
林黛玉和薛宝钗作为薄命司“金陵十二钗正册”中的女子,其结局必然是悲剧性的,至于是否是这样的悲惨死亡,也是有可能的,证据可以从第八回关于贾宝玉佩戴的玉的一首诗的尾句看出:“白骨如山忘姓氏,无非公子与红妆。
”,这里的公子显然是玉的主人即贾宝玉,而红妆是指和贾宝玉最亲近的女子,其中必然有薛宝钗这个贾宝玉的妻子和林黛玉这个贾宝玉的真爱,“白骨如山”暗示了他们结局的悲惨性,而且秦可卿的自缢而死也说明这种死法是贵族女子所能接受的,这样黛玉和宝钗悲惨死亡的结局就并不突兀违和了。
毕竟最后一支《红楼梦》曲“飞鸟各投林”末句“好一似食尽鸟投林,落了片白茫茫大地真干净!”正暗示了红楼梦这部书大悲剧的结局。
}\par
\hop
宝玉看了仍不解。
\jia{世之好事者争传《推背图》之说,
\zhu{
推背图:书名。相传为唐代李淳风与袁天纲共同编著的图谶,预言历代变革兴衰之事。
六十图,每图附七言诗一首,编至第六十图袁推李背停止,故称为「推背图」。
}
想前人断不肯煽惑愚迷,即有此说,亦非常人供谈之物。
此回悉借其法,为儿女子数运之机。
无可以供茶酒之物,
\zhu{无可以供茶酒之物:令人费解,联合上下文,可能和祖先祭祀和神灵崇拜有关。}
亦无干涉政事,真奇想奇笔。
}待要问时,情知他必不肯泄漏,待要丢下,又不舍。
遂又往后看时,只见画着一张弓,
\zhu{“弓”谐音“宫闱”的“宫”字。}
弓上挂一香橼。
\zhu{
橼:音“元”。香橼:即枸橼(音“举元”),常绿小乔木或大灌木,有短刺,叶子卵圆形,花瓣里面白色,外面淡紫色。
果实长圆形,黄色,果皮粗而厚。
供观赏,果实、种子、叶和根都可入药。
也称这种植物的果实。 
“橼”谐元春的“元”字。
}也有一首歌词云:\par
\hop
二十年来辨是非,榴花开处照宫闱。
\par
三春争及初春景,\jia{显极。
}虎兔相逢大梦归。
\par
\zhu{这首是元春判词。
三春:这里隐指迎春、探春、惜春。
初春:指元春。
争及:怎及。
“虎兔”,杨本作“虎兕”,已卯本原抄作“虎兕”但“兕”又点改为“兔”字。
兕(音“四”):犀牛类猛兽。
大梦归:死亡。
}\par
\ping{首联上句说元春一生谨言慎行二十多年,暗示了元春二十多岁即死去。
下句的榴指石榴,石榴多子,有多子多福的寓意,下句说榴花在宫里盛开,但是尚未结果,即暗示元春入宫受宠怀孕,但是尚未生产即死去。
}\par
\ping{尾联下句的“虎兔”,己本、杨本作“虎兕”。
若作“虎兔”解,可能是暗示元春死于寅虎年末、卯兔年初;若作“虎兕”解,可能是暗示元春死于两派政治势力的恶斗之中。
第十八回,元妃省亲点戏第二出《乞巧》。
《乞巧》即清初洪升《长生殿》传奇中的一出。
剧本演唐玄宗与杨贵妃的悲剧故事。
以杨贵妃喻贾元春,可见两人都是政治斗争的牺牲品。
}\par
\par
\hop
后面又画着两人放风筝,一片大海,一只大船,船中有一女子掩面泣涕之状。
也有四句写云:\par
才自精明志自高,生于末世运偏消。
\jia{感叹句,自寓。
}
\zhu{运偏消:命运偏偏愈来愈不济。}
\par
清明涕送江边望,千里东风一梦遥。
\jia{好句!}\par
\zhu{这首是探春判词。
画面暗指探春远嫁海隅,犹如断线的风筝,一去不返。
后二句诗与此意同。
}\par
\hop
后面又画几缕飞云,一湾逝水。
其词曰:\par
\hop
富贵又何为?襁褓之间父母违。
\zhu{
襁(音“抢”):背孩子用的系带;褓(音“保”):包孩子用的小被。
襁褓之间:指婴儿时期。
}
\par
展眼吊斜晖,湘江水逝楚云飞。
\zhu{
吊:凭吊,伤悼。
湘江水逝楚云飞:藏“湘”“云”二字,并暗用宋玉《高唐赋》中楚怀王梦会巫山神女事,喻夫妻生活的短暂,与该判词画面含意相同。
宋玉《高唐赋》中楚怀王梦会巫山神女事:“先王尝游高唐,怠而昼寝,梦见一妇人,曰:‘妾巫山之女也,为高唐之客。
闻君游高唐,愿荐枕席。
’王因幸之。
去而辞曰:‘妾在巫山之阳、高丘之阻,旦为朝云,暮为行雨。
朝朝暮暮,阳台之下。
”“云雨”、“巫山”、“巫阳”、“高唐”、“阳台”这些词暗示性行为的来源在此。
}
\par
\zhu{这首是史湘云判词。
前二句说史湘云自幼父母双亡,家庭的富贵井不能给她以温暖。
后二句说史湘云婚后好景不长,转眼之间夫妻离散。
}\par
\hop
后面又画着一块美玉,落在泥垢之中。
其断语云:\par
\hop
欲洁何曾洁,云空未必空。
\zhu{
洁:既指清洁,亦指佛教所说的净。
佛教宣扬现实世界是污秽的,唯有天堂佛国才算“净土”,所以佛教又称净教。
妙玉有“洁癖”,又身在佛门,故云欲“洁”。
空:超脱尘缘。
}
\par
可怜金玉质,终陷淖泥中。
\zhu{
金玉质:喻妙玉“出身不凡,心性高洁”。
淖(音“闹”):泥沼,烂泥。
}
\par
\zhu{这首是妙玉判词。
画面“一块美玉”寓其名,“落在泥垢之中”喻其结局。
后二句诗与此意同。
}\par
\ping{首联上句“欲洁何曾洁”,是指妙玉有洁癖,即“欲洁”,嫌弃穷苦粗鲁的乡下人刘姥姥,但是最终结局是“终陷淖泥中”,不得清洁而终,即“何曾洁”。
下句“云空未必空”,是指妙玉标榜自己超脱尘缘,即“云空”,但是对人分三六九等,区别对待,鄙视穷苦粗鲁的乡下人刘姥姥,并没有超脱尘缘,即“未必空”。
}\par
\hop
后面忽画一恶狼,追扑一美女,欲啖之意。
\zhu{啖(音“旦”):吃。}
其书云:\par
\hop
子系中山狼,得志便猖狂。
\jia{好句!}
\zhu{
子:旧时对男子的尊称。
系:是。
“子”“系”又合而成“孙(孫)”字,指迎春的丈夫孙绍祖。
“中山狼”:古代寓言,赵简子在中山(国名)打猎,把一只狼赶得走投无路,东郭先生将狼藏进袋中救了它;赵简子一走,狼反而要吃掉东郭先生(见明代马中锡《东田集》)。
后遂以中山狼比喻忘恩负义的人。
}
\par
金闺花柳质,一载赴黄粱。
\zhu{
赴黄粱:这里喻死亡。
唐人沈既济《枕中记》说:寒儒卢生枕在道士吕翁给他的一个神奇的枕上睡去,梦中享尽荣华富贵,梦醒,还不到蒸熟一顿黄粱米饭的时间,后以喻人生如梦。
}
\par
\zhu{这首是迎春判词。
画面与判词均暗示迎春嫁了忘恩负义的凶恶丈夫,致被折磨而死。
}\par
\hop
后面便是一所古庙,里面有一美人在内看经独坐。
其判云:\par
\hop
勘破三春景不长,缁衣顿改昔年妆。
\zhu{
勘破:看破。
三春:这里隐指迎春、探春、惜春。
缁(音“资”)衣:黑色的衣服,这里指僧尼服装。
}\par
可怜绣户侯门女,独卧青灯古佛旁。
\jia{好句!}\par
\zhu{这首是惜春判词。
画面与判词暗示惜春的结局是出家为尼。
据脂批,惜春为尼后过着“缁衣乞食”的生活。
青灯:佛前海灯,即长明灯,供于寺庙佛像前,灯内大量贮油,中燃一焰,长年不灭。
}\par
\hop
后面便是一片冰山,上有一只雌凤。
其判曰:\par
\hop
凡鸟偏从末世来,都知爱慕此生才。
\par
一从二令三人木,\jia{拆字法。
}哭向金陵事更哀。
\par
\zhu{这首是王熙凤判词。
画面的“雌凤”象征王熙凤,“一片冰山”喻王熙凤倚作靠山的财势似冰山难以持久。
“凡鸟”合而成“鳯(凤)”字,点其名。
事出《世说新语·简傲》:嵇康与吕安是好友,一次吕安去拜访嵇康,康不在,其兄嵇喜出门迎接,吕安不入,在门上题一“凤”字而去,嵇喜很高兴,以为称自己是凤凰,其实吕安嘲笑他是“凡鸟”。
一从二令三人木:难确知其含义。
或谓指贾琏对王熙凤态度变化的三个阶段:始则听从,续则使令,最后休弃(“人木”合成“休”字)。
据脂批,贾府“事败”,王熙凤曾落入“狱神庙”,后短命而死。
}\par
\ping{何为“哭向金陵事更哀”?脂评暗示用拆字法把尾联上句的“人木”合成“休”字,即王熙凤被休弃落难时,曾向自己的娘家,即金陵四大家族之一的王家,哭着寻求帮助,然而“事更哀”,即并没有得到帮助,反而使得事情更加糟糕。
联系到下文暗示王熙凤的女儿巧姐命运的第十一支《红楼梦》曲“留馀庆”中的“休似俺那爱银钱、忘骨肉的狠舅奸兄!”,可明确得知,王熙凤的兄弟,也就是巧姐的舅舅,不念骨肉亲情,贪财卖掉巧姐。
可能的情况如下:王熙凤被休弃,寻求金陵娘家兄弟的帮助照顾巧姐,但是反而被薄情贪财的兄弟欺骗,自己的女儿被其所卖,从而使本就悲惨的事情更加哀苦。
}\par
\hop
后面又有一座荒村野店,有一美人在那里纺绩。
其判云:\par
\hop
势败休云贵,家亡莫论亲。
\jia{非经历过者,此二句则云纸上谈兵。
过来人那得不哭!}\par
偶因济刘氏,巧得遇恩人。
\par
\zhu{这首是巧姐判词。
画面暗指巧姐的结局是成为以纺绩为生的乡村妇女。
判词前二句写巧姐在贾府势败后被“狠舅奸兄”所卖。
后二句写巧姐为刘姥姥所救。
巧:语意双关,含巧姐之“巧”与凑巧之“巧”。
恩人:指刘姥姥。
}\par
\hop
诗后又画一盆茂兰,旁有一位凤冠霞帔的美人。
\zhu{凤冠霞帔:古时后妃的冠饰。明清时亦作为嫁服。}
也有判云:\par
\hop
桃李春风结子完,到头谁似一盆兰。
\par
如冰水好空相妒,枉与他人作笑谈。
\jia{真心实语。
}\par
\zhu{这首是李纨判词。
画面暗示李纨晚年因子得贵、诰命加身。
(诰命:本指皇帝赐爵授官的诏令,在此义同“命妇”,代指受皇帝封赠的贵妇人。
)首句“桃李”、“完”寓李纨二字,全句喻说李纨早寡,
她刚生下贾兰不久,丈夫贾珠就死了,
所以她短暂的婚姻生活就像春风中的桃李花一样,一到结了果实,景色也就完了。
次句寓贾兰的“兰”字,兼指将来贾府诸子孙中唯贾兰显贵,后二句句意难以确定,或谓化用唐代僧人寒山《无题》诗“欲识生死譬,且将冰水比。
水结即成冰,冰消返成水。
”说李纨一生三从四德,晚年荣华方至,却随即死去,只留得一个诰封虚名,白白地给世人作谈资笑料。
}\par
\ping{尾联上句的“如冰水好”,一方面指李纨坚贞守寡,如“槁木死灰”般压抑情欲,寂寞孤独,像冰水一样清洁而又清冷,即使她晚年因子得贵、诰命加身,却无福消受,旋即死去;另一方面指李纨像冰水一样冷漠无情,下文第十二支《红楼梦》曲“晚韶华”中“虽说是、人生莫受老来贫,也须要阴骘积儿孙。
”,即有为了“莫受老来贫”吝啬不积阴骘即阴德,被指责“须要阴骘积儿孙”。
尾联上句的“空相妒”中的“相”,是“互相”之意,也可表示动作偏指一方。
李纨因何被嫉妒,又嫉妒谁呢?首先分析因何被嫉妒。
从这个判词和红楼梦曲可以看出,李纨守寡教子,子荣母贵;从第四十五回李纨和凤姐的对话可以得知,李纨因孤儿寡母,在经济待遇上受到贾府的特殊关照,很有钱,但是并不愿意自己掏钱帮姊妹们办诗社,而是来找凤姐要钱。
从红楼梦曲“人生莫受老来贫,也须要阴骘积儿孙”可知这笔钱在贾府事败抄家之后,并没有随之抄没,而是依旧留在李纨的手中,但是李纨吝啬薄情,不愿拿这笔钱接济落难的贾府众人。
那么第二个问题是,李纨嫉妒谁呢?王夫人不管事了,目前是侄女王熙凤管家。
李纨作为王夫人的长儿媳,论亲疏要比王熙凤与王夫人更亲,但是却没从王夫人那里继承管家权,其中的奥妙在第六十五回兴儿向尤二姐介绍贾府情况时已经透露:“我们家的规矩又大,寡妇奶奶们不管事,只宜清净守节。
”,即由于李纨丈夫贾珠的早亡,在妻以夫贵的传统社会,顿时失去了竞争管家权力的依傍,李纨对凤姐得宠掌权而自己被冷落并不十分满意,证据如下:第三十九回李纨道“凤丫头也是有造化的。
想当初你珠大爷在日,何曾也没两个人……若有一个守得住,我倒有个膀臂。
”第四十五回李纨笑道:“你(凤姐)今儿又招我来了。
给平儿拾鞋也不要,你们两个只该换一个过子才是。
”第三十九回李纨揽着他(平儿)笑道:“可惜这么个好体面模样儿,命却平常,只落得屋里使唤。
不知道的人,谁不拿你当作奶奶太太看。
”}\par
\hop
后面又画着高楼大厦,有一美人悬梁自缢。
其判云:\par
\hop
情天情海幻情身,情既相逢必主淫。
\par
漫言不肖皆荣出,造衅开端实在宁。
\zhu{漫言:同“漫说”、“漫道”,相当于“不要说”。
}\par
\zhu{这首是秦可卿判词。
根据脂批,小说第十三回回目原为:“秦可卿淫丧天香楼”。
画上所画当指此。
脂批又云:“老朽因(秦可卿)有魂托凤姐贾家后事二件……其言其意则令人悲切感服,姑赦之,因命芹溪(雪芹)删去。
”但曹雪芹虽删去了这段情节,却在判词和画中仍保留了初稿里关于秦可卿结局的某些暗示。
情天情海:与“太虚幻境”的匾额“孽海情天”义同,喻世间风月情多。
幻情身:幻变的情的化身。
后两句意谓别以为不长进的东西都出自荣国府,造祸开端的其实是宁国府里的人,指贾珍等伤风坏俗的秽行。
}\par
\hop
宝玉还欲看时,那仙姑知他天分高明,性情颖慧,\jia{通部中笔笔贬宝玉,人人嘲宝玉,语语谤宝玉,今却于警幻意中忽写出此八字来,真是意外之意。
此法亦别书中所无。
}恐把仙机泄漏,遂掩了卷册,笑向宝玉道:“且随我去游玩奇景,\jia{是哄小儿语,细甚。
}何必在此打这闷葫芦!”\jia{为前文“葫芦庙”一点。
}\par
宝玉恍恍惚惚,不觉弃了卷册,\jia{是梦中景况,细极。
}又随了警幻来至后面。
但见珠帘绣幕,画栋雕檐,说不尽那光摇朱户金铺地,雪照琼窗玉作宫。
更见仙花馥郁,异草芬芳,真好个所在。
\jia{已为省亲别墅画下图式矣。
}又听警幻笑道:“你们快出来迎接贵客!”一语未了,只见房中又走出几个仙子来,皆是荷袂蹁跹,羽衣飘舞,姣若春花,媚如秋月。
一见了宝玉,都怨谤警幻道:“我们不知系何‘贵客’,忙的接了出来!姐姐曾说今日今时必有绛珠妹子\jia{绛珠为谁氏?请观者细思首回。
}\ping{明写宝玉神游太虚幻境,虚写黛玉神游太虚幻境。
}
的生魂前来游玩,故我等久待。
何故反引这浊物来污染这清净女儿之境?”\jia{奇笔摅奇文。
\zhu{摅:音“书“,抒发、发表。}
作书者视女儿珍贵之至,不知今时女儿可知?余为作者痴心一哭,又为近之自弃自败之女儿一恨。
}宝玉听如此说,便唬得欲退不能退,\zhu{唬:同“吓”。
}
\jia{贵公子不怒而反退,却是宝玉天分中一段情痴。
}\qi{贵公子岂容人如此厌弃,反不怒而反欲退,实实写尽宝玉天分中一段情痴来。
若是薛阿呆至此闻是语,则警幻之辈共成齑粉矣。
\zhu{齑:音“机”。齑粉:碎成粉屑。}
一笑。
}果觉自形污秽不堪。
警幻忙携住宝玉的手,\jia{妙!警幻自是个多情种子。
}向众姊妹道:“你等不知原委:今日原欲往荣府去接绛珠,适从宁府所过,偶遇宁荣二公之灵,嘱吾云:‘吾家自国朝定鼎以来,\zhu{定鼎:传说夏禹曾收九州之金,铸造九鼎,夏商周三代都把它们作为传国的重器。
后世因称新朝定都建国为定鼎。
}功名奕世,\zhu{
奕:音“义”,重、累。
奕世:一代接一代,世代绵延。
}富贵传流,虽历百年,奈运终数尽,不可挽回者。
故近之子孙虽多,竟无一可以继业。
\jia{这是作者真正一把眼泪。
}其中惟嫡孙宝玉一人,禀性乖张,生情怪谲,虽聪明灵慧,略可望成,无奈吾家运数合终,恐无人规引入正。
幸仙姑偶来,万望先以情欲声色等事警其痴顽,\jia{二公真无可奈何,开一觉世觉人之路也。
}或能使彼跳出迷人圈子,然后入于正路,亦吾兄弟之幸矣。
’如此嘱吾,故发慈心,引彼至此。
先以彼家上、中、下三等女子之终身册籍,令彼熟玩,尚未觉悟。
故引彼再至此处,令其再历饮馔声色之幻,或冀将来一悟,亦未可知也。
”\jia{一段叙出宁、荣二公,足见作者深意。
}\par


说毕,携了宝玉入室。
但闻一缕幽香,竟不知所焚何物。
宝玉遂不禁相问,警幻冷笑道:“此香尘世中既无,尔何能知!此香乃系诸名山胜境内初生异卉之精,合各种宝林珠树之油所制,名为‘群芳髓’。
”\jia{好香!}宝玉听了,自是羡慕。
已而大家入座,小鬟捧上茶来。
宝玉自觉清香味异,纯美非常,因又问何名。
警幻道:“此茶出在放春山遣香洞,又以仙花灵叶上所带宿露而烹。
此茶名曰‘千红一窟’。
”\jia{隐“哭”字。
}宝玉听了,点头称赏。
因看房内,瑶琴、宝鼎、古画、新诗,无所不有,更喜窗下亦有唾绒,\zhu{唾绒:古代妇女刺绣,每当换线停针,用齿咬断绣线,口中常沾留线绒,随口吐出,俗谓唾绒。
李煜词《一斛珠》:“烂嚼红茸,笑向檀郎唾。
”“茸”与“绒”通。
}奁间时渍粉污。
\qi{是宝玉心事。
}\zhu{奁:音“连”,古代妇女梳妆用的镜匣和盛其他化妆品的器皿。
时:时常,经常。
}壁上也有一副对联,书云:\par
\hop
幽微灵秀地,\jia{女儿之心,女儿之境。
}\par
无可奈何天。
\jia{两句尽矣。
撰通部大书不难,最难是此等处,可知皆从无可奈何而有。
}\par
\hop
宝玉看毕,无不羡慕。
因又请问众仙姑姓名:一名痴梦仙姑,一名钟情大士,\zhu{大士:佛教称佛和菩萨为大士。
}一名引愁金女,一名度恨菩提,\zhu{菩提:佛教名词。
梵文音译,意译为觉悟、成佛。
释迦在毕钵罗树下觉悟成佛,佛家遂称该处为菩提场,该树为菩提树。
}各各道号不一。
少刻,有小鬟来调桌安椅,设摆酒馔。
真是:琼浆满泛玻璃盏,玉液浓斟琥珀杯。
更不用再说那肴馔之盛。
宝玉因闻得此酒清香甘冽,异乎寻常,又不禁相问。
警幻道:“此酒乃是百花之蕊,万木之汁,加以麟髓之醅,
\zhu{醅:音“胚”,未经过滤的酒。}
凤乳之麯酿成,\zhu{
麯:音“曲”,酿酒用的发酵物,多用大麦麸皮等制成。
这里“麟髓之醅”、“凤乳之麯”均极言酿造仙酒的原料之珍异。
}因名为‘万艳同杯’。
”\jia{与“千红一窟”一对,隐“悲”字。
}宝玉称赏不迭。
\par
饮酒间,又有十二个舞女上来,请问演何词曲。
警幻道:“就将新制《红楼梦》十二支演上来。
”舞女们答应了,便轻敲檀板,\zhu{檀板:乐器名。
即拍板,亦名牙板。
因用檀木制成,故名檀板,因其色红,亦称红牙板。
}款按银筝。
\zhu{款:动作缓慢、舒徐的样子。
按:弹筝的动作。
筝:一种弦乐器。
}
听他歌道是:\par
\hop
开辟鸿蒙……\jia{故作顿挫摇摆。
}\par
\hop
方歌了一句,警幻便说道:“此曲不比尘世中所填传奇之曲,\zhu{传奇之曲:明代以后通称南戏为传奇。
}必有生旦净末之别,\zhu{生旦净末:传统戏曲中的脚色类型,主要分为生、旦、净、丑四类,或生、旦、净、末、丑五类,总称为“行当”。
演员扮演人物,皆按“行当”,各有自己的表演程式(即法则),不能随意混用。
}又有南北九宫之限。
\zhu{南北九宫之限:南北九宫,指古代戏曲的宫调(即调式)。
南:指南曲(传奇);北:指北曲(杂剧);九宫:即九个宫调(正宫、中吕、南吕、仙吕、黄钟五宫,大石调、双调、商调、越调四调,合为九宫调)。
戏剧的曲牌,是受宫调限制的,某一曲牌属于某一宫调之内,不能放入其它宫调来用。
有的曲牌可以兼入两宫,但要按曲谱规定。
所以戏剧的宫调限制是很严的。
}此或咏叹一人,或感怀一事,偶成一曲,即可谱入管弦。
若非个中人,\zhu{个中人:指处在局中,洞悉内情的人。
在这里作“行家”解。
}\jia{三字要紧。
不知谁是个中人。
宝玉即个中人乎?然则石头亦个中人乎?作者亦系个中人乎?观者亦个中人乎?}不知其中之妙。
料尔亦未必深明此调,若不先阅其稿,后听其歌,翻成嚼蜡矣。
”\jia{警幻是个极会看戏人。
近之大老观戏,必先翻阅角本。
\zhu{角本:脚本,戏剧表演或电影、电视摄制等所依据的底本。}
目睹其词,耳听彼歌,却从警幻处学来。
}说毕,回头命小鬟取了《红楼梦》的原稿来,递与宝玉。
宝玉揭开,一面目视其文,一面耳聆其歌曰:\jia{作者能处,惯于自站地步,又惯于陡起波澜,又惯于故为曲折,最是行文秘诀。
}\par
\hop
第一支\quad 红楼梦引子\par
开辟鸿蒙,\zhu{开辟鸿蒙:开天辟地以来。
鸿蒙:旧指宇宙形成以前的原始浑沌状态。
}谁为情种?\jia{非作者为谁?余又曰:“亦非作者,乃石头耳。
”}都只为风月情浓。
趁着这奈何天,伤怀日,寂寥时,试遣愚\jia{“愚”字自谦得妙!}衷。
\zhu{遣:排遣,抒发。
愚:“我”的自谦词。
衷:衷曲,情怀。
}因此上,演出这怀金悼玉的《红楼梦》。
\jia{读此几句,翻厌近之传奇中必用开场副末等套,
\zhu{副末:一种宋元时戏剧的角色。常与副净出现,专司开场或制造效果。}
累赘太甚。
}\jia{“怀金悼玉”,大有深意。
}\par
\zhu{《红楼梦》十二支曲与金陵十二钗册子判词互为补充,预示了书中主要人物的命运和结局。
怀金悼玉:以薛宝钗(金)和林黛玉(玉)代指金陵十二钗。
}\par
\ping{怀金悼玉:薛宝钗(金)和林黛玉(玉)是和贾宝玉关系最亲近的两个人,薛宝钗是贾宝玉现实的妻子,林黛玉是贾宝玉理想的爱人,贾宝玉在婚后“空对着、山中高士晶莹雪,终不忘、世外仙姝寂寞林”,怀金,意为在现实中面对薛宝钗;悼玉,意为在心中永远哀悼念念不忘林黛玉。
“怀金悼玉”综合地反映了贾宝玉纠结矛盾的心态,也是宝黛钗三人爱情悲剧的真实写照。
}
\par
\hop
第二支\quad 终身误\par
都道是金玉良姻,\zhu{金玉良姻:指贾宝玉与薛宝钗的姻缘,书中有金锁配玉的说法。
}俺只念木石前盟。
\zhu{木石前盟:指贾宝玉与林黛玉的爱情,书中有神瑛侍者以甘露灌溉绛珠仙草的神话描写。
}空对着、山中高士晶莹雪,终不忘、世外仙姝寂寞林。
\ping{雪,谐音“薛”,暗指薛宝钗。
林,暗指林黛玉。
}叹人间,美中不足今方信。
纵然是齐眉举案,
\zhu{
齐眉举案:《后汉书·梁鸿传》:东汉孟光送饭给丈夫梁鸿时,总是将木盘高举,与眉平齐,后遂以“举案齐眉”喻妻子对丈夫的恭顺,成为妇德的楷模。
}
到底意难平。
\jia{语句泼撒,不负自创北曲。
}\par
\zhu{终身误曲名意即误了终身。
曲子从贾宝玉婚后仍念念不忘死去的林黛玉,写薛宝钗婚后境遇的冷落和难堪。
}
\hop
第三支\quad 枉凝眉\par
一个是阆苑仙葩,\zhu{阆苑仙葩:指林黛玉。
阆(音“郎”)苑:神仙的园林;仙葩(音“趴”):仙花。
}一个是美玉无瑕。
\zhu{美玉无瑕:指贾宝玉。
瑕:玉的疵病。
}若说没奇缘,今生偏又遇着他;若说有奇缘,如何心事终虚化?一个枉自嗟呀,\zhu{嗟呀:伤感叹息。
}一个空劳牵挂。
一个是水中月,一个是镜中花。
想眼中能有多少泪珠儿,怎经得秋流到冬尽,春流到夏!\par
\zhu{枉凝眉曲名意谓徒然悲愁。
曲子从宝黛爱情遇变故而破灭,写林黛玉泪尽而死的悲惨命运。
}\par
\hop
宝玉听了此曲,散漫无稽,\zhu{稽:考证,考核。
}不见得好处,\jia{自批驳,妙极!}但其声韵凄惋,竟能销魂醉魄。
因此也不察其原委,问其来历,就暂以此释闷而已。
\jia{妙!设言世人亦应如此法看此《红楼梦》一书,更不必追究其隐寓。
}因又看下面道:\par
\hop
第四支\quad 恨无常\par
喜荣华正好,恨无常又到。
\zhu{无常:本佛教用语,指世间一切事物忽生忽灭,变幻无定,后讹称勾命鬼。
这里指元春的死亡,兼有风云变幻的意味。
}眼睁睁、把万事全抛,荡悠悠、把芳魂消耗。
望家乡,路远山高。
故向爹娘梦里相寻告:儿命已入黄泉,\zhu{黄泉:旧时谓天玄地黄,称地下水为黄泉,后用以代指冥间。
}天伦呵,\zhu{天伦:旧时指父子、兄弟等天然的亲属关系,这里是父亲的代称。
}须要退步抽身早!\jia{悲险之至!}\par
\zhu{恨无常曲名有不得寿终与荣辱无定双重意思。
曲子从元妃的暴死,写贾府的即将大祸临头。
}\par
\ping{
第十八回元妃省亲时,演的第二出戏是《乞巧》。
这折戏写杨贵妃虽“蒙陛下宠眷”,却“恩移爱更”。于是唐明皇对她信誓旦旦:“在天愿作比翼鸟,在地愿为连理枝”。
但杨贵妃在马嵬之变成为牺牲品,暗示元妃“宠难凭”即最后失宠惨死的结局。
“望家乡,路远山高”是形象化的比喻与夸张:一方面喻指元春被“送到那不得见人的去处”,另一方面隐喻人鬼殊途、阴阳之隔。
在贾母和王夫人有例行省视元妃的机会时,元妃为什么直到死后才向爹娘“梦里寻告”呢?可能是因为元妃死前已失宠幽禁监视了起来,例行省视已被取消。\hang
元妃失势乃至死亡的原因大致有三个方面:一是因元春晋妃而得意忘形、作威作福的“杨国忠”们,包括贾珍、贾琏、贾赦、凤姐;二是第七十二回凤姐梦到方相为别家娘娘夺走贾家的锦,暗示忠顺亲王在元妃和别家娘娘宫闱斗争失败后通过抄家夺走贾家的荣华富贵;三是“虎兕相逢”暗示的皇朝上层斗争的激化,具体情形可能如此:边陲爆发了战乱(康、雍、乾三朝与西北准噶尔等部的战事从未或歇,第二十六回小贾兰“演习骑射”之伏,第五十回联诗有“赐裘怜抚戍,加絮念征徭”),贾府乃在“武荫之属”,必须从戎出征,但是子孙武业荒废不能应付厥职,再加上贾府祸事已发,最终被抄没治罪。
}
\par
\hop
第五支\quad 分骨肉\par
一帆风雨路三千,把骨肉家园齐来抛闪。
\zhu{抛闪:抛开。
}恐哭损残年,\zhu{残年:晚年,指老年人。
}告爹娘,休把儿悬念。
自古穷通皆有定,\zhu{穷通:人生遭遇的窘困和显达。
定:指命中注定。
}离合岂无缘?\zhu{缘:缘分,机缘。
}从今分两地,各自保平安。
奴去也,\zhu{奴:旧时女子的自称。
}莫牵连!\qi{探卿声口如闻。
}\zhu{牵连:牵挂留连。
}\par
\zhu{分骨肉曲名即骨肉分离的意思。
曲子从探春远嫁海隅时对父母的强颜劝慰,写她与骨肉亲人分离时的悲苦心境。
}\par
\hop
第六支\quad 乐中悲\par
襁褓中,父母叹双亡。
\jia{意真辞切,过来人见之不免失声。
}纵居那绮罗丛,\zhu{绮罗丛:代指富贵家庭。
}谁知娇养?幸生来、英豪阔大宽宏量,从未将儿女私情略萦心上。
好一似、霁月光风耀玉堂。
\qi{堪与湘卿作照。
\zhu{湘卿:史湘云。}
}\zhu{霁月光风:雨过天晴时明净景象,喻史湘云胸怀开朗,磊落光明。
光风:《楚辞·招魂》:“光风转蕙”。
王逸注:“光风,谓雨已(止)日出而风,草木有光也。
”}厮配得才貌仙郎,\zhu{厮配:相配。
才貌仙郎:据脂批,当指卫若兰。
}博得个地久天长,准折得幼年时坎坷形状。
\zhu{准折:抵销,弥补,抵偿。
坎坷:道路不平,喻人生道路的曲折多艰。
}终久是云散高唐,水涸湘江。
这是尘寰中消长数应当,\zhu{
寰:音“环“,广大的空间或地域。如:「寰宇」、「人寰」。
尘寰:人间罪恶太多,故佛家称人间为「尘寰」。
消长:《易·泰》:“君子道长,小人道消。
”消:灭;长:生,这里指盛衰变化。
数:气数,运数。
}何必枉悲伤!\jia{悲壮之极,北曲中不能多得。
}\par
\zhu{乐中悲曲名谓乐中寓悲。
写史湘云虽生于富贵之家,但自幼父母双亡,虽嫁得“才貌仙郎”,又中途离散。
“云散高唐,水涸湘江”两句喻史湘云的夫妻离散,晚景孤凄。
高唐:宋玉《高唐赋》中楚怀王梦会巫山神女事:“先王尝游高唐,怠而昼寝,梦见一妇人,曰:‘妾巫山之女也,为高唐之客。
闻君游高唐,愿荐枕席。
’王因幸之。
去而辞曰:‘妾在巫山之阳、高丘之阻,旦为朝云,暮为行雨。
朝朝暮暮,阳台之下。
”“云雨”、“巫山”、“巫阳”、“高唐”、“阳台”这些词暗示性行为的来源在此。
水涸(音“合”)湘江:传说舜南巡死于苍梧,二妃随征,溺于湘江,俗呼湘君(见汉刘向《列女传》)。
湘妃竹,又称斑竹。
产于湖南、广西,竹上有紫色斑点。
传说舜帝南巡,死于苍梧,其妃湘夫人追至,哭甚哀,以泪挥竹,故竹上斑点若泪痕。
见晋代张华《博物志》。
}\par
\hop
第七支\quad 世难容\par
气质美如兰,才华阜比仙。
\zhu{阜:音“复”,盛,丰富,兴盛。
}\jia{妙卿实当得起。
\zhu{妙卿:妙玉。}
}天生成孤僻人皆罕。
\zhu{罕:纳罕,诧异,惊奇。
}你道是、啖肉食腥膻,\zhu{啖(音“旦”):吃。
膻:羊膻气。
}\jia{绝妙!曲文填词中不能多见。
}视绮罗俗厌。
却不知太高人愈妒,过洁世同嫌。
\jia{至语。
}可叹这、青灯古殿人将老,辜负了、红粉朱楼春色阑。
\zhu{红粉:胭脂、香粉,代指年轻女子。
朱楼:指富贵人家女子住的绣楼。
春色阑:春色将尽,喻女子的青春将逝。
}到头来,依旧是风尘肮脏违心愿。
\zhu{风尘肮脏:一说“风尘”指扰攘的尘世;肮脏又作抗脏,不屈不阿的意思。
文天祥《得儿女消息诗》:“肮脏到头方是汉,婷婷更欲向何人?”一说“风尘”犹云烟花,旧指娼妓的生活;肮脏作龌龊不洁解。
}好一似、无瑕白玉遭泥陷,又何须王孙公子叹无缘。
\par
\zhu{世难容曲名意谓难为世俗所容。
写妙玉的为人及其不幸遭际。
}\par
\hop
第八支\quad 喜冤家\qi{“冤家”上加一“喜”字,真新真奇!}\par
中山狼,无情兽,全不念当日根由。
一味的骄奢淫荡贪还构。
\zhu{贪还构:词意难确指,或系贪婪和构陷的意思。
}
觑着那,\zhu{觑(音“去”):看。
}侯门艳质同蒲柳;\zhu{侯门艳质:犹言侯门千金小姐。
蒲柳:即水杨,易生易凋,旧时常用以比喻本性低贱或易于衰朽的东西,此取前一义。
这里是说孙绍祖作践迎春,不把她当作贵族小姐对待。
}作践的,公府千金似下流。
叹芳魂艳魄,一载荡悠悠。
\jia{题只十二钗,却无人不有,无事不备。
}\par
\zhu{喜冤家曲名意谓喜庆婚嫁招来冤家对头。
写迎春的婚后不幸遭际。
}\par
\hop
第九支\quad 虚花悟\par
将那三春看破,桃红柳绿待如何?把这韶华打灭,\zhu{韶华:美好时光,喻青春年华。
}觅那清淡天和。
\zhu{天和:自然的和气,亦即所谓元气。
觅天和:犹言修道养性。
}说什么天上夭桃盛,\qi{此休恰甚。
\zhu{休:停止。
“说什么”意思是“休说”,“说什么……杏蕊多”意思是不要说什么荣华富贵。
}}云中杏蕊多。
\zhu{天上夭桃、云中杏蕊:喻荣华富贵。
《诗·周南·桃夭》:“桃之夭夭,灼灼其华。
”夭夭:茂盛艳丽的样子。
《汉武帝内传》:七月七日西王母设天厨宴请汉武帝,以玉盘盛仙桃七颗,言此桃三千年一生实。
唐代高蟾《下第后上永崇高侍郎》诗以“天上碧桃和露种,日边红杏倚云栽”喻在朝显贵。
}到头来谁见把秋捱过?\zhu{捱:同“挨”,熬,撑。
}则看那白杨村里人呜咽,\zhu{
则看:只见。
白杨村:指坟地,古人在墓地多种白杨。
}青枫林下鬼吟哦。
\zhu{青枫林:用杜甫《梦李白》“魂来枫林青”之意。
}更兼着连天衰草遮坟墓。
这的是昨贫今富人劳碌,
\zhu{的是:真是。}
春荣秋谢花折磨。
似这般生关死劫谁能躲?闻说道西方宝树唤婆娑,\zhu{婆娑:娑音“缩”,似为梵文“婆颇娑”的省称,意即光明;或谓即“婆罗”,一种常绿乔木,相传佛祖释迦牟尼在此树下涅槃(逝世)。
}上结着长生果。
\zhu{长生果:虚拟的一种食之能长生不老的果实;或喻解脱人世一切痛苦而修成正果。
}
\jia{末句开句收句。
\zhu{最后一句,既留有回味(开),又收结全篇(收)。
}}\qi{喝醒大众,是极。
}\par
\zhu{虚花悟曲名意谓参悟到良辰美景皆虚幻,亦即“色空”的禅理。
写惜春因看破贾府好景不长而决意皈依佛门。
}\par
\hop
第十支\quad 聪明累\par
机关算尽太聪明,\zhu{机关:心机,权术。
黄庭坚《牧童诗》:“骑牛远远过前村,短笛横吹隔垅闻。
多少长安名利客,机关用尽不如君。
”}反算了卿卿性命。
\jia{警拔之句。
}\zhu{卿卿:夫妻间的爱称。
《世说新语·惑溺》:“王安丰妇常卿安丰,安丰曰:‘妇人卿婿,于礼为不敬,后勿复尔。
’妇曰:‘亲卿爱卿,是以卿卿,我不卿卿,谁当卿卿?’遂恒听之。
”这里用“卿卿”有嘲弄之意。
}生前心已碎,死后性空灵。
家富人宁,终有个家亡人散各奔腾。
枉费了、意悬悬半世心;\zhu{意悬悬:提心吊胆、时刻劳神。
}好一似、荡悠悠三更梦。
\jia{过来人睹此,宁不放声一哭?}忽喇喇似大厦倾,
\zhu{喇:拟声词,形容风、雨等的声音。}
昏惨惨似灯将尽。
呀!一场欢喜忽悲辛。
叹人世,终难定!\jia{见得到。
}\par
\zhu{聪明累曲名即聪明反为聪明误之意。
写王熙凤的悲惨结局和贾府一败涂地的情景。
}\par
\hop
第十一支\quad 留馀庆\par
留馀庆,\zhu{馀庆:与“阴功”义近,意为前辈的善行而使后辈获得善报。
《易·坤》:“积善之家,必有馀庆。
”}留馀庆,忽遇恩人;幸娘亲,幸娘亲,积得阴功。
劝人生,济困扶穷,休似俺那爱银钱、忘骨肉的狠舅奸兄!\zhu{奸兄:似指曾得凤姐好处的贾蔷之流。
续书指贾芸,恐非。
脂批云:“醉金刚一回文字伏芸哥仗义探庵”,可知贾芸后来对贾府有仗义行为。
}\ping{这里说的是巧姐被“狠舅奸兄”所卖,幸得刘姥姥搭救。
这里的“奸兄”,可能是指贾芹。
第五十三回,贾芹在贾府当差手头有钱,依旧来领接济贾府穷子弟的年物。
贾珍指责他过于贪心,顺带提到贾芹在差事上“为王称霸起来,夜夜招聚匪类赌钱,养老婆小子。
”从贾芹即使有钱也领救济,可以看出贾芹的贪财,有犯罪的动机;从贾芹交接匪类为非作歹,可以看出贾芹有犯罪的能力。
另外,贾珍谈到贾芹赌钱,一个可能就是贾芹变卖巧姐偿还赌债。
}正是乘除加减,\zhu{乘除加减:意谓人生的荣枯消长,浮沉赏罚,皆有定数,丝毫不爽。
}上有苍穹。
\zhu{苍穹:青天。
}\par
\zhu{留馀庆曲名意谓前辈留下的德泽。
写贾府势败家亡时骨肉相残及巧姐由刘姥姥救出火坑事。
}\par
\hop
第十二支\quad 晚韶华\par
镜里恩情,\jia{起得妙!}\zhu{镜里恩情:指李纨早寡。
}更那堪梦里功名!\zhu{梦里功名:似指贾兰“爵禄高登”后她即死去。
}那美韶华去之何迅!再休提绣帐鸳衾。
\zhu{绣帐鸳衾:代指夫妻生活。
}只这带珠冠,披凤袄,\zhu{珠冠、凤袄:旧时诰命夫人穿戴的服饰。
}也抵不了无常性命。
虽说是、人生莫受老来贫,也须要阴骘积儿孙。
\zhu{阴骘(音“治”):阴德。
}气昂昂头戴簪缨,气昂昂头戴簪缨,光灿灿胸悬金印;威赫赫爵禄高登,威赫赫爵禄高登,昏惨惨黄泉路近。
问古来将相可还存?也只是、虚名儿与后人钦敬。
\par
\zhu{晚韶华曲名寓“夕阳无限好,只是近黄昏”之意。
写李纨一生的枯荣变化。
}\par
\ping{李纨虽然教子有方,其子贾兰爵禄高登,但是也不过是虚名罢了,功名利禄都将散去。
李纨为了达到“莫受老来贫”的目标而冷漠吝啬,可是她为之努力一生的目标不过是泡影。
}\par
\hop
第十三支\quad 好事终\par
画梁春尽落香尘。
\zhu{画梁春尽落香尘:暗指秦可卿于天香楼悬梁自尽。
}\jia{六朝妙句。
}擅风情,秉月貌,便是败家的根本。
箕裘颓堕皆从敬,\jia{深意,他人不解。
}\zhu{箕裘(音“基球”):簸箕和皮袍。
《礼记·学记》:“良冶之子,必学为裘;良弓之子,必学为箕。
”意思是善于冶炼的人家,必定先要子弟学会缝补皮袍,为冶炼金属、烧陶土修补器具作准备;善于造弓的人家,必定先要子弟学会做簸箕,为弄弯木竹、兽角制成弓作准备。
后人遂常用“箕裘”来比喻祖先的事业。
箕裘颓堕:指贾府的儿孙不能继承祖业。
敬:指贾敬。
}家事消亡首罪宁。
\zhu{宁:指宁国府。
}宿孽总因情。
\zhu{宿孽:犹言祸根。
}\jia{是作者具菩萨之心,秉刀斧之笔,撰成此书,一字不可更,一语不可少。
}\par
\zhu{好事终曲名意谓情事终了,含有嘲讽意味。
曲子从秦可卿的悬梁自缢,写贾府纲常毁堕,道德败坏。
}\par
\ping{贾敬进士出身却抛弃家庭事业去庙里修仙,致使疏于管教的宁国府秽乱不堪,埋下了家事消亡的祸根。
}\par
\hop
第十四支\quad 收尾\quad 飞鸟各投林\jia{收尾愈觉悲惨可畏。
}\par
为官的,家业凋零;富贵的,金银散尽。
\jia{二句先总宁荣。
}有恩的,死里逃生;无情的,分明报应。
欠命的,命已还;欠泪的,泪已尽。
冤冤相报岂非轻,分离聚合皆前定。
欲知命短问前生,老来富贵也真侥幸。
看破的,遁入空门;痴迷的,枉送了性命。
\jia{将通部女子一总。
}好一似食尽鸟投林,落了片白茫茫大地真干净!\jia{又照看“葫芦庙”。
\zhu{
葫芦庙:葫芦庙炸供失火导致甄家“烧成一片瓦砾场”。
可能贾家最后也是在一场大火中化为灰烬,“落了片白茫茫大地真干净”。
}
与“树倒猢狲散”反照。
\zhu{树倒猢狲散:第十三回秦可卿托梦王熙凤和第二十二回贾母灯谜处的脂批都提到了这个俗语,都在暗示贾府的结局。}
}\par
\zhu{飞鸟各投林曲名喻家败人散各奔东西。
此曲总写贾宝玉,和金陵十二钗等的不幸结局和贾府最终“树倒猢狲散”的衰败景象。
据脂批透露,贾府“事败抄没”后“子孙流散”,“一败涂地”。
贾宝玉“悬崖撒手”,“弃而为僧”。
}\par
\hop
歌毕,还要歌副曲。
\jia{是极!香菱、晴雯辈岂可无,亦不必再。
}警幻见宝玉甚无趣味,\qi{自站地步。
}因叹:“痴儿竟尚未悟!”那宝玉忙止歌姬不必再唱,自觉朦胧恍惚,告醉求卧。
警幻便命撤去残席,送宝玉至一香闺绣阁之中,其间铺陈之盛,乃素所未见之物。
更可骇者,早有一位女子在内,其鲜艳妩媚,有似乎宝钗,风流袅娜,则又如黛玉。
\jia{难得双兼,妙极!}\ping{日有所思夜有所梦,宝钗黛玉这两个人是贾宝玉隐秘的性幻想对象,所以会出现在宝玉的春梦里。
}正不知何意,忽警幻道:“尘世中多少富贵之家,那些绿窗风月,绣阁烟霞,皆被淫污纨袴与那些流荡女子悉皆玷辱。
\jia{真极!}更可恨者,自古来多少轻薄浪子,皆以‘好色不淫’为饰,
\zhu{淫:过度、沉浸、放纵;或指不正当的男女关系。}
又以‘情而不淫’作案,\zhu{情而不淫:感情志趣相投,却不流于淫乱。
案:事件,这里指淫乱之事。
}\qi{“色而不淫”四字已滥熟于各小说中,今却特贬其说,批驳出矫饰之非,可谓至切至当,亦可以唤醒众人,勿为前人之矫词所惑也。
}此皆饰非掩丑之语也。
好色即淫,知情更淫。
是以巫山之会,云雨之欢,皆由既悦其色,复恋其情所致也。
\jia{“色而不淫”,今翻案,奇甚!}吾所爱汝者,乃天下古今第一淫人也!”\jia{多大胆量敢作如此之文!}\par
宝玉听了,唬的忙答道:“仙姑错了。
我因懒于读书,家父母尚每垂训饬,\zhu{饬:音“赤”,整顿,整治。
}岂敢再冒‘淫’字?况且年纪尚小,不知‘淫’字为何物。
”\jia{绛芸轩中诸事情景由此而生。
}警幻道:“非也。
淫虽一理,意则有别。
如世之好淫者,不过悦容貌,喜歌舞,调笑无厌,云雨无时,恨不能尽天下之美女供我片时之趣兴,\jia{说得恳切恰当之至!}此皆皮肤淫滥之蠢物耳。
如尔则天分中生成一段痴情,吾辈推之为‘意淫’。
\jia{二字新雅。
}‘意淫’二字,惟心会而不可口传,可神通而不能语达。
\jia{按宝玉一生心性,只不过是体贴二字,故曰“意淫”。
}汝今独得此二字,在闺阁中,固可为良友,然于世道中未免迂阔怪诡,\zhu{迂阔:不切合实际。
}百口嘲谤,万目睚眦。
\zhu{睚眦(音“牙自”):瞪眼、怒目而视。
也引伸为很小的怨隙。
《史记·范睢传》:“一饭之德必偿,睚眦之怨必报。
”}今既遇令祖宁荣二公剖腹深嘱,吾不忍君独为我闺阁增光,见弃于世道,是以特引前来,醉以灵酒,沁以仙茗,警以妙曲,再将吾妹一人,乳名兼美\jia{妙!盖指薛林而言也。
}字可卿者,许配于汝。
\ping{本回前文有“鲜艳妩媚,有似乎宝钗,风流袅娜,则又如黛玉”的描述,由此可知宝玉梦境里的“可卿”,乳名兼美,即兼有薛宝钗和林黛玉两个人的美。
这位梦中的“可卿”很可能是以现实中贾宝玉的侄儿媳妇秦可卿为原型的,这么说来,秦可卿也是贾宝玉性幻想的对象,所以也出现在了宝玉的春梦里。
}今夕良时,即可成姻。
不过令汝领略此仙闺幻境之风光尚然如此,何况尘境之情景哉?而今后万万解释,\zhu{解释:这里是领悟、不受困惑的意思。
}改悟前情,将谨勤有用的工夫,置身于经济之道。
\foot{按:第五回结尾部分,各本与甲戌本间有一些异文,常引起研究者的注意。
现据己本择要对照如下(他本个别文字有出入):“将谨勤有用的工夫,置身于经济之道”:“留意于孔孟之间,委身于经济之道”。
}\zhu{经济之道:指封建社会的治国理民,经邦济世之道。
}
”\qi{说出此二句,警幻亦腐矣,然亦不得不然耳。
}说毕,便秘授以云雨之事,\qi{这是情之未了一着,
\zhu{一着:计策、方法。}
不得不说破。
}推宝玉入帐。
\par
那宝玉恍恍惚惚,依警幻所嘱之言,未免有阳台、巫峡之会。
\qi{如此方免累赘。
}\zhu{宋玉《高唐赋》中楚怀王梦会巫山神女事:“先王尝游高唐,怠而昼寝,梦见一妇人,曰:‘妾巫山之女也,为高唐之客。
闻君游高唐,愿荐枕席。
’王因幸之。
去而辞曰:‘妾在巫山之阳、高丘之阻,旦为朝云,暮为行雨。
朝朝暮暮,阳台之下。
”“云雨”、“巫山”、“巫阳”、“高唐”、“阳台”这些词暗示性行为的来源在此。
}数日来\foot{按:第五回结尾部分,各本与甲戌本间有一些异文,常引起研究者的注意。
现据己本择要对照如下(他本个别文字有出入):“未免有阳台、巫峡之会。
数日来,”:“未免有儿女之事,难以尽述。
至次日,便”。
},柔情缱绻,\zhu{缱绻(音“浅犬”):犹言缠绵。
语出《诗经·大雅·民劳》,本为牢固相结之意。
后多用以形容情投意合、难舍难分的样子。
}软语温存,与可卿难解难分。
\par
那日,警幻携宝玉、可卿闲游。
至一个所在,但见荆榛遍地,\qi{略露心迹。
}狼虎同群。
\qi{凶极!试问观者此系何处。
}忽尔大河阻路,黑水淌洋,又无桥梁可通。
\jia{若有桥梁可通,则世路人情犹不算艰难。
}宝玉正自徬徨,只听警幻道:“宝玉休前进,作速回头要紧!”\jia{机锋。
}宝玉忙止步问道:“此系何处?”警幻道:“此即迷津也。
\zhu{迷津:佛家谓三界(欲界、色界、无色界)六道(天道、人道、阿修罗道、畜生道、饿鬼道、地狱道)都是迷误虚妄的境界,故称迷津;世间众生,都陷溺于“迷津”之中,须赖佛家教义,觉迷情海,慈航普渡。
后用“迷津”比喻人沉溺于迷途之中。
津:江河的渡口。
}深有万丈,遥亘千里,中无舟楫可通,\qi{可思。
}只有一个木筏,乃木居士掌舵,灰侍者撑篙,\qi{特用“形如槁木,心如死灰”句以消其念,可谓善于读矣。
}不受金银之谢,但遇有缘者渡之。
尔今偶游至此,如堕落其中,则深负我从前一番以情悟道、守理衷情之言矣。
”\qi{看他忽转笔作此语,则知此后皆是自悔。
}
宝玉方欲回言,只听迷津内水响如雷,竟有一夜叉般怪物窜出,\zhu{夜叉:梵语音译,亦作药叉。
佛教多泛指恶鬼。
后来借以形容相貌丑陋、性情凶恶的人。
}直扑而来\foot{按:第五回结尾部分,各本与甲戌本间有一些异文,常引起研究者的注意。
现据己本择要对照如下(他本个别文字有出入):“那日,警幻携宝玉”一段:“因二人携手出去游玩之时,忽至一个所在,但见荆榛遍地,狼虎同群,迎面一道黑溪阻路,并无桥梁可通。
正在犹豫之间,忽见警幻从后追来,告道:‘快休前进,作速回头要紧!’宝玉忙止步问道:‘此系何处?’警幻道:‘此即迷津也。
……尔今偶游至此,设如堕落其中,则深负我从前谆谆警戒之语矣。
’话犹未了,只听迷津内水响如雷,竟有许多夜叉海鬼将宝玉拖将下去。
”}
。
吓得宝玉汗下如雨,一面失声喊叫:“可卿救我!可卿救我!”慌得袭人、媚人等上来扶起,拉手说:“宝玉别怕,我们在这里!”\qi{接得无痕迹。
历来小说中之梦未见此一醒。
}\par
秦氏在外听见,连忙进来,一面说:“丫鬟们,好生看着猫儿狗儿打架!”\qi{细,又是照应前文。
}又闻宝玉口中连叫:“可卿救我”,\jia{云龙作雨,不知何为龙,何为云,何为雨。
\zhu{云龙作雨:指《红楼梦》在情节安排上出人意表,神妙变化的艺术手法。}
}因纳闷道:“我的小名这里没人知道,他如何从梦里叫出来?”正是:\par
\hop
一场幽梦同谁近,千古情人独我痴\foot{“正是……”原无,据庚、戚、蒙、舒等本补。
己、杨本联语作:“梦同谁诉离愁恨,千古情人独我知。
”}。
\par
\hop
\qi{总评:将一部全盘点出几个,以陪衬宝玉。
使宝玉从此倍偏倍痴,倍聪明倍潇洒,亦非突如其来。
作者真妙心妙口,妙笔妙人。
}
\dai{009}{贾宝玉神游太虚幻境看判词}
\dai{010}{贾宝玉和可卿缠绵,迷津见夜叉}
\sun{p5-1}{贾宝玉神游太虚幻境}{宝玉在贾蓉媳妇秦氏房中歇息,图上侧:梦中恍惚跟着秦氏来到朱栏玉砌、绿树清溪的一处。
一美人自称警幻仙姑,邀请宝玉随她一游。
宝玉于是转过“太虚幻境”牌坊。
图右下:宝玉在“薄命司”见他家上中下三等女子的终身册籍,图左下:宝玉之后又见众仙女}
\sun{p5-2}{曲演《红楼梦》,宝玉误堕迷津}{图右侧:警幻仙姑请宝玉品茗饮酒。
这时又有十二个歌姬上来,为其演奏新制的《红搂梦》十二支曲子。
随后,仙姑又将其妹表字可卿者,许配给宝玉。
图左侧:次日,宝玉与可卿携手游玩之时,忽至一处,但见荆榛遍地,狼虎同群。
忽尔大河阻路,黑水淌洋,警幻从后追来,喝道:“快休前进,作速回头要紧!”话犹未了,只听迷津内如雷声滚滚,有许多夜叉海鬼直扑而来。
}