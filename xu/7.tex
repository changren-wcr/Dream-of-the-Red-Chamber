\section*{悬崖撒手}
1、羁候所门外(秋)\par
雁行阵阵。大门缓缓打开,苍白、憔悴的宝玉蹒跚而出。字幕(叠):第二十七集:悬崖撒手。
半空里传来几声哀鸣,宝玉抬头望去。一只失群的孤雁在夕阳里摇摇晃晃地扇动着翅膀。“宝二爷!”有人朗声招呼着宝玉。
宝玉应声回过头来。倪二笑嘻嘻地一抱拳:“宝二爷,恭喜恭喜!”宝玉疑问地看着倪二:“你是……?”倪二:“醉金刚倪二!”宝玉惊喜地:“原来你就是……,早听芸儿说起过,多亏了你照应!……你这是……?”
倪二咧嘴笑笑:“芸二爷走前再三嘱咐我,替他来接宝二爷。”宝玉不禁眼圈儿一红:“……他和小红……差不多该到了吧?”

2、驿道(秋)\par
西风残照,寒鸦数点。一辆马车远远行来,贾芸骑着马跟在车后。车棚内,小红深情地看着贾芸。贾芸往前面指了指。
小红回头看过去。道旁界碑上,刻着三个大字:平安州。小红酸楚地笑了。

3、街口小屋内(晚)\par
残杯冷炙错陈桌上。宝玉、倪二相对嘿然。一灯荧荧,把微弱、昏黄的光铺向茅椽蓬牖、瓦灶绳床。倪二微带醉意,伸手拍了拍宝玉:
“宝二爷,别难受了!亏了皇上的恩典,把政老爷杀头的罪,改成了流刑!那烟瘴地面虽说苦,可总比挨一刀的滋味好受!”
宝玉叹了口气:“……家亡人散!展眼间就剩下我一个人了……”倪二:“所以我劝宝二爷听芸二爷一句,到平安州去,跟他们一起过。”
宝玉苦笑着摇摇头。倪二:“可你一个人怎么活呢?”宝玉:“自食其力!”倪二惊异地看着宝玉,“宝二爷,我是个粗人,说句话你别见怪。”
宝玉看着倪二。倪二:“常日里听人说,宝二爷的本事都用在娘儿们身上了,成天花儿粉的,书都不好好念,这会子要自食其力?”
宝玉欲言又止。倪二拍拍脑袋:“噢——对了,对了!卖字,宝二爷可以卖字!”宝玉摇摇头:“我和笔墨的缘分已经尽了。”
倪二疑问地看着宝玉。宝玉站起身来,环视一周,半晌,斩钉截铁地:“我就在这儿干!”
倪二“唿”地一下站起来:“说笑话吧,二爷!我是让你在这儿临时落个脚,等着芸二爷安顿好了来接你……”
宝玉一把摘下挂在墙上的梆子:“就干这个!”倪二:“打更看街?”说着哈哈大笑。宝玉把梆子往绳床上一扔:“你笑什么?”
倪二“噗”地一屁股坐下:“好!我跟你说说都是些什么活儿,你掂量掂量。”宝玉坐下。倪二目不转睛地盯着宝玉:“夜里打更,睡不成囫囵觉;天不亮就得忙着吆喝车马——中间儿是官儿们过的,一般百姓走岔了道儿来不及回避,得唯你是问;傍晚儿车马稀了,又得忙着扫街泼水……”
宝玉冷冷一笑:“比狱神庙里怎样?”倪二:“这……,好吧!明儿我到营里给你挂个号儿,你干两天试试!”“哗啦”一声,老三惶惶推门而入。
老三:“宝二爷,刚得着信儿,你们太太……”宝玉倏地立起:“太太怎么了?”老三:“……跟着政老爷去,半道儿上……殁了!”
宝玉默默走到窗前,怔怔地看着夜空。宝玉噙着泪水,喃喃地:“早去……早好,省得……受罪……”
老三从怀里掏出一个用旧布裹着的什么,踌躇着递给宝玉:“二爷……”宝玉接过,慢慢打开,一愣:“嗯?”
玻璃绣球灯在昏暗的烛光下反射着五彩斑驳的幽光。老三轻声:“……对不住你,二爷!”倪二狐疑地看着老三:“怎么回事?”
老三嗫嚅着:“……这是……”宝玉把话接过来:“这是我存在他那儿的。”说着,泪水夺眶而出。

4、桥头(晨)\par
一个货郎摇着鼓,挑着鼓担下桥。宝玉的目光跟随着鼓担。货郎停步歇担,笑嘻嘻地:“要点儿什么?”
宝玉犹豫地:“要……”说着,不由自主地把手伸向一盒盒脂粉。货郎不解地看着宝玉。宝玉略有些尴尬地把手移向一根根小蜡烛:“要……要蜡烛。”

5、锦香院门前\par
门楼上悬挂着几支彩灯,下面嵌着一块匾额,上书“锦香院”三字。门前轿马纷攘,不时的有锦衣少年簇拥着绿蛾红粉进进出出。
门内传出一个女子清扬圆润的唱曲声:滴不尽相思血泪抛红豆,开不完春柳春花满画楼,睡不稳纱窗风雨黄昏后,忘不了新愁与旧愁……
宝玉远远定过,听见唱曲声,一愣,不由得停步,侧耳谛听。唱曲声在继续:咽不下玉粒金莼噎满喉 照不见菱花镜里形容瘦……
宝玉怔怔地听着。(闪回)宝玉唱“红豆曲”,锦香院的妓女云儿怀抱琵琶,轻掐檀槽。冯紫英、蒋玉函等齐声喝彩。薛蟠摇头咂嘴。(闪回完)
宝玉泪水盈眶。唱曲声在继续:展不开的眉头, 捱不明的更漏……过往行人诧异地看着宝玉。宝玉若有觉察地抹抹眼睛,转身快步离去。
唱曲声渐渐模糊:“呀!恰便似遮不住的青山隐隐,流不断的绿水悠悠……”

6、荣宁街\par
人声嘈杂。过往行人停步朝喧闹处张望。三、五个年青人从远处跑来,一边跑一边询问路人:“前边出什么事了?”一个老翁摇摇头:“不知道。”

7、荣国府正门前\par
喧笑声中,宝玉被人群团团围住,推来搡去。一个年青人高声喊着:“宝哥儿,你的那块玉呢?摔给我们看看呀!”
人群哄然响应:“对对,拿出来!”“摔呀!”高台阶上,十几个男仆开心地大笑。
一个妇人尖声叫着:“哎——,我这儿有才调好的胭脂,你吃不吃呀?”人群哄笑。
一个中年汉子笑着使劲推了宝玉一把。宝玉踉跄着倒地,呻吟了一声。中年汉子高声嚷着:“快,快喊姐姐妹妹就不疼了!”人群哄笑。
宝玉从地上爬起来,满面泪痕。各种各样的笑脸在宝玉跟前晃动着。

8、大观园围墙外(黄昏)\par
虎皮石上的粉墙坍了一个缺口。衣衫不整、鬓发散乱的宝玉站在缺口处,怔怔地朝园子里看着。
荒草丛生的大观园里,死一般的寂静。隐隐在望的潇湘馆,落叶萧萧,寒烟漠漠。
(幻觉)黛玉嗔怪的声音:“我给你的那个荷包也给他们了?你明儿再也别想要我的东西了!”
宝玉眼圈儿红红的,遥望着潇湘馆。(闪回)黛玉拿起剪刀,嚓嚓几下,把一个才做了一半的香袋儿剪破。宝玉眼眶里闪动着泪花。
(闪回)月夜下,病体支离的黛玉扶着紫鹃立在阶前,含泪微笑着:“……我去年答应过给你做个香袋儿,等我好了,我……”
宝玉“呜”的一下趴在断墙上,捶着墙恸哭失声。潇湘馆渐渐淹没在暮色里……

9、锦香院门前(夜)\par
夜阑更深,万籁俱寂。一盏风灯随着梆声晃动。杂役打扮的宝玉击柝行来。

10、街口(傍晚)\par
日影西斜。官道上,车马簇簇,尘土飞扬。杂役打扮的宝玉吃力地挑着两桶水,歪歪斜斜地走来。
宝玉把水桶搁在道边,抡起水瓢往街上泼水。一辆华丽的马车驶来。宝玉抡起的水瓢没有收住,“哗”地一瓢水泼在马背上。
宝玉:“哎呀!”赶车人怒声喝骂:“瞎了你的狗眼!”随手一鞭,“啪”一下抽在宝玉脸上。
宝玉“哎哟”一声,急忙捂脸。鲜血顺着指缝流下。车棚内响起一阵男男女女幸灾乐祸的笑声。
宝玉咬着牙。恨恨地看着远去的马车。“哐——”地一声锣响,喝道的衙役汹汹而来。行人、车马纷纷回避。
宝玉回头一惊,急忙跪在道边。一乘官轿,前呼后拥,缓缓行近。一个衙役抬脚踹翻两只水桶。
水桶滚下,泥水溅了宝玉满身满脸。宝玉跪在泥水里,动也不动。脸上的鲜血“叭哒叭哒”地滴落在面前。

11、街口小屋内(晚)\par
满屋浓烟。宝玉坐在瓦灶前,一面流泪咳嗽,一面往灶底添着柴草。“哗啦”一声,房门被推开。宝玉惊问:“谁?”
浓烟里,倪二趔趄着走进,醉熏熏地:“怎么……回事?这么多……烟!”宝玉咳嗽着起身:“是倪二哥!怎么这两天老没见着你?”
倪二往木凳上一坐:“哎哟”一声,急忙用双手一撑。宝玉:“怎么了?”倪二咧着嘴吸着气破口大骂:“那个杂种王八羔子,把他爷爷……给打了!”
宝玉吃惊地:“谁?”倪二:“前儿我……喝了点儿酒,躺在当街睡……觉,那个杂种……大轿过来,我没……起来,我说:‘老子喝酒是……是自己的钱!喝醉了躺的是……皇上的地!谁敢……管……老子!’那个王八羔子就……把我……关起来……打了一顿!”
说着又痛得吸了口气。宝玉:“是谁?”倪二瞪起布满血丝的眼睛,咬牙切齿地:“贵府上的……大恩人!”
宝玉急切地:“究竟是谁?”倪二一字一顿地:“贾雨村!”宝玉一愣:“他?”倪二被呛得咳嗽起来,歪歪斜斜地走到灶前,“扑通”跪下,一把拽出堵在灶底的柴草,用一根柴棍儿挑了挑灶底。
“烘”地一下,灶底吐出了火苗。倪二歪斜着脸,看着火苗,一咬牙:“明儿是……八月十五,我醉金刚伺候你……过节!”灶底的火苗越烧越旺。

12、桥头(中秋节·晚)\par
月出东山,清辉万里。杂役打扮的宝玉拎着梆子,伫立在桥头,怔怔地看着莹澈的满月。

13、荣国府前角门(中秋节·晚)\par
彩灯辉煌,弦歌绰约。轿马接踵而至,次第进入角门。街旁暗影里,男仆打扮的倪二眯着眼睛朝角门窥看。

14、桥头\par
宝玉从怀里掏出玻璃绣球灯,小心翼翼地捧在手上。

15、荣国府前角门\par
倪二混在一群跟着轿马的男仆中走进角门。

16、桥头\par
燃亮的玻璃绣球灯在月光下闪烁苔五彩陆离的光芒。

17、船上\par
一个女子擎着香从舱内走出,默默跪在船头,对月祝祷。舱内红烛高烧,桌上摆着大盘的月饼和各色菜肴。
一个便装官员同着几个请客围桌而坐,觥筹交错、笑语喧哗。几个婢女侍立左右。
船头跪拜的女子慢慢抬起头来,平视远方。香烟缕缕,在月光下缭绕。女子突然一愣,吃惊地看着远处桥头上的五彩光芒。
女子站起身来。喃喃地:“……玻璃绣球灯?……不,不会的。……是……是玻璃绣球灯!”
女子转身朝船尾跑去。船尾,一个中年船夫口里哼着小调,双手摇橹。女子跑来,一把抓住橹柄,急切地:“求求大爷,到前面桥下……略停一停。”
船夫不解地:“嗯?”女子急忙从腕上褪下两只翡翠手镯,递过去:“求求大爷……”

18、桥头\par
宝玉一只手擎着玻璃绣球灯,另一只手攥着衣袖,在灯上轻轻地擦拭着。忽明忽略的灯光照射着宝玉脸颊上一道长长的鞭痕。
突然,桥下传来一声惊问:“桥上拿灯的是谁?”宝玉一愣,急忙朝桥下张望。桥下,一条木船缓缓傍岸。
一个女子立在船头:“是……贾家的人吗?”宝玉惊异地:“你是谁?”船头女子:“是贾家的人吗?”宝玉急忙奔向桥下。

19、桥下\par
宝玉仔细辨认着船头女子:“你是……?”灯光映在宝玉的脸上。女子突然惊呼:“二哥哥!”宝玉惊疑地:“你是……谁?”
女子激动地:“二哥哥!我是……”宝玉:“谁?”女子转身跑到舱口,一把拔出插在舱门上的灯笼,跑回船头,把灯笼高高地举在脸旁,带着哭声:“二哥哥!”
宝玉喃喃地:“湘云?”湘云颤声:“是我!”宝玉不顾一切地趟进没膝深的水里,一下子扑在船帮上:“云妹妹!”
湘云“扑通”跪在船板上:“二哥哥……”兄妹二人百感交集,一个在船上,一个在水中,伸出的手刚能够在一起,互相紧紧抓住,痛哭失声。
舱内,灯红酒绿,拇战正酣。湘云抽泣着:“二哥哥,你……怎么一个人在这儿?”宝玉睁着泪眼,直直地看着湘云,说不出话。
湘云:“老太太、太太都……好吧?”宝玉哽咽着:“都……死了。”湘云一下闭上眼睛,泪水啪啪滴落。
湘云睁开眼睛:“……琏二嫂子呢?”宝玉:“……死了。”湘云嘴唇翕翕地抖着:“林姐姐呢?”宝玉:“……死了。”
湘云声音轻得几乎听不见:“……宝姐姐……?”宝玉摇了摇头,簌簌流着泪水:“……覆巢之下,安有完卵,……死的死……散的散!就剩下我……”
湘云趴在船板上“呜呜”哭起来。凉风飒飒,粼粼碧波拍打着船舷。远处飘来呜呜咽咽的笛声。
湘云慢慢抬起头来:“……那年中秋,我和林姐姐在凹晶溪馆联诗,是最后一次了……”宝玉含泪看着湘云。
湘云:“我记得……我的最后一句是‘寒塘渡鹤影’,林姐姐对了句‘冷月葬花魂’,谁知道,竞然都成了谶语!……今儿又是中秋了,我在水上飘流,她……连尸骨都……,……想再跟她拌几句嘴……也不能够了……”
宝玉无言唏嘘。凄清忧怨的笛声,断断续续地飘来。湘云突然拾起头:“二哥哥!今儿是团圆节,老天有眼,让我遇上你!咱们……再也别分开了!”
宝玉一下攥紧了湘云的手。一阵喝道声、杂沓的脚步声从桥上传来。宝玉激动地:“……再也不分开了!”

20、桥上\par
一乘大轿落地,前后簇拥的执事、护卫站了满满一桥。一个护卫从桥头探出身子,往桥下一指,“在那儿!”五、六个护卫快步往桥下跑去。

21、桥下\par
湘云:“……二哥哥,快把我赎出去吧!”五、六个护卫趟水下河,扭住宝玉,怒喝:“走!快走!”
湘云惊叫:“二哥哥!”宝玉:“云妹妹!”护卫们把宝玉拖上岸去。湘云哭喊:“二哥哥!”
船舱里的人纷纷跑出:“怎么回事?怎么回事?”便服官员抬头往桥上一看,大吃一惊,忙转身一把揪住船夫:“谁让你靠的岸?”
一个护卫指着船上怒喝:“船!摇走!”船夫急忙奔向船尾。岸上,宝玉挣脱了护卫,一下扑在水里,哭喊着:“云妹妹!”
木船缓缓离岸。船头,湘云发疯般地哭喊着往水里扑去,被几个人死死拉住。护卫们喝骂着从水里拖起宝玉。宝玉拼命挣扎。

22、船上\par
湘云撕心裂肺地哭喊着:“二哥哥!赎我……”船尾,船夫使劲摇着橹。船头,湘云抬头,远远看见桥头的玻璃绣球灯上下左右剧烈地扭动着、摇晃着。

23、桥头\par
护卫们拖着宝玉上桥。宝玉嘶声哭喊着:“云妹妹——”一个护卫夺下宝玉手中的玻璃绣球灯,“啪”地一下摔在桥头。
远远传来湘云撕裂人心的哭喊声:“……二哥哥……赎我……”护卫们把宝玉拖到大轿前,喝令:“跪下!”
一个护卫上前单腿一跪,气喘吁吁地:“启禀王爷,带来了。”宝玉一惊:“王爷?”
月光、灯笼交相辉映之下,北静王水溶一股怒气,端坐轿中。宝玉“唿”地一下站起来,哭喊一声:“王爷!”
左右齐声喝斥:“跪下!”一个护卫“啪”地一脚,把宝玉踹倒跪下。水溶厉声地:“大胆的东西!本藩路过,为何不在桥头伺候?”
宝玉哀哀哭泣。水溶:“来呀!”左右打雷般地应了一声。水溶:“捆起来!送巡检衙门!”
左右又如打雷般地应了一声。几个护卫过来,一脚踹翻宝玉,七手八脚地捆着。
宝玉挣扎着、哭喊着:“王爷!是我!”护卫们拖起宝玉就走。水溶一愣:“慢着!”
护卫们停步。水溶:“带过来!”护卫们推搡着宝玉上前。水溶看看左右:“灯!”
两只灯笼挑在宝玉身旁。宝玉鬓发散乱,满身泥水,嘴角挂着一条血渍。水溶吃惊地:“你是……宝玉?”
宝玉“扑通”跪下,呜呜大哭。水溶:“快!快松绑!”护卫们急忙解下宝玉身上的绳子。
水溶歉疚地:“怎么是你!嗐!你怎么……干了这个?”宝玉渐渐止住哭泣。水溶:“你出来……为什么不去找我?”
宝玉:“带罪之身,愧见王爷。”水溶:“唉呀!你真是太……,我派人到处找你!”
宝玉感动地:“王爷……”水溶:“咱们……回府细谈吧!”水溶把眼左右一扫:“看轿!”
一个侍从走到宝玉面前一躬身。宝玉急忙摇头:“不,不不!我不去……”水溶不解地:“怎么?”
宝玉含泪:“我要去找……我的……表妹。”水溶:“谁?”宝玉:“……史湘云。”水溶惊异地:“怎么?方才就是……?”
宝玉垂泪,使劲点了点头。水溶歉疚地低下头去,接着抬起头来:“我派人去找,你……跟我回去。”
宝玉摇摇头:“不!……王爷的厚恩,虽肝脑涂地,无以相报。可是……”水溶眼圈儿一红,急忙摇手:“别这么说……”
宝玉沉痛地:“……百年公府,瓦解冰销。如今,我孑然一身,了无牵挂。太白有云:‘弃我去者,昨日之日不可留。’王爷,让我……走自己的路吧……”
水溶低眉沉吟,半晌,含泪抬头:“……好吧,既这么说,我……不留你!”水溶转过脸去,朝着轿旁手捧八宝攒金木盒的侍从一摆手。
侍从连忙进前。水溶指指木盒:“打开!”侍从打开木盒。盒内满贮光彩夺目的珠宝翠钻。
水溶:“这是方才在大明宫侍宴,圣上赐给的。你……拿着。”宝玉后退一步:“不不,我……”
水溶:“……折变了,作……盘缠。”宝玉的泪水涌出了眼眶。(闪回)湘云跪在船头,撕心裂肺地哭喊着:“二哥哥!赎我……”
宝玉拭了拭泪:“好,我……自己拿。”宝玉脱下外衣,从盒里抓了一把,用外衣包起来。
水溶:“都拿上!”宝玉:“够了。”水溶:“你……”宝玉咽着泪水,双膝跪倒:“谢王爷!”
水溶垂泪:“我……等你回来。”宝玉立起身来,后退几步,在路边跪下。大轿抬起。
宝玉忽然想起了什么:“王爷!”水溶抬头。宝玉立起进前,扶着轿杆:“……有一件事,不知当问不当问。”
水溶:“说吧。”宝玉:“我们娘娘……究竟是怎么回事?”水溶簌簌流泪,连连摇着手:“别……别问了!说不得了……说不得了……”
大轿在执事、护卫的簇拥下颤颤行去。宝玉怔怔地目送着大轿。桂魄西斜,青光铺满空落的桥头。
宝玉慢慢下桥,俯身捡起一片片玻璃绣球灯的碎片,包在外衣里。突然,远处人喊马嘶,烈焰卷着黑烟腾空而起。
宝玉本能地喊了一声:“哎呀!荣国府!”拔脚朝着火处跑去。宝玉跑出十几步,突然停住,慢慢转身,走上桥头,遥望着被大火映红了的天空。
远处,风声、火声、哭喊声、尖叫声乱成一片。宝玉木然伫立。

24、水边(清晨)\par
布履短褐的宝玉远远走来。河水分成两条叉道,宝玉犹豫地向两边张望。宝玉沿着一条叉道的岸边走去。

25、小镇(深秋)\par
尖风呼啸。宝玉抱着一只破瓢,顶风走来。街旁一个小酒店,当门支着一口大锅,锅里煮着大块的卤肉,热气在屋里弥漫着。
铺面外挑着一个酒幌,旁边挂着两块木牌,一块写着“万里香”,一块写着“百年老汤”。
宝玉在酒店门口停步,瑟瑟抖着,眼睛直直地盯着肉锅。宝玉解开外衣,摸了摸腰间缠着的布袋子,犹豫着。
(闪回)湘云跪在船头,撕心裂肺地哭喊着:“二哥哥!赎我……”宝玉系上外衣,蹭到肉锅旁,把手里的破瓢伸过去:行行好……”
掌锅的伙计瞪着眼睛,一扬手里的铜勺:“滚!臭花子!”宝玉后退了两步,慢慢转身走去。

26、村居(冬)\par
凛冽的寒风撕扯营房顶的茅草。宝玉端着破瓢站在门前。门开着一条小缝儿,一个老妇伸出手来,把半碗剩菜场倒在破瓢里,门“砰”地一下关上了。
宝玉端起破瓢喝了一口,皱了皱眉,一咬牙“咕咚咕咚”几口把菜场灌了下去,又用手抓起瓢里的烂菜根,塞进嘴里。

27、破庙(雪夜)\par
雪花从破窗外飘进。宝玉围着一块破毡,蜷缩在墙脚里。

28、山泉旁(春)\par
泉水淙淙。宝玉双手掬起泉水,大口喝着。泉边开满了野花。宝玉坐在一块石头上,解下腰里的布袋子,小心翼翼地打开。
珠宝翠钻熠熠闪光。宝玉拣起玻璃绣球灯的碎片,怔怔地看着。

29、池塘边(夏)\par
赤日炎炎。衣衫褴褛的宝玉双手捂住腰间拼命地奔逃。几条恶狗狂吠着追咬。
池塘里爬上一群光屁股孩子,叫着、笑着,把一团团烂泥甩在宝玉身上。

30、江边(秋)\par
阴风怒号,浊浪排空。形容枯槁的宝玉坐在礁石上,失神地望着浪峰里的几点白帆。

31、江边(秋·黄昏)\par
千里澄江似练。宝玉坐在礁石上,茫然俯视着缓缓东去的逝水。(闪回)荣国府威严的兽头大门。荣国府上空的浓烟烈火。
凤尾森森、龙吟细细的潇湘馆。荒凉的颓垣断井。贾母慈祥的笑脸。上下挥舞的板子。元妃归省的仪仗。
草荐裹着的凤姐被扔进雪坑。泪光点点、娇喘微微的黛玉。娴静的宝钗。狰狞的狱神塑像。湘云眠芍。宝琴立雪。
拿着胭脂膏要往嘴里送的宝玉。“啪”的一鞭抽在宝玉脸上,鲜血顺着指缝流下。男妇老幼开心地大笑着的脸。(闪回完)
礁石上,宝玉痛苦地哼了一声,一下抱住自己的头。流水在礁石旁叹息着、呜咽着……仿佛从天边飘来一阵暗哑的歌声:
世人都晓神仙好,惟有功名忘不了;古今将相在何方?荒冢一堆草没了……
宝玉慢慢抬头。遥远的荒原上,麻履鹑衣的甄士隐飘然而来,口中哼着“好了歌”:世人都晓神仙好,只有金银忘不了;终朝只恨聚无多,及到多时眼闭了!世人都晓神仙好,只有姣妻忘不了;君生日日说恩情,君死又随人去了……
宝玉目不转睛地看着甄士隐从礁石旁蹒跚而过。歌声渐渐远去:世人都晓神仙好,只有儿孙忘不了;痴心父母古来多,孝顺儿孙谁见了?
甄士隐的背影在半轮落日里晃动着,歌声随风飘散了……宝玉托着珠宝翠钻,“嘿嘿”冷笑起来。
宝玉捏起一片碎玻璃,笑出了眼泪。宝玉站在礁石上,哭着、笑着。宝玉托起珠宝翠钻,“哗”地一下,抛入水中。
宝玉的笑声戛然而止。水面上漾起的片片涟漪,扩大着,扩大着,终于渐渐消失了……

32、山路上\par
衣衫破碎、目光炯炯的宝玉冷笑着走来。

33、荒野\par
一座新坟,一个披麻带孝的妇人跪在坟前哭号。宝玉冷笑。

34、村头\par
娶亲的队伍吹打着走来。宝玉看着花轿冷笑。

35、街市\par
宝玉举着破瓢,昂首走进一家店铺。

36、街市\par
宝玉被人从朱漆大门内推出滚下台阶。宝玉爬起来,捡起破瓢,拍拍屁股,冷笑着离去。

37、锦香院门前\par
徐娘半老的妓女云儿吃惊的脸。宝玉冷笑。

38、荣宁街\par
火后的荣宁二府,只剩下一片瓦砾。瓦砾堆里,长满了荒草。宝玉冷笑。

39、蒋宅院门外(冬夜)\par
带着啸音的北风呼啸着,卷起地上积雪,和天上的飞雪搅在一起,到处回旋着、铺洒着。
宝玉一手抱着破瓢,一手拄着棍子,搏击风雪,艰难地走来。宝正站在门外,回头看了看肆虐的风雪,把棍子一夹,伸手使劲拍门。

40、蒋宅正房内\par
烛光明亮,炭火熊熊。蒋玉菡、花袭人坐在桌旁拥炉对酌。一个小丫鬟不时地给二人斟酒。
拍门声传来。蒋玉菡、花袭人惊疑地对看了一眼。蒋玉菡:“怎么这样敲门?”花袭人:“这个时候会有谁来?”
蒋玉菡对小丫鬟:“你出去看看。”小丫鬟:“哎。”

41、院门外\par
宝玉用拳头砸门。院门“吱”地一声,开了一条缝儿。宝玉推门而入。

42、院内\par
小丫鬟惊问:“你找谁?”宝玉不答话,抬脚就往里走。小丫鬟一把拉住宝玉:“哎哎……”
小丫鬟举起灯笼一照。蓬头垢面的宝玉,双目熠熠闪光。小丫鬟“啊”地惊叫一声,转身就跑。一边跑,一边喊着:“大爷!大爷!快来呀……”
“哗啦”一声,正房门打开,蒋玉菡急步走出:“怎么回事?”小丫鬟指着宝玉,结结巴巴地:“他……他……”
蒋玉函接过灯笼,走到宝玉面前上下照照:“你要干什么?”宝玉平静地伸出手里的破瓢。
蒋玉菡回头对小丫鬟笑笑:“是个叫花子,看把你吓的!……给他拿点儿吃的,打发他走吧。”
蒋玉菡笑嘻嘻地看了宝玉一眼,转身欲回屋,忽然好象感觉到了什么。蒋玉菡转过身来,把灯笼举到宝玉面前,疑问地打量着宝玉。
宝玉毫无表情地看着蒋玉菡。蒋玉菡紧张地在宝玉脸上寻找着什么。(闪回)唱着“红豆曲”的宝玉。蓬头垢面的宝玉。
蒋玉菡摇摇头:“……不,不会是……”(闪回)明眸皓齿,面如满月的宝玉。蓬头垢面的宝玉。
小丫鬟走来,把一大碗饭菜倒向宝玉的破瓢。宝玉伸瓢接过。(闪回)宝玉伸手接过蒋玉菡递给的大红汗巾子。
蒋玉菡脱口而出:“是!……没错!一定是!”宝玉伸手抓起瓢里的饭菜往嘴边送去。
蒋玉菡急忙一把拉住:“慢着!你是不是……宝……”宝玉挣开蒋玉菡,把手里抓着的饭菜送进口中。
蒋玉菡激动地:“你……还认识我吗?”宝玉只顾抓起饭菜往嘴里填着。蒋玉菡:“我……我是……琪官儿!”
宝玉仿佛没有听见一般。蒋玉菡一把夺下破瓢扔在地下,拉起宝玉朝屋门走去。

43、正房内\par
蒋玉菡拉着宝玉进门。袭人吃惊地站起来:“你怎么……把个花子领进来了?”
蒋玉菡激动地:“花子?你仔细看看,他是谁?”袭人上下打量着宝玉。
宝玉嚼着嘴里的饭菜,使劲咽了下去。袭人的嘴唇哆嗦起来,喃喃地:“……是……”
宝玉眼睛扫视着房内。袭人“哇”地一声,扑倒在宝玉面前:“二爷!”宝玉的目光落在饭桌上。
袭人跪着往前挪了两步,紧紧抱住宝玉的腿,呜呜大哭。蒋玉菡含泪看着。
袭人使劲儿攥着宝玉已经冻硬了的破衣襟,哭喊着:“二爷!你……怎么落到这一步……”
宝玉挣脱开袭人,朝饭桌走去。蒋玉菡急忙抹抹眼泪,扶起袭人,哽咽着:“……先让二爷吃点儿东西,回头再……”
袭人呜咽着靠在墙上。蒋玉菡朝小丫鬟摆摆手:“快!烫酒……换菜!”小丫鬟答应一声,快步朝后面走去。
蒋玉菡:“二爷略等一等,就……”宝玉已经狼吞虎咽地吃起来了。蒋玉菡眼圈儿一红,泪水一下子涌了出来。

44、院内\par
大雪如幕。地上的破瓢渐渐被落雪埋住了……

45、正房内\par
宝玉穿着一身簇新的冬衣,闭着眼睛,坐在一张红木圈椅上,身子靠着椅背,腿上搭着一条大红汗巾子和一条松花汗巾子。
袭人啜泣着坐在椅旁,用梳子慢慢梳理着宝玉脏乱的头发。椅前地下放着一个盛满热水的铜盆,宝玉的双脚浸在热水里。
蒋玉菡蹲在铜盆前,挽着衣袖,细心地给宝玉洗着冻烂的双脚。宝玉平静地闭着眼睛。
蒋玉菡抚摸着宝玉脚上的裂口,泪水“啪哒啪哒”地掉在盆里。袭人给宝玉梳着头,不时抬起手背抹去腮边的泪水:
“……那年,我母亲、我哥哥要赎我回家,我说过……至死也不回去的。……原想着一心一意伺候二爷,能……长长远远的,可……到底……。那时候,我只有铁了心一死了!
可……学金钏儿的样儿吧,又觉着辜负了……老太太、太太待我的恩惠;……学司棋吧,又觉着对不起……母亲和哥哥……。况且……我又不是晴雯那样的刚性儿……”
蒋玉菡默默地给宝玉穿好鞋袜,把铜盆挪过一边。袭人看了看蒋玉菡,哽咽着:“……后来他娶了我,我哭过……闹过……不肯……,他待我好……没难为我……。
……后来,我见他拿着我原先的松花汗巾子,又知道我的那条大红汗巾子原是他的,才……”
袭人给宝玉挽好发髻,拿起一块帕子拭了拭泪,长吁了一口气:“……可见人一辈子怎么样,是有个定数的。……总算天开眼,又碰上二爷了!
……还有件喜事儿:宝姑娘……宝二奶奶和麝月头年儿给卖到这边儿来了,可巧让他碰上了,好说歹说,昨儿到底给赎出来了。
……道儿不好走,安顿在客店里,原说是雪一停就去接的。……二爷来了,让他……这会子就去接!明儿一早就……”
宝玉鼾声大作。袭人一愣,忙招呼蒋玉菡:“快着!搀到屋里炕上去!”蒋玉菡轻轻抱起宝玉,袭人在旁边扶着,走进卧房,把宝玉平放在炕上。
蒋玉菡悄声:“你招呼着二爷,我这就叫小厮套车。”袭人:“路上……当心!”
蒋玉菡点点头,转身走出。袭人掀起被子,轻轻盖在宝玉身上。(闪回)怡红院里,宝玉躺在小填漆床上,袭人从他项上摘下通灵玉,用自己的手帕包好,塞在枕下。
袭人看着宝玉憔悴的面庞,泪水滴落在被头上……

46、院门外(凌晨)\par
夜初雪霁,晨光熹微。两辆马车先后停住。蒋玉菡从一辆马车内跳下车,急忙走到另一辆马车后面,打起帘子。
麝月搀着宝钗下车。三人匆匆走进院门。

47、院内\par
院内一片静谧。三人“咯吱咯吱”踩着积雪定向房门。

48、正房内\par
袭人歪在红木圈椅上熟睡着,腮边挂着道道泪痕。三人进门。袭人惊醒,揉着眼睛站起来,激动地喊了声:“宝二奶奶!”
接着往卧房里指指,压低声音:“还睡着呢。”蒋玉菡打起门帘,宝钗、麝月、袭人轻轻走进。
袭人一声惊呼:“哎呀!人呢?”炕上空空的,锦被掀在一边。蒋玉菡:“快……找找!”说着转身跑出卧房。
袭人跟着跑出。宝钗走到炕边,一下坐在炕上,手抓着锦被,泪水慢慢涌出眼眶。麝月不知所措地:“二奶奶……”
蒋玉菡急步走进卧房,面色苍白:“我去追!”说着转身就走。宝钗:“不!”蒋玉菡回头。袭人走进。
宝钗咽着泪水:“……不要去……他……不会回来了……”袭人“扑通”一下跪在宝钗面前,号啕大哭:“二奶奶!……”
宝钗默默地从怀里掏出那块丝络网着的通灵宝玉,托在手上,失神地凝视着。泪水“啪啪”滴落在玉上……

49、驿道(晨)\par
漫天皆白。宝玉穿着一身簇新的冬衣,一手抱着破瓢,一只手拄着棍子,踩着厚厚的积雪,踽踽独行。
远远地,一队人马押着一辆囚车迎面行来。宝玉停步,目不转睛地看着囚车。
囚车内,贾雨村锒铛面坐,头被木枷锁在车外。囚车旁,当日应天府的门子拽棍挎刀,面色冷峻。
贾雨村凝视着路旁的宝玉,嘴角微微搐动了一下。宝玉冷笑起来。这是参透人生的冷笑!
笑声越来越响,充满了整个世界,在天地间久久地回荡着……歌声起:
为官的,家业凋零;富贵的,金银散尽。
有恩的,死里逃生;无情的,分明报应。
欠命的,命已还;欠泪的,泪已尽。
冤冤相报岂非轻,分离聚合皆前定。
欲知命短问前生,老来富贵也真侥幸。
看破的,遁入空门;痴迷的,枉送了性命。
好一似食尽鸟投林,落了片白茫茫大地真干净!
歌声里,宝玉向远处走去,走去。天地连成白茫茫一片,无边无际……