\section*{误窃通灵}

1、荣国府·贾母院(冬)\par
贾琏转过大理石插屏匆匆走进内院。正房台阶上坐着的几个丫鬟慌忙立起。字幕(叠):第二十二集:误窃通灵

2、贾母正房内\par
贾母闭着眼睛斜靠在榻上,鸳鸯轻轻地给贾母捶着肩背。一丫鬟禀报:“老太太,琏二爷来了。”
贾琏进门。贾母睁开眼睛。“老太太!”贾琏压低了的声音里透出抑制不住的兴奋,“喜事儿,老太太!”
贾母欲起身:“嗯?”鸳鸯连忙搀起贾母。贾琏:“老爷才刚打发人来说,今年京察,工部保举老爷‘一等称职’,
吏部带领引见,圣上念老爷勤俭谨慎,优先升放了外任!”贾母:“哦?”
贾琏:“老爷吩咐先告诉老太太,让老太太高兴高兴,老爷谢了恩就回来!”
贾母:“快着!快预备着庆贺庆贺!凤姐儿呢?快去找她来!”
贾琏:“我才打发人找去了。”凤姐恰好进门,笑吟吟地:“什么事儿这么勾魂儿似的勾我?”
贾母:“你看看这个猴儿,怎么就象是打地底下钻出来的似的,亏着没说她的坏话。”

3、荣国府议事厅外\par
手执对牌的丫鬟、仆妇进进出出。厅内传出凤姐的一声怒喝:“滚出去!”一个婆子悻悻地自厅内走出。

4、议事厅内\par
凤姐上首端坐,余怒未消。平儿给下面站着的丫鬟、婆子们使了个眼色。
众人退出。平儿悄声:“奶奶保重着些儿,身子才好了,……再说,她是大太太那边儿的,要是回去多个嘴传个话儿……”
一道阴影掠过凤姐的眉梢。门外一声咳嗽。平儿看了凤姐一眼,匆匆走出。

5、议事厅外\par
旺儿媳妇迎上来,悄悄递过一张当票:“平姑娘,给,当票。”
平儿:“银子呢?”“取回来了。讨二奶奶的示下,是不是……”
旺儿媳妇伸出两个手指,“还照着这个数的利息放出去?”
平儿急忙摆手:“不行不行!老爷升迁,说话就要赴任,几下里开销,正愁没地方抓挠银子呢!”

6、府内夹道\par
邢夫人气呼呼地走来,后面跟着被凤姐骂出议事厅的那个婆子。

7、议事厅外\par
平儿:“放出去的月钱赶紧把本息都收回来,不知道是哪个混帐乱嚼舌头,刚才正为这些事儿生气呢。
万一闹出来……”邢夫人走来。
平儿急忙把话打住。平儿、旺儿媳妇闪在一边,垂手而立:“太太……”
邢夫人不理,径向厅内走去。

8、议事厅内\par
凤姐嗒然而坐。邢大人带着婆子进门。凤姐吃了一惊,连忙离座,强笑着:“太太……”

9、议事厅外\par
平儿给旺儿媳妇使了个眼色,旺儿媳妇急忙离去。林之孝家的走来,笑嘻嘻地:“平姑娘!外头……”
平儿摆手。林之孝家的不解地看着平儿。
平儿悄悄指指厅内。林之孝家的朝厅内看了看,压低声音:“外头问明儿家里给老爷饯行,席安在哪儿?”
平儿略一沉吟:“……嗯……我去请老太太的示下。”

10、贾母房内\par
平儿垂手站着。贾母:“……酒席就摆在园子里,不拘哪一处吧,宽敞就好。
……才听说怡红院枯死的那棵海棠忽然这时侯开了,正好散了席都过去赏花儿……”

11、藕香榭\par
竹曲桥底,流澌浮漂。桥上,手捧盘盏的丫鬟往来穿梭。
冻云漠漠。欢声笑语不时在空中荡漾……

12、怡红院\par
半株海棠红绿争妍。麝月拿着花剪,细心地剪去枯萎的枝叶。
宋妈走过。麝月一眼瞥见:“宋妈妈,快把这些枯枝子收拾收拾,
待会儿席散了老太太要过来赏花呢!”“嗳嗳。”宋妈答应着,拿过靠在墙上的长把竹枝扫帚,轻轻扫去地下的败叶枯枝。
袭人扶着醉眼迷离的宝玉进门。袭人:“麝月!快过来搀一把,二爷喝多了!”麝月忙丢下花剪走过来搀住宝玉,嘻嘻笑着:“怎么了?喝了这么多?”

13、宝玉卧房内\par
“二爷今儿
袭人、麝月搀着宝玉走进卧房,服侍宝玉躺在小填漆床上。
麝月忙给宝玉脱靴。袭人把宝玉罩在外面的一件玄色狐腿外褂解下来。
麝月过来帮着扶起宝玉,袭人从宝玉颈上摘下玉来,用自己的罗帕包了,塞在枕下,接着,又把宝玉的一件狐腋箭袖脱了。 麝月揭起锦被,轻轻给宝玉盖好。麝月悄声:“怎么喝成这样儿?“
袭人:“今儿席上二爷连着写了两首好诗,老爷一高兴,多赏了几杯酒。”
院门口人声嘈杂。一个小丫鬟在院子里高声喊着:“袭人姐姐,袭人姐姐!老太太来了!”
袭人、麝月急忙走出卧房。

14、院内\par
鸳鸳搀着贾母进院门。袭人、麝月急忙迎上:“给老太太请安。”
邢夫人、王夫人、李纨、探春、惜春、赵姨娘、贾政、贾赦、贾琏、贾环、贾兰并一二十个丫鬟婆子随后陆续说笑着进院门。
贾母:“宝玉呢?”袭人:“回老太太话,宝二爷睡下了。”

15、宝玉卧房内\par
宝玉轻鼾渐起。

16、院内\par
修剪过的“女儿棠”,虽然只活了半边,却仍然是丝垂翠缕、葩吐丹砂。
众人围观,七嘴八舌地议论着:“真是希罕事儿!”“就是,已经枯死了,怎么又活了!”
“也没人浇它!”“啧啧……”贾母用手轻轻地捻着花瓣儿,凑近看着。
“怪事儿!怎么这个节气还能开花儿?”“真是的!”
两个丫鬟抬过宝玉日常的卧榻。鸳鸯扶着贾母坐下。
贾赦微微摇头:“这花开得不是时候,据我看,未必是什么好事儿!不如砍了去!”
贾政微微一笑:“见怪不怪,其怪自败。不用砍,随它去吧。”
贾母:“谁在那儿胡说呢?什么怪不怪的!要是有好事儿你们享去,要是不好,我一个人当去。”
邢夫人微微皱了皱眉。贾赦讪讪地退下。贾母:“你们知道什么?这花儿应该在三月里开的,
如今虽说是十一月,因为节气迟,还算十月,应着小阳春的天气,这花儿开是因为暖和!”
王夫人陪笑着:“老太太见得多,说得是,也不为奇怪。”
邢夫人:“听说这棵海棠枯死了快一年了,怎么这会子开了?这里头必有原故,不知道应在什么事儿上。”
李纨笑着:“老太太、太太说得都是。据我的糊涂想法儿,必是有喜事临门,这花儿先来报个信儿。”
众人都凑趣地笑起来。
平儿托着两匹红纱进门:“老太太!”众人回头。
平儿笑嘻嘻地走近贾母:“我们奶奶知道老太太在这里赏花,自己不得来,叫奴才来服侍老太太、太太们。……
还有两匹红送给宝二爷包裹这花儿,当作贺札。”
袭人连忙上去接了,呈给贾母看。贾母笑着:“偏是凤丫头行出点子事儿来,叫人看着又体面、又新鲜。”
袭人对平儿笑笑:“回去替宝二爷给二奶奶道谢。”
贾母:“凤丫头没来,少了个逗趣儿的。昨儿还好好的,怎么说病就病了。”
邢夫人嘴唇微微一撇。平儿给袭人使了个眼色,袭人走过来。
平儿悄声:“奶奶说,这花儿开得奇怪,叫你铰块红绸子挂挂避邪。以后也别当成希奇事乱说了。”
袭人看着平儿,微微点头。

17、院内(黄昏)\par
罗幕轻寒。宝玉鬓发尨茸,睡眼惺忪,披着一件裹圆的皮袄,伫立阶前,惆怅地看着在北风中瑟瑟发抖的“女儿棠”。
歌声起(哼鸣)。
(幻觉)艳丽的海棠花前,晴雯含笑而立。(闪回)晴雯大笑着,一把抢过麝月的扇子撕成两半。
(闪回)灯下,晴雯带病补裘,力尽神危。(闪回)晴雯挽着头发闯进房内,拎起箱底,尽力一倒。(闪回)芦席土炕上,晴雯强展星眸。
歌声:茜纱窗下,我本无缘,黄土陇中,卿何薄命……
(幻觉)盛开的“女儿棠”,花叶上挂满了晶莹的露珠,在阳光的照射下,闪烁着奇瑰的光彩……
“二爷。”袭人不知什么时候走来,站在宝玉身旁。宝玉:“嗯?”
“你怎么起来了?”宝玉:“……”
歌声渐隐。宝玉:“今儿都是谁来了?”
袭人:“老太太、大太太、太太、大老爷、老爷,还有大奶奶、琏二爷……”
宝玉:“林妹妹呢?”袭人:“林姑娘没来,说是身上不舒服……”
宝玉:“病了?……我去看看她。”宝玉急步走出院门。
袭人:“哎呀,二爷!穿好衣服……”

18、宝玉卧房内\par
袭人端着烛台进来,目光落在拉开的妆台抽屉上:“嗯?”袭人急忙走到妆台前,借着烛光往抽屉里看了看,转身喊了一声:“麝月!”麝月闻声走进。
“这抽屉是你开的?”麝月看看抽屉:“不是我。”“是二爷开的?”“没见二爷开!”
袭人眉头一皱:“秋纹病着,碧痕告假回家了,除了这几个人,别人是不许进这屋的……”
麝月:“快看看少了东西没有?”袭人拉出抽屉放在妆台上,仔细翻检着里面的东西。麝月:“好象没少什么。”
袭人:“不对!……那块“祖母绿”哪儿去了?”麝月紧张地:“啊?”

19、潇湘馆\par
曲槛疏竹。袭人呆呆地站着。宝玉走出房门,黛玉、紫鹃跟出。
宝玉:“外头冷,妹妹快进屋吧。”黛千耽心地:“袭人姐姐,出了什么事儿?”袭人强笑笑:“没……”

20、怡红院\par
丫鬟、婆子站了一地。灯火通明,鸦雀无声。
麝月铁青着脸,目光扫视着众人。 

21、潇湘馆\par
黛玉双手托腮,满腹疑虑地坐在书案前。
紫鹃用蜡剪儿剔着烛花:“……姑娘别乱猜疑了,不会出什么事儿的。”
黛玉抬起头来,呆呆地看着紫鹃。
紫鹃笑嘻嘻地说着什么,黛玉耳边响起的却是另外的声音:
(画外音)黛玉:“袭人姐姐,出了什么事儿?”
(画外音)袭人强笑笑:“没……”

22、怡红院\par
嘤嘤啜泣声。众人在麝月的监视下互相搜检。
宝玉、袭人进门。宝玉:“这是干什么?都别搜了!”
麝月:“哎,二爷……”宝玉:“抽屉是我拉开的。”
麝月惊异地看着宝玉。宝玉:“……那块‘祖母绿’是我今儿一早带在身上,路上丢了。”
宋妈舒了口气:“嗳哟我的妈呀,吓死我了。”
麝月:“不对吧二爷,今儿后晌我收拾妆台还……”宝玉急忙打断:“你知道什么!”转身对众人:“都回去吧。”
众人离去。宝玉看着袭人、麝月埋怨道:“什么劳什骨子,也值得这么兴师动众的!还嫌这阵子撵出去的人少怎么的?
万一要是闹开了,又得饶上几个遭殃的。这事儿……都别再提了。”
麝月张嘴还要说什么,袭人忙给她使了个眼色。

23、王夫人正房内(清晨)\par
曈昽日影,斜透帘栊。
文案上摊放着一本本《四书集注》。宝玉没精打彩地站着。
王夫人:“一大清早找你来,有正经事跟你说,你好好听着。”
宝玉:“我听着呢。”王夫人:“老爷说话就要赴任了,别指望老爷一走你就放了野马了。
送老爷走了以后,你每天早晚过我这里来。”
宝玉不解地看着王夫人。王夫人指着文案:“把这些个书都给我背会讲明了。宝玉皱起眉头,一声不响。
王夫人:“我的儿,你也给我争口气!眼看着一天大似一天的了,不挣上个前程可怎么好!”
宝玉心不在焉地看着文案。王夫人轻轻叹了口气。

24、怡红院\par
院子里笑语喧哗。宝玉进院门。麝月一眼瞥见,高声嚷着:“二爷回来了!”
众丫鬟一窝蜂笑着拥上来。宝玉诧异地:“什么事儿这么高兴?”
麝月把背在后面的手突然举到宝玉面前:“你看!这是什么?”
麝月的手里攥着一块罗帕。宝玉看了看麝月,抬起手来,轻轻打开罗帕。
“祖母绿”在阳光下熠熠闪光。宝玉:“怎么……”
袭人在廊下笑吟吟地站着。麝月:“二爷早起过去念书,我给二爷收拾床,怎么一掀枕头,一眼就看见这个。”
宋妈咧嘴笑着:“也不知是哪个小蹄子使的坏,花姑娘审了一早上也没人认帐!”
一个小丫鬟噙着泪花,笑着:“管他是谁吧。横竖东西没丢,大家都干净了。”
宋妈嘿嘿傻笑。小丫鬟打趣地:“宋妈妈这会子又笑了,昨儿晚上怎么脸拉得这么长?连一根皱纹儿都找不着了!”
众哄笑。宋妈佯嗔着一扬巴掌:“我把你这个小蹄子……”小丫鬟格格笑着把头一缩。

25、怡红院门外\par
黛玉、紫鹃含笑听着院子里的笑语。紫鹃悄悄捅了一下黛玉:“我说不会出什么事儿吧?姑娘白白地一夜没睡,
干陪了这么多眼泪。”黛玉忍俊不禁地瞪了紫鹃一眼:“……走吧。”紫鹃做了个鬼脸儿,随黛玉离去。

26、院内\par
麝月托着罗帕,“我第一眼看见这个,还寻思着是二爷的玉呢!”众笑。
袭人笑着朝宝玉胸前瞥了一眼,愣了一下,渐渐收起了笑容。
麝月指着宝玉,笑着对众人道:“咱们这位好心的爷还往自己身上揽呢,硬说丢在外头了……”众笑。
宝玉冷笑一声,指着众人:“我说你们这些人哪,‘天下本无事,庸人自扰之’。有这会子笑的,昨儿晚上为什么吓成那样?
什么好东西!丢了就丢了,得了就得了,本来也……”
众人“轰”地一下围上来,七嘴八舌地:“哎哎,不对不对!”
“话可不是这样说……”“前儿邢姑娘刚丢了东西,咱们这里又……”
袭人分开众人,急步走到宝玉面前。小丫鬟笑着:“哎哎,都别乱嚷嚷了!听花大姐姐说!”
袭人两眼直盯盯地看着宝玉胸前。宝玉莫名其妙地低头看看,复又抬头看看袭人。
袭人轻声:“二爷,你的玉呢?”宝玉笑笑:“看看,又来了。我说你们是让人偷怕了还是……”
袭人:“二爷!”众人的目光“唰”地一下集中在宝玉胸前。宝玉下意识地用手摸了摸胸前:“嗯?”
所有的人都屏住了呼吸,眼睛死死盯住宝玉。一种不祥的预感笼罩着院子。
宝玉歪着脑袋想了想,“嗤”地一下笑起来:“是了,都是太太逼着念书,早起慌慌张张地过去,玉也忘了带了。”
“啊?!”麝月急忙转身朝宝玉卧房内奔去。众人“唿”地一下跟着跑向房内。

27、宝玉卧房内\par
麝月一把掀开枕头。枕下空无一物。
“啊?!”,众人不约而同地倒吸了一口凉气,呆住了。袭人环视着众人,强笑笑:“谁把这件东西给藏起来了?
……玩儿是玩儿,笑是笑,别真弄丢了,那可就大家都活不成了。”
众人面面厮觑,一声不响。袭人额上沁出了冷汗。
麝月“哇“地一声哭起来,一屁股坐在地下。
袭人微微颤抖着:”别……别哭,快……快找一找……快找!”
麝月急忙从地下爬起来,擦擦眼泪,抓起忱头上下抖着。
袭人朝着目瞪口呆的众人:“还……还愣着干什么?快找哇!”
众人如梦初醒,“哗”地一下子散开,七手八脚,翻箱倒柜,乱成一团。
麝月拎起被子抖着。一丫鬟爬进床下,两手在地上乱摸。
一丫鬟踩着凳子,拉开帐顶,朝里面看。
一丫鬟挪开茶几。一丫鬟拉开藤屉子春凳。
一丫鬟弯着腰顺墙边找过去……宋妈扎煞着两手,嘴里叨叨咕咕地在房内来回转圈儿……
袭人拉开妆台抽屉……麝月掀起床单……
抽屉一个个拉开……被褥一层层揭起……
满屋子哭声、拍打声、推拉家俱声……宝玉站在门口摇手阻止着众人,嘴里说着什么……
宝玉叹了口气,走出房门。
……
房内一片狼籍。所有的东西都离开了原位。袭人两只手撑着妆台,额上沁满了冷汗。麝月满面泪痕,失神地坐在空床上。
宋妈抱着一个枕头坐在地下。五、六个丫鬟倚墙呆立。袭人用手背拭了拭额上的冷汗……
(闪回)袭人从宝玉颈上摘下玉来,用罗帕包好,塞入枕下。
(闪回)麝月手里托着打开的罗帕,“祖母绿”在阳光下熠熠闪光。
(闪回)被拉开的妆台抽屉。(闪回)满院子的人在赏花。(闪回)麝月一把掀开枕头,枕下空无一物。
(闪回)满院子的人。(闪问)抽屉。(闪回)满院子的人。(闪回)空无一物的枕下。(闪回)满院子的人。
袭人猛地转过身来:“昨儿谁进过屋子?”众人吃了一惊,愣愣地看着袭人。
袭人一把抓过一个小丫鬟:“你看见设有?”小丫鬟怔怔地:“看见什么?”
“谁进过屋子!”“什么时候?”“看花儿的时候!”
小丫鬟害怕地:“没……没有。”袭人:“你干什么呢?”
小丫鬟:“我给老太太端着茶呢。”袭人对另一丫鬟:“你呢?”
另一丫鬟赶紧摇摇头:“我给太太、奶奶们搬凳子呢。”又一丫鬟连忙过来:“我帮宋妈烧茶呢,不信你问……”
又一丫鬟:“玉钏儿姐姐找我说话呢……”袭人走到缩在墙角的一个小丫鬟面前:“你?”
小丫鬟“哇”地一声哭了。“别哭!”袭人直直盯住小丫头:“你干什么呢?”
小丫鬟赶紧擦了擦眼泪抽答着:“我一直跟着姐姐呢,姐姐怎么忘了?”
袭人转过身,叹了口气。宋妈从地上爬起来:“花姑娘,昨儿来的人多,说不定让谁拣了去了。”
袭人摇摇头:“进来的谁不知道这是个命根子?谁敢拣了去呢?再说,又是塞在枕头底下的。”
宋妈:“再不就是谁故意摸走了,吓唬着姑娘们玩儿?”袭人抬头看了看麝月。
麝月慢慢站起来:“……要不,就去问问昨儿进来的人?”袭人点点头。
宋妈:“那就快着吧!”麝月慌着要往外走。袭人:“慢着!”
麝月不解地看着袭人。袭人:“好歹先别声张,先在园子里各处问问,若是有姐妹们吓我们玩儿呢,
你们给她磕头要了回来,若是小丫鬟偷了去,问出来也不回上头,不论把什么送她换了出来都行。”
麝月点点头,急忙走出房门。众丫鬟看了看袭人,也随后慌着跟了出去。袭人追出房门:“哎哎,麝月!”

28、院内\par
麝月停步回头。袭人:“先别到府里问去,找不成再惹出些风波来,就更不好了。”
麝月答应着,带着几个丫鬟走出院门。院内树草凋零。唯有“女儿棠”新绿扶红,风姿绰约。
袭人立在廊下,呆呆地看着“女儿棠”。
(闪回)平儿悄声:“奶奶说,这花儿开得奇怪……”“女儿棠”嫩绿的枝叶。
(画外音)平儿悄声:“奶奶说,这花儿开得奇怪……”“女儿棠”鲜红的花瓣儿。袭人转身跑入房内。
院子里凄冷荒落。袭人手里拿着一块红绸子,急步走出房门。“女儿棠“花红得象要滴出血来。
袭人面色惨白,颤抖着,把红绸子挂在“女儿棠”上……

29、潇湘馆\par
疏竹影里,紫鹃用一个小磁盅喂着架上的鹦鹉。宝玉进院门。
紫鹃回头:“宝二爷?”宝玉询问地看看紫鹃。紫鹃笑嘻嘻地点点头。
宝玉朝房内走去。

30、黛玉房内\par
黛玉倚案行书。宝玉走到黛玉身旁,“怎么妹妹也想着‘蟾官折桂’了,这会子这么用功?”
黛玉抬头来,抿嘴一笑:“不敢。比不得宝二爷,都念到第三本《诗经》了,什么‘呦呦鹿鸣,荷叶浮萍’。”
宝玉“扑嗤”一下笑了。黛玉笑着咳嗽起来。
宝玉忙收住笑:“哎呀……”黛玉摇摇手,拿起罗帕擦擦眼睛,“不妨事的……”
宝玉俯身看了看案上的书:“哎?这是什么书?我怎么一个字也不认得?”
黛玉微微一笑。宝玉:“妹妹近日愈发大进了,看起天书来了!”
黛玉:“好个念书的人,连个琴谱都没见过。”
宝玉:“噢,这就是琴谱?”黛玉冷笑一声:“亏你房里还挂着一张琴呢!”
宝玉:“那不过是挂着摆摆样子,老爷书房里还挂着几张呢。……怎么你还藏着这个本事?”
黛玉:“我小时候在扬州学过一点儿,这么多年不弄,快忘光了。前儿紫鹃叨噔大书架子,翻出来一套琴谱,今儿闲翻翻。”
宝玉连忙从墙上琴槽子里抠出琴来。黛玉:“哎哎,干什么?”
宝玉把琴安放在长条几上,走到黛玉面前拱手一揖:“敬聆松风。”
黛玉:“别酸文假醋的了,连个琴谱都看不懂,我可不愿意对……”
宝玉笑嘻嘻地:“对牛弹琴?不然不然。昔日高山流水,得遇知音,那钟子期就一定能看得懂琴谱么?”
黛玉怔怔地看着宝玉:“(心声)高山流水……”
琴声起……一泓泻玉绕阶缘屋盘旋竹下而出。架上鹦鹉仿佛在侧耳聆听。
纤纤素手理结丝桐。宝玉深情的目光。一丝不安从黛玉的眸子里闪过。宝玉空荡荡的胸前。
(幻觉)挂着玉的胸前。宝玉空荡荡的胸前。琴声渐促,发出不谐和音。黛玉疑问的目光。
宝玉下意识地用手摸了摸胸前。琴声急促。

31、怡红院\par
琴声急促。一个丫鬟奔进院门。袭人急步迎上。丫鬟擦擦汗,摇头。
又一个丫鬟奔进院门。袭人迎上。丫鬟摇头。
又一个丫鬟奔进院门。袭人迎上。摇头。
袭人急步走出院门。麝月气喘吁吁地站住,颓丧地摇头。袭人一下坐在门槛上。琴声戛然而止。

32、潇湘馆\par
宝玉惊异地看着崩断的琴弦。

33、怡红院\par
李纨、探春、袭人并一群丫鬟、婆子们围站在院子里。
探春:“赶快把园子几个门都关了,再打发人各处找找,谁找出来,重重地赏银!”
袭人:“……三姑娘,我有一句话不知道当说不当说。”
探春:“都到这会子了,还拘什么?说吧。”
袭人:“这事儿难保没有人使坏,昨儿进来的人当中会不会……”
李纨:“有谁跟宝兄弟不对呢?”
袭人看着探春:“……”探春略一沉吟:“……环儿?……难说。……侍书!”
探春的丫鬟侍书连忙走上来:“姑娘。”
探春:“你悄悄地去找平姑娘,告诉她丢玉的事儿。就说是我说的,让她想个法子把环儿三爷带到这儿来。可千万别声张。”
侍书答应了一声,转飞就走差点和匆匆进门的麝月撞个满怀。麝月:“二爷回来了。”宝玉进门。
袭人迎上来刚要说什么。宝玉埋怨地:“我说你们这些人哪,一点子事也要闹得满世界都不安!什么劳什骨子,丢了正好,我早就不想要它了。”
探春:“你说得倒轻巧,可怎么向老太太、老爷、太太交待?”

31、王夫人正房内\par
玉钏儿用一个小漆茶盘托着两盅茶进门。贾政、王夫人对面坐在临窗大炕上。
王夫人:“……老爷明儿就要上路了,有几件事还要跟老爷商量商量。”
贾政接过茶盅:“说吧。”王夫人接过茶盅。
玉钏儿退下。王夫人:“头一件,宝玉和园子里的姐妹们都大了,
总在一处耳鬓厮磨的,难免惹出些闲话来。我想……,过了年就先把宝玉挪出来。”
贾政微微颔首:“嗯。”王夫人:“第二件,宝玉现已经过了舞象之年,
还没有进学,实在让人焦心。求老爷临走前再教导教导他。逼着他收收心。
要是有一天,他能象珠儿活着的时候那么知道读书上进,我也就……”
王夫人哽咽着,用绢子擦擦眼睛。
贾政摇了摇头,叹了口气:“知其子者莫若父。宝玉这孩子,若论聪明才智,比环儿强十倍,
若论八股举业一道,怕不是这块材料。……当年,太爷原打算让我从科甲出身的。
谁知太爷临终时遗本一上,圣上体恤先臣,额外恩赐了官儿。
说起来,贾门还没有过一个从举业上发迹的,这大概也是贾门的定数。”
王夫人:“可是……”贾政:“你看他昨儿在席上做的那两首诗,空灵娟逸,清俊通脱。
细评起来,也还不算十分玷辱了祖宗。”王夫人叹了口气。
贾政呷了口茶:“还有呢?”
王夫人:“还有一件,宝玉的亲事也该议了,虽然有老太太做主,
可是……还想听听老爷的意思……”
玉钏儿匆匆走进房内,神色慌张。贾政:“嗯?”
玉钏儿:“老爷,太太。府里头风言风语,说园子里丢了东西。”
王夫人:“哦?”玉钏儿:“听说……是宝二爷的玉……”贾政:“什么?”
玉钏儿:“说是……宝二爷的玉丢了。”王夫人“唿”地一下站起来。

35、怡红院\par
一个小丫鬟跑进院门,神色紧张地:“来了,来了!”
探春:“大家都装作没事儿一样!……二哥哥,你先到后边儿呆会儿去。”
众人匆匆散开。贾环、平儿、侍书进院门。
贾环:“三姐姐。”探春:“哟,环儿兄弟来了,快屋里坐。”
贾环笑嘻嘻地:“三姐姐叫我来,果真是让我入你们的诗社?”
探春瞥了平儿一眼。平儿暗暗点头。探春:“哦,进屋说吧。”
贾环面带得意之色走入房内。平儿急忙拉着探春耳语。

36、房内\par
贾环坐在文案旁的一张圈椅上。袭人捧上茶来:“三爷。”
贾环忙起接茶:“劳动姐姐了。”探春和平儿进门。
探春笑嘻嘻地:“听二哥哥说,你近来文思大进,写了不少好诗呢。”
贾环忙欠身:“哪里,二哥哥谬奖了。”
探春:“听说,你们兄弟这阵子越发和睦了。”贾环:“……哦,这是自然的。”探春:“还常常一块儿玩笑?”
贾环:“……哦,是的是的。”探春:“那……,我有个事儿问问你。”贾环:“什么事?”
探春:“……,昨儿赏花的时候,二哥哥的玉不见了,你瞧见了没有?”
贾环“唿”地站起来,瞪着眼睛看着平儿:“噢,原来叫我来是为这个!”
探春:“你到底见没见呢?”贾环急得紫涨了脸:“人家丢了东西,怎么三姐姐偏查问我,我是犯过案的贼么?”
平儿急忙解释:“不是这么说,怕三爷要拿了去吓唬她们玩儿,所以顺便问问。”
贾环:“他的玉在他身上,要问问他去,我不知道!……得了什么不来问我,丢了什么就来问我!”
说着,起身就走。平儿:“哎哎,三爷别走……”贾环一径出门去了。
宝玉从内室急步走出:“环儿一去,必定嚷得满世界里都知道了,这可不是闹事了吗?”
袭人急得直哭:“要是上头知道了,我们这些人就要粉身碎骨了!”
平儿:“我看,这事儿要瞒也是瞒不住的,快商量商量怎么回上头吧。”
宝玉:“你们也不用商量,就说我砸了就完了。”
平儿:“我的爷,好轻巧话儿!上头要问怎么砸的呢?那砸破的碎碴儿呢?”
宝玉:“不然,就说我前儿出门丢了。”
李纨:“嗯,这大概还混得过去。”探春:“不行不行,既是前儿丢的,为什么当天不回?”
宝玉:“……”众人面面厮觑。

37、荣国府前角门外\par
面带喜色的仆人们进进出出地忙碌着。
男仆李十儿一只脚蹬着大板凳,比比划划地说着什么。板凳上坐着四、五个男仆,聚精会神地听着。
李十儿:“……京官儿有几个发了财的?要发财还得放外任!”
男仆们点头:“那是那是。”李十儿:“咱们这京城里头,都知道孙府底子薄吧?”
男仆们点头。李十儿:“你看人家一趟盐务回来!嗯?还别说主子,就说奴才吧!
我那个表兄弟儿,不过跟着主子看个门儿,回来的时候,箱子包袱不算,单是白花花的银子……”
李十儿伸出一只巴掌在众人眼前晃了晃:“就捞了这个数!”众人唏嘘惊叹不已。

38、大观园中路上\par
王夫人带着玉钏儿急步走来。

39、荣国府前角门外\par
一个华冠丽服的男仆:“李十儿,又吹什么呢?”
李十儿回头,眼睛一亮:“嘿!瞧瞧!里外三新了!”男仆咧嘴一笑:“跟着老爷赴任,总不能太寒酸了。”
李十儿拎起男仆的衣襟,一字一顿地:“有钱做新衣裳,那欠我的钱,也该有的还了吧?”
男仆正色道:“啧!怎么哏皮子这么浅!我就没有个发财的时候了?”
一乘暖轿远远过来。李十儿抬眼瞥见。急忙迎上去。轿内端坐着面带笑容的贾琏。
李十儿跟在轿旁,恭敬地:“二爷!”轿帘掀起一条缝。
贾琏:“李十儿?”李十儿谄媚地:“二爷,我那点儿孝敬二爷的意思……,二爷……过目了吧?”
贾琏微微一笑。李十儿:“我那个事儿……?”贾琏:“回过老爷了。”
李十儿急切地:“老爷的意思是……?”贾琏:“跟着去,可得好好办事!”
李十儿惊喜地:“谢二爷!”角门外男仆们垂手侍立。暧轿进门。

40、赵姨娘房门外过道上\par
贾环站着抹眼泪。赵姨娘往房内让着邢夫人:“……求大太太给评评这个理儿,他们丢了东西自己不找,
怎么叫人背地里拷问环儿?”邢夫人停步:“谁把环儿找去拷问的?”赵姨娘:“平儿!”
邢大人诧异地:“不能吧!哪有奴才拷问主子的道理?”
赵姨娘:“人家是个有脸的奴才,环儿是个没脸的主子。”
贾环:“她不过是仗着琏二嫂子的势力!”邢夫人冷笑一声:”什么有脸的没脸的?凭她仗着谁的势力,给她脸才有脸!
你带着环儿找她去,啐在她脸上!”赵姨娘:“我……”
邢夫人:“怕什么?你去我随后就去,倒要看看谁是有脸的!”

41、怡红院·房内\par
啜泣声。素云进房门:“太太来了!”李纨连忙迎出。

42、院内\par
王夫人带着玉钏儿进院门。李纨:“太太……”

43、房内\par
王夫人进门。袭人、麝月并众丫鬟满面泪痕,齐齐跪下。
王夫人倒吸了一口凉气:“啊?这么说,是真的?”
宝玉急忙上前:“太太,这事儿不与她们相干。是我前儿出门儿,丢在外头了。”
王夫人:“为什么当时不找?”宝玉:“我让茗烟儿在外头找来着。……我怕她们惊慌,没告诉她们。”
王夫人:“胡说!如今脱换衣服不都是她们服侍的么?大凡出门儿,手巾荷包短了,还要问问,这玉不见了,就不问问么?”
平儿进门:“二奶奶来了!”凤姐姣怯怯地扶着小红进门:“请太太安。”李纨:“你病还没好,又过来干什么?”

44、贾政书房\par
贾政端坐闭目养神。贾琏兴冲冲地进门:“老爷。”贾政睁开眼睛。
贾琏:“雨村说,明儿一早去长亭送老爷。”贾政仿佛没有听见,忧心忡忡地看着贾琏。贾琏渐渐收了笑容。

45、怡红院·房内\par
袭人等跪在地下无声地哭泣着。凤姐:“……咱们家人多手杂,这阵子又到处乱糟糟的。
依我说,先别吵嚷。太太想,偷玉的人明知查出来就是个死,他若是一着急,先把玉给毁了可怎么处呢?
不如就说宝玉不喜欢,搁丢了,也没什么要紧。暗地里派人去各处察访,想法子哄骗出来。
那时候玉也可得,罪名也好定。只是有一条,千万别叫老太太和老爷知道。”
王夫人叹了口气:“老爷已经知追了。”凤姐:“啊?……”门外传来贾环的哭声。
众人朝门口看去。赵姨娘的声音:“我今儿索性就把你交给那起洑上水的,该杀该剐,随人家吧!”
探春皱着眉头叹了口气。赵姨娘带着贾环进门,看见王夫人和凤姐,一下愣住了。
王夫人怒容满面。凤姐瞥了一眼王夫人,厉声对赵姨娘:“这是什么时候?容得你在这里撒泼!
告诉你,这回丢的可是个命根子!凭你是谁,凡是昨儿来过这里的都得问问,怎么就不能问问环儿?下一个还要问你呢!”
赵姨娘害怕地往后退退。“你还想问谁呢?”邢夫人悻悻地出现在房门口。
凤姐一惊:“哦,太太……”邢夫人:“凡是来过的都得问问?昨儿我也来了,你还要问我不成?
昨儿老太太也来了,你还想问老太太不成!”
凤姐勉强陪笑:“太太,我不是……”邢夫人:“你没来,你干净!别人就都是贼么?”众人惊愕地看着邢夫人。
邢夫人:“我看你也太张狂了些!”凤姐的泪水“唰”地一下涌满了眼眶。王夫人嘴唇哆嗦着说不出话来。
李纨忙走过来:“大太太先诸请坐,有话慢慢说。”平儿也急忙陪着笑脸:“太太别生气,奶奶怎么敢指着太太说呢?
本来……”邢大人眼一瞪:“主子们说话呢,你插什么嘴?这是什么规矩!”
平儿捂着脸退下去。赵姨娘暗暗得意。

46、院内\par
黛玉不知什么时候来了,扶着紫鹃站在房门口廊下。
房内传出邢夫人的声音:“既然说了,不论主子奴才都得问,那好,索性就关了大门,一处一处地搜!”
黛玉惆怅地看了紫鹃一眼。

47、房内\par
李纨:“大太太别多心,这事儿……”邢夫人斜睨着凤姐,冷笑一声:“她倒撇清了,昨儿没来!”

48、房门外廊下\par
黛玉下意识地抓紧了紫鹃的手臂,紧张地:“说谁呢?”紫鹃:“姑娘怎么了?昨儿没来的又不是一个两个。”

49、房内\par
凤姐拷酉啜泣。宝玉着急地:“我说实话吧,那劳什骨子让我给砸了。”
邢夫人:“哼!实话不实话的吧。反正,那玉就是不丢,早晚也是个砸!”

50、房门外廊下\par
黛玉打了个哆嗦, 一下把嘴唇咬住。

51、房内\par
邢夫人冷冷地:“砸了也不是一回两回了!”

52、廊下\par
黛玉脸色惨白。

53、房内\par
王夫人泪流满面,浑身颤抖。
(幻觉)摔砸通灵玉声、杂沓的脚步声、惊叫声、哭泣声……声音嘎然而止。众人惊呼:“太太!……” 

54、户外\par
阴霾沉沉,雪花飘飘。

55、贾母院外穿堂(晨)\par
雪霁日出。几个仆妇在穿堂外执帚扫雪。
凤姐扶着小红自穿堂走出。扫帚划过,仿佛有什么东西在雪地上一闪。
凤姐一愣,走近一步仔细瞧了瞧。通灵玉在旭日的照射下闪着幽光!
凤姐失声叫了起来:“玉!玉!”众仆妇闻声急忙围上来。凤姐俯身抓起通灵玉,兴奋地托着:
“是宝兄弟的玉!找着啦——”

56、潇湘馆\par
雪雁慌慌张张地跑进院门:“紫鹃姐姐,紫鹃姐姐!”
紫鹃急步自房内迎出:“轻点儿!姑娘又说了好一阵子胡话,才安静了。……宝二爷呢?怎么?没找来?”
雪雁:“宝二爷搬走了!”紫鹃:“什么?”
雪雁:“宝二爷昨儿晚上就被太太逼着搬出去了!”紫鹃:“啊?!”

57、黛玉卧房内\par
黛玉昏昏沉沉地躺在床上。长条几上,崩断的琴弦在微微颤动……