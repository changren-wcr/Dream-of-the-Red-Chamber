\chapter{金兰契互剖金兰语 \quad 风雨夕闷制风雨词}
\zhu{金兰契:比喻情投意合的知心朋友。
金,喻坚;兰,喻香;契,契合,意气相合。
《易·系辞上》:“	二人同心,其利断金;同心之言,其臭(嗅)如兰。
”}
\par
\qi{富贵荣华春暖,梦破黄\sout{粮}[粱]愁晚。
金玉作楼台,也是戏场妆点。
\zhu{
不要被表面的“富贵荣华”所迷惑,那不过是“戏场妆点”,很快就要“梦破黄粱”了。
本回赖家摆酒摆戏三日庆贺赖嬷嬷的小子选任州官,故词中以“戏场妆点”作比。
}
莫缓,莫缓,遗却灵光不远。
\zhu{
最后两句似乎是说黛玉离“眼泪还债”而死的日子不远了。
}
}\par
话说凤姐儿正抚恤平儿,忽见众姊妹进来,忙让坐了,平儿斟上茶来。
凤姐儿笑道:“今儿来的这么齐,倒像下帖子请了来的。
”探春笑道:“我们有两件事:一件是我的,一件是四妹妹的,还夹着老太太的话。
”凤姐儿笑道:“有什么事,这么要紧?”探春笑道:“我们起了个诗社,头一社就不齐全,众人脸软,所以就乱了。
我想必得你去作个监社御史,铁面无私才好。
再四妹妹为画园子,用的东西这般那般不全,回了老太太,老太太说:‘只怕后头楼底下还有当年剩下的,找一找,若有呢拿出来,若没有,叫人买去。
’”凤姐笑道:“我又不会作什么湿的干的,要我吃东西去不成?”探春道:“你虽不会作,也不要你作。
你只监察着我们里头有偷安怠惰的,该怎么样罚他就是了。
”凤姐儿笑道:“你们别哄我,我猜着了,那里是请我作监社御史!分明是叫我作个进钱的铜商。
\zhu{进钱的铜商:进钱:供给钱。
进,这里是进奉的意思。
铜商:西汉邓通受宠于汉文帝,得赐蜀郡严道铜山,可自行铸钱,成为西汉大富商。
见《汉书·邓通传》。
故后以铜商代指富商。
}你们弄什么社,必是要轮流作东道的。
你们的月钱不够花了,想出这个法子来拗了我去,
\zhu{拗[niù]:向相反或不顺的方向扭转。}
好和我要钱。
可是这个主意?”一席话说的众人都笑起来了。
李纨笑道:“真真你是个水晶心肝玻璃人。
”凤姐儿笑道:“亏你是个大嫂子呢!把姑娘们原交给你带着念书学规矩针线的,他们不好,你要劝。
这会子他们起诗社,能用几个钱,你就不管了?老太太、太太罢了,原是老封君。
\zhu{封君:古代受封邑的贵族的通称。
《汉书·食货志下》引颜师古注:“封君,受封邑者,谓公主及列侯之属也。
”自晋以后,历代均有封典之制,即皇帝按照官员的等级分别给他本人及其妻室、父祖以荣誉封赠,凡受到这种封典的都叫“封君”。
}你一个月十两银子的月钱,比我们多两倍银子。
老太太、太太还说你寡妇失业的,可怜,不够用,又有个小子,足的又添了十两,和老太太、太太平等。
又给你园子地,各人取租子。
年终分年例,你又是上上分儿。
你娘儿们,主子奴才共总没十个人,吃的穿的仍旧是官中的。
一年通共算起来,也有四五百银子。
这会子你就每年拿出一二百两银子来陪他们顽顽,能几年的限?\ping{李纨确实有些吝啬。
}他们各人出了阁,难道还要你赔不成?这会子你怕花钱,调唆他们来闹我,我乐得去吃一个河涸海干,我还通不知道呢!”\zhu{通:全部,整个。
这里形容程度之深。
}\par
李纨笑道:“你们听听,我说了一句,他就疯了,说了两车的无赖泥腿市俗专会打细算盘、分斤拨两的话出来。
\geng{心直口拙之人急了,恨不得将万句话来并成一句,说死那人,毕肖!}这东西亏他托生在诗书大宦名门之家做小姐,出了嫁又是这样,他还是这么着;若是生在贫寒小户人家,作个小子,还不知怎么下作贫嘴恶舌的呢!天下人都被你算计了去!昨儿还打平儿呢,亏你伸的出手来!那黄汤难道灌丧了狗肚子里去了?\zhu{黄汤:指酒(骂人喝酒时说)。
}气的我只要给平儿打抱不平儿。
忖夺了半日,
\zhu{忖夺:犹忖度。}
好容易‘狗长尾巴尖儿’的好日子,\zhu{“狗长尾巴尖儿”的好日子:代指生日。
俗传小狗在胎里,一到尾巴长足便生下来。
这是对别人生日的玩笑话。
}又怕老太太心里不受用,因此没来,究竟气还未平。
你今儿又招我来了。
给平儿拾鞋也不要,你们两个只该换一个过子才是。
”\ping{李纨顾左右而言它,倒是没就自己的财产辩驳,所以凤姐算的应该是很精准。
}说的众人都笑了。
凤姐儿忙笑道:“竟不是为诗为画来找我这脸子,竟是为平儿来报仇的。
竟不承望平儿有你这一位仗腰子的人。
早知道,便有鬼拉着我的手打他,我也不打了。
平姑娘,过来!我当着大奶奶姑娘们替你赔个不是,担待我酒后无德罢。
”\zhu{酒后无德:醉后糊涂,耍酒疯,德行不好。
}说着,众人又都笑起来了。
李纨笑问平儿道:“如何?我说必定要给你争争气才罢。
”平儿笑道:“虽如此,奶奶们取笑,我禁不起。
”李纨道:“什么禁不起,有我呢。
快拿了钥匙,叫你主子开了楼房找东西去。
”\par
凤姐儿笑道:“好嫂子,你且同他们回园子里去。
才要把这米帐合算一算,那边大太太又打发人来叫,又不知有什么话说,须得过去走一趟。
还有年下你们添补的衣服,还没打点给他们做去。
”李纨笑道:“这些事情我都不管,你只把我的事完了我好歇着去,省得这些姑娘小姐闹我。
”凤姐忙笑道:“好嫂子,赏我一点空儿。
你是最疼我的,怎么今儿为平儿就不疼我了?往常你还劝我说,事情虽多,也该保养身子,捡点着偷空儿歇歇,你今儿反倒逼我的命了。
况且误了别人的年下衣裳无碍,他姊妹们的若误了,却是你的责任,老太太岂不怪你不管闲事,\ping{不管闲事:不理会、不干涉和自己无关的事情。这里是说李纨只关心自己的事情,不管众姊妹的事情。
}这一句现成的话也不说?我宁可自己落不是,岂敢带累你呢。
”李纨笑道:“你们听听,说的好不好?把他会说话的!我且问你:这诗社你到底管不管?”凤姐儿笑道:“这是什么话,我不入社花几个钱,不成了大观园的反叛了,还想在这里吃饭不成?明儿一早就到任,下马拜了印,先放下五十两银子给你们慢慢作会社东道。
过后几天,我又不作诗作文,只不过是个俗人罢了。
‘监察’也罢,不‘监察’也罢,有了钱了,你们还撵出我来!”说的众人又都笑起来。
凤姐儿道:“过会子我开了楼房,凡有这些东西都叫人搬出来你们看,若使得,留着使,若少什么,照你们单子,我叫人替你们买去就是了。
画绢我就裁出来。
那图样没有在太太跟前,还在那边珍大爷那里呢。
说给你们,别碰钉子去。
我打发人取了来,一并叫人连绢交给相公们矾去。
如何?”李纨点首笑道:“这难为你,果然这样还罢了。
既如此,咱们家去罢,等着他不送了去再来闹他。
”说着,便带了他姊妹就走。
凤姐儿道:“这些事再没两个人,都是宝玉生出来的。
”李纨听了,忙回身笑道:“正是为宝玉来,反忘了他。
头一社是他误了。
我们脸软,你说该怎么罚他?”凤姐想了一想,说道:“没有别的法子,只叫他把你们各人屋子里的地罚他扫一遍才好。
”众人都笑道:“这话不差。
”\par
说着才要回去,只见一个小丫头扶了赖嬷嬷进来。
凤姐儿等忙站起来,笑道:“大娘坐。
”又都向他道喜。
赖嬷嬷向炕沿上坐了,笑道:“我也喜,主子们也喜。
若不是主子们的恩典,我们这喜从何来?昨儿奶奶又打发彩哥儿赏东西,
\zhu{彩哥儿:凤姐的童仆彩明。}
我孙子在门上朝上磕了头了。
”李纨笑道:“多早晚上任去?”赖嬷嬷叹道:“我那里管他们,由他们去罢!前儿在家里给我磕头,我没好话,我说:‘哥哥儿,你别说你是官儿了,横行霸道的!你今年活了三十岁,虽然是人家的奴才,一落娘胎胞,主子恩典,放你出来,上托着主子的洪福,下托着你老子娘,也是公子哥儿似的读书认字,也是丫头、老婆、奶子捧凤凰似的,长了这么大。
你那里知道那“奴才”两字是怎么写的!只知道享福,也不知道你爷爷和你老子受的那苦恼,熬了两三辈子,好容易挣出你这么个东西来。
从小儿三灾八难,花的银子也照样打出你这么个银人儿来了。
到二十岁上,又蒙主子的恩典,许你捐个前程在身上。
你看那正根正苗的忍饥挨饿的要多少?你一个奴才秧子,仔细折了福!如今乐了十年,不知怎么弄神弄鬼的,求了主子,又选了出来。
州县官儿虽小,事情却大,为那一州的州官,就是那一方的父母。
你不安分守己,尽忠报国,孝敬主子,只怕天也不容你。
’”
\ping{“尽忠报国”是虚,“孝敬主子”是实。}
李纨凤姐儿都笑道:“你也多虑。
我们看他也就好了。
先那几年还进来了两次,这有好几年没来了,年下生日,只见他的名字就罢了。
前儿给老太太、太太磕头来,在老太太那院里,见他又穿着新官的服色,倒发的威武了,比先时也胖了。
他这一得了官,正该你乐呢,反倒愁起这些来!他不好,还有他父亲呢,你只受用你的就完了。
闲了坐个轿子进来,和老太太斗一日牌,说一天话儿,谁好意思的委屈了你。
家去一般也是楼房厦厅,谁不敬你,自然也是老封君似的了。
”\par
平儿斟上茶来,赖嬷嬷忙站起来接了,笑道:“姑娘不管叫那个孩子倒来罢了,又折受我。
”
\zhu{
折受:旧谓享受非份而折福叫“折受”。
这里用作谦词,是无福承受、于心不安的意思。
}
说着,一面吃茶,一面又道:“奶奶不知道。
这些小孩子们全要管的严。
饶这么严,\zhu{饶:即使,尽管,表示让步关系。
}他们还偷空儿闹个乱子来叫大人操心。
知道的说小孩子们淘气;不知道的,人家就说仗着财势欺人,连主子名声也不好。
恨的我没法儿,常把他老子叫来骂一顿,才好些。
”因又指宝玉道:“不怕你嫌我,如今老爷不过这么管你一管,老太太护在头里。
当日老爷小时挨你爷爷的打,谁没看见的。
老爷小时,何曾像你这么天不怕地不怕的了。
还有那大老爷,虽然淘气,也没像你这扎窝子的样儿,\zhu{扎窝子:本指飞鸟钻在巢中,不肯出来,这里喻留恋家庭小天地,不思有所作为。
}也是天天打。
\ping{棍棒教育下出了情感表达障碍的贾政,受到父亲的影响,贾政只会认为父亲对于孩子凶恶的态度才是正确的,以至于心中泛起爱意时反而更加凶神恶煞,这在第十七回时展现的很多。
}还有东府里你珍哥儿的爷爷,那才是火上浇油的性子,说声恼了,什么儿子,竟是审贼!\ping{审贼教育把贾敬审成了进士,副作用是这个进士总想升天。
}如今我眼里看着,耳朵里听着,那珍大爷管儿子倒也像当日老祖宗的规矩,只是管的到三不着两的。
\zhu{到三不着两:也作“着三不着两”、“道三不着两”,谓说话或行事轻重失宜,考虑不周,注意这里,顾不到那里。
}
他自己也不管一管自己,这些兄弟侄儿怎么怨的不怕他?你心里明白,喜欢我说,不明白,嘴里不好意思,心里不知怎么骂我呢!”\par
正说着,只见赖大家的来了,接着周瑞家的张材家的都进来回事情。
凤姐儿笑道:“媳妇来接婆婆来了。
”赖大家的笑道:“不是接他老人家,倒是打听打听奶奶姑娘们赏脸不赏脸?”赖嬷嬷听了,笑道:“可是我糊涂了,正经说的话且不说,且说陈谷子烂芝麻的混捣熟。
\zhu{混捣熟:絮絮叨叨地说一些别人听厌了的陈词滥调。
}因为我们小子选了出来,众亲友要给他贺喜,少不得家里摆个酒。
我想,摆一日酒,请这个也不是,请那个也不是。
又想了一想,托主子洪福,想不到的这样荣耀,就倾了家,我也是愿意的。
因此吩咐他老子连摆三日酒:头一日,在我们破花园子里摆几席酒,一台戏,请老太太、太太们、奶奶姑娘们去散一日闷;外头大厅上一台戏,摆几席酒,请老爷们、爷们去增增光;第二日再请亲友;第三日再把我们两府里的伴儿请一请。
热闹三天,也是托着主子的洪福一场,光辉光辉。
”李纨凤姐儿都笑道:“多早晚的日子?我们必去,只怕老太太高兴要去也定不得。
”赖大家的忙道:“择了十四的日子,只看我们奶奶的老脸罢了。
”凤姐笑道:“别人我不知道,我是一定去的。
先说下,我是没有贺礼的,也不知道放赏,吃完了一走,可别笑话。
”赖大家的笑道:“奶奶说那里话?奶奶要赏,赏我们三二万银子就有了。
”
\ping{
赖大家虽然是贾家的佣人,但是相当富有,有自己的宅院,还有私人花园。孙子当官连请三天宴席。
贾府里这种老管家,确实是贾母说的“财主”。
}
\par
赖嬷嬷笑道:“我才去请老太太,老太太也说去,可算我这脸还好。
”说毕又叮咛了一回,方起身要走,因看见周瑞家的,便想起一事来,因说道:“可是还有一句话问奶奶,这周嫂子的儿子犯了什么不是,撵了他不用?”凤姐儿听了,笑道:“正是我要告诉你媳妇,事情多也忘了。
赖嫂子回去说给你老头子,两府里不许收留他小子,叫他各人去罢。
”\zhu{各人:自己。
}赖大家的只得答应着。
周瑞家的忙跪下央求。
赖嬷嬷忙道:“什么事?说给我评评。
”\ping{赖嬷嬷俨然是半个主子像,地位很高。
}凤姐儿道:“前日我生日,里头还没吃酒,他小子先醉了。
老娘那边送了礼来,他不说在外头张罗,他倒坐着骂人,礼也不送进来。
两个女人进来了,他才带着小幺们往里抬。
小幺们倒好,他拿的一盒子倒失了手,撒了一院子馒头。
人去了,打发彩明去说他,他倒骂了彩明一顿。
这样无法无天的忘八羔子,
\zhu{忘八:即“王八”,乌龟或鳖的俗称,骂人的话,指妻子有外遇的男人。}
不撵了作什么!”赖嬷嬷笑道:“我当什么事情,原来为这个。
奶奶听我说:他有不是,打他骂他,使他改过,撵了去断乎使不得。
他又比不得是咱们家的家生子儿,他现是太太的陪房。
奶奶只顾撵了他,太太脸上不好看。
依我说,奶奶教导他几板子,以戒下次,仍旧留着才是。
不看他娘,也看太太。
”凤姐儿听说,便向赖大家的说道:“既这样,打他四十棍,以后不许他吃酒。
”赖大家的答应了。
周瑞家的磕头起来,又要与赖嬷嬷磕头,赖大家的拉着方罢。
然后他三人去了,李纨等也就回园中来。
\par
至晚,果然凤姐命人找了许多旧收的画具出来,送至园中。
宝钗等选了一回,各色东西可用的只有一半,将那一半又开了单子,与凤姐儿去照样置买,不必细说。
\par
一日,外面矾了绢,起了稿子进来。
宝玉每日便在惜春这里帮忙。
\geng{自忙不暇,又加上一“帮”字,可笑可笑。
所谓《春秋》笔法。
\zhu{《春秋》笔法:《春秋》是孔子根据鲁史撰修的编年体史书。
古代学者说它“以一字为褒贬”,含有“微言大义”。
后来就把文笔深隐曲折、意含褒贬叫“春秋笔法”。
}}探春、李纨、迎春、宝钗等也多往那里闲坐,一则观画,二则便于会面。
宝钗因见天气凉爽,夜复渐长,\geng{“复”字妙,补出宝钗每年夜长之事,皆《春秋》字法也。
}遂至母亲房中商议打点些针线来。
日间至贾母处王夫人处省候两次,不免又承色陪坐半时,\zhu{承色:顺承迎合父母长辈以博欢心。
}园中姊妹处也要度时闲话一回,故日间不大得闲,每夜灯下女工必至三更方寝。
\geng{\sout{代}[伏]下\sout{收夕}[后文]。
}\geng{写针线下“商议”二字,直将寡母训女多少温存活现在纸上。
不写阿呆兄,已见阿呆兄终日醉饱优游,怒则吼,喜则跃,家务一概无闻之形景毕露矣。
《春秋》笔法。
}黛玉每岁至春分秋分之后,必犯嗽疾;今秋又遇贾母高兴,多游玩了两次,未免过劳了神,近日又复嗽起来,觉得比往常又重,所以总不出门,只在自己房中将养。
\zhu{将养:保养,调养。
}有时闷了,又盼个姊妹来说些闲话排遣;及至宝钗等来望候他,说不得三五句话又厌烦了。
众人都体谅他病中,且素日形体娇弱,禁不得一些委屈,所以他接待不周,礼数粗忽,也都不苛责。
\par
这日宝钗来望他,因说起这病症来。
宝钗道:“这里走的几个太医虽都还好,只是你吃他们的药总不见效,不如再请一个高明的人来瞧一瞧,治好了岂不好?每年间闹一春一夏,又不老又不小,成什么?不是个常法。
”黛玉道:“不中用。
我知道我这样病是不能好的了。
且别说病,只论好的日子我是怎么形景,就可知了。
”宝钗点头道:“可正是这话。
古人说:‘食谷者生。
’\zhu{“食谷者生”:中医认为食五谷,可以添养精神气血。
}你素日吃的竟不能添养精神气血,也不是好事。
”黛玉叹道:“‘死生有命,富贵在天’,也不是人力可强的。
今年比往年反觉又重了些似的。
”说话之间,已咳嗽了两三次。
宝钗道:“昨儿我看你那药方上,人参肉桂觉得太多了。
虽说益气补神,也不宜太热。
依我说,先以平肝健胃为要,肝火一平,不能克土,\zhu{肝火一平,不能克土:中医理论以五行的生克致化来说明五脏之间的相互关系。
金生水,水生木,木生火,火生土,土生金;金克木,木克土,土克水,水克火,火克金。
肝属木,脾胃属土,木与土是相克关系,木能生火,肝火太旺,要伤及脾土。
肝火一平,使之不能再克伤脾胃,就能和顺地摄取食物的营养。
}胃气无病,饮食就可以养人了。
每日早起拿上等燕窝一两,冰糖五钱,用银铫子熬出粥来,\zhu{铫(音“掉”)子:一种带柄有短嘴的小锅。
}若吃惯了,比药还强,最是滋阴补气的。
”\par
黛玉叹道:“你素日待人,固然是极好的,然我最是个多心的人,只当你心里藏奸。
从前日你说看杂书不好,又劝我那些好话,竟大感激你。
往日竟是我错了,实在误到如今。
细细算来,我母亲去世的早,又无姊妹兄弟,我长了今年十五岁,\geng{黛玉才十五岁,记清。
}竟没一个人像你前日的话教导我。
怨不得云丫头说你好,我往日见他赞你,我还不受用,昨儿我亲自经过,才知道了。
比如若是你说了那个,我再不轻放过你的;你竟不介意,反劝我那些话,可知我竟自误了。
\ping{黛玉逐渐妥协进入成人世界,和宝钗和解并倾诉心事。
黛钗之间旧有的矛盾也就是迅速成长的少年在不同年龄段心态的差异吧。
}若不是从前日看出来,今日这话,再不对你说。
你方才说叫我吃燕窝粥的话,虽然燕窝易得,但只我因身上不好了,每年犯这个病,也没什么要紧的去处。
请大夫,熬药,人参肉桂,已经闹了个天翻地覆,这会子我又兴出新文来熬什么燕窝粥,\zhu{新文:新花样。
}老太太、太太、凤姐姐这三个人便没话说,那些底下的婆子丫头们,未免不嫌我太多事了。
你看这里这些人,因见老太太多疼了宝玉和凤丫头两个,他们尚虎视眈眈,背地里言三语四的,何况于我?况我又不是他们这里正经主子,原是无依无靠投奔了来的,他们已经多嫌着我了。
如今我还不知进退,何苦叫他们咒我?”宝钗道:“这样说,我也是和你一样。
”黛玉道:“你如何比我?你又有母亲,又有哥哥,这里又有买卖地土,家里又仍旧有房有地。
你不过是亲戚的情分,白住了这里,一应大小事情,又不沾他们一文半个,要走就走了。
我是一无所有,吃穿用度,一草一纸,皆是和他们家的姑娘一样,那起小人岂有不多嫌的。
”宝钗笑道:“将来也不过多费得一副嫁妆罢了,如今也愁不到这里。
”\geng{宝钗此一戏,直抵过通部黛玉之戏宝钗矣,又恳切,又真情,又平和,又雅致,又不穿凿,又不牵强。
黛玉因识得宝钗后方吐真情,宝钗亦识得黛玉后方肯戏也。
此是大关节大章法,非细心看不出。
}\geng{细思二人此时好看之极,真是儿女小窗中喁喁也。
\zhu{喁:音“鱼”,模拟小声说话的声音。
}}黛玉听了,不觉红了脸,笑道:“人家才拿你当个正经人,把心里的烦难告诉你听,你反拿我取笑儿。
”宝钗笑道:“虽是取笑儿,却也是真话。
你放心,我在这里一日,我与你消遣一日。
你有什么委屈烦难,只管告诉我,我能解的,自然替你解一日。
我虽有个哥哥,你也是知道的,只有个母亲比你略强些。
咱们也算同病相怜。
你也是个明白人,何必作‘司马牛之叹’?\zhu{司马牛之叹:司马牛是孔子的学生,名耕,字子牛,他曾感叹说:“人皆有兄弟,我独亡(无)。
”见《论语·颜渊》。
后常以此代指没有兄弟。
}\geng{通部众人必从宝钗之评方定,然宝钗亦必从颦儿之评始可,何妙之至!}你才说的也是,多一事不如省一事。
我明日家去和妈妈说了,只怕我们家里还有,与你送几两,每日叫丫头们就熬了,又便宜,又不惊师动众的。
”黛玉忙笑道:“东西事小,难得你多情如此。
”宝钗道:“这有什么放在口里的!只愁我人人跟前失于应候罢了。
只怕你烦了,我且去了。
”黛玉道:“晚上再来和我说句话儿。
”宝钗答应着便去了,不在话下。
\par
这里黛玉喝了两口稀粥,仍歪在床上,不想日未落时天就变了,淅淅沥沥下起雨来。
秋霖脉脉,\zhu{霖:音“林”,长时间连降不停的大雨。
脉脉:这里指细雨连绵的样子。
}阴晴不定,那天渐渐的黄昏,且阴的沉黑,兼着那雨滴竹梢,更觉凄凉。
知宝钗不能来,便在灯下随便拿了一本书,却是《乐府杂稿》,\zhu{《乐府杂稿》:未详。
疑为作者虚拟的书名。
乐府:本为汉武帝所立的官署,专司搜集诗歌配管弦以入乐。
后世因称乐府官署保存的和一些能入乐的诗歌为“乐府”。
}有《秋闺怨》、《别离怨》等词。
黛玉不觉心有所感,亦不禁发于章句,遂成《代别离》一首,\zhu{《代别离》:代:拟作。
这里的《代别离》是拟上文所说的《别离怨》之类的作品。
有时也指用别人口气写诗来抒发自己的情怀。
}拟《春江花月夜》之格,\zhu{拟《春江花月夜》之格:
这里的《秋窗风雨夕》是拟张若虚《春江花月夜》的格调,故云。
}
乃名其词曰《秋窗风雨夕》。
其词曰:\par
\hop
秋花惨淡秋草黄,耿耿秋灯秋夜长。
\zhu{耿耿:隐隐有些光亮的样子;喻心中有所思虑而展转不寐。
}\par
已觉秋窗秋不尽,那堪风雨助凄凉!\par
助秋风雨来何速!惊破秋窗秋梦绿。
\zhu{下句意谓秋天来临,草木将衰萎枯黄。
但秋之初来,人尚不觉,故秋梦之中还是一片绿色,而今风雨相催,摇撼秋窗,惊破了绿色的梦,由此而产生物色变衰,年华易老之感。
}\par
抱得秋情不忍眠,自向秋屏移泪烛。
\par
泪烛摇摇爇短檠,牵愁照恨动离情。
\zhu{爇短檠:谓烛将燃尽,烧及灯台。
爇:音“若”,燃烧;点燃。
檠:音“情”,烛台,灯架。
}\par
谁家秋院无风入?何处秋窗无雨声?\par
罗衾不奈秋风力,残漏声催秋雨急。
\par
连宵脉脉复飕飕,灯前似伴离人泣。
\zhu{脉脉:这里指细雨连绵的样子。
飕飕:形容风吹、雨打的声音,形容寒冷的样子。
}\par
寒烟小院转萧条,疏竹虚窗时滴沥。
\zhu{虚窗:指寂寥冷落的窗子。}
\par
不知风雨几时休,已教泪洒纱窗湿。
\par
\hop
吟罢搁笔,方要安寝,丫鬟报说:“宝二爷来了。
”一语未完,只见宝玉头上戴着大箬笠,\zhu{箬笠[ruòlì]:用箬竹的叶或篾编结成的宽边帽,用来遮雨和遮阳光。
箬竹[ruòzhú]:竹子的一种。叶子宽而大,可用来编制器具或斗笠,也可用来包粽子。
篾[miè]:劈成条状的薄竹片。
}身上披着蓑衣。
黛玉不觉笑了:“那里来的渔翁!”宝玉忙问:“今儿好些?\geng{一句。
}吃了药没有?\geng{两句。
}今儿一日吃了多少饭?”\geng{三句。
}一面说,一面摘了笠,脱了蓑衣,忙一手举起灯来,一手遮住灯光,向黛玉脸上照了一照,觑着眼细瞧了一瞧,笑道:“今儿气色好了些。
”\par
黛玉看脱了蓑衣,里面只穿半旧红绫短袄,系着绿汗巾子,膝下露出油绿绸撒花裤子,底下是掐金满绣的绵纱袜子,\zhu{
掐金:用金线绣出图案或花样的边缘。
}靸着蝴蝶落花鞋。
\zhu{靸:音“洒”,穿鞋时把鞋后帮踩在脚后跟下,拖着走。
}黛玉问道:“上头怕雨,底下这鞋袜子是不怕雨的?也倒干净。
”宝玉笑道:“我这一套是全的。
有一双棠木屐,\zhu{棠木屐:棠木制作的屐,下有高齿,雨天当套鞋用。
棠:即棠梨,也叫杜梨,落叶乔木,木质坚韧。
屐:音“基”,木鞋。
}才穿了来,脱在廊檐上了。
”黛玉又看那蓑衣斗笠不是寻常市卖的,十分细致轻巧,因说道:“是什么草编的?怪道穿上不像那刺猬似的。
”宝玉道:“这三样都是北静王送的。
他闲了下雨时在家里也是这样。
你喜欢这个,我也弄一套来送你。
别的都罢了,惟有这斗笠有趣,竟是活的。
上头的这顶儿是活的,冬天下雪,戴上帽子,就把竹信子抽了,\zhu{竹信子:竹子做的插销。
信子亦作“芯子”。
}去下顶子来,只剩了这圈子。
下雪时男女都戴得,我送你一顶,冬天下雪戴。
”黛玉笑道:“我不要他。
戴上那个,成个画儿上画的和戏上扮的渔婆了。
”及说了出来,方想起话未忖夺,
\zhu{忖夺:犹忖度。}
与方才说宝玉的话相连,后悔不及,羞的脸飞红,便伏在桌上嗽个不住。
\geng{妙极之文。
使黛玉自己直说出夫妻来,\zhu{宝玉先自称“渔翁”,黛玉后自称“渔婆”。
}却又云“画的”“扮的”,本是闲谈,却是暗隐不吉之兆。
所谓“画儿中爱宠”是也,谁曰不然?}\par
宝玉却不留心,\geng{必云“不留心”方好,方是宝玉。
若着心则又有何文字?且直是一时时猎色一贼矣。
}因见案上有诗,遂拿起来看了一遍,又不禁叫好。
黛玉听了,忙起来夺在手内,向灯上烧了。
宝玉笑道:“我已背熟了,烧也无碍。
”黛玉道:“我也好了些,多谢你一天来几次瞧我,下雨还来。
这会子夜深了,我也要歇着,你且请回去,明儿再来。
”宝玉听说,回手向怀中掏出一个核桃大小的一个金表来,瞧了一瞧,那针已指到戌末亥初之间,
\zhu{戌末亥初:晚上九点左右。}
忙又揣了,说道:“原该歇了,又扰的你劳了半日神。
”说着,披蓑戴笠出去了,又翻身进来问道:“你想什么吃,告诉我,我明儿一早回老太太,岂不比老婆子们说的明白?”\geng{直与后部宝钗之文遥遥针对。
\ping{
“后部宝钗之文”当指钗玉婚后有关情节,前八十回中没有这样的文字。
这句批语意思很明显:钗玉婚后宝玉对宝钗并非仇敌相见,对宝钗的温情他报以体贴、关怀,犹如体贴黛玉一样。
当然具体情节内涵不一定要与黛玉相同,但其文理情致却是一致的。
}}\geng{想彼姊妹房中婆子丫鬟皆有,随便皆可遣使,今宝玉独云“婆子”而不云“丫鬟”者,心内已度定丫鬟之为人,一言一事,无论大小,是方无错谬者也,一何可笑!}黛玉笑道:“等我夜里想着了,明儿早起告诉你。
你听雨越发紧了,快去罢。
可有人跟着没有?”有两个婆子答应:“有人,外面拿着伞点着灯笼呢。
”黛玉笑道:“这个天点灯笼?”宝玉道:“不相干,是明瓦的,\zhu{明瓦:古时未有玻璃,用蛎壳磨成半透明的薄片,嵌于窗间或灯架上以透光照明,谓之“明瓦”。
}不怕雨。
”黛玉听了,回手向书架上把个玻璃绣球灯拿了下来,命点一支小蜡来,递与宝玉,道:“这个又比那个亮,正是雨里点的。
”宝玉道:“我也有这么一个,怕他们失脚滑倒了打破了,所以没点来。
”黛玉道:“跌了灯值钱,跌了人值钱?你又穿不惯木屐子。
那灯笼命他们前头点着。
这个又轻巧又亮,原是雨里自己拿着的,你自己手里拿着这个,岂不好?明儿再送来。
就失了手也有限的,怎么忽然又变出这‘剖腹藏珠’的脾气来!”\zhu{剖腹藏珠:《资治通鉴·唐太宗贞观元年》:“上(唐太宗)谓侍臣曰:‘吾闻西域贾胡得美珠,剖身以藏之。
’彼之爱珠而不爱其身。
”后以喻为物伤身,轻重倒置。
}宝玉听说,连忙接了过来,前头两个婆子打着伞提着明瓦灯,后头还有两个小丫鬟打着伞。
宝玉便将这个灯递与一个小丫头捧着,宝玉扶着他的肩,一径去了。
\par
就有蘅芜苑的一个婆子,也打着伞提着灯,送了一大包上等燕窝来,还有一包子洁粉梅片雪花洋糖。
\zhu{
梅片:即“冰片”,又被称为“龙脑”,龙脑香树脂的加工品,呈白色晶体状,气味清凉。
洁粉雪花:形容冰片之白。
洋糖:指从外国进口的机制糖;另一种说法是指绵白糖。
洁粉梅片雪花洋糖:白糖加上冰片。
}
说:“这比买的强。
姑娘说了:姑娘先吃着,完了再送来。
”黛玉回说“费心”,命他外头坐了吃茶。
婆子笑道:“不吃茶了,我还有事呢。
”黛玉笑道:“我也知道你们忙。
如今天又凉,夜又长,越发该会个夜局,痛赌两场了。
”婆子笑道:“不瞒姑娘说,今年我大沾光儿了。
横竖每夜各处有几个上夜的人,误了更也不好,不如会个夜局,又坐了更,又解闷儿。
今儿又是我的头家,
\zhu{
头家:聚赌抽头的人。
抽头:向赢钱的赌徒抽取一部分的利益给提供赌博场所的人。
}
如今园门关了,就该上场了。
”\geng{几句闲话,将潭潭大宅夜间所有之事描写一尽。
虽偌大一园,且值秋冬之夜,岂不寥落哉?今用老妪数语,更写得每夜深人定之后,各处[灯]光灿烂、人烟簇集,柳陌\sout{之}[小]巷之中,或提灯同酒,或寒月烹茶者,竟仍有络绎人迹不绝,不但不见寥落,且觉更胜于日间繁华矣。
此是大宅妙景,不可不写出。
又伏下后文,
\zhu{第七十三回,贾母彻查赌博。}
且又衬出后文之冷落。
此闲话中写出,正是不写之写也。
脂砚斋评。
}黛玉听说笑道:“难为你。
误了你发财,冒雨送来。
”命人给他几百钱打些酒吃,避避雨气。
那婆子笑道:“又破费姑娘赏酒吃。
”说着,磕了一个头,外面接了钱,打伞去了。
\par
紫鹃收起燕窝,然后移灯下帘,伏侍黛玉睡下。
黛玉自在枕上感念宝钗,一时又羡他有母兄;一面又想宝玉虽素习和睦,终有嫌疑。
又听见窗外竹梢焦叶之上,雨声淅沥,清寒透幕,不觉又滴下泪来。
直到四更将阑,
\zhu{阑:将尽。}
方渐渐的睡了。
暂且无话。
要知端的——\par
\qi{总评:请看赖大,则知贵家奴婢身份,而本主毫不以为过分,习惯自然,故是有之。
见者当自度是否可也。
}
\dai{089}{金兰契互剖金兰语}
\dai{090}{黛玉送宝玉玻璃绣球灯,宝玉披蓑衣离开潇湘馆}
\sun{p45-1}{金兰契互剖金兰语,宝玉披蓑衣冒雨探视}{黛玉咳嗽比往日更甚,宝钗来看她,彼此互诉心曲。
宝钗走后,黛玉作《秋窗风雨夕》,方要安寝,宝玉戴着大箬笠,身上披着蓑衣来看黛玉。
黛玉不觉笑道:“哪里来的渔翁!”}