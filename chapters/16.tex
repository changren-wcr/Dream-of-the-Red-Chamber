\chapter{贾元春才选凤藻宫\quad 秦鲸卿夭逝黄泉路}
\jia{幼儿小女之死,得情之正气,又为痴贪辈一针灸。
\hang
凤姐恶迹多端,莫大于此件者:受赃婚以致人命。
\hang
贾府连日闹热非常,宝玉无见无闻,却是宝玉正文。
夹写秦、智数句,下半回方不突然。
\hang
黛玉回,方解宝玉为秦钟之忧闷,是天然之章法。
平儿借香菱答话,是补菱姐近来着落。
赵妪讨情闲文,却引出通部脉络。
所谓由小及大,譬如登高必自卑之意。
\hang
细思大观园一事,若从如何奉旨起造,又如何分派众人,从头细细直写将来,几千样细事,如何能顺笔一气写清?又将落于死板拮据之乡,故只用琏凤夫妻二人一问一答,上用赵妪讨情作引,下文蓉蔷来说事作收,馀者随笔顺笔略一点染,则耀然洞彻矣。
此是避难法。
\hang
大观园用省亲事出题,是大关键处,方见大手笔行文之立意。
\hang
借省亲事写南巡,出脱心中多少忆昔感今。
\hang
极热闹极忙中写秦钟夭逝,可知除“情”字,俱非宝玉正文。
\hang
大鬼小鬼论势利兴衰,骂尽攒炎附势之辈。
}\par
\qi{请看财势与情根,万物难逃造化门。
旷典传来空好听。
\zhu{旷:空绝。旷典:谓稀有难逢的盛大典礼;前所未有的典制。这里是指元春省亲之事。}
那如知己解温存?}\par
诗曰:……\par
\hop
却说宝玉见收拾了外书房,约定与秦钟读夜书。
偏那秦钟秉性最弱,因在郊外受了些风霜,又与智能儿偷期绻缱,未免失于调养,\geng{勿笑。
这样无能,却是写与人看。
}回来时便咳嗽伤风,懒进饮食,大有不胜之态,遂不敢出门,只在家中养息。
\jia{为下文伏线。
}宝玉便扫了兴头,只得付于无可奈何,且自静候大愈时再约。
\jia{所谓“好事多磨”也。
[脂砚。
]
\foot{方括号内的署名表示此署名甲戌本没有但己、庚本有,据补。
下同。
}}\par
那凤姐儿已是得了云光的回信,俱已妥协。
老尼达知张家,\zhu{达:通晓。
}果然那守备忍气吞声的收了前聘之物。
谁知那个张财主虽如此爱势贪财,却养了一个知义多情的女儿,\geng{所谓“老鸦窝里出凤凰”,此女是在十二钗之外副者。
}闻得父母退了亲事,他便一条绳索悄悄的自缢了。
那守备之子闻得金哥自缢,他也是个极多情的,遂也投河而死。
\geng{灭一双美满夫妻。
}
只落得张李两家没趣,真是人财两空。
这里凤姐却坐享了三千两,\geng{如何消缴?\zhu{消缴:这里是消除罪孽的意思。
}造孽者不知,自有知者。
}王夫人等连一点消息也不知道。
自此凤姐胆识愈壮,以后有了这样的事,便恣意的作为起来,也不消多记。
\jia{一段收拾过。
阿凤心机胆量,真与雨村是一对乱世之奸雄。
后文不必细写其事,则知其平生之作为。
回首时,无怪乎其惨痛之态,使天下痴心人同来一警,或可期共入于恬然自得之乡矣。
[脂砚。
]}\par
一日,正是贾政的生辰,宁荣二处人丁都齐集庆贺,热闹非常。
忽有门吏忙忙进来,至席前报说:“有六宫都太监夏老爷来降旨。
”\zhu{六宫都太监:六宫:皇后与妃嫔所居之处。
都太监:太监的总管,作者虚拟的官名。
都:总管之谓。
}吓得贾赦、贾政等一干人不知是何消息,忙止了戏文,撤去酒席,摆香案启中门跪接。
早见六宫都监夏守忠乘马而至,前后左右又有许多内监跟从。
那夏守忠也不曾负诏捧敕,至檐前下马,满面笑容,走至厅上,南面而立,口内说:“特旨:立刻宣贾政入朝,在临敬殿陛见。
”\zhu{陛见:臣下谒见皇帝。
陛:音“币”,宫殿的台阶。
}说毕,也不及吃茶,便乘马去了。
贾政等不知是何兆头,只得急忙更衣入朝。
\geng{泼天喜事却如此开宗。
出人意料外之文也。
壬午季春。
}\par
贾母等合家人等心中皆惶惶不定,不住的使人飞马来往报信。
有两个时辰工夫,忽见赖大等三四个管家喘吁吁跑进仪门报喜,
\zhu{
仪门:旧时官衙、府第的大门之内的门,具装饰作用。
一说,旁门也可称仪门。
}
又说“奉老爷命,速请老太太带领太太等进朝谢恩”等语。
那时贾母正心神不定,在大堂廊下伫立,\geng{慈母爱子写尽。
回廊下伫立与“日暮倚庐仍怅望”对景,余掩卷而泣。
}\geng{“日暮倚庐仍怅望”,南汉先生句也。
}
\zhu{
日暮倚庐仍怅望:脂批所引诗句,不知见于何本。南汉先生未能确证何人。
}
邢夫人、王夫人、尤氏、李纨、凤姐、迎春姊妹以及薛姨妈等皆在一处。
听如此信至,贾母便唤进赖大来细问端的。
\zhu{端的:究竟、详情。}
赖大禀道:“小的们只在临敬门外伺候,里头的信息一概不能得知。
后来还是夏太监出来道喜,说咱家大小姐晋封为凤藻宫尚书,\zhu{凤藻宫尚书:凤藻宫:作者虚拟的宫名。
尚书:官名,三国时魏国曾设女尚书之职。
唐代白居易《上阳白发人》诗中有“大家(皇帝)遥赐尚书号”之句。
清代无此例。
}加封贤德妃。
后来老爷出来亦如此吩咐小的。
如今老爷又往东宫去了,速请老太太领着太太们去谢恩。
”贾母等听了方心神安定,不免又都洋洋喜气盈腮。
\geng{字眼,留神。
亦人之常情。
}于是都按品大妆起来。
贾母带领邢夫人、王夫人、尤氏,一共四乘大轿入朝。
贾赦、贾珍亦换了朝服,带领贾蓉、贾蔷奉侍贾母大轿前往。
于是宁荣二处上下里外,莫不欣然踊跃,\chen{秦氏生魂先告凤姐矣。
}个个面上皆有得意之状,言笑鼎沸不绝。
\par
谁知近日水月庵的智能私逃进城,\jia{好笔仗,好机轴。
}\jia{忽然接水月庵,似大脱泄。
及读至后,方知为紧收。
此大段有如歌疾调迫之际,忽闻戛然檀板截断,真见其大力量处,却便于写宝玉之文。
}找至秦钟家下看视秦钟,不意被秦业知觉,将智能逐出,将秦钟打了一顿,自己气的老病发作,三五日的光景呜呼死了。
秦钟本自怯弱,又值带病未愈,受了笞打,今见老父气死,此时悔痛无及,更又添了许多症候。
因此宝玉心中怅然如有所失。
\geng{凡用宝玉收拾,俱是大关键。
}虽闻得元春晋封之事,亦未解得愁闷。
\jia{眼前多少[热闹]文字不写,却从万人意外撰出一段悲伤,是别人不屑写者,亦别人之不能处。
}贾母等如何谢恩,如何回家,亲朋如何来庆贺,宁荣两处近日如何热闹,众人如何得意,独他一个皆视有如无,毫不曾介意。
\geng{的的真真宝玉。
}因此众人嘲他越发呆了。
\jia{大奇至妙之文,却用宝玉一人,连用五“如何”,隐过多少繁华势利等文。
试思若不如此,必至种种写到,其死板拮据、琐碎杂乱,何可胜哉?故只借宝玉一人如此一写,省却多少闲文,却有无限烟波。
}
\geng{越发呆了。
}\par
且喜贾琏与黛玉回来,先遣人来报信,明日就可到家,宝玉听了,方略有些喜意。
\jia{不如此,后文秦钟死去,将何以慰宝玉?}细问原由,方知贾雨村也进京陛见,皆由王子腾累上保本,\zhu{累:重叠,积累,这里指多次。
保本:封建官吏向皇帝保荐人才的奏本。
}此来候补京缺,与贾琏是同宗弟兄,又与黛玉有师徒之谊,故同路作伴而来。
林如海已葬入祖坟了,诸事停妥,贾琏方进京的。
本该出月到家,因闻得元春喜信,遂昼夜兼程而进,一路俱各平安。
宝玉只问得黛玉“平安”二字,馀者也就不在意了。
\jia{又从天外写出一段离合来,总为掩过宁、荣两处许多琐细闲笔。
处处交代清楚,方好起大观园也。
}\par
好容易\geng{三字是宝玉心中。
}盼至明日午错,
\zhu{午错:正午已过。}
果报:“琏二爷和林姑娘进府了。
”见面时彼此悲喜交接,未免又大哭一阵,后又致喜庆之词。
\jia{世界上亦如此,不独书中瞬息。
观此便可省悟。
}宝玉心中品度黛玉,越发出落的超逸了。
黛玉又带了许多书籍来,忙着打扫卧室,安插器具,又将些纸笔等物分送宝钗、迎春、宝玉等人。
宝玉又将北静王所赠鹡鸰香串珍重取出来,转赠黛玉。
黛玉说:“什么臭男人拿过的!我不要他。
”\ping{北静王亲口说:“此系前日圣上亲赐鹡鸰香”,这个“臭男人”可能不仅有北静王,还有当朝皇帝。
}
遂掷而不取。
宝玉只得收回,暂且无话。
\jia{略一点黛玉性情,赶忙收住,正留为后文地步。
}\par
且说贾琏自回家参见过众人,回至房中。
正值凤姐近日多事之时,无片刻闲暇之工,\jia{补阿凤二句最不可少。
}见贾琏远路归来,少不得拨冗接待,\geng{写得尖利刻薄。
}房内无外人,便笑道:“国舅老爷大喜!国舅老爷一路风尘辛苦。
\jia{娇音如闻,俏态如见,少年夫妻常事,的确有之。
}小的听见昨日的头起报马来报,\zhu{报马:这里代指报告消息的人。
}说今日大驾归府,略预备了一杯水酒掸尘,\geng{却是为下文作引。
}不知可赐光谬领否?”贾琏笑道:“岂敢岂敢,多承多承!”\geng{一言答不上,蠢才蠢才!}一面平儿与众丫鬟参拜毕,献茶。
贾琏遂问别后家中的事,又谢凤姐操持劳碌。
凤姐道:“我那里照管得这些事!见识又浅,口角又夯,心肠又直率,人家给个棒槌,我就认作针。
脸又软,搁不住人给两句好话,心里就慈悲了。
况且又无经历过大事,胆子又小,太太略有些不自在,就吓得我连觉也睡不着了。
我苦辞了几回,太太又不容辞,倒反说我图受用了,不肯习学了。
殊不知我是捻着一把汗儿呢。
一句也不敢多说,一步也不敢多走。
\jia{此等文字,作者尽力写来,欲诸公认识阿凤,好看后文,勿为泛泛看过。
}你是知道的,咱们家所有的这些管家奶奶们,那一位是好缠的?\jia{独这一句不假。
[脂砚。
]}
错一点儿他们就笑话打趣,偏一点儿他们就指桑说槐的抱怨。
‘坐山观虎’、‘借剑杀人’、‘引风吹火’、‘站干岸儿’、‘推倒油瓶不扶’,都是全挂子的武艺。
况且我年纪轻,头等不压众,怨不得不放我在眼里。
更可笑\geng{三字是得意口气。
}那府里忽然蓉儿媳妇死了,珍大哥又再三再四的在太太跟前跪着讨情,只要请我帮他几日;我是再四推辞,太太断不依,只得从命。
依旧被我闹了个马仰人翻,\geng{得意之至口气。
}更不成个体统,至今珍大哥还抱怨后悔呢。
你这一来了,明儿你见了他,好歹描补描补,\zhu{描补:这里指说话办事有不周到处,事后加以解释弥补。
}就说我年纪小,原没见过世面,谁叫大爷错委他的。
”\jia{阿凤之待琏兄如弄小儿,可思之至。
}\geng{阿凤之弄琏兄如弄小儿,可怕可畏!若生于小户,落在贫家,琏兄死矣!}\ping{强势如凤姐,亦在丈夫面前装弱,要丈夫疼她。
}\par
正说着,\jia{又用断法方妙。
盖此等文断不可无,亦不可太多。
}只听外间有人说话,凤姐便问:“是谁?”平儿进来回道:“姨太太打发香菱妹子来问我一句话,我已经说了,打发他回去了。
”贾琏笑道:“正是呢,方才我见姨妈去,不防和一个年轻的小媳妇子撞了个对面,生的好齐整模样。
\geng{酒色之徒。
}我疑惑咱家并无此人,说话时因问姨妈,谁知就是上京来买的那小丫头,名叫香菱的,竟与薛大傻子作了房里人,开了脸,\zhu{开脸:旧俗女子出嫁时用线绞净脸上的汗毛,修齐鬓角,叫作“开脸”。
}
越发出挑的标致了。
那薛大傻子真玷辱了他。
”\jia{垂涎如见。
试问兄宁有不玷平儿乎?[脂砚。
]}凤姐道:“嗳!\geng{如闻。
}往苏杭走了一趟回来,也该见些世面了,\jia{这“世面”二字,单指女色也。
}还是这么眼馋肚饱的。
你要爱他,不值什么,我去拿平儿换了他来如何?\jia{奇谈,是阿凤口中方有此等语句。
}\jia{用平儿口头谎言,写补菱卿一项实事,并无一丝痕迹,而有作者多少机括。
}那薛老大\jia{又一样称呼,各得神理。
}也是‘吃着碗里望着锅里’的,这一年来的光景,他为要香菱不能到手,\jia{补前文之未到,且并将香菱身分写出。
[脂砚。
]}和姨妈打了多少饥荒。
\zhu{打饥荒:此指纠缠不休,找麻烦。
}也因姨妈看着香菱的模样儿好还是末则,其为人行事,却又比别的女孩儿不同,温柔安静,差不多的主子姑娘也跟他不上呢,\jia{何曾不是主子姑娘?盖卿不知来历也,作者必用阿凤一赞,方知莲卿尊重不虚。
}故此摆酒请客的费事,明堂正道的与他作了妾。
过了没半月,也看的马棚风一般了,\zhu{马棚风:比喻习以为常。
不当一回事。
}我倒心里可惜了的。
”\jia{一段纳宠之文,偏于阿凤口中补出,亦奸猾幻妙之至!}一语未了,二门上小厮传报:“老爷在大书房等二爷呢。
”贾琏听了,忙忙整衣出去。
\par
这里凤姐乃问平儿:“方才姨妈有什么事,巴巴的打发香菱来?”\jia{必有此一问。
}平儿笑道:“那里来的香菱,是我借他暂撒个谎。
\jia{卿何尝谎言?的是补菱姐正文。
}奶奶说说,旺儿嫂子越发连个承算也没了。
”\chen{此处系平儿捣鬼。
}说着,又走至凤姐身边,悄悄说道:\geng{如闻如见。
}
“奶奶的那利钱银子,迟不送来,早不送来,这会子二爷在家,他且送这个来了。
\jia{总是补遗。
}幸亏我在堂屋里撞见,不然时走了来回奶奶,二爷倘或问奶奶是什么利钱,奶奶自然不肯瞒二爷的,\jia{平姐欺看书人了。
}\geng{可儿可儿,凤姐竟被他哄了。
}少不得照实告诉二爷。
我们二爷那脾气,油锅里的钱还要找出来花呢,听见奶奶有了这个梯己,
\zhu{
梯己:意即私人的、贴心的。
私蓄亦可称作“梯己”。
}
他还不放心的花了呢?所以我赶着接了过来,叫我说了他两句。
谁知奶奶偏听见了问,我就撒谎说香菱了。
”\jia{一段平儿的见识作用,不枉阿凤生平刮目,又伏下多少后文,补尽前文未到。
}凤姐听了笑道:“我说呢,姨妈知道你二爷来了,忽喇八的反打发个房里人来了?\zhu{忽喇八:忽然、凭空。
}原来你这蹄子肏鬼。
”\zhu{蹄子:骂女人的话。
}\geng{疼极反骂。
}\par
说话时,贾琏已进来,凤姐便命摆上酒馔来,夫妻对坐。
凤姐虽善饮,却不敢任兴,\jia{百忙中又点出大家规范,所谓无不周详,无不贴切。
}只陪着贾琏。
一时贾琏的乳母赵嬷嬷走来,贾琏与凤姐忙让他一同吃酒,令其上炕去。
赵嬷嬷执意不肯。
平儿等早已炕沿下设下一杌子,\zhu{杌(音“物”):小凳子。
}又有一小脚踏,赵嬷嬷在脚踏上坐了。
贾琏向桌上拣两盘肴馔,与他放在杌上自吃。
凤姐又道:“妈妈很咬不动那个,倒没的硌了他的牙。
”\zhu{硌(硌音“各”)牙:牙齿嚼到硬东西而感到难受。
}\geng{何处着想?却是自然有的。
}因向平儿道:“早起我说那一碗火腿炖肘子很烂,正好给妈妈吃,你怎么不取去赶着叫他们热来?”又道:“妈妈,你尝一尝你儿子带来的惠泉酒。
”\zhu{惠泉酒:惠泉水所酿的酒。
惠泉在江苏无锡惠山第一峰下,号称“天下第二泉”。
}\geng{补点不到之文,像极!}赵嬷嬷道:“我喝呢,奶奶也喝一钟。
怕什么,只不要过多了就是了。
\jia{宝玉之李嬷,此处偏又写一赵嬷,特犯不犯。
\zhu{犯:重复。
特犯不犯:第一个“犯”,指的是写了两个奶妈,确实有重复之嫌。
第二个“犯”,指的是虽然写了两个奶妈,但是行文内容并不重复,每个奶妈都有自己的人物特点,并非数量上的简单堆砌。
}先有梨香院一回,今又写此一回,两两遥对,却无一笔相重,一事合掌。
}我这会子跑来,倒也不为酒饭,倒有一件正经事,奶奶好歹记在心里,疼顾我些罢。
我们的爷,只是嘴里说的好,到了跟前就忘了我们。
幸亏我从小儿奶了你这么大。
\zhu{幸亏:这里应该是“亏”的意思。
}我也老了,有的是那两个儿子,你就另眼照看他们些,别人也不敢呲牙儿的。
\zhu{呲(呲音“资”)牙儿:掀唇露齿。
这里引申为议论讥诮别人。
}\geng{为蔷、蓉作引。
}我还再四的求了你几遍,你答应的倒好,到如今还是燥屎。
\zhu{燥屎:歇后语:“燥屎——干搁着”。
此指对受托之事漫不经心,搁置未办。
}\geng{有是乎?}这如今又从天上跑出这样一件大喜事来,那里用不着人?所以倒是来求奶奶是正经。
靠着我们爷,只怕我还饿死了呢。
”\ping{第二回:“谁知自娶了他令夫人之后,倒上下无一人不称颂他夫人的,琏爷倒退了一射之地。
”从引文可见王熙凤的强势,求贾琏没用,求凤姐有用。
第二十四回:贾芸对凤姐说道:“求叔叔(贾琏)这事,婶子(凤姐)休提,我昨儿正后悔呢。
早知这样,我竟一起头求婶子,这会子也早完了。
谁承望叔叔竟不能的。
”从引文更加印证了求贾琏不如求凤姐。
帮奶妈儿子讨个差事这件事微妙增加了夫妻的矛盾,贾琏听得这说法,心里岂会舒坦?}\par
凤姐笑道:“妈妈你放心,两个奶哥哥都交给我。
\zhu{奶哥哥:乳母的比自己年长的儿子;又称奶兄。}
你从小儿奶的,你还有什么不知道他那脾气的?拿着皮肉倒往那不相干的外人身上贴。
可是现放着奶哥哥,那一个不比人强?你疼顾照看他们,谁敢说个‘不’字儿?\geng{会送情。
}没的白便宜了外人。
——我这话也说错了,我们看着是‘外人’,你却是看着是‘内人’一样呢。
”\zhu{内人:对人称自己的妻子为“内人”。
}\geng{可儿可儿!}说的满屋里人都笑了。
赵嬷嬷也笑个不住,又念佛道:“可是屋子里跑出青天来了!若说‘内人’‘外人’这些混帐事,我们爷是没有,\jia{千真万真,是没有。
一笑。
}\geng{有是语,像极,毕肖。
乳母护子。
}不过是脸软心慈,搁不住人求两句罢了。
”凤姐笑道:“可不是呢,有‘内人’求的他才慈软呢,他在咱们娘儿们跟前才是刚硬呢!”\ping{讽刺贾琏在外包养女人,把“外人”即娼妓小蜜当作“内人”即妻子般看待,百般体贴照顾;却对自己的妻子“刚硬”。
}赵嬷嬷笑道:“奶奶说的太尽情了,我也乐了。
再吃一杯好酒。
从此我们奶奶作了主,我就没的愁了。
”\par
贾琏此时没好意思,只是讪笑吃酒,说“胡说”二字,“快盛饭来,吃碗子还要往珍大爷那边去商议事呢。
”凤姐道:“可是。
别误了正事。
才刚老爷叫你说什么?”\ji{一段赵妪讨情闲文,却引出通部脉络。
所谓由小及大,譬如登高必自卑之意。
细思大观园一事,若从如何奉旨起造,又如何分派众人,从头细细直写将来,几千样细事,如何能顺笔一气写清?又将落于死板拮据之乡,故只用琏凤夫妻二人一问一答,上用赵妪讨情作引,下用蓉蔷来说事作收,馀者随笔顺笔略一点染,则耀然洞彻矣。
此是避难法。
}贾琏道:“就为省亲。
”\zhu{省(省音“醒”)亲:探望父母等长辈尊亲。
省:探望问安。
}\jia{二字醒眼之极,却只如此写来。
}\geng{大观园用省亲事出题,是大关键事,方见大手笔行文之立意。
畸笏。
}凤姐忙问道:\jia{“忙”字最要紧,特于凤姐口中出此字,可知事关巨要,非同浅细,是此书中正眼矣。
}“省亲的事竟准了不成?”\jia{问得珍重,可知是万人意外之事。
[脂砚。
]}贾琏笑道:“虽不十分准,也有八分准了。
”\jia{如此故顿一笔,更妙!见得事关重大,非一语可了者,亦是大篇文章,抑扬顿挫之至。
}凤姐笑道:“可见当今的隆恩。
\zhu{当今:意同“今上”,即当朝皇帝。
}历来听书看戏,古时从来未有的。
”\jia{于闺阁中作此语,直与《击壤》同声。
[脂砚。
]\zhu{论衡·卷八·艺增篇:“有年五十击壤于路者,观者曰:‘大哉!尧德乎!’击壤者曰:‘吾日出而作,日入而息,凿井而饮,耕田而食,尧何等力!”原文是赞颂尧的德政,用在此处意在赞颂当朝皇帝的德政。
}}赵嬷嬷又接口道:“可是呢,我也老糊涂了。
我听见上上下下吵嚷了这些日子,什么省亲不省亲,我也不理论他去;如今又说省亲,到底是怎么个原故?”\jia{补近日之事,启下回之文。
}\jia{赵嬷一问是文章家进一步门庭法则。
}\geng{自政老生日,用降旨截住,贾母等进朝如此热闹,用秦业死岔开,只写几个“如何”,将泼天喜事交代完了,紧接黛玉回,琏、凤闲话,以老妪勾出省亲事来。
其千头万绪,合榫贯连,无一毫痕迹,如此等,是书多多,不能枚举。
想兄在青埂峰上,经煅炼后,参透重关至恒河沙数。
\zhu{
重关:佛教用语,为禅宗悟道三关之一。
恒河沙数:比喻数量多得像恒河里的沙子一样无法计数。
}
如否,余曰万不能有此机括,有此笔力,恨不得面问果否。
叹叹!丁亥春。
畸笏叟。
}贾琏道:\jia{大观园一篇大文,千头万绪,从何处写起,今故用贾琏夫妻问答之间,闲闲叙出,观者已省大半。
\zhu{省:音“醒”,领悟、明白。}
后再用蓉、蔷二人重一渲染。
便省却多少赘瘤笔墨。
此是避难法。
}“如今当今体贴万人之心,
\zhu{当今:旧时称当代的皇帝。}
世上至大莫如‘孝’字,\zhu{世上至大莫如孝字:《孝经·圣治章》:“人之行莫大于孝。
”封建帝王“以孝治天下”,目的在由孝及忠,以巩固封建宗法制和君主制的统治。
}想来父母儿女之性,皆是一理,不是贵贱上分别的。
当今自为日夜侍奉太上皇、皇太后,尚不能略尽孝意,因见宫里嫔妃、才人等皆是入宫多年,以致抛离父母音容,岂有不思想之理?在儿女,思想父母是分所应当。
想父母在家,若只管思念儿女,竟不能一见,倘因此成疾致病,甚至死亡,皆由朕躬禁锢,\zhu{朕(朕音“振”)躬:皇帝自称。
朕:本古人自称之词,意同“我”,至秦始皇始定为皇帝专用的自称,后代沿之。
躬:自身。
}不能使其遂天伦之愿,亦大伤天和之事。
故启奏太上皇、皇太后,每月逢二六日期,准其椒房眷属入宫请候看视。
\zhu{椒房:汉代后妃住的宫室用花椒和泥涂壁,取其温暖有香气;又因花椒结实多,兼有希求多子之意。
后以椒房代指后妃居处或后妃。
}于是太上皇、皇太后大喜,深赞当今至孝纯仁,体天格物。
\zhu{
体:设身处地为人着想,如“体谅”、“体察“。
体天:体察上天的意志。
格物:穷究事物的道理;纠正人的行为。
}
因此二位老圣人又下旨意,说椒房眷属入宫,未免有国体仪制,母女尚不能惬怀。
\zhu{惬怀:称心如意。
}竟大开方便之恩,特降谕诸椒房贵戚,除二六日入宫之恩外,凡有重宇别院之家,可以驻跸关防之处,\zhu{驻跸:帝王后妃在宫外的停留驻扎。
跸:音“毕”,帝王后妃出行,戒严清道,以防常人停留窥视。
关防:出于礼制和保安的需要而采取的分隔内外的措施。
清代内务府有执掌“关防”的机构和人员。
}不妨启请内廷銮舆入其私第,\zhu{銮舆:皇帝、后妃所乘的宫车。
}
庶可略尽骨肉私情、天伦中之至性。
此旨一下,谁不踊跃感戴?现今周贵人的父亲已在家里动了工了,\zhu{贵人:妃嫔称号的一种。
}修盖省亲别院呢。
又有吴贵妃的父亲吴天佑家,也往城外踏看地方去了。
\jia{又一样布置。
}这岂不有八九分了?”\par
赵嬷嬷道:“阿弥陀佛!原来如此。
这样说,咱们家也要预备接咱们大小姐了?”\geng{文忠公之嬷。
\zhu{文忠公:指苏轼,南宋时加赐苏轼谥号文忠。
苏轼有一篇纪念自己乳母的《乳母任氏墓志铭》:“赵郡苏轼子瞻之乳母任氏,名采莲,眉之眉山人。
父遂,母李氏。
事先夫人三十有五年,工巧勤俭,至老不衰。
乳亡姊八娘与轼,养视轼之子迈、迨、过,皆有恩劳。
从轼官于杭、密、徐、湖,谪于黄。
元丰三年八月壬寅,卒于黄之临皋亭,享年七十有二。
十月壬午,葬于黄之东阜黄冈县之北。
铭曰:生有以养之,不必其子也。
死有以葬之,不必其里也。
我祭其从与享之,其魂气无不之也。
”苏轼的奶妈任氏长期在苏家做佣人,照顾了苏家三代人,是苏家的历史见证者。
而贾琏的奶妈赵嬷嬷也是很小就到贾府做佣人了,后文赵嬷嬷回忆自己才记事儿的时候贾府接驾的情形可以佐证,因此赵嬷嬷也是贾家的历史见证者。
此条批语意在点出两者的相似之处,以文忠公苏轼的奶妈任氏类比贾琏的奶妈赵嬷嬷。
}}贾琏道:“这何用说呢!不然,这会子忙的是什么?”\jia{一段闲谈中补出多少文章。
真是费长房“壶中天地”也。
\zhu{后汉书·卷八十二下·方术列传第七十二下:费长房者,汝南人也。
曾为市掾。
市中有老翁卖药,悬一壶于肆头,及市罢,辄跳入壶中。
市人莫之见,唯长房于楼上睹之,异焉,因往再拜奉酒脯。
翁知长房之意其神也,谓之曰:“子明日可更来。
”长房旦日复诣翁,翁乃与俱入壶中。
唯见玉堂严丽,旨酒甘肴,盈衍其中,共饮毕而出。
这里的意思是用费长房在老翁的壶里看到壮丽的景观的典故,比喻在这一小段闲谈中能看出很多信息。
}}凤姐笑道:“若果如此,我可也见个大世面了。
可恨我小几岁年纪,若早生二三十年,如今这些老人家也不薄我没见世面了。
\jia{忽接入此句,不知何意,似属无谓。
}说起当年太祖皇帝仿舜巡的故事,\zhu{舜巡:古时天子巡行四方,祭山川,施教化,谓之“巡狩”。
相传帝舜曾南巡至苍梧之野,故这里称皇帝的巡行叫“舜巡”。
}比一部书还热闹,\geng{既知舜巡而又说热闹,此妇人女子口头也。
}我偏没造化赶上。
”\geng{不用忙,往后看。
}赵嬷嬷道:“嗳哟哟,那可是千载希逢的!那时候我才记事儿,咱们贾府正在姑苏、扬州一带监造海舫,
\zhu{舫:音“仿”,船。}
修理海塘,只预备接驾一次,\geng{又要瞒人。
\zhu{曹家接驾康熙南巡之事似漏非漏。}
}把银子都花的淌海水似的!说起来……”凤姐忙接道: 
\jia{又截得好。
“忙”字妙!上文“说起来”必未完,粗心看去则说疑阙,殊不知正传神处。
}“我们王府也预备过一次。
那时我爷爷单管各国进贡朝贺的事,凡有的外国人来,都是我们家养活。
\jia{点出阿凤所有外国奇玩等物。
}粤、闽、滇、浙所有的洋船货物,都是我们家的。
”\par
赵嬷嬷道:“那是谁不知道的?如今还有个口号儿呢,说‘东海少了白玉床,龙王来请江南王’,\geng{应前“葫芦案”。
\zhu{
第四回的护官符:贾不假,白玉为堂金作马。
阿房宫,三百里,住不下金陵一个史。
丰年好大雪,珍珠如土金如铁。
东海缺少白玉床,龙王来请金陵王。
}
}这说的就是奶奶府上了。
还有如今现在江南的甄家,\jia{甄家正是大关键、大节目,勿作泛泛口头语看。
}嗳哟哟,\geng{口气如闻。
}好势派!独他家接驾四次。
\geng{点正题正文。
}\ping{作者曹雪芹的祖父曹寅接待了康熙六次南巡中的四次。
}若不是我们亲眼看见,告诉谁谁也不信的。
别讲银子成了土泥,\geng{极力一写,非夸也,可想而知。
}凭是世上所有的,没有不是堆山塞海的,‘罪过可惜’四个字,竟顾不得了。
”\geng{真有是事,经过见过。
}凤姐道:“我常听见我们太爷们也这样说,岂有不信的。
\geng{对证。
}只纳罕他家怎么就这么富贵呢?”赵嬷嬷道:“告诉奶奶一句话,也不过是拿着皇帝家的银子往皇帝身上使罢了!\jia{是不忘本之言。
}谁家有那些钱买这个虚热闹去?”\jia{最要紧语。
人苦不自知。
能作是语者吾未尝见。
}\par
正说的热闹,王夫人又打发人来瞧凤姐吃了饭不曾。
凤姐便知有事等他,忙忙的吃了半碗饭,漱口要走,\geng{好顿挫。
}又有二门上小厮们回:“东府里蓉、蔷二位哥儿来了。
”贾琏才漱了口,平儿捧着盆盥手,见他二人来了,便问:“什么话?快说。
”凤姐且止步稍候,听他二人回些什么。
贾蓉先回说:“我父亲打发我来回叔叔:老爷们已经议定了,\geng{简净之至!}从东边一带,借着东府里的花园起,转至北边,一共丈量准了,三里半大,可以盖造省亲别院了。
\geng{园基乃一部之主,必当如此写清。
}已经传人画图样去了,\geng{后一图伏线。
\zhu{这句话的意思是,这张大观园的设计图,为此后的故事发展做了铺垫。
}大观园系玉兄与十二钗之太虚幻境,岂可草率?}明日就得。
叔叔才回家,未免劳乏,不用过我们那边去,\geng{应前贾琏口中。
\zhu{“贾琏口中”指的是贾琏说的皇帝准许元春省亲之事。}
}有话明日一早再请过去面议。
”贾琏笑着说道:“多谢大爷费心体谅,我就从命不过去了。
正经是这个主意才省事,盖的也容易;若采置别处地方去,那更费事,且倒不成体统。
你回去说,这样很好,若老爷们再要改时,全仗大爷谏阻,万不可另寻地方。
明日一早我给大爷请安去,再议细话。
”贾蓉忙应几个“是”。
\geng{园已定矣。
}\par
贾蔷又近前回说:“下姑苏割聘教习,\zhu{割聘:大致是聘请的意思。
}采买女孩子,置办乐器行头等事,\zhu{行:音“形“。行头:演戏所用的服装、道具等。
}大爷派了侄儿,\geng{“画蔷”一回伏线。
\zhu{指的是第三十回贾蔷和被贾府买来唱戏的龄官之间的情缘。
}}带领着来管家两个儿子,还有单聘仁、卜固修两个清客相公,\ping{单聘仁:谐音“善骗人”。
卜固修:谐音“不顾羞”。
}一同前往,所以命我来见叔叔。
”\geng{凡各物事,工价重大兼伏隐着情字者,莫如此件。
故园定后便先写此一件,馀便不必细写矣。
}贾琏听了,将贾蔷打量了打量,\geng{有神。
}笑道:“你能在这一行么?\geng{勾下文。
}这个事虽不甚大,里头大有藏掖的。
”\zhu{藏掖:隐匿,这里指营私舞弊的机会。
}\jia{射利人微露心迹。
\zhu{射:追求,攫取。
}}\geng{射利语,可叹!是亲侄。
}贾蔷笑道:“只好学习着办罢了。
”\par
贾蓉在身旁灯影下悄拉凤姐的衣襟,凤姐会意,\ping{暧昧如画。
}因笑道:“你也太操心了,难道大爷比咱们还不会用人?偏你又怕他不在行了。
谁都是在行的?孩子们已长的这么大了,‘没吃过猪肉,也看见过猪跑’。
大爷派他去,原不过是个坐纛旗儿,\zhu{坐纛旗儿:纛:音“道”,古代军中大旗。
坐纛旗儿:即主帅,这里借指主事的人。
}难道认真的叫他去讲价钱、会经纪去呢!\zhu{经纪:买卖。
商人亦称经纪人。
这里指旧时为买卖双方撮合交易从中赚取佣金的人。
}
依我说就很好。
”贾琏道:“自然是这样。
并不是我驳回,少不得替他筹算筹算。
”因问:“这项银子动那一处的?”贾蔷道:“才也议到这里。
赖爷爷\jia{此等称呼,令人酸鼻。
}\geng{好称呼。
}说,
\ping{有体面的老管家的身份地位相当高。}
竟不用从京里带下去,江南甄家还收着我们五万银子。
明日写一封书信,会票我们带去,\zhu{会票:即“汇票”,唐时号曰:“飞钱”,明清始有“会票”之称。
会票不仅用于汇款,且用作清偿债务的凭证。
}
先支三万,下剩二万存着,等置办花烛彩灯并各色帘栊帐幔的使费。
”贾琏点头道:“这个主意好。
”\geng{《石头记》中多作心传神会之文,不必道明。
一道明白,便入庸俗之套。
}\par
凤姐便向贾蔷道:\jia{再不略让一步,正是阿凤一生短处。
[脂砚。
]}“既这样,我有两个在行妥当人,你就带他们去办,这个便宜了你呢。
”\zhu{便宜(便音“变”):方便合适。
}贾蔷忙陪笑道:“正要和婶子讨两个人呢,\jia{写贾蔷乖处。
[脂砚。
]}这可巧了。
”因问名字。
凤姐便问赵嬷嬷。
彼时赵嬷嬷已听呆了话,平儿忙笑推他,\meng{真是强将手下无弱兵。
至精至细。
}他才醒悟过来,忙说:“一个叫赵天梁,一个叫赵天栋。
”凤姐道:“可别忘了,我可干我的去了。
”说着便出去了。
贾蓉忙赶出来,又悄悄向凤姐道:“婶子要带什么东西\foot{诸本此后有“分付我,开个账给蔷兄弟带了去,叫他按账置办了来。
”按贾蓉并非如此罗嗦之人,此处当以甲戌本原文点到为止较胜。
故不从增。
}?” 凤姐笑\geng{有神。
}道:“别放你娘的屁!\geng{像极,的是阿凤。
}我的东西还没处撂呢,稀罕你们鬼鬼祟祟的?”说着一迳去了。
\zhu{迳:同“径”。
}\jia{阿凤欺人处如此。
}\jia{忽又写到利弊,真令人一叹。
[脂砚。
]}\geng{从头至尾细看阿凤之待蓉、蔷,可为一体一党,然尚作如此语欺蓉,其待他人可知矣。
}\par
这里贾蔷也悄问贾琏:“要什么东西?顺便织来孝敬叔叔。
”\zhu{织:曹雪芹祖父曹寅在江南担任织造官,此处用“织”可能暗露家族传统。
}\ping{走关系拿到肥差,必须给恩主回报。
}贾琏笑道:“你别兴头。
\zhu{兴头:得意洋洋。
}才学着办事,倒先学会这把戏。
我短了什么,少不得写信去告诉你,\geng{又作此语,不犯阿凤。
}且不要论到这里。
”说毕,打发他二人去了。
接着回事的人来,不止三四次,贾琏害乏,便传与二门上,一应不许传报,俱等明日料理。
凤姐至三更时分方下来安歇,\geng{好文章,一句内隐两处若许事情。
\zhu{
一、贾琏疲于应对家务事,而凤姐却非常热衷。
二、凤姐熬夜导致身体差,为后文凤姐因病不能管事甚至流产、死亡埋下伏笔。
三、贾琏长期出差回家第一天,应付处理了很多事情,还没有来得及和凤姐单独相处。
贾琏把事情推脱到明天,可能想要和凤姐鸳梦重温,但是凤姐却对管理更加上心,没有满足贾琏。
这可能种下了贾琏出轨的种子。
}
}一宿无话。
\par
次日早贾琏起来,见过贾赦、贾政,便往宁府中来,合同老管事人等,并几位世交门下清客相公,审察两府地方,缮画省亲殿宇,
\zhu{缮:抄写。}
一面参度办理人丁。
自此后,各行匠役齐集,金银铜锡以及土木砖瓦之物,搬运移送不歇。
\meng{一总。
}先令匠役拆宁府会芳园墙垣楼阁,直接入荣府东大院中。
荣府东边所有下人一带群房尽已拆去。
当日宁荣二宅,虽有一小巷界断不通,\jia{补明,使观者如身临足到。
}然这小巷亦系私地,并非官道,故可以连属。
会芳园本是从北角墙下引来一股活水,今亦无烦再引。
\jia{园中诸景,最要紧是水,亦必写明方妙。
余最鄙近之修造园亭者,徒以顽石土堆为佳,不知引泉一道。
甚至丹青,唯知乱作山石树木,不知画泉之法,亦是恨事。
脂砚斋。
}其山石树木虽不敷用,\zhu{敷:够,足。
}贾赦住的乃是荣府旧园,其中竹树山石以及亭榭栏杆等物,皆可挪就前来。
如此两处又甚近,凑来一处,省得许多财力,纵亦不敷,所添亦有限。
全亏一个老明公号山子野\jia{妙号,随事生名。
}者,\zhu{明公:本用以称呼有学识有地位的人,后泛作对人的尊称,如同“先生”。
}一一筹画起造。
\par
贾政不惯于俗务,\geng{这也少不得的一节文字,省下笔来好作别样。
}只凭贾赦、贾珍、贾琏、赖大、来升、林之孝、吴新登、詹光、程日兴等几人安插摆布。
凡堆山凿池、起楼竖阁、种竹栽花一应点景等事,又有山子野制度。
\zhu{制度:这里作动词用,规划调度的意思。
}下朝闲暇,不过各处看望看望,最要紧处和贾赦商议商议便罢了。
贾赦只在家高卧,有芥豆之事,\zhu{芥:音“界”,小草,比喻轻微纤细的事物。
}贾珍等或自去回明,或写略节;\zhu{略节:简要的书面报告。
}
或有话说,便传呼贾琏、赖大等来领命。
贾蓉单管打造金银器皿。
\meng{好差。
}贾蔷已起身往姑苏去了。
贾珍、赖大等又点人丁,开册籍,监工等事,一笔不能写到,不过是喧阗热闹非常而已。
\zhu{喧阗:音“宣田”,声音喧哗、噪杂。
}暂且无话。
\par
且说宝玉近因家中有这等大事,贾政不来问他的书,\geng{一笔不漏。
}
心中是件畅事。
无奈秦钟之病一日重似一日,也着实悬心,不能乐业。
\jia{“天下本无事,庸人自扰之”,世上人个个如此,又非此情钟意切。
}\jia{偏于极热闹处写出大不得意之文,却无丝毫牵强,且有许多令人笑不了、哭不了、叹不了、悔不了,唯以大白酬我作者。
\zhu{白:酒杯。
}[壬午季春。
畸笏。
]}这日一早起来,才梳洗毕,意欲回了贾母去望候秦钟,忽见茗烟在二门照壁前探头缩脑,
\zhu{
照壁:坐落于门内外,与门相对,作为屏障,以区隔内外,操纵观感视线的一座短墙。
也作“影壁”、“照墙”。照壁有以砖、石、琉璃砌筑的,也有木制的。
}
宝玉忙出来问他:“作什么?”茗烟道:“秦相公不中用了!”\jia{从茗烟口中写出,省却多少闲文。
}宝玉听说,唬了一跳,忙问道:“我昨儿才瞧了他来了,\geng{点常去。
}还明明白白的,怎么就不中用了?”茗烟道:“我也不知道,才刚是他家的老头子特来告诉我的。
”宝玉听了,忙转身回明贾母。
贾母吩咐:“好生派妥当人跟去,到那里尽一尽同窗之情就回来,不许多耽搁了。
”宝玉听了,忙忙的更衣出来,车犹未备,\jia{顿一笔,方不板。
}急的满厅乱转。
一时催促的车到,忙上了车,李贵、茗烟等跟随。
来至秦钟门首,悄无一人,\jia{目睹萧条景况。
}遂蜂拥至内室,唬的秦钟的两个远房婶子并几个弟兄都藏之不迭。
\jia{妙!这婶母、兄弟是特来等分绝户家私的,不表可知。
}\par
此时,秦钟已发过两三次昏了,移床易箦多时矣。
\zhu{易箦:易:更换。
箦:音“责”,竹席。
孔子弟子曾参恪守礼制,病危时一定要人换掉大夫才能寝用的华美竹席(见《礼记·檀弓上》)。
后因称人之将死为“易箦”。
}宝玉一见,便不禁失声。
\jia{余亦欲哭。
}李贵忙劝道:“不可,不可,秦相公是弱症,未免炕上挺扛的骨头不受用,\geng{李贵亦能道此等语。
}所以暂且挪下来松散些。
哥儿如此,岂不反添了他的病。
”宝玉听了,方忍住。
近前见秦钟面如白蜡,宝玉叫道:“鲸兄!宝玉来了。
”连叫三声,秦钟不睬。
宝玉又道:“宝玉来了!”\par
那秦钟早已魂魄离身,只剩得一口悠悠馀气在胸,正见许多鬼判持牌提索来捉他。
\jia{看至此一句令人失望,再看至后面数语,方知作者故意借世俗愚谈愚论设譬,喝醒天下迷人,翻成千古未见之奇文奇笔。
}\geng{《石头记》一部中皆是近情近理必有之事,必有之言。
又如此等荒唐不经之谈,间亦有之,是作者故意游戏之笔,聊以破色取笑,
\zhu{破:解除、去除。色:生气发怒,改变脸色。}
非如别书认真说鬼话也。
}那秦钟魂魄那里就肯去,又记念着家中无人掌管家务,\jia{扯淡之极,令人发一大笑。
}\jia{余谓诸公莫笑,且请再思。
\ping{批书人在行间页眉互怼,何其相似于视频弹幕的互怼。
}}又记挂着父母还有留积下的三四千两银子,\jia{更属可笑,更可痛哭。
}\ping{第八回:“(秦业)只是宦囊羞涩,……说不得东拼西凑的恭恭敬敬封了二十四两贽见礼。
”从引文可见秦家并不富裕,与此处“三四千两银子”矛盾,可能是秦钟临死前已经糊涂了,或者是作者的笔误。
}又记挂着智能尚无下落,\jia{忽从死人心中补出活人原由,更奇更奇。
}
因此百般求告鬼判。
无奈这些鬼判都不肯徇私,反叱咤秦钟道:“亏你还是读过书的人,岂不知俗语说的:‘阎王叫你三更死,谁敢留你到五更。
’\geng{可想鬼不读书,信矣哉!}我们阴间,上下都是铁面无私的,不比你们阳间,瞻情顾意,\geng{写杀了。
}有许多的关碍处。
”\zhu{关碍:妨碍,阻碍。
}\par
正闹着,那秦钟的魂魄忽听见“宝玉来了”四字,又央求道:“列位神差,略发慈悲,让我回去,和这一个好朋友说一句话就来的。
”众鬼道:“又是什么好朋友?”秦钟道:“不瞒列位,就是荣国公孙子,小名宝玉的。
”都判官听了,先就唬慌起来,忙喝骂鬼使道:“我说你们放回了他去走走罢,你们断不依我的话,如今只等他请出个运旺时盛的人来才罢。
”\jia{如闻其声。
试问谁曾见都判来,观此则又见一都判跳出来。
调侃世情固深,然游戏笔墨一至于此,真可压倒古今小说。
}\jia{这才算是小说。
}众鬼见都判如此,也都忙了手脚,一面又抱怨道:“你老人家先是那等雷霆电雹,原来见不得‘宝玉’二字。
\jia{调侃“宝玉”二字,极妙![脂砚。
]}
\jia{世人见“宝玉”而不动心者为谁?}\chen{大可发笑。
}依我们愚见,他是阳间,我们是阴间,怕他也无益于我们。
”\jia{神鬼也讲有益无益。
}\lie{此章无非笑趋势之人。
}都判道:“放屁!俗语说的好,‘天下的官管天下的事’,阴阳本无二理\foot{庚本此句前另有“自古人鬼之道却是一般”一句,语意与本句重复,且前面众鬼也只说“阴间”、“阳间”,不提“人”“鬼”,则该语应系后人所增,或系批语混入正文。
}。
\ji{更妙!愈不通愈妙,愈错会意愈奇。
脂砚。
}别管他阴也罢,阳也罢,敬着点没错了的。
”\geng{名曰捣鬼。
}\ping{刚刚还说“我们阴间,上下都是铁面无私的,不比你们阳间,瞻情顾意”,这里岂不自己打自己的脸?}众鬼听说,只得将秦魂放回……哼了一声,微开双目,见宝玉在侧,乃勉强叹道:“怎么不肯早来?\geng{千言万语只此一句。
}再迟一步也不能见了。
”宝玉忙携手垂泪道:“有什么话,留下两句。
”\ji{只此句便足矣。
}秦钟道:“并无别话。
以前你我见识自为高过世人,我今日才知自误。
\ji{谁不悔迟!}以后还该立志功名,以荣耀显达为是。
”\geng{此刻无此二语,亦非玉兄之知己。
}
\geng{观者至此,必料秦钟另有异样奇语,然却只以此二语为嘱。
试思若不如此为嘱,不但不近人情,亦且太露穿凿。
读此则知全是悔迟之恨。
}\ping{只有立志功名,荣耀显达,才能摆脱对于家庭物质上的依赖,才有机会冲破封建家庭的束缚,自由恋爱。
秦钟临死之时,反思自己和智能横遭打断的感情,意识到这一点并规劝宝玉,和秦可卿临死前托梦王熙凤安排贾府退路一样,都是将死之人的大智慧大领悟,可惜王熙凤和贾宝玉都没有完全照做。
贾宝玉能有条件衣食无忧的谈恋爱,关键在于祖先靠“仕途经济学问”打下一份家业。
物质上依赖祖先的家业,必然在人生道路的选择包括婚姻上受制于家庭。
宝玉此后依旧没有改变,继续厌烦仕途经济学问,只知道躺在祖先的功劳簿上睡觉,为之后婚姻不得自由的悲剧埋下伏笔。
第七十一回:宝玉道:“……事事我常劝你,总别听那些俗语,想那俗事,只管安富尊荣才是。
……”尤氏道:“谁都像你,真是一心无挂碍,只知道和姊妹们顽笑,饿了吃,困了睡,再过几年,不过还是这样,一点后事也不虑。
”宝玉笑道:“我能够和姊妹们过一日是一日,死了就完了。
什么后事不后事。
”从引文可见,宝玉没有长远规划,只知坐享富贵。
第六十二回:黛玉道:“……咱们家里也太花费了。
我虽不管事,心里每常闲了,替你们一算计,出的多进的少,如今若不省俭,必致后手不接。
”宝玉笑道:“凭他怎么后手不接,也短不了咱们两个人的。
”黛玉听了,转身就往厅上寻宝钗说笑去了。
从引文可见,连知己黛玉都看不上宝玉,话不投机,转身离开。
}说毕,便长叹一声,萧然长逝。
\ji{若是细述一番,则不成《石头记》之文矣。
}下回分解。
\ping{本回名字“贾元春才选凤藻宫,秦鲸卿夭逝黄泉路”,一个人的极盛与一个人的夭折并列,繁华和凋零并列。
}\par
\qi{总评:大凡有势者未尝有意欺人。
然群小蜂起,浸润左右,伏首下气,奴颜婢膝,或激或顺,不计事之可否,以要一时之利。
有势者自任豪爽,斗露才华,未审利害,高下其手,偶有成就,一试再试,习以为常,则物理人情皆所不论。
又财货丰馀,衣食无忧,则所乐者必旷世所无。
要其必获,一笑百万,是所不惜。
其不知排场已立,收敛实难,从此勉强,至成蹇窘,\zhu{蹇:音“捡”,困苦,不顺利。
}时衰运败,百计颠翻。
昔年豪爽,今朝指背。
\zhu{指背:大致意思是衰败后被人远远地指着后背说闲话。
}此千古英雄同一慨叹者。
大抵作者发大慈大悲愿,欲诸公开巨眼,\zhu{巨眼:比喻善于鉴别的眼力。
}得见毫微,塞本穷源,\zhu{塞:可能是“溯”的错讹。
“塞(溯)本穷源”的意思是追寻源头。
}以成无碍极乐之至意也。
}
\dai{031}{赵嬷嬷陪贾琏凤姐吃饭谈及省亲之事}
\dai{032}{秦鲸卿夭逝黄泉路}
\sun{p16-1}{贾元春才选凤藻宫,林黛玉却赠鹡鸰香串}{图右侧:六宫都监夏守忠前来传旨,宣贾政入朝。
贾政等也猜不出是何来头,只得急忙更衣入朝,全家人都惶惶不安。
不久,赖大传来喜讯,原来是元春被封凤藻宫尚书,加封贤德妃,于是又转为皆大欢喜。
图左侧:贾琏与黛玉回京进府,见了黛玉,宝玉诚心诚意将北静王所赠鹡鸰香串送与黛玉,黛玉却不屑掷地,说道:“什么臭男人拿过的,我不要它!”}
\sun{p16-2}{王熙凤接风迎贾琏,秦鲸卿夭逝黄泉路}{图右侧:贾琏回家,凤姐少不得摆上酒馔接风。
贾琏奶母赵嬷嬷来求凤姐给两个儿子谋个差事。
原来元春省亲之事近日有了眉目,荣宁两府上下已忙成一片。
恰在这时,贾蓉、贾蔷过来回禀建造省亲别院事宜。
图左侧:宝玉听闻秦钟不中用了,忙乘车来至秦家,李贵、茗烟等跟随,蜂拥至内室,唬的秦钟的两个远房婶子并几个弟兄都藏之不迭。
那秦钟早已魂魄离身,正见许多鬼判持牌提索来捉他,都判官听闻宝玉来探视,遂喝骂鬼使,放秦钟魂魄回去见宝玉最后一面。
}