\chapter{大观园试才题对额\quad 荣国府归省庆元宵}
\qi{一物珍藏见至情,
\zhu{宝玉将黛玉所赠荷包珍藏在衣服里面的情节。}
豪华每向闹中争。
黛林宝薛传佳句,豪宴仙缘留趣名。
为剪荷包绾两意,\zhu{绾:音“碗”,系。
}屈从优女结三生。
\zhu{三生:指前生、今生和来生,这是佛教宣扬转世投胎的迷信说法。
这句说龄官与贾蔷暗生情愫。
}可怜转眼皆虚话,云自飘飘月自明。
}\par
[话说宝玉来]至院外\foot{按:己、庚本第十七至十八回未分回,其余诸本已分回,但位置不同。
此处依戚、蒙、列、杨等本分回,并补回首套语数字。
后人所拟回目实在太差,现试用一种新的方式。
},就有跟贾政的几个小厮上来拦腰抱住,都说:“今儿亏我们,老爷才喜欢,老太太打发人出来问了几遍,都亏我们回说喜欢;\geng{下人口气毕肖。
}不然,若老太太叫你进去,就不得展才了。
人人都说,你才那些诗比世人的都强。
今儿得了这样的彩头,\zhu{彩头:好运气;也指获得的奖品、赏物。
}该赏我们了。
”宝玉笑道:“每人一吊钱。
”众人道:“谁没见那一吊钱!\geng{钱亦有没用处。
}把这荷包赏了罢。
”说着,一个上来解荷包,那一个就解扇囊,不容分说,将宝玉所佩之物尽行解去。
又道:“好生送上去罢。
”一个抱了起来,几个围绕,送至贾母二门前。
\geng{好收煞。
}那时贾母已命人看了几次。
众奶娘丫鬟跟上来,见过贾母,知不曾难为着他,心中自是喜欢。
\par
少时袭人倒了茶来,见身边佩物一件无存,\geng{袭人在玉兄一身,无时不照察到。
}因笑道:“带的东西又是那起没脸的东西们解了去了。
”林黛玉听说,走来瞧瞧,果然一件无存,因向宝玉道:“我给你的那个荷包也给他们了?\geng{又起楼阁。
}你明儿再想我的东西,可不能够了!”说毕,赌气回房,将前日宝玉所烦他作的那个香袋儿——才做了一半——赌气拿过来就铰。
宝玉见他生气,便知不妥,忙赶过来,早剪破了。
宝玉已见过这香囊,虽尚未完,却十分精巧,费了许多工夫,今见无故剪了,却也可气。
因忙把衣领解了,从里面红袄襟上将黛玉所给的那荷包解了下来,递与黛玉瞧道:“你瞧瞧,这是什么!我那一回把你的东西给人了?”林黛玉见他如此珍重,带在里面,\ji{按理论之,则是“天下本无事,庸人自扰之”。
若以儿女子之情论之,则是必有之事,必有之理。
又系今古小说中不能写到写得,谈情者亦不能说出讲出,情痴之至文也!}可知是怕人拿去之意,因此又自悔莽撞,未见皂白就剪了香袋,\zhu{皂白:即青红皂白,比喻是非,情由等。
}\ji{情痴之至!若无此悔便是一庸俗小性之女子矣。
}因此又愧又气,低头一言不发。
宝玉道:“你也不用剪,我知道你是懒待给我东西。
我连这荷包奉还,何如?”说着,掷向他怀中便走。
\ji{这却难怪。
}黛玉见如此,越发气起来,声咽气堵,又汪汪的滚下泪来,\ji{怒之极正是情之极。
}拿起荷包来又剪。
宝玉见他如此,忙回身抢住,笑道:“好妹妹,饶了他罢!”\ji{这方是宝玉。
}黛玉将剪子一摔,拭泪说道:“你不用同我好一阵歹一阵的,要恼,就撂开手。
这当了什么!”说着,赌气上床,面向里倒下拭泪。
禁不住宝玉上来“妹妹”长“妹妹”短赔不是。
\par
前面贾母一片声找宝玉。
众奶娘丫鬟们忙回说:“在林姑娘房里呢。
”贾母听说道:“好,好,好!让他姊妹们一处顽顽罢。
才他老子拘了他这半天,让他开心一会子罢。
只别叫他们拌嘴,不许扭了他。
”众人答应着。
黛玉被宝玉缠不过,只得起来道:“你的意思不叫我安生,我就离了你。
”说着往外就走。
宝玉笑道:“你到那里,我跟到那里。
”一面仍拿起荷包来带上。
黛玉伸手抢道:“你说不要了,这会子又带上,我也替你怪臊的!”说着,嗤的一声笑了。
宝玉道:“好妹妹,明儿另替我作个香袋儿罢。
”黛玉道:“那也只瞧我高兴罢了。
”一面说,一面二人出房,到王夫人上房中去了,\ji{一段点过近日二玉公案,断不可少。
}
可巧宝钗亦在那里。
\par
此时王夫人那边热闹非常。
\ji{四字特补近日千忙万冗,多少花团锦簇文字。
}原来贾蔷已从姑苏采买了十二个女孩子,并聘了教习,以及行头等事来了。
\zhu{行头:演戏所用的服装、道具等。}
那时薛姨妈另迁于东北上一所幽静房舍居住,将梨香院早已腾挪出来,
\zhu{梨香院:
梨园位于长安宫廷禁苑梨园旁,故得名梨园。
玄宗曾选坐部伎子弟三百人和宫女数百人于此学歌舞。后梨园泛称戏班或演戏之所。
}
另行修理了,就令教习在此教演女戏。
又另派家中旧有曾演学过歌唱的众女人们,如今皆已皤然老妪了,\zhu{
皤:音“婆”,白。
皤然:头发银白的样子。
}\ji{又补出当日宁、荣在世之事,所谓此是末世之时也。
}着他们带领管理。
就令贾蔷总理其日用出入银钱等事,以及诸凡大小所需之物料帐目。
\ji{补出女戏一段,又伏一案。
\zhu{伏一案:指贾蔷和龄官的情缘。}
}又有林之孝家的来回:“采访聘买得十个小尼姑、小道姑都有了,连新作的二十分道袍也有了。
外有一个带发修行的,本是苏州人氏,\ping{妙玉乃苏州人士,巧的是黛玉、香菱也来自苏州。
}祖上也是读书仕宦之家。
因生了这位姑娘自小多病,买了许多替身儿皆不中用,\zhu{替身儿:迷信习俗以为命中有灾难的人,可以用舍身出家做僧尼的办法来消灾。
官僚、地主家庭往往买穷人家子女代替出家,叫做“替身”。
第三回:黛玉介绍自己病情时,说:“那一年我才三岁时,听得说来了一个癞头和尚,说要化我去出家,我父母固是不从。他又说:‘既舍不得他,只怕他的病一生也不能好的了。”
由此可见,因病而自己出家或者让别人代替出家,在当时很常见。
}足的这位姑娘亲自入了空门,\zhu{足的:到底。
}方才好了,所以带发修行,今年才十八岁,法名妙玉。
\geng{妙玉世外人也,故笔笔带写,妙极妥极!畸笏。
}\ji{妙卿出现。
至此细数十二钗,以贾家四艳再加薛林二冠有六,添秦可卿有七,熙凤有八,李纨有九,今又加妙玉,仅得十人矣。
后有史湘云与熙凤之女巧姐儿者,共十二人,雪芹题曰“金陵十二钗”,盖本宗《红楼梦》十二曲之义。
后宝琴、岫烟、李纹、李绮皆陪客也,《红楼梦》中所谓副十二钗是也。
又有又副册三段词,乃晴雯、袭人、香菱三人而已,馀未多及,想为金钏、玉钏、鸳鸯、茜雪、平儿等人无疑矣。
观者不待言可知,故不必多费笔墨。
}\geng{\sout{树处}[副册]引十二钗总未的确,皆系漫拟也。
至末回警幻情榜,方知正副、再副及三四副芳讳。
壬午季春。
畸笏。
}如今父母俱已亡故,身边只有两个老嬷嬷,一个小丫头伏侍。
文墨也极通,经文也不用学了,模样儿又极好。
因听见长安都中有观音遗迹并贝叶遗文,\zhu{贝叶遗文:古代写在贝叶上的佛经。
贝叶:贝多树的叶子。
古时印度僧人多用以写佛教经文。
}去岁随了师父上来,\ji{因此方使妙卿入都。
}现在西门外牟尼院住着。
他师父极精演先天神数,\zhu{先天神数:北宋理学家邵雍,根据《易传》关于八卦形成的解释,参杂道教思想,虚构了一个世界构造的图式,叫“先天八卦图”,用以推测宇宙和人事的变化。
据说这种图式和所据的“象数”原理,在没有天地以前,就已存在,故其学称“先天学”。
“先天神数”即指此类学说。
}
于去冬圆寂了。
\zhu{圆寂:佛教用语。
一作“灭度”,梵文“涅槃”的意译。
原意是佛教所说的烦恼寂灭、功德圆满的最高境界。
后用来称佛或僧侣的逝世。
}妙玉本欲扶灵回乡的,他师父临寂遗言,说他‘衣食起居不宜回乡,在此静居,后来自然有你的结果’。
所以他竟未回。
”王夫人不等回完,便说:“既这样,我们何不接了他来。
”林之孝家的回道:“请他,他说:‘侯门公府,必以贵势压人,我再不去的。
’”\ji{补出妙卿身世不凡,心性高洁。
}王夫人笑道:“他既是官宦小姐,自然骄傲些,就下个帖子请他何妨。
”林之孝家的答应了出去,命书启相公写请帖去请妙玉。
\zhu{书启相公:旧时官署中掌管书牍的人。
}次日遣人备车轿去接等后话,暂且搁过,此时不能表白。
\zhu{表白:向人说明、解释。}
\ji{补尼道一段,又伏一案\foot{按:甲辰本及程甲、乙本在此分回。
}\zhu{伏一案:妙玉出场,引出后文妙玉和贾府的互动。}。
}\par
当下又有人回,工程上等着糊东西的纱绫,请凤姐去开楼拣纱绫;又有人来回,请凤姐开库,收金银器皿。
连王夫人并上房丫鬟等众,皆一时不得闲的。
宝钗便说:“咱们别在这里碍手碍脚,找探丫头去。
”说着,同宝玉黛玉往迎春等房中来闲顽,无话。
\par
王夫人等日日忙乱,直到十月将尽,幸皆全备:各处监管都交清帐目;各处古董文玩,皆已陈设齐备;采办鸟雀的,自仙鹤、孔雀以及鹿、兔、鸡、鹅等类,悉已买全,交于园中各处像景饲养;\zhu{像景:适应景物。
}贾蔷那边也演出二十出杂戏来;小尼姑、道姑也都学会了念几卷经咒。
贾政方略心意宽畅,\ji{好极!可见智者居心无一时弛怠!}又请贾母等进园,色色斟酌,\zhu{色色:样样。
}点缀妥当,再无一些遗漏不当之处了。
于是贾政方择日题本。
\zhu{题本:明清时,各衙门用正式文书向皇帝奏事叫题本;非正式的叫奏本。
本;臣下奏事的文书。
}\ji{至此方完大观园工程公案,观者则为大观园费尽精神,余则为若许笔墨却只因一个葬花冢。
}
本上之日,奉朱批准奏:次年正月十五日上元之日,恩准贵妃省亲。
贾府领了此恩旨,益发昼夜不闲,年也不曾好生过的。
\ji{一语带过。
是以“岁首祭宗祀,元宵开夜宴”一回留在后文细写。
}\par
展眼元宵在迩,自正月初八日,就有太监出来先看方向:何处更衣,何处燕坐,\zhu{燕坐:闲坐。
燕:安闲。
}何处受礼,何处开宴,何处退息。
\zhu{退息:从某处退下来休息。}
又有巡察地方总理关防太监等,\zhu{关防:出于礼制和保安的需要而采取的分隔内外的措施。
清代内务府有执掌“关防”的机构和人员。
}带了许多小太监出来,各处关防,挡围幕,指示贾宅人员何处退,何处跪,何处进膳,何处启事,种种仪注不一。
\zhu{仪注:礼仪制度和规范。
}外面又有工部官员并五城兵备道打扫街道,撵逐闲人。
贾赦等督率匠人扎花灯烟火之类,至十四日,俱已停妥。
这一夜,上下通不曾睡。
\par
至十五日五鼓,自贾母等有爵者,俱各按品服大妆。
园内各处,帐舞蟠龙,帘飞彩凤,金银焕彩,珠宝争辉,\ji{是元宵之夕,不写灯月而灯光月色满纸矣。
}鼎焚百合之香,瓶插长春之蕊,\ji{抵一篇大赋。
}静悄无人咳嗽。
\ji{有此句方足。
}贾赦等在西街门外,贾母等在荣府大门外。
街头巷口,俱系围幕挡严。
正等的不耐烦,忽一太监坐大马而来,\ji{有是礼。
}贾母忙接入,问其消息。
太监道:“早多着呢!未初刻用过晚膳,
\zhu{未初刻:下午一点。}
未正二刻还到宝灵宫拜佛,\ji{暗贴王夫人,细。
}
\zhu{未正二刻:下午两点半。}
酉初刻进大明宫领宴看灯方请旨,
\zhu{酉初刻:下午五点。}
只怕戌初才起身呢。
\zhu{戌初:晚上七点。}
”凤姐听了道:\geng{自然当家人先说话。
}
“既是这么着,老太太、太太且请回房,等是时候再来也不迟。
”于是贾母等暂且自便,园中悉赖凤姐照理。
又命执事人带领太监们去吃酒饭。
\par
一时传人一担一担的挑进蜡烛来,各处点灯。
方点完时,忽听外边马跑之声。
\ji{静极故闻之。
细极。
}一时,有十来个太监都喘吁吁跑来拍手儿。
\ji{画出内家风范。
《石头记》最难之处别书中摸不着。
}这些太监会意,\geng{难得他写的出,是经过之人也。
}都知道是“来了,来了”,各按方向站住。
贾赦领合族子侄在西街门外,贾母领合族女眷在大门外迎接。
半日静悄悄的。
忽见一对红衣太监骑马缓缓的走来,\ji{形容毕肖。
}至西街门下了马,将马赶出围幕之外,便垂手面西站住。
\ji{形容毕肖。
}半日又是一对,亦是如此。
少时便来了十来对,\ping{这里的太监让我想起每次阅兵最开始给阅兵队伍定位的标兵。
}方闻得隐隐细乐之声。
一对对龙旌凤翣,雉羽夔头,\zhu{
旌[jīng]:古代指一种旗帜,旗杆顶上有牦牛尾或五色羽毛作为装饰。
翣:音“煞”,用野鸡或孔雀羽毛编成的大掌扇。
雉:野鸡。
夔:音“葵”,古代传说中灵异动物。
龙旌凤翣,雉羽夔头:帝后的仪仗用物。
}又有销金提炉焚着御香;然后一把曲柄七凤黄金伞过来,便是冠袍带履。
又有随事太监捧着香珠、绣帕、漱盂、拂尘等类。
\zhu{拂尘:形如马尾,后有持柄,用以拂拭尘土,或驱赶蝇蚊,俗称“蝇甩子”。
古时多用麈(麈:音“主”,一种似骆驼的鹿类动物)兽之尾制成,故又称麈尾。
}
一队队过完,后面方是八个太监抬着一顶金顶金黄绣凤版舆,\zhu{舆:车。
版舆:指皇室仪制中贵妃所坐的轿,木质,漆成金黄。
}缓缓行来。
贾母等连忙路旁跪下。
\geng{一丝不乱。
}早飞跑过几个太监来,扶起贾母、邢夫人、王夫人来。
那版舆抬进大门、入仪门往东,
\zhu{
仪门:旧时官衙、府第的大门之内的门,具装饰作用。
一说,旁门也可称仪门。
}
去到一所院落门前,有执拂太监跪请下舆更衣。
\zhu{拂:拂尘。
}于是抬舆入门,太监等散去,只有昭容、彩嫔等引领元春下舆。
只见院内各色花灯熌灼,\geng{元春目中。
}\zhu{熌:同“闪”。
}皆系纱绫扎成,精致非常。
上面有一匾灯,
\zhu{匾灯:匾额腔体内部设置灯烛,使得能在夜里看清题写之字。}
写着“体仁沐德”四字。
元春入室,更衣毕复出,上舆进园。
只见园中香烟缭绕,花彩缤纷,处处灯光相映,时时细乐声喧,说不尽这太平景象、富贵风流。
此时自己回想当初在大荒山中,青埂峰下,那等凄凉寂寞;若不亏癞僧、跛道二人携来到此,又安能得见这般世面。
本欲作一篇《灯月赋》、《省亲颂》,以志今日之事,但又恐入了别书的俗套。
按此时之景,即作一赋一赞,也不能形容得尽其妙;即不作赋赞,其豪华富丽,观者诸公亦可想而知矣。
所以倒是省了这工夫纸墨,且说正紧的为是。
\ji{自“此时”以下皆石头之语,真是千奇百怪之文。
}\geng{如此繁华盛极、花团锦簇之文,忽用石兄自语截住,是何笔力!令人安得不拍案叫绝。
试阅历来诸小说中有如此章法乎?}\par
且说贾妃在轿内看此园内外如此豪华,因默默叹息奢华过费。
忽又见执拂太监跪请登舟。
贾妃乃下舆。
只见清流一带,势如游龙,两边石栏上,皆系水晶玻璃各色风灯,\zhu{风灯:有罩能防风的灯。
}点的如银光雪浪;上面柳杏诸树虽无花叶,然皆用通草绸绫纸绢依势作成,\zhu{通草:即通脱木,五加科小乔木。
}粘于枝上的,每一株悬灯数盏;更兼池中荷荇凫鹭之属,亦皆系螺蚌羽毛之类作就的。
诸灯上下争辉,真系玻璃世界,珠宝乾坤。
船上亦系各种精致盆景诸灯,珠帘绣幕,桂楫兰桡,\zhu{
楫(音“集”)、桡(音“饶”)都是船桨。
桂、兰都是香木。
桂楫兰桡:指华美的船只。
}自不必说。
已而入一石港,
\zhu{港:这里读作“哄”四声,指桥下涵洞。}
港上一面匾灯,明现着“蓼汀花溆”四字。
按此四字,并“有凤来仪”等处,皆系上回贾政偶然一试宝玉之课艺才情耳,何今日认真用此匾联?况贾政世代诗书,来往诸客屏侍坐陪者,\zhu{屏侍坐陪:在屏风前和座位上侍候陪伴。
}悉皆才技之流,岂无一名手题撰,竟用小儿一戏之辞苟且搪塞?\geng{驳得好!}真似暴发新荣之家,滥使银钱,一味抹油涂朱,毕则大书“前门绿柳垂金锁,后户青山列锦屏”之类,则以为大雅可观,岂《石头记》中通部所表之宁荣贾府所为哉!据此论之,竟大相矛盾了。
诸公不知,待蠢物\ji{石兄自谦,妙!可代答云“岂敢!”}将原委说明,大家方知。
\geng{《石头记》惯用特犯不犯之笔,读之真令人惊心骇目。
}\par
当日这贾妃未入宫时,自幼亦系贾母教养。
后来添了宝玉,贾妃乃长姊,宝玉为弱弟,贾妃之心上念母年将迈,始得此弟,是以怜爱宝玉,与诸弟待之不同。
且同随贾母,刻未暂离。
那宝玉未入学堂之先,三四岁时,已得贾妃手引口传,教授了几本书、数千字在腹内了。
\geng{批书人领过此教,故批至此竟放声大哭,俺先姊仙逝太早,不然余何得为废人耶?}其名分虽系姊弟,其情状有如母子。
自入宫后,时时带信出来与父母说:“千万好生扶养,不严不能成器,过严恐生不虞,且致父母之忧。
”眷念切爱之心,
\zhu{切[qiè]:两物相磨。引申为贴近、接近。}
刻未能忘。
前日贾政闻塾师背后赞宝玉偏才尽有,贾政未信,适巧遇园已落成,令其题撰,聊一试其情思之清浊。
其所拟之匾联虽非妙句,在幼童为之,亦或可取。
即另使名公大笔为之,固不费难,然想来倒不如这本家风味有趣。
\geng{转得好。
}更使贾妃见之,知系其爱弟所为,亦或不负其素日切望之意。
\ji{一驳一解,跌宕摇曳之至,且写得父母兄弟体贴恋爱之情,淋漓痛切,真是天伦至情。
}\geng{有是论。
}因有这段原委,故此竟用了宝玉所题之联额。
那日虽未曾题完,后来亦曾补拟。
\ji{一句补前文之不暇,启后文之苗裔。
\zhu{苗裔:后代。
}至后文凹晶馆黛玉口中又一补,\zhu{第七十六回,黛玉和湘云凹晶馆联诗。}所谓“一击空谷,八方皆应”。
}\par
闲文少叙,且说贾妃看了四字,笑道:“‘花溆’二字便妥,何必‘蓼汀’?”
\ping{
元春为何去掉“蓼汀花溆”中的“蓼汀”?
说法一:“蓼汀”取自唐诗“暮天新雁起汀洲,红蓼花开水国愁”,红花开放却凄凉萧瑟。“蓼汀”二字太过凄凉,所以元春删去。
说法二:这是因为“花溆”的“溆”字,其形似“钗”,其音似“薛”;而“蓼汀”二字反切就是“林”字。
\zhu{反切:我国传统的注音方法,用两个字的音拼合出另一个字的音,上字取声母,下字取韵母和声调。
如用“多贡”反切,即取“多”的声母d,“贡”的韵母和声调 òng,就能拼合成“栋”的读音 dòng。也说切或反。
}
暗含元春对于宝钗和黛玉的不同态度。
}
侍座太监听了,忙下小舟登岸,飞传与贾政。
贾政听了,即忙移换。
\ji{换的周到可悦。
}一时,舟临内岸,复弃舟上舆,便见琳宫绰约,\zhu{绰约:美丽多姿。
}桂殿巍峨。
石牌坊上明显“天仙宝境”四字,\ji{不得不用俗。
}
贾妃忙命换“省亲别墅”四字。
\ji{妙!是特留此四字与彼自命。
}于是进入行宫。
\zhu{行宫:古代皇帝后妃外出,临时下榻之处。
}但见庭燎烧空,\zhu{庭燎:古代贵族庭院中用以照明的大烛,用松、竹、苇等捆扎成束,灌以油脂。
}\ji{“庭燎”最恰。
}香屑布地,火树琪花,\zhu{火树琪花:形容灯火之盛。
琪:美玉。
}金窗玉槛。
说不尽帘卷虾须,毯铺鱼獭,\zhu{虾须帘:用虾须做成的帘子。
一说以细竹丝编制的帘子为虾须帘。
鱼獭[tǎ]:即水獭,皮毛珍贵。
鱼獭毯:用水獭皮做的毯子。
}鼎飘麝脑之香,屏列雉尾之扇。
真是:\par
\hop
金门玉户神仙府,桂殿兰宫妃子家。
\par
\hop
贾妃乃问:“此殿何无匾额?”随侍太监跪启曰:“此系正殿,外臣未敢擅拟。
”贾妃点头不语。
礼仪太监跪请升座受礼,两陛乐起。
\zhu{陛[bì]:台阶;特指帝王宫殿的台阶。}
礼仪太监二人引贾赦、贾政等于月台下排班,\zhu{月台:古代建筑正殿前的露天平台,三面有台阶可上。
}殿上昭容传谕曰:“免。
”太监引贾赦等退出。
又有太监引荣国太君及女眷等自东阶升月台上排班,\zhu{太君:古代官员的母亲的封号;后用来尊称对方的母亲。}
\ji{一丝不乱,精致大方。
有如欧阳公九九。
\zhu{
有如欧阳公九九:“九九”或是九成误笔,《九成宫醴泉铭》乃欧阳询楷书最具代表性作品,被后世誉为“天下第一楷书”。
欧阳询的正楷骨气劲峭,法度严整,此或是以欧公书法严整喻小说描写井然有序。
}}昭容再谕曰:“免。
”于是引退。
\par
\chai{yuanchun}{元春归省}
茶已三献,贾妃降座,乐止。
退入侧殿更衣,方备省亲车驾出园。
至贾母正室,欲行家礼,贾母等俱跪止不迭。
贾妃满眼垂泪,方彼此上前厮见,一手搀贾母,一手搀王夫人,三个人满心里皆有许多话,只是俱说不出,只管呜咽对泣。
\ji{《石头记》得力擅长全是此等地方。
}\geng{非经历过如何写得出!壬午春。
}邢夫人、李纨、王熙凤、迎、探、惜三姊妹等,俱在旁围绕,垂泪无言。
半日,贾妃方忍悲强笑,安慰贾母、王夫人道:“当日既送我到那不得见人的去处,好容易今日回家娘儿们一会,不说说笑笑,反倒哭起来。
一会子我去了,又不知多早晚才来!”\ping{前八十回贾元春没有再回来过,后八十回贾府衰落迹象明显,应该更不可能了。
}说到这句,不禁又哽咽起来。
\ji{追魂摄魄。
《石头记》传神摹影全在此等地方,他书中不得有此见识。
}邢夫人等忙上来解劝。
\ji{说完不可,不先说不可,说之不痛不可,最难说者是此时贾妃口中之语。
只如此一说,方千贴万妥,一字不可更改,一字不可增减,入情入神之至!}贾母等让贾妃归座,又逐次一一见过,又不免哭泣一番。
然后东西两府掌家执事人丁等在厅外行礼,及两府掌家执事媳妇领丫鬟等行礼毕。
贾妃因问:“薛姨妈、宝钗、黛玉因何不见?”\chen{谅前信息皆知,故有此问。
}王夫人启曰:“外眷无职,未敢擅入。
”\ji{所谓诗书世家,守礼如此。
偏是暴发,骄妄自大。
}贾妃听了,忙命快请。
\ji{又谦之如此,真是世界好人物。
}一时薛姨妈等进来,欲行国礼,亦命免过,上前各叙阔别寒温。
又有贾妃原带进宫去的丫鬟抱琴等\ji{前所谓贾家四钗之环,暗以琴棋书画排行,至此始全。
\zhu{元春的抱琴、迎春的司棋、探春的侍书、惜春的入画。}
}上来叩见,贾母等连忙扶起,命人别室款待。
执事太监及彩嫔、昭容各侍从人等,宁国府及贾赦那宅两处自有人款待,只留三四个小太监答应。
母女姊妹深叙些离别情景,\ji{“深”字妙!}及家务私情。
\par
又有贾政至帘外问安,贾妃垂帘行参等事。
\zhu{参:古代下级见上级叫参。
这里贾妃是上级,而贾政是下级。
}又隔帘含泪谓其父曰:“田舍之家,虽齑盐布帛,\zhu{齑(齑音“基”)盐布帛:形容生活的清苦。
齑:切碎的腌莱。
齑盐:泛指粗茶淡饭。
布帛:原指棉织品和丝织品,这里泛指素衣布裳。
}终能聚天伦之乐;今虽富贵已极,骨肉各方,然终无意趣!”\ping{贾元春前面已经说了“当日既送我到那不得见人的去处”,这时又说“终无意趣”,可见其对于宫廷生活的排斥,只是不得已为了贾府的未来牺牲了自己的幸福。
但是说话的时候,宫里的太监在侧,贾元春难道不怕这些话传到皇帝的耳朵里吗?我觉得可能的解释是,第十六回通过贾琏的口说到“如今当今体贴万人之心,世上至大莫如‘孝’字”,在以孝治国的局势下,贾元春这样的话语是“孝”的表现,是被当今皇帝所认可的,允许嫔妃回家省亲也是皇帝彰显孝道的行为。
皇帝希望臣民把对于父母的“孝”升华为对于皇帝的“忠”,自古忠孝不能两全,皇帝更看重的是在忠孝的矛盾下,毅然决然选择忠的行为,这样才能体现忠的可贵,而不是完全不考虑孝,这样的没有人性的忠是廉价的,更可能是别有所图的伪忠。
《韩非子·难一》:管仲有病,桓公往问之,曰:“仲父病,不幸卒于大命,将奚以告寡人?”管仲曰:“微君言,臣故将谒之。
愿君去竖刁,除易牙,远卫公子开方。
易牙为君主味,惟人肉未尝,易牙烝其子首而进之。
夫人情莫不爱其子,今弗爱其子,安能爱君?君妒而好内,竖刁自宫以治内。
人情莫不爱其身,身且不爱,安能爱君?开方事君十五年,齐、卫之间不容数日行,弃其母,久宦不归。
其母不爱,安能爱君?臣闻之:‘矜伪不长,盖虚不久。
'愿君去此三子者也。
”管仲卒死,桓公弗行。
及桓公死,虫出户不葬。
}贾政亦含泪启道:“臣草莽寒门,鸠群鸦属之中,岂意得征凤鸾之瑞。
\zhu{征凤鸾之瑞:意谓出现了能呈祥瑞的鸾凤。
征:迹象;证验。
凤鸾:喻指元春。
}\geng{此语犹在耳。
}今贵人上锡天恩,\zhu{锡:音“次”,赐。
}下昭祖德,此皆山川日月之精奇、祖宗之远德钟于一人,幸及政夫妇。
且今上启天地生物之大德,垂古今未有之旷恩,虽肝脑涂地,臣子岂能得报于万一!惟朝乾夕惕,\zhu{朝乾夕惕:从早到晚兢兢业业,不敢稍有懈怠。
乾:音“前”,“乾乾”的简省,自强不息的意思。
惕:音“替”,小心谨慎。
《易·乾》:“君子终日乾乾,夕惕若厉,无咎。
”}忠于厥职外,\zhu{厥:音“掘”,其,相当于“他的”、“那个”。
}愿我君万寿千秋,乃天下苍生之同幸也。
贵妃切勿以政夫妇残犁为念,\zhu{“残犁”二字,无法理解,应为“残年”之错别字。
庚辰本和己卯本皆作“残犁”,戚序本、蒙古王府本作“残黎”,甲辰本为“残年”。
考之,古代似无“残犁”之说法;“残黎”倒是一个常用词,“黎”是黎民百姓的意思,“残黎”常用来形容在经历国家大灾大难之后侥幸残留下来的人们,如《明史·熊廷弼传》云:“不然,支撑宁、前、锦、义间,扶伤救败,收拾残黎,犹可图桑榆之效。
”将“残黎”一词用在元春省亲时贾政跟元春的对话中,不仅不合情理,甚至有些反动。
唯有“残年”一词用在此处可以说得通。
“犁”字可能是“年”的古体字“秊”之讹误。
}懑愤金怀,\zhu{懑(懑音“闷”)愤金怀:心里烦闷郁愤的意思。
“金”是表示尊重的修饰词。
}更祈自加珍爱。
惟业业兢兢,勤慎恭肃以侍上,庶不负上体贴眷爱如此之隆恩也。
”贾妃亦嘱“只以国事为重,暇时保养,切勿记念”等语。
贾政又启:“园中所有亭台轩馆,皆系宝玉所题。
如果有一二稍可寓目者,\zhu{寓目:亲眼看一看。
}请别赐名为幸。
”元妃听了宝玉能题,便含笑说:“果进益了。
”贾政退出。
贾妃见宝、林二人亦发比别姊妹不同,真是姣花软玉一般。
因问:“宝玉为何不进见?”\ji{至此方出宝玉。
}贾母乃启:“无谕,外男不敢擅入。
”\ping{弟弟原是算外男?}元妃命快引进来。
小太监出去引宝玉进来,先行国礼毕,元妃命他进前,携手揽于怀内,又抚其头颈,\geng{作书人将批书人哭坏了。
}笑道:“比先竟长了好些……”一语未终,泪如雨下。
\ji{只此一句便补足前面许多文字。
}\par
尤氏、凤姐等上来启道:“筵宴齐备,请贵妃游幸。
”元妃等起身,命宝玉导引,遂同诸人步至园门前。
早见灯光火树之中,诸般罗列非常。
进园来先从“有凤来仪”、“红香绿玉”、“杏帘在望”、“蘅芷清芬”等处,登楼步阁,涉水缘山,百般眺览徘徊。
一处处铺陈不一,一桩桩点缀新奇。
贾妃极加奖赞,又劝:“以后不可太奢,此皆过分之极。
”已而至正殿,谕免礼归座,大开筵宴。
贾母等在下相陪,尤氏、李纨、凤姐等亲捧羹把盏。
\par
元妃乃命传笔砚伺候,亲搦湘管,\zhu{搦:音“诺”,握,拿。
湘管:毛笔。
}择其几处最喜者赐名。
按其书云:\par
\hop
“顾恩思义”\quad{\footnotesize 匾额}\par
天地启宏慈,赤子苍头同感戴;\zhu{赤子苍头:泛指老百姓。
赤子:指初生的婴儿。
苍头:原指老年的奴仆,这里指老年人。
}\par
古今垂旷典,\zhu{旷典:空前的大恩典。
}九州万国被恩荣。
{\footnotesize 此一匾一联书于正殿。
}\ji{是贵妃口气。
}\par
“大观园”\quad{\footnotesize 园之名}\par
“有凤来仪”赐名曰“潇湘馆”。
\zhu{
此处为林黛玉未来所居之地。有凤来仪:赞黛玉是“人中之凤”。
潇湘:用湘妃哭舜,泪染斑竹的典故,结合黛玉“潇湘妃子”之别号,可知暗示黛玉为宝玉哭干眼泪。
}
\par
“红香绿玉”改作“怡红快绿”。
即名曰“怡红院”。
\par
“蘅芷清芬”赐名曰“蘅芜苑”。
\zhu{
此处为未来薛宝钗所居之地。
蘅芜苑:谐音“恨无缘”、“恒无缘”。蘅芷清芬:谐音“恒止情分”。
此处之命名,可能暗示宝钗和宝玉婚姻的结局。
这里用楚辞“美人香草”(以咏赞香草与美人以寄托作者之理想与感慨)的象征手法来衬托、象征宝钗的性格,用这些香草为宝钗安排一个生活居住环境。
}
\par
“杏帘在望”赐名曰“浣葛山庄”。
\zhu{
杏帘在望:谐音“幸怜在望”,即后妃们所盼望的“宠幸”。
“浣葛山庄”则是古拙而远离政治意味的山野民间称谓。
}
\par
\hop
正楼曰“大观楼”,东面飞楼曰“缀锦阁”,西面斜楼曰“含芳阁”;更有“蓼风轩”、“藕香榭”、\ji{雅而新。
}“紫菱洲”、“荇叶渚”等名;又有四字的匾额十数个,诸如“梨花春雨”、“桐剪秋风”、“荻芦夜雪”等名,此时悉难全记。
\ji{故意留下秋爽斋、凸碧山堂、凹晶溪馆、暖香坞等处为后文另换眼目之地步。
}又命旧有匾联俱不必摘去。
于是先题一绝云:\par
\hop
衔山抱水建来精,多少工夫筑始成。
\par
天上人间诸景备,芳园应锡大观名。
\zhu{锡:音“次”,赐。
}\ji{诗却平平,盖彼不长于此也,故只如此。
}\par	
\hop
写毕,向诸姐妹笑道:“我素乏捷才,且不长于吟咏,妹辈素所深知。
今夜聊以塞责,不负斯景而已。
异日少暇,必补撰《大观园记》并《省亲颂》等文,以记今日之事。
妹辈亦各题一匾一诗,随才之长短,亦暂吟成,不可因我微才所缚。
且喜宝玉竟知题咏,是我意外之想。
此中‘潇湘馆’、‘蘅芜苑’二处,我所极爱,次之‘怡红院’、‘浣葛山庄’,此四大处,必得别有章句题咏方妙。
前所题之联虽佳,如今再各赋五言律一首,使我当面试过,方不负我自幼教授之苦心。
”宝玉只得答应了,下来自去构思。
\par
迎、探、惜三人之中,要算探春又出于姊妹之上,然自忖亦难与薛林争衡,\ji{只一语便写出宝黛二人,又写出探卿知己知彼,伏下后文多少地步。
}
只得勉强随众塞责而已。
李纨也勉强凑成一律。
\ji{不表薛、林可知。
}贾妃先挨次看姊妹们的,写道是:\par
\hop
旷性怡情\quad {\footnotesize 匾额}\quad 迎 春\par
园成景备特精奇,奉命羞题额旷怡。
\par
谁信世间有此境,游来宁不畅神思?\zhu{宁不:怎不。
}\par
\hop
万象争辉\quad {\footnotesize 匾额}\quad 探 春\par
名园筑出势巍巍,奉命何惭学浅微。
\par
精妙一时言不出,果然万物生光辉。
\par
\hop
文章造化\zhu{文章造化:指大观园的精巧华美巧夺天工。
文章:文采;花纹。
造化:创造化育万物的自然界。
}\quad {\footnotesize 匾额}\quad 惜 春\par
山水横拖千里外,楼台高起五云中。
\zhu{横拖:横亘延伸。
}\par
园修日月光辉里,景夺文章造化功。
\zhu{日月光辉:喻皇帝后妃的恩泽有如日月的光辉。
}\ji{更牵强。
三首之中还算探卿略有作意,故后文写出许多意外妙文。
}\par
\hop
文采风流\quad {\footnotesize 匾额}\quad 李 纨\par
秀水明山抱复回,风流文采胜蓬莱。
\ji{起好!}\zhu{蓬莱:仙山名。
《史记·秦始皇本纪》:“海中有三神山,名曰蓬莱、方丈、瀛洲。
”《汉书·郊祀志》:“此三神山者,其传在渤海中,……诸仙人及不死之药皆在焉。
其物禽兽尽白,而黄金白银为宫阙。
”}\par
绿裁歌扇迷芳草,红衬湘裙舞落梅。
\ji{凑成。
}\zhu{这一联说绿绢裁制的歌扇同碧草一色,迷离难分;红装衬着湘裙舞动如梅花落瓣,随风飞回。
歌扇:古时女子歌唱常以扇遮面,故称歌扇。
湘裙:疑为“缃[xiāng]裙”,即浅黄色细绢所制之裙;一说为湘绣所做的裙子。
}\par
珠玉自应传盛世,神仙何幸下瑶台。
\zhu{珠玉:喻精采的诗文。
这里借指大观园题咏。
瑶台:神仙所居之处,这里代指皇宫。
“神仙何幸下瑶台”喻元春归省。
}\par
名园一自邀游赏,未许凡人到此来。
\zhu{邀:迎候。
}\ji{此四诗列于前,正为滃托下韵也。
}\zhu{滃:音“翁”,三声,云气腾涌,水大的样子。
滃托:烘托。
}\par
\hop
凝晖钟瑞\zhu{凝晖钟瑞:阳光瑞气凝集会聚之意。
晖:日光,喻皇恩。
钟:聚。
}\quad {\footnotesize 匾额}\ji{便有含蓄。
 }\quad 薛宝钗\par
芳园筑向帝城西,华日祥云笼罩奇。
\par
高柳喜迁莺出谷,修篁时待凤来仪。
\ji{恰极!}\zhu{这一联上句说高高的柳树欢迎黄莺从幽谷中飞来。
化用《诗·小雅·伐木》:“伐木丁丁,鸟鸣嘤嘤,出自幽谷,迁于乔木。
”这里喻元春出深闺进宫闱。
下句说修长的丛竹时刻等待凤凰的到临。
传说凤凰食竹实,呈祥瑞。
这里喻元春归省。
篁:音“黄”,竹林。
修:长。
}\par
文风已著宸游夕,孝化应隆归省时。
\zhu{这一联说宸游之夕朝廷倡导诗礼之风已经彰著,归省之时以孝教育感化万民之德更加隆盛。
著:表现得显著。
宸游:皇帝后妃出外巡游。
宸:北辰所居,即帝王所居,代指帝王。
孝化:封建孝道的教化作用。
}\par
睿藻仙才盈彩笔,自惭何敢再为辞?\zhu{睿藻:称颂帝王所作诗文的用语,此指元春题咏。
睿:明智。
藻:文辞。
}\ji{好诗!此不过颂圣应酬耳,犹未见长,以后渐知。
}\par
\hop
世外仙源\quad {\footnotesize 匾额}\ji{落想便不与人同。
 }\quad 林黛玉\par
名园筑何处,仙境别红尘。
\par
借得山川秀,添来景物新。
\ji{所谓“信手拈来无不是”。
}\ji{阿颦自是一种心思。
}\par
香融金谷酒,花媚玉堂人。
\zhu{金谷酒:晋代石崇有金谷园,常与宾客游宴其中,命各赋诗,“不能者,罚酒三斗”(见其《金谷诗序》)。
又李白《春夜宴桃李园序》:“不有佳作,何伸雅怀?如诗不成,罚依金谷酒数。
”这里借指大观园开筵赋诗。
玉堂人:指元春。
玉堂:嫔妃所居之处。
}\par
何幸邀恩宠,宫车过往频?\ji{末二首是应制诗。
}\ji{余谓宝林此作未见长,何也?盖后文别有惊人之句也。
在宝卿有生不屑为此,在黛卿实不足一为。
}\ping{可能是预想宫车频繁过往,今后贾元春还会回来省亲,但是在前八十回宫车统共就来了这么一次。
也可能是说元妃省亲规模宏大,车队很长。
}\par
\hop
贾妃看毕,称赏一番,又笑道:“终是薛林二妹之作与众不同,非愚姊妹可同列者。
”\ping{愚姊妹即是迎春,探春,惜春三姐妹,这里一方面是因为薛林二姐妹的诗才拔群,另一方面也是因为薛林是贾府外客,作为贾府长女的贾元春,这样做也是谦虚。
}原来林黛玉安心今夜大展奇才,将众人压倒,\ji{这却何必,然尤物方如此。
}不想贾妃只命一匾一咏,倒不好违谕多作,只胡乱作一首五言律应景罢了。
\ji{请看前诗,却云是胡乱应景。
}\par
彼时宝玉尚未作完,只刚做了“潇湘馆”与“蘅芜苑”二首,正作“怡红院”一首,起草内有“绿玉春犹卷”一句。
宝钗转眼瞥见,便趁众人不理论,\zhu{理论:理会。
}急忙回身悄推他道:“他\ji{此“他”字指贾妃。
}\geng{这样章法,又是不曾见过的。
}因不喜‘红香绿玉’四字,改了‘怡红快绿’;你这会子偏用‘绿玉’二字,岂不是有意和他争驰了?\ping{第十九回,贾宝玉以“香芋(香玉)”比喻林黛玉,贾元春去掉匾额里的“香玉”两个字,表现出对于“香玉”的厌烦,可能暗示了贾元春并不喜欢林黛玉。
另一种说法是贾元春不喜欢奢靡浮华,从前文修改“天仙宝境”为“省亲别墅”和后文贾元春的告诫“倘明岁天恩仍许归省,万不可如此奢华靡费了。
”可以看出。
“香”“玉”所代表的浮华靡丽并不为富贵已极的元春所喜爱,“怡”“快”所代表的舒畅愉快的朴素幸福正是元春所缺少的所珍视的。
}况且蕉叶之说也颇多,再想一个字改了罢。
”宝玉见宝钗如此说,便拭汗道:\ji{想见其构思之苦,方是至情。
最厌近之小说中满纸“神童”、“天分”等语。
}“我这会子总想不起什么典故出处来。
”宝钗笑道:“你只把‘绿玉’的‘玉’字改作‘蜡’字就是了。
”宝玉道:“‘绿蜡’\geng{好极!}可有出处?”宝钗见问,悄悄的咂嘴点头\geng{媚极!艳极!}笑道:“亏你今夜不过如此,将来金殿对策,\zhu{金殿对策:金殿:即金銮殿,皇帝受朝见的殿堂。
对策:原指汉代被荐举的人对答皇帝有关政治、经义的策问。
清代科举制度,会试后还要参加由皇帝主持的殿试,殿试的题目为策问。
}你大约连‘赵钱孙李’都忘了呢!\zhu{赵钱孙李:《百家姓》的头一句。
《百家姓》:北宋时编的集姓氏为四言韵语的书,作者佚名。
旧时流行的启蒙课本之一。
}\ji{有得宝卿奚落。
但\sout{就}[孰]谓宝卿无情?只是较阿颦施之特正耳。
}唐钱珝咏芭蕉诗头一句‘冷烛无烟绿蜡干’,\zhu{钱珝咏芭蕉诗:唐代诗人钱珝(珝:音“许”)《未展芭蕉》诗:“冷烛无烟绿蜡干,芳心犹卷怯春寒。
”宝玉题“怡红快绿”一诗中“绿蜡春犹卷”句,不但“绿蜡”二字本此,“春犹卷”三字也从此诗第二句化出。
可见宝玉的诗句原是一起构思的。
小说写他原用“绿玉”,据宝钗意见改为“绿蜡”,实为借此穿插对话,勾画人物。
}你都忘了不成?”\ji{此等处便用硬证实处,最是大力量,但不知是何心思,是从何落想,穿插到如此玲珑锦绣地步。
}\geng{如此穿插安得不令人拍案叫绝!壬午季春。
}\chen{乃翁前何多敏捷,今见乃姐何反迟钝,未免怯才,拘紧人所必有之耳。
}宝玉听了,不觉洞开心臆,笑道:“该死,该死!现成眼前之物偏倒想不起来了,真可谓‘一字师’了。
\zhu{一字师:唐代诗僧齐己《早梅》诗:“前村深雪里,昨夜数枝开。
”郑谷看了后,改“数枝”为“一枝”,齐己钦服下拜。
时人称郑谷为“一字师”。
}从此后我只叫你师父,再不叫姐姐了。
\ping{
改“玉”为“蜡”,也是把有富贵意象的“玉”改为了有清冷意象的“蜡”,这也是迎合了贾元春的喜好。
}
”宝钗亦悄悄的笑道:“还不快作上去,只管姐姐妹妹的。
谁是你姐姐?那上头穿黄袍的才是你姐姐,你又认我这姐姐来了。
”一面说笑,因说笑又怕他耽延工夫,遂抽身走开了。
\ji{一段忙中闲文,已是好看之极,出人意外。
}宝玉只得续成,共有了三首。
\par
此时林黛玉未得展其抱负,自是不快。
因见宝玉独作四律,大费神思,何不代他作两首,也省他些精神不到之处。
\ji{写黛卿之情思,待宝玉却又如此,是与前文特犯不犯之处。
}\geng{偏又写一样,是何心意构思而得?畸笏。
}
想着,便也走至宝玉案旁,悄问:“可都有了?”宝玉道:“才有了三首,只少‘杏帘在望’一首了。
”黛玉道:“既如此,你只抄录前三首罢。
赶你写完那三首,我也替你作出这首了。
”说毕,低头一想,早已吟成一律,\ji{瞧他写阿颦只如此,便妙极。
}便写在纸条上,搓成个团子,掷在他跟前。
\geng{纸条送递系童生秘诀,黛卿自何处学得?一笑。
丁亥春。
}
\chen{姐姐做试官尚用枪手,难怪世间之代倩多耳。
\zhu{代倩[qìng]:谓科举考试时请人代笔作弊。
}}\ping{同样是帮助贾宝玉作诗,薛宝钗帮助贾宝玉的方式是辅助启发,而林黛玉帮助贾宝玉的方式是包办代替,传统妻子的角色应该是像薛宝钗那样在背后辅佐丈夫,默默付出,而不是像林黛玉那样过于张扬外露,强势能干。
贾元春之前已经通过删除“红香绿玉”牌匾上的“香玉”二字(第十九回中贾宝玉以“香芋(香玉)”比喻林黛玉),展示了自己更加倾向于传统的薛宝钗作贾宝玉的妻子,这可能就是她偏爱薛宝钗的原因之一。
另一个原因可能是贾元春进宫前后,得知了贾琏的妻子王熙凤的光芒盖住了丈夫贾琏,所以不希望贾宝玉也像贾琏那样活在过于强势能干的妻子的阴影之下。
不可否认的是,作为贾元春母亲王夫人对于薛宝钗的偏好,可能影响了贾元春自己的判断。
}宝玉打开一看,只觉此首比自己所作的三首高过十倍,真是喜出望外,\ji{这等文字,亦是观书者望外之想。
}
遂忙恭楷呈上。
贾妃看道:\par
\hop
有凤来仪\quad {\footnotesize 臣宝玉谨题}\par
秀玉初成实,堪宜待凤凰。
\zhu{秀玉:喻竹。
竹子别称绿玉。
}\ji{起便拿得住。
}\par
竿竿青欲滴,个个绿生凉。
\zhu{个个:竹叶簇聚,形状像许多“个”字。
}\par
迸砌防阶水,穿帘碍鼎香。
\zhu{这一联即“妨阶水之迸砌,碍鼎香之穿帘”。
砌:石阶的边沿。
两句都写竹子长得茂密,既挡住了阶下泉水不使溅上阶沿,又留住了鼎炉香烟不使穿帘散去。
}\ji{妙句!古云:“竹密何妨水过”,今偏翻案。
}\par
莫摇清碎影,好梦昼初长。
\par
\hop
蘅芷清芬\zhu{蘅芷:杜蘅白芷,都是香草。
}\par
蘅芜满净苑,萝薜助芬芳。
\ji{“助”字妙!通部书所以皆善炼字。
}\par
软衬三春草,柔拖一缕香。
\zhu{此联承上,写满苑异草,牵藤引蔓,柔软如春日嫩草,吐露出一缕芳香。
}\ji{刻画入妙。
}\par
轻烟迷曲径,冷翠滴回廊。
\zhu{轻烟:藤蔓纤柔,夹缠萦绕,犹如轻烟。
冷翠:草上露珠,清冷碧翠。
}\ji{甜脆满颊。
}\par
谁谓池塘曲,谢家幽梦长。
\zhu{南朝诗人谢灵运《登池上楼》诗有“池塘生春草”之句,相传为梦中所得。
这联意谓:谁说只有写过“池塘生春草”名句的谢灵运才有触发诗兴的好梦呢!
}\par
\hop
怡红快绿\par
深庭长日静,两两出婵娟。
\zhu{两两:指芭蕉与海棠。
}\ji{双起双敲,读此首始信前云“有蕉无棠不可,有棠无蕉更不可”等批非泛泛妄批驳他人,到自己身上则无能为之论也。
}\par
绿蜡\ji{本是“玉”字,此遵宝卿改,似较“玉”字佳。
}春犹卷,\ji{是蕉。
}红妆夜未眠。
\ji{是海棠。
}\zhu{这一联上句说春天蕉叶卷而未舒,犹如翠烛;下句写海棠入夜犹开,像少女未眠。
苏轼《海棠》诗:“只恐夜深花睡去,故烧高烛照红妆”,以红妆喻海棠。
}\par
凭栏垂绛袖,\ji{是海棠之情。
}倚石护青烟。
\ji{是芭蕉之神。
何得如此工恰自然?真是好诗,却是好书。
}\zhu{这一联上句说槛外海棠,红花如凭栏美人垂下的大红衫袖;下句说石旁芭蕉,绿叶像回护的青烟。
}\par
对立东风里,\ji{双收。
}主人应解怜。
\zhu{解怜:懂得爱惜。
}\ji{归到主人方不落空。
}\ji{王梅隐云:“咏物体又难双承双落,一味双拿则不免牵强。
”此首可谓诗题两称,极工、极切、极流利妩媚。
\zhu{王梅隐是否和第十三回批语中的“王隐梅”是同一人,待考。}
}\par
\hop
杏帘在望\par
杏帘招客饮,在望有山庄。
\ji{分题作一气呵成,格调熟练,自是阿颦口气。
}\par
菱荇鹅儿水,桑榆燕子梁。
\ji{阿颦之心臆才情原与人别,亦不是从读书中得来。
}\zhu{这一联上句说鹅儿在长着菱荇的水面上嬉戏。
下句意思是燕子飞越桑榆之间,忙忙碌碌地在梁上筑巢。
菱荇:菱角、荇菜。
}\par
一畦春韭绿,十里稻花香。
\par
盛世无饥馁,何须耕织忙。
\zhu{馁:音“内”三声,饥饿。
}\ji{以幻入幻,顺水推舟,且不失应制,所以称阿颦。
}\par
\hop
贾妃看毕,喜之不尽,说:“果然进益了!”又指“杏帘”一首为前三首之冠。
遂将“浣葛山庄”改为“稻香村”。
\ji{如此服善,妙!}\geng{仍用玉兄前拟“稻香村”,却如此幻笔幻体,文章之格式至矣尽矣!壬午春。
}又命探春另以彩笺誊录出方才一共十数首诗,出令太监传与外厢。
\zhu{外厢:外面。
}贾政等看了,都称颂不已。
贾政又进《归省颂》。
元妃又命以琼酥金脍等物,\zhu{脍:音“快”,细切的肉、鱼。
琼酥金脍:宫中精美小食品。
}赐与宝玉并贾兰。
\ji{百忙中点出贾兰,一人不落。
}此时贾兰极幼,未达诸事,只不过随母依叔行礼,故无别传。
贾环从年内染病未痊,自有闲处调养,故亦无传。
\ji{补明,方不遗失。
}\par
那时贾蔷带领十二个女戏,在楼下正等的不耐烦,只见一太监飞来说:“作完了诗,快拿戏目来!”贾蔷急将锦册呈上,并十二个花名单子。
少时,太监出来,只点了四出戏:\par
\hop
第一出《豪宴》;\zhu{《豪宴》是清初李玉《一捧雪》传奇中的一出。
剧本演明代莫怀古因玉杯“一捧雪”,被奸邪害得家破人亡的故事。
第四十八回雨村诬坐石呆子致死,抄没古扇送给贾赦。
贾家后来之败与贾赦、雨村当有直接关系。
}\ji{《一捧雪》中。
伏贾家之败。
}\par
第二出《乞巧》;\zhu{《乞巧》即清初洪升《长生殿》传奇中的一出。
剧本演唐玄宗与杨贵妃的悲剧故事。
杨贵妃是在安史之乱中于马嵬被逼缢死。暗示元春的死类似。
}\ji{《长生殿》中。
伏元妃之死。
}\ping{以杨贵妃喻贾元春,可见俩人都是政治斗争的牺牲品。
}\par
第三出《仙缘》;\zhu{《仙缘》即明代汤显祖《邯郸记》中《合仙》一出。
剧本演吕洞宾下凡度卢生上天,代替何仙姑天门扫花的故事。
甄宝玉送玉可能是点醒宝玉的契机,预示宝玉像卢生一样看破红尘出家。“送玉”可能意为“送宝玉出家”。
}\ji{《邯郸梦》中。
伏甄宝玉送玉。
}\par
第四出《离魂》。
\zhu{《离魂》是汤显祖《牡丹亭》改编本中的一出。
即明代汤显祖《牡丹亭》第二十出《闹殇》。写杜丽娘后园寻梦,杳无所得,归来哀情难遣,积郁成病。
“伤春病到深秋。”中秋之夕,欲一见那皎皎月轮,倩春香开轩望去。但见秋雨淅沥,西风梧叶,丽娘心病难医,
心曲无诉,终于带着对爱情徒然的渴望,告别了围于身边的双亲,在这连宵夜雨中离开人世。
该剧情节凄恻,故脂批谓“伏黛玉死”。
}\ji{《牡丹亭》中。
伏黛玉死。
}\ji{所点之戏剧伏四事,乃通部书之大过节、大关键。
}\par
\zhu{
按封建礼法,在元妃省亲时,这些戏中有些情节是不能演的。
作者在这里或有通过戏名,暗示贾府和主要人物结局的用意}\par
\hop
贾蔷忙张罗扮演起来。
一个个歌欺裂石之音,\zhu{歌欺裂石之音:欺:超过。
裂石之音:比喻声音的激越。
}舞有天魔之态。
\zhu{天魔舞:本为唐代一种宫廷舞乐,王建《宫词》:“十六天魔舞袖长。
”元顺帝至正十四年制天魔舞,系宫廷大型队舞,以宫女十六人,盛妆扮成菩萨相,有多种乐器伴奏,应节而舞。
}虽是妆演的形容,却作尽悲欢情状。
\ji{二句毕矣。
}刚演完了,一太监执一金盘糕点之属进来,问:“谁是龄官?”贾蔷便知是赐龄官之物,喜的忙接了,\ji{何喜之有?伏下后面许多文字,只用一“喜”字。
\zhu{暗示贾蔷和龄官悄生情愫,为后文埋下伏笔。}
}命龄官叩头。
太监又道:“贵妃有谕,说:‘龄官极好,再作两出戏,不拘那两出就是了。
’”贾蔷忙答应了,因命龄官做《游园》、《惊梦》二出。
\zhu{《游园》、《惊梦》:原本《牡丹亭》中的《惊梦》一出,到了演出本分为《游园》、《惊梦》两出。
写杜丽娘在春香怂恿下私去后花园赏春,大自然的美景与生机唤醒她那青春的活力,
她第一次明确意识到自由的宝贵,意识到自己命运的可悲。
她“春情难遣”,渴望着得到理想的爱人,可现实却是由父母“一例一例”地去排选名门子弟,
去抛舍她的青春。她郁郁而回,在梦中见到了理想的情人——柳梦梅,“两情相合”“千般爱惜,万种温存”。
醒来,情人杳然,只有母亲的责问,只有那对甜蜜梦境的永久忆想。
}龄官自为此二出原非本角之戏,执意不作,定要作《相约》、《相骂》二出。
\zhu{《相约》、《相骂》:明代《钗钏记》传奇中的两出。
《钗钏记》写皇甫吟善文,然贫甚,岳父史直欲悔亲,经历波折后皇甫吟和碧桃终成眷属。
《相约》写云香奉碧桃小姐之命来皇甫吟家,得见吟母张氏,遂留下口信,约吟在中秋夜到后花园内会面,以赠信物和聘资。《相骂》则写云香再到皇甫家,张氏听人谗言,以为欲陷害其子,怒相诟骂,云香亦不示弱,遂生口角,至推推搡搡,不欢而散。
}
\ji{《钗钏记》中。
总隐后文不尽风月等文。
}\ji{按近之俗语云:“宁养千军,不养一戏。
”盖甚言优伶之不可养之意也。
大抵一班之中,此一人技业稍优出众,此一人则拿腔作势、辖众恃能,种种可恶,使主人逐之不舍责之不可,虽欲不怜而实不能不怜,虽欲不爱而实不能不爱。
余历梨园子弟广矣,个个皆然,亦曾与惯养梨园诸世家兄弟谈议及此,众皆知其事而皆不能言。
今阅《石头记》至“原非本角之戏,执意不作”二语,便见其恃能压众、乔酸娇妒,淋漓满纸矣。
复至“情悟梨香院”一回更将和盘托出,与余三十年前目睹身亲之人现形于纸上。
使言《石头记》之为书,情之至极、言之至恰,然非领略过乃事、迷陷过乃情,即观此,茫然嚼蜡,亦不知其神妙也。
}贾蔷扭他不过,\ji{如何反扭他不过?其中便隐许多文字。
}\ping{贵妃并没有指定具体戏名,贾蔷要求龄官做《游园》、《惊梦》二出,但是龄官自己执意不作,定要作《相约》、《相骂》二出。
确定戏名的环节是在贾蔷和龄官之间进行的,而非在贵妃和龄官之间进行的,这里应该是暗示贾蔷和龄官暗生情愫,而非表明龄官不畏权贵。
}只得依他作了。
贾妃甚喜,命“不可难为了这女孩子,好生教习”,\ji{可知尤物了。
}额外赏了两匹宫缎、两个荷包并金银锞子、\zhu{锞子(锞音“课”):旧时做货币用的小金锭或银锭。
}食物之类。
\ji{又伏下一个尤物,一段新文。
}然后撤筵,将未到之处复又游顽。
忽见山环佛寺,忙另盥手进去焚香拜佛,又题一匾云:“苦海慈航”。
\zhu{苦海慈航:佛教宣扬现实世界如同苦海,劝人出家是菩萨的慈悲行为,就像用船救人超渡苦海。
}\ji{寓通部人事。
一篇热文,却如此冷收。
}又额外加恩与一班幽尼女道。
\par
少时,太监跪启:“赐物俱齐,请验等例。
”\zhu{等:等级,次序。
例:旧例,惯例。
}乃呈上略节。
贾妃从头看了,俱甚妥协,即命照此遵行。
太监听了,下来一一发放。
原来贾母的是金、玉如意各一柄,沉香拐拄一根,伽楠念珠一串,“富贵长春”宫缎四匹,“福寿绵长”宫绸四匹,紫金“笔锭如意”锞十锭,\zhu{笔锭如意:金锭子上的字样,以“笔锭”谐音“必定”,意谓一定吉祥、事事如意。
}“吉庆有鱼”银锞十锭。
\zhu{吉庆有鱼:银锭上的字样,以“鱼”谐音“馀”。
}邢夫人、王夫人二分,只减了如意、拐、珠四样。
贾敬、贾赦、贾政等,每分御制新书二部,宝墨二匣,金、银爵各二只,\zhu{爵:古代的三脚酒器。
}
表礼按前。
\zhu{表礼:旧日赠送或赏赐的礼物。
}宝钗、黛玉诸姊妹等,每人新书一部,宝砚一方,新样格式金银锞二对。
宝玉亦同此。
\ji{此中忽夹上宝玉,可思。
}贾兰则是金银项圈二个,金银锞二对。
尤氏、李纨、凤姐等,皆金银锞四锭,表礼四端。
\zhu{端:量词,布帛的长度单位。
}
外表礼二十四端,清钱一百串,
\zhu{
清钱:清代官局所铸铜钱,形式、成分、重量、文字皆有定制,是为“制钱”。
凡市场上流通之铜钱,其千文中全为官铸制钱,清纯一色,而不混夹民间私铸之小钱,称为“清钱”。
}
是赐与贾母、王夫人及诸姊妹房中奶娘众丫鬟的。
贾珍、贾琏、贾环、贾蓉等,皆是表礼一分,金锞一双。
其馀彩缎百端,\zhu{端:布帛长度单位。
}金银千两,御酒华筵,是赐东西两府凡园中管理工程、陈设、答应及司戏、掌灯诸人的。
\zhu{陈设、答应:此处均指近侍当差之人。
}外有清钱五百串,是赐厨役、优伶、百戏、杂行人丁的。
\par
众人谢恩已毕,执事太监启道:“时已丑正三刻,请驾回銮。
”\zhu{北宋时开始将每个时辰分为“初”、“正”两部分,分十二时辰为二十四,称“小时”。
子时从当天夜里十一点到第二天凌晨一点,子初从夜里十一点开始到零点,子正从零点开始到凌晨一点。
以此类推,子丑寅卯辰巳午未申酉戌亥代表一天中的十二个时辰。
清代正式规定一昼夜为九十六刻,每个时辰八刻,又区分为上四刻和下四刻。
汉代皇宫中值班人员分五个班次,按时更换,叫“五更”,由此便把一夜分为五更,每更为一个时辰。
戌时为一更,亥时为二更,子时为三更,丑时为四更,寅时为五更。
由于古代报更使用击鼓方式,故又以鼓指代更。
如杜甫《阁夜》:“五更鼓角声悲壮,三峡星河影动摇。
”又如,白居易的《长恨歌》:“迟迟钟鼓初长夜,耿耿星河欲曙天。
”其中的“鼓角”、“钟鼓”都是古时用来打更的器具。
丑正三刻根据推断可以知道是凌晨两点四十五。
}\ping{根据前文太监的话:“酉初刻进大明宫领宴看灯方请旨,只怕戌初才起身呢”,也就是下午五点才去请旨,晚上七点才起身。
结合当时是元宵节,天短夜长,起身之时已经天黑了,第二天凌晨两点四十五,元春又回宫了。
黑夜中归乡,在家短暂停留之后,又在黑夜中离开,真是一人身处黑暗,然后又消失在黑暗中。
}贾妃听了,不由的满眼又滚下泪来。
却又勉强堆笑,拉住贾母、王夫人的手,紧紧的不忍释放,\ji{使人鼻酸。
}再四叮咛:“不须记挂,好生自养。
如今天恩浩荡,一月许进内省视一次,见面是尽有的,何必伤惨。
倘明岁天恩仍许归省,万不可如此奢华靡费了。
”\ji{妙极之谶。试看别书中专能故用一不祥之语为谶,今偏不然,只有如此现成一语,便是不再之谶。只看他用一“倘”字,便隐讳自然之至。
}贾母等已哭的哽噎难言了。
贾妃虽不忍别,怎奈皇家规范,违错不得,只得忍心上舆去了。
这里诸人好容易将贾母、王夫人安慰解劝,搀扶出园去了。
正是——\geng{一回离合悲欢夹写之文,真如山阴道上令人应接不暇,尚有许多忙中闲、闲中忙小波澜,一丝不漏,一笔不苟。
}\par
\qi{总评:此回铺排,非身经历、开巨眼、\zhu{巨眼:比喻善于鉴别的眼力。
}伸大笔,则必有所滞罣牵强,\zhu{罣:同“挂”。
}岂能如此触处成趣,立后文之根,足本文之情者?且借象说法,学我佛阐经,代天女散花,以成此奇文妙趣,惟不得与四才子书之作者,
\zhu{
四才子书:可能是六才子书。
清朝金圣叹集《庄子》、《离骚》、《史记》、杜甫之律诗、《水浒传》、《西厢记》为六才子书。
金圣叹《三国演义序》:「余尝集才子书者六,其目曰:《庄》也,《骚》也,马之《史记》也,杜之律诗也,《水浒》也,《西厢》也。」
}
同时讨论臧否,\zhu{臧否:音“脏匹”,善恶,得失,引申为评论人物的好坏。
}为可恨耳。
}
\dai{034}{林黛玉误会宝玉错铰香袋}
\dai{035}{元春省亲见家人}
\dai{036}{元春题名}
\sun{p18-1}{因得彩解袋赏小厮,黛玉莽撞自悔铰袋}{图右下:宝玉来到院外,就有小厮来抱住,道:“方才老太太打发人问了几遍,我们回说老爷喜欢,要不然老太太叫你进去了,哪里还能施展才华!今儿得了彩头,该赏我们了。
”一不容分说,一个个上来解荷包,解扇袋,将宝玉身上所有尽行解去。
图中部:宝玉见过贾母,贾母知不曾难为着他,心中自是喜欢。
图左上:宝玉来到自己房里,袭人见身边饰物一件不存,笑道:“必是那些没脸的东西们解了去。
”黛玉听说,生气回房,将尚未做完的一只香袋儿,拿起剪子就铰。
直到宝玉从内衣襟上解下她送的荷包,黛玉方才后悔自己莽撞。
}
\sun{p18-2}{宝黛同往上房,妙玉带发修行}{图右侧:黛玉被宝玉缠不过,只得起来道:“你的意思不叫我安生,我就离了你。
”说着往外就走。
宝玉笑道:“你到那里,我跟到那里。
”一面说,一面二人出房,到王夫人上房中去了。
图左侧:出身读书仕宦之家,文墨极通,模样极好的妙玉带发修行。
}
\sun{p18-3}{贾府跪迎贵妃省亲}{自贾母等有爵者,俱各按品服大妆。
街头巷口,俱系围幕挡严。
一时,有十来个太监都喘吁吁跑来拍手儿。
这些太监会意,都知道是“来了,来了”,各按方向站住。
贾赦领合族子侄在西街门外,贾母领合族女眷在大门外迎接。
半日静悄悄的。
忽见一对红衣太监骑马缓缓的走来,至西街门下了马,将马赶出围幕之外,便垂手面西站住。
半日又是一对,亦是如此。
少时便来了十来对。
}
\sun{p18-4}{贵妃前导仪仗执事}{方闻得隐隐细乐之声。
一对对龙旌凤翣,雉羽夔头,又有销金提炉焚着御香;然后一把曲柄七凤黄金伞过来,便是冠袍带履。
又有随事太监捧着香珠、绣帕、漱盂、拂尘等类。
}
\sun{p18-5}{贵妃乘坐版舆,太监扶起亲人}{一队队过完,后面方是八个太监抬着一顶金顶金黄绣凤版舆,缓缓行来。
贾母等连忙路旁跪下。
早飞跑过几个太监来,扶起贾母、邢夫人、王夫人来。
}
\sun{p18-6}{贵妃乘舟游幸大观园}{执拂太监跪请登舟。
贾妃乃下舆。
只见清流一带,势如游龙,两边石栏上,皆系水晶玻璃各色风灯,点的如银光雪浪;上面柳杏诸树虽无花叶,然皆用通草绸绫纸绢依势作成,粘于枝上的,每一株悬灯数盏;更兼池中荷荇凫鹭之属,亦皆系螺蚌羽毛之类作就的。
诸灯上下争辉,真系玻璃世界,珠宝乾坤。
船上亦系各种精致盆景诸灯,珠帘绣幕,桂楫兰桡,自不必说。
已而入一石港,港上一面匾灯,明现着“蓼汀花溆”四字。
}
\sun{p18-7}{开筵宴元春赐名,试诗才姊妹题咏}{元春进入正殿,与贾母等亲人相聚,继而大开筵宴,贾母等在下相陪,尤氏李纨凤姐等捧羹把盏。
过后,元春亲搦湘管,择其几处最喜者赐名,题园之总名“大观园”, 其余改题为“潇湘馆”、“怡红院”、“蘅芜苑”等等。
又命众姐妹和宝玉各赋题咏,元妃看完后喜之不尽,撒筵后行赏众人,洒泪惜别。
}

