\chapter{凸碧堂品笛感凄清 \quad 凹晶馆联诗悲寂寞}
\qi{此回着笔最难,不叙中秋夜宴则漏,叙夜宴又与上元相犯;\zhu{上元:元宵节,阴历正月十五日。
犯:重复。
}不叙诸人酬和则俗,叙酬和又与起社相犯。
诸人在贾政前吟诗,诸人各自为一席,又非礼。
\zhu{这句话的意思是,在上一回中,贾母看中秋赴宴人丁稀少,不让大家分开坐而是围坐一桌。
}既叙夜宴再叙酬和,不漏不俗,更不相犯。
云行月移,水流花放,别有机括,\zhu{机括:弩上控制箭矢发射的机件。
泛指机械发动、开启的部分。
这里比喻心机、计谋。
}深宜玩索。
}\par
话说贾赦贾政带领贾珍等散去不提。
且说贾母这里命将围屏撤去,两席并而为一。
众媳妇另行擦桌整果,更杯洗箸,陈设一番。
贾母等都添了衣,盥漱吃茶,方又入坐,团团围绕。
贾母看时,宝钗姊妹二人不在坐内,知他们家去圆月去了,且李纨凤姐二人又病着,少了四个人,便觉冷清了好些。
\geng{不想这次中秋反写得十分凄楚。
}贾母因笑道:“往年你老爷们不在家,咱们越性请过姨太太来,大家赏月,却十分闹热。
忽一时想起你老爷来,又不免想到母子、夫妻、儿女不能一处,也都没兴。
及至今年你老爷来了,正该大家团圆取乐,又不便请他们娘儿们来说说笑笑。
况且他们今年又添了两口人,\zhu{两口人:指薛宝琴和薛蝌。
}也难丢了他们跑到这里来。
偏又把凤丫头病了,有他一人来说说笑笑,还抵得十个人的空儿。
可见天下事总难十全。
”说毕,不觉长叹一声,遂命拿大杯来斟热酒。
\ping{想要欢乐,底色是愁。
}王夫人笑道:“今日得母子团圆,自比往年有趣。
往年娘儿们虽多,终不似今年自己骨肉齐全的好。
”贾母笑道:“正是为此,所以才高兴拿大杯来吃酒。
你们也换大杯才是。
”邢夫人等只得换上大杯来。
因夜深体乏,且不能胜酒,未免都有些倦意,无奈贾母兴犹未阑,只得陪饮。
\par
贾母又命将罽毡铺于阶上,\zhu{罽(音“计”):毛织的毯子。
}命将月饼、西瓜、果品等类都叫搬下去,令丫头媳妇们也都团团围坐赏月。
贾母因见月至中天,比先越发精彩可爱,因说:“如此好月,不可不闻笛。
”因命人将十番上女孩子传来。
\zhu{十番:又称十番锣鼓,一种用乐器合奏的套曲。
李斗《扬州画舫录》:“是乐不用小锣、金锣、铙钹、号筒,只用笛、管、箫、弦、提琴、云锣、汤锣、木鱼、檀板、大鼓十种,故名十番鼓。
番者更番之谓。
……若夹用锣铙之属,则为粗细十番。
”}
贾母道:“音乐多了,反失雅致,只用吹笛的远远的吹起来就够了。
”\ping{贾母还想着让仆人们一起围坐赏月,简直是贾家慈悲代表了。
但是贾母既然已经在上一回子女送菜的时候说了要厉行节俭,不要大摆排场,而现在还蓄养歌姬,着实令人感到矛盾。
}
说毕,刚才去吹时,只见跟邢夫人的媳妇走来向邢夫人前说了两句话。
贾母便问:“说什么事?”那媳妇便回说:“方才大老爷出去,被石头绊了一下,崴了腿。
”贾母听说,忙命两个婆子快看去,又命邢夫人快去。
邢夫人遂告辞起身。
贾母便又说:“珍哥媳妇也趁着便就家去罢,我也就睡了。
”尤氏笑道:“我今日不回去了,定要和老祖宗吃一夜。
”贾母笑道:“使不得,使不得。
你们小夫妻家,今夜不要团圆团圆,如何为我耽搁了。
”尤氏红了脸,笑道:“老祖宗说的我们太不堪了。
我们虽然年轻,已经是十来年的夫妻,也奔四十岁的人了。
况且孝服未满,陪着老太太顽一夜还罢了,岂有自去团圆的理。
”贾母听说,笑道:“这话很是,我倒也忘了孝未满。
可怜你公公已是二年多了,\geng{不是算贾敬,却是算赦死期也。
\ping{伏贾府事败,也就这二年的事。
}}可是我倒忘了,该罚我一大杯。
既这样,你就越性别送,陪着我罢了。
你叫蓉儿媳妇送去,就顺便回去罢。
”尤氏说了。
蓉妻答应着,送出邢夫人,一同至大门,各自上车回去。
不在话下。
\par
这里贾母仍带众人赏了一回桂花,又入席换暖酒来。
正说着闲话,猛不防只听那壁厢桂花树下,\zhu{壁廂:边、旁、面。
“那壁厢”就是那一边。
}呜呜咽咽,悠悠扬扬,吹出笛声来。
趁着这明月清风,天空地净,真令人烦心顿解,万虑齐除,都肃然危坐,默默相赏。
听约两盏茶时,方才止住,大家称赞不已。
于是遂又斟上暖酒来。
贾母笑道:“果然可听么?”众人笑道:“实在可听。
我们也想不到这样,须得老太太带领着,我们也得开些心胸。
”贾母道:“这还不大好,须得拣那曲谱越慢的吹来越好。
”说着,便将自己吃的一个内造瓜仁油松穰月饼,
\zhu{
瓜仁油松穰月饼:袁枚《随园食单》“刘方伯月饼”条曾经介绍做法道:“用山东飞面作酥为皮,中用松子仁、核桃仁、瓜子仁为细末,微加冰糖和猪油作馅。”评价是:“食之不觉甚甜,用香松柔腻,迥异寻常。” 
}
又命斟一大杯热酒,送给谱笛之人,慢慢的吃了再细细的吹一套来。
媳妇们答应了,方送去,只见方才瞧贾赦的两个婆子回来了,说:“右脚面上白肿了些,如今调服了药,疼的好些了,也不甚大关系。
”贾母点头叹道:“我也太操心。
打紧说我偏心,我反这样。
”因就将方才贾赦的笑话说与王夫人尤氏等听。
王夫人等因笑劝道:“这原是酒后大家说笑,不留心也是有的,岂有敢说老太太之理。
老太太自当解释才是。
”
\zhu{解释:劝解疏通。}
只见鸳鸯拿了软巾兜与大斗篷来,说:“夜深了,恐露水下来,风吹了头,须要添了这个。
坐坐也该歇了。
”贾母道:“偏今儿高兴,你又来催。
难道我醉了不成,偏到天亮!”因命再斟酒来。
一面戴上兜巾,披了斗篷,大家陪着又饮,说些笑话。
只听桂花阴里,呜呜咽咽,袅袅悠悠,又发出一缕笛音来,果真比先越发凄凉。
大家都寂然而坐。
夜静月明,且笛声悲怨,贾母年老带酒之人,听此声音,不免有触于心,禁不住堕下泪来。
众人此时都不禁凄凉寂历之意,\zhu{历:逐个,一一地。
引申为分明的、清晰的。
如“历历在目”、“晴川历历汉阳树,春草萋萋鹦鹉洲”。
寂历:寂静空旷。
}半日,方知贾母伤感,才忙转身陪笑,发语解释。
\geng{“转身”妙!画出对月听笛如痴如呆、不觉尊长在上之形景来。
}又命暖酒,且住了笛。
\par
尤氏笑道:“我也就学一个笑话,说与老太太解解闷。
”贾母勉强笑道:“这样更好,快说来我听。
”尤氏乃说道:“一家子养了四个儿子:大儿子只一个眼睛,二儿子只一个耳朵,三儿子只一个鼻子眼,四儿子倒都齐全,偏又是个哑叭。
”\zhu{哑叭:哑巴。
}正说到这里,只见贾母已朦胧双眼,似有睡去之态。
\geng{总写出凄凉无兴景况来。
}尤氏方住了,忙和王夫人轻轻的请醒。
\ping{
尤氏的笑话说了半天,未进入主题,故事中几个儿子没有一个是零件功能齐全的,这更令人倒胃口和扫兴。
元宵家宴贾母正有人丁不全之叹,说这样的笑话更犯忌不合时宜,道出了家族的命运。
}
贾母睁眼笑道:“我不困,白闭闭眼养神。
\zhu{白:单单,只是。
}你们只管说,我听着呢。
”王夫人等笑道:“夜已四更了,风露也大,请老太太安歇罢。
明日再赏十六,也不辜负这月色。
”贾母道:“那里就四更了?”王夫人笑道:“实已四更,他们姊妹们熬不过,都去睡了。
”贾母听说,细看了一看,果然都散了,只有探春在此。
贾母笑道:“也罢。
你们也熬不惯,况且弱的弱,病的病,去了倒省心。
只是三丫头可怜见的,尚还等着。
你也去罢,我们散了。
”说着,便起身,吃了一口清茶,便有预备下的竹椅小轿,便围着斗篷坐上,两个婆子搭起,众人围随出园去了。
不在话下。
\par
这里众媳妇收拾杯盘碗盏时,却少了个细茶杯,各处寻觅不见,又问众人:“必是谁失手打了。
撂在那里,告诉我拿了磁瓦去交收是证见,不然又说偷起来。
”众人都说:“没有打了,只怕跟姑娘的人打了,也未可知。
你细想想,或问问他们去。
”一语提醒了这管家伙的媳妇,因笑道:“是了,那一会记得是翠缕拿着的。
我去问他。
”说着便去找时,刚下了甬道,就遇见了紫鹃和翠缕来了。
\geng{妙!又书一个。
}翠缕便问道:“老太太散了,可知我们姑娘那去了?”\geng{更妙!}这媳妇道:“我来问那一个茶钟往那里去了,你们倒问我要姑娘。
”翠缕笑道:“我因倒茶给姑娘吃的,展眼回头,就连姑娘也没了。
”那媳妇道:“太太才说都睡觉去了。
你不知那里顽去了,还不知道呢。
”翠缕向紫鹃道:“断乎没有悄悄的睡去之理,只怕在那里走了一走。
如今见老太太散了,赶过前边送去,也未可知。
我们且往前边找找去。
有了姑娘,自然你的茶钟也有了。
你明日一早再找,有什么忙的。
”媳妇笑道:“有了下落就不必忙了,明儿就和你要罢。
”说毕回去,仍查收家伙。
这里紫鹃和翠缕便往贾母处来。
不在话下。
\par
原来黛玉和湘云二人并未去睡觉。
只因黛玉见贾府中许多人赏月,贾母犹叹人少,不似当年热闹,又提宝钗姊妹家去母女弟兄自去赏月等语,不觉对景感怀,自去俯栏垂泪。
宝玉近因晴雯病势甚重,诸务无心,\geng{带一笔,妙!更觉谨密不漏。
}王夫人再四遣他去睡,他也便去了。
探春又因近日家事着恼,无暇游玩。
虽有迎春惜春二人,偏又素日不大甚合。
所以只剩了湘云一人宽慰他,因说:“你是个明白人,何必作此形像自苦。
我也和你一样,我就不似你这样心窄。
何况你又多病,还不自己保养。
可恨宝姐姐,姊妹天天说亲道热,早已说今年中秋要大家一处赏月,必要起社,大家联句,到今日便弃了咱们,自己赏月去了。
社也散了,诗也不作了。
倒是他们父子叔侄纵横起来。
你可知宋太祖说的好:‘卧榻之侧,岂许他人酣睡。
’\zhu{卧榻之侧,岂许他人酣睡:喻自己势力范围,不许别人插足。
《宋史纪事本末·平江南》:南唐后主李煜遣徐铉向宋太祖乞求缓师,“帝按剑怒曰:‘不须多言,江南主亦有何罪,但天下一家,卧榻之侧,岂容他人酣睡耶!’”这里借喻大观园中作诗雅事,向来是姑娘姐妹吟咏展才,岂容“他们父子叔侄纵横起来”。
}他们不作,咱们两个竟联起句来,明日羞他们一羞。
”\par
黛玉见他这般劝慰,不肯负他的豪兴,因笑道:“你看这里这等人声嘈杂,有何诗兴。
”湘云笑道:“这山上赏月虽好,终不及近水赏月更妙。
你知道这山坡底下就是池沿,山坳里近水一个所在就是凹晶馆。
可知当日盖这园子时就有学问。
这山之高处,就叫凸碧;山之低洼近水处,就叫作凹晶。
这‘凸’‘凹’二字,历来用的人最少。
如今直用作轩馆之名,更觉新鲜,不落窠臼。
可知这两处一上一下,一明一暗,一高一矮,一山一水,竟是特因玩月而设此处。
有爱那山高月小的,便往这里来;有爱那皓月清波的,便往那里去。
只是这两个字俗念作‘洼’‘拱’二音,便说俗了,不大见用,只陆放翁用了一个‘凹’字,说‘古砚微凹聚墨多’,还有人批他俗,岂不可笑。
”林黛玉道:“也不只放翁才用,古人中用者太多。
如江淹《青苔赋》,\zhu{《青苔赋》:江淹:南朝梁文学家。
他的《青苔赋》有“悲凹险兮,唯流水而驰鹜”的句子。
}东方朔《神异经》,\zhu{《神异经》:东方朔,西汉武帝时人,善辞赋,性诙谐。
《神异经》是托名东方朔作的一部志怪小说,其中有“北方荒中有石湖,方千里,……其湖无凹凸,平满无高下”的话。
}
以至《画记》上云张僧繇画一乘寺的故事,\zhu{《画记》上云张僧繇画一乘寺:张僧繇(繇音“由”):南朝梁武帝时著名画家,开创了佛像绘画及雕刻的中国风格,《历代名画记》中列为“上品”。
他曾在南京一乘寺门上用古印度技法画凹凸花,即以类似色按浓淡配置而产生了浮雕般的效果,远望如凹凸,近看却平。
历来并无《画记》一书,只韩愈有文名《画记》,记人物及马,而张彦远《历代名画记》中“张僧繇”条并未提及凹凸花,可见此处实误,宜看作小说家言,乃泛指记画之作。
}不可胜举。
只是今人不知,误作俗字用了。
实和你说罢,这两个字还是我拟的呢。
因那年试宝玉,因他拟了几处,也有存的,也有删改的,也有尚未拟的。
这是后来我们大家把这没有名色的也都拟出来了,注了出处,写了这房屋的坐落,一并带进去与大姐姐瞧了。
他又带出来,命给舅舅瞧过。
谁知舅舅倒喜欢起来,又说:‘早知这样,那日该就叫他姊妹一并拟了,岂不有趣。
’所以凡我拟的,一字不改都用了。
如今就往凹晶馆去看看。
”\par
说着,二人便同下了山坡。
只一转弯,就是池沿,沿上一带竹栏相接,直通着那边藕香榭的路径。
\geng{点明,妙!不然此园竟有多大地亩了。
\zhu{这条脂评的意思是,通过湘云和黛玉的路线和视角,可以看出园子各处建筑景色布局紧凑,并不是那种占地面积很大但是内部空旷的园子。
}}因这几间就在此山怀抱之中,乃凸碧山庄之退居,
\zhu{退居:指供宾客临时休息的处所。}
因洼而近水,故颜其额曰“凹晶溪馆”。
\zhu{颜额:题额。}
因此处房宇不多,且又矮小,故只有两个老婆子上夜。
今日打听得凸碧山庄的人应差,与他们无干,这两个老婆子关了月饼果品并犒赏的酒食来,\zhu{关:领取。
}二人吃得既醉且饱,早已息灯睡了。
\geng{妙极!此书有进一步写法。
如王夫人云“他姊妹可怜,那里像当日林姑妈那样”,又如贾母云“如今人少,那里有当日人多”等数语,此谓进一步法也。
有退一步法,如宝钗之对邢岫烟云“此一时也,彼一时也,如今比不得先的话了,只好随实守分”,又如凤姐之对平儿云“如今我也看明白了,我如今也要作好好先生罢”等类,此谓退一步法也。
今又方收拾过贾母高乐,却又写出二婆子高乐,此[进]一步之实事也。
如前文海棠诗四首已足,忽又用湘云独成二律反压卷,此又进一步实事也。
所谓“法法皆全,丝丝不爽”也。
\zhu{
进一步法:作者已经表现一层境界或表述一件事,但意犹未尽,故而又做进一步的表述。
退一步法:以此时的情境,叙述反衬出不如过去的景况。
}
}\par
黛玉湘云见息了灯,湘云笑道:“倒是他们睡了好。
咱们就在这卷棚底下赏这水月如何?”\zhu{卷棚:建筑术语,指一种没有正脊的屋面做法,即屋面两坡的连接处不用正脊压盖,而呈一个弧形的转折。
}二人遂在两个湘妃竹墩上坐下。
只见天上一轮皓月,池中一轮水月,上下争辉,如置身于晶宫鲛室之内。
\zhu{鲛室:谓鲛人水中居室。
鲛人:神话传说中的人鱼。
}微风一过,粼粼然池面皱碧铺纹,真令人神清气净。
湘云笑道:“怎得这会子坐上船吃酒倒好。
这要是我家里这样,我就立刻坐船了。
”黛玉笑道:“正是古人常说的好,‘事若求全何所乐’。
据我说,这也罢了,偏要坐船起来。
”湘云笑道:“得陇望蜀,人之常情。
可知那些老人家说的不错。
说贫穷之家自为富贵之家事事趁心,告诉他说竟不能遂心,他们不肯信的;必得亲历其境,他方知觉了。
就如咱们两个,虽父母不在,然却也忝在富贵之乡,\zhu{忝:有愧于,辱,这里用作谦词。
成语有“恭列门墙”(愧在师门)。
}只你我竟有许多不遂心的事。
”黛玉笑道:“不但你我不能趁心,就连老太太、太太以至宝玉探丫头等人,无论事大事小,有理无理,其不能各遂其心者,同一理也,何况你我旅居客寄之人哉!”\geng{以\sout{立}[理]未[有]不怡然得享自然之乐者矣。
书中若干女子从主及婢,未\sout{有}
必各有所觉、各有所试、各有所长者,皆未如宝\sout{宝}[玉]无可关切筹划,可叹。
\zhu{此批错别字较多。
大概的意思是说,宝玉身为男人,只知安享富贵,对时务毫不关心,甚至不如书中的女子。
}}湘云听说,恐怕黛玉又伤感起来,忙道:“休说这些闲话,咱们且联诗。
”\par
正说间,只听笛韵悠扬起来。
黛玉笑道:“今日老太太、太太高兴了,这笛子吹的有趣,到是助咱们的兴趣了。
\geng{妙!正是吹笛之时。
勿认作又一处之笛也。
}咱两个都爱五言,就还是五言排律罢。
”\zhu{律:是律诗的简称,每首八句,中间四句为“对仗”。
每句五字的叫五言律;每句七字的叫七言律。
超过八句的律诗,叫排律。
}湘云道:“限何韵?”黛玉笑道:“咱们数这个栏杆的直棍,这头到那头为止。
他是第几根就用第几韵。
若十六根,便是‘一先’起。
这可新鲜?”湘云笑道:“这倒别致。
”于是二人起身,便从头数至尽头,止得十三根。
湘云道:“偏又是‘十三元’了。
\zhu{近体诗所用的诗韵,共分一〇六韵部。
各部都以该韵部的第一个字作为此韵部的名称。
“十三元”即上平声中以“元”字起首的第十三韵部的简称。
上平声从“一东”到“十五删”共十五个韵部,下平声第一个韵部为“一先”,
所以黛玉说“若十六根,便是‘一先’起。”
}这韵少,作排律只怕牵强不能押韵呢。
少不得你先起一句罢了。
”黛玉笑道:“倒要试试咱们谁强谁弱,只是没有纸笔记。
”湘云道:“不妨,明儿再写。
只怕这一点聪明还有。
”黛玉道:“我先起一句现成的俗语罢。
”因念道:\par
\hop
三五中秋夕,\par
\hop
湘云想了一想,道:\par
\hop
清游拟上元。
撒天箕斗灿,\zhu{拟上元:可与上元节相比。
上元:元宵节,阴历正月十五日。
箕(音“基”)斗:星宿名,南箕北斗,这里泛指群星。
}\par
\hop
林黛玉笑道:\par
\hop
匝地管弦繁。
\zhu{匝:周;遍。
}几处狂飞盏,\par
\hop
湘云笑道:“这一句‘几处狂飞盏’有些意思。
这倒要对的好呢。
”想了一想,笑道:\par
\hop
谁家不启轩。
\zhu{轩:窗子。
孟浩然《过故人庄》:“开轩面场圃,把酒话桑麻。
”}轻寒风剪剪,
\zhu{剪剪:形容轻微而带寒意的风。}
\par
\hop
黛玉道:“对的比我的却好。
只是底下这句又说熟话了,就该加劲说了去才是。
”湘云道:“诗多韵险,也要铺陈些才是。
纵有好的,且留在后头。
”黛玉笑道:“到后头没有好的,我看你羞不羞。
”因联道:\par
\hop
良夜景暄暄。
\zhu{暄,音“宣”,暖和。
这里应该是就心情而言。
}争饼嘲黄发,\zhu{下句即“嘲黄发之争饼”。
黄发:老年人。
争饼:争中秋的月饼。
此典生僻,有人以为或指唐僖宗命御厨以红绫扎饼赐曲江新科进士之事,因徐演有“莫欺老缺残牙齿,曾吃红绫饼馅来”的诗。
见宋秦再思《洛中记异》。
}\par
\hop
湘云笑道:“这句不好,是你杜撰,用俗事来难我了。
”黛玉笑道:“我说你不曾见过书呢。
吃饼是旧典,唐书唐志你看了来再说。
”湘云笑道:“这也难不倒我,我也有了。
”因联道:\par
\hop
分瓜笑绿嫒。
香新荣玉桂,\zhu{上句即“笑绿媛之分瓜”。
绿媛:年轻女子;绿,指绿鬓,乌黑的头发。
分瓜:指切西瓜。
《燕京岁时记》:“八月十五日祭月。
其祭,果饼必圆,分瓜必牙错。
”“凡中秋供月,西瓜必参差切之,如莲花瓣状。
”下句意谓盛开的桂花飘散清香。
}\par
\hop
黛玉笑道:“分瓜可是实实的你杜撰了。
”湘云笑道:“明日咱们对查了出来大家看看,这会子别耽误工夫。
”
\ping{
苏轼也曾杜撰过典故。1057年22岁的苏轼参加礼部考试,写就的《刑赏忠厚之至论》有云:“皋陶曰‘杀之’三,尧曰‘宥之’三。”诸主文皆不知其出处……欧公问其出处,东坡笑曰:“想当然尔。”数公大笑。
}
黛玉笑道:“虽如此,下句也不好,不犯着又用‘玉桂’‘金兰’等字样来塞责。
”因联道:\par
\hop
色健茂金萱。
\zhu{萱草:又名忘忧草,俗称金针菜,也叫黄花菜。
古人常用萱堂代指母亲。
这句的意思是繁茂的萱草闪耀光彩。
}蜡烛辉琼宴,\par
\hop
湘云笑道:“‘金萱’二字便宜了你,省了多少力。
这样现成的韵被你得了,只是不犯着替他们颂圣去。
况且下句你也是塞责了。
”黛玉笑道:“你不说‘玉桂’,我难道强对个‘金萱’么?再也要铺陈些富丽,方才是即景之实事。
”湘云只得又联道:\par
\hop
觥筹乱绮园。
\zhu{
觥:古代的一种酒器;筹:行酒令的筹码。
绮:有花纹的丝织品,这里指美丽,华丽。
}分曹尊一令,\zhu{下句意谓尊令官一人之命,分出对手,行射覆、猜拳等酒令。
分曹:分出对手,如射覆中分射者和覆者。
曹,这里指对偶,即互作对手的人。
}\par
\hop
黛玉笑道:“下句好,只是难对些。
”因想了一想,联道:\par
\hop
射覆听三宣。
\zhu{射覆:射:猜。
覆:遮盖;隐藏。
射覆:原为古时的一种猜谜游戏,用碗盆等把某物遮盖起来,猜中者胜。
后来也作为酒令的一种,如这里覆者先用诗文、成语、典故等隐寓某一事物,射者猜度,用也隐寓该事物的另一诗文、成语、典故等揭谜底,若射者猜不出或猜错以及覆者误判射者的猜度时,都要罚酒。
宣:宣布酒令。
书中有“三宣牙牌令”。
}骰彩红成点,\par
\hop
湘云笑道:“‘三宣’有趣,竟化俗成雅了。
只是下句又说上骰子。
”少不得联道:\par
\hop
传花鼓滥喧。
晴光摇院宇,
\zhu{晴光:指月光。}
\par
\hop
黛玉笑道:“对的却好。
下句又溜了,只管拿些风月来塞责。
”湘云道:“究竟没说到月上,也要点缀点缀,方不落题。
”黛玉道:“且姑存之,明日再斟酌。
”因联道:\par
\hop
素彩接乾坤。
\zhu{素彩:指月光。}
赏罚无宾主,\par
\hop
湘云道:“又说他们作什么,不如说咱们。
”只得联道:\par
\hop
吟诗序仲昆。
\zhu{仲昆:排名次;定高低。
}构思时倚槛,\zhu{槛:音“剑”,栏杆。
}\par
\hop
黛玉道:“这可以入上你我了。
”因联道:\par
\hop
拟景或依门。
酒尽情犹在,\par
\hop
湘云说道:“是时候了。
”乃联道:\par
\hop
更残乐已谖。
渐闻语笑寂,\zhu{谖:音“宣”,忘记,引申为止歇。
}\par
\hop
黛玉说道:“这时候可知一步难似一步了。
”因联道:\par
\hop
空剩雪霜痕。
阶露团朝菌,\zhu{雪霜痕:代指月光。
朝菌:一种朝生暮死的菌类。
《庄子·逍遥游》:“朝菌不知晦朔”。
}\par
\hop
湘云笑道:“这一句怎么押韵,让我想想。
”因起身负手,想了一想,笑道:“够了,幸而想出一个字来,几乎败了。
”因联道:\par
\hop
庭烟敛夕棔。
秋湍泻石髓,\zhu{棔:音“昏”,即合欢树,一名马缨花,又叫夜合花。
夜间叶子成对相合。
石髓:即石钟乳,石上多孔隙。
}\par
\hop
黛玉听了,不禁也起身叫妙,说:“这促狭鬼,\zhu{促狭:刁钻机灵,爱捉弄人。
}果然留下好的。
这会子才说‘棔’字,亏你想得出。
”湘云道:“幸而昨日看历朝文选见了这个字,我不知是何树,因要查一查。
宝姐姐说不用查,这就是如今俗叫作明开夜合的。
我信不及,到底查了一查,果然不错。
看来宝姐姐知道的竟多。
”黛玉笑道:“‘棔’字用在此时更恰,也还罢了。
只是‘秋湍’一句亏你好想。
只这一句,别的都要抹倒。
我少不得打起精神来对一句,只是再不能似这一句了。
”因想了一想,道:\par
\hop
风叶聚云根。
\zhu{云根:指山石。
古人认为山间云气生于山石,故称之为云根。
}宝婺情孤洁,\zhu{婺:音“务”,婺女。
星宿名。
这里代指秋星。
这句意谓秋星孤高晶洁。
}\par
\hop
湘云道:“这对的也还好。
只是下一句你也溜了,幸而是景中情,不单用‘宝婺’来塞责。
”因联道:\par
\hop
银蟾气吐吞。
\zhu{银蟾:指月中蟾蜍。
《后汉书》注:羿请不死之药于西王母,姮娥窃之以奔月,是为蟾蜍。
气吐吞:古人把云层遮月而过说成是月中蟾蜍在吞吐云气。
这句意谓月中的银蟾吞吐云气,遮住月亮的光辉。
}药经灵兔捣,\zhu{灵兔:传说月中有白兔捣药。
}\par
\hop
黛玉不语点头,半日随念道:\par
\hop
人向广寒奔。
\zhu{广寒:广寒宫,即月宫。
人:传说嫦娥偷吃不死药而奔月。
}犯斗邀牛女,\zhu{邀:见面。
牛女:牵牛、织女两星或“牛郎织女”的省称。
}\par
\hop
湘云也望月点首,联道:\par
\hop
乘槎待帝孙。
\zhu{
槎[chá]:木制的筏子。
帝孙,也叫天孙,即织女星。
这一句和上一句用了同一个传说。
乘槎游仙的传说,见于《博物志》:银河与海相通,居海岛者,年年八月定期可见有木筏从水上来去。
有人便带了粮食,登上木筏而去,结果碰到了牛郎织女。
海上客乘槎游仙回来后,曾问方士严君平。
严说:“某年月日,客星犯牵牛宿。
”一算,正是他到天河的时候。
}虚盈轮莫定,\zhu{虚盈:指月的圆缺。
轮:月轮;月亮。
这句意谓月亮的圆缺变换不停。
}\par
\hop
黛玉笑道:“又用比兴了。
”\zhu{比兴:比,譬喻,以彼物比此物,有象征的效果。
兴,寄托,为触景生情,因事寄兴,有暗示的效果。
比、兴为诗经六义中的两类,古代儒者认为这两种手法便于描写和反映现实,并适合于表现社会政治内容。
}因联道:\par
\hop
晦朔魄空存。
\zhu{朔:阴历每月初一。
晦:阴历每月最末一天叫晦。
望:指月光满盈时,即农历小月十五日,大月十六日。
魄:月魄,指月的实体。
这句意谓每当晦、朔,月魄无光。
}壶漏声将涸,\par
\hop
湘云方欲联时,黛玉指池中黑影与湘云看道:“你看那河里怎么像个人在黑影里去了,敢是个鬼罢?”湘云笑道:“可是又见鬼了。
我是不怕鬼的,等我打他一下。
”因弯腰拾了一块小石片向那池中打去,只听打得水响,一个大圆圈将月影荡散复聚者几次。
\geng{写得出。
试思若非亲历其境者如何摹写得如此。
}只听那黑影里嘎然一声,却飞起一个大白鹤来,\geng{写得出。
}直往藕香榭去了。
黛玉笑道:“原来是他,猛然想不到,反吓了一跳。
”湘云笑道:“这个鹤有趣,倒助了我了。
”因联道:\par
\hop
窗灯焰已昏。
寒塘渡鹤影,\zhu{“寒塘”句:秋夜寒塘掠过飞鹤的身影。
}\par
\hop
林黛玉听了,又叫好,又跺足,说:“了不得,这鹤真是助他的了!这一句更比‘秋湍’不同,叫我对什么才好?‘影’字只有一个‘魂’字可对,况且‘寒塘渡鹤’何等自然,何等现成,何等有景且又新鲜,我竟要搁笔了。
”湘云笑道:“大家细想就有了,不然就放着明日再联也可。
”黛玉只看天,不理他,半日,猛然笑道:“你不必说嘴,我也有了,你听听。
”因对道:\par
\hop
冷月葬花魂。
\par
\zhu{“冷月葬花魂”,庚辰本原作“冷月葬死魂”,“死”点改为“诗”。
列藏、甲辰本作“冷月葬诗魂”。
戚序、蒙府、杨本均作“冷月葬花魂”。
“死”或以为系“花”形讹,或以为是“诗”音讹。
“花魂”、“诗魂”各有出典,仔细品味,一、“诗魂”关人事,“鹤影”和“花魂”写景,用“花魂”对“鹤影”的工对,要比用“诗魂”对“鹤影”的宽对,对仗更加工整;二、第二十六回:“花魂默默无情绪,鸟梦痴痴何处惊”,“葬花魂”对应了黛玉《葬花吟》,其中有“昨宵庭外悲歌发,知是花魂与鸟魂”、“花魂鸟魂总难留,鸟自无言花自羞”,由此可见,花鸟相对,是作者一贯文风。
三、作者以花喻人,以花落喻人亡,以红消香断喻红颜薄命,突出体现在黛玉的《葬花吟》和《桃花行》,都是黛玉夭亡的谶语,这里用“葬花魂”,同样暗示了黛玉的悲剧命运。
四、从《红楼梦》继承我国丰富的文学遗产来看,也证明“葬花魂”是原文。
明代有名的天才早熟的才女叶小鸾,她十七岁就不幸夭亡。
其父叶绍袁在他所著的《续窈闻记》中记载了这样的一个故事:叶小鸾死后,某大师召来她的灵魂,女魂表示愿从师受戒。
大师说,受戒之先,必须审戒,便审问她生前种种罪过。
她都一一以诗句相答,语极绮丽。
比如,师问:“曾犯杀否?”女答:“犯。
”师问如何。
她说:“曾呼小玉除花虱,也遣轻纨坏蝶衣。
”师问:“曾犯淫否?”女答:“犯。
——晓镜偷窥眉曲曲,春裙新绣鸟双双。
”师问:“曾恶口否?”女答:“犯。
——生怕帘开讥燕子,为怜花谢骂东风。
”如此等等,共问答了十个问题,最后一个问题是:“曾犯痴否?”女答:“犯。
——勉弃珠环收汉玉,戏捐粉盒葬花魂。
”师大为赞叹说:“实在你只有一种罪,就是会做绮丽的诗。
”在这里,天真无邪、才华洋溢,而又不幸早夭的叶小鸾,不是与大观园里才冠群芳的林黛玉颇有相似之处吗?特别是以“葬花魂”为“痴”,不是更使人联想到《红楼梦》中有关葬花情节的描写吗?值得注意的是曹雪芹是曾经看过《续窈闻记》的。
综上,“花魂”艺术上稍胜,今从之。
}\par
\hop
湘云拍手赞道:“果然好极!非此不能对。
好个‘葬花魂’!”因又叹道:“诗固新奇,只是太颓丧了些。
你现病着,不该作此过于清奇诡谲之语。
”黛玉笑道:“不如此,如何压倒你。
下句竟还未得,只为用工在这一句了。
”\par
一语未了,只见栏外山石后转出一个人来,笑道:“好诗,好诗,果然太悲凉了。
不必再往下联,若底下只这样去,反不显这两句了,倒觉得堆砌牵强。
”二人不防,倒唬了一跳。
细看时,不是别人,却是妙玉。
二人皆诧异,\geng{原可诧异,余亦诧异。
}因问:“你如何到了这里?”妙玉笑道:“我听见你们大家赏月,又吹的好笛,我也出来玩赏这清池皓月。
顺脚走到这里,忽听见你两个联诗,更觉清雅异常,故此听住了。
只是方才我听见这一首中,有几句虽好,只是过于颓败凄楚。
此亦关人之气数而有,\zhu{关:关系,涉及。
}所以我出来止住。
如今老太太都已早散了,满园的人想俱已睡熟了,你两个的丫头还不知在那里找你们呢。
你们也不怕冷了?快同我来,到我那里去吃杯茶,只怕就天亮了。
”黛玉笑道:“谁知道就这个时候了。
”\par
三人遂一同来至栊翠庵中。
只见龛焰犹青,\zhu{龛:音“刊”,供奉神佛像或祖先牌位的石室或橱柜。
龛焰:指龛中的灯火。
}炉香未烬。
几个老嬷嬷也都睡了,只有小丫鬟在蒲团上垂头打盹。
\zhu{蒲团:用蒲草、高粱叶或玉米皮等编成的圆形垫子。
是农村中常见的坐具,也是僧尼、道士坐禅或跪拜的用具。
}妙玉唤他起来,现去烹茶。
忽听叩门之声,小丫鬟忙去开门看时,却是紫鹃翠缕与几个老嬷嬷来找他姊妹两个。
进来见他们正吃茶,因都笑道:“要我们好找,一个园里走遍了,连姨太太那里都找到了。
才到了那山坡底下小亭里找时,可巧那里上夜的正睡醒了。
我们问他们,他们说,方才亭外头棚下两个人说话,后来又添了一个,听见说大家往庵里去。
我们就知是这里了。
”妙玉忙命小丫鬟引他们到那边去坐着歇息吃茶。
自取了笔砚纸墨出来,将方才的诗命他二人念着,遂从头写出来。
黛玉见他今日十分高兴,便笑道:“从来没见你这样高兴,我也不敢唐突请教。
这还可以见教否?
\zhu{见教:指教(我)。}
若不堪时,便就烧了;若或可政,\zhu{政:通“正”。
}即请改正改正。
”妙玉笑道:“也不敢妄加评赞。
只是这才有了二十二韵。
我意思想着你二位警句已出,再若续时,恐后力不加。
我竟要续貂,\zhu{续貂:古代近侍官员以貂尾为冠饰,朝廷滥任官吏,时人讽刺曰:“貂不足,狗尾续”,见《晋书·赵王伦传》。
后常以“续貂”作为自谦之词,表示佳作在前,难以为继。
}又恐有玷。
”
\zhu{玷[diàn]:白玉上面的污点;使有污点。}
黛玉从没见妙玉作过诗,今见他高兴如此,忙说:“果然如此,我们的虽不好,亦可以带好了。
”妙玉道:“如今收结,到底还该归到本来面目上去。
若只管丢了真情真事且去搜奇捡怪,一则失了咱们的闺阁面目,\zhu{闺阁面目:指诗之格调合乎闺阁小姐的身分情趣。
}二则也与题目无涉了。
”二人皆道极是。
妙玉遂提笔一挥而就,递与他二人道:“休要见笑。
依我必须如此,方翻转过来,虽前头有凄楚之句,亦无甚碍了。
”二人接了看时,只见他续道:\par
\hop
香篆销金鼎,脂冰腻玉盆。
\zhu{香篆(篆音“撰”):即篆香,一种记时用的盘香,形似篆文。
脂冰:即冰脂,亦即凝脂(古文“冰”“凝”通),这里指凝固了的蜡油。
腻玉盆:凝于烛盆中。
另一种说法,脂冰是指冰雪般的肌肤上的香脂。
那么腻玉盆的意思应该是盥洗时肌肤上的香脂留在了盆里。
}\par
箫增嫠妇泣,衾倩侍儿温。
\zhu{
嫠(音“离”)妇:寡妇。
上句意谓呜呜洞箫,增添寡妇泣声的悲戚。
倩:音“庆”,请别人代自己做事。
}\par
空帐悬文凤,闲屏掩彩鸳。
\par
露浓苔更滑,霜重竹难扪。
\par
犹步萦纡沼,还登寂历原。
\zhu{
纡[yū]:弯曲;曲折。
萦纡:回旋盘曲。
历:逐个,一一地。
引申为分明的、清晰的。
如“历历在目”、“晴川历历汉阳树,春草萋萋鹦鹉洲”。
寂历:寂静空旷。
}\par
石奇神鬼搏,木怪虎狼蹲。
\zhu{上句意谓石头奇形怪状,像神鬼在搏斗。
}\par
赑屃朝光透,罘罳晓露屯。
\zhu{赑屃:音“币戏”,传说其力大,能负重,故大碑的石座多雕作它的形状。
这里代指石碑。
罘罳:音“浮思”,古代设在宫门外或城角上多孔的屏障,用以瞭望和防御。
这里泛指门外用作屏障的有孔的篱垣。
屯:凝聚。
}\par
振林千树鸟,啼谷一声猿。
\par
歧熟焉忘径,泉知不问源。
\zhu{歧熟焉忘径:歧:道路分岔处。
《列子·说符》:“大道以多歧亡羊,学者以多方丧生。
”后常以“歧路亡羊”比喻事理复杂多变,没有正确方向容易误入迷途,就有可能像邻人走失羊一样,再也找不回来。
这里是反其意而用,意谓岔道都很熟悉,哪会迷路呢?
}\par
钟鸣栊翠寺,鸡唱稻香村。
\par
有兴悲何继,无愁意岂烦。
\par
芳情只自遣,雅趣向谁言。
\par
彻旦休云倦,烹茶更细论。
\zhu{细论:仔细品评。
杜甫《春日忆李白》诗:“何时一尊酒,重与细论文?”}\par
\hop
后书:右中秋夜大观园即景联句三十五韵。
\zhu{右:通「佑」,辅助。
因为这首诗是湘云黛玉口述了前半部分,而由妙玉笔录并续了后半部分,从妙玉的角度,是“辅助”了湘云黛玉。
}\par
黛玉湘云二人皆赞赏不已,说:“可见我们天天是舍近而求远。
现有这样诗仙在此,却天天去纸上谈兵。
”妙玉笑道:“明日再润色。
此时想也快天亮了,到底要歇息歇息才是。
”林史二人听说,便起身告辞,带领丫鬟出来。
妙玉送至门外,看他们去远,方掩门进来。
不在话下。
\par
这里翠缕向湘云道:“大奶奶那里还有人等着咱们睡去呢。
如今还是那里去好?”湘云笑道:“你顺路告诉他们,叫他们睡罢。
我这一去未免惊动病人,不如闹林姑娘半夜去罢。
”说着,大家走至潇湘馆中,有一半人已睡去。
二人进去,方才卸妆宽衣,盥漱已毕,方上床安歇。
紫鹃放下绡帐,移灯掩门出去。
谁知湘云有择席之病,虽在枕上,只是睡不着。
黛玉又是个心血不足常常失眠的,今日又错过困头,自然也是睡不着。
二人在枕上翻来覆去。
黛玉因问道:“怎么你还没睡着?”湘云微笑道:“我有择席的病,况且走了困,只好躺躺罢。
你怎么也睡不着?”黛玉叹道:\geng{一“笑”一“叹”,只二字便写出平日之形景。
}“我这睡不着也并非今日,大约一年之中,通共也只好睡十夜满足的。
”湘云道:“都是你病的原故,所以……”不知下文什么——\par
\qi{总评:诗词清远闲旷,自是慧业才人,
\zhu{慧业:佛教用语,指智慧的业缘。业缘:佛教用语,善恶果报的因缘。}
何须赘评?须看他众人联句填词时,各人性情,各人意见,叙来恰肖其人;二人联诗时,一番讥评,一番叹赏,叙来更得其神。
再看漏永吟残,\zhu{漏永:即“更长漏永”。
更,古代夜间的计时单位,一夜分五更。
漏,漏壶,古代的计时器,内盛以水,均匀地向下滴。
更长漏永形容漫长的夜晚。
残:残缺,不完整。
吟残是指黛玉和湘云联诗未完成而妙玉来了。
}忽开一洞天福地,\zhu{洞天福地:神仙所住的地方。
比喻名山胜境。
}字字出人意表。
\zhu{表:外。
}\hang
只一品笛,疑有疑无,若近若远,有无限逸致。
}

\dai{151}{尤氏讲笑话,贾母朦胧似睡}
\dai{152}{凹晶馆联诗悲寂寞}
\sun{p76-1}{凹晶馆联诗悲寂寞}{湘云和黛玉去凹晶溪馆赏月,但见清池皓月,上下争辉,微风一过,粼粼然池面皱碧铺纹,真令人神清气净。
二人便联起诗来。
兴致正高时,妙玉走来,三人同去栊翠庵叙话。
}