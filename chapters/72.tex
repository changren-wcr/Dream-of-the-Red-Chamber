\chapter{王熙凤恃强羞说病 \quad 来旺妇倚势霸成亲}
\qi{此回似着意似不着意,似接续似不接续,在画师为浓淡相间,在墨客为骨肉停匀,\zhu{停匀:均衡;均匀。
}在乐工为笙歌间作,在文坛为养局为别调。
\zhu{养局:词义不明,根据上下文,应该是指不按主线一气直叙到底,而是穿插支线,起到了蓄势铺垫的作用。
}前后文气,至此一歇。
}\par
且说鸳鸯出了角门,脸上犹红,心内突突的,真是意外之事。
因想这事非常,若说出来,奸盗相连,关系人命,还保不住带累了旁人。
横竖与自己无干,且藏在心内,不说与一人知道。
回房复了贾母的命,大家安息。
从此凡晚间便不大往园中来。
因思园中尚有这样奇事,何况别处,因此连别处也不大轻走动了。
\par
原来那司棋因从小儿和他姑表兄弟在一处顽笑起住时,小儿戏言,便都订下将来不娶不嫁。
近年大了,彼此又出落的品貌风流。
常时司棋回家时,二人眉来眼去,旧情不忘,只不能入手。
又彼此生怕父母不从,二人便设法彼此里外买嘱园内老婆子们留门看道,今日趁乱方初次入港。
\zhu{入港:男女发生性关系的隐晦表达。
}虽未成双,却也海誓山盟,私传表记,\zhu{表记:纪念品、信物。
}已有无限风情了。
忽被鸳鸯惊散,那小厮早穿花度柳,从角门出去了。
司棋一夜不曾睡着,又后悔不来。
\zhu{后悔不来:表示事情已经发生,无法追悔或补救。}
直至次日见了鸳鸯,自是脸上一红一白,百般过不去。
心内怀着鬼胎,茶饭无心,起坐恍惚。
挨了两日,竟不听见有动静,方略放下了心。
这日晚间,忽有个婆子来悄告诉他道:“你兄弟竟逃走了,三四天没归家。
如今打发人四处找他呢。
”司棋听了,气个倒仰,因思道:“纵是闹了出来,也该死在一处。
他自为是男人,先就走了,可见是个没情意的。
”因此又添了一层气。
次日便觉心内不快,百般支持不住,一头睡倒,恹恹的成了大病。
\par
鸳鸯闻知那边无故走了一个小厮,园内司棋又病重,要往外挪,心下料定是二人惧罪之故,“生怕我说出来,方吓到这样。
”因此自己反过意不去,指着来望候司棋,支出人去,反自己立身发誓,与司棋说:“我告诉一个人,立刻现死现报!你只管放心养病,别白糟踏了小命儿。
”司棋一把拉住,哭道:“我的姐姐,咱们从小儿耳鬓厮磨,你不曾拿我当外人待,我也不敢待慢了你。
\zhu{待慢:疏失、招待不周全。
也作“怠慢”。
}如今我虽一着走错,你若果然不告诉一个人,你就是我的亲娘一样。
从此后我活一日是你给我一日,我的病好之后,把你立个长生牌位,我天天焚香礼拜,保佑你一生福寿双全。
我若死了时,变驴变狗报答你。
再俗语说:‘千里搭长棚,没有不散的筵席。
’再过三二年,咱们都是要离这里的。
俗语又说:‘浮萍尚有相逢日,人岂全无见面时。
’倘或日后咱们遇见了,那时我又怎么报你的德行。
”一面说,一面哭。
这一席话反把鸳鸯说的心酸,也哭起来了。
因点头道:“正是这话。
我又不是管事的人,何苦我坏你的声名,我白去献勤。
况且这事我自己也不便开口向人说。
你只放心。
从此养好了,可要安分守己,再不许胡行乱作了。
”司棋在枕上点首不绝。
\par
鸳鸯又安慰了他一番,方出来。
因知贾琏不在家中,又因这两日凤姐儿声色怠惰了些,不似往日一样,因顺路也来望候。
因进入凤姐院门,二门上的人见是他来,便立身待他进去。
鸳鸯刚至堂屋中,只见平儿从里间出来,见了他来,忙上来悄声笑道:“才吃了一口饭歇了午睡,你且这屋里略坐坐。
”鸳鸯听了,只得同平儿到东边房里来。
小丫头倒了茶来。
鸳鸯因悄问:“你奶奶这两日是怎么了?我看他懒懒的。
”平儿见问,因房内无人,便叹道:“他这懒懒的也不止今日了,这有一月之前便是这样。
又兼这几日忙乱了几天,又受了些闲气,从新又勾起来。
\zhu{从新:重新。
}
这两日比先又添了些病,所以支持不住,便露出马脚来了。
”\zhu{露出马脚:暴露出真相或漏洞。
}鸳鸯忙道:“既这样,怎么不早请大夫来治?”平儿叹道:“我的姐姐,你还不知道他的脾气的。
别说请大夫来吃药。
我看不过,白问了一声身上觉怎么样,\zhu{白:单单,只是。
}他就动了气,反说我咒他病了。
饶这样,\zhu{饶:即使,尽管,表示让步关系。
}天天还是察三访四,自己再不肯看破些且养身子。
”鸳鸯道:“虽然如此,到底该请大夫来瞧瞧是什么病,也都好放心。
”平儿道:“我的姐姐,说起病来,据我看也不是什么小症候。
”鸳鸯忙道:“是什么病呢?”平儿见问,又往前凑了一凑,向耳边说道:“只从上月行了经之后,这一个月竟沥沥淅淅的没有止住。
这可是大病不是?”鸳鸯听了,忙答道:“嗳哟!依你这话,这可不成了血山崩了。
”\zhu{血山崩:中医病症名。
妇女不在行经期间,阴道内大量出血,或月经刚停,仍续见下血、淋沥不断叫“崩漏”。
出血量多而来势急剧的叫“血崩”或“血山崩”。
}平儿忙啐了一口,又悄笑道:“你女孩儿家,这是怎么说的,倒会咒人呢。
”鸳鸯见说,不禁红了脸,又悄笑道:“究竟我也不知什么是崩不崩的,你倒忘了不成,先我姐姐不是害这病死了。
我也不知是什么病,因无心听见妈和亲家妈说,我还纳闷,后来也是听见妈细说原故,才明白了一二分。
”平儿笑道:“你该知道的,我竟也忘了。
”\par
二人正说着,只见小丫头进来向平儿道:“方才朱大娘又来了。
我们回了他奶奶才歇午觉,他往太太上头去了。
”平儿听了点头。
鸳鸯问:“那一个朱大娘?”平儿道:“就是官媒婆那朱嫂子。
\zhu{官媒婆:旧时衙门中的女差役。
承办择配女犯或官僚贵族之家放出婚配的女奴,还承担女犯的押解伴送等事。
官媒婆也指以做媒为业的妇女。
}因有什么孙大人家来和咱们求亲,所以他这两日天天弄个帖子来赖死赖活。
”一语未了,小丫头跑来说:“二爷进来了。
”说话之间,贾琏已走至堂屋门,口内唤平儿。
平儿答应着才迎出去,贾琏已找至这间房内来。
至门前,忽见鸳鸯坐在炕上,便煞住脚,笑道:“鸳鸯姐姐,今儿贵脚踏贱地。
”鸳鸯只坐着,笑道:“来请爷奶奶的安,偏又不在家的不在家,睡觉的睡觉。
”贾琏笑道:“姐姐一年到头辛苦伏侍老太太,我还没看你去,那里还敢劳动来看我们。
正是巧的很,我才要找姐姐去。
因为穿着这袍子热,先来换了夹袍子再过去找姐姐,\zhu{夹:有面有里,中间不衬垫絮类。
}不想天可怜,省我走这一趟,姐姐先在这里等我了。
”一面说,一面在椅上坐下。
\par
鸳鸯因问:“又有什么说的?”贾琏未语先笑道:“因有一件事,我竟忘了,只怕姐姐还记得。
上年老太太生日,曾有一个外路和尚来孝敬一个蜡油冻的佛手,\zhu{外路和尚:即“行脚僧”、云游四方的和尚。
蜡油冻的佛手:用黄色蜜蜡冻石雕刻成的佛手。
冻石:是一种半透明的名贵石头。
佛手:即“佛手柑”,色黄,形如手指攥聚,有浓香。
}因老太太爱,就即刻拿过来摆着了。
因前日老太太生日,我看古董帐上还有这一笔,却不知此时这件东西着落何方。
古董房里的人也回过我两次,等我问准了好注上一笔。
所以我问姐姐,如今还是老太太摆着呢,还是交到谁手里去了呢?”鸳鸯听说,便道:“老太太摆了几日厌烦了,就给了你们奶奶。
你这会子又问我来。
我连日子还记得,还是我打发了老王家的送来的。
你忘了,或是问你们奶奶和平儿。
”平儿正拿衣服,听见如此说,忙出来回说:“交过来了,现在楼上放着呢。
奶奶已经打发过人出去说过给了这屋里,他们发昏,没记上,又来叨登这些没要紧的事。
”贾琏听说,笑道:“既然给了你奶奶,我怎么不知道,你们就昧下了。
”平儿道:“奶奶告诉二爷,二爷还要送人,奶奶不肯,好容易留下的。
这会子自己忘了,倒说我们昧下。
那是什么好东西,什么没有的物儿。
比那强十倍的东西也没昧下一遭,这会子爱上那不值钱的!”贾琏垂头含笑想了一想,拍手道:“我如今竟糊涂了!丢三忘四,惹人抱怨,竟大不像先了。
”鸳鸯笑道:“也怨不得。
事情又多,口舌又杂,你再喝上两杯酒,那里清楚的许多。
”一面说,一面就起身要去。
\par
贾琏忙也立身说道:“好姐姐,再坐一坐,兄弟还有事相求。
”说着便骂小丫头:“怎么不潗好茶来!\zhu{潗茶:同“沏茶”。
}快拿干净盖碗,\zhu{盖碗:一种上有盖、下有托,中有碗的汉族茶具。
又称“三才碗”、“三才杯”,盖为天、托为地、碗为人,暗含天地人和之意。
}把昨儿进上的新茶潗一碗来。
”\zhu{上:皇帝。
}说着向鸳鸯道:“这两日因老太太的千秋,所有的几千两银子都使了。
几处房租地税通在九月才得,\zhu{通:全、都。
}这会子竟接不上。
明儿又要送南安府里的礼,又要预备娘娘的重阳节礼,还有几家红白大礼,至少还得三二千两银子用,一时难去支借。
\ping{
从贾府财政上因强撑台面而入不敷出,显露出贾府衰败气象。
第十五回,王熙凤弄权铁槛寺,王熙凤对净虚说:“……这三千银子,不过是给打发说去的小厮作盘缠,使他赚几个辛苦钱,我一个钱也不要他的。
便是三万两,我此刻也拿的出来。
”第十六回,预备元春省亲,贾蔷道:“……赖爷爷说,竟不用从京里带下去,江南甄家还收着我们五万银子。
明日写一封书信,会票我们带去,先支三万,下剩二万存着,等置办花烛彩灯并各色帘栊帐幔的使费。
”}俗语说,‘求人不如求己’。
说不得,姐姐担个不是,暂且把老太太查不着的金银家伙偷着运出一箱子来,暂押千数两银子支腾过去。
不上半年的光景,银子来了,我就赎了交还,断不能叫姐姐落不是。
”鸳鸯听了,笑道:“你倒会变法儿,亏你怎么想来。
”贾琏笑道:“不是我扯谎,若论除了姐姐,也还有人手里管的起千数两银子的,只是他们为人都不如你明白有胆量。
我若和他们一说,反吓住了他们。
所以我‘宁撞金钟一下,不打破鼓三千’。
”一语未了,忽有贾母那边的小丫头子忙忙走来找鸳鸯,说:“老太太找姐姐半日,我们那里没找到,却在这里。
”鸳鸯听说,忙的且去见贾母。
\par
贾琏见他去了,只得回来瞧凤姐。
谁知凤姐已醒了,听他和鸳鸯借当,自己不便答话,只躺在榻上。
听见鸳鸯去了,贾琏进来,凤姐因问道:“他可应准了?”贾琏笑道:“虽然未应准,却有几分成手,\zhu{成手:成功,得手。
}须得你晚上再和他一说,就十分成了。
”凤姐笑道:“我不管这事。
倘或说准了,这会子说得好听,到有了钱的时节,你就丢在脖子后头,谁去和你打饥荒去。
\zhu{打饥荒:此指纠缠不休,找麻烦。
}倘或老太太知道了,倒把我这几年的脸面都丢了。
”贾琏笑道:“好人,你若说定了,我谢你如何?”凤姐笑道:“你说,谢我什么?”贾琏笑道:“你说要什么就要什么。
”平儿一旁笑道:“奶奶倒不要谢的。
昨儿正说要作一件什么事,恰少一二百银子使,不如借了来,奶奶拿一二百银子,岂不两全其美。
”凤姐笑道:“幸亏提起我来,就是这样也罢。
”贾琏笑道:“你们太也狠了。
你们这会子别说一千两的当头,\zhu{当头:向当铺借钱时所用的抵押品。
}就是现银子要三五千,只怕也难不倒。
我不和你们借就罢了。
这会子烦你说一句话,还要个利钱,真真了不得。
”凤姐听了,翻身起来说:“我有三千五万,不是赚的你的。
如今里里外外上上下下背着我嚼说我的不少,就差你来说了,可知没家亲引不出外鬼来。\zhu{
家亲:原指已故的亲人,义同“家神”。《地藏菩萨本愿经·如来赞叹品第六》:“或夜梦恶鬼,乃及家亲。”这里借喻“内鬼”。俗语有“家(内)神通外鬼”之语,意即内外勾结。
没家亲引不出外鬼:俗语。
意思是说没有内部的人捣乱,就不会引来外面敌视或仇视自己的人,没有家里的人说坏话,就引不出外面的流言蜚语。
}
我们王家可那里来的钱,都是你们贾家赚的。
别叫我恶心了。
你们看着你们家,什么石崇、邓通!把我王家的地缝子扫一扫,就够你们过一辈子呢。
说出来的话也不怕臊!现有对证:把太太和我的嫁妆细看看,比一比你们的,那一样是配不上你们的。
”贾琏笑道:“说句顽话就急了。
这有什么这样的,要使一二百两银子值什么,多的没有,这还有,先拿进来,你使了再说,如何?”凤姐道:“我又不等着衔口垫背,\zhu{衔口垫背:旧俗,殓葬时给死者口中含珠玉或米粮,叫“衔口”;在死者褥下放钱物叫“垫背”。
}忙了什么。
”贾琏道:“何苦来,不犯着这样肝火盛。
”
\zhu{肝火:指容易急躁发怒的情绪。}
凤姐听了,又自笑起来,“不是我着急,你说的话戳人的心。
我因为我想着后日是尤二姐的周年,我们好了一场,虽不能别的,到底给他上个坟烧张纸,也是姊妹一场。
他虽没留下个男女,也要‘前人撒土迷了后人的眼’才是。
”\zhu{前人撒土迷了后人的眼:俗语。
“前人撒土”是引子,“迷了后人的眼”才是本意:遮掩后人的眼睛,做做样子给人看。
清·李光庭《乡言解颐》卷一“土”条:“前人撒土迷后人眼,谓含糊了事者也。”
“也要”在蒙府本和戚序本中作“也不要”。
“也不要”用在这里的意思是说给尤二姐办周年,总要说得过去才行,不能含糊了事。
从书中前后叙述和凤姐话意,都看不出有“隆重”去给尤二姐上坟的意思,“也要”也可以解释得通。
}一语倒把贾琏说没了话,低头打算了半晌,方道:“难为你想的周全,我竟忘了。
既是后日才用,若明日得了这个,你随便使多少就是了。
”\ping{凤姐会做好人,设计害死尤二姐,回头略施小恩小惠就能让贾琏感动。
}\par
一语未了,只见旺儿媳妇走进来。
凤姐便问:“可成了没有?”旺儿媳妇道:“竟不中用。
我说须得奶奶作主就成了。
”贾琏便问:“又是什么事?”凤姐儿见问,便说道:“不是什么大事。
旺儿有个小子,今年十七岁了,还没得女人,因要求太太房里的彩霞,不知太太心里怎么样,就没有计较得。
\zhu{计较:打算。
得:得到。
}前日太太见彩霞大了,二则又多病多灾的,因此开恩打发他出去了,给他老子娘随便自己拣女婿去罢。
因此旺儿媳妇来求我。
我想他两家也就算门当户对的,一说去自然成的,谁知他这会子来了,说不中用。
”贾琏道:“这是什么大事,比彩霞好的多着呢。
”旺儿家的陪笑道:“爷虽如此说,连他家还看不起我们,别人越发看不起我们了。
好容易相看准一个媳妇,我只说求爷奶奶的恩典,替作成了。
奶奶又说他必肯的,我就烦了人走过去试一试,谁知白讨了没趣。
若论那孩子倒好,据我素日私意儿试他,他心里没有甚说的,只是他老子娘两个老东西太心高了些。
”一语戳动了凤姐和贾琏,凤姐因见贾琏在此,且不作一声,只看贾琏的光景。
贾琏心中有事,那里把这点子事放在心里。
待要不管,只是看着他是凤姐儿的陪房,且又素日出过力的,脸上实在过不去,因说道:“什么大事,只管咕咕唧唧的。
你放心且去,我明儿作媒打发两个有体面的人,一面说,一面带着定礼去,就说我的主意。
他十分不依,叫他来见我。
”旺儿家的看着凤姐,凤姐便扭嘴儿。
旺儿家的会意,忙爬下就给贾琏磕头谢恩。
\ping{彩霞的婚姻不仅是父母包办,更是主子包办。
}\par
贾琏忙道:“你只给你姑娘磕头。
我虽如此说了这样行,到底也得你姑娘打发个人叫他女人上来,和他好说更好些。
虽然他们必依,然这事也不可霸道了。
”凤姐忙道:“连你还这样开恩操心呢,我倒反袖手旁观不成。
旺儿家你听见,说了这事,你也忙忙的给我完了事来。
说给你男人,外头所有的帐,一概赶今年年底下收了进来,少一个钱我也不依的。
我的名声不好,再放一年,都要生吃了我呢。
”旺儿媳妇笑道:“奶奶也太胆小了。
谁敢议论奶奶,若收了时,公道说,我们倒还省些事,不大得罪人。
”凤姐冷笑道:“我也是一场痴心白使了。
我真个的还等钱作什么,不过为的是日用出的多,进的少。
这屋里有的没的,我和你姑爷一月的月钱,再连上四个丫头的月钱,通共一二十两银子,还不够三五天的使用呢。
若不是我千凑万挪的,早不知道到什么破窑里去了。
如今倒落了一个放帐破落户的名儿。
\geng{可知放帐乃发。
\ping{全书涉及到凤姐的放账有以下三处描写:第三回里王夫人曾问王熙凤“月钱放过了不曾”。
此句有脂砚斋的旁批:“不见后文,不见此笔之妙。
”第十六回里平儿将送利钱来的旺儿媳妇支开,就是因为贾琏在家要有意隐瞒。
第三十九回里袭人催促月钱发放,平儿谈及王熙凤挪用月钱放债,并嘱咐袭人不要泄密。
到本回凤姐对自己放债之事不再讳莫如深,提到了针对自己放债的流言蜚语,也不再躲着贾琏。
由此可见,凤姐挪用月钱放债由来已久,一开始还是遮遮掩掩,连贾琏也不知道,到后来逐渐为大家所知道,甚至有了“放帐破落户”的骂名。
凤姐放债的目的究竟是为了补贴贾府开支还是中饱私囊了呢?第三十九回里袭人认为王熙凤放债只是为自己敛财:“难道他还短钱使,还没个足厌?何苦还操这心。
”平儿“何曾不是呢”的呼应,证明了袭人理解无误。
如果放债真的是为缓解府内经济困难,她何至于如此隐秘。
反过来,王熙凤占公家便宜倒是常有的事。
她曾请大家吃小荷叶儿汤,由厨房买单,贾母就批评她“拿着官中的钱你做人”。
前前后后为凤姐放债的是旺儿,此外王熙凤受馒头庵老尼之托拆散守备之子与金哥的婚事,其间伪造文书、走节度使云光门路等事,旺儿“两日工夫俱已妥协”,王熙凤因此“坐享了三千两”。
凤姐赚钱的私密事情都由旺儿打理,所以要加意笼络,这才非要替旺儿的儿子说成这门亲事。
}所谓“此家儿知耻恶”之事也。
\zhu{家儿:子弟。
这里应该是指王熙凤。
}}既这样,我就收了回来。
我比谁不会花钱,咱们以后就坐着花,到多早晚是多早晚。
这不是样儿:前儿老太太生日,太太急了两个月,想不出法儿来,还是我提了一句,后楼上现有些没要紧的大铜锡家伙四五箱子,拿去弄了三百银子,才把太太遮羞礼儿搪过去了。
\ping{三百两拿不出,贾府财政捉襟见肘。}
我是你们知道的,那一个金自鸣钟卖了五百六十两银子。
没有半个月,大事小事倒有十来件,白填在里头。
今儿外头也短住了,\zhu{短:引申为不足,缺乏。
}不知是谁的主意,搜寻上老太太了。
明儿再过一年,各人搜寻到头面衣服,可就好了!”旺儿媳妇笑道:“那一位太太奶奶的头面衣服折变了不够过一辈子的,只是不肯罢了。
”\geng{闲语,补出近日诸事。
}凤姐道:“不是我说没了能耐的话,要像这样,我竟不能了。
昨晚上忽然作了一个梦,说来也可笑,\geng{反说“可笑”,妙甚!若必以此梦为凶兆,则思反落套,
\zhu{落套:落入俗套。}
非红楼之梦矣。
}梦见一个人,虽然面善,却又不知名姓,\geng{是以前授方相之旧,\zhu{
方相:古代逐疫驱鬼和山川精怪的神灵。神像丑陋恐怖,出丧时常置于行列前开道。
}数十年后矣。
}
找我。
问他作什么,他说娘娘打发他来要一百匹锦。
我问他是那一位娘娘,他说的又不是咱们家的娘娘。
我就不肯给他,他就上来夺。
正夺着,就醒了。
”\geng{妙!实家常触景闲梦,必有之理,却是江淹才尽之兆也,\zhu{江淹才尽:即“江郎才尽”,江郎指南朝梁文人江淹。
江郎才尽原指江淹少以诗文著称,独步当代。
晚年时,偶憩于一亭中,梦见郭璞索还五色笔,自此江淹作诗绝无佳句。
典出《南史·卷五九·江淹传》。
}可伤。
}
旺儿家的笑道:“这是奶奶的日间操心,常应候宫里的事。
”\geng{淡淡抹去,妙!}\par
一语未了,人回:“夏太府打发了一个小内监来说话。
”\zhu{
太府:官名,这里指太监职位较高者。
内监:内宫的太监。
}贾琏听了,忙皱眉道:“又是什么话,一年他们也搬够了。
”凤姐道:“你藏起来,等我见他,若是小事罢了,若是大事,我自有话回他。
”贾琏便躲入内套间去。
这里凤姐命人带进小太监来,让他椅子上坐了吃茶,因问何事。
那小太监便说:“夏爷爷因今儿偶见一所房子,如今竟短二百两银子,打发我来问舅奶奶家里,有现成的银子暂借一二百,过一两日就送过来。
”\geng{可谓“密处不容针”。
}凤姐儿听了,笑道:“什么是送过来,有的是银子,只管先兑了去。
改日等我们短了,再借去也是一样。
”\ping{不论贾家是否富余,掌权的太监都可以同样地以借为幌子索贿。
}\ping{太监靠近权力核心,不能得罪,需要好生伺候。
}小太监道:“夏爷爷还说了,上两回还有一千二百两银子没送来,等今年年底下,自然一齐都送过来。
”凤姐笑道:“你夏爷爷好小气,这也值得提在心上。
我说一句话,不怕他多心,若都这样记清了还我们,不知还了多少了。
只怕没有,若有,只管拿去。
”因叫旺儿媳妇来,“出去不管那里先支二百两来。
”旺儿媳妇会意,因笑道:“我才因别处支不动,才来和奶奶支的。
”凤姐道:“你们只会里头来要钱,叫你们外头算去就不能了。
”说着叫平儿,“把我那两个金项圈拿出去,暂且押四百两银子。
”\ping{用典当的钱打发太监,一方面表示自己家真的穷,希望太监去别处搜刮油水,不要再蚊子腿上找肉吃;另一方面表示即使我家这么穷,我也要满足太监的欲求,更加突出了自己的孝顺。
}平儿答应了,去半日,果然拿了一个锦盒子来,里面两个锦袱包着。
打开时,一个金累丝攒珠的,\zhu{累:堆叠,重叠。
}那珍珠都有莲子大小,一个点翠嵌宝石的。
\zhu{点翠:首饰类贴以翡翠羽者。}
两个都与宫中之物不离上下。
\geng{是太监眼中看、心中评。
}一时拿去,果然拿了四百两银子来。
凤姐命与小太监打叠起一半,那一半命人与了旺儿媳妇,命他拿去办八月中秋的节。
\geng{过下伏脉。
}那小太监便告辞了,凤姐命人替他拿着银子,送出大门去了。
这里贾琏出来笑道:“这一起外祟何日是了!”\zhu{祟:音“岁”,灾祸。
}凤姐笑道:“刚说着,就来了一股子。
”贾琏道:“昨儿周太监来,张口一千两。
我略应慢了些,他就不自在。
将来得罪人之处不少。
这会子再发个三二百万的财就好了。
”
\ping{可能是林家的财产被贾府以抚养林黛玉的名义霸占了。第十三回,林如海身染重疾,写书要黛玉回南。贾母着贾琏相送。不料林姑老爷九月初三殁了,从扬州送灵回苏州。年底,贾琏将黛玉带回荣国府。可以猜测,因为林家已经没了至亲,一定是贾琏帮着林黛玉处理了家产。折变的银子应该随着黛玉带回荣国府。}
一面说,一面平儿伏侍凤姐另洗了面、更衣,往贾母处去伺候晚饭。
\par
这里贾琏出来,刚至外书房,忽见林之孝走来。
贾琏因问何事。
林之孝说道:“方才听得雨村降了,却不知因何事,只怕未必真。
”贾琏道:“真不真,他那官儿也未必保得长。
将来有事\foot{此处从庚本原文。
“将来有事”,联系上下文,当理解为“我们家将来有事”,诸本均作“雨村将来要出事”解,因而各补充若干字以足文义,非是。
},怕宁可疏远着他好。
”林之孝道:“何尝不是,只是一时难以疏远。
如今东府大爷和他更好,老爷又喜欢他,时常来往,那个不知。
”贾琏道:“横竖不和他谋事,也不相干。
你去再打听真了,是为什么。
”林之孝答应了,却不动身,坐在下面椅子上,且说些闲话。
因又说起家道艰难,便趁势又说:“人口太重了。
不如拣个空日回明老太太老爷,把这些出过力的老家人用不着的,开恩放几家出去。
一则他们各有营运,二则家里一年也省些口粮月钱。
再者里头的姑娘也太多。
俗语说:‘一时比不得一时。
’如今说不得先时的例了,少不得大家委屈些,该使八个的使六个,该使四个的便使两个。
若各房算起来,一年也可以省得许多月米月钱。
况且里头的女孩子们一半都太大了,也该配人的。
配人成了房,岂不又孳生出人来。
”贾琏道:“我也这样想着,只是老爷才回家来,多少大事未回,那里议到这个上头。
前儿官媒拿了个庚帖来求亲,\zhu{庚帖:也叫“年庚帖子”。
旧时订婚,男女双方互相交换的一种红色柬帖,上写订婚者的姓名、籍贯、生辰八字及祖宗三代等。
}太太还说老爷才来家,每日欢天喜地的说骨肉完聚,忽然就提起这事,恐老爷又伤心,所以且不叫提这事。
”林之孝道:“这也是正理,太太想的周到。
”\par
贾琏道:“正是,提起这话我想起了一件事来。
我们旺儿的小子要说太太房里的彩霞。
他昨儿求我,我想什么大事,不管谁去说一声去。
这会子有谁闲着,你打发个人去说一声,就说我的话\foot{原作“这会子有谁闲着,我打发个人去说一声,就说我的话”,诸本(除蒙本外)均同。
蒙本作“这会子谁去呢?你闲着,就打发个人去说一声”,语虽不佳,“打发个人去”的是“你”不是“我”,是对的。
按贾琏既认为这是小事,自不必亲自派人,让林之孝随便打发个闲着的人去说,“就说我的话”就足够了。
下文“林之孝听了,只得应着”可证。
现酌参蒙本改“我”为“你”字。
}。
”林之孝听了,只得应着,半晌笑道:“依我说,二爷竟别管这件事。
旺儿的那小儿子虽然年轻,在外头吃酒赌钱,无所不至。
虽说都是奴才们,到底是一辈子的事。
彩霞那孩子这几年我虽没见,听得越发出挑的好了,何苦来白糟踏一个人。
”贾琏道:“他小儿子原会吃酒,不成人?”\zhu{成人:成器、成材。
}林之孝冷笑道:“岂只吃酒赌钱,在外头无所不为。
我们看他是奶奶的人,也只见一半不见一半罢了。
”贾琏道:“我竟不知道这些事。
既这样,那里还给他老婆,且给他一顿棍,锁起来,再问他老子娘。
”林之孝笑道:“何必在这一时。
那是错也等他再生事,我们自然回爷处治。
如今且恕他。
”贾琏不语,一时林之孝出去。
\par
晚间,凤姐已命人唤了彩霞之母来说媒。
那彩霞之母满心纵不愿意,见凤姐亲自和他说,何等体面,\geng{今时人因图此现在体面,误了多少女儿,此正是为今时女儿一\sout{笑}[哭]。
}便心不由意的满口应了出去。
今凤姐问贾琏可说了没有,贾琏因说:“我原要说的,打听得他小儿子大不成人,故还不曾说。
若果然不成人,且管教他两日,再给他老婆不迟。
”凤姐听说,便说:“你听见谁说他不成人?”贾琏道:“不过是家里的人,还有谁。
”凤姐笑道:“我们王家的人,连我还不中你们的意,何况奴才呢。
我才已竟和他母亲说了,\zhu{竟:泛指结束,完毕。
甲辰本作“已经”。
}他娘已经欢天喜地应了,难道又叫进他来不要了不成?”贾琏道:“既你说了,又何必退,明儿说给他老子好生管他就是了。
”这里说话不提。
\par
且说彩霞因前日出去,等父母择人,心中虽是与贾环有旧,尚未作准。
今日又见旺儿每每来求亲,早闻得旺儿之子酗酒赌博,而且容颜丑陋,一技不知,自此心中越发懊恼。
生恐旺儿仗凤姐之势,一时作成,终身为患,不免心中急躁。
遂至晚间悄命他妹子小霞\geng{霞大小,奇奇怪怪之文,更觉有趣。
}进二门来找赵姨娘,问了端的。
\zhu{端的:究竟、详情。
}赵姨娘素日深与彩霞契合,巴不得与了贾环,方有个膀臂,不承望王夫人又放了出去。
每唆贾环去讨,\zhu{唆[suō]:指使、怂恿。
}一则贾环羞口难开,二则贾环也不大甚在意,不过是个丫头,他去了,将来自然还有,\geng{这是世人之情,亦是丈夫之情。
}遂迁延住不说,\zhu{迁延:拖延。
}
意思便丢开。
无奈赵姨娘又不舍,又见他妹子来问,是晚得空,便先求了贾政。
\geng{这是使人想不到之文,却是大家必有之事。
}贾政因说道:“且忙什么,等他们再念一二年书再放人不迟。
我已经看中了两个丫头,一个与宝玉,一个给环儿。
只是年纪还小,又怕他们误了书,所以再等一二年。
”\geng{妙文。
又写出贾老儿女之情。
细思一部书总不写贾老,则不成文,若不如此写,则又非贾老。
}赵姨娘道:“宝玉已有了二年了,老爷还不知道?”贾政听了忙问道:“谁给的?”\ping{王夫人私下给袭人姨娘待遇的事情可能要被揭开。
}赵姨娘方欲说话,只听外面一声响,不知何物,大家吃了一惊不小。
要知端的,且听下回分解。
\par
\qi{总评:夏雨冬风,常不解其何自来、何自去?鸳鸯与司棋相哭发誓,事已瓦释冰消,\zhu{瓦释冰消:即“冰消瓦解”,冰一样地消融,瓦一样地分解。
比喻崩溃、分裂或失败、离散。
}及平地风波一起,措手不及,亦不解何自来何自去。
}
\dai{143}{贾琏向鸳鸯借当}
\dai{144}{来旺妇倚势霸成亲}
\sun{p72-1}{贾琏借当渡难关}{
图中间:鸳鸯安慰司棋出来,顺路来望候凤姐,二门上的人立身待他进去。 
图右侧:平儿告诉鸳鸯,凤姐近日因劳累受气,又添了新病。
正说话间,贾琏回来了,见了鸳鸯,热情有余,其实是有事相求,求鸳鸯把老太太的金银家伙偷着运出一箱来,暂押千数两银子,日后再赎。
图左侧:鸳鸯偷运贾母的金银家伙给贾琏。
}