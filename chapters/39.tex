\chapter{村姥姥是信口开河 \quad 情哥哥偏寻根究底}
\qi{只为贫寒不拣行,\zhu{不拣行:降低人格尊严。
}富家趋入且逢迎。
岂知着意无名利,便是三才最上乘。
\zhu{三才:天、地、人。
上乘:比喻上等的事物和境界。
}}\par
话说众人见平儿来了,都说:“你们奶奶作什么呢,怎么不来了?”平儿笑道:“他那里得空儿来。
因为说没有好生吃得,又不得来,所以叫我来问还有没有,叫我要几个拿了家去吃罢。
”湘云道:“有,多着呢。
”忙令人拿了十个极大的。
平儿道:“多拿几个团脐的。
”
\zhu{团脐:雌蟹腹甲形圆,称团脐。雄蟹腹甲形尖,称尖脐。故团脐、尖脐有时亦指雌蟹和雄蟹。}
众人又拉平儿坐,平儿不肯。
李纨拉着他笑道:“偏要你坐。
”拉着他身边坐下,端了一杯酒送到他嘴边。
平儿忙喝了一口就要走。
李纨道:“偏不许你去。
显见得只有凤丫头,就不听我的话了。
”说着又命嬷嬷们:“先送了盒子去,就说我留下平儿了。
”那婆子一时拿了盒子回来说:“二奶奶说,叫奶奶和姑娘们别笑话要嘴吃。
这个盒子里是方才舅太太那里送来的菱粉糕和鸡油卷儿,给奶奶姑娘们吃的。
”又向平儿道:“说使你来你就贪住顽不去了。
劝你少喝一杯儿罢。
”平儿笑道:“多喝了又把我怎么样?”一面说,一面只管喝,又吃螃蟹。
李纨揽着他笑道:“可惜这么个好体面模样儿,命却平常,只落得屋里使唤。
不知道的人,谁不拿你当作奶奶太太看。
”\ping{李纨挑拨离间,反映了李纨嫉妒当家的王熙凤。
}\par
平儿一面和宝钗湘云等吃喝,一面回头笑道:“奶奶,别只摸的我怪痒的。
”李氏道:“嗳哟!这硬的是什么?”平儿道:“钥匙。
”李氏道:“什么钥匙?要紧梯己东西怕人偷了去,\zhu{梯己:意即私人的。
}却带在身上。
\ping{钥匙是当家人管家的法宝,这体现出李纨对凤姐管家的羡慕了。
}我成日家和人说笑,\zhu{家:一作“价”,语尾助词,无义。
}有个唐僧取经,就有个白马来驮他;\zhu{有个唐僧取经,就有个白马来驮他:唐代僧人玄奘,曾去天竺(即印度)取经。
龙王三太子化成白马,驮着唐僧去西天取经的故事,见明代吴承恩《西游记》第十五回。
}刘智远打天下,就有个瓜精来送盔甲;\zhu{刘智远打天下,就有个瓜精来送盔甲:刘智远,五代时后汉王朝的建立者。
“瓜精送盔甲”见明初无名氏的南戏《白兔记》第十五出《看瓜》。
}有个凤丫头,就有个你。
你就是你奶奶的一把总钥匙,还要这钥匙作什么。
”平儿笑道:“奶奶吃了酒,又拿了我来打趣着取笑儿了。
”宝钗笑道:“这倒是真话。
我们没事评论起人来,你们这几个都是百个里头挑不出一个来,妙在各人有各人的好处。
”李纨道:“大小都有个天理。
比如老太太屋里,要没那个鸳鸯如何使得。
从太太起,那一个敢驳老太太的回,现在他敢驳回。
偏老太太只听他一个人的话。
老太太那些穿戴的,别人不记得,他都记得,要不是他经管着,不知叫人诓骗了多少去呢。
那孩子心也公道,虽然这样,倒常替人说好话儿,还倒不依势欺人的。
”惜春笑道:“老太太昨儿还说呢,他比我们还强呢。
”平儿道:“那原是个好的,我们那里比的上他。
”宝玉道:“太太屋里的彩霞,是个老实人。
”探春道:“可不是,外头老实,心里有数儿。
太太是那么佛爷似的,事情上不留心,他都知道。
凡百一应事都是他提着太太行。
\zhu{凡百:表示概括的词,全部,所有的意思。
}连老爷在家出外去的一应大小事,他都知道。
太太忘了,他背地里告诉太太。
”李纨道:“那也罢了。
”指着宝玉道:“这一个小爷屋里要不是袭人,你们度量到个什么田地!凤丫头就是楚霸王,也得这两只膀子好举千斤鼎。
\zhu{楚霸王举千斤鼎:楚霸王即项羽,名籍,战国末楚国贵族之后,参加了秦末农民起义,秦亡后,自立为西楚霸王。
《史记·项羽本纪》说:“籍长八尺馀,力能扛鼎。
”}他不是这丫头,就得这么周到了!”平儿笑道:“先时陪了四个丫头,死的死,去的去,只剩下我一个孤鬼了。
”李纨道:“你倒是有造化的。
凤丫头也是有造化的。
想当初你珠大爷在日,何曾也没两个人。
你们看我还是那容不下人的?天天只见他两个不自在。
所以你珠大爷一没了,趁年轻我都打发了。
若有一个守得住,我倒有个膀臂。
”\ping{自伤身世,心里想要热闹有权威,但是事实是清冷靠边站。
李纨无论是看戏还是生活,都很注意配对的问题:唐僧配白马、刘智远配瓜精、凤姐配平儿、贾母配鸳鸯、王夫人配彩霞、宝玉配袭人。
她觉得自己很孤独,没有依靠。
}说着滴下泪来。
众人都道:“又何必伤心,不如散了倒好。
”说着便都洗了手,大家约往贾母王夫人处问安。
\par
众婆子丫头打扫亭子,收拾杯盘。
袭人和平儿同往前去,让平儿到房里坐坐,再喝一杯茶。
平儿说:“不喝茶了,再来吧。
”说着便要出去。
袭人又叫住问道:“这个月的月钱,连老太太和太太还没放呢,是为什么?”\ping{袭人知道太太的月钱还有没下发,是因为自己的月钱是从王夫人的月钱中取的。
}平儿见问,忙转身至袭人跟前,见方近无人,
\zhu{方近:临近,不远的地方。}
才悄悄说道:“你快别问,横竖再迟几天就放了。
”袭人笑道:“这是为什么,唬得你这样?”平儿悄悄告诉他道:“这个月的月钱,我们奶奶早已支了,放给人使呢。
等别处的利钱收了来,凑齐了才放呢。
因为是你,我才告诉你,你可不许告诉一个人去。
”袭人道:“难道他还短钱使,还没个足厌?何苦还操这心。
”平儿笑道:“何曾不是呢。
这几年拿着这一项银子,翻出有几百来了。
他的公费月例又使不着,十两八两零碎攒了放出去,只他这梯己利钱,一年不到,上千的银子呢。
”\ping{凤姐挪用公款放贷牟利。
其中以众人的月例为本钱,这几年的利息“翻出有几百来了”,而以凤姐的体己为本钱,一年不到的利息收入“上千的银子”。相对凤姐的体己利钱,月例的利钱,显然太少了。
}袭人笑道:“拿着我们的钱,你们主子奴才赚利钱,哄的我们呆呆的等着。
”平儿道:“你又说没良心的话。
你难道还少钱使?”袭人道:“我虽不少,只是我也没地方使去,就只预备我们那一个。
”平儿道:“你倘若有要紧的事用钱使时,我那里还有几两银子,你先拿来使,明儿我扣下你的就是了。
”袭人道:“此时也用不着,怕一时要用起来不够了,我打发人去取就是了。
”\par
平儿答应着,一径出了园门,来至家内,只见凤姐儿不在房里。
忽见上回来打抽丰的那刘姥姥和板儿又来了,\zhu{打抽丰:也叫“打秋风”,旧时利用各种关系取得有钱人的赠与。
本含“分肥”的意思。
一说旧时衙役于秋风起时以作棉衣为名向富户募钱。
}坐在那边屋里,还有张材家的周瑞家的陪着,又有两三个丫头在地下倒口袋里的枣子倭瓜并些野菜。
\zhu{倭瓜:南瓜。
}众人见他进来,都忙站起来了。
\ji{妙文!上回是先见平儿后见凤姐,此则先见凤姐后见平儿也。
何错综巧妙得情得理之至耶?}刘姥姥因上次来过,知道平儿的身分,忙跳下地来问“姑娘好”,又说:“家里都问好。
早要来请姑奶奶的安看姑娘来的,因为庄家忙。
好容易今年多打了两石粮食,瓜果菜蔬也丰盛。
这是头一起摘下来的,并没敢卖呢,留的尖儿孝敬姑奶奶姑娘们尝尝。
\zhu{尖儿:上好的,也称“尖子”。
}姑娘们天天山珍海味的也吃腻了,这个吃个野意儿,也算是我们的穷心。
”\par
平儿忙道:“多谢费心。
”又让坐,自己也坐了。
又让“张婶子周大娘坐”,又令小丫头子倒茶去。
周瑞张材两家的因笑道:“姑娘今儿脸上有些春色,眼圈儿都红了。
”平儿笑道:“可不是。
我原是不吃的,大奶奶和姑娘们只是拉着死灌,不得已喝了两盅,脸就红了。
”张材家的笑道:“我倒想着要吃呢,又没人让我。
明儿再有人请姑娘,可带了我去罢。
”说着大家都笑了。
\par
周瑞家的道:“早起我就看见那螃蟹了,一斤只好秤两个三个。
这么三大篓,想是有七八十斤呢。
”周瑞家的道:“若是上上下下只怕还不够。
”平儿道:“那里够,不过都是有名儿的吃两个子。
那些散众的,也有摸得着的,也有摸不着的。
”刘姥姥道:“这样螃蟹,今年就值五分一斤。
十斤五钱,五五二两五,三五一十五,再搭上酒菜,一共倒有二十多两银子。
\zhu{
刘姥姥算账的时候,把八十斤分成五十斤加三十斤。
十斤五钱,“五五二两五”指五十斤就要二十五钱,即二两五银子。
“三五一十五”是指剩下的三十斤螃蟹就要一十五钱,即一两五银子。
二两五加一两五等于四两,算得螃蟹价为四两银子。
关于酒菜的价格,贾府正为了螃蟹而大开宴席,“上面一桌”“东边一桌”“西边靠门一桌”“又令人在边廊上摆了两桌”,
邓云乡先生在《红楼识小录·螃蟹账》一文中便指出,“以三两一桌计”,加起来就有十五两。
所以刘姥姥说“一共倒有二十多两银子”。
}
阿弥陀佛!这一顿的钱够我们庄家人过一年了。
”\par
平儿因问:“想是见过奶奶了?”\ji{写平儿伶俐如此。
}刘姥姥道:“见过了,叫我们等着呢。
”说着又往窗外看天气,\ji{是八月中当开窗时,细致之甚。
}说道:“天好早晚了,\zhu{早晚:偏向于“晚”。
}我们也去罢,别出不去城才是饥荒呢。
”
\zhu{饥荒:麻烦事,祸患。}
周瑞家的道:“这话倒是,我替你瞧瞧去。
”说着一径去了,半日方来,笑道:“可是你老的福来了,竟投了这两个人的缘了。
”平儿等问怎么样,周瑞家的笑道:“二奶奶在老太太的跟前呢。
我原是悄悄的告诉二奶奶,‘刘姥姥要家去呢,怕晚了赶不出城去。
’二奶奶说:‘大远的,难为他扛了那些沉东西来,晚了就住一夜明儿再去。
’这可不是投上二奶奶的缘了。
这也罢了,偏生老太太又听见了,问刘姥姥是谁。
二奶奶便回明白了。
老太太说:‘我正想个积古的老人家说话儿,请了来我见一见。
’这可不是想不到天上缘分了。
”说着,催刘姥姥下来前去。
刘姥姥道:“我这生像儿怎好见的。
好嫂子,你就说我去了罢。
”平儿忙道:“你快去罢,不相干的。
我们老太太最是惜老怜贫的,比不得那个狂三诈四的那些人。
想是你怯上,我和周大娘送你去。
”说着,同周瑞家的引了刘姥姥往贾母这边来。
\par
二门口该班的小厮们见了平儿出来,都站起来了,又有两个跑上来,赶着平儿叫“姑娘”。
\ji{想这一个“姑娘”非下称上之“姑娘”也,
\zhu{下称上:侄辈对姑姑的称呼。}
按北俗以姑母曰“姑姑”,南俗曰“娘娘”,此“姑娘”定是“姑姑”“娘娘”之称。
每见大家风俗多有小童称少主妾曰“姑姑”“娘娘”者。
按此书中若干人说话语气及动用器物饮食诸类,皆东南西北互相兼用,此“姑娘”之称亦南北相兼而用无疑矣。
}平儿问:“又说什么?”那小厮笑道:“这会子也好早晚了,
\zhu{早晚:偏向于“晚”。}
我妈病了,等着我去请大夫。
好姑娘,我讨半日假可使的?”平儿道:“你们倒好,都商议定了,一天一个告假,又不回奶奶,只和我胡缠。
前儿住儿去了,二爷偏生叫他,叫不着,我应起来了,还说我作了情。
你今儿又来了。
”\ji{分明几回没写到贾琏,今忽闲中一语便补得贾琏这边天天闹热,令人却如看见听见一般。
所谓不写之写也。
刘姥姥眼中耳中又一番识面,
\zhu{识面:世面。这里说刘姥姥在被平儿带到贾母那里的路上,又长了见识。}
奇妙之甚!}周瑞家的道:“当真的他妈病了,姑娘也替他应着,放了他罢。
”平儿道:“明儿一早来。
听着,我还要使你呢,再睡的日头晒着屁股再来!你这一去,带个信儿给旺儿,就说奶奶的话,问着他那剩的利钱。
明儿若不交了来,奶奶也不要了,就越性送他使罢。
”
\ping{这是宽容还是用反话威胁呢?}
\ji{交代过袭人的话,看他如此说,真比凤姐又甚一层。
李纨之语不谬也。
不知阿凤何等福得此一人。
}那小厮欢天喜地答应去了。
\par
平儿等来至贾母房中,彼时大观园中姊妹们都在贾母前承奉。
\ji{妙极!连宝玉一并算入姊妹队中了。
}刘姥姥进去,只见满屋里珠围翠绕,花枝招展,并不知都系何人。
只见一张榻上歪着一位老婆婆,身后坐着一个纱罗裹的美人一般的一个丫鬟在那里捶腿,凤姐儿站着正说笑。
\ji{奇奇怪怪文章。
在刘姥姥眼中以为阿凤至尊至贵,普天下人独该站着说,阿凤独坐才是。
如何今见阿凤独站哉?真妙文字。
}刘姥姥便知是贾母了,忙上来陪着笑,福了几福,\zhu{福:这里指旧日女子与人相见时的一种礼节,也叫“万福”。
行礼时上身略弯,两手抱拳在胸前右上方上下移动。
句中前一福字作动词,后一福字作名词。
}口里说:“请老寿星安。
”\ji{更妙!贾母之号何其多耶?在诸人口中则曰“老太太”,在阿凤口中则曰“老祖宗”,在僧尼口中则曰“老菩萨”,在刘姥姥口中则曰“老寿星”,\sout{者}[看]去似有数人,想去则皆贾母,难得如此各尽其妙。
刘姥姥亦善应接。
}贾母亦欠身问好,又命周瑞家的端过椅子来坐着。
那板儿仍是怯人,不知问候。
\ji{“仍”字妙!盖有上文故也。
不知教训者来看此句。
}贾母道:“老亲家,你今年多大年纪了?”\ji{神妙之极!看官至此必愁贾母以何相称,谁知公然曰“老亲家”。
何等现成,何等大方,何等有情理。
若云作者心中编出,余断断不信。
何也?盖编得出者,断不能有这等情理。
}刘姥姥忙立身答道:“我今年七十五了。
”贾母向众人道:“这么大年纪了,还这么健朗。
比我大好几岁呢。
我要到这么大年纪,还不知怎么动不得呢。
”刘姥姥笑道:“我们生来是受苦的人,老太太生来是享福的。
若我们也这样,那些庄家活也没人作了。
”贾母道:“眼睛牙齿都还好?”刘姥姥道:“都还好,就是今年左边的槽牙活动了。
”贾母道:“我老了,都不中用了,眼也花,耳也聋,记性也没了。
你们这些老亲戚,我都不记得了。
亲戚们来了,我怕人笑我,我都不会,不过嚼的动的吃两口,睡一觉,闷了时和这些孙子孙女儿顽笑一回就完了。
”刘姥姥笑道:“这正是老太太的福了。
我们想这么着也不能。
”贾母道:“什么福,不过是个老废物罢了。
”说的大家都笑了。
贾母又笑道:“我才听见凤哥儿说,你带了好些瓜菜来,叫他快收拾去了,我正想个地里现撷的瓜儿菜儿吃。
\zhu{撷:音“协”,摘取。
}外头买的,不像你们田地里的好吃。
”刘姥姥笑道:“这是野意儿,不过吃个新鲜。
依我们想鱼肉吃,只是吃不起。
”贾母又道:“今儿既认着了亲,别空空儿的就去。
不嫌我这里,就住一两天再去。
我们也有个园子,园子里头也有果子,你明日也尝尝,带些家去,你也算看亲戚一趟。
”凤姐儿见贾母喜欢,也忙留道:“我们这里虽不比你们的场院大,空屋子还有两间。
你住两天罢,把你们那里的新闻故事儿说些与我们老太太听听。
”贾母笑道:“凤丫头别拿他取笑儿。
他是乡屯里的人,老实,那里搁的住你打趣他。
”说着,又命人去先抓果子与板儿吃。
板儿见人多了,又不敢吃。
贾母又命拿些钱给他,叫小幺儿们带他外头顽去。
刘姥姥吃了茶,便把些乡村中所见所闻的事情说与贾母,贾母益发得了趣味。
正说着,凤姐儿便令人来请刘姥姥吃晚饭。
贾母又将自己的菜拣了几样,命人送过去与刘姥姥吃。
凤姐知道合了贾母的心,吃了饭便又打发过来。
鸳鸯忙令老婆子带了刘姥姥去洗了澡,自己挑了两件随常的衣服令给刘姥姥换上。
\ji{一段鸳鸯身份、权势、心机,只写贾母也。
}那刘姥姥那里见过这般行事,忙换了衣裳出来,坐在贾母榻前,又搜寻些话出来说。
彼时宝玉姊妹们也都在这里坐着,他们何曾听见过这些话,自觉比那些瞽目先生说的书还好听。
\zhu{
瞽[gǔ]:眼瞎;失明。
瞽目先生:说书唱曲的盲艺人。
}\par
那刘姥姥虽是个村野人,却生来的有些见识,况且年纪老了,世情上经历过的,见头一个贾母高兴,第二见这些哥儿姐儿们都爱听,便没了说的也编出些话来讲。
因说道:“我们村庄上种地种菜,每年每日,春夏秋冬,风里雨里,那有个坐着的空儿,天天都是在那地头子上作歇马凉亭,\zhu{歇马凉亭:本指旧时驿路上供行人歇马休息的亭子,这里是说农民把地头当作“歇马凉亭”来休息。
}什么奇奇怪怪的事不见呢。
就像去年冬天,接连下了几天雪,地下压了三四尺深。
我那日起的早,还没出房门,只听外头柴草响。
我想着必定是有人偷柴草来了。
我爬着窗户眼儿一瞧,却不是我们村庄上的人。
”贾母道:“必定是过路的客人们冷了,见现成的柴,抽些烤火去也是有的。
”刘姥姥笑道:“也并不是客人,所以说来奇怪。
老寿星当个什么人?原来是一个十七八岁的极标致的一个小姑娘,梳着溜油光的头,穿着大红袄儿,白绫裙子——”\ji{刘姥姥的口气如此。
}刚说到这里,忽听外面人吵嚷起来,又说:“不相干的,别唬着老太太。
”贾母等听了,忙问怎么了,丫鬟回说:“南院马棚里走了水,\zhu{走了水:即“失火”的意思。
旧日迷信,忌讳说“失火”,故用“走水”来代替,取水能灭火的意思。
}不相干,已经救下去了。
”贾母最胆小的,听了这个话,忙起身扶了人出至廊上来瞧,只见东南上火光犹亮。
贾母唬的口内念佛,忙命人去火神跟前烧香。
王夫人等也忙都过来请安,又回说“已经下去了,老太太请进房去罢。
”贾母足的看着火光息了方领众人进来。
\zhu{足的:足足的、到底的。
}\ji{一段为后回作引,然偏于宝玉爱听时截住。
\zhu{
有人觉得“后回”即为后半部某一回贾府发生大火灾。
但是这条脂批在面对贾府大火灾的时候,语气轻描淡写,也没有发出“哀哉伤哉”、“悲夫”、“叹叹”等一类感慨。为贾府大火灾千里伏线说不通。
这条脂批强调文章“截住”的写法。本回刘姥姥在信口开河地编故事,“忽听外面人吵嚷起来”,叙事被“截住”;下一回行酒令时刘姥姥正说“花儿落了结个大倭瓜”时,“只听外面乱嚷”,叙事也被“截住”。
本回和下一回关于刘姥姥的叙事,都是被突发事件“截住”的,两回的写法都极其相似,所以有“为后回作引”的评论。
\hang
两回的写法类似,但是情节上却有不同。本回交代了突发事件时马棚失火,但是下一回即四十回因突发事件戛然而止后并没有在四十一回交代发生了什么突发事件。
可以推测在四十回和四十一回交界处迷失了一段原稿。最大的可能是原稿装订成册后,在借阅过程中有了破损,致使四十回末了或者四十一回开头掉了一页(不论是四回、五回、十回、二十回装订成一册,第四十回总在一册的末了,四十一回总在一册的开头)。
\hang
如果再仔细地考察一下,不难发现此处所缺的一页,不在四十回的末了,恰好在四十一回开头。第四十一回的回目异文:“庚辰本”题作《栊翠庵茶品梅花雪,怡红院劫遇母蝗虫》,“戚序本”作《贾宝玉品茶栊翠庵,刘老妪醉卧怡红院》,“程乙本”作《贾宝玉品茶栊翠庵,刘姥姥醉卧怡红院》。
“庚辰本”的回目拟得差强人意。“梅花雪”与“母蝗虫”不成对仗,刘姥姥待到后半部“贾府事败”将有一番作为,用“母蝗虫”指代刘姥姥显然过于挖苦。这证明“庚辰本”的回目应为补拟的,且补拟者甚至也不可能是脂砚斋,“为后回作引”证明脂砚斋评论时原稿尚完好。“庚辰本”最接近原作面貌,回目尚为补拟,其他各版本回目异文应皆出后人之手,因此可以断定所缺一页正是第四十一回开头。
联系到第六回回目《贾宝玉初试云雨情,刘姥姥一进荣国府》,“程乙本”回目名《贾宝玉品茶栊翠庵,刘姥姥醉卧怡红院》应为最妥。
}
}宝玉且忙着问刘姥姥:“那女孩儿大雪地作什么抽柴草?倘或冻出病来呢?”贾母道:“都是才说抽柴草惹出火来了,你还问呢。
别说这个了,再说别的罢。
”宝玉听说,心内虽不乐,也只得罢了。
\par
刘姥姥便又想了一篇,说道:“我们庄子东边庄上,有个老奶奶子,今年九十多岁了。
他天天吃斋念佛,谁知就感动了观音菩萨夜里来托梦说:‘你这样虔心,原来你该绝后的,如今奏了玉皇,给你个孙子。
’原来这老奶奶只有一个儿子,这儿子也只一个儿子,好容易养到十七八岁上死了,哭的什么似的。
后果然又养了一个,\zhu{养:生孩子。
}今年才十三四岁,生的雪团儿一般,聪明伶俐非常。
可见这些神佛是有的。
”这一夕话,\zhu{一夕话:一席话。
}实合了贾母王夫人的心事,连王夫人也都听住了。
\par
宝玉心中只记挂着抽柴的故事,因闷闷的心中筹画。
探春因问他:“昨日扰了史大妹妹,咱们回去商议着邀一社,又还了席,也请老太太赏菊花,何如?”宝玉笑道:“老太太说了,还要摆酒还史妹妹的席,叫咱们作陪呢。
等着吃了老太太的,咱们再请不迟。
”探春道:“越往前去越冷了,老太太未必高兴。
”宝玉道:“老太太又喜欢下雨下雪的。
不如咱们等下头场雪,请老太太赏雪岂不好?咱们雪下吟诗,也更有趣了。
”林黛玉忙笑道:“咱们雪下吟诗?依我说,还不如弄一捆柴火,雪下抽柴,还更有趣儿呢。
”说着,宝钗等都笑了。
宝玉瞅了他一眼,也不答话。
\par
一时散了,背地里宝玉足的拉了刘姥姥,\zhu{足的:足足的、到底的。
}细问那女孩儿是谁。
刘姥姥只得编了告诉他道:“那原是我们庄北沿地埂子上有一个小祠堂里供的,不是神佛,当先有个什么老爷。
”说着又想名姓。
宝玉道:“不拘什么名姓,你不必想了,只说原故就是了。
”刘姥姥道:“这老爷没有儿子,只有一位小姐,名叫茗玉。
小姐知书识字,老爷太太爱如珍宝。
可惜这茗玉小姐生到十七岁,一病死了。
”\ping{也是一个早逝的“玉”。
}宝玉听了,跌足叹惜,\zhu{跌足:跺脚。
}又问后来怎么样。
刘姥姥道:“因为老爷太太思念不尽,便盖了这祠堂,塑了这茗玉小姐的像,派了人烧香拨火。
如今日久年深的,人也没了,庙也烂了,那个像就成了精。
”宝玉忙道:“不是成精,规矩这样人是虽死不死的。
”刘姥姥道:“阿弥陀佛!原来如此。
不是哥儿说,我们都当他成精。
他时常变了人出来各村庄店道上闲逛。
\zhu{店:犹“站”。常用作集镇之名。如:长辛店,驻马店。
}我才说这抽柴火的就是他了。
我们村庄上的人还商议着要打了这塑像平了庙呢。
”宝玉忙道:“快别如此。
若平了庙,罪过不小。
”刘姥姥道:“幸亏哥儿告诉我,我明儿回去告诉他们就是了。
”宝玉道:“我们老太太、太太都是善人,合家大小也都好善喜舍,最爱修庙塑神的。
我明儿做一个疏头,\zhu{疏头:旧时称分条陈述事情的文字及僧道拜忏所焚化的祝文等叫“疏”,也称“疏头”。
这里指修庙募捐的“启事”。
拜忏:也叫“礼忏”,是佛教的规仪,是礼拜与忏悔的省称。
佛教信徒在忏悔时,依照忏法诵念忏悔文,表示忏悔违戒之罪,并礼拜诸佛、菩萨,求其宽恕所造诸恶业,表示改过迁善之决心。
}替你化些布施,你就做香头,\zhu{香头:寺庙中管香火的头目。
}攒了钱把这庙修盖,再装潢了泥像,每月给你香火钱烧香岂不好?”刘姥姥道:“若这样,我托那小姐的福,也有几个钱使了。
”宝玉又问他地名庄名,来往远近,坐落何方。
刘姥姥便顺口胡诌了出来。
\par
宝玉信以为真,回至房中,盘算了一夜。
次日一早,便出来给了茗烟几百钱,按着刘姥姥说的方向地名,着茗烟去先踏看明白,回来再做主意。
那茗烟去后,宝玉左等也不来,右等也不来,急的热锅上的蚂蚁一般。
好容易等到日落,方见茗烟兴兴头头的回来。
宝玉忙道:“可有庙了?”茗烟笑道:“爷听的不明白,叫我好找。
那地名座落不似爷说的一样,所以找了一日,找到东北上田埂子上才有一个破庙。
”宝玉听说,喜的眉开眼笑,忙说道:“刘姥姥有年纪的人,一时错记了也是有的。
你且说你见的。
”茗烟道:“那庙门却倒是朝南开,也是稀破的。
我找的正没好气,一见这个,我说‘可好了’,连忙进去。
一看泥胎,唬的我跑出来了,活似真的一般。
”宝玉喜的笑道:“他能变化人了,自然有些生气。
”茗烟拍手道:“那里有什么女孩儿,竟是一位青脸红发的瘟神爷。
”宝玉听了,啐了一口,骂道:“真是一个无用的杀才!
\zhu{杀才:该杀的,骂人之词;冤家,常用于称所爱的人。}
这点子事也干不来。
”茗烟道:“二爷又不知看了什么书,或者听了谁的混话,信真了,把这件没头脑的事派我去碰头,怎么说我没用呢?”宝玉见他急了,忙抚慰他道:“你别急。
改日闲了你再找去。
若是他哄我们呢,自然没了,若真是有的,你岂不也积了阴骘。
我必重重的赏你。
”正说着,只见二门上的小厮来说:“老太太房里的姑娘们站在二门口找二爷呢。
”\par
\qi{总评:此回第一写势利之好财,\zhu{王熙凤放贷。
}第二写穷苦趋势之求财。
\zhu{刘姥姥二进荣国府。
}且文章不得雷同,先既有诗社,而今不得不用套坡公听鬼之遗事,
\ping{坡公听鬼:
苏轼,字子瞻,号东坡居士。
宋·叶梦得《避暑录话》:
“子瞻在黄州及岭表,每旦起,不招客相与语,则必出而访客,所与游者亦不尽择,各随其人高下,谈谐放荡,不复为畛畦,有不能谈者则强之说鬼,或辞无有则曰:姑妄言之。”
苏东坡让人“姑妄言之”、“强之说鬼”的故事,可以类比本回刘姥姥为讨贾府上下的开心,信口开河地编故事。
}以振其馀响,即此以点染宝玉之痴。
其文真如环转,无端倪可指。
\zhu{端倪:事情的头绪。
《庄子·大宗师》:“反覆终始,不知端倪。
”推寻事物的本末终始。
}}
\dai{077}{刘姥姥带瓜果拜见凤姐}
\dai{078}{村姥姥是信口开河}
\sun{p39-1}{刘姥姥二进荣国府,史太君亲见刘姥姥}{图右侧:刘姥姥携孙子板儿来到贾府,送来自家野菜瓜枣。
贾母听说道: “正想个积古的老人家说话。
”遂传见刘姥姥。
刘姥栳有些犹豫,平儿说:“我们老太太最是怜老惜贫的。
”图左侧:刘姥姥来到贾母房中,满屋里珠围翠绕,花枝招展的,不知都是何人。
只见一张榻上,独歪着一位老婆婆,便知是贾母了,忙上来请安。
}