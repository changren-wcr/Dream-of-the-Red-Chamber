\chapter{白玉钏亲尝莲叶羹 \quad 黄金莺巧结梅花络}
\qi{情因相爱反相伤,何事人多不揣量。
黛玉徘徊还自苦,莲羹甘受使儿狂。
\zhu{使儿:这里可能是指丫鬟侍女。
蒙府本“狂”作“枉”。“狂”与“枉”字形相近,容易抄误,不过这两个字都能解得通。
狂:即玉钏儿作为丫鬟敢于对主人宝玉“满脸怒色”的态度。
枉:玉钏儿把金钏儿之死怪罪到宝玉头上其实是冤枉了他。
}}\par
话说宝钗分明听见林黛玉刻薄他,因记挂着母亲哥哥,并不回头,一径去了。
这里林黛玉还自立于花阴之下,远远的却向怡红院内望着,只见李宫裁、迎春、探春、惜春并各项人等都向怡红院内去过之后,一起一起的散尽了,只不见凤姐儿来,心里自己盘算道:“如何他不来瞧宝玉?便是有事缠住了,他必定也是要来打个花胡哨,\zhu{打个花胡哨:虚情假意地敷衍一下。
}讨老太太和太太的好儿才是。
今儿这早晚不来,必有原故。
”一面猜疑,一面抬头再看时,只见花花簇簇一群人又向怡红院内来了。
定睛看时,只见贾母搭着凤姐儿的手,后头邢夫人王夫人跟着周姨娘并丫鬟媳妇等人都进院去了。
黛玉看了不觉点头,想起有父母的人的好处来,早又泪珠满面。
少顷,只见宝钗薛姨妈等也进去了。
忽见紫鹃从背后走来,说道:“姑娘吃药去罢,开水又冷了。
”黛玉道:“你到底要怎么样?只是催,我吃不吃,管你什么相干!”紫鹃笑道:“咳嗽的才好了些,又不吃药了。
如今虽然是五月里,\meng{闺中相怜之情,令人羡慕之至。
}天气热,到底也该还小心些。
大清早起,在这个潮地方站了半日,也该回去歇息歇息了。
”一句话提醒了黛玉,方觉得有点腿酸,呆了半日,方慢慢的扶着紫鹃,回潇湘馆来。
\par
一进院门,只见满地下竹影参差,苔痕浓淡,不觉又想起《西厢记》中所云“幽僻处可有人行,点苍苔白露泠泠”二句来,\zhu{泠泠:音“灵灵”,形容清凉。
}因暗暗的叹道:“双文,双文,\zhu{双文:即《西厢记》里的崔莺莺,因莺莺的名字是用两个“莺”字叠成。
}诚为命薄人矣。
然你虽命薄,尚有孀母弱弟;今日林黛玉之命薄,一并连孀母弱弟俱无。
古人云‘佳人命薄’,然我又非佳人,何命薄胜于双文哉!”一面想,一面只管走,不防廊上的鹦哥见林黛玉来了,嘎的一声扑了下来,倒吓了一跳,因说道:“作死的,又扇了我一头灰。
”那鹦哥仍飞上架去,便叫:“雪雁,快掀帘子,姑娘来了。
”黛玉便止住步,以手扣架道:“添了食水不曾?”那鹦哥便长叹一声,竟大似林黛玉素日吁嗟音韵,接着念道:“侬今葬花人笑痴,他年葬侬知是谁?试看春尽花渐落,便是红颜老死时。
一朝春尽红颜老,花落人亡两不知!”\meng{哭成的句子,到今日听了,竟作一场笑话。
}黛玉紫鹃听了都笑起来。
紫鹃笑道:“这都是素日姑娘念的,难为他怎么记了。
”黛玉便令将架摘下来,另挂在月洞窗外的钩上,
\zhu{月洞窗:圆月形的窗子。}
于是进了屋子,在月洞窗内坐了。
吃毕药,只见窗外竹影映入纱来,满屋内阴阴翠润,几簟生凉。
\zhu{几:低矮的桌子。
簟:音“电”,竹席。
}
黛玉无可释闷,便隔着纱窗调逗鹦哥作戏,又将素日所喜的诗词也教与他念。
这且不在话下。
\par
且说薛宝钗来至家中,只见母亲正自梳头呢。
一见他来了,便说道:“你大清早起跑来作什么?”宝钗道:“我瞧瞧妈身上好不好。
昨儿我去了,不知他可又过来闹了没有?”一面说,一面在他母亲身旁坐了,由不得哭将起来。
薛姨妈见他一哭,自己撑不住,也就哭了一场,一面又劝他:“我的儿,你别委曲了,你等我处分他。
你要有个好歹,我指望那一个来!”薛蟠在外边听见,连忙跑了过来,对着宝钗,左一个揖,右一个揖,只说:“好妹妹,恕我这一次罢!原是我昨儿吃了酒,回来的晚了,路上撞客着了,\zhu{撞客:旧时迷信认为突然神智昏迷、胡言乱语,是鬼、神附体,俗称“撞客”。
}来家未醒,不知胡说了什么,连自己也不知道,怨不得你生气。
”宝钗原是掩面哭的,听如此说,由不得又好笑了,遂抬头向地下啐了一口,说道:“你不用做这些像生儿。
\zhu{像生儿:原指对客观事物的声音、状态等的模拟仿效。
这里指做戏似地装模作样,引人发笑。
}我知道你的心里多嫌我们娘儿两个,是要变着法儿叫我们离了你,你就心净了。
”薛蟠听说,连忙笑道:“妹妹这话从那里说起来的,这样我连立足之地都没了。
妹妹从来不是这样多心说歪话的人。
”薛姨妈忙又接着道:“你只会听见你妹妹的歪话,难道昨儿晚上你说的那话就应该的不成?当真是你发昏了!”薛蟠道:“妈也不必生气,妹妹也不用烦恼,从今以后我再不同他们一处吃酒闲逛如何?”宝钗笑道:“这不明白过来了!”\meng{亲生兄妹,形景逼真贴切。
}薛姨妈道:“你要有这个横劲,那龙也下蛋了。
”
\ping{宝钗鼓励赞赏哥哥的转变,而薛姨妈则是嘲讽质疑。薛姨妈在教育孩子方面不如宝钗。}
薛蟠道:“我若再和他们一处逛,妹妹听见了只管啐我,再叫我畜生,不是人,如何?何苦来,为我一个人,娘儿两个天天操心!妈为我生气还有可恕,若只管叫妹妹为我操心,我更不是人了。
如今父亲没了,我不能多孝顺妈多疼妹妹,反教娘生气妹妹烦恼,真连个畜生也不如了。
”口里说着,眼睛里禁不起也滚下泪来。
\meng{又是一样哭法,不过是情之所致。
}
薛姨妈本不哭了,听他一说又勾起伤心来。
宝钗勉强笑道:“你闹够了,这会子又招着妈哭起来了。
”薛蟠听说,忙收了泪,笑道:“我何曾招妈哭来!罢,罢,罢,丢下这个别提了。
叫香菱来倒茶妹妹吃。
”宝钗道:“我也不吃茶,等妈洗了手,我们就过去了。
”薛蟠道:“妹妹的项圈我瞧瞧,只怕该炸一炸去了。
”\zhu{炸一炸:金银器物旧了,经淬火加工使它重现光泽,叫作“炸”。
}宝钗道:“黄澄澄的又炸他作什么?”薛蟠又道:“妹妹如今也该添补些衣裳了。
要什么颜色花样,告诉我。
”宝钗道:“连那些衣服我还没穿遍了,又做什么?”\meng{一写骨肉悔过之情,一写本等贞静之女。
}一时薛姨妈换了衣裳,拉着宝钗进去,薛蟠方出去了。
\par
这里薛姨妈和宝钗进园来瞧宝玉,到了怡红院中,只见抱厦里外回廊上许多丫鬟老婆站着,便知贾母等都在这里。
母女两个进来,大家见过了,只见宝玉躺在榻上。
薛姨妈问他可好些。
宝玉忙欲欠身,口里答应着“好些”,又说:“只管惊动姨娘、姐姐,我禁不起。
”薛姨娘忙扶他睡下,又问他:“想什么,只管告诉我。
”宝玉笑道:“我想起来,自然和姨娘要去的。
”王夫人又问:“你想什么吃?回来好给你送来的。
”宝玉笑道:“也倒不想什么吃,倒是那一回做的那小荷叶儿小莲蓬儿的汤还好些。
”凤姐一旁笑道:“听听,口味不算高贵,只是太磨牙了。
\zhu{磨牙:啰嗦,麻烦,费劲。}
巴巴的想这个吃了。
”贾母便一叠声的叫人做去。
凤姐儿笑道:“老祖宗别急,等我想一想这模子谁收着呢。
”因回头吩咐个婆子去问管厨房的要去。
那婆子去了半天,来回说:“管厨房的说,四副汤模子都交上来了。
\zhu{汤模子:一种特制的成套金属模具,用来将湿面等压印成各种花样入汤。}
”凤姐儿听说,想了一想,道:“我记得交上来了,就不记得交给谁了,多半在茶房里。
”一面又遣人去问管茶房的,也不曾收。
次后还是管金银器皿的送了来。
\par
薛姨妈先接过来瞧时,原来是个小匣子,里面装着四副银模子,都有一尺多长,一寸见方,
\zhu{尺:一市尺等于三分之一米,合十市寸。}
上面凿着有豆子大小,也有菊花的,也有梅花的,也有莲蓬的,也有菱角的,共有三四十样,打的十分精巧。
因笑向贾母王夫人道:“你们府上也都想绝了,吃碗汤还有这些样子。
若不说出来,我见这个也不认得这是作什么用的。
”凤姐儿也不等人说话,便笑道:“姑妈那里晓得,这是旧年备膳,他们想的法儿。
不知弄些什么面印出来,借点新荷叶的清香,全仗着好汤,究竟没意思,谁家常吃他了。
那一回呈样的作了一回,
\zhu{呈样:做出样品呈送给上面的人审看。}
他今日怎么想起来了。
”说着接了过来,递与个妇人,吩咐厨房里立刻拿几只鸡,另外添了东西,做出十来碗来。
王夫人道:“要这些做什么?”凤姐儿笑道:“有个原故:这一宗东西家常不大作,今儿宝兄弟提起来了,单做给他吃,老太太、姑妈、太太都不吃,似乎不大好。
不如借势儿弄些大家吃,托赖连我也上个俊儿。
”\zhu{托赖:托庇,依赖。
上个俊儿:尝个新、沾点光的意思。
}贾母听了,笑道:“猴儿,把你乖的!拿着官中的钱你做人。
”\zhu{
官中:指大家庭所共有的。
做人:做好人,卖人情。
}说的大家笑了。
凤姐也忙笑道:“这不相干。
这个小东道我还孝敬的起。
”便回头吩咐妇人,“说给厨房里,只管好生添补着做了,在我的帐上来领银子。
”妇人答应着去了。
\par
宝钗一旁笑道:“我来了这么几年,留神看起来,凤丫头凭他怎么巧,再巧不过老太太去。
”贾母听说,便答道:“我如今老了,那里还巧什么。
当日我像凤哥儿这么大年纪,比他还来得呢。
他如今虽说不如我们,也就算好了,比你姨娘强远了。
你姨娘可怜见的,不大说话,和木头似的,在公婆跟前就不大显好。
\ping{此时王夫人在场。
}凤儿嘴乖,怎么怨得人疼他。
”宝玉笑道:“若这么说,不大说话的就不疼了?”
\ping{
奶奶说自己的母亲王夫人“和木头似的”,宝玉赶紧给母亲找补。
宝玉很怕有人得不到爱,在某个人被赞美的时候,他立刻会想到那个没有被赞美的人。
}
贾母道:“不大说话的又有不大说话的可疼之处,嘴乖的也有一宗可嫌的,倒不如不说话的好。
”宝玉笑道:“这就是了。
我说大嫂子倒不大说话呢,老太太也是和凤姐姐的一样看待。
若是单是会说话的可疼,这些姊妹里头也只是凤姐姐和林妹妹可疼了。
”贾母道:“提起姊妹,不是我当着姨太太的面奉承,千真万真,从我们家四个女孩儿算起,全不如宝丫头。
”\ping{贾母又把林黛玉算成了自己家的女孩儿,而把薛宝钗看成了别人家的女孩儿。
虽说不是奉承,其实就是奉承。
}薛姨妈听说,忙笑道:“这话是老太太说偏了。
”王夫人忙又笑道:“老太太时常背地里和我说宝丫头好,这倒不是假话。
”宝玉勾着贾母原为赞林黛玉的,不想反赞起宝钗来,倒也意出望外,便看着宝钗一笑。
\ping{
宝玉原是希望贾母赞林黛玉的,按说贾母赞宝钗,他应该觉得不舒服。
但是宝玉却“意出望外”,欣然接受,是因为宝玉觉得两个人都好棒!
}
宝钗早扭过头去和袭人说话去了。
\par
忽有人来请吃饭,贾母方立起身来,命宝玉好生养着,又把丫头们嘱咐了一回,方扶着凤姐儿,让着薛姨妈,大家出房去了。
因问汤好了不曾,又问薛姨妈等:“想什么吃,只管告诉我,我有本事叫凤丫头弄了来咱们吃。
”薛姨妈笑道:“老太太也会怄他的。
\zhu{怄[òu]:引逗,招惹,捉弄,引人发笑或使人生气。
}时常他弄了东西孝敬,究竟又吃不了多少。
”凤姐儿笑道:“姑妈倒别这样说。
我们老祖宗只是嫌人肉酸,若不嫌人肉酸,早已把我还吃了呢。
”\par
一句话没说了,引的贾母众人都哈哈的笑起来。
宝玉在房里也撑不住笑了。
袭人笑道:“真真的二奶奶的这张嘴怕死人!”宝玉伸手拉着袭人笑道:“你站了这半日,可乏了?”一面说,一面拉他身旁坐了。
袭人笑道:“可是又忘了。
趁宝姑娘在院子里,你和他说,烦他莺儿来打上几根络子。
\zhu{络:音“涝”。
络子:用线编织成的网袋。
}
”宝玉笑道:“亏你提起来。
”说着,便仰头向窗外道:“宝姐姐,吃过饭叫莺儿来,烦他打几根络子,可得闲儿?”宝钗听见,回头道:“怎么不得闲儿,一会叫他来就是了。
”贾母等尚未听真,都止步问宝钗。
宝钗说明了,大家方明白。
贾母又说道:“好孩子,叫他来替你兄弟作几根。
你要无人使唤,我那里闲着的丫头多呢,你喜欢谁,只管叫了来使唤。
”薛姨妈宝钗等都笑道:“只管叫他来作就是了,有什么使唤的去处。
他天天也是闲着淘气。
”\par
大家说着,往前迈步正走,忽见史湘云、平儿、香菱等在山石边掐凤仙花呢,见了他们走来,都迎上来了。
少顷至园外,王夫人恐贾母乏了,便欲让至上房内坐。
贾母也觉腿酸,便点头依允。
王夫人便令丫头忙先去铺设坐位。
那时赵姨娘推病,只有周姨娘与众婆娘丫头们忙着打帘子,立靠背,铺褥子。
贾母扶着凤姐儿进来,与薛姨妈分宾主坐了。
薛宝钗史湘云坐在下面。
王夫人亲捧了茶奉与贾母,李宫裁奉与薛姨妈。
贾母向王夫人道:“让他们小妯娌伏侍,
\zhu{妯娌:音“轴里”,兄弟之妻相互的称呼。}
你在那里坐了,好说话儿。
”王夫人方向一张小杌子上坐下,\zhu{杌(音“物”):小凳子。
}便吩咐凤姐儿道:“老太太的饭在这里放,添了东西来。
”凤姐儿答应出去,便令人去贾母那边告诉,那边的婆娘忙往外传了,丫头们忙都赶过来。
王夫人便令“请姑娘们去”。
请了半天,只有探春惜春两个来了;迎春身上不耐烦,不吃饭;林黛玉自不消说,平素十顿饭只好吃五顿,众人也不着意了。
少顷饭至,众人调放了桌子。
凤姐儿用手巾裹着一把牙箸站在地下,笑道:“老祖宗和姑妈不用让,还听我说就是了。
”贾母笑向薛姨妈道:“我们就是这样。
”薛姨妈笑着应了。
于是凤姐放了四双:上面两双是贾母薛姨妈,两边是薛宝钗史湘云的。
王夫人李宫裁等都站在地下看着放菜。
凤姐先忙着要干净家伙来,替宝玉拣菜。
\meng{家庭之间,亦复如此。
}\par
少顷,荷叶汤来,贾母看过了。
王夫人回头见玉钏儿在那边,便令玉钏与宝玉送去。
凤姐道:“他一个人拿不去。
”可巧莺儿和喜儿都来了。
宝钗知道他们已吃了饭,便向莺儿道:“宝兄弟正叫你去打络子,你们两个一同去罢。
”莺儿答应,同着玉钏儿出来。
莺儿道:“这么远,怪热的,怎么端了去?”玉钏笑道:“你放心,我自有道理。
”说着,便令一个婆子来,将汤饭等物放在一个捧盒里,\meng{大家气象。
}令他端了跟着,他两个却空着手走。
一直到了怡红院门内,玉钏儿方接了过来,同莺儿进入宝玉房中。
袭人、麝月、秋纹三个人正和宝玉顽笑呢,见他两个来了,都忙起来,笑道:“你两个怎么来的这么碰巧,一齐来了。
”一面说,一面接了下来。
玉钏便向一张杌子上坐了,莺儿不敢坐下。
\meng{两人不一样写,真是各进其文于后。
}袭人便忙端了个脚踏来,\meng{宝卿之婢,自应与众不同。
}莺儿还不敢坐。
\ping{玉钏是宝玉母亲的丫鬟,而莺儿是宝玉表姐的丫鬟,尊卑不同。
}宝玉见莺儿来了,却倒十分欢喜;忽见了玉钏儿,便想到他姐姐金钏儿身上,又是伤心,又是惭愧,便把莺儿丢下,且和玉钏儿说话。
袭人见把莺儿不理,恐莺儿没好意思的,\meng{能事者。
}又见莺儿不肯坐,便拉了莺儿出来,到那边房里去吃茶说话儿去了。
\par
这里麝月等预备了碗箸来伺候吃饭。
宝玉只是不吃,问玉钏儿道:“你母亲身子好?”玉钏儿满脸怒色,正眼也不看宝玉,半日,方说了一个“好”字。
宝玉便觉没趣,半日,只得又陪笑问道:\meng{何等涵度。
}\ping{宝玉情不情,可贵可叹。
}
“谁叫你给我送来的?”玉钏儿道:“不过是奶奶太太们!”宝玉见他还是这样哭丧,便知他是为金钏儿的原故;待要虚心下气磨转他,又见人多,不好下气的,\meng{金钏儿如若有知,该何等感激!}因而变尽方法,将人都支出去,然后又陪笑问长问短。
\par
那玉钏儿先虽不悦,只管见宝玉一些性子没有,凭他怎么丧谤,\zhu{丧谤:恶声恶气、出语伤人。
}他还是温存和气,自己倒不好意思的了,脸上方有三分喜色。
\meng{我看到此处,也着实不过意。
}宝玉便笑求他:“好姐姐,你把那汤拿了来我尝尝。
”玉钏儿道:“我从不会喂人东西,等他们来了再吃。
”宝玉笑道:“我不是要你喂我。
我因为走不动,你递给我吃了,你好赶早儿回去交代了,你好吃饭的。
我只管耽误时候,你岂不饿坏了。
你要懒待动,我少不了忍了疼下去取来。
”说着便要下床来,扎挣起来,\zhu{扎挣:勉强支持。
}禁不住嗳哟之声。
玉钏儿见他这般,忍不住起身说道:“躺下罢!那世里造了来的业,\zhu{业:同“孽”,罪过、邪恶的意思。
}这会子现世现报。
教我那一个眼睛看的上!”\meng{偏于此间写此不情之态,以表白多情之苦。
}一面说,一面哧的一声又笑了,端过汤来。
\par
宝玉笑道:“好姐姐,你要生气只管在这里生罢,见了老太太、太太可放和气些,若还这样,你就又捱骂了。
”\zhu{捱:同“挨”。
}玉钏儿道:“吃罢,吃罢!不用和我甜嘴蜜舌的,我可不信这样话!”说着,催宝玉喝了两口汤。
宝玉故意说:“不好吃,不吃了。
”玉钏儿道:“阿弥陀佛!这还不好吃,什么好吃?”宝玉道:“一点味儿也没有,你不信,尝一尝就知道了。
”玉钏儿真就赌气尝了一尝。
宝玉笑道:“这可好吃了。
”玉钏儿听说,方解过意来,原是宝玉哄他吃一口,便说道:“你既说不好吃,这会子说好吃也不给你吃了。
”宝玉只管央求陪笑要吃,\meng{写尽多情人无限委屈柔肠。
}玉钏儿又不给他,一面又叫人打发吃饭。
\par
丫头方进来时,忽有人来回话:“傅二爷家的两个嬷嬷来请安,来见二爷。
”宝玉听说,便知是通判傅试家的嬷嬷来了。
\zhu{傅试:谐音“趋炎附势”。}
那傅试原是贾政的门生,历年来都赖贾家的名势得意,贾政也着实看待,故与别个门生不同,他那里常遣人来走动。
宝玉素习最厌愚男蠢女的,今日却如何又令两个婆子过来?其中原来有个原故:只因那宝玉闻得傅试有个妹子,名唤傅秋芳,
\zhu{秋芳:青春韶华已过。}
也是个琼闺秀玉,常闻人传说才貌俱全,虽自未亲睹,然遐思遥爱之心十分诚敬,不命他们进来,恐薄了傅秋芳,\ji{痴想。
}因此连忙命让进来。
\par
那傅试原是暴发的,因傅秋芳有几分姿色,聪明过人,那傅试安心仗着妹妹要与豪门贵族结姻,不肯轻易许人,所以耽误到如今。
目今傅秋芳年已二十三岁,尚未许人。
争奈那些豪门贵族又嫌他穷酸,\zhu{争奈:怎奈。
}根基浅薄,不肯求配。
\meng{大抵诸色非情不生,非情不合。
情之表见于爱,爱众则心无定象,心不定则诸幻丛生,诸魔蜂起,则汲汲乎流于无情。
\zhu{汲汲:急切追求的样子。
}此宝玉之多情而不情之案,\zhu{案:案例,事件。
多情而不情:宝玉有“情极之毒”,由“多情”转向“无情”,走向冷酷决绝的反面,演变为世人尚不忍心的“悬崖撒手”、“弃而为僧”。宝玉是怎样为这样的情所驱使而走上出家的道路的呢?
从本回文末批语“爱河之深无底,何可泛滥,一溺其中,非死不止。且泛爱者不专,新旧叠增,岂能尽了?其多情之心不能不流于无情之地”可以看出,宝玉对“多情”的耽溺和对“泛爱”的不加节制,将无法保持情感的稳固与恒常,终于成为“无情”。
此外第三十回有批语“爱众不常,多情不寿”。
宝玉的“泛爱”在此回之后的“识分定情悟梨香院”情节中有了自悟自省。
}
凡我同人其留意!}\ping{第一回:择膏粱,谁承望流落在烟花巷!}那傅试与贾家亲密,也自有一段心事。
\ping{可能是想要和贾家联姻。
}今日遣来的两个婆子偏生是极无知识的,闻得宝玉要见,进来只刚问了好,说了没两句话。
那玉钏见生人来,也不和宝玉厮闹了,手里端着汤只顾听话。
宝玉又只顾和婆子说话,一面吃饭,一面伸手去要汤。
两个人的眼睛都看着人,不想伸猛了手,便将碗碰翻,将汤泼了宝玉手上。
玉钏儿倒不曾烫着,唬了一跳,忙笑了,“这是怎么说!”慌的丫头们忙上来接碗。
宝玉自己烫了手倒不觉的,却只管问玉钏儿:“烫了那里了?疼不疼?”\meng{多情人每于苦恼时不自觉,反说彼家苦恼。
爱之至、惜之深之故也。
}玉钏儿和众人都笑了。
玉钏儿道:“你自己烫了,只管问我。
”宝玉听说,方觉自己烫了。
众人上来连忙收拾。
宝玉也不吃饭了,洗手吃茶,又和那两个婆子说了两句话。
然后两个婆子告辞出去,晴雯等送至桥边方回。
\par
那两个婆子见没人了,一行走,一行谈论。
这一个笑道:“怪道有人说他家宝玉是外像好里头糊涂,中看不中吃的,果然有些呆气。
他自己烫了手,倒问人疼不疼,这可不是个呆子?”那一个又笑道:“我前一回来,听见他家里许多人抱怨,千真万真的有些呆气。
大雨淋的水鸡似的,他反告诉别人:‘下雨了,快避雨去罢。
’你说可笑不可笑?\zhu{第三十回,宝玉看龄官在地上画“蔷”字,突然下雨,宝玉提醒她赶紧去避雨,自己却淋湿了。
}时常没人在跟前,就自哭自笑的;看见燕子,就和燕子说话;河里看见了鱼,就和鱼说话;见了星星月亮,不是长吁短叹,就是咕咕哝哝的。
且是连一点刚性也没有,连那些毛丫头的气都受的。
爱惜东西,连个线头儿都是好的;糟踏起来,那怕值千值万的都不管了。
”\meng{如人饮水,冷暖自知。
其中深意味,岂能持告君?}两个人一面说,一面走出园来,辞别诸人回去,不在话下。
\ji{宝玉之为人,非此一论,亦描写不尽;宝玉之不肖,非此一鄙,亦形容不到。
试问作者是丑宝玉乎?是赞宝玉乎?试问观者是喜宝玉乎?是恶宝玉乎?}\par
如今且说袭人见人去了,便携了莺儿过来,问宝玉打什么络子。
宝玉笑向莺儿道:“才只顾说话,就忘了你。
烦你来不为别的,却为替我打几根络子。
”莺儿道:“装什么的络子?”宝玉见问,便笑道:“不管装什么的,你都每样打几个罢。
”\meng{富家子弟每多有如是语,只不自觉耳。
}莺儿拍手笑道:“这还了得!要这样,十年也打不完了。
”宝玉笑道:“好姐姐,你闲着也没事,都替我打了罢。
”袭人笑道:“那里一时都打得完,如今先拣要紧的打两个罢。
”莺儿道:“什么要紧,不过是扇子、香坠儿、汗巾子。
”宝玉道:“汗巾子就好。
”莺儿道:“汗巾子是什么颜色的?”宝玉道:“大红的。
”莺儿道:“大红的须是黑络子才好看的,或是石青的才压的住颜色。
”\zhu{石青:淡灰青色。
}宝玉道:“松花色配什么?”\zhu{松花:偏黑的深绿色。
}莺儿道:“松花配桃红。
”宝玉笑道:“这才娇艳。
再要雅淡之中带些娇艳。
”莺儿道:“葱绿柳黄是我最爱的。
”宝玉道:“也罢了,也打一条桃红,再打一条葱绿。
”莺儿道:“什么花样呢?”宝玉道:“共有几样花样?”莺儿道:“一炷香、朝天凳、象眼块、方胜、连环、梅花、柳叶。
”\zhu{-一灶香……柳叶:这里是各种编织图案的名称。
一灶香:直线形。
朝天凳:梯形。
象眼块:菱形。
方胜:一角相叠的两个菱形。
连环:两个套连的圆环。
梅花、柳叶:梅花形、柳叶形的图样。
}宝玉道:“前儿你替三姑娘打的那花样是什么?”莺儿道:“那是攒心梅花。
”\zhu{攒心:即向中心聚拢。
}宝玉道:“就是那样好。
”一面说,一面叫袭人,刚拿了线来,窗外婆子说“姑娘们的饭都有了。
”宝玉道:“你们吃饭去,快吃了来罢。
”袭人笑道:“有客在这里,我们怎好去的!”\meng{人情物理,一丝不乱。
}莺儿一面理线,一面笑道:“这话又打那里说起,正经快吃了来罢。
”袭人等听说方去了,只留下两个小丫头听呼唤。
\par
宝玉一面看莺儿打络子,一面说闲话,因问他:“十几岁了?”莺儿手里打着,一面答话说:“十六岁了。
”宝玉道:“你本姓什么?”莺儿道:“姓黄。
”宝玉笑道:“这个名姓倒对了,果然是个黄莺儿。
”莺儿笑道:“我的名字本来是两个字,叫作金莺。
姑娘嫌拗口,就单叫莺儿,如今就叫开了。
”宝玉道:“宝姐姐也算疼你了。
明儿宝姐姐出阁,少不得是你跟去了。
”莺儿抿嘴一笑。
宝玉笑道:“我常常和袭人说,明儿不知那一个有福的消受你们主子奴才两个呢。
”\meng{是有心?是无心?}莺儿笑道:“你还不知道,我们姑娘有几样世人都没有的好处呢,模样儿还在次。
”
\ping{莺儿配合宝钗对宝玉发起攻势,吊起宝玉的胃口。}
宝玉见莺儿娇憨婉转,语笑如痴,早不胜其情了,那更提起宝钗来!便问他道:“好处在那里?好姐姐,细细告诉我听。
”莺儿笑道:“我告诉你,你可不许又告诉他去。
”\meng{闺房闲话,着实幽韵。
}宝玉笑道:“这个自然的。
”正说着,只听外头说道:“怎么这样静悄悄的!”二人回头看时,不是别人,正是宝钗来了。
宝玉忙让坐。
宝钗坐了,因问莺儿“打什么呢?”一面问,一面向他手里去瞧,才打了半截。
宝钗笑道:“这有什么趣儿,倒不如打个络子把玉络上呢。
”一句话提醒了宝玉,便拍手笑道:“倒是姐姐说得是,我就忘了。
只是配个什么颜色才好?”宝钗道:“若用杂色断然使不得,大红又犯了色,\zhu{犯了色:色彩不协调。
}
黄的又不起眼,黑的又过暗。
等我想个法儿:把那金线拿来,配着黑珠儿线,一根一根的拈上,打成络子,这才好看。
”\ping{宝钗提出,用金线把玉络上,暗示用自己的金锁连结宝玉的玉,隐秘的表达了自己对于宝玉的爱,希望金玉良缘能够实现。
}\par
宝玉听说,喜之不尽,一叠声便叫袭人来取金线。
正值袭人端了两碗菜走进来,告诉宝玉道:“今儿奇怪,才刚太太打发人给我送了两碗菜来。
”宝玉笑道:“必定是今儿菜多,送来给你们大家吃的。
”袭人道:“不是,指名给我送来的,还不叫我过去磕头。
这可是奇了。
”宝钗笑道:“给你的,你就吃了,这有什么可猜疑的。
”袭人笑道:“从来没有的事,倒叫我不好意思的。
”宝钗抿嘴一笑,说道:“这就不好意思了?\meng{宝\sout{玉}[钗]之慧性灵心。
}明儿比这个更叫你不好意思的还有呢。
”袭人听了话内有因,素知宝钗不是轻嘴薄舌奚落人的,自己方想起上日王夫人的意思来,\ping{王夫人提出要把袭人的待遇提高到姨娘的待遇。
这样私密的事情,宝钗竟然略知一二,可见王夫人对宝钗的钟爱信任。
}便不再提,将菜与宝玉看了,说:“洗了手来拿线。
”说毕,便一直的出去了。
吃过饭,洗了手,进来拿金线与莺儿打络子。
此时宝钗早被薛蟠遣人来请出去了。
\par
这里宝玉正看着打络子,忽见邢夫人那边遣了两个丫鬟送了两样果子来与他吃,问他“可走得了?若走得动,叫哥儿明儿过来散散心,太太着实记挂着呢。
”宝玉忙道:“若走得了,必请太太的安去。
疼的比先好些,请太太放心罢。
”一面叫他两个坐下,一面又叫秋纹来,把才拿来的那果子拿一半送与林姑娘去。
秋纹答应了,刚欲去时,只听黛玉在院内说话,宝玉忙叫:“快请。
”要知端的,且听下回分解。
\par
\qi{总评:此回是以情说法,警醒世人。
黛玉因情凝思默度,忘其有身,忘其有病;而宝玉千屈万折,因情忘其尊卑,忘其痛苦,并忘其性情。
爱河之深无底,何可泛滥,一溺其中,非死不止。
且泛爱者不专,新旧叠增,岂能尽了?其多情之心不能不流于无情之地。
究其立意,倏忽千里而自不觉。
诚可悲乎!}
\dai{069}{白玉钏亲尝莲叶羹}
\dai{070}{黄金莺巧结梅花络}
\sun{p34-1}{情中情因情感妹妹,错里错以错劝哥哥}{图下侧:宝玉昏昏沉沉中恍惚听得悲切之声,睁眼一看,却是黛玉,只见她两个眼晴肿得桃儿一般,满脸泪光。
此时黛玉虽不是嚎啕大哭,气噎喉堵,更觉厉害。
半天,方抽抽噎噎地道:“你可都改了罢!”图左上:宝钗听袭人说是薛蟠告发的琪官之事,信以为真,便告母亲。
从外面喝酒回来的薛蟠,闻得此言,深感冤枉,一时性起,就要大闹, 亏得宝钗劝住。
图右上:次日,黛玉见贾母及众人纷纷来看宝玉, 想起有父母的好处,早又泪珠满面。
又听廊下鹦哥学舌,便令将鹦哥挂在窗下,聊以释闷。
}