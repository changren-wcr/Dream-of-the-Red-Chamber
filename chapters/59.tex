\chapter{柳叶渚边嗔莺咤燕 \quad 绛芸轩里召将飞符}
\qi{山无起伏,便是顽山;\zhu{顽山:指浑沦未破未被开发的山。
}水无潆洄,便是死水。
此回于前文叙过事,字字应;于后回未叙事,语语伏。
是上下关节。
至铸鼎象物手段,\zhu{铸鼎象物:
夏禹曾经收九州之金,铸成上有百物之图形的九个鼎。见《左传·宣公三年》。旧时用此比喻君王有德。
这里用来小说刻画人物描写场景毕肖。
}则在下回施展。
}\par
话说宝玉听说贾母等回来,遂多添了一件衣服,拄杖前边来,都见过了。
贾母等因每日辛苦,都要早些歇息,一宿无话,次日五鼓,又往朝中去。
\par
离送灵日不远,鸳鸯、琥珀、翡翠、玻璃四人都忙着打点贾母之物,玉钏、彩云、彩霞等皆打叠王夫人之物,当面查点与跟随的管事媳妇们。
跟随的一共大小六个丫鬟,十个老婆子媳妇子,男人不算。
连日收拾驮轿器械。
\zhu{驮轿:也叫骡驮轿,前后两匹牲口抬着走的轿子。
}鸳鸯与玉钏儿皆不随去,只看屋子。
一面先几日预发帐幔铺陈之物,先有四五个媳妇并几个男人领了出来,坐了几辆车绕道先至下处,铺陈安插等候。
\par
临日,贾母带着蓉妻坐一乘驮轿,王夫人在后亦坐一乘驮轿,贾珍骑马率了众家丁护卫。
又有几辆大车与婆子丫鬟等坐,并放些随换的衣包等件。
是日薛姨妈尤氏率领诸人直送至大门外方回。
贾琏恐路上不便,一面打发了他父母起身赶上贾母王夫人驮轿,\zhu{打发:安排。
这里是子对父,不应该解释成派遣。
}自己也随后带领家丁押后跟来。
\par
荣府内赖大添派人丁上夜,将两处厅院都关了,一应出入人等皆走西边小角门。
日落时,便命关了仪门,\zhu{
仪门:旧时官衙、府第的大门之内的门,具装饰作用。
一说,旁门也可称仪门。
}不放人出入。
园中前后东西角门亦皆关锁,只留王夫人大房之后常系他姊妹出入之门、东边通薛姨妈的角门,这两门因在内院,不必关锁。
里面鸳鸯和玉钏儿也各将上房关了,自领丫鬟婆子下房去安歇。
每日林之孝之妻进来,带领十来个婆子上夜,穿堂内又添了许多小厮们坐更打梆子,\zhu{坐更[gēng]:夜间警卫。
梆子:巡更或旧时衙门用以集散人众所敲的响器。
用竹子或挖空的木头制成。
}已安插得十分妥当。
\par
一日清晓,宝钗春困已醒,搴帷下榻,\zhu{搴帷:音“千维”,撩起帷幕。
}微觉轻寒,启户视之,见园中土润苔青,原来五更时落了几点微雨。
于是唤起湘云等人来,一面梳洗,湘云因说两腮作痒,恐又犯了杏癍癣,因问宝钗要些蔷薇硝来。
\zhu{
癣:西医认为它是指侵犯皮肤、毛发和指甲的浅部真菌感染。
传播途径或因与病人或猫、狗等动物直接接触,或因穿戴别人的衣帽、鞋袜和理发工具的间接接触。
中医认为,癣是因风热湿邪,侵袭皮肤,郁久风盛而起,以瘙痒为主要症状的一类皮肤病。
这里所说的杏癍癣,是癣病中的一种,由于这种病多发于春季,所以也叫春癣。
硝:硝石、芒硝、朴硝等矿物盐的统称。
蔷薇硝:掺入蔷薇花等研制的硝。
用蔷薇硝擦春癣:《本草纲目》说蔷薇主治“痈疽恶疮,结肉跌筋”。
蔷薇根也可治“痈疽疥癣”。
李时珍说,它“能入阳明经,除风热湿热,生肌杀虫,故痈疽疮癣,古方常用”。
}宝钗道:“前儿剩的都给了妹子。
”因说:“颦儿配了许多,我正要和他要些,因今年竟没发痒,就忘了。
”因命莺儿去取些来。
莺儿应了才去时,蕊官便说:“我同你去,顺便瞧瞧藕官。
”说着,一径同莺儿出了蘅芜苑。
\par
二人你言我语,一面行走,一面说笑,不觉到了柳叶渚,顺着柳堤走来。
因见柳叶才吐浅碧,丝若垂金,莺儿便笑道:“你会拿着柳条子编东西不会?”蕊官笑道:“编什么东西?”莺儿道:“什么编不得?顽的使的都可。
等我摘些下来,带着这叶子编个花篮儿,采了各色花放在里头,才是好顽呢。
”说着,且不去取硝,且伸手挽翠披金,采了许多的嫩条,命蕊官拿着。
他却一行走一行编花篮,随路见花便采一二枝,编出一个玲珑过梁的篮子。
\zhu{梁:人体或物体隆起的部分。“过梁”应该是指篮子上编的提手。
}枝上自有本来翠叶满布,将花放上,却也别致有趣。
喜的蕊官笑道:“姐姐,给了我罢。
”莺儿道:“这一个咱们送林姑娘,回来咱们再多采些,编几个大家顽。
”说着,来至潇湘馆中。
\par
黛玉也正晨妆,见了篮子,便笑说:“这个新鲜花篮是谁编的?”莺儿笑说:“我编了送姑娘顽的。
”黛玉接了笑道:“怪道人赞你的手巧,这顽意儿却也别致。
”一面瞧了,一面便命紫鹃挂在那里。
莺儿又问候了薛姨妈,方和黛玉要硝。
黛玉忙命紫鹃包了一包,递与莺儿。
黛玉又道:“我好了,今日要出去逛逛。
你回去说与姐姐,不用过来问候妈了,也不敢劳他来瞧我,梳了头同妈都往你那里去,连饭也端了那里去吃,大家热闹些。
”\par
莺儿答应了出来,便到紫鹃房中找蕊官。
只见藕官与蕊官二人正说得高兴,不能相舍,因说:“姑娘也去呢,藕官先同我们去等着岂不好?”紫鹃听如此说,便也说道:“这话倒是,他这里淘气的也可厌。
”
一面说,一面便将黛玉的匙箸用一块洋巾包了,交与藕官道:“你先带了这个去,也算一趟差了。
”\par
藕官接了,笑嘻嘻同他二人出来,一径顺着柳堤走来。
莺儿便又采些柳条,越性坐在山石上编起来,又命蕊官先送了硝去再来。
他二人只顾爱看他编,那里舍得去。
莺儿只顾催说:“你们再不去,我也不编了。
”藕官便说:“我同你去了,再快回来。
”二人方去了。
\par
这里莺儿正编,只见何婆的女儿春燕走来,笑问:“姐姐编什么呢?”正说着,蕊、藕二人也到了。
春燕便向藕官道:“前儿你到底烧什么纸?被我姨妈看见了,要告你没告成,倒被宝玉赖了他一大些不是,气的他一五一十告诉我妈。
你们在外头这二三年积了些什么仇恨,如今还不解开?”藕官冷笑道:“有什么仇恨?他们不知足,反怨我们了。
在外头这两年,别的东西不算,只算我们的米菜,不知赚了多少家去,合家子吃不了,还有每日买东买西赚的钱。
在外逢我们使他们一使儿,就怨天怨地的。
你说说可有良心?”春燕笑道:“他是我的姨妈,也不好向着外人反说他的。
怨不得宝玉说:‘女孩儿未出嫁,是颗无价之宝珠;出了嫁,不知怎么就变出许多的毛病来,虽是颗珠子,却没有光彩宝色,是颗死珠了;再老了,更变的不是珠子,竟是鱼眼睛了。
分明一个人,怎么变出三样来?’这话虽是混话,倒也有些不差。
别人不知道,只说我妈和姨妈,他老姊妹两个,如今越老了越把钱看的真了。
先时老姐儿两个在家抱怨没个差使,没个进益,幸亏有了这园子,把我挑进来,可巧把我分到怡红院。
家里省了我一个人的费用不算外,每月还有四五百钱的馀剩,这也还说不够。
后来老姊妹二人都派到梨香院去照看他们,藕官认了我姨妈,芳官认了我妈,这几年着实宽裕了。
如今挪进来也算撒开手了,还只无厌。
你说好笑不好笑?我姨妈刚和藕官吵了,接着我妈为洗头就和芳官吵。
芳官连要洗头也不给他洗。
昨日得月钱,推不去了,买了东西先叫我洗。
我想了一想:我自有钱,就没钱要洗时,不管袭人、晴雯、麝月,那一个跟前和他们说一声,也都容易,何必借这个光儿?好没意思。
所以我不洗。
他又叫我妹妹小鸠儿洗了,\zhu{鸠:鸟名,也称鸬鸠、斑鸠。
《诗经·召南·鹊巢》:“维鹊有巢,维鸠盈之。
”}才叫芳官,果然就吵起来。
接着又要给宝玉吹汤,你说可笑死了人?我见他一进来,我就告诉那些规矩。
他只不信,只要强做知道的,足的讨个没趣儿。
幸亏园里的人多,没人分记的清楚谁是谁的亲故。
若有人记得,只有我们一家人吵,什么意思呢?你这会子又跑来弄这个。
这一带地上的东西都是我姑娘管着,\zhu{姑娘:姑姑。
}一得了这地方,比得了永远基业还利害,每日早起晚睡,自己辛苦了还不算,每日逼着我们来照看,生恐有人糟踏,又怕误了我的差使。
如今进来了,老姑嫂两个照看得谨谨慎慎,一根草也不许人动。
你还掐这些花儿,又折他的嫩树,他们即刻就来,仔细他们抱怨。
”莺儿道:“别人乱折乱掐使不得,独我使得。
自从分了地基之后,每日里各房皆有分例,吃的不用算,单管花草顽意儿。
谁管什么,每日谁就把各房里姑娘丫头戴的,必要各色送些折枝的去,还有插瓶的。
惟有我们说了:‘一概不用送,等要什么再和你们要。
’究竟没有要过一次。
我今便掐些,他们也不好意思说的。
”\par
一语未了,他姑娘果然拄了拐走来。
莺儿春燕等忙让坐。
那婆子见采了许多嫩柳,又见藕官等都采了许多鲜花,心内便不受用;看着莺儿编,又不好说什么,便说春燕道:“我叫你来照看照看,你就贪住顽不去了。
倘或叫起你来,你又说我使你了,拿我做隐身符儿你来乐。
”\zhu{隐身符儿:迷信认为能使身体隐匿不被人见的符录(符录:道士巫师所画的一种图形或线条,相传可以役鬼神,辟病邪)。
在这里是挡箭牌的意思。
}春燕道:“你老又使我,又怕,这会子反说我。
难道把我劈做八瓣子不成?”莺儿笑道:“姑妈,你别信小燕的话。
这都是他摘下来的,烦我给他编,我撵他,他不去。
”春燕笑道:“你可少顽儿,你只顾顽儿,他老人家就认真了。
”那婆子本是愚顽之辈,兼之年近昏耄,\zhu{昏耄:昏聩糊涂的老年。
耄:音“冒”,老。
}惟利是命,一概情面不管,正心疼肝断,无计可施,听莺儿如此说,便以老卖老,拿起拄杖来向春燕身上击上几下,骂道:“小蹄子,我说着你,你还和我强嘴儿呢。
你妈恨的牙根痒痒,要撕你的肉吃呢。
你还来和我强梆子似的。
”\zhu{梆子:打击乐器。
用两根长短不同的硬木棒制成,两手各执其一,互击发音以按节拍。
强梆子:可能是指说话声高,节奏快,气势汹汹,类似于前文的“强嘴儿”,即争吵的意思。
}打的春燕又愧又急,哭道:“莺儿姐姐顽话,你老就认真打我。
我妈为什么恨我?我又没烧胡了洗脸水,有什么不是!”莺儿本是顽话,忽见婆子认真动了气,忙上去拉住,笑道:“我才是顽话,你老人家打他,我岂不愧?”那婆子道:“姑娘,你别管我们的事,难道为姑娘在这里,不许我管孩子不成?”莺儿听见这般蠢话,便赌气红了脸,撒了手冷笑道:“你老人家要管,那一刻管不得,偏我说了一句顽话就管他了。
我看你老管去!”说着,便坐下,仍编柳篮子。
\ping{婆子明面上打侄女春燕出气,暗地里不满莺儿折枝摘花。
婆子不敢直接冲比自己地位高的宝钗贴身丫鬟莺儿发火,只能打侄女春燕杀鸡吓猴旁敲侧击。
}\par
偏又有春燕的娘出来找他,喊道:“你不来舀水,在那里做什么呢?”那婆子便接声儿道:“你来瞧瞧,你的女儿连我也不服了!在那里排揎我呢。
”\zhu{排揎(揎音“宣”):数落、责难的意思。
}那婆子一面走过来说:“姑奶奶,又怎么了?我们丫头眼里没娘罢了,连姑妈也没了不成?”莺儿见他娘来了,只得又说原故。
他姑娘那里容人说话,便将石上的花柳与他娘瞧道:“你瞧瞧,你女儿这么大孩子顽的。
他先领着人糟踏我,我怎么说人?”他娘也正为芳官之气未平,又恨春燕不遂他的心,便走上来打耳刮子,骂道:“小娼妇,你能上去了几年?你也跟那起轻狂浪小妇学,怎么就管不得你们了?干的我管不得,\zhu{干的:干女儿。
}你是我屄里掉出来的,难道也不敢管你不成!既是你们这起蹄子到的去的地方我到不去,你就该死在那里伺候,又跑出来浪汉。
”一面又抓起柳条子来,直送到他脸上,问道:“这叫作什么?这编的是你娘的屄!”\ping{这难道不是我骂我自己吗?}莺儿忙道:“那是我们编的,你老别指桑骂槐。
”那婆子深妒袭人晴雯一干人,\ping{埋下伏笔,晴雯被举报诬陷。
}已知凡房中大些的丫鬟都比他们有些体统权势,凡见了这一干人,心中又畏又让,未免又气又恨,亦且迁怒于众,复又看见了藕官,又是他令姊的冤家,四处凑成一股怒气。
\par
那春燕啼哭着往怡红院去了。
他娘又恐问他为何哭,怕他又说出自己打他,又要受晴雯等之气,不免着起急来,又忙喊道:“你回来!我告诉你再去。
”春燕那里肯回来?急的他娘跑了去又拉他。
他回头看见,便也往前飞跑。
他娘只顾赶他,不防脚下被青苔滑倒,引的莺儿三个人反都笑了。
莺儿便赌气将花柳皆掷于河中,自回房去。
这里把个婆子心疼的只念佛,又骂:“促狭小蹄子!糟踏了花儿,雷也是要打的。
”自己且掐花与各房送去不提。
\par
却说春燕一直跑入院中,顶头遇见袭人往黛玉处去问安。
春燕便一把抱住袭人,说:“姑娘救我!我娘又打我呢。
”袭人见他娘来了,不免生气,便说道:“三日两头儿打了干的打亲的,还是卖弄你女儿多,还是认真不知王法?”这婆子来了几日,见袭人不言不语是好性的,便说道:“姑娘你不知道,别管我们闲事!都是你们纵的,这会子还管什么?”说着,便又赶着打。
袭人气的转身进来,见麝月正在海棠下晾手巾,听得如此喊闹,便说:“姐姐别管,看他怎样。
”一面使眼色与春燕,春燕会意,便直奔了宝玉去。
众人都笑说:“这可是没有的事都闹出来了。
”麝月向婆子道:“你再略煞一煞气儿,难道这些人的脸面,和你讨一个情还讨不下来不成?”那婆子见他女儿奔到宝玉身边去,又见宝玉拉了春燕的手说:“别怕,有我呢。
”春燕又一行哭,又一行说,把方才莺儿等事都说出来。
宝玉越发急起来,说:“你只在这里闹也罢了,怎么连亲戚也都得罪起来?”麝月又向婆子及众人道:“怨不得这嫂子说我们管不着他们的事,我们虽无知错管了,如今请出一个管得着的人来管一管,嫂子就心服口服,也知道规矩了。
”便回头叫小丫头子:“去把平儿给我们叫来!平儿不得闲就把林大娘叫了来。
”那小丫头应了就走。
众媳妇上来笑说:“嫂子,快求姑娘们叫回那孩子罢。
平姑娘来了,可就不好了。
”那婆子说道:“凭你那个平姑娘来也凭个理,没有娘管女儿大家管着娘的。
”众人笑道:“你当是那个平姑娘?是二奶奶屋里的平姑娘。
他有情呢,说你两句;他一翻脸,嫂子你吃不了兜着走!”\par
说话之间,只见小丫头子回来说:“平姑娘正有事,问我作什么,我告诉了他,他说:‘既这样,且撵他出去,告诉了林大娘在角门外打他四十板子就是了。
’”那婆子听如此说,自不舍得出去,便又泪流满面,央告袭人等说:“好容易我进来了,况且我是寡妇,家里没人,正好一心无挂的在里头伏侍姑娘们。
姑娘们也便宜,\zhu{便宜:音“变仪”,方便,顺当。
}我家里也省些搅过。
\zhu{搅(音“嚼”)过:即日常的吃穿用度。
}
我这一去,又要去自己生火过活,\zhu{过活:度日,生活。
}将来不免又没了过活。
”\zhu{
过活:靠着生活的财产。
}袭人见他如此,早又心软了,便说:“你既要在这里,又不守规矩,又不听说,又乱打人。
那里弄你这个不晓事的来,天天斗口,也叫人笑话,失了体统。
”晴雯道:“理他呢,打发去了是正经。
谁和他去对嘴对舌的。
”\ping{晴雯心直口快,得罪婆子,埋下被算计诬陷的种子。
}那婆子又央众人道:“我虽错了,姑娘们吩咐了,我以后改过。
姑娘们那不是行好积德。
”一面又央春燕道:“原是我为打你起的,究竟没打成你,我如今反受了罪?你也替我说说。
”宝玉见如此可怜,只得留下,吩咐他不可再闹。
那婆子走来一一的谢过了下去。
\par
只见平儿走来,问系何事。
\ping{前文传话的小丫头回来得非同寻常地快,很可能是打发别人去请平儿来,自己却假传“圣旨”,吓唬婆子。
这里平儿根本不知道,证明了这一点。
}袭人等忙说:“已完了,不必再提。
”平儿笑道:“‘得饶人处且饶人’,得省的将就省些事也罢了。
能去了几日,只听各处大小人儿都作起反来了,一处不了又一处,叫我不知管那一处的是。
”袭人笑道:“我只说我们这里反了,原来还有几处。
”平儿笑道:“这算什么。
正和珍大奶奶算呢,这三四日的工夫,一共大小出来了八九件了。
你这里是极小的,算不起数儿来,还有大的可气可笑之事。
”不知袭人问他果系何事,且听下回分解。
\par
\qi{总评:苏堤柳暖,阆苑春浓,\zhu{阆(音“郎”)苑:神仙的园林。
}兼之晨妆初罢,疏雨梧桐,正可借软草以慰佳人,采奇花以寄公子。
不意莺嗔燕怒,逗起波涛,婆子长舌,丫鬟碎语,群相聚讼,\zhu{讼:争辩是非,打官司。
}又是一样烘云托月法。
\zhu{烘云托月:烘云托月本是中国画中一种艺术表现手法,即画月亮并不涂白色或勾勒,而是用较淡或极淡的水墨去烘染月亮周围的云朵,即烘染云朵而衬托出月亮来,简言之即烘托法。这实际是文学创作中的衬托法或反衬法。}
}
\dai{117}{莺儿编花篮,春燕成替罪羊挨打,莺儿赌气将花柳掷于河中}
\dai{118}{宝玉庇护春燕,婆子流泪害怕被撵走}
\sun{p59-1}{莺儿取硝路编花篮}{莺儿、蕊官来到柳叶渚,顺着柳堤走来,见柳叶才吐浅碧,丝若垂金。
莺儿便采了许多嫩条,令蕊官拿着,自己就编起来,又采花一二枝,编出一个玲珑的篮子。
这时,何婆的女儿春燕走来。
}
\sun{p59-2}{柳叶渚边嗔莺咤燕,绛芸轩里召将飞符}{图右侧:春燕姑妈拄杖而至,见采了许多嫩柳和鲜花,心疼不已,不便怪罪莺儿,就拿春燕出气。
此时何婆又来了,因正生芳官的气,见状也拿春燕撒气。
图左侧:春燕哭着跑回怡红院。
何婆赶着打来,宝玉见了拉着春燕的手说:“别怕,有我呢。
”麝月见何婆气急败坏,叫小丫头去把平儿叫来。
}