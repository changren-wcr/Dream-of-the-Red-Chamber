\chapter{敏探春兴利除宿弊 \quad 时宝钗小惠全大体}
\qi{叙入梦景,极迷离,却极分明。
牛鬼蛇神不犯笔端,\zhu{这句话的意思是,笔端不写牛鬼蛇神。
}全从至情至理中写出,《齐谐》莫能载也。
\zhu{《齐谐》:古书名,《庄子·逍遥游》:“齐谐者,志怪者也。
” 后志怪之书以及敷演此类故事的戏剧,多以“齐谐”为名。
亦指谈笑说怪。
}}\par
话说平儿陪着凤姐儿吃了饭,伏侍盥漱毕,方往探春处来。
只见院中寂静,只有丫鬟婆子诸内壸近人在窗外听候。
\zhu{内壸:即内室。
壸:音“捆”,通“阃”。
宫中的间道,引申为内宫的代称。
}\par
平儿进入厅中,他姊妹三人正议论些家务,说的便是年内赖大家请吃酒他家花园中事故。
见他来了,探春便命他脚踏上坐了,因说道:“我想的事不为别的,因想着我们一月有二两月银外,丫头们又另有月钱。
前儿又有人回,要我们一月所用的头油脂粉,每人又是二两。
这又同才刚学里的八两一样,重重叠叠,事虽小,钱有限,看起来也不妥当。
你奶奶怎么就没想到这个?”\par
平儿笑道:“这有个原故:姑娘们所用的这些东西,自然是该有分例。
每月买办买了,令女人们各房交与我们收管,不过预备姑娘们使用就罢了,没有一个我们天天各人拿钱找人买头油又是脂粉去的理。
所以外头买办总领了去,按月使女人按房交与我们的。
姑娘们的每月这二两,原不是为买这些的,原为的是一时当家的奶奶太太或不在,或不得闲,姑娘们偶然一时可巧要几个钱使,省得找人去。
这原是恐怕姑娘们受委屈,可知这个钱并不是买这个才有的。
如今我冷眼看着,各房里的我们的姊妹都是现拿钱买这些东西的,竟有一半。
我就疑惑,不是买办脱了空,\zhu{脱空:落空;没有着落;弄虚作假。
}迟些日子,就是买的不是正经货,弄些使不得的东西来搪塞。
”探春李纨都笑道:“你也留心看出来了。
脱空是没有的,也不敢,只是迟些日子;催急了,不知那里弄些来,不过是个名儿,其实使不得,依然得现买。
就用这二两银子,另叫别人的奶妈子的或是弟兄哥哥的儿子买了来才使得。
若使了官中的人,依然是那一样的。
不知他们是什么法子,是铺子里坏了不要的,他们都弄了来,单预备给我们?”平儿笑道:“买办买的是那样的,他买了好的来,买办岂肯和他善开交,又说他使坏心要夺这买办了。
所以他们也只得如此,宁可得罪了里头,不肯得罪了外头办事的人。
\ping{自古以来,采购就是肥差。
根本原因是事后信息不对称和委托人/代理人之间利益不一致导致的道德风险问题,采购方作为直接接触卖家的代理人,比委托人有更多的私人信息,由此可以获得信息租金。
如果采用报销的制度设计,委托人按照采购和卖家协商的价格完成交易,那么存在采购和卖家串通作弊,标高售价。
比正常价格高的那部分,先成为卖家的利润,卖家再把其中一部分以回扣的形式给采购。
如果采取承包的制度设计,委托人给承包人固定承包金额,要求承包人交付某产品,那么采购就会为了省钱,买劣质低价的产品,多剩下来承包金。
采购的产品质量一般不够好,一方面是因为采购和委托人利益不一致,采购为了多赚钱以次充好,和委托人的利益相冲突;另一方面是事前信息不对称导致的次品问题(柠檬市场),委托人不知道产品质量的好坏,好的价格高,差的价格低,在质量不确定的情况下,出价平均化,介于高低之间,只有价格低的差品才能完成交易。
同样的,销售也是肥差。
通过低于市场价甚至成本价把产品卖给客户是利益输送的方式之一。
}姑娘们只能可使奶妈妈们,他们也就不敢闲话了。
”探春道:“因此我心中不自在。
钱费两起,东西又白丢一半,通算起来,反费了两折子,不如竟把买办的每月蠲了为是。
此是一件事。
第二件,年里往赖大家去,你也去的,你看他那小园子比咱们这个如何?”平儿笑道:“还没有咱们这一半大,树木花草也少多了。
”探春道:“我因和他家女儿说闲话儿,谁知那么个园子,除他们戴的花、吃的笋菜鱼虾之外,一年还有人包了去,年终足有二百两银子剩。
从那日我才知道,一个破荷叶,一根枯草根子,都是值钱的。
”\par
宝钗笑道:“真真膏粱纨绮之谈。
虽是千金小姐,原不知这事,但你们都念过书识字的,竟没看见朱夫子有一篇《不自弃文》不成?”\zhu{《不自弃文》:见于《朱子文集大全类编》。
大意为天下之物即便是顽石、蝮蛇、粪便、草灰等等皆因其有一节之可取而不为世之所弃,“今人而见弃焉,特其自弃尔”。
故人不应自弃,不宜“怨天尤人”而当“反求诸己”,思“祖德”、念“父功”,作成自身事业,以求“于身不弃,于人无愧,祖父不失其贻谋,子孙不沦于困辱”,从而保存和发展其祖宗的基业。
}探春笑道:“虽看过,那不过是勉人自励,虚比浮词,
\zhu{虚比浮词:虚夸不实的文辞。}
那里都真有的?”宝钗道:“朱子都有虚比浮词?那句句都是有的。
你才办了两天时事,就利欲熏心,把朱子都看虚浮了。
你再出去见了那些利弊大事,越发把孔子也看虚了!”探春笑道:“你这样一个通人,\zhu{通人:博古通今之人。
}竟没看见子书?当日《姬子》有云:‘登利禄之场,处运筹之界者,窃尧舜之词,背孔孟之道。
’”
\zhu{窃:这里应该是指窃用圣人之言为自己背书。背:违反;不遵守。}
宝钗笑道:“底下一句呢?”探春笑道:“如今只断章取意,念出底下一句,我自己骂我自己不成?”\zhu{《姬子》:中国古代诸子百家之中,未见有姬子之书,想是作者的杜撰。
因上文宝钗先提到朱子、孔子,探春欲以一个压得过朱子、孔子的人物来对答,便说是“姬子”,盖因周公姓姬,作为取笑之谈。
至于那几句讥嘲之辞,颇类李贽著作中的某些议论。
如:“口谈道德而心存高官、志在巨富;既已得高官巨富矣,仍讲道德说仁义自若。”(见《焚书·又与焦弱侯书》)。
“阳为道学,阴为富贵;被服儒雅,行若狗彘。”(见《续焚书·三教归儒说》)。
探春说“我自己骂我自己”,是因为《姬子》纯属虚构,哪里真的还有下文?作者借此点出所谓“姬子”即是探春自已。
探春借“姬子”说出自己未来理家立业的宣言,亦即她的那“一番道理”。
“姬”是古代对妇女的美称。把探春誉为“姬子”,暗寓“美人”亦可称“子”,与孔子、朱子们分庭抗礼,并以此“背孔孟之道”,足见作者对“姬子”探春的那套道理和做法是很推崇赞赏的。
}宝钗道:“天下没有不可用的东西;既可用,便值钱。
难为你是个聪敏人,这些正事大节目事竟没经历,\zhu{大节目:指事物的关键之处或主要部分。
}也可惜迟了。
”\ji{反点题,文法中又一变体也。
\zhu{
宝钗用“聪敏”评价探春,呼应题目前半句“敏探春兴利除宿弊”。
“反点题”中的“反”的含义是宝钗的的评价重点不在“聪敏”,而在于说探春见世面少且迟。
}
}李纨笑道:“叫了人家来,不说正事,且你们对讲学问。
”宝钗道:“学问中便是正事。
此刻于小事上用学问一提,那小事越发作高一层了。
不拿学问提着,便都流入市俗去了。
”\par
三人只是取笑之谈,说了笑了一回,便仍谈正事。
\ji{作者又用金蝉脱壳之法。
}探春因又接说道:“咱们这园子只算比他们的多一半,加一倍算,一年就有四百银子的利息。
若此时也出脱生发银子,自然小器,不是咱们这样人家的事。
若派出两个一定的人来,既有许多值钱之物,一味任人作践,也似乎暴殄天物。
不如在园子里所有的老妈妈中,拣出几个本分老诚能知园圃的事,派准他们收拾料理,也不必要他们交租纳税,只问他们一年可以孝敬些什么。
一则园子有专定之人修理,花木自有一年好似一年的,也不用临时忙乱;二则也不至作践,白辜负了东西;三则老妈妈们也可借此小补,不枉年日在园中辛苦;四则亦可以省了这些花儿匠山子匠打扫人等的工费。
将此有馀,以补不足,未为不可。
”宝钗正在地下看壁上的字画,听如此说一则,便点一回头,说完,便笑道:“善哉,三年之内无饥馑矣!”李纨笑道:“好主意。
这果一行,太太必喜欢。
省钱事小,第一有人打扫,专司其职,又许他们去卖钱。
使之以权,动之以利,再无不尽职的了。
”\ping{责任到人,包产到户。
}平儿道:“这件事须得姑娘说出来。
我们奶奶虽有此心,也未必好出口。
此刻姑娘们在园里住着,不能多弄些玩意儿去陪衬,反叫人去监管修理,图省钱,这话断不好出口。
”\par
宝钗忙走过来,摸着他的脸笑道:“你张开嘴,我瞧瞧你的牙齿舌头是什么作的。
从早起来到这会子,你说这些话,一套一个样子,也不奉承三姑娘,也没见你说奶奶才短想不到,也并没有三姑娘说一句,你就说一句是;横竖三姑娘一套话出,你就有一套话进去;总是三姑娘想的到的,你奶奶也想到了,只是必有个不可办的原故。
这会子又是因姑娘住的园子,不好因省钱令人去监管。
你们想想这话,若果真交与人弄钱去的,那人自然是一枝花也不许掐,一个果子也不许动了,姑娘们分中自然不敢,\zhu{分中:犹分内,本分以内。
}天天与小姑娘们就吵不清。
他这远愁近虑,不亢不卑。
他奶奶便不是和咱们好,听他这一番话,也必要自愧的变好了,不和也变和了。
”探春笑道:“我早起一肚子气,听他来了,忽然想起他主子来,素日当家使出来的好撒野的人,我见了他便生了气。
谁知他来了,避猫鼠儿似的站了半日,怪可怜的。
接着又说了那么些话,不说他主子待我好,倒说‘不枉姑娘待我们奶奶素日的情意了’。
\ping{
虽然凤姐管家的时候,实际上是照顾探春的,如果以施恩者的身份要求探春报恩,就显得居高临下了。
以求关照的低姿态,更能得人怜。
}
这一句,不但没了气,我倒愧了,又伤起心来。
我细想,我一个女孩儿家,自己还闹得没人疼没人顾的,我那里还有好处去待人。
”口内说到这里,不免又流下泪来。
李纨等见他说的恳切,又想他素日赵姨娘每生诽谤,在王夫人跟前亦为赵姨娘所累,亦都不免流下泪来,都忙劝道:“趁今日清净,大家商议两件兴利剔弊的事,也不枉太太委托一场。
又提这没要紧的事做什么?”平儿忙道:“我已明白了。
姑娘竟说谁好,竟一派人就完了。
”\zhu{竟:径,直接。
}
探春道:“虽如此说,也须得回你奶奶一声。
我们这里搜剔小遗,\zhu{遗:遗漏,忽略。
}已经不当,皆因你奶奶是个明白人,我才这样行,若是糊涂多蛊多妒的,\zhu{多蛊多妒:居心歹毒,多所猜疑和妒忌。
蛊:音“古”,毒虫。
《本草纲目·
虫部四》李时珍集解引陈藏器曰:“取百虫入瓮中.经年开之,必有一虫食尽诸虫,即此名为蛊”。
}我也不肯,倒像抓他乖一般。
\zhu{乖:谬误。
}岂可不商议了行。
”平儿笑道:“既这样,我去告诉一声。
”说着去了,半日方回来,笑说:“我说是白走一趟,这样好事,奶奶岂有不依的。
”\par
探春听了,便和李纨命人将园中所有婆子的名单要来,大家参度,大概定了几个。
又将他们一齐传来,李纨大概告诉与他们。
众人听了,无不愿意,也有说:“那一片竹子单交给我,一年工夫,明年又是一片。
除了家里吃的笋,一年还可交些钱粮。
”这一个说:“那一片稻地交给我,一年这些顽的大小雀鸟的粮食不必动官中钱粮,我还可以交钱粮。
”探春才要说话,人回:“大夫来了,进园瞧姑娘。
”众婆子只得去接大夫。
平儿忙说:“单你们,有一百个也不成个体统,难道没有两个管事的头脑带进大夫来?”回事的那人说:“有,吴大娘和单大娘他两个在西南角上聚锦门等着呢。
”平儿听说,方罢了。
\par
众婆子去后,探春问宝钗如何。
宝钗笑答道:“幸于始者怠于终,缮其辞者嗜其利。
”\zhu{幸于始者怠于终,缮其辞者嗜其利:幸:庆幸,这里是指因有利可图而感到侥幸。
缮:修补、整治。
嗜:特殊爱好。
全句意思是:开头因侥幸获利而兴头很高的人,最终是会懈怠的;嘴上说得好听的人,特别爱占便宜。
}探春听了点头称赞,便向册上指出几人来与他三人看。
平儿忙去取笔砚来。
他三人说道:“这一个老祝妈是个妥当的,况他老头子和他儿子代代都是管打扫竹子,如今竟把这所有的竹子交与他。
\ping{老祝妈:祝谐音“竹”,正好管竹子,因事起名。
}这一个老田妈本是种庄稼的,稻香村一带凡有菜蔬稻稗之类,\zhu{稗:音“拜”,稻田里的一种杂草。
}虽是顽意儿,不必认真大治大耕,也须得他去,再一按时加些培植,\zhu{一:另,又。
}岂不更好?”\ping{老田妈:姓田正好管田,因事起名。
}探春又笑道:“可惜,蘅芜苑和怡红院这两处大地方竟没有出利息之物。
”李纨忙笑道:“蘅芜苑更利害。
如今香料铺并大市大庙卖的各处香料香草儿,都不是这些东西?算起来比别的利息更大。
怡红院别说别的,单只说春夏天一季玫瑰花,共下多少花?还有一带篱笆上蔷薇、月季、宝相、\zhu{宝相:花名,属蔷薇科。
}金银藤,单这没要紧的草花干了,卖到茶叶铺药铺去,也值几个钱。
”探春笑道:“原来如此。
只是弄香草的没有在行的人。
”平儿忙笑道:“跟宝姑娘的莺儿他妈就是会弄这个的,上回他还采了些晒干了编成花篮葫芦给我顽的,姑娘倒忘了不成?”宝钗笑道:“我才赞你,你到来捉弄我了。
”三人都诧异,都问这是为何。
宝钗道:“断断使不得!你们这里多少得用的人,一个一个闲着没事办,这会子我又弄个人来,叫那起人连我也看小了。
我倒替你们想出一个人来:怡红院有个老叶妈,他就是茗烟的娘。
那是个诚实老人家,他又和我们莺儿的娘极好,不如把这事交与叶妈。
他有不知的,不必咱们说,他就找莺儿的娘去商议了。
那怕叶妈全不管,竟交与那一个,那是他们私情儿,有人说闲话,也就怨不到咱们身上了。
如此一行,你们办的又至公,于事又甚妥。
”李纨平儿都道:“是极。
”\ji{宝钗此等非与凤姐一样,此是随时俯仰,彼则逸才逾蹈也。
\zhu{
此:指的是宝钗。
彼:指的是凤姐。
俯仰:应付周旋。
逸才:指出众的人才、出众的才能。
逾:越过,超越。
蹈:踩,踏。
逸才逾蹈:锋芒毕露,才华超众。
}}探春笑道:“虽如此,只怕他们见利忘义。
”\ji{这是探春敏智过人处,此讽亦不可少。
}平儿笑道:“不相干,前儿莺儿还认了叶妈做干娘,请吃饭吃酒,两家和厚的好的很呢。
”\ji{夹写大观园中多少儿女家常闲景,此亦补前文之不足也。
}探春听了,方罢了。
又共同斟酌出几人来,俱是他四人素昔冷眼取中的,用笔圈出。
\par
一时婆子们来回大夫已去,将药方送上去。
三人看了,一面遣人送出去取药,监派调服,一面探春与李纨明示诸人:某人管某处,按四季除家中定例用多少外,馀者任凭你们采取了去取利,年终算帐。
探春笑道:“我又想起一件事:若年终算帐归钱时,自然归到帐房,仍是上头又添一层管主,还在他们手心里,又剥一层皮。
这如今我们兴出这事来派了你们,已是跨过他们的头去了,心里有气,只说不出来;你们年终去归帐,他还不捉弄你们等什么?再者,这一年间管什么的,主子有一全分,他们就得半分。
这是家里的旧例,人所共知的,别的偷着的在外。
\zhu{偷着的在外:在外和账房暗中舞弊。}
如今这园子里是我的新创,竟别入他们手,每年归帐,竟归到里头来才好。
”宝钗笑道:“依我说,里头也不用归帐。
这个多了那个少了,倒多了事。
不如问他们谁领这一分的,他就揽一宗事去。
不过是园里的人的动用。
我替你们算出来了,有限的几宗事:不过是头油、胭粉、香、纸,每一位姑娘几个丫头,都是有定例的;再者,各处笤帚、撮簸、\zhu{撮簸:即簸箕,用来簸粮食或撮垃圾等的一种器具。
}掸子并大小禽鸟、鹿、兔吃的粮食。
不过这几样,都是他们包了去,不用帐房去领钱。
你算算,就省下多少来?”平儿笑道:“这几宗虽小,一年通共算了,也省的下四百两银子。
”\par
宝钗笑道:“却又来,一年四百,二年八百两,取租的房子也能看得了几间,薄地也可添几亩。
虽然还有敷馀的,但他们既辛苦闹一年,也要叫他们剩些,粘补粘补自家。
虽是兴利节用为纲,然亦不可太啬。
纵再省上二三百银子,失了大体统也不像。
所以如此一行,外头帐房里一年少出四五百银子,也不觉得很艰啬了,他们里头却也得些小补。
这些没营生的妈妈们也宽裕了,园子里花木,也可以每年滋长蕃盛,你们也得了可使之物。
这庶几不失大体。
\zhu{庶几:希望,但愿。
}若一味要省时,那里不搜寻出几个钱来。
凡有些馀利的,一概入了官中,那时里外怨声载道,岂不失了你们这样人家的大体?如今这园里几十个老妈妈们,若只给了这个,那剩的也必抱怨不公。
我才说的,他们只供给这个几样,也未免太宽裕了。
一年竟除这个之外,他每人不论有馀无馀,只叫他拿出若干贯钱来,大家凑齐,单散与园中这些妈妈们。
他们虽不料理这些,却日夜也是在园中照看当差之人,关门闭户,起早睡晚,大雨大雪,姑娘们出入,抬轿子,撑船,拉冰床,\zhu{冰床:在冰上滑行用的小坐床,也称“冰排子”。
}一应粗糙活计,都是他们的差使。
一年在园里辛苦到头,这园内既有出息,也是分内该沾带些的。
还有一句至小的话,越发说破了:你们只管了自己宽裕,不分与他们些,他们虽不敢明怨,心里却都不服,只用假公济私的多摘你们几个果子,多掐几枝花儿,你们有冤还没处诉。
他们也沾带了些利息,你们有照顾不到,他们就替你照顾了。
”\ping{宝钗处事妥帖安稳,人人想到。
}\par
众婆子听了这个议论,又去了帐房受辖制,又不与凤姐儿去算帐,一年不过多拿出若干贯钱来,各各欢喜异常,都齐说:“愿意。
强如出去被他揉搓着,还得拿出钱来呢。
”那不得管地的听了每年终又无故得分钱,也都喜欢起来,口内说:“他们辛苦收拾,是该剩些钱粘补的。
我们怎么好‘稳坐吃三注’的?”\zhu{稳坐吃三注:不费气力而稳得多方钱财的意思。
注:赌注,用来赌博的财物。
三注:指押在上门、下门和天门三个位置上的赌注。
}\par
宝钗笑道:“妈妈们也别推辞了,这原是分内应当的。
你们只要日夜辛苦些,别躲懒纵放人吃酒赌钱就是了。
不然,我也不该管这事;你们一般听见,
\zhu{一般:一样,同样。}
姨娘亲口嘱托我三五回,说大奶奶如今又不得闲儿,别的姑娘又小,托我照看照看。
我若不依,分明是叫姨娘操心。
你们奶奶又多病多痛,家务也忙。
我原是个闲人,便是个街坊邻居,也要帮着些,何况是亲姨娘托我。
我免不得去小就大,讲不起众人嫌我。
倘或我只顾了小分沽名钓誉,那时酒醉赌博生出事来,我怎么见姨娘?你们那时后悔也迟了,就连你们素日的老脸也都丢了。
这些姑娘小姐们,这么一所大花园子,都是你们照看,皆因看得你们是三四代的老妈妈,最是循规遵矩的,原该大家齐心,顾些体统。
你们反纵放别人任意吃酒赌博,姨娘听见了,教训一场犹可,倘或被那几个管家娘子听见了,他们也不用回姨娘,竟教导你们一番。
你们这年老的反受了年小的教训,虽是他们是管家,管的着你们,何如自己存些体统,他们如何得来作践。
所以我如今替你们想出这个额外的进益来,也为大家齐心把这园里周全的谨谨慎慎,使那些有权执事的看见这般严肃谨慎,且不用他们操心,他们心里岂不敬伏。
也不枉替你们筹画进益,既能夺他们之权,生你们之利,岂不能行无为之治,分他们之忧。
你们去细想想这话。
”家人都欢声鼎沸说:“姑娘说的很是。
从此姑娘奶奶只管放心,姑娘奶奶这样疼顾我们,我们再要不体上情,天地也不容了。
”\par
刚说着,只见林之孝家的进来说:“江南甄府里家眷昨日到京,今日进宫朝贺。
此刻先遣人来送礼请安。
”\ping{传说中的甄家终于在现实里出现了,从前幻梦迷雾中的贾(假)府要彻底显露出千疮百孔的真(甄)相了。
}说着,便将礼单送上去。
探春接了,看道是:“上用的妆缎蟒缎十二匹,上用杂色缎十二匹,上用各色纱十二匹,上用宫绸十二匹,官用各色缎纱绸绫二十四匹。
”李纨也看过,说:“用上等封儿赏他。
”因又命人回了贾母。
贾母便命人叫李纨、探春、宝钗等也都过来,将礼物看了。
李纨收过,一边吩咐内库上人说:“等太太回来看了再收。
”贾母因说:“这甄家又不与别家相同,上等赏封赏男人,只怕展眼又打发女人来请安,预备下尺头。
”\zhu{尺头:绸缎衣料。
}一语未完,果然人回:“甄府四个女人来请安。
”贾母听了,忙命人带进来。
\par
那四个人都是四十往上的年纪,穿戴之物,皆比主子不甚差别。
请安问好毕,贾母命拿了四个脚踏来,他四人谢了坐,待宝钗等坐了,方都坐下。
贾母便问:“多早晚进京的?”四人忙起身回说:“昨儿进的京。
今日太太带了姑娘进宫请安去了,故令女人们来请安,问候姑娘们。
”贾母笑问道:“这些年没进京,也不想到今年来。
”四人也都笑回道:“正是,今年是奉旨进京的。
”贾母问道:“家眷都来了?”四人回说:“老太太和哥儿、两位小姐并别位太太都没来,就只太太带了三姑娘来了。
”贾母道:“有人家没有?”四人道:“尚没有。
”贾母笑道:“你们大姑娘和二姑娘这两家,都和我们家甚好。
”四人笑道:“正是。
每年姑娘们有信回去说,全亏府上照看。
”贾母笑道:“什么照看,原是世交,又是老亲,原应当的。
你们二姑娘更好,更不自尊自大,所以我们才走的亲密。
”四人笑道:“这是老太太过谦了。
”贾母又问:“你这哥儿也跟着你们老太太?”四人回说:“也是跟着老太太。
”贾母道:“几岁了?”又问:“上学不曾?”四人笑说:“今年十三岁。
因长得齐整,老太太很疼。
自幼淘气异常,天天逃学,老爷太太也不便十分管教。
”贾母笑道:“也不成了我们家的了!你这哥儿叫什么名字?”四人道:“因老太太当作宝贝一样,他又生的白,老太太便叫作宝玉。
”贾母便向李纨等道:“偏也叫作个宝玉。
”李纨忙欠身笑道:“从古至今,同时隔代重名的很多。
”四人也笑道:“起了这小名儿之后,我们上下都疑惑,不知那位亲友家也倒似曾有一个的。
只是这十来年没进京来,却记不得真了。
”贾母笑道:“岂敢,就是我的孙子。
——人来。
”众媳妇丫头答应了一声,走近几步。
贾母笑道:“园里把咱们的宝玉叫了来,给这四个管家娘子瞧瞧,比他们的宝玉如何?”\par
众媳妇听了,忙去了,半刻围了宝玉进来。
四人一见,忙起身笑道:“唬了我们一跳。
若是我们不进府来,倘若别处遇见,还只道我们的宝玉后赶着也进了京了呢。
”一面说,一面都上来拉他的手,问长问短。
宝玉忙也笑问好。
贾母笑道:“比你们的长的如何?”李纨等笑道:“四位妈妈才一说,可知是模样相仿了。
”贾母笑道:“那有这样巧事?大家子孩子们再养的娇嫩,除了脸上有残疾十分黑丑的,大概看去都是一样的齐整。
这也没有什么怪处。
”四人笑道:“如今看来,模样是一样。
据老太太说,淘气也一样。
我们看来,这位哥儿性情却比我们的好些。
”贾母忙问:“怎见得?”四人笑道:“方才我们拉哥儿的手说话便知。
我们那一个只说我们糊涂,慢说拉手,\zhu{慢说:别说;不要说。
表示让步、转折之意。
}他的东西我们略动一动也不依。
所使唤的人都是女孩子们。
”四人未说完,李纨姊妹等禁不住都失声笑出来。
贾母也笑道:“我们这会子也打发人去见了你们宝玉,若拉他的手,他也自然勉强忍耐一时。
可知你我这样人家的孩子们,凭他们有什么刁钻古怪的毛病儿,见了外人,必是要还出正经礼数来的。
若他不还正经礼数,也断不容他刁钻去了。
就是大人溺爱的,是他一则生的得人意,二则见人礼数竟比大人行出来的不错,使人见了可爱可怜,背地里所以才纵他一点子。
若一味他只管没里没外,不与大人争光,凭他生的怎样,也是该打死的。
”四人听了,都笑道:“老太太这话正是。
虽然我们宝玉淘气古怪,有时见了人客,规矩礼数更比大人有礼。
所以无人见了不爱,只说为什么还打他。
殊不知他在家里无法无天,大人想不到的话偏会说,想不到的事他偏要行,所以老爷太太恨的无法。
就是弄性,也是小孩子的常情,胡乱花费,这也是公子哥儿的常情,怕上学,也是小孩子的常情,都还治的过来。
第一,天生下来这一种刁钻古怪的脾气,如何使得。
”一语未了,人回:“太太回来了。
”王夫人进来问过安。
他四人请了安,大概说了两句。
贾母便命歇歇去。
王夫人亲捧过茶,方退出。
四人告辞了贾母,便往王夫人处来,说了一会家务,打发他们回去,不必细说。
\par
这里贾母喜的逢人便告诉,也有一个宝玉,也却一般行景。
\zhu{行景:状况、情形。
}众人都为天下之大,世宦之多,同名者也甚多,祖母溺爱孙者也古今所有常事耳,不是什么罕事,故皆不介意。
独宝玉是个迂阔呆公子的性情,\zhu{迂阔:不切合实际。
}自为是那四人承悦贾母之词。
后至蘅芜苑去看湘云病去,史湘云说他:“你放心闹罢,先是‘单丝不成线,独树不成林’,如今有了个对子,闹急了,再打很了,你逃走到南京找那一个去。
”宝玉道:“那里的谎话你也信了,偏又有个宝玉了?”湘云道:“怎么列国有个蔺相如,汉朝又有个司马相如呢?”宝玉笑道:“这也罢了,偏又模样儿也一样,这是没有的事。
”湘云道:“怎么匡人看见孔子,只当是阳虎呢?”\zhu{匡人看见孔子只当是阳虎:匡:春秋时卫国的地方,在今河南省长垣县境。
阳虎:即阳货,春秋时鲁国人,季孙氏家臣。
据《史记·孔子世家》载,孔子的相貌像阳虎,因阳虎欺压过匡人,所以孔子过匡,匡人曾把他当成阳虎围困了几天。
}宝玉笑道:“孔子、阳虎虽同貌,却不同名;蔺与司马虽同名,而又不同貌;偏我和他就两样俱同不成?”湘云没了话答对,因笑道:“你只会胡搅,我也不和你分证。
有也罢,没也罢,与我无干。
”说着便睡下了。
\par
宝玉心中便又疑惑起来:若说必无,然亦似有;若说必有,又并无目睹。
心中闷了,回至房中榻上默默盘算,不觉就忽忽的睡去,不觉竟到了一座花园之内。
宝玉诧异道:“除了我们大观园,竟又有这一个园子?”\ji{写园可知。
}正疑惑间,从那边来了几个女儿,都是丫鬟。
宝玉又诧异道:“除了鸳鸯、袭人、平儿之外,也竟还有这一干人?”\ji{写人可知。
妙在并不说“更强”二字。
}只见那些丫鬟笑道:“宝玉怎么跑到这里来了?”宝玉只当是说他,自己忙来陪笑说道:“因我偶步到此,不知是那位世交的花园,好姐姐们,带我逛逛。
”众丫鬟都笑道:“原来不是咱家的宝玉。
他生的倒也还干净,\ji{妙。
在玉卿身上只落了这两个字,亦不奇了。
}嘴儿也倒乖觉。
”宝玉听了,忙道:“姐姐们,这里也更还有个宝玉?”丫鬟们忙道:“宝玉二字,我们是奉老太太、太太之命,为保佑他延寿消灾的。
我叫他,他听见喜欢。
你是那里远方来的臭小厮,也乱叫起他来。
仔细你的臭肉,打不烂你的。
”又一个丫鬟笑道:“咱们快走罢,别叫宝玉看见,又说同这臭小厮说了话,把咱熏臭了。
”说着一径去了。
\par
宝玉纳闷道:“从来没有人如此涂毒我,
\zhu{涂毒:毒害,引申为侮辱。}
他们如何更这样?真亦有我这样一个人不成?”一面想,一面顺步早到了一所院内。
宝玉又诧异道:“除了怡红院,也更还有这么一个院落。
”忽上了台矶,进入屋内,只见榻上有一个人卧着,那边有几个女孩儿做针线,也有嘻笑顽耍的。
只见榻上那个少年叹了一声。
一个丫鬟笑问道:“宝玉,你不睡又叹什么?想必为你妹妹病了,你又胡愁乱恨呢。
”宝玉听说,心下也便吃惊。
只见榻上少年说道:“我听见老太太说,长安都中也有个宝玉,和我一样的性情,我只不信。
我才作了一个梦,竟梦中到了都中一个花园子里头,遇见几个姐姐,都叫我臭小厮,不理我。
好容易找到他房里头,偏他睡觉,空有皮囊,真性不知那去了。
”宝玉听说,忙说道:“我因找宝玉来到这里。
原来你就是宝玉?”榻上的忙下来拉住:“原来你就是宝玉?这可不是梦里了。
”宝玉道:“这如何是梦?真且又真了。
”一语未了,只见人来说:“老爷叫宝玉。
”唬得二人皆慌了。
一个宝玉就走,一个宝玉便忙叫:“宝玉快回来,快回来!”\par
袭人在旁听他梦中自唤,忙推醒他,笑问道:“宝玉在那里?”此时宝玉虽醒,神意尚恍惚,因向门外指说:“才出去了。
”袭人笑道:“那是你梦迷了。
你揉眼细瞧,是镜子里照的你影儿。
”宝玉向前瞧了一瞧,原是那嵌的大镜对面相照,自己也笑了。
早有人捧过漱盂茶卤来,\zhu{茶卤:这里指用以漱口的浓酽茶汁。
}
漱了口。
麝月道:“怪道老太太常嘱咐说小人屋里不可多有镜子。
小人魂不全,有镜子照多了,睡觉惊恐作胡梦。
如今倒在大镜子那里安了一张床。
有时放下镜套还好;往前去,天热困倦不定,那里想的到放他,比如方才就忘了。
自然是先躺下照着影儿顽的,一时合上眼,自然是胡梦颠倒;不然如何得看着自己叫着自己的名字?不如明儿挪进床来是正经。
”一语未了,只见王夫人遣人来叫宝玉,不知有何话说——\ji{此下紧接“慧紫鹃试忙玉”。
}\par
\qi{总评:探春看得透,拿得定,说得出,办得来,是有才干者,故赠以“敏”字;宝钗认的真,用的当,责的专,待的厚,是善知人者,故赠以“识”字。
“敏”与“识”合,何事不济?(按:回目“时宝钗”戚、蒙本作“识宝钗”。
\zhu{时:合时宜的。
另一说:通“是”。
善,好。
识:见识,知识。
})\hang
叙园圃事极板重,却极活泼。
营心孔方,\zhu{孔方:钱的谑称。
旧时铜钱外圆,中有方孔,故名。
}带以图记,\zhu{带以图记:意思可能是描写生动,如一幅画一样跃然纸上。
顺便带出了大观园中稻香村、蘅芜苑、怡红院等的蔬菜稻稗、香料香草香花等,所以批语认为“叙园圃事极板重,却极活泼”。
}劳形案牍,\zhu{刘禹锡《陋室铭》:无丝竹之乱耳,无案牍之劳形。
}不费讴吟。
\zhu{讴吟:歌唱吟咏。
费:耗损。
上两句的意思大概是,时务繁忙不损害吟咏的雅兴。
}高人焉肯以书香混于铜臭也哉!\ping{
本回:李纨笑道:“叫了人家来,不说正事,且你们对讲学问。
”宝钗道:“学问中便是正事。
此刻于小事上用学问一提,那小事越发作高一层了。
不拿学问提着,便都流入市俗去了。
”宝钗要在日常事务(铜臭)的处理中用理论(书香)拔高眼界。
贾府衰败的深层原因不仅在于缺乏书中“运筹谋划”之人,更重要的还在于带有浓厚贵族意识的贾氏儿孙们对于“数字管理”(铜臭)的本能的蔑视。
例如贾宝玉作为贾府的“高人”,只读书清谈,不肯“混于铜臭”。
}}
\dai{111}{敏探春兴利除宿弊}
\dai{112}{时宝钗小惠全大体}
\sun{p56-1}{探春发火,宝钗赶来}{图左侧:赵姨娘胡闹之后,探春面有怒色,平儿一边垂手默侍。
图中部:宝钗也从上房中来。
}
\sun{p56-2}{甄府上京探亲请安,镜中现影甄贾相会}{图右侧:江南甄家进京,派了四个女人来请安。
闲谈之间得知其府上也有一个公子叫宝玉,忙唤宝玉来见,果然毕肖。
图左侧:宝玉昏睡中恍惚置身一陌生宅院,园内丫鬟不相识,并不理他,又见园内人物情景与怡红院相似,榻上一叫宝玉的声称梦见自己。
两人相见正兴奋之时,忽听老爷叫,心里一急,醒来方知是梦。
}