\chapter{杏子阴假凤泣虚凰 \quad 茜纱窗真情揆痴理}
\zhu{假凤虚凰:凤凰,传说中象征祥瑞的神鸟,雄的叫凤,雌的叫凰,常用来比喻夫妻。
因藕官和菂官在戏中扮演夫妻,在生活中也俨若夫妻,但她们都是女孩子,故称“假凤虚凰”。
揆:音“葵”,推测、揣度。
}
\par
\qi{用清明烧纸徐徐引入园内烧纸,较之前文用燕窝隔回照应,别有草蛇灰线之趣,令人不觉。
前文一接,怪蛇出水;此文一引,春云吐岫。
\zhu{岫:音“秀”,山洞;峰峦。
}}\par
话说他三人因见探春等进来,忙将此话掩住不提。
探春等问候过,大家说笑了一会方散。
\par
谁知上回所表的那位老太妃已薨,\zhu{薨:音“轰”,《礼记·曲礼下》:“天子死曰崩,诸侯曰薨。
”皇妃之死也叫薨。
}凡诰命等皆入朝随班按爵守制。
\zhu{守制:按照封建居丧制度守丧。
封建时代皇帝后妃之丧,臣民都要守制。
守制期间,对官员以至民间的宴乐嫁娶都有种种限制,下文的“凡养优伶男女者一概蠲免遣发”就是其中的一项。
}
敕谕天下:凡有爵之家,一年内不得筵宴音乐,庶民皆三月不得婚嫁。
贾母、邢、王、尤、许婆媳祖孙等皆每日入朝随祭,\zhu{许:可能是贾蓉新娶的媳妇。
}至未正以后方回。
\zhu{未正:下午两点钟。}
在大内偏宫二十一日后,方请灵入先陵,地名曰孝慈县。
\ji{随事命名。
}
这陵离都来往得十来日之功,如今请灵至此,还要停放数日,方入地宫,\zhu{地宫:皇帝陵寝中的地下宫殿。
}
故得一月光景。
\ji{周到细腻之至。
}\ji{真细之至,不独写侯府得理,亦且将皇宫赫赫,写得令人不敢坐阅。
}宁府贾珍夫妻二人,也少不得是要去的。
两府无人,因此大家计议,家中无主,便报了尤氏产育,将他腾挪出来,协理荣宁两处事体。
因又托了薛姨妈在园内照管他姊妹丫鬟。
薛姨妈只得也挪进园来。
因宝钗处有湘云香菱;李纨处目今李婶母女虽去,然有时亦来住三五日不定,贾母又将宝琴送与他去照管;
\ping{
宝琴在贾府的地位下降了。可以揣测贾母之前格外照顾宝琴的目的是主动施恩惠于薛姨妈,促使薛姨妈改变在宝玉婚姻上支持王夫人(宝钗)的态度,转而支持贾母(黛玉)。
前一回薛姨妈当中表示要给宝黛结合做媒人,贾母的目的达到了,所以宝琴就失去了特殊的关照。
}
迎春处有岫烟;探春因家务冗杂,且不时有赵姨娘与贾环来嘈聒,
\zhu{聒[guō]:(声音)嘈杂扰人。}
甚不方便;惜春处房屋狭小;况贾母又千叮咛万嘱咐托他照管林黛玉,薛姨妈素习也最怜爱他的,今既巧遇这事,便挪至潇湘馆来和黛玉同房,一应药饵饮食十分经心。
\zhu{药饵:药物。
}黛玉感戴不尽,以后便亦如宝钗之呼,连宝钗前亦直以姐姐呼之,宝琴前直以妹妹呼之,俨似同胞共出,较诸人更似亲切。
贾母见如此,也十分喜悦放心。
薛姨妈只不过照管他姊妹,禁约得丫头辈,一应家中大小事务也不肯多口。
尤氏虽天天过来,也不过应名点卯,亦不肯乱作威福,且他家内上下也只剩他一个料理,再者每日还要照管贾母王夫人的下处一应所需饮馔铺设之物,所以也甚操劳。
\par
当下荣宁两处主人既如此不暇,并两处执事人等,或有人跟随入朝的,或有朝外照理下处事务的,又有先踩踏下处的,\zhu{踩踏:这里是实地察看的意思。
}也都各各忙乱。
因此两处下人无了正经头绪,也都偷安,或乘隙结党,与权暂执事者窃弄威福。
荣府只留得赖大并几个管事照管外务。
这赖大手下常用几个人已去,虽另委人,都是些生的,只觉不顺手。
且他们无知,或赚骗无节,\zhu{赚:诳骗。
}
或呈告无据,或举荐无因,种种不善,在在生事,\zhu{在在:处处,到处。
}也难备述。
\par
又见各官宦家,凡养优伶男女者,一概蠲免遣发,尤氏等便议定,待王夫人回家回明,也欲遣发十二个女孩子,又说:“这些人原是买的,如今虽不学唱,尽可留着使唤,令其教习们自去也罢了。
”王夫人因说:“这学戏的倒比不得使唤的,他们也是好人家的儿女,因无能卖了做这事,装丑弄鬼的几年。
如今有这机会,不如给他们几两银子盘缠,各自去罢。
当日祖宗手里都是有这例的。
咱们如今损阴坏德,而且还小器。
如今虽有几个老的还在,那是他们各有原故,不肯回去的,所以才留下使唤,大了配了咱们家的小厮们了。
”尤氏道:“如今我们也去问他十二个,有愿意回去的,就带了信儿,叫上父母来,亲自来领回去,给他们几两银子盘缠方妥。
若不叫上他父母亲人来,只怕有混账人顶名冒领出去又转卖了,岂不辜负了这恩典。
若有不愿意回去的,就留下。
”王夫人笑道:“这话妥当。
”\par
尤氏等又遣人告诉了凤姐儿。
\ji{看他任意鄙俚诙谐之中,
\zhu{
鄙俚[bǐlǐ]:粗俗;浅陋。
}
必有一个“礼”字还清,足见是大家形景。
}一面说与总理房中,每教习给银八两,令其自便。
凡梨香院一应物件,查清注册收明,派人上夜。
将十二个女孩子叫来面问,倒有一多半不愿意回家的:也有说父母虽有,他只以卖我们为事,这一去还被他卖了;也有父母已亡,或被叔伯兄弟所卖的;也有说无人可投的;也有说恋恩不舍的。
所愿去者止四五人。
王夫人听了,只得留下。
将去者四五人皆令其干娘领回家去,单等他亲父母来领;将不愿去者分散在园中使唤。
贾母便留下文官自使,将正旦芳官指与宝玉,将小旦蕊官送了宝钗,将小生藕官指与了黛玉,将大花面葵官送了湘云,将小花面荳官送了宝琴,\zhu{荳:同“豆”。
}将老外艾官送了探春,尤氏便讨了老旦茄官去。
\zhu{行当的介绍,参见后文单独整段注释。}
\ping{十二官之中,龄官是被浓墨重彩刻画的,第十八回元妃省亲,龄官受赏,贾蔷要求龄官做《游园》、《惊梦》二出,但是龄官自己执意不作,定要作《相约》、《相骂》二出。
第三十六回,宝玉看到贾蔷龄官情意缠绵,心有所感。
但是之后龄官再也没有出现。
第三十六回,龄官说:“今儿我咳嗽出两口血来,太太叫大夫来瞧,不说替我细问问,你且弄这个来取笑。
偏生我这没人管没人理的,又偏病。
”由此可以推测龄官不是被父母领走重获自由,而是因病去世。
从后文可知,藕官私下烧纸祭祀死去的菂官,可知确有死亡发生,可能龄官也是其中之一。
另外,三十六回之后,贾蔷也随之消失了,龄官在第三十六回说:“你们家把好好的人弄了来,关在这牢坑里学这个劳什子……”可见龄官早有离开的打算和想法,可能在病治好之后,贾蔷和龄官有情人终成眷属,离开大观园了。
}当下各得其所,就如倦鸟出笼,每日园中游戏。
众人皆知他们不能针黹,不惯使用,皆不大责备。
其中或有一二个知事的,愁将来无应时之技,亦将本技丢开,便学起针黹纺绩女工诸务。
\par
\zhu{
    汉剧十大行综述:行当是一种基本功与表演模式的分类集成,至嘉庆末,据《汉口竹枝词》所载,汉剧形成了规范的十大行当:一末、二净、三生、四旦、五丑、六外、七小、八贴、九夫、十杂,沿袭至今。
“一末”的“苍”,“二净”的“刚”,“三生”的方正,“四旦”的“庄”,“五丑”的诙谐、机趣,“六外”的潇洒与机敏,“七小”的温文俊雅,“八贴”的玲珑娇俏,“九夫”的贤良沉稳,“十杂”的勇猛暴烈,分工明细,各具风采。
《扬州画舫录》云:“梨园以副末开场,为领班,副末以下,老生、正生、老外、大面、二面,七人为男角色;老旦、正旦、小旦、八贴,四人为女角色;打诨一人为杂。
此江湖十二角色,元院本旧制也。
”\hang
一末:一末为老年生角,多扮演老年的帝王、宰相、学士、高官、贤士、义仆等正面人物。
其所饰人物多半为年高德重、忠贞刚直的,衣着装扮则随剧情而变。
但其须发为白或灰白者为多。
\hang
二净:净角即花面,以粉墨涂面而演出者,多扮演直臣名将,兼扮奸雄暴君。
\hang
三生:明徐渭曰:“生,即男子之通称也。
”生为中年生角,多扮演慷慨激昂、肃穆忠贞的正面人物。
所扮人物以文为主,即便是武戏,仍以文唱为主。
\hang
四旦:多扮演大家闺秀、中年妇女、皇后王妃、贞女节妇。
\hang
五丑:俗谓小花脸,敷粉墨于面而打诨者,可扮演各种不同类型的角色,如年老的贫婆、幼稚的娃娃、昏聩糊涂的帝王、急公好义的狱卒、奸邪的小吏、耿直的老翁等。
\hang
六外:亦称外角,外为外末、外旦之省称。
言之于正色之外,又加某色充之,或扮男,或扮女。
元曲皆称外。
后专谓扮演男子者为外,扮演女子者为贴。
明徐渭曰:“生之外又一生也。
”《椒生随笔》谓:“外角者,乃局外之人演其事也。
”汉剧以男角挂须者为外。
\hang
七小:即小生,以年轻不挂须的为七小,云集山人谓:“小者,小于老生,正生也。
”如闺秀旦称小旦,无须丑称小丑之类。
七小也包括文武小生。
\hang
八贴:即花旦,元曲称贴旦,系别于正旦而言。
明代传奇始省称曰贴,谓于正色之外,又加他色以充之。
明徐渭曰:“旦之外贴一旦也。
”多扮演年轻的少女、风骚泼辣的少妇。
\hang
九夫:即老旦,为夫人之省称,多扮演老年妇女。
\hang
十杂:即大花脸,与二净分工,二净以庄严胜,十杂以雄浑胜。
云集山人云:杂者,“言其脸乱杂无章也”。
多扮演勇猛憨直的武将和飞扬跋扈的权臣。
}\par
一日正是朝中大祭,贾母等五更便去了,先到下处用些点心小食,然后入朝。
早膳已毕,方退至下处,用过早饭,略歇片刻,复入朝待中晚二祭完毕,方出至下处歇息,用过晚饭方回家。
可巧这下处乃是一个大官的家庙,乃比丘尼焚修,\zhu{比丘尼:梵语,指已经出家受大戒的女子,俗称“尼姑”。
焚修:焚香修行,泛指净俢。
}房舍极多极净。
东西二院,荣府便赁了东院,北静王府便赁了西院。
太妃少妃每日宴息,见贾母等在东院,彼此同出同入,都有照应。
外面细事不消细述。
\par
且说大观园中因贾母王夫人天天不在家内,又送灵去一月方回,各丫鬟婆子皆有闲空,多在园内游玩。
更又将梨香院内伏侍的众婆子一概撤回,并散在园内听使,更觉园内人多了几十个。
因文官等一干人或心性高傲,或倚势凌下,或拣衣挑食,或口角锋芒,大概不安分守理者多。
因此众婆子无不含怨,只是口中不敢与他们分证。
如今散了学,\zhu{散了学:不再学戏唱戏。
}大家称了愿,也有丢开手的,也有心地狭窄犹怀旧怨的,因将众人皆分在各房名下,不敢来厮侵。
\zhu{厮:互相。
}\par
可巧这日乃是清明之日,贾琏已备下年例祭祀,带领贾环、贾琮、贾兰三人去往铁槛寺祭柩烧纸。
宁府贾蓉也同族中几人各办祭祀前往。
因宝玉未大愈,故不曾去得。
\ping{管事的大人走了,甚至连小孩子都走了,园内真的成为宝玉和姊妹的天下了。
}饭后发倦,袭人因说:“天气甚好,你且出去逛逛,省得丢下粥碗就睡,存在心里。
”宝玉听说,只得拄了一支杖,靸着鞋,
\zhu{靸:音“洒”,穿鞋时把鞋后帮踩在脚后跟下,拖着走。}
步出院外。
\ji{画出病势。
}因近日将园中分与众婆子料理,各司各业,皆在忙时,也有修竹的,也有\wu 树的,\zhu{
\wu(音“乌”):除田草的刀,在这里作动词用。
\wu 树:一种园林工艺,将树木的旧枝大部砍去使其另发新枝。
}也有栽花的,也有种豆的,池中又有驾娘们行着船夹泥种藕。
\zhu{夹泥:即捞取河底的烂泥作肥料。
}香菱、湘云、宝琴与丫鬟等都坐在山石上,瞧他们取乐。
宝玉也慢慢行来。
湘云见了他来,忙笑说:“快把这船打出去,他们是接林妹妹的。
”众人都笑起来。
宝玉红了脸,也笑道:“人家的病,谁是好意的,你也形容着取笑儿。
”湘云笑道:“病也比人家另一样,原招笑儿,反说起人来。
”说着,宝玉便也坐下,看着众人忙乱了一回。
湘云因说:“这里有风,石头上又冷,坐坐去罢。
”\par
宝玉便也正要去瞧林黛玉,便起身拄拐辞了他们,从沁芳桥一带堤上走来。
只见柳垂金线,桃吐丹霞,山石之后,一株大杏树,花已全落,叶稠阴翠,上面已结了豆子大小的许多小杏。
宝玉因想道:“能病了几天,竟把杏花辜负了!不觉倒‘绿叶成荫子满枝’了!”\zhu{绿叶成荫子满枝:喻少女已嫁并生儿育女。
}因此仰望杏子不舍。
又想起邢岫烟已择了夫婿一事,虽说是男女大事,不可不行,但未免又少了一个好女儿。
不过两年,便也要“绿叶成荫子满枝”了。
再过几日,这杏树子落枝空,再几年,岫烟未免乌发如银,红颜似槁了,\zhu{槁:音“搞”,草木枯干,引申为羸瘦憔悴。
}
因此不免伤心,只管对杏流泪叹息。
\ji{近之淫书满纸伤春,究竟不知伤春原委。
看他并不提“伤春”字样,却艳恨秾愁,香流满纸矣。
}正悲叹时,忽有一个雀儿飞来,落于枝上乱啼。
宝玉又发了呆性,心下想道:“这雀儿必定是杏花正开时他曾来过,今见无花空有子叶,故也乱啼。
这声韵必是啼哭之声,可恨公冶长不在眼前,\zhu{公冶长:孔子弟子,传说能通鸟语。
}不能问他。
但不知明年再发时,这个雀儿可还记得飞到这里来与杏花一会了?”\par
正胡思间,忽见一股火光从山石那边发出,将雀儿惊飞。
宝玉吃一大惊,又听那边有人喊道:“藕官,你要死,怎弄些纸钱进来烧?我回去回奶奶们去,仔细你的肉!”宝玉听了,益发疑惑起来,忙转过山石看时,只见藕官满面泪痕,蹲在那里,手里还拿着火,守着些纸钱灰作悲。
宝玉忙问道:“你与谁烧纸钱?快不要在这里烧。
你或是为父母兄弟,你告诉我姓名,外头去叫小厮们打了包袱写上名姓去烧。
”\zhu{包袱:祭扫时的焚化品。
清代富察敦崇《燕京岁时记》:“十月初一日,乃都人祭扫之候,俗谓之送寒衣……今则以包袱代之,有寒衣之名,无寒衣之实矣。
包袱者,以冥镪封于纸函中,题其姓名行辈,如前所云。
”烧包袱:焚化用锡箔折叠的元宝、锞子、纸钱等。
}藕官见了宝玉,只不作一声。
宝玉数问不答,忽见一婆子恶恨恨走来拉藕官,口内说道:“我已经回了奶奶们了,奶奶气的了不得。
”藕官听了,终是孩气,怕辱没了没脸,便不肯去。
婆子道:“我说你们别太兴头过馀了,如今还比你们在外头随心乱闹呢。
这是尺寸地方儿。
”\zhu{尺寸地方儿:讲分寸规矩的地方。
这里指贵族府第的内宅。
}指宝玉道:“连我们的爷还守规矩呢,你是什么阿物儿,跑来胡闹。
怕也不中用,跟我快走罢!”\ji{如何?必是含怨之人。
又拉上宝玉,画出小人得意来。
}宝玉忙道:“他并没烧纸钱,原是林妹妹叫他来烧那烂字纸的。
你没看真,反错告了他。
”\ping{宝玉这看谁漂亮喜欢谁就包庇谁的劲儿啊,让他管事准完,毫无规矩可言。
}\par
藕官正没了主意,见了宝玉,也正添了畏惧,忽听他反掩饰,心内转忧成喜,也便硬着口说道:“你很看真是纸钱了么?我烧的是林姑娘写坏了的字纸!”那婆子听如此,亦发狠起来,便弯腰向纸灰中拣那不曾化尽的遗纸,拣了两点在手内,说道:“你还嘴硬,有据有证在这里。
我只和你厅上讲去!”说着,拉了袖子,就拽着要走。
宝玉忙把藕官拉住,用拄杖敲开那婆子的手,说道:“你只管拿了那个回去。
实告诉你:我昨夜作了一个梦,梦见杏花神和我要一挂白纸钱,不可叫本房人烧,要一个生人替我烧了,我的病就好的快。
所以我请了白钱,巴巴儿的和林姑娘烦了他来,替我烧了祝赞。
\zhu{祝赞:祷告于神,祈求福佑。
}原不许一个人知道的,所以我今日才能起来,偏你看见了。
我这会子又不好了,都是你冲了!你还要告他去。
藕官,只管去,见了他们你就照依我这话说。
等老太太回来,我就说他故意来冲神祗,保佑我早死。
”藕官听了益发得了主意,反倒拉着婆子要走。
那婆子听了这话,忙丢下纸钱,陪笑央告宝玉道:“我原不知道,二爷若回了老太太,我这老婆子岂不完了?我如今回奶奶们去,就说是爷祭神,我看错了。
”宝玉道:“你也不许再回去了,我便不说。
”婆子道:“我已经回了,叫我来带他,我怎好不回去的。
也罢,就说我已经叫到了他,林姑娘叫了去了。
”宝玉想了一想,方点头应允。
那婆子只得去了。
\ping{院内不允许烧纸的规定能管下人,管不了主子,法治小于人治。
林黛玉莫名其妙背锅,成了包庇下人犯法的主子。
}\par
这里宝玉问他:“到底是为谁烧纸?我想来若是为父母兄弟,你们皆烦人外头烧过了,这里烧这几张,必有私自的情理。
”藕官因方才护庇之情感激于衷,便知他是自已一流的人物,便含泪说道:“我这事,除了你屋里的芳官并宝姑娘的蕊官,并没第三个人知道。
今日被你遇见,又有这段意思,少不得也告诉了你,只不许再对人言讲。
”又哭道:“我也不便和你面说,你只回去背人悄问芳官就知道了。
”说毕,佯常而去。
\zhu{佯常:亦作“佯长”、“扬长”,大模大样地离开的样子。
}\par
宝玉听了,心下纳闷,\ji{连观书者亦纳闷。
}只得踱到潇湘馆,瞧黛玉益发瘦的可怜,问起来,比往日已算大愈了。
\ji{好,若只管病亦不好。
}
黛玉见他也比先大瘦了,想起往日之事,不免流下泪来,些微谈了谈,便催宝玉去歇息调养。
宝玉只得回来。
因记挂着要问芳官那原委,偏有湘云香菱来了,正和袭人芳官说笑,不好叫他,恐人又盘诘,只得耐着。
\par
一时芳官又跟了他干娘去洗头。
他干娘偏又先叫了他亲女儿洗过了后,才叫芳官洗。
芳官见了这般,便说他偏心,“把你女儿剩水给我洗。
我一个月的月钱都是你拿着,沾我的光不算,反倒给我剩东剩西的。
”他干娘羞愧变成恼,便骂他:“不识抬举的东西!怪不得人人说戏子没一个好缠的。
凭你甚么好人,入了这一行,都弄坏了。
这一点子屄崽子,也挑幺挑六,咸屄淡话,咬群的骡子似的!”娘儿两个吵起来。
袭人忙打发人去说:“少乱嚷,瞅着老太太不在家,一个个连句安静话也不说。
”晴雯因说:“都是芳官不省事,不知狂的什么。
也不是会两出戏,倒像杀了贼王、擒了反叛来的。
”\ping{若是这群小戏子都狂,那可能是因为他们技艺精湛,把从前演过的角色的心理留在自己身上了呢。
}袭人道:“一个巴掌拍不响,老的也太不公些,小的也太可恶些。
”宝玉道:“怨不得芳官。
自古说:‘物不平则鸣。
’\zhu{物不平则鸣:喻人遇到不公平的境遇就要发泄申诉。
出自韩愈《送孟东野序》。
}\ji{自来经语未遭如是用也。
\zhu{自来:从来,历来。
经语:应该是指这句话引经据典。
}}他少亲失眷的,在这里没人照看,赚了他的钱,又作践他,如何怪得?”因又向袭人道:“他一月多少钱?以后不如你收了过来照管他,岂不省事?”袭人道:“我要照看他那里不照看了,又要他那几个钱才照看他?没的讨人骂去了。
”说着,便起身至那屋里取了一瓶花露油并些鸡卵、香皂、头绳之类,
\zhu{
鸡卵:《本草纲目》:“鸡子白、猪胆,沐头解,少顷洗去,光泽不燥”。
}
叫一个婆子来送给芳官去,叫他另要水自洗,不要吵闹了。
他干娘益发羞愧,便说芳官“没良心,花掰我克扣你的钱。
”\zhu{花掰:胡编瞎说的意思。
}便向他身上拍了几把,芳官便哭起来。
宝玉便走出,袭人忙劝:“作什么?我去说他。
”晴雯忙先过来,指他干娘说道:“你老人家太不省事。
你不给他洗头的东西,我们饶给他东西,\zhu{饶:额外增添。
}你不自臊,还有脸打他。
他要还在学里学艺,你也敢打他不成!”那婆子便说:“一日叫娘,终身是母。
他排场我,\zhu{排场:在这里义同“排揎”(揎音“宣”),数落、责难的意思。
}我就打得!”\par
袭人唤麝月道:“我不会和人拌嘴,晴雯性太急,你快过去震吓他两句。
”麝月听了,忙过来说道:“你且别嚷。
我且问你,别说我们这一处,你看满园子里,谁在主子屋里教导过女儿的?便是你的亲女儿,既分了房,有了主子,自有主子打得骂得,再者大些的姑娘姐姐们打得骂得,谁许老子娘又半中间管闲事了?都这样管,又要叫他们跟着我们学什么?越老越没了规矩!你见前儿坠儿的娘来吵,你也来跟他学?你们放心,因连日这个病那个病,老太太又不得闲心,所以我没回。
等两日消闲了,咱们痛回一回,大家把威风煞一煞儿才好。
宝玉才好了些,连我们不敢大声说话,你反打的人狼号鬼叫的。
上头能出了几日门,你们就无法无天的,眼睛里没了我们,再两天你们就该打我们了。
他不要你这干娘,怕粪草埋了他不成?”宝玉恨的用拄杖敲着门槛子说道:“这些老婆子都是些铁心石头肠子,也是件大奇的事。
不能照看,反倒折挫,天长地久,如何是好!”\ji{画出宝玉来。
}晴雯道:“什么‘如何是好’,都撵了出去,不要这些中看不中吃的!”那婆子羞愧难当,一言不发。
那芳官只穿着海棠红的小棉袄,底下丝绸撒花袷裤,\zhu{袷:同“夹”。
袷裤:即“夹裤”,有面有里,中间不衬垫絮类的裤子。
}敞着裤腿,\ji{四字奇想,写得纸上跳出一个女优来。
\zhu{
女优:旧称女演员。
}}一头乌油似的头发披在脑后,哭的泪人一般。
麝月笑道:“把一个莺莺小姐,反弄成拷打红娘了!
\zhu{红娘是《西厢记》里崔莺莺的丫头,为莺莺和张生穿针引线,被老夫人发现后拷打了一顿。}
这会子又不妆扮了,还是这么松怠怠的。
”宝玉道:“他这本来面目极好,倒别弄紧衬了。
”\zhu{紧衬:密合紧贴。
}晴雯过去拉了他,替他洗净了发,用手巾拧干,松松的挽了一个慵妆髻,\zhu{慵妆髻:一种蓬松而偏垂一边的发髻。
}命他穿了衣服过这边来了。
\par
接着司内厨的婆子来问:“晚饭有了,可送不送?”小丫头听了,进来问袭人。
袭人笑道:“方才胡吵了一阵,也没留心听钟几下了。
”晴雯道:“那劳什子又不知怎么了,又得去收拾。
”说着,便拿过表来瞧了一瞧说:“略等半钟茶的工夫就是了。
”小丫头去了。
麝月笑道:“提起淘气,芳官也该打几下。
昨儿是他摆弄了那坠子半日,\zhu{坠子:摆钟下面的摆锤。
}就坏了。
”说话之间,便将食具打点现成。
一时小丫头子捧了盒子进来站住。
晴雯麝月揭开看时,还是只四样小菜。
晴雯笑道:“已经好了,还不给两样清淡菜吃。
这稀饭咸菜闹到多早晚?”一面摆好,一面又看那盒中,却有一碗火腿鲜笋汤,忙端了放在宝玉跟前。
宝玉便就桌上喝了一口,\ji{画出病人。
}说:“好烫!”袭人笑道:“菩萨,能几日不见荤,馋的这样起来。
”一面说,一面忙端起轻轻用口吹。
\ji{画。
}因见芳官在侧,便递与芳官,笑道:“你也学着些伏侍,别一味呆憨呆睡。
口劲轻着,别吹上唾沫星儿。
”芳官依言果吹了几口,甚妥。
\par
他干娘也忙端饭在门外伺候。
向日芳官等一到时原从外边认的,\zhu{向日:往日;从前。
}就同往梨香院去了。
这干婆子原系荣府三等人物,不过令其与他们浆洗,皆不曾入内答应,故此不知内帏规矩。
今亦托赖他们方入园中,随女归房。
这婆子先领过麝月的排场,方知了一二分,生恐不令芳官认他做干娘,便有许多失利之处,故心中只要买转他们。
\zhu{买转:犹买通。
}今见芳官吹汤,便忙跑进来笑道:“他不老成,仔细打了碗,让我吹罢。
”一面说,一面就接。
晴雯忙喊:“出去!你让他砸了碗,也轮不到你吹。
你什么空儿跑到这里槅子来了?还不出去。
”一面又骂小丫头们:“瞎了心的,他不知道,你们也不说给他!”小丫头们都说:“我们撵他,他不出去;说他,他又不信。
如今带累我们受气,你可信了?我们到的地方儿,有你到的一半,还有你一半到不去的呢。
何况又跑到我们到不去的地方还不算,又去伸手动嘴的了。
”一面说,一面推他出去。
阶下几个等空盒家伙的婆子见他出来,都笑道:“嫂子也没用镜子照一照,就进去了。
”羞的那婆子又恨又气,只得忍耐下去。
\par
芳官吹了几口,宝玉笑道:“好了,仔细伤了气。
你尝一口,可好了?”芳官只当是顽话,只是笑看着袭人等。
袭人道:“你就尝一口何妨。
”晴雯笑道:“你瞧我尝。
”说着就喝了一口。
芳官见如此,自己也便尝了一口,说:“好了。
”递与宝玉。
宝玉喝了半碗,吃了几片笋,又吃了半碗粥就罢了。
众人拣收出去了。
小丫头捧了沐盆,盥漱已毕,袭人等出去吃饭。
宝玉使个眼色与芳官,芳官本自伶俐,又学几年戏,何事不知?便装说头疼不吃饭了。
袭人道:“既不吃饭,你就在屋里作伴儿,把这粥给你留着,一时饿了再吃。
”说着,都去了。
\par
这里宝玉和他只二人,宝玉便将方才从火光发起,如何见了藕官,又如何谎言护庇,又如何藕官叫我问你,从头至尾,细细的告诉他一遍,又问他祭的果系何人。
芳官听了,满面含笑,又叹一口气,说道:“这事说来可笑又可叹。
”宝玉听了,忙问如何。
芳官笑道:“你说他祭的是谁?祭的是死了的菂官。
”\zhu{菂:音“第”,莲子。
}宝玉道:“这是友谊,也应当的。
”芳官笑道:“那里是友谊?他竟是疯傻的想头,说他自己是小生,菂官是小旦,常做夫妻,虽说是假的,每日那些曲文排场,皆是真正温存体贴之事,故此二人就疯了,虽不做戏,寻常饮食起坐,两个人竟是你恩我爱。
\ping{这就是戏痴,把自己完全带入戏曲中,分不清现实和虚幻。
这一段可能是霸王别姬的灵感来源吧,只不过这里是两个唱戏的女孩子之间的迷失性别的爱,而霸王别姬里面是两个男孩子之间的故事。
}
菂官一死,他哭的死去活来,至今不忘,所以每节烧纸。
后来补了蕊官,我们见他一般的温柔体贴,也曾问他得新弃旧的。
他说:‘这又有个大道理。
比如男子丧了妻,或有必当续弦者,也必要续弦为是。
便只是不把死的丢过不提,便是情深意重了。
若一味因死的不续,孤守一世,妨了大节,也不是理,死者反不安了。
’你说可是又疯又呆?说来可是可笑?”宝玉听说了这篇呆话,独合了他的呆性,\ping{作者明写藕官对同性之爱的态度,暗写宝玉的婚恋观。
“比如男子丧了妻,或有必当续弦者,也必要续弦为是。
便只是不把死的丢过不提,便是情深意重了”埋下伏笔,黛玉死后宝玉还会和宝钗结婚,但是心里永远忘不了黛玉。
}不觉又是欢喜,又是悲叹,又称奇道绝,说:“天既生这样人,又何用我这须眉浊物玷辱世界。
”因又忙拉芳官嘱道:“既如此说,我也有一句话嘱咐他,我若亲对面与他讲未免不便,须得你告诉他。
”芳官问何事。
宝玉道:“以后断不可烧纸钱。
这纸钱原是后人异端,不是孔子的遗训。
\ping{宝玉虽然离经叛道,但是受传统儒家教育浸润多年,依然十分尊敬孔子。
第二十回:(宝玉)因有这个呆念在心,把一切男子都看成混沌浊物,可有可无。
只是父亲叔伯兄弟中,因孔子是亘古第一人说下的,不可忤慢,只得要听他这句话。
第五十一回:宝玉笑道:“松柏不敢比。
连孔子都说:‘岁寒然后知松柏之后凋也。
’可知这两件东西高雅,不怕羞臊的才拿他混比呢。
”}以后逢时按节,只备一个炉,到日随便焚香,一心诚虔,就可感格了。
\zhu{感格:感动、感应的意思。
格:感通。
}
\ping{第四十三回,宝玉去水仙庵含泪祭奠金钏。}
愚人原不知,无论神佛死人,必要分出等例,各式各例的。
殊不知只一‘诚心’二字为主。
即值仓皇流离之日,虽连香亦无,随便有土有草,只以洁净,便可为祭,不独死者享祭,便是神鬼也来享的。
\ping{贾府败落,宝玉流离,仓皇之中也不忘虔心祭奠。
}你瞧瞧我那案上,只设一炉,不论日期,时常焚香。
他们皆不知原故,我心里却各有所因。
\ping{宝玉应该是还没有忘记因自己而死的金钏吧。
}随便有清茶便供一钟茶,有新水就供一盏水,或有鲜花,或有鲜果,甚至荤羹腥菜,只要心诚意洁,便是佛也都可来享,所以说,只在敬不在虚名。
以后快命他不可再烧纸。
”芳官听了,便答应着。
一时吃过饭,便有人回:“老太太、太太回来了——”\par
\qi{总评:道理彻上彻下,提笔左潆右拂,\zhu{潆:音“营”,水流环绕回旋的样子。
}浩浩千万言不绝。
又恐后人溺词失旨,特自注一句以结穴,
\zhu{结穴:喻指事物要害所在。这里指全文最后的若干句。
这里应该是指文末宝玉阐述自己婚姻观的几句话。
}
曰诚曰信。
\hang
杏子林对禽惜花一席话,
\zhu{杏子林对禽惜花:指宝玉在杏树下看到花已全落,听到雀啼枝头,感慨岫烟定亲后红颜必然老去。}
仿佛茂叔庭草不除襟怀。
\zhu{茂叔庭草不除:周敦颐,字茂叔。周敦颐的“观天地生物气象”强调“庭前草不除”以“观天地生生不已”。}}
\dai{115}{杏子阴假凤泣虚凰}
\dai{116}{干娘打骂芳官,麝月训诫,宝玉恨恼}
\sun{p58-1}{众姊妹坐山石玩乐,杏子阴假凤泣虚凰}{图右侧:宝玉拄杖步出院外。
众婆子有栽花的、种豆的,池中又有驾娘们行着船夹泥种藕。
香菱、湘云、宝琴与丫鬟等都坐在山石上,瞧他们取乐。
图左侧:忽见一股火光,原是藕官在烧纸祭奠,一婆子欲拿其问罪,宝玉急忙编排理由遮掩庇护。
}