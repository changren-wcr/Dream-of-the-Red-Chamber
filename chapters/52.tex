\chapter{俏平儿情掩虾须镯 \quad 勇晴雯病补雀金裘}
\qi{写黛玉弱症的是弱症,写晴雯时症的是时症;写湘云性快的是快性,写晴雯性傲的是傲性。
彼何人斯?而具肖物手段如此。
}\par
贾母道:“正是这话了。
上次我要说这话,我见你们的大事多,如今又添出这些事来,你们固然不敢抱怨,未免想着我只顾疼这些小孙子孙女儿们,就不体贴你们这当家人了。
你既这么说出来,更好了。
”因此时薛姨妈李婶都在座,邢夫人及尤氏婆媳也都过来请安,还未过去,贾母向王夫人等说道:“今儿我才说这话,素日我不说,一则怕逞了凤丫头的脸,\zhu{逞……脸:因受宠而骄纵。
}二则众人不伏。
今日你们都在这里,都是经过妯娌姑嫂的,还有他这样想的到的没有?”薛姨妈、李婶、尤氏等齐笑说:“真个少有。
别人不过是礼上面子情儿,实在他是真疼小叔子小姑子。
就是老太太跟前,也是真孝顺。
”贾母点头叹道:“我虽疼他,我又怕他太伶俐也不是好事。
”凤姐儿忙笑道:“这话老祖宗说差了。
世人都说太伶俐聪明,怕活不长。
世人都说得,人人都信,独老祖宗不当说,不当信。
老祖宗只有伶俐聪明过我十倍的,怎么如今这样福寿双全的?只怕我明儿还胜老祖宗一倍呢!我活一千岁后,等老祖宗归了西,我才死呢。
”\ping{按照“越伶俐活得越短”的俗语,贾母比凤姐伶俐聪明,所以凤姐应该比贾母活得时间还要长。
而贾母已经福寿双全,所以说凤姐可以活“一千岁”。
这其实是用了归谬法(反证法),先假设俗语是对的,推理出不合理的结果,由此证明俗语的荒诞不经。
}
贾母笑道:“众人都死了,单剩下咱们两个老妖精,有什么意思。
”说的众人都笑了。
\par
宝玉因记挂着晴雯、袭人等事,便先回园里来。
到房中,药香满屋,一人不见,只见晴雯独卧于炕上,脸面烧的飞红,又摸了一摸,只觉烫手。
忙又向炉上将手烘暖,伸进被去摸了一摸身上,也是火烧。
因说道:“别人去了也罢,麝月秋纹也这样无情,各自去了?”晴雯道:“秋纹是我撵了他去吃饭的,麝月是方才平儿来找他出去了。
两人鬼鬼祟祟的,不知说什么。
必是说我病了不出去。
”宝玉道:“平儿不是那样人。
况且他并不知你病特来瞧你,想来一定是找麝月来说话,偶然见你病了,随口说特瞧你的病,这也是人情乖觉取和的常事。
便不出去,有不是,与他何干?你们素日又好,断不肯为这无干的事伤和气。
”晴雯道:“这话也是,只是疑他为什么忽然间瞒起我来。
”\geng{宝玉一篇推情度理之谈以射正事,
\zhu{射:用言语文字暗示。
如“影射”。
}
不知何如。
\zhu{何如:如何;怎么样。}
}宝玉笑道:“让我从后门出去,到那窗根下听听说些什么,来告诉你。
”说着,果然从后门出去,至窗下潜听。
\par
只闻麝月悄问道:“你怎么就得了的?”\geng{妙!这才有神理,是平儿说过一半了。
若此时从\sout{宝玉}[平儿]口中从头说起一原一故,直是二人特等宝玉来听方说起也。
}
平儿道:“那日洗手时不见了,二奶奶就不许吵嚷,出了园子,即刻就传给园里各处的妈妈们小心查访。
我们只疑惑邢姑娘的丫头,本来又穷,只怕小孩子家没见过,拿了起来也是有的。
再不料定是你们这里的。
幸而二奶奶没有在屋里,你们这里的宋妈妈去了,拿着这支镯子,说是小丫头子坠儿偷起来的,被他看见,来回二奶奶的。
\geng{妙极!红玉既有归结,坠儿岂可不表哉?可知“奸贼”二字是相连的。
故“情”字原非正道,坠儿原不情也,不过一愚人耳,可以传奸即可以为盗。
\zhu{第二十六回、第二十七回,坠儿作为中介,替贾芸和小红暗地里传递手帕互传情思,这就是批书人所说的“传奸”。
}二次小窃皆出于宝玉房中,
\zhu{后文马上提到良儿偷玉。}
亦大有深意在焉。
}我赶着忙接了镯子,想了一想:宝玉是偏在你们身上留心用意、争胜要强的,那一年有一个良儿偷玉,刚冷了一二年间,还有人提起来趁愿,
\zhu{趁愿:即称愿。满足愿望,幸灾乐祸。}
这会子又跑出一个偷金子的来了。
而且更偷到街坊家去了。
偏是他这样,偏是他的人打嘴。
所以我倒忙叮咛宋妈,千万别告诉宝玉,只当没有这事,别和一个人提起。
第二件,老太太、太太听了也生气。
三则袭人和你们也不好看。
所以我回二奶奶,只说:‘我往大奶奶那里去的,谁知镯子褪了口,丢在草根底下,雪深了没看见。
今儿雪化尽了,黄澄澄的映着日头,还在那里呢,我就拣了起来。
’二奶奶也就信了,所以我来告诉你们。
你们以后防着他些,别使唤他到别处去。
等袭人回来,你们商议着,变个法子打发出去就完了。
”麝月道:“这小娼妇也见过些东西,怎么这么眼皮子浅。
”平儿道:“究竟这镯子能多少重,原是二奶奶说的,这叫做‘虾须镯’,倒是这颗珠子还罢了。
晴雯那蹄子是块爆炭,\ping{侧面描写,从他人口中得知晴雯的性格。
}要告诉了他,他是忍不住的。
一时气了,或打或骂,依旧嚷出来不好,所以单告诉你留心就是了。
”说着便作辞而去。
\par
宝玉听了,又喜又气又叹。
喜的是平儿竟能体贴自己;气的是坠儿小窃;叹的是坠儿那样一个伶俐人,作出这丑事来。
因而回至房中,把平儿之话一长一短告诉了晴雯。
又说:“他说你是个要强的,如今病着,听了这话越发要添病,等好了再告诉你。
”晴雯听了,果然气的蛾眉倒蹙,凤眼圆睁,即时就叫坠儿。
\ping{坠儿这名字就有下坠堕落之意,也算是一种暗示。
晴雯气也是气她不争气吧,明明是个有眼界的伶俐人,偏要做腌臜事儿,自甘堕落。
}宝玉忙劝道:“你这一喊出来,岂不辜负了平儿待你我之心了。
不如领他这个情,过后打发他就完了。
”晴雯道:“虽如此说,只是这口气如何忍得!”宝玉道:“这有什么气的?你只养病就是了。
”\par
晴雯服了药,至晚间又服二和,\zhu{和[huò]:量词。一服中药煎的次数,一次叫“一和”。    
}夜间虽有些汗,还未见效,仍是发烧,头疼鼻塞声重。
次日,王太医又来诊视,另加减汤剂。
虽然稍减了烧,仍是头疼。
宝玉便命麝月:“取鼻烟来,给他嗅些,痛打几个嚏喷,就通了关窍。
”麝月果真去取了一个金镶双扣金星玻璃的一个扁盒来,\zhu{金星玻璃:我国清代玻璃器的一个品种,始创于乾隆时期,它是利用某些金属物质在玻璃中溶解度很小的特性,在一定的温度下即从玻璃中析出的原理,制出一种含有结晶颗粒而呈现出金属闪光点的玻璃。
清代金星玻璃中的金属颗粒一般是铜,闪射出金光。
金镶双扣金星玻璃:应描述的是该扁盒的开关,可能指以两个镶金之金星玻璃互卡的珠扣式设计(今仍见于许多小提包)。
}
递与宝玉。
宝玉便揭翻盒扇,里面有西洋珐琅的黄发赤身女子,\zhu{珐琅:音“发廊”,用石英、长石、硝石和碳酸钠等加上铅和锡的氧化物烧制成的像釉子的物质。
用它涂在铜质或银质器物上,经过烧制,形成不同颜色的釉质表面,既可防锈,又可作为装饰。
釉子:釉音“又”,以石英、长石、硼砂、黏土等为原料,磨成粉末,加水调制而成的物质,用来涂在陶瓷半成品的表面,烧制后发出玻璃光泽,并能增加陶瓷的机械强度和绝缘性能。
}两肋又有肉翅,里面盛着些真正汪恰洋烟。
\zhu{汪恰洋烟:鼻烟的一种。
汪恰语源未详。一说可能是 virginia 或是 virgin 的音译;或说可能是法文 vierge 的译音。
}\geng{汪恰,西洋一等宝烟也。
}晴雯只顾看画儿,宝玉道:“嗅些,走了气就不好了。
”晴雯听说,忙用指甲挑了些嗅入鼻中,不怎样。
便又多多挑了些嗅入。
忽觉鼻中一股酸辣透入囟门,\zhu{囟门:初生婴儿的头顶前部。
因颅骨尚未成熟愈合,故可看到脑部血管的跳动。
}接连打了五六个嚏喷,眼泪鼻涕登时齐流。
\geng{写得出。
}晴雯忙收了盒子,笑道:“了不得,好爽快!拿纸来。
”早有小丫头子递过一搭子细纸,晴雯便一张一张的拿来醒鼻子。
宝玉笑问:“如何?”晴雯笑道:“果觉通快些,只是太阳还疼。
”\zhu{太阳:太阳穴。
}宝玉笑道:“越性尽用西洋药治一治,只怕就好了。
”说着,便命麝月:“和二奶奶要去,就说我说了:姐姐那里常有那西洋贴头疼的膏子药,叫做‘依弗哪’,找寻一点儿。
”麝月答应了,去了半日,果拿了半节来。
便去找了一块红缎子角儿,铰了两块指顶大的圆式,将那药烤和了,用簪挺摊上。
\zhu{挺:通“梃”,棍棒。
}晴雯自拿着一面靶镜,\zhu{靶:柄。
靶镜:带柄的镜子。
}贴在两太阳上。
麝月笑道:“病的蓬头鬼一样,如今贴了这个,倒俏皮了。
二奶奶贴惯了,倒不大显。
”说毕,又向宝玉道:“二奶奶说了:明日是舅老爷生日,太太说了叫你去呢。
明儿穿什么衣裳?今儿晚上好打点齐备了,省得明儿早起费手。
”宝玉道:“什么顺手就是什么罢了。
一年闹生日也闹不清。
”说着,便起身出房,往惜春房中去看画。
\par
刚到院门外边,忽见宝琴的小丫鬟名小螺者从那边过去,宝玉忙赶上问:“那去?”小螺笑道:“我们二位姑娘都在林姑娘房里呢,我如今也往那里去。
”宝玉听了,转步也便同他往潇湘馆来。
不但宝钗姊妹在此,且连邢岫烟也在那里,四人围坐在熏笼上叙家常。
紫鹃倒坐在暖阁里,
\zhu{倒坐:侧身而坐。}
临窗作针黹。
一见他来,都笑说:“又来了一个!可没了你的坐处了。
”宝玉笑道:“好一副‘冬闺集艳图’!可惜我迟来了一步。
横竖这屋子比各屋子暖,这椅子上坐着并不冷。
”说着,便坐在黛玉常坐的搭着灰鼠椅搭一张椅上。
因见暖阁之中有一玉石条盆,
\zhu{条盆:长方形的花盆。}
里面攒三聚五栽着一盆单瓣水仙,点着宣石,\zhu{宣石:产于安徽宁国县(旧属宣城),石质坚硬,色泽洁白,多用于叠假山。
}便极口赞:“好花!这屋子越发暖,这花香的越清香。
昨日未见。
”黛玉因说道:“这是你家的大总管赖大婶子送薛二姑娘的,两盆腊梅、两盆水仙。
他送了我一盆水仙,他送了蕉丫头一盆腊梅。
我原不要的,又恐辜负了他的心。
你若要,我转送你如何?”宝玉道:“我屋里却有两盆,只是不及这个。
琴妹妹送你的,如何又转送人,这个断使不得。
”黛玉道:“我一日药吊子不离火,我竟是药培着呢,那里还搁的住花香来熏?越发弱了。
况且这屋子里一股药香,反把这花香搅坏了。
不如你抬了去,这花也清净了,没杂味来搅他。
”宝玉笑道:“我屋里今儿也有病人煎药呢,你怎么知道的?”\ping{“你怎么知道的”前后衔接不顺。
从前文黛玉以自己屋里有药香为由要送花给宝玉可以看出,黛玉是不知道宝玉房中晴雯生病煎药的事情的。
宝玉说黛玉知道说不通。“你怎么知道的”可能应该改成“你怎么会不知道”。
宝玉认为和黛玉熟知契合,黛玉肯定知道晴雯因病煎药之事。但黛玉还说送花,倒显得像不知道似的。
}黛玉笑道:“这话奇了,我原是无心的话,谁知你屋里的事?你不早来听说古记,\zhu{古记:值得凭吊纪念的旧时景物事迹叫“古记儿”,在这里义近故事、传说。
}
这会子来了,自惊自怪的。
”
\ping{宝玉惊怪后方知宝玉屋内病人煎药之事。对宝玉认为自己理应知道煎药的事传达的亲密而感到害羞,故冷淡处理说不知你的事。后文黛玉立刻捂脸怕臊。}
\par
宝玉笑道:“咱们明儿下一社又有了题目了,就咏水仙腊梅。
”黛玉听了,笑道:“罢,罢!我再不敢作诗了,作一回,罚一回,没的怪羞的。
”说着,便两手握起脸来。
宝玉笑道:“何苦来!又奚落我作什么。
我还不怕臊呢,你倒握起脸来了。
”宝钗因笑道:“下次我邀一社,四个诗题,四个词题。
每人四首诗,四阕词。
头一个诗题《咏〈太极图〉》,\zhu{《咏〈太极图〉》:《太极图》:北宋周敦颐绘制的对宇宙万物创成变化的图解,混杂着儒家和道家的思想。
他的《太极图说》认为太极是宇宙本体,由太极的一动一静,产生阴阳五行和宇宙万物。
这种客观唯心主义的思想,成为程朱理学的理论基础。
清代乾隆将周敦颐著作列在御纂《性理精义》的卷首,颁布学宫,甚至以“太极图”为题命臣僚赋诗作文。
小说此处的描写反映了当时崇奉理学的风气;由于《太极图说》形式上是推衍《易经》的,所以下文说《咏〈太极图〉》这样的诗题只能弄些《易经》上的话“颠来倒去生填”,窒息才情,束缚思想,不可能做出好诗来。
}
限一先的韵,五言律,要把一先的韵都用尽了,一个不许剩。
”宝琴笑道:“这一说,可知是姐姐不是真心起社了,这分明难人。
若论起来,也强扭的出来,不过颠来倒去弄些《易经》上的话生填,\zhu{《易经》:简称《易》,也叫《周易》,儒家经典之一。
文字简约,语义玄奥。
}究竟有何趣味。
我八岁时节,跟我父亲到西海沿子上买洋货,谁知有个真真国的女孩子,才十五岁,那脸面就和那西洋画上的美人一样,也披着黄头发,打着联垂,\zhu{联垂:辫子。
}满头带的都是珊瑚、猫儿眼、祖母绿这些宝石;
\zhu{猫儿眼:即猫睛石,一种宝石。作为装饰品时,多磨成圆球形,看上去很像猫的眼睛。}
身上穿着金丝织的锁子甲洋锦袄袖;\zhu{
锁子甲:用铁链衔接,互相密扣缀和而成的衣形。穿起来柔和便利,比大型坚甲轻巧。五环相互,一环受镞,诸环拱护,故箭不能入。
金丝织的锁子甲:并非真正的铠甲,而是一种仿锁子甲形制以金丝织成的衣服。
洋锦袄袖:锁子甲的袄袖用洋锦制成。
}带着倭刀,也是镶金嵌宝的,实在画儿上的也没他好看。
有人说他通中国的诗书,会讲五经,能作诗填词,因此我父亲央烦了一位通事官,\zhu{通事官:翻译官。
\ping{这个注释并不合理,因为真真国的女孩子“通诗书,讲五经,填诗词”,中文水平很高,并不需要有一个翻译官居中翻译。
可能是中间牵线搭桥的中介。
}}烦他写了一张字,就写的是他作的诗。
”众人都称奇道异。
宝玉忙笑道:“好妹妹,你拿出来我瞧瞧。
”宝琴笑道:“在南京收着呢,此时那里去取来?”宝玉听了,大失所望,便说:“没福得见这世面。
”黛玉笑拉宝琴道:“你别哄我们。
我知道你这一来,你的这些东西未必放在家里,自然都是要带了来的,这会子又扯谎说没带来。
他们虽信,我是不信的。
”宝琴便红了脸,低头微笑不语。
\ping{宝琴入京和梅翰林儿子成婚,所以会把东西全都带过来。}
宝钗笑道:“偏这个颦儿惯说这些白话,\zhu{白话:点破真相的大实话;没有根据或不能实现的话。
}把你就伶俐的。
”黛玉道:“若带了来,就给我们见识见识也罢了。
”宝钗笑道:“箱子笼子一大堆还没理清,知道在那个里头呢!等过日收拾清了,找出来大家再看就是了。
”又向宝琴道:“你若记得,何不念念我们听听?”\ping{宝琴为何找借口不愿意拿出来真真国女孩子作的诗呢?况且一开始是宝琴自己提及这首诗的,后面却遮遮掩掩,令人费解。
是因为宝琴东西太多而不好找或者懒得找出来呢?还是因为害羞不想拿出私密的闺阁字样来给大家看呢?或者真真国女孩子作诗是宝琴编的故事呢?宝琴虚构故事作诗在之前是有例子的,第五十一回,宝琴所作的十首怀古诗,第九首《蒲东寺怀古》借用了《西厢记》的故事,第十首《梅花观怀古》借用了《牡丹亭》的故事,都不是宝琴自己亲自到访的地方,而是一种虚构的创作手法。
这里宝琴借虚构的真真国女儿之口,抒发自己的情感,是有可能的。
宝钗估计知道宝琴这是编的故事,肯定拿不出来真凭实据,为了帮宝琴解围,就以东西太多没收拾清楚为由敷衍过去。
另外,如果这个故事是编造的,那么就解释了之前在通事官的翻译上存在的疑惑:真真国的女孩子“通诗书,讲五经,填诗词”,中文水平很高,并不需要有一个翻译官居中翻译。
正是因为宝琴想要在姐妹显露自己经历丰富,所以编造了这样一个真真国女孩子的故事。
因为临时编造,没有考虑周全,所以有这样的自相矛盾的地方。
}宝琴方答道:“记得是首五言律,外国的女子也就难为他了。
”宝钗道:“你且别念,等把云儿叫了来,也叫他听听。
”说着,便叫小螺来吩咐道:“你到我那里去,就说我们这里有一个外国美人来了,作的好诗,请你这‘诗疯子’来瞧去,再把我们‘诗呆子’也带来。
”小螺笑着去了。
\par
半日,只听湘云笑问:“那一个外国美人来了?”一头说,一头果和香菱来了。
众人笑道:“人未见形,先已闻声。
”宝琴等忙让坐,遂把方才的话重叙了一遍。
湘云笑道:“快念来听听。
”宝琴因念道:\par
\hop
昨夜朱楼梦,今宵水国吟。
\zhu{朱楼:即红楼,指代贵族之家。宵:夜。水国:环海之地,岛国。}
\par
岛云蒸大海,岚气接丛林。
\zhu{上句意思是海水蒸腾而成岛上的云。
岚:音“兰”,山林中的雾气,亦指岛上景象。
}\par
月本无今古,情缘自浅深。
\zhu{缘:因为。
自:本有。
这两句意谓月亮本无古今之别,因为人的感情有深浅不同,所以多情人便会对月发生感慨。
}\par
汉南春历历,焉得不关心。
\zhu{汉南:本指汉水之南,这里非实指,是用典,说人生易老,俯仰今昔,不堪迟暮之感。
语出北朝庾信《枯树赋》:“昔年移柳,依依汉南;今看摇落,凄怆江潭;树犹如此,人何以堪!”
赋的后两句又用桓温北征途中,见前所植柳树已十围,因慨叹流涕事。
历历:历历在目,清晰可见。这句说,回想起来,昔时情景如在眼前。
关心:关切、动心。
}\par
\hop
众人听了,都道:“难为他!竟比我们中国人还强。
”一语未了,只见麝月走来说:“太太打发人来告诉二爷,明儿一早往舅舅那里去,就说太太身上不大好,不得亲自来。
”宝玉忙站起来答应道:“是。
”因问宝钗宝琴可去。
宝钗道:“我们不去。
昨儿单送了礼去了。
”大家说了一回方散。
\par
宝玉因让诸姊妹先行,自己落后。
黛玉便又叫住他问道:“袭人到底多早晚回来。
”宝玉道:“自然等送了殡才来呢。
”黛玉还有话说,又不曾出口,出了一回神,便说道:“你去罢。
”宝玉也觉心里有许多话,只是口里不知要说什么,想了一想,也笑道:“明日再说罢。
”一面下了阶矶,低头正欲迈步,复又忙回身问道:“如今的夜越发长了,你一夜咳嗽几遍?醒几次?”\geng{此皆好笑之极,无味扯淡之极,回思则皆沥血滴髓之至情至神也。
岂别部偷寒送暖,私奔暗约,一味淫情浪态之小说可比哉?}黛玉道:“昨儿夜里好了,只嗽两遍,却只睡了四更一个更次,就再不能睡了。
”宝玉又笑道:“正是有句要紧的话,这会子才想起来。
”一面说,一面便挨过身来,悄悄道:“我想宝姐姐送你的燕窝——”一语未了,只见赵姨娘走了进来瞧黛玉,问:“姑娘这两天好?”黛玉便知他是从探春处来,从门前过,顺路的人情。
黛玉忙陪笑让坐,说:“难得姨娘想着,怪冷的,亲自走来。
”又忙命倒茶,一面又使眼色与宝玉。
宝玉会意,便走了出来。
\ping{
宝黛关于燕窝未完之语,在第五十七回有了回应。
宝玉道:“也没什么要紧。不过我想着宝姐姐也是客中,既吃燕窝,又不可间断,若只管和他要,也太托实。
虽不便和太太要,我已经在老太太跟前略露了个风声,只怕老太太和凤姐姐说了。我告诉他的,竟没告诉完了他。
如今我听见一日给你们一两燕窝,这也就完了。”
}
\par
正值吃晚饭时,见了王夫人,王夫人又嘱咐他早去。
宝玉回来,看晴雯吃了药。
此夕宝玉便不命晴雯挪出暖阁来,自己便在晴雯外边。
又命将熏笼抬至暖阁前,麝月便在熏笼上。
一宿无话。
\par
至次日,天未明时,晴雯便叫醒麝月道:“你也该醒了,只是睡不够!你出去叫人给他预备茶水,我叫醒他就是了。
”麝月忙披衣起来道:“咱们叫起他来,穿好衣裳,抬过这火箱去,
\zhu{火箱:即熏笼。}
再叫他们进来。
老嬷嬷们已经说过,不叫他在这屋里,怕过了病气。
如今他们见咱们挤在一处,又该唠叨了。
”晴雯道:“我也是这么说呢。
”二人才叫时,宝玉已醒了,忙起身披衣。
麝月先叫进小丫头子来,收拾妥当了,才命秋纹檀云等进来,一同伏侍宝玉梳洗毕。
麝月道:“天又阴阴的,只怕有雪,穿那一套毡的罢。
”宝玉点头,即时换了衣裳。
小丫头便用小茶盘捧了一盖碗建莲红枣儿汤来,\zhu{盖碗:一种上有盖、下有托,中有碗的汉族茶具。
又称“三才碗”、“三才杯”,盖为天、托为地、碗为人,暗含天地人和之意。
建莲:莲子中最上品者为“建莲”,即产自福建建宁的莲子。
}宝玉喝了两口。
麝月又捧过一小碟法制紫姜来,\zhu{
法制:按传统方法制作,转为地道的、标准的之意。
法制紫姜:用嫩姜制作的酱菜。
}
宝玉噙了一块。
又嘱咐了晴雯一回,便往贾母处来。
\par
贾母犹未起来,知道宝玉出门,便开了房门,命宝玉进去。
宝玉见贾母身后宝琴面向里也睡未醒。
贾母见宝玉身上穿着荔色哆啰呢的天马箭袖,\zhu{荔色:紫红色。
哆啰呢:一种西洋传入的阔幅呢料。
天马:天马皮,沙狐腹下之皮。
箭袖:原为便于射箭穿的窄袖衣服,这里指男子穿的一种服式。
}大红猩猩毡盘金彩绣石青妆缎沿边的排穗褂子。
\zhu{猩猩毡:猩红色毛毡。
盘金:用金线在织物上盘出花样。
彩绣:彩色丝绣。
石青:淡灰青色。
妆缎:又叫“妆花缎”,织有图案的绸缎。
排穗褂子:下摆垂有一排流苏穗儿的衣服。
}贾母道:“下雪呢么?”宝玉道:“天阴着,还没下呢!”贾母便命鸳鸯来:“把昨儿那一件乌云豹的氅衣给他罢。
”\zhu{乌云豹:即沙狐颔下之皮。
氅[chǎng]衣:类似斗篷的无袖御寒外衣。
}鸳鸯答应了,走去果取了一件来。
宝玉看时,金翠辉煌,碧彩闪灼,又不似宝琴所披之凫靥裘。
只听贾母笑道:“这叫作‘雀金呢’,\zhu{雀金呢:用孔雀毛拈线加金缕织成的厚衣料。
}这是哦啰斯国拿孔雀毛拈了线织的。
\zhu{哦啰斯:俄罗斯。
}前儿把那一件野鸭子的给了你小妹妹,\geng{“小”字更妙!盖王夫人之末女也。
}这件给你罢。
”\ping{宝琴野鸭子,宝玉孔雀,还是挺大差距。
}宝玉磕了一个头,便披在身上。
贾母笑道:“你先给你娘瞧瞧去再去。
”宝玉答应了,便出来,只见鸳鸯站在地下揉眼睛。
因自那日鸳鸯发誓决绝之后,他总不和宝玉讲话。
宝玉正自日夜不安,此时见他又要回避,宝玉便上来笑道:“好姐姐,你瞧瞧,我穿着这个好不好。
”鸳鸯一摔手,便进贾母房中来了。
宝玉只得到了王夫人房中,与王夫人看了,然后又回至园中,与晴雯麝月看过后,至贾母房中回说:“太太看了,只说可惜了的,叫我仔细穿,别遭踏了他。
”贾母道:“就剩下了这一件,你遭踏了也再没了。
这会子特给你做这个也是没有的事。
”说着又嘱咐他:“不许多吃酒,早些回来。
”宝玉应了几个“是”。
\par
老嬷嬷跟至厅上,只见宝玉的奶兄李贵和王荣、张若锦、赵亦华、钱启、周瑞六个人,\zhu{奶兄:奶妈的儿子。
}带着茗烟、伴鹤、锄药、扫红四个小厮,背着衣包,抱着坐褥,笼着一匹雕鞍彩辔的白马,\zhu{辔:音“配”,驾驭牲口的缰绳。
}早已伺候多时了。
老嬷嬷又吩咐了他六人些话,六个人忙答应了几个“是”,忙捧鞭坠镫。
\zhu{坠镫:亦作“坠蹬”,向下拉正马镫,侍候尊长上马。
亦表示对人敬仰,甘执贱役之意。
}宝玉慢慢的上了马,李贵和王荣笼着嚼环,\zhu{嚼环:即嚼子。为了便于驾驭牲口和驯服动物,或防止动物伤人,横放在牲口或动物嘴里的小铁链或其它形状的铁制品,两端连在笼头或缰绳上,多用于马、骡子、牛等大个的牲口,偶尔也能用于犬科动物。
马嚼环\foot{
\footPic{马嚼环}{jiaozi.jpg}{0.5}
}:勒在马口里的小铁链,也称马嚼子,借以控制马匹的活动。
骑手一拉缰绳,马嚼子就被拉进马嘴巴里,骑手就这样来控制马匹的行进速度或者让马停步。
}钱启周瑞二人在前引导,张若锦、赵亦华在两边紧贴宝玉后身。
宝玉在马上笑道:“周哥,钱哥,咱们打这角门走罢,省得到了老爷的书房门口又下来。
”周瑞侧身笑道:“老爷不在家,书房天天锁着的,爷可以不用下来罢了。
”宝玉笑道:“虽锁着,也要下来的。
”钱启李贵等都笑道:“爷说的是。
便托懒不下来,倘或遇见赖大爷林二爷,虽不好说爷,也劝两句。
有的不是,都派在我们身上,又说我们不教爷礼了。
”周瑞钱启便一直出角门来。
\par
正说话时,顶头果见赖大进来。
宝玉忙笼住马,意欲下来。
赖大忙上来抱住腿。
宝玉便在镫上站起来,笑携他的手,说了几句话。
接着又见一个小厮带着二三十个拿扫帚簸箕的人进来,见了宝玉,都顺墙垂手立住,独那为首的小厮打千儿,
\zhu{打千儿:旧时满族男子向人请安,左膝前屈,右腿后弯,上身微俯,右手下垂,行半跪礼。}
请了一个安。
宝玉不识名姓,只微笑点了点头儿。
马已过去,\geng{总为后文伏线。
\ping{此句令人费解。
}}那人方带人去了。
于是出了角门,门外又有李贵等六人的小厮并几个马夫,早预备下十来匹马专候。
一出了角门,李贵等都各上了马,前引傍围的一阵烟去了,不在话下。
\par
这里晴雯吃了药,仍不见病退,急的乱骂大夫,说:“只会骗人的钱,一剂好药也不给人吃。
”\geng{奇文。
真娇憨女儿之语也。
}麝月笑劝他道:“你太性急了,俗语说:‘病来如山倒,病去如抽丝。
’又不是老君的仙丹,那有这样灵药!你只静养几天,自然好了。
你越急越着手。
”\zhu{着手:附着手上,引申为棘手。
}晴雯又骂小丫头子们:“那里钻沙去了!\zhu{钻沙:贝甲类钻进沙里不易寻找,这里喻小丫头们都跑的找不见了。
}瞅我病了,都大胆子走了。
明儿我好了,一个一个的才揭你们的皮呢!”唬的小丫头子篆儿忙进来问:“姑娘作什么?”\geng{此“姑娘”亦“姑姑”“娘娘”之称,亦如贾琏处小厮呼平儿,皆南北互用一语也。
脂砚。
\zhu{第三十九回脂评:每见大家风俗多有小童称少主妾曰“姑姑”“娘娘”者。}
}晴雯道:“别人都死绝了,就剩了你不成?”说着,只见坠儿也\ceng 了进来。
\zhu{\ceng:也作蹭,这里是行动缓慢、欲行又止的样子。
}晴雯道:“你瞧瞧这小蹄子,不问他还不来呢。
这里又放月钱了,又散果子了,你该跑在头里了。
你往前些,我不是老虎吃了你!”坠儿只得前凑。
晴雯便冷不防欠身一把将他的手抓住,\geng{是病卧之时。
}向枕边取了一丈青,\zhu{一丈青:兼带挖耳勺的细长簪子,一头尖细,一头较粗,顶端作小杓,即“耳挖子”。
}向他手上乱戳,口内骂道:“要这爪子作什么?拈不得针,拿不动线,只会偷嘴吃。
眼皮子又浅,爪子又轻,打嘴现世的,不如戳烂了!”坠儿疼的乱哭乱喊。
麝月忙拉开坠儿,按晴雯睡下,笑道:“才出了汗,又作死。
等你好了,要打多少打不的?这会子闹什么!”晴雯便命人叫宋嬷嬷进来,说道:“宝二爷才告诉了我,叫我告诉你们,坠儿很懒,宝二爷当面使他,他拨嘴儿不动,
\zhu{拨嘴儿:拌口舌,争辩。}
连袭人使他,他背后骂他。
今儿务必打发他出去,明儿宝二爷亲自回太太就是了。
”\ping{晴雯行事暴躁,俨然女主人的样子。
}宋嬷嬷听了,心下便知镯子事发,因笑道:“虽如此说,也等花姑娘回来知道了,再打发他。
”晴雯道:“宝二爷今儿千叮咛万嘱咐的,什么‘花姑娘’‘草姑娘’,我们自然有道理。
你只依我的话,快叫他家的人来领他出去。
”\ping{袭人势大,晴雯吃醋。
}麝月道:“这也罢了。
早也去,晚也去,带了去早清净一日。
”\par
宋嬷嬷听了,只得出去唤了他母亲来,打点了他的东西,又来见晴雯等,说道:“姑娘们怎么了,你侄女儿不好,\geng{“侄女”二字妙,余前注不谬。
\zhu{前文,篆儿忙进来问:“姑娘作什么?”,后面有批语,指出这里的“姑娘”是“姑姑”的意思,所以这里称“侄女”。
}}你们教导他,怎么撵出去?也到底给我们留个脸儿。
”晴雯道:“你这话只等宝玉来问他,与我们无干。
”那媳妇冷笑道:“我有胆子问他去!他那一件事不是听姑娘们的调停?他纵依了,姑娘们不依,也未必中用。
比如方才说话,虽是背地里,姑娘就直叫他的名字。
在姑娘们就使得,在我们就成了野人了。
”晴雯听说,一发急红了脸,说道:“我叫了他的名字了,你在老太太跟前告我去,说我撒野,也撵出我去。
”麝月忙道:“嫂子,你只管带了人出去,有话再说。
这个地方岂有你叫喊讲礼的?你见谁和我们讲过礼?别说嫂子你,就是赖奶奶林大娘,也得担待我们三分。
便是叫名字,从小儿直到如今,都是老太太吩咐过的,你们也知道的,恐怕难养活,巴巴的写了他的小名儿,\ping{“宝玉”是个小名儿,也不知大名是什么。
}各处贴着叫万人叫去,为的是好养活。
连挑水挑粪花子都叫得,\zhu{花子:乞丐。
也称为“叫花子”。
}何况我们!连昨儿林大娘叫了一声‘爷’,老太太还说他呢,此是一件。
二则,我们这些人常回老太太的话去,可不叫着名字回话,难道也称‘爷’?那一日不把宝玉两个字念二百遍,偏嫂子又来挑这个了!过一日嫂子闲了,在老太太、太太跟前,听听我们当着面儿叫他就知道了。
嫂子原也不得在老太太、太太跟前当些体统差事,成年家只在三门外头混,\zhu{家:一作“价”,语尾助词,无义。
}怪不得不知我们里头的规矩。
这里不是嫂子久站的,再一会,不用我们说话,就有人来问你了。
有什么分证话,且带了他去,你回了林大娘,叫他来找二爷说话。
家里上千的人,你也跑来,我也跑来,我们认人问姓,还认不清呢!”说着,便叫小丫头子:“拿了擦地的布来擦地!”那媳妇听了,无言可对,亦不敢久立,赌气带了坠儿就走。
宋妈妈忙道:“怪道你这嫂子不知规矩,你女儿在这屋里一场,临去时,也给姑娘们磕个头。
没有别的谢礼,——便有谢礼,他们也不希罕,——不过磕个头,尽了心。
怎么说走就走?”坠儿听了,只得翻身进来,给他两个磕了两个头,又找秋纹等。
他们也不睬他。
那媳妇嗐声叹气,不敢多言,抱恨而去。
\par
晴雯方才又闪了风,着了气,反觉更不好了,翻腾至掌灯,刚安静了些。
只见宝玉回来,进门就嗐声跺脚。
麝月忙问原故,宝玉道:“今儿老太太喜喜欢欢的给了这个褂子,谁知不防后襟子上烧了一块,幸而天晚了,老太太、太太都不理论。
”一面说,一面脱下来。
麝月瞧时,果见有指顶大的烧眼,说:“这必定是手炉里的火迸上了。
这不值什么,赶着叫人悄悄的拿出去,叫个能干织补匠人织上就是了。
”说着便用包袱包了,交与一个妈妈送出去。
说:“赶天亮就有才好。
千万别给老太太、太太知道。
”婆子去了半日,仍旧拿回来,说:“不但能干织补匠人,就连裁缝绣匠并作女工的问了,都不认得这是什么,都不敢揽。
”麝月道:“这怎么样呢!明儿不穿也罢了。
”宝玉道:“明儿是正日子,老太太、太太说了,还叫穿这个去呢。
偏头一日烧了,岂不扫兴。
”\par
晴雯听了半日,忍不住翻身说道:“拿来我瞧瞧罢。
没个福气穿就罢了,这会子又着急。
”宝玉笑道:“这话倒说的是。
”说着,便递与晴雯,又移过灯来,细看了一会。
晴雯道:“这是孔雀金线织的,如今咱们也拿孔雀金线就像界线似的界密了,\zhu{界线:手工刺绣和织补工艺中所用的一种纵横线织法。
}只怕还可混得过去。
”麝月笑道:“孔雀线现成的,但这里除了你,还有谁会界线?”晴雯道:“说不得,我挣命罢了。
”宝玉忙道:“这如何使得!才好了些,如何做得活。
”晴雯道:“不用你蝎蝎螫螫的,\zhu{
蝎蝎螫螫:用人们对蝎螫的惊恐神情,形容过分的担心、惶恐、大惊小怪。
}我自知道。
”一面说,一面坐起来,挽了一挽头发,披了衣裳,只觉头重身轻,满眼金星乱迸,实实撑不住。
若不做,又怕宝玉着急,少不得恨命咬牙捱着。
便命麝月只帮着拈线。
晴雯先拿了一根比一比,笑道:“这虽不很像,若补上,也不很显。
”宝玉道:“这就很好,那里又找哦啰嘶国的裁缝去。
”\geng{妙谈!}
晴雯先将里子拆开,用茶杯口大的一个竹弓钉牢在背面,再将破口四边用金刀刮的散松松的,然后用针纫了两条,分出经纬,亦如界线之法,先界出地子后,
\zhu{地子:底子,花纹或文字的衬托面。}
依本衣之纹来回织补。
补两针,又看看,织补两针,又端详端详。
无奈头晕眼黑,气喘神虚,补不上三五针,伏在枕上歇一会。
宝玉在旁,一时又问:“吃些滚水不吃?”一时又命:“歇一歇。
”一时又拿一件灰鼠斗篷替他披在背上,一时又命拿个拐枕与他靠着。
\zhu{拐枕:像枕头那样的用品,供坐在炕上或床上支靠身体。
}急的晴雯央道:“小祖宗!你只管睡罢。
再熬上半夜,明儿把眼睛抠搂了,\zhu{抠搂:眼窝下陷(因熬夜或疲惫等导致)。
}
怎么处!”宝玉见他着急,只得胡乱睡下,仍睡不着。
一时只听自鸣钟已敲了四下,\geng{按“四下”乃寅正初刻,“寅”此样写法,避讳也。
\zhu{作者曹雪芹是江宁织造曹寅之孙。
}}刚刚补完;又用小牙刷慢慢的剔出绒毛来。
麝月道:“这就很好,若不留心,再看不出的。
”宝玉忙要了瞧瞧,说道:“真真一样了。
”晴雯已嗽了几阵,好容易补完了,说了一声:“补虽补了,到底不像,我也再不能了!”嗳哟了一声,便身不由主倒下。
要知端的,且听下回分解。
\par
\chai{qingwen}{晴雯补裘}
\qi{总评:此回前幅以药香、花香联络为章法,\zhu{前幅:前半部分。
}后幅以西洋鼻烟、西洋依弗哪药、西洋画儿、西洋诗、西洋哦啰嘶国雀金裘联络为章法,极穿插映带之妙。
\hang
写宝玉写不尽,却于仆从上描写一番。
于管家见时描写一番,于园工诸人上描写一番。
园中马是慢慢行,出门后又是一阵烟,大家气象、公子局度如画。
\zhu{局度:才干气度,格局。
}\hang
中一段写黛玉与宝玉满怀愁绪,有口难言,说不出一种凄凉,真是吴道子画顶上圆光。
\zhu{吴道子:唐代著名画家。
吴道子画顶上圆光:出自《梦溪笔谈·书画》“吴道子尝画佛,留其圆光,当大会中,对万众举手一挥,圆中运规,观者莫不惊呼。
画家为之自有法,但以肩倚壁,尽臂挥之,自然中规(吴氏徒手所画之圆其实是利用了圆规的原理)。
”这里是以吴道子的绘画技艺高超,比喻作者写作技术高妙。
}}
\dai{103}{晴雯病中撵走坠儿}
\dai{104}{勇晴雯病补雀金裘}
\sun{p52-1}{晴雯病卧,宝玉偷听,冬闺集艳}{图右下:晴雯偶感风寒,宝玉忙让她在里间暖阁躺着,悄悄请了大夫进来看病。
图右上:宝玉偷听麝月和平儿对话,原来说的是坠儿偷镯子一事。
图左侧:又一日晨,宝玉跟了丫鬟小螺到潇湘馆来,见宝钗姊妹、邢岫烟在叙家常,便也来凑趣。
}
\sun{p52-2}{鸳鸯回避宝玉,晴雯病补雀金裘}{图右侧:贾母送宝玉一件金翠辉煌的雀金裘,出来时遇见鸳鸯,宝玉便上来笑道:“好姐姐,你瞧瞧,我穿着这个好不好。
”鸳鸯一摔手,便进贾母房中来了。
图左侧:雀金裘后襟子上烧了一块,晴雯挣扎着坐起来带病修补。
}