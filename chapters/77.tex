\chapter{俏丫鬟抱屈夭风流 \quad 美优伶斩情归水月}
\qi{司棋一事,前文着实写来,此却随笔收去;晴雯一事,前文不过带叙,此却竭力发挥。
前文借晴雯一衬,文不寂寞;此文借司棋一引,文愈曲折。
}\par
话说王夫人见中秋已过,凤姐病已比先减了,虽未大愈,可以出入行走得了,仍命大夫每日诊脉服药,又开了丸药方子来配调经养荣丸。
因用上等人参二两,王夫人命人取时,翻寻了半日,只向小匣内寻了几枝簪挺粗细的。
\zhu{挺:通“梃”,棍棒。
}王夫人看了嫌不好,命再找去,又找了一大包须末出来。
王夫人焦躁道:“用不着偏有,但用着了,再找不着。
成日家我说叫你们查一查,都归拢在一处。
你们白不听,就随手混撂。
你们不知他的好处,用起来得多少换买来还不中使呢。
”\zhu{换:商行行话,指银两易物单位。
下文“三十换”即三十两银子换一两货物(如人参)。
}彩云道:“想是没了,就只有这个。
上次那边的太太来寻了些去,太太都给过去了。
”王夫人道:“没有的话,你再细找找。
”彩云只得又去找,拿了几包药材来说:“我们不认得这个,请太太自看。
除这个再没有了。
”王夫人打开看时,也都忘了,不知都是什么药,并没有一枝人参。
因一面遣人去问凤姐有无,凤姐来说:“也只有些参膏。
芦须虽有几枝,\zhu{参膏芦须:参膏:用次参或碎参熬的膏。
芦:人参顶部长叶处,亦称“参芦”。
须:人参的细根。
}也不是上好的,每日还要煎药里用呢。
”王夫人听了,只得向邢夫人那里问去。
邢夫人说:“因上次没了,才往这里来寻,早已用完了。
”王夫人没法,只得亲身过来请问贾母。
贾母忙命鸳鸯取出当日所馀的来,竟还有一大包,皆有手指头粗细的,遂称二两与王夫人。
王夫人出来交与周瑞家的拿去,令小厮送与医生家去,又命将那几包不能辨得的药也带了去,命医生认了,各包记号了来。
\geng{此等皆家常细事,岂是揣摩得者。
}\par
一时,周瑞家的又拿了进来说:“这几包都各包好记上名字了。
但这一包人参固然是上好的,如今就连三十换也不能得这样的了,\zhu{“三十换”,唯独列藏本作“八十换”。
清代民间可买卖的官参以四等参最好,而乾隆二十八年的人参价格,高者三十二换,次亦值二十五换,至三、四十年则达四、五十换,乾隆末年已到七、八十换。
赵翼于嘉庆元年写《人参诗》时,更称“白金三百两易一两,尚不得佳者”。
嘉庆十五年的四等参值二百四十换,道光初更达四百换。
从清代持续飙升的人参价格,可以发现在红楼梦写的年代,上好人参值三十换尚合理,列藏本抄写于乾隆后半叶,抄手可能因人参价格高涨,为符合原书中的语境,遂径自改抄为“八十换”。
另从人参价格推断该抄本的抄写时间不应晚于嘉庆元年。
}但年代太陈了。
这东西比别的不同,凭是怎样好的,只过一百年后,便自己就成了灰了。
如今这个虽未成灰,然已成了朽糟烂木,也无性力的了。
\ping{贾家就像一百年的人参一样,自动失去了它的效力走向衰亡。}
请太太收了这个,倒不拘粗细,好歹再换些新的倒好。
”王夫人听了,低头不语,半日才说:“这可没法了,只好去买二两来罢。
”也无心看那些,只命:“都收了罢。
”因向周瑞家的说:“你就去说给外头人们,拣好的换二两来。
倘一时老太太问,你们只说用的是老太太的,不必多说。
”周瑞家的方才要去时,宝钗因在坐,乃笑道:“姨娘且住。
如今外头卖的人参都没好的。
虽有一枝全的,他们也必截做两三段,镶嵌上芦泡须枝,\zhu{芦泡须枝:人参各部位的名称见页脚图片\foot{\footPic{人参各部位的名称}{renshen.png}{1.0}},大部分红楼梦校注本多将“芦泡须枝”之意略过不谈,少数解释为炮制过的人参,或谓“芦泡”是位于人参主根上端的芦头,并含混称因其“形似泡”。
“芦泡”在古代文献中很罕见,仅红楼梦曾出现过一次,此应指的是人参的芦头\foot{\footPic{人参各部位的名称}{renshen.png}{1.0}}、泡丁、参须和参枝(指参须上端较粗的部位)等部位。
当时参行多动手脚将一枝全的参身截做两三段,再镶嵌上其他芦头、参须和参枝加以改造。
黄叔灿初刊于嘉庆十三年的《参谱》中,尝称“泡者轻也,谓其无肉,外皮而朴(按:指粗糙如木皮)”, 嘉庆《直语补证》中亦谓“凡物虚大谓之泡”,知“泡”字乃用来形容人参主根或支根的内虚松软(即所谓“质薄肉少”,是生长不完全或虫蛀所致)。
又,道光《说文通训定声》称“以丁入物亦曰丁”,据此,“泡丁”应指在参皮之下以人工塞入参条,
参条因药效差且价钱廉,故常被拿来鱼目​​混珠或制造泡丁。
传统医书中的许多药方,在使用人参时常强调要先除去芦头(所谓“去芦”),否则将造成呕吐,幷直指“参条、参须不过得参之余气,危险之证断难倚仗”。
这些部位的药效髙低,亦显示在其售价上,如乾隆四十五年六月各类人参在南方每斤值银为四等人参1440两,五等人参1120两,渣末440两,泡丁400两,芦须(芦头或参须)120两,其中“渣末”乃运参之箱底所留下的零星杂末,间亦包含一些参枝。
}掺匀了好卖,看不得粗细。
我们铺子里常和参行交易,如今我去和妈说了,叫哥哥去托个伙计过去和参行商议说明,叫他把未作的原枝好参兑二两来。
不妨咱们多使几两银子,也得了好的。
”王夫人笑道:“倒是你明白。
就难为你亲自走一趟更好。
”于是宝钗去了,半日回来说:“已遣人去,赶晚就有回信的。
明日一早去配也不迟。
”王夫人自是喜悦,因说道:“‘卖油的娘子水梳头’,\zhu{俗语,原意是说经营某种物品的人反而舍不得用,义同“卖席的睡土炕”。
这里的“油”是指“发油”,例如第二十八回蒋玉菡说道:“……女儿愁,无钱去打桂花油。
……”,第六十二回湘云说道:“这鸭头不是那丫头,头上那讨桂花油。
”}自来家里有好的,不知给了人多少。
这会子轮到自己用,反倒各处求人去了。
”说毕长叹。
宝钗笑道:“这东西虽然值钱,究竟不过是药,原该济众散人才是。
咱们比不得那没见世面的人家,得了这个,就珍藏密敛的。
”\geng{调侃语。
}
王夫人点头道:“这话极是。
”\par
一时宝钗去后,因见无别人在室,遂唤周瑞家的来,问前日园中搜检的事情可得个下落。
周瑞家的是已和凤姐等人商议停妥,一字不隐,遂回明王夫人。
王夫人听了,虽惊且怒,却又作难,因思司棋系迎春之人,皆系那边的人,只得令人去回邢夫人。
周瑞家的回道:“前日那边太太嗔着王善保家的多事,打了几个嘴巴子,如今他也装病在家,不肯出头了。
况且又是他外孙女儿,自己打了嘴,他只好装个忘了,日久平服了再说。
如今我们过去回时,恐怕又多心,倒像似咱们多事似的。
不如直把司棋带过去,一并连赃证与那边太太瞧了,不过打一顿配了人,再指个丫头来,岂不省事。
如今白告诉去,那边太太再推三阻四的,又说‘既这样你太太就该料理,又来说什么’,岂不反耽搁了。
倘那丫头瞅空寻了死,反不好了。
如今看了两三天,人都有个偷懒的,倘一时不到,岂不倒弄出事来。
”王夫人想了一想,说:“这也倒是。
快办了这一件,再办咱们家的那些妖精。
”\par
周瑞家的听说,会齐了那几个媳妇,先到迎春房里,回迎春道:“太太们说了,司棋大了,连日他娘求了太太,太太已赏了他娘配人,今日叫他出去,另挑好的与姑娘使。
”说着,便命司棋打点走路。
迎春听了,含泪似有不舍之意,因前夜已闻得别的丫鬟悄悄的说了原故,虽数年之情难舍,但事关风化,亦无可如何了。
那司棋也曾求了迎春,实指望迎春能死保赦下的,只是迎春语言迟慢,耳软心活,是不能作主的。
司棋见了这般,知不能免,因哭道:“姑娘好狠心!哄了我这两日,如今怎么连一句话也没有?”周瑞家的等说道:“你还要姑娘留你不成?便留下,你也难见园里的人了。
依我们的好话,快快收了这样子,倒是人不知鬼不觉的去罢,大家体面些。
”迎春含泪道:“我知道你干了什么大不是,我还十分说情留下,岂不连我也完了。
你瞧入画也是几年的人,怎么说去就去了。
自然不止你两个,想这园里凡大的都要去呢。
依我说,将来终有一散,不如你各人去罢。
”\zhu{各人:自己。
}周瑞家的道:“所以到底是姑娘明白。
明儿还有打发的人呢,你放心罢。
”司棋无法,只得含泪与迎春磕头,和众姊妹告别,又向迎春耳根说:“好歹打听我要受罪,替我说个情儿,就是主仆一场!”迎春亦含泪答应:“放心。
”\par
于是周瑞家的人等带了司棋出了院门,又命两个婆子将司棋所有的东西都与他拿着。
走了没几步,后头只见绣橘赶来,一面也擦着泪,一面递与司棋一个绢包说:“这是姑娘给你的。
主仆一场,如今一旦分离,这个与你作个想念罢。
”司棋接了,不觉更哭起来了,又和绣橘哭了一回。
周瑞家的不耐烦,只管催促,二人只得散了。
司棋因又哭告道:“婶子大娘们,好歹略徇个情儿,
\zhu{徇情:为照顾私情而不讲原则。}
如今且歇一歇,让我到相好的姊妹跟前辞一辞,也是我们这几年好了一场。
”周瑞家的等皆各有事务,作这些事便是不得已了,况且又深恨他们素日大样,\zhu{大样:自高自大,轻狂傲慢。
}如今那里有工夫听他的话,因冷笑道:“我劝你走罢,别拉拉扯扯的了。
我们还有正经事呢。
谁是你一个衣包里爬出来的,\zhu{一个衣包里爬出来的:指同胞兄弟姊妹。
}辞他们作什么,他们看你的笑声还看不了呢。
你不过是挨一会是一会罢了,难道就算了不成!依我快走罢。
”一面说,一面总不住脚,直带着往后角门出去了。
司棋无奈,又不敢再说,只得跟了出来。
\par
可巧正值宝玉从外而入,一见带了司棋出去,又见后面抱着些东西,料着此去再不能来了。
 因闻得上夜之事,又兼晴雯之病亦因那日加重,细问晴雯,又不说是为何。
上日又见入画已去,今又见司棋亦走,不觉如丧魂魄一般,因忙拦住问道:“那里去?”周瑞家的等皆知宝玉素日行为,又恐唠叨误事,因笑道:“不干你事,快念书去罢。
”宝玉笑道:“好姐姐们,且站一站,我有道理。
”周瑞家的便道:“太太不许少捱一刻,又有什么道理。
我们只知遵太太的话,管不得许多。
”司棋见了宝玉,因拉住哭道:“他们做不得主,你好歹求求太太去。
”宝玉不禁也伤心,含泪说道:“我不知你作了什么大事,晴雯也病了,如今你又去。
都要去了,这却怎么的好。
”\geng{宝玉之语全作囫囵意,\zhu{囫囵:笼统含糊。
}最是极无味之语,偏是极浓极有情之语也。
只合如此写方是宝玉,稍有真切则不是宝玉了。
}周瑞家的发躁向司棋道:“你如今不是副小姐了,若不听话,我就打得你。
别想着往日有姑娘护着,任你们作耗。
越说着,还不好好走。
如今和小爷们拉拉扯扯,成个什么体统!”那几个媳妇不由分说,拉着司棋便出去了。
\par
宝玉又恐他们去告舌,\zhu{告舌:搬弄唇舌。
}恨的只瞪着他们,看已去远,方指着恨道:“奇怪,奇怪,怎么这些人只一嫁了汉子,染了男人的气味,就这样混帐起来,比男人更可杀了!” \geng{“染了男人的气味”实有此情理,非躬亲阅历者亦不知此语之妙。
}守园门的婆子听了,也不禁好笑起来,因问道:“这样说,凡女儿个个是好的了,女人个个是坏的了?”宝玉点头道:“不错,不错!”婆子们笑道:“还有一句话我们糊涂不解,倒要请问请问。
”方欲说时,只见几个老婆子走来,忙说道:“你们小心,传齐了伺候着。
此刻太太亲自来园里,在那里查人呢。
只怕还查到这里来呢。
又吩咐快叫怡红院的晴雯姑娘的哥嫂来,在这里等着领出他妹妹去。
”因笑道:“阿弥陀佛!今日天睁了眼,把这一个祸害妖精退送了,大家清净些。
”宝玉一闻得王夫人进来清查,便料定晴雯也保不住了,早飞也似的赶了去,所以这后来趁愿之语竟未得听见。
\par
宝玉及到了怡红院,只见一群人在那里,王夫人在屋里坐着,一脸怒色,见宝玉也不理。
晴雯四五日水米不曾沾牙,恹恹弱息,如今现从炕上拉了下来,蓬头垢面,两个女人才架起来去了。
王夫人吩咐,只许把他贴身衣服撂出去,馀者好衣服留下给好丫头们穿。
又命把这里所有的丫头们都叫来一一过目。
\ping{在本回王夫人撵走司棋:于是周瑞家的人等带了司棋出了院门,又命两个婆子将司棋所有的东西都与他拿着。
在本回王夫人撵走四儿和芳官:(王夫人)因喝命:“唤他干娘来领去,就赏他外头自寻个女婿去吧。
把他的东西一概给他。
”王夫人只让晴雯带着贴身衣服出去,可见王夫人最恨晴雯,红颜惹祸。
}\par
原来王夫人自那日着恼之后,王善保家的就趁势告倒了晴雯。
本处有人和园中不睦的,也就随机趁便下了些话。
王夫人皆记在心中。
因节间有事,故忍了两日,今日特来亲自阅人。
一则为晴雯犹可,二则因竟有人指宝玉为由,说他大了,已解人事,都由屋里的丫头们不长进教习坏了。
因这事更比晴雯一人较甚,\geng{暗伏一段“更比”。
觉烟迷雾罩之中更有无限溪山矣。
}\ping{后一件事“较甚”,所以前一件关于晴雯则“犹可”,即“不那么严重,还可以接受”。
袭人和宝玉偷试云雨,更加危险。
}乃从袭人起以至于极小作粗活的小丫头们,个个亲自看了一遍。
因问:“谁是和宝玉一日的生日?”本人不敢答应,老嬷嬷指道:“这一个蕙香,又叫作四儿的,是同宝玉一日生日的。
”王夫人细看了一看,虽比不上晴雯一半,却有几分水秀。
视其行止,聪明皆露在外面,且也打扮的不同。
王夫人冷笑道:“这也是个不怕臊的。
他背地里说的,同日生日就是夫妻。
这可是你说的?打量我隔的远,都不知道呢。
可知道我身子虽不大来,我的心耳神意时时都在这里。
难道我通共一个宝玉,就白放心凭你们勾引坏了不成!”
这个四儿见王夫人说着他素日和宝玉的私语,不禁红了脸,低头垂泪。
王夫人即命也快把他家的人叫来,领出去配人。
又问,“谁是耶律雄奴?”老嬷嬷们便将芳官指出。
王夫人道:“唱戏的女孩子,自然是狐狸精了!上次放你们,你们又懒待出去,可就该安分守己才是。
你就成精鼓捣起来,调唆着宝玉无所不为。
”芳官哭辩道:“并不敢调唆什么。
”王夫人笑道:“你还强嘴。
我且问你,前年我们往皇陵上去,是谁调唆宝玉要柳家的丫头五儿了?幸而那丫头短命死了,不然进来了,你们又连伙聚党遭害这园子呢。
\zhu{遭害:指遭受祸害。
}你连你干娘都欺倒了,岂止别人!”因喝命:“唤他干娘来领去,就赏他外头自寻个女婿去吧。
把他的东西一概给他。
”又吩咐上年凡有姑娘们分的唱戏的女孩子们,一概不许留在园里,都令其各人干娘带出,自行聘嫁。
一语传出,这些干娘皆感恩趁愿不尽,都约齐与王夫人磕头领去。
\ping{干娘白捡的干女儿可以通过婚姻的方式卖掉。}
\par
王夫人又满屋里搜检宝玉之物。
凡略有眼生之物,一并命收的收,卷的卷,着人拿到自己房内去了。
因说:“这才干净,省得旁人口舌。
”因又吩咐袭人麝月等人:“你们小心!往后再有一点分外之事,我一概不饶。
因叫人查看了,今年不宜迁挪,暂且挨过今年,明年一并给我仍旧搬出去心净。
”\geng{一段神奇鬼讶之文不知从何想来,王夫人从来未理家务,岂不一木偶哉?且前文隐隐约约已有无限口舌,浸润之谮原非一日矣。
\zhu{谮[zèn]:毁谤、诬谄。
}若无此一番更变,不独终无散场之局,且亦大不近乎情理。
况此亦是余旧日目睹亲闻,作者身历之现成文字,非捏造而成者,故迥不与小说之离合悲欢窠臼相对。
\zhu{窠臼:音“科旧”,指旧式建筑门下承受转轴的臼形小坑。
比喻陈旧、一成不变的规格模式。
}想遭零落之大族儿子见此,虽事有各殊,然其情理似亦有默契于心者焉。
此一段不独批此,直从抄检大观园及贾母对月兴尽生悲皆可附者也。
}说毕,茶也不吃,遂带领众人又往别处去阅人。
暂且说不到后文。
\par
如今且说宝玉只当王夫人不过来搜检搜检,无甚大事,谁知竟这样雷嗔电怒的来了。
所责之事皆系平日之语,一字不爽,\zhu{爽:违背,不合。
引申为过失,差错。
}料必不能挽回的。
虽心下恨不能一死,但王夫人盛怒之际,自不敢多言一句,多动一步,一直跟送王夫人到沁芳亭。
王夫人命:“回去好生念念那书,仔细明儿问你。
才已发下狠了。
”\zhu{发狠:恼怒,动气。
下决心,不顾一切。
}宝玉听如此说,方回来,一路打算:“谁这样犯舌?\zhu{犯舌:犹多嘴。
}况这里事也无人知道,如何就都说着了。
”一面想,一面进来,只见袭人在那里垂泪。
且去了第一等的人,岂不伤心,便倒在床上也哭起来。
袭人知他心内别的还犹可,独有晴雯是第一件大事,乃推他劝道:“哭也不中用了。
你起来我告诉你,晴雯已经好了,他这一家去,倒心净养几天。
你果然舍不得他,等太太气消了,你再求老太太,慢慢的叫进来也不难。
不过太太偶然信了人的诽言,一时气头上如此罢了。
”宝玉哭道:“我究竟不知晴雯犯了何等滔天大罪!”\geng{余亦不知,盖此等冤实非晴雯一人也。
}袭人道:“太太只嫌他生的太好了,未免轻佻些。
在太太是深知这样美人似的人必不安静,所以恨嫌他,像我们这粗粗笨笨的倒好。
”宝玉道:“这也罢了。
咱们私自顽话怎么也知道了?又没外人走风的,这可奇怪。
”袭人道:“你有甚忌讳的,一时高兴了,你就不管有人无人了。
我也曾使过眼色,也曾递过暗号,倒被那别人已知道了,你反不觉。
”宝玉道:“怎么人人的不是太太都知道,单不挑出你和麝月秋纹来?”袭人听了这话,心内一动,低头半日,无可回答,\ping{诛心一问,暗示袭人在晴雯事件中的作用。
}因便笑道:“正是呢。
若论我们也有顽笑不留心的孟浪去处,\zhu{孟浪:冒失、越礼。
}怎么太太竟忘了?想是还有别的事,等完了再发放我们,也未可知。
”宝玉笑道:“你是头一个出了名的至善至贤之人,他两个又是你陶冶教育的,焉得还有孟浪该罚之处!只是芳官尚小,过于伶俐些,未免倚强压倒了人,惹人厌。
四儿是我误了他,还是那年我和你拌嘴的那日起,叫上来作些细活,未免夺占了地位,故有今日。
\zhu{第二十一回,贾宝玉早起就去林黛玉和史湘云卧室玩闹,并且就在那里梳洗了,回来之后发现袭人生气了不理自己,赌气不理袭人及众丫鬟,所以让四儿服侍自己。
}只是晴雯也是和你一样,从小儿在老太太屋里过来的,虽然他生得比人强,也没甚妨碍去处。
就是他的性情爽利,口角锋芒些,究竟也不曾得罪你们。
想是他过于生得好了,反被这好所误。
”说毕,复又哭起来。
\par
袭人细揣此话,好似宝玉有疑他之意,竟不好再劝,因叹道:“天知道罢了。
此时也查不出人来了,白哭一会子也无益。
倒是养着精神,等老太太喜欢时,回明白了再要来是正理。
”宝玉冷笑道:“你不必虚宽我的心。
等到太太平服了再瞧势头去要时,知他的病等得等不得。
他自幼上来娇生惯养,何尝受过一日委屈。
连我知道他的性格,还时常冲撞了他。
他这一下去,就如同一盆才抽出嫩箭来的兰花送到猪窝里去一般。
况又是一身重病,里头一肚子的闷气。
他又没有亲爷热娘,只有一个醉泥鳅姑舅哥哥。
他这一去,一时也不惯的,那里还等得几日。
知道还能见他一面两面不能了!”说着又越发伤心起来。
袭人笑道:“可是你‘只许州官放火,不许百姓点灯’。
\zhu{只许州官放火,不许百姓点灯:宋代某州官名田登,忌讳和“登”同音的字,令百姓改称“灯”为“火”,每逢正月十五“放灯”,官榜写作“放火”。
百姓讽刺说“只许州官放火,不许百姓点灯”,见宋代陆游《老学庵笔记》。
后用以喻只许自己胡作非为,不许别人正当行动。
}我们偶然说一句略妨碍些的话,就说是不利之谈,你如今好好的咒他,是该的了!他便比别人娇些,也不至这样起来。
”宝玉道:“不是我妄口咒他,今年春天已有兆头的。
”袭人忙问何兆。
宝玉道:“这阶下好好的一株海棠花,竟无故死了半边,我就知有异事,果然应在他身上。
”袭人听了,又笑起来,因说道:“我待不说,又撑不住,你太也婆婆妈妈的了。
这样的话,岂是你读书的男人说的。
草木怎又关系起人来?若不婆婆妈妈的,真也成了个呆子了。
”\zhu{按:“若不婆婆妈妈的,真也成了个呆子了”各本均为正文,疑是批语混入,近是。
}\par
宝玉叹道:“你们那里知道,不但草木,凡天下之物,皆是有情有理的,也和人一样,得了知己,便极有灵验的。
若用大题目比,就有孔子庙前之桧,\zhu{桧:音“贵”,也名“桧柏”、“圆柏”,常绿乔木。
孔子庙前之桧,相传为孔子生前所种,当晋永嘉之乱时忽然枯死,到隋统一天下又复活。
}坟前之蓍,\zhu{蓍:音“诗”,蓍草,古代用蓍草茎占卜,传说孔子坟前的蓍草最为灵验。
}诸葛祠前之柏,\zhu{诸葛:诸葛亮。
相传诸葛亮庙前的柏树在唐末开始枯萎,到宋初又复活。
}岳武穆坟前之松。
\zhu{岳武穆:即岳飞。
宋代抗金名将,被奸相秦桧所害,后谥“武穆”。
相传岳坟前的树木为岳飞英灵所感,枝都朝南生长,心向南宋。
}这都是堂堂正大随人之正气,千古不磨之物。
世乱则萎,世治则荣,几千百年了,枯而复生者几次。
这岂不是兆应?小题目比,就有杨太真沉香亭之木芍药,\zhu{木芍药:即牡丹。
唐明皇曾与杨贵妃在沉香亭北赏牡丹,李白作《清平调》三章,以歌其事。
其一:云想衣裳花想容,春风拂槛露华浓。
若非群玉山头见,会向瑶台月下逢。
其二:一枝红艳露凝香,云雨巫山枉断肠。
借问汉宫谁得似?可怜飞燕倚新妆!其三:名花倾国两相欢,长得君王带笑看。
解释春风无限恨,沉香亭北倚阑干。
}
端正楼之相思树,\zhu{端正楼:位于骊山的华清宫,是当年杨贵妃梳妆的地方。
相思树:或指端正楼前的琪树。
安史之乱后,唐明皇见琪树而思念死在马嵬驿的杨贵妃。
}
王昭君冢上之草,\zhu{王昭君冢上之草:王昭君墓又称“青冢”,在今内蒙古自治区呼和浩特市南大黑河岸上。
或谓:“塞草皆白,唯此冢草青,故名”(见《大同府志》);或谓:“墓无草木,远而望之,冥蒙作黛色,故曰青冢”(见清宋荦《筠廊偶记》)。
一说,蒙语“呼和”意谓“青”,“浩特”意谓“城”,昭君葬该地,故名“青冢”。
}
岂不也有灵验。
所以这海棠亦应其人欲亡,故先就死了半边。
”袭人听了这篇痴话,又可笑,又可叹,因笑道:“真真的这话越发说上我的气来了。
那晴雯是个什么东西,就费这样心思,比出这些正经人来!还有一说,他纵好,也灭不过我的次序去。
\ping{袭人敌视晴雯得宠,坚决维护自己的地位。}
便是这海棠,也该先来比我,也还轮不到他。
想是我要死了。
”宝玉听说,忙握他的嘴,劝道:“这是何苦!一个未清,你又这样起来。
罢了,再别提这事,别弄的去了三个,又饶上一个。
”袭人听说,心下暗喜道:“若不如此,你也不能了局。
”宝玉乃道:“从此休提起,全当他们三个死了,不过如此。
况且死了的也曾有过,也没见我怎么样,\zhu{“死了的”可能是投井自尽的金钏。
}
此一理也。
\geng{宝玉至终一着全作如是想,\zhu{至终一着:令人费解,“着”可能是“生”的错讹,“至终一着”即为“至终一生”,自始至终的意思。
另:“着”可能是“省”的笔误,“至终一着”即为“至终一省”,最后醒悟的意思。
}所以始于情终于悟者。
既能终于悟而止,则情不得滥漫而涉于淫佚之事矣。
一人前事,一人了法,皆非“弃竹而复悯笋”之意。
\zhu{弃竹而复悯笋:舍弃大的而看重小的,得不偿失;另一种说法:笋长大为竹,比喻怜新弃旧、见异思迁。
}}
如今且说现在的,倒是把他的东西,作瞒上不瞒下,悄悄的打发人送出去与了他。
再或有咱们常时积攒下的钱,拿几吊出去给他养病,也是你姊妹好了一场。
”袭人听了,笑道:“你太把我们看的又小器又没人心了。
这话还等你说,我才已将他素日所有的衣裳以至各什各物总打点下了,都放在那里。
\ping{袭人已经知道,晴雯这次被撵出去,肯定不能再回来了。
}如今白日里人多眼杂,又恐生事,且等到晚上,悄悄的叫宋妈给他拿出去。
我还有攒下的几吊钱也给他罢。
”宝玉听了,感谢不尽。
袭人笑道:“我原是久已出了名的贤人,连这一点子好名儿还不会买来不成!”\ping{袭人自己道出自己是略施小惠,买好名声。
}宝玉听他方才的话,忙陪笑抚慰一时。
晚间果密遣宋妈送去。
\par
宝玉将一切人稳住,便独自得便出了后角门,央一个老婆子带他到晴雯家去瞧瞧。
\ping{宝玉也知稳住人再走,也算成长,原本觉得所有女儿都是好人,现在对袭人产生了怀疑,也知道她们之间也是存在倾轧,开始防备她们。
可以说是彻底的童年的消逝。
}先是这婆子百般不肯,只说怕人知道,“回了太太,我还吃饭不吃饭!”无奈宝玉死活央告,又许他些钱,那婆子方带了他来。
这晴雯当日系赖大家用银子买的,那时晴雯才得十岁,尚未留头。
\zhu{留头:又叫“留满头”。
旧时女子幼年剃发,随着年事增长,先留顶心头发,再留全发,叫做“留头”。
}因常跟赖嬷嬷进来,贾母见他生得伶俐标致,十分喜爱。
故此赖嬷嬷就孝敬了贾母使唤,后来所以到了宝玉房里。
这晴雯进来时,也不记得家乡父母。
只知有个姑舅哥哥,专能庖宰,也沦落在外,故又求了赖家的收买进来吃工食。
\zhu{吃工食:靠干活吃饭。
}赖家的见晴雯虽到贾母跟前,千伶百俐,嘴尖性大,却倒还不忘旧,\geng{只此一句便是晴雯正传。
可知晴雯为聪明风流所害也。
一篇为晴雯写传,是哭晴雯也。
}\geng{非哭晴雯,乃哭风流也。
}故又将他姑舅哥哥收买进来,把家里一个女孩子配了他。
成了房后,谁知他姑舅哥哥一朝身安泰,就忘却当年流落时,任意吃死酒,家小也不顾。
偏又娶了个多情美色之妻,见他不顾身命,不知风月,一味死吃酒,便不免有蒹葭倚玉之叹,\zhu{蒹葭倚玉:即“蒹葭倚玉树”。
蒹葭,音“兼家”,芦苇,喻质之贱。
玉树,喻质之贵。
《世说新语·容止》:魏明帝使毛曾与夏侯玄共坐,时人谓“蒹葭倚玉树”。
蒹葭指毛曾,玉树指夏侯玄,谓两人品德才貌极不相称。
这里趣指多浑虫不配和灯姑娘结为夫妇。
}红颜寂寞之悲。
又见他器量宽宏,\geng{趣极!“器量宽宏”如此用,真扫地矣。
}并无嫉衾妒枕之意,这媳妇遂恣情纵欲,满宅内便延揽英雄,收纳材俊,上上下下竟有一半是他考试过的。
若问他夫妻姓甚名谁,便是上回贾琏所接见的多浑虫灯姑娘儿的便是了。
\zhu{灯姑娘:即“多姑娘”。}
\geng{奇奇怪怪,左盘右旋,千丝万缕,皆自一体也。
}目今晴雯只有这一门亲戚,所以出来就在他家。
\par
此时多浑虫外头去了,那灯姑娘吃了饭去串门子,只剩下晴雯一人,在外间房内爬着。
\geng{总哭晴雯。
}宝玉命那婆子在院门瞭哨,\zhu{瞭哨:放哨。
}他独自掀起草帘\geng{“草帘”。
}进来,一眼就看见晴雯睡在芦席土炕上,\geng{“芦席土炕”。
}幸而衾褥还是旧日铺的。
心内不知自己怎么才好,因上来含泪伸手轻轻拉他,悄唤两声。
当下晴雯又因着了风,又受了他哥嫂的歹话,病上加病,嗽了一日,才朦胧睡了。
忽闻有人唤他,强展星眸,一见是宝玉,又惊又喜,又悲又痛,忙一把死攥住他的手。
哽咽了半日,方说出半句话来:“我只当不得见你了。
”接着便嗽个不住。
宝玉也只有哽咽之分。
\par
晴雯道:“阿弥陀佛,你来的好,且把那茶倒半碗我喝。
渴了这半日,叫半个人也叫不着。
”宝玉听说,忙拭泪问:“茶在那里?”晴雯道:“那炉台上就是。
”宝玉看时,虽有个黑沙吊子,\zhu{沙吊子:也作沙铫子(铫:音“吊”),用砂土烧制成的阔口壶,煎药或烧开水用,短嘴,直柄,有盖,质地较薄。
}却不像个茶壶。
只得桌上去拿了一个碗,也甚大甚粗,不像个茶碗,未到手内,先就闻得油膻之气。
\geng{不独为晴雯一哭,且为宝玉一哭亦可。
}宝玉只得拿了来,先拿些水洗了两次,复又用水汕过,\zhu{汕:音“善”,冲洗,冲刷。
}方提起沙壶斟了半碗。
\zhu{沙壶:即“沙吊子”。
}看时,绛红的,\zhu{绛:音“匠”,大红色。
}
也太不成茶。
晴雯扶枕道:“快给我喝一口罢!这就是茶了。
那里比得咱们的茶!”宝玉听说,先自己尝了一尝,并无清香,且无茶味,只一味苦涩,略有茶意而已。
尝毕,方递与晴雯。
只见晴雯如得了甘露一般,一气都灌下去了。
\ping{“如得了甘露一般”,一方面是太久没有喝水,另一方面是宝玉的深情就像甘露一样。}
宝玉心下暗道:“往常那样好茶,他尚有不如意之处;今日这样。
看来,可知古人说的‘饱饫烹宰,饥餍糟糠’,\zhu{饫:音“玉”,饱食。
烹宰:代指鱼肉美食。
餍:满足。
}又道是‘饭饱弄粥’,\zhu{饭饱弄粥:俗语。
干饭吃够了,想要吃点稀粥了。
意思是说吃腻了精美的饭食,想换点清淡的东西来调节一下口味。
}可见都不错了。
”\geng{妙!通篇宝玉最恶书者,每因女子之所历始信其可,此谓触类旁通之妙诀矣。
}一面想,一面流泪问道:“你有什么说的,趁着没人告诉我。
”晴雯呜咽道:“有什么可说的!不过挨一刻是一刻,挨一日是一日。
我已知横竖不过三五日的光景,就好回去了。
只是一件,我死也不甘心的:我虽生的比别人略好些,并没有私情密意勾引你怎样,如何一口死咬定了我是个狐狸精!我太不服。
今日既已担了虚名,而且临死,不是我说一句后悔的话,早知如此,我当日也另有个道理。
不料痴心傻意,只说大家横竖是在一处。
不想平空里生出这一节话来,有冤无处诉。
”说毕又哭。
\par
宝玉拉着他的手,只觉瘦如枯柴,腕上犹戴着四个银镯,因泣道:“且卸下这个来,等好了再戴上罢。
”因与他卸下来,塞在枕下。
又说:“可惜这两个指甲,好容易长了二寸长,\zhu{寸:量词。
计算长度的单位。
公制一寸等于十公分。
}这一病好了,又损好些。
”
\ping{晴雯留这么长的指甲,是不能做粗活的,晴雯俨然是小姐的身份了。}
晴雯拭泪,就伸手取了剪刀,将左手上两根葱管一般的指甲齐根铰下;又伸手向被内将贴身穿着的一件旧红绫袄脱下,并指甲都与宝玉道:“这个你收了,以后就如见我一般。
快把你的袄儿脱下来我穿。
我将来在棺材内独自躺着,也就像还在怡红院的一样了。
论理不该如此,只是担了虚名,我可也是无可如何了。
”宝玉听说,忙宽衣换上,藏了指甲。
晴雯又哭道:“回去他们看见了要问,不必撒谎,就说是我的。
既担了虚名,越性如此,也不过这样了。
”\lie{晴雯此举胜袭人多矣,真一字一哭也,又何必鱼水相得而后为情哉?}\par
一语未了,只见他嫂子笑嘻嘻掀帘进来,道:“好呀,你两个的话,我已都听见了。
”又向宝玉道:“你一个作主子的,跑到下人房里作什么?看我年轻又俊,敢是来调戏我么?”宝玉听说,吓的忙陪笑央道:“好姐姐,快别大声。
他伏侍我一场,我私自来瞧瞧他。
”灯姑娘便一手拉了宝玉进里间来,笑道:“你不叫嚷也容易,\zhu{“不叫嚷”的应该是“我”而非“你”。
}只是依我一件事。
”说着,便坐在炕沿上,却紧紧的将宝玉搂入怀中。
宝玉如何见过这个,心内早突突的跳起来了,急的满面红涨,又羞又怕,只说:“好姐姐,别闹。
”\geng{如闻如见,“别闹”二字活跳。
}灯姑娘乜斜醉眼,\zhu{乜(音“咩”)斜 :眯着眼睛,斜眼看人。
}笑道:“呸!成日家听见你风月场中惯作工夫的,怎么今日就反讪起来。
”\zhu{讪:音“善”,羞惭,难为情 。
}宝玉红了脸,笑道:“姐姐放手,有话咱们好说。
外头有老妈妈,听见什么意思。
”灯姑娘笑道:“我早进来了,却叫婆子去园门等着呢。
我等什么似的,今儿等着了你。
虽然闻名,不如见面,空长了一个好模样儿,竟是没药信的炮仗,只好装幌子罢了,倒比我还发讪怕羞。
可知人的嘴一概听不得的。
就比如方才我们姑娘下来,我也料定你们素日偷鸡盗狗的。
我进来一会在窗下细听,屋内只你二人,若有偷鸡盗狗的事,岂有不谈及于此,谁知你两个竟还是各不相扰。
可知天下委屈事也不少。
如今我反后悔错怪了你们。
既然如此,你但放心。
以后你只管来,我也不罗唣你。
”\zhu{罗唣:即“啰唣”,音“罗造”,骚扰,吵闹。
}\par
宝玉听说,才放下心来,方起身整衣央道:“好姐姐,你千万照看他两天。
我如今去了。
”说毕出来,又告诉晴雯。
二人自是依依不舍,也少不得一别。
晴雯知宝玉难行,遂用被蒙头,总不理他,宝玉方出来。
意欲到芳官、四儿处去,无奈天黑,出来了半日,恐里面人找他不见,又恐生事,遂且进园来了,明日再作计较。
因乃至后角门,小厮正抱铺盖,里边嬷嬷们正查人,若再迟一步也就关了。
\par
宝玉进入园中,且喜无人知道。
到了自己房内,告诉袭人只说在薛姨妈家去的,也就罢了。
一时铺床,袭人不得不问今日怎么睡。
宝玉道:“不管怎么睡罢了。
”原来这一二年间袭人因王夫人看重了他了,越发自要尊重。
凡背人之处,或夜晚之间,总不与宝玉狎昵,较先幼时反倒疏远了。
\ping{晴雯前车之鉴。
}况虽无大事办理,然一应针线并宝玉及诸小丫头们凡出入银钱衣履什物等事,也甚烦琐;且有吐血旧症虽愈,然每因劳碌风寒所感,即嗽中带血,故迩来夜间总不与宝玉同房。
\zhu{迩:音“耳”,近。
}宝玉夜间常醒,又极胆小,每醒必唤人。
因晴雯睡卧警醒,且举动轻便,故夜晚一应茶水起坐呼唤之任皆悉委他一人,所以宝玉外床只是他睡。
今他去了,袭人只得要问,因思此任比日间紧要之意。
宝玉既答不管怎样,袭人只得还依旧年之例,遂仍将自己铺盖搬来设于床外。
\par
宝玉发了一晚上呆。
\geng{一句足矣。
}及催他睡下,袭人等也都睡后,听着宝玉在枕上长吁短叹,复去翻来,直至三更以后。
方渐渐的安顿了,略有齁声。
袭人方放心,也就朦胧睡着。
没半盏茶时,只听宝玉叫“晴雯”。
袭人忙睁开眼连声答应,问作什么。
宝玉因要吃茶。
袭人忙下去向盆内蘸过手,从暖壶内倒了半盏茶来吃过。
宝玉乃笑道:\geng{“笑”字好极,有文章,盖恐冷落袭人也。
}“我近来叫惯了他,却忘了是你。
”袭人笑道:“他一乍来时你也曾睡梦中直叫我,半年后才改了。
我知道这晴雯人虽去了,这两个字只怕是不能去的。
”说着,大家又卧下。
宝玉又翻转了一个更次,至五更方睡去时,只见晴雯从外头走来,仍是往日形景,进来笑向宝玉道:“你们好生过罢,我从此就别过了。
”说毕,翻身便走。
宝玉忙叫时,又将袭人叫醒。
袭人还只当他惯了口乱叫,却见宝玉哭了,说道:“晴雯死了。
”袭人笑道:“这是那里的话!你就知道胡闹,被人听着什么意思。
”宝玉那里肯听,恨不得一时亮了就遣人去问信。
\par
及至天亮时,就有王夫人房里小丫头立等叫开前角门传王夫人的话:“即时叫起宝玉,快洗脸,换了衣裳快来,因今儿有人请老爷寻秋赏桂花,老爷因喜欢他前儿作得诗好,故此要带他们去。
这都是太太的话,一句别错了。
你们快飞跑告诉他去,立逼他快来,老爷在上屋里还等他吃面茶呢。
\zhu{面茶:一种小吃。
面粉加油炒熟,再加入开水冲或煮成糊状,吃时加糖或麻酱、椒盐等。
}环哥儿已来了。
快跑,快跑。
再着一个人去叫兰哥儿,也要这等说。
”里面的婆子听一句,应一句,一面扣扭子,一面开门。
一面早有两三个人一行扣衣,一行分头去了。
\par
袭人听得叩院门,便知有事,忙一面命人问时,自己已起来了。
听得这话,促人来舀了面汤,\zhu{面汤:洗脸的热水。
}催宝玉起来盥漱。
他自去取衣。
因思跟贾政出门,便不肯拿出十分出色的新鲜衣履来,只拿那二等成色的来。
宝玉此时亦无法,只得忙忙的前来。
果然贾政在那里吃茶,十分喜悦。
宝玉忙行了省晨之礼。
\zhu{省(音“醒”):家庭日常礼节。
子女对父母早上问安叫“省”,晚上服侍就寝叫“定”。
见《礼记·曲礼上》:“凡为人子之礼,冬温而夏清,昏定而晨省。
”}贾环贾兰二人也都见过宝玉。
贾政命坐吃茶,向环兰二人道:“宝玉读书不如你两个,论题联和诗这种聪明,你们皆不及他。
今日此去,未免强你们做诗,宝玉须听便助他们两个。
”王夫人等自来不曾听见这等考语,真是意外之喜。
\par
一时候他父子二人等去了,方欲过贾母这边来时,就有芳官等三个的干娘走来,回说:“芳官自前日蒙太太的恩典赏了出去,他就疯了似的,茶也不吃,饭也不用,勾引上藕官、蕊官,三个人寻死觅活,只要剪了头发做尼姑去。
我只当是小孩子家一时出去不惯也是有的,不过隔两日就好了。
谁知越闹越凶,打骂着也不怕。
实在没法,所以来求太太,或是就依他们做尼姑去,或教导他们一顿,赏给别人作女儿去罢,我们也没这福。
”王夫人听了道:“胡说!那里由得他们起来,佛门也是轻易人进去的!每人打一顿给他们,看还闹不闹了!”\par
当下因八月十五日各庙内上供去,皆有各庙内的尼姑来送供尖之例,\zhu{供尖:指供品的顶端部分。
以其馈人,以示祝福。
}王夫人曾于十五日就留下水月庵的智通与地藏庵的圆信住两日,至今日未回,听得此信,巴不得又拐两个女孩子去作活使唤,因都向王夫人道:“咱们府上到底是善人家。
因太太好善,所以感应得这些小姑娘们皆如此。
虽说佛门轻易难入,也要知道佛法平等。
我佛立愿,原是一切众生无论鸡犬皆要度他,无奈迷人不醒。
若果有善根能醒悟,即可以超脱轮回。
所以经上现有虎狼蛇虫得道者就不少。
如今这两三个姑娘既然无父无母,家乡又远,他们既经了这富贵,又想从小儿命苦入了这风流行次,将来知道终身怎么样,所以苦海回头,出家修修来世,也是他们的高意。
太太倒不要限了善念。
”\par
王夫人原是个好善的,先听彼等之语不肯听其自由者,因思芳官等不过皆系小儿女一时不遂之谈\foot{庚本原作“一时不遂之但”,另笔点去“之”字,旁添“心故有此意”。
诸本均作“不遂之谈”,“但”或系“谈”的音讹。
唯“不遂之谈”略显生硬,是否的当,有待推敲。
而庚本旁改文字显非原文,暂从诸本改。
},恐将来熬不得清净,反致获罪。
今听这两个拐子的话大近情理;且近日家中多故,又有邢夫人遣人来知会,明日接迎春家去住两日,以备人家相看;且又有官媒婆来求说探春等事,\zhu{官媒婆:旧时衙门中的女差役。
承办择配女犯或官僚贵族之家放出婚配的女奴,还承担女犯的押解伴送等事。
官媒婆也指以做媒为业的妇女。
}
心绪正烦,那里着意在这些小事上。
既听此言,便笑答道:“你两个既这等说,你们就带了作徒弟去如何?”两个姑子听了,念一声佛道:“善哉!善哉!若如此,可是你老人家阴德不小。
”说毕,便稽首拜谢。
\zhu{稽首:音“起首”,一种俯首至地的最敬礼。
}王夫人道:“既这样,你们问他们去。
若果真心,即上来当着我拜了师父去罢。
”\par
这三个女人听了出去,果然将他三人带来。
王夫人问之再三,他三人已是立定主意,遂与两个姑子叩了头,又拜辞了王夫人。
王夫人见他们意皆决断,知不可强了,反倒伤心可怜,\ping{王夫人信佛,但是同时认为出家为尼遁入佛门是女性很差的结局,不如出嫁过世俗生活,这就很矛盾了,这体现了王夫人信佛,不是因为信奉佛教经义,而是有很功利的世俗现实目的。
}忙命人取了些东西来赍赏了他们,\zhu{赍:音“鸡”,赠送。
}又送了两个姑子些礼物。
从此芳官跟了水月庵的智通,蕊官、藕官二人跟了地藏庵的圆信,各自出家去了。
再听下回分解。
\par
\qi{总评:看晴雯与宝玉永绝一段,的是消魂文字;看宝玉几番呆论,真是至诚种子;看宝玉给晴雯斟茶,又真是呆公子。
前文叙袭人奔丧时,宝玉夜来吃茶,先呼袭人,\zhu{第五十一回:晴雯自在熏笼上,麝月便在暖阁外边。
至三更以后,宝玉睡梦之中,便叫袭人。
叫了两声,无人答应,自己醒了,方想起袭人不在家,自己也好笑起来。
}此又夜来吃茶,先呼晴雯。
字字龙跳天门,虎卧凤阙;语语婴儿恋母,稚鸟寻巢。
}

\dai{153}{司棋被撵,宝玉欲拦}
\dai{154}{宝玉探视晴雯,互赠信物}
\sun{p77-1}{司棋被撵,宝玉欲拦}{王夫人命将司棋逐出大观园。
周瑞家的领令行事,迎春含泪似有不舍之意。
正值宝玉从外而入,司棋便哭求宝玉。
宝玉见婆子说话蛮横,又见后面抱着些东西,知道不能回来了,不觉如丧魂魄一般。
那几个媳妇不由分说,拉着司棋走了。
}
\sun{p77-2}{王夫人撵走晴雯审查丫头}{宝玉到了怡红院,王夫人在屋里坐着,一脸怒色,见宝玉也不理。
晴雯被两个女人架起来去了。
王夫人只许把他贴身衣服撂出去,好衣服留下给好丫头们穿,又命把所有的丫头们都叫来一一过目。
}