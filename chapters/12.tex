\chapter{王熙凤毒设相思局\quad 贾天祥正照风月鉴}
\qi{反正从来总一心,镜光至意两相寻。
有朝敲破蒙头瓮,绿水青山任好春。
}\par
话说凤姐正与平儿说话,只见有人回说:“瑞大爷来了。
”凤姐急命:\geng{立意追命。
}“快请进来。
”贾瑞见往里让,心中喜出望外,急忙进来,见了凤姐,满面陪笑,\geng{如蛇。
\zhu{这条评语的意思大概是形容贾瑞色心得到满足后得意之形和企图更进一步的丑态。}
}连连问好。
凤姐儿也假意殷勤,让坐让茶。
\par
贾瑞见凤姐如此打扮,益发酥倒,因饧了眼问道:
\zhu{饧:音“行”,眼睛半睁半闭或呆滞无神,形容眼色朦胧。}
“二哥哥怎么还不回来?”凤姐道:“不知什么原故。
”贾瑞笑道:“别是路上有人绊住了脚了,\meng{旁敲远引。
}舍不得回来也未可知?”凤姐道:“也未可知。
男人家见一个爱一个也是有的。
”\meng{这是钩。
}贾瑞笑道:\ji{如闻其声。
}
“嫂子这话错了,我就不这样。
”\ji{渐渐入港。
\zhu{入港:说话投机。
}}凤姐笑道:“像你这样的人能有几个呢,十个里也挑不出一个来。
”\geng{勿作正面看为幸。
畸笏。
}
\meng{游鱼虽有入瓮之志,无钩不能上岸;一上钩来,欲去亦不可得。
}贾瑞听了,喜的抓耳挠腮,又道:“嫂子天天也闷的很?”凤姐道:“正是呢,只盼个人来说话解解闷儿。
”贾瑞笑道:“我倒天天闲着,天天过来替嫂子解解闲闷可好不好?”凤姐笑道:“你哄我呢,你那里肯往我这里来?”贾瑞道:“我在嫂子跟前,若有一点谎话,天打雷劈!只因素日闻得人说,嫂子是个利害人,在你跟前一点也错不得,所以唬住了我。
如今见嫂子最是个有说有笑极疼人的,\ji{奇妙!}我怎么不来,——死了也愿意!”\geng{这倒不假。
}凤姐笑道:“果然你是个明白人,比贾蓉两个强远了。
我看他那样清秀,只当他们心里明白,谁知竟是两个糊涂虫,\geng{反文,着眼。
\zhu{
凤姐在这里佯装对贾蓉贾蔷两人不解风情而惋惜,实际上暗示了和凤姐关系暧昧的真是这两人。
第六回贾蓉向凤姐借炕屏就体现了贾蓉和凤姐的暧昧;
第九回提及贾蓉和贾蔷“最相亲厚”,甚至到奴仆造言诽谤的程度;
本回后文受凤姐指派捉弄贾瑞的正是这两人。
}
}一点不知人心。
”\par
贾瑞听这话,越发撞在心坎儿上,由不得又往前凑了一凑,\meng{写呆人痴性活现。
}觑着眼看凤姐带的荷包,然后又问戴着什么戒指。
凤姐悄悄道:“放尊重着,别叫丫头们看了笑话。
”贾瑞如听纶音佛语一般,\zhu{纶(音“伦”)音:皇帝的命令。
后常以“纶音”代指“圣旨”。
纶:古代缚印玺或帷幕用的带子。
}
忙往后退。
\ping{色胆包天但又战战兢兢。
}凤姐笑道:“你该去了。
”\ji{叫“去”,正是叫“来”也。
}贾瑞道:“我再坐一坐儿。
好狠心的嫂子!”凤姐又悄悄的道:“大天白日,人来人往,你就在这里也不方便。
你且去,等着晚上起了更你来,悄悄的在西边穿堂儿等我。
”\geng{先写穿堂,只知房舍之大,岂料有许多用处。
}
\meng{凡人在平静时,物来言至,无不照见。
若迷于一事一物,虽风雷交作,有所不闻。
\zhu{
冷静清醒状态下的人,对于周围反应灵敏;
而一旦沉湎着迷,心智被彻底占据,就会对周围毫无反应。
}
即“穿堂儿等”之一语,府第非比凡常,关启门户,必要查看,且更夫仆妇,势必往来,岂容人藏过于其间?只因色迷,闻声连诺,不能有回思之暇,信可悲夫!}贾瑞听了,如得珍宝,忙问道:“你别哄我。
但只那里人过的多,怎么好躲的?”凤姐道:“你只放心。
我把上夜的小厮们都放了假,两边门一关,再没别人了。
”贾瑞听了,喜之不尽,忙忙的告辞而去,心内以为得手。
\geng{未必。
}\par
盼到晚上,果然黑地里摸入荣府,趁掩门时,钻入穿堂。
果见漆黑无人,往贾母那边去的门户已锁,倒只有向东的门未关。
贾瑞侧耳听着,半日不见人来,忽听咯登一声,东边的门也倒关了。
\geng{平平略施小计。
}
贾瑞急的也不敢则声,\zhu{则:做。
则声:开口发言、出声。
}只得悄悄的出来,将门撼了撼,关得铁桶一般。
此时要求出去,亦不能够。
\meng{此大抵是凤姐调遣。
不先为点明者,可以少许多事故,又可以藏拙。
}南北皆是大房墙,要跳亦无攀援。
这屋内又是过门风,空落落;现是腊月天气,夜又长,朔风凛凛,侵肌裂骨,一夜几乎不曾冻死。
\geng{可为偷情一戒。
}\meng{教导之法、慈悲之心尽矣,无奈迷徒
不悟何!}好容易盼到早晨,只见一个老婆子先将东门开了,进去又叫西门。
贾瑞瞅他背着脸,一溜烟抱着肩跑了出来,幸而天气尚早,人都未起,从后门一径跑回家去。
\par
原来贾瑞父母早亡,只有他祖父代儒教养。
那代儒素日教训最严,\geng{教训最严,奈其心何!一叹。
}不许贾瑞多走一步,生怕他在外吃酒赌钱,有误学业。
今忽见他一夜不归,只料定他在外非饮即赌,嫖娼宿妓,\geng{辗转灵活,一人不放,一笔不肖。
\zhu{肖:类似;相像。一笔不肖:意思应该是“没有一笔是不像的”。}
}那里想到这段公案,\geng{世人万万想不到,况老学究乎!}因此气了一夜。
贾瑞也捻着一把汗,少不得回来撒慌,只说:“往舅舅家去了,天黑了,留我住了一夜。
”代儒道:“自来出门,非禀我不敢擅出,如何昨日私自去了?据此亦该打,何况是撒谎!”\geng{处处点父母痴心、子孙不肖。
此书系自愧而成。
}因此,发狠到底打了三四十板,不许吃饭,令他跪在院内读文章,定要补出十天工课来方罢。
贾瑞直冻了一夜,今又遭了苦打,且饿着肚子跪在风地里念文章,\meng{教令何尝不好,孽种故此不同。
}其苦万状。
\ji{祸福无门,唯人自招。
}\par
此时贾瑞前心犹是未改,\geng{四字是寻死之根。
}\geng{苦海无边,回头是岸。
若个能回头也?叹叹!壬午春。
畸笏。
}再想不到是凤姐捉弄他。
过后两日,得了空,便仍来找凤姐。
凤姐故意抱怨他失信,贾瑞急的赌身发誓。
\ping{误堕情网,不可自拔,智商下线,执迷不悟。
}凤姐因见他自投罗网,\geng{可谓因人而使。
}少不得再寻别计令他知改,\geng{四字是作者明阿凤身份,勿得轻轻看过。
}故又约他道:“今日晚上,你别在那里了。
你在我这房后小过道子里那间空屋里等我,可别冒撞了。
”\ji{伏的妙!
\zhu{伏贾瑞误把贾蓉当作凤姐欲行非礼。}
}贾瑞道:“果真?”凤姐道:“谁可哄你,你不信就别来。
”\geng{紧一句。
}\meng{大士心肠。
\zhu{大士:佛教称佛和菩萨为大士。}
}贾瑞道:“来,来,来。
死也要来!”\ji{不差。
}凤姐道:“这会子你先去罢。
”贾瑞料定晚间必妥,\geng{未必。
}此时先去了。
凤姐在这里便点兵派将,\geng{四字用得新,必有新文字好看。
}
\meng{新文,最妙!}设下圈套。
\par
那贾瑞只盼不到晚上,偏生家里有亲戚又来了,\ji{专能忙中写闲,狡猾之甚!}直等吃了晚饭才去,那天已有掌灯时候。
又等他祖父安歇了,方溜进荣府,直往那夹道中屋子里来等着,热锅上的蚂蚁一般,\meng{有心人记着,其实苦恼。
}只是干转。
左等不见人影,右听也没声音,心下自思:“别是又不来了,又冻我一夜不成?”\meng{似醒非醒语。
}正自胡猜,只见黑魆魆的来了一个人,\zhu{魆:音“虚”。
黑魆魆:形容黑暗。
}\geng{真到了。
}贾瑞便意定是凤姐,不管皂白,饿虎一般,等那人刚至门前,便如猫儿捕鼠的一般,抱住叫道:“亲嫂子,等死我了。
”说着,抱到屋里炕上就亲嘴扯裤子,满口里“亲娘”“亲爹”的乱叫起来。
\meng{丑态可笑。
}那人只不做声,\geng{好极!}贾瑞拉了自己裤子,硬帮帮的就想顶入。
\geng{将到矣。
}忽然灯光一闪,只见贾蔷举着个捻子照道:\zhu{捻(音“辗”)子:这里指引火用的纸卷儿。
}“谁在屋里?”只见炕上那人笑道:“瑞大叔要臊我呢。
\zhu{臊:音“扫”四声,羞,难为情。}
”贾瑞一见,却是贾蓉,\ji{奇绝!}真臊的无地可入,\geng{亦未必真。
}不知要怎么样才好,回身就要跑,被贾蔷一把揪住道:“别走!如今琏二婶已经告到太太跟前,\geng{好题目。
}说你无故调戏他。
\geng{调戏还有“有故”?一笑。
}他暂用了个脱身计,哄你在这边等着,太太气死过去,\geng{好大题目。
}因此叫我来拿你。
刚才你又拦住他,没的说,跟我去见太太!”\par
贾瑞听了,魂不附体,只说:“好侄儿,只说没有见我,明日我重重的谢你。
”贾蔷道:“你若谢我,放你不值什么,只不知你谢我多少?况且口说无凭,写一文契来。
”贾瑞道:“这如何落纸呢?”\geng{也知写不得。
一叹!}贾蔷道:“这也不妨,写一个赌钱输了外人账目,借头家银若干两便罢。
\zhu{
头家:聚赌抽头的人。
抽头:向赢钱的赌徒抽取一部分的利益给提供赌博场所的人。
也称为“拈头”。
聚赌抽头所得的钱叫头儿钱。
}
”贾瑞道:“这也容易。
只是此时无纸笔。
”贾蔷道:“这也容易。
”说罢,翻身出来,纸笔现成,\geng{二字妙!}\ping{凤姐调兵遣将,蓉蔷有备而来。
}拿来命贾瑞写。
他两个作好作歹,\zhu{作好作歹:做好人,又做恶人,比喻用各种方式反复劝说。
}只写了五十两银,然后画了押,贾蔷收起来。
然后撕罗贾蓉。
\zhu{撕罗:调停,解决。}
\meng{可怜至此!好事者当自度。
}贾蓉先咬定牙不依,只说:“明日告诉族中的人评评理。
”贾瑞急的至于叩头。
贾蔷做好做歹的,\meng{此是加一倍法。
}也写了一张五十两欠契才罢。
贾蔷又道:“如今要放你,我就担着不是。
\ji{又生波澜。
}老太太那边的门早已关了,老爷正在厅上看南京的东西,那一条路定难过去,如今只好走后门。
若这一走,倘或遇见了人,连我也完了。
等我们先去哨探哨探,再来领你。
这屋你还藏不得,少时就来堆东西。
等我寻个地方。
”说毕,拉着贾瑞,仍熄了灯,\ji{细。
}出至院外,摸着大台矶底下,说道:“这窝儿里好,你只蹲着,别哼一声,等我们来再动。
”\geng{未必如此收场。
}说毕,二人去了。
\par
贾瑞此时身不由己,只得蹲在那里。
心下正盘算,只听头顶上一声响,哗拉拉一净桶尿粪从上面直泼下来,
\zhu{净桶:马桶。}
可巧浇了他一头一身,贾瑞撑不住嗳哟了一声,忙又掩住口,\ji{更奇。
}不敢声张,满头满脸浑身皆是尿屎,冰冷打战。
\geng{余料必有新奇解恨文字收场,方是《石头记》笔力。
}
\geng{瑞奴实当如是报之。
}
\geng{
此一节可入《西厢记》批评内十大快中。
\zhu{
十大快:当指《西厢记·拷艳》一折前金圣叹的总批:
“昔与斫山,同客共住,霖雨十日,对床无聊,因约赌说快事,
以破积闷。“惟金圣叹所举快事计三十三则,畸笏所记或有误,或撮其要以十数概括之。
金圣叹本人亦未具体指出多少则:
“于是反自追索,犹忆得数则……”
“快”即畅快、高兴(之事)。脂评认为凤姐戏弄贾瑞为大快之事。
}
畸笏。
}\meng{这也未必不是预为埋伏者。
总是慈悲设教,遇难教者,不得不现三头六臂,并吃人心、喝人血之相,以警戒之耳。
}只见贾蔷跑来叫:“快走,快走!”贾瑞如得了命,三步两步从后门跑到家里,天已三更,只得叫门。
开门人见他这般光景,问是怎的。
少不得撒谎说:“黑了,失脚掉在茅厕里了。
”一面到自己房中更衣洗濯,心下方想到是凤姐顽他,因此发一回恨;再想想凤姐的模样儿,\geng{欲根未断。
}又恨不得一时搂在怀,一夜竟不曾合眼。
\ping{人的爱恨真是复杂,若此被玩弄还不死心,浅薄至此,对很多人来说好皮囊甚至是唯一能获得其爱意的方法。
}\par
自此满心想凤姐,\geng{此刻还不回头,真自寻死路矣。
}\meng{孙行者非有紧箍儿,虽老君之炉、五行之山,何尝屈其一二?}只不敢往荣府去了。
贾蓉两个常常的来索银子,他又怕祖父知道,正是相思尚且难禁,更又添了债务;日间工课又紧,他二十来岁之人,尚未娶亲,迩来想着凤姐,\zhu{迩:音“耳”,近。
}未免有那指头告了消乏等事;\zhu{指头告了消乏:手淫。
了音“聊”,三声。
}更兼两回冻恼奔波,\ji{写得历历病源,如何不死?}因此三五下里夹攻,\geng{所谓步步紧。
}不觉就得了一病:心内发膨胀,口内无滋味,脚下如绵,眼中似醋,黑夜作烧,白昼常倦,下溺连精,嗽痰带血。
诸如此症,不上一年,都添全了。
\geng{简洁之至!}于是不能支持,一头睡倒,合上眼还只梦魂颠倒,满口乱说胡话,惊怖异常。
百般请医治疗,诸如肉桂、附子、鳖甲、麦冬、玉竹等药,吃了有几十斤下去,也不见个动静。
\ji{说得有趣。
}\par
倏又腊尽春回,\zhu{倏:音“述”,急速。
}这病更又沉重。
代儒也着了忙,各处请医疗治,皆不见效。
因后来吃“独参汤”,\zhu{独参汤:中医方剂名。
治元气大亏、阳气暴脱的危症。
在一般方剂中,各种药配伍,人参用量不过数钱,独参汤独用人参一味,重或一、二两,取其功专而力大。
}代儒如何有这力量,只得往荣府来寻。
王夫人命凤姐秤二两给他,\ji{王夫人之慈若是。
}凤姐回说:“前儿新近都替老太太配了药,那整的太太又说留着送杨提督的太太配药,
\zhu{整的:整个的。}
偏生昨儿我已送了去了。
”王夫人道:“就是咱们这边没了,你打发个人往你婆婆那边问问,或是你珍大哥哥那府里再寻些来,凑着给人家。
吃好了,救人一命,也是你的好处。
”\ji{夹写王夫人。
}凤姐听了,也不遣人去寻,只得将些渣末泡须凑了几钱,\zhu{
渣末泡须\foot{\footPic{人参各部位的名称}{renshen.png}{1.0}}:
渣末:运参之箱底所留下的零星杂末,间亦包含一些参枝。
泡须:泡丁以及芦须,是较差的人参等级。
}命人送去,只说:\meng{“只说”。
}“太太送来的,再也没了。
\ping{王夫人的善心被凤姐歪曲成了狠心。}
”然后回王夫人说:“都寻了来,共凑了有二两多送去。
”\ji{然便有二两独参汤,贾瑞固亦不能微好,又岂能望好,但凤姐之毒何如是?终是瑞之自失。
}\ping{欺下瞒上。
}\par
那贾瑞此时要命心胜,无药不吃,只是白花钱,不见效。
忽然这日有个跛足道人\ji{自甄士隐随君一去,别来无恙否?}来化斋,口称专治冤业之症。
\zhu{冤业之症:迷信说法,由于“结冤造孽”而得的病症。
业:同“孽”,罪过、邪恶的意思。
}贾瑞偏生在内就听见了,\ping{在内能听,魂已出窍,离死不远。
}直着声叫喊\ji{如闻其声,吾不忍听也。
}说:“快请进那位菩萨来救我!”一面叫,一面在枕上叩首。
\ji{如见其形,吾不忍看也。
}众人只得带了那道士进来。
贾瑞一把拉住,连叫:“菩萨救我!”\ji{人之将死,其言也哀,作者如何下笔?}那道士叹道:“你这病非药可医!我有个宝贝与你,你天天看时,此命可保矣。
”说毕,从褡裢中\ji{妙极!此褡裢犹是士隐所抢背者乎?}
\zhu{褡裢:一种中间开口而两端装钱物的长方口袋,小的可以挂在腰带上,大的可以搭在肩膀上。}
取出一面镜子来\ji{凡看书人从此细心体贴,方许你看,否则此书哭矣。
}——两面皆可照人,\ji{此书表里皆有喻也。
}
镜把上面錾着“风月宝鉴”四字\ji{明点。
}——递与贾瑞道:“这物出自太虚幻境空灵殿上,警幻仙子所制,\ji{言此书原系空虚幻设。
}\geng{与“红楼梦”呼应。
}专治邪思妄动之症,\ji{毕真。
}有济世保生之功。
\ji{毕真。
}所以带他到世上,单与那些聪明俊杰、风雅王孙等看照。
\ji{所谓无能纨绔是也。
\zhu{“无能纨绔”指的是“风雅王孙”。}
}千万不可照正面,\ji{观者记之,不要看这书正面,方是会看。
}
\geng{谁人识得此句!}只照他的背面,\ji{记之。
}要紧,要紧!三日后吾来收取,管叫你好了。
”说毕,佯常而去,\zhu{佯:音“羊”。佯常而去:大模大样地离去。
佯常:同“扬长”。
或谓佯常亦作“常佯”、“倘佯”,意为徘徊、盘旋,自由自在地往来。
}众人苦留不住。
\par
贾瑞收了镜子,想道:“这道士倒有些意思,我何不照一照试试。
”想毕,拿起“风月鉴”来,向反面一照,只见一个骷髅立在里面,\ji{所谓“好知青冢骷髅骨,就是红楼掩面人”是也。
作者好苦心思。
}唬得贾瑞连忙掩了,骂:“道士混账,如何吓我!我倒再照照正面是什么。
”想着,又将正面一照,只见凤姐站在里面招手\geng{可怕是“招手”二字。
}叫他。
\ji{奇绝!}贾瑞心中一喜,荡悠悠的觉得进了镜子,\ji{写得奇峭,真好笔墨。
}
与凤姐云雨一番,凤姐仍送他出来。
到了床上,“嗳哟”了一声,一睁眼,镜子从手里掉过来,仍是反面立着一个骷髅。
贾瑞自觉汗津津的,底下已遗了一滩精。
\meng{此一句力如龙象,意谓:正面你方才已自领略了,你也当思想反面才是。
}心中到底不足,又翻过正面来,只见凤姐还招手叫他,他又进去。
如此三四次。
到了这次,刚要出镜子来,只见两个人走来,拿铁锁把他套住,拉了就走。
\ji{所谓醉生梦死也。
}贾瑞叫道:“让我拿了镜子再走!”\ji{可怜!大众齐来看此。
}\meng{这是作书者之立意,要写情种,故于此试一深写之。
在贾瑞则是求仁而得仁,未尝不含笑九泉,虽死后亦不解脱者,悲矣!}——只说了这句,就再不能说话了。
\par
旁边伏侍贾瑞的众人,只见他先还拿着镜子照,落下来,仍睁开眼拾在手内,末后镜子落下来便不动了。
众人上来看看,已没了气,身子底下冰凉渍湿一大滩精,这才忙着穿衣抬床。
\zhu{抬床:类似于“移床易箦”。易箦:易:更换。
箦:音“责”,竹席。
孔子弟子曾参恪守礼制,病危时一定要人换掉大夫才能寝用的华美竹席(见《礼记·檀弓上》)。
后因称人之将死为“易箦”。
依客家风俗,死者弥留断气后,家人立即将他用木板抬到正厅上,俗称“出厅下”,
并为死者抹身更衣穿鞋袜,尸席男左女右。
如在外死亡,一律放在门口,不得入室,俗谓“冷尸不得入室”。
}
代儒夫妇哭的死去活来,大骂道士,“是何妖镜!\ji{此书不免腐儒一谤。
}若不早毁此物,\ji{凡野史俱可毁,独此书不可毁。
}遗害于世不小。
”\ji{腐儒。
}遂命架火来烧,只听镜内哭道:“谁叫你们瞧正面了!你们自己以假为真,何苦来烧我?”\ji{观者记之。
}正哭着,只见那跛足道人从外跑来,喊道:“谁毁‘风月鉴’,吾来救也!”说着,直入中堂,抢入手内,飘然去了。
\par
当下,代儒料理丧事,各处去报丧。
三日起经,\zhu{起经:旧俗,人死后第三天,开始请和尚道士念经,叫起经。
}七日发引,\zhu{发引:出殡时,送丧人牵着引索作前导,把灵柩从停放的地方运出,叫发引。
引:牵引灵柩的索子。
}寄灵于铁槛寺,\ji{所谓“铁门限”是也。
先安一开路之人,以备秦氏仙柩有方也。
}日后带回原籍。
当下贾家众人齐来吊问,荣府贾赦赠银二十两,贾政亦是二十两,宁国府贾珍亦有二十两,别者族中人贫富不等,或三两五两,不可胜数。
另有各同窗家分资,也凑了二三十两。
代儒家道虽然淡薄,倒也丰丰富富完了此事。
\par
谁知这年冬底\foot{按:“冬底”,各本均同,但与上下文时间不衔接。
吴克歧假托古本作“八月底”,林冠夫理校为“五月底”,可参考。
},林如海的书信寄来,却为身染重疾,写书特来接林黛玉回去。
\meng{须要林黛玉长住,偏要暂离。
}贾母听了,未免又加忧闷,只得忙忙的打点黛玉起身。
宝玉大不自在,争奈父女之情,\zhu{争奈:怎奈。
}也不好拦劝。
于是贾母定要贾琏送他去,仍叫带回来。
一应土仪盘缠,\zhu{土仪盘缠:用土产作为赠人的礼物叫土仪。
仪:礼物。
盘缠:即盘川,旅费。
}不消烦说,自然要妥贴。
作速择了日期,贾琏与林黛玉辞别了贾母等,带领仆从,登舟往扬州去了。
要知端的,\zhu{端的:究竟、详情。
}且听下回分解。
\par
\geng{此回忽遣黛玉去者,正为下回可儿之文也。
若不遣去,只写可儿、
\zhu{可儿:指秦可卿。}
阿凤等人,却置黛玉于荣府,成何文哉?故必遣去,方好放笔写秦,方不脱节。
况黛玉乃书中正人,秦为陪客,岂因陪而失正耶?后大观园方是宝玉、宝钗、黛玉等正经文字,前皆系陪衬之文也。
}\par
\qi{总评:儒家正心,道者炼心,释辈戒心。
可见此心无有不到,无不能入者,独畏其入于邪而不反,故用心炼戒以缚之。
请看贾瑞一起念,及至于死,专诚不二,虽经两次警教,毫无反悔,可谓痴子,可谓愚情。
相乃可思,不能相而独欲思,岂逃倾颓?
\ping{单相思只能感动自己,没有好结果。}
作者以此作一新样情种,以助解者生笑,以为痴者设一棒喝耳!}
\dai{023}{贾蓉假扮凤姐戏耍贾瑞}
\dai{024}{贾天祥正照风月鉴}
\sun{p12-1}{王熙凤毒设相思局,贾天祥正照风月鉴}{凤姐见贾瑞不识好歹,佯装亲近欲设计害他。
笫一次,让贾瑞在穿堂冻了一夜。
见他邪心不改,图右侧:第二次骗他夜宿空室,被贾蓉捉住,罚银子后又被浇了一桶粪便。
如此一来,便病倒了。
百般请医疗治,吃了几十斤药下去,仍不见效。
图左侧:忽一日有个跛足道人送来“风月宝鉴”,专治邪思妄动之症。
贾瑞不听医嘱,照了镜的正面,结果命归西天,代儒夫妇哭得死去活来,大骂道士,令人架起火来烧那镜子。
这时那跛足道人忽又现身,抢了镜子,飘然走了。
}