\chapter{尴尬人难免尴尬事 \quad 鸳鸯女誓绝鸳鸯偶}
\geng{此回亦有本而笔,非泛泛之笔也。
\hang
只看他题纲用“尴尬”二字于邢夫人,可知包藏含蓄文字之中,莫能量也。
}\par
\qi{裹脚与缠头,\zhu{裹脚、缠头:代指女人。
}欲觅终身伴。
顾影自为怜,静住深深院。
好事不称心,恶语将人慢。
\zhu{慢:傲慢,不敬。
贾赦因鸳鸯不从,就发出恶言威胁。
也可以从鸳鸯的角度解读,即鸳鸯对前来说合的嫂子“恶语”相加。
}誓死守香闺,远却杨花片。
\zhu{远却杨花片:杨花指情爱,过去说女人水性杨花,意思是女人心浮意软,容易被情爱所动,改变初衷。
这里说要远却杨花,当然是说鸳鸯不嫁的决心很坚决了。
}}\par
话说林黛玉直到四更将阑,方渐渐的睡去,暂且无话。
\par
如今且说凤姐儿因见邢夫人叫他,不知何事,忙另穿戴了一番,坐车过来。
邢夫人将房内人遣出,悄向凤姐儿道:“叫你来不为别事,有一件为难的事,老爷托我,我不得主意,先和你商议。
老爷因看上了老太太的鸳鸯,要他在房里,叫我和老太太讨去。
我想这倒平常有的事,只是怕老太太不给,你可有法子?”凤姐儿听了,忙道:“依我说,竟别碰这个钉子去。
老太太离了鸳鸯,饭也吃不下去的,那里就舍得了?况且平日说起闲话来,老太太常说,老爷如今上了年纪,作什么左一个小老婆右一个小老婆放在屋里,没的耽误了人家。
放着身子不保养,官儿也不好生作去,成日家和小老婆喝酒。
太太听这话,很喜欢老爷呢?这会子回避还恐回避不及,倒拿草棍儿戳老虎的鼻子眼儿去了!太太别恼,我是不敢去的。
明放着不中用,而且反招出没意思来。
老爷如今上了年纪,行事不妥,太太该劝才是。
比不得年轻,作这些事无碍。
如今兄弟、侄儿、儿子、孙子一大群,还这么闹起来,怎样见人呢?”\ping{凤姐考虑的思路很典型,就是贾府最高领导人的好恶,其他都是次要考虑。
}邢夫人冷笑道:“大家子三房四妾的也多,偏咱们就使不得?我劝了也未必依。
就是老太太心爱的丫头,这么胡子苍白了又作了官的一个大儿子,要了作房里人,也未必好驳回的。
我叫了你来,不过商议商议,你先派上了一篇不是。
也有叫你去的理?自然是我说去。
你倒说我不劝,你还不知道那性子的,劝不成,先和我恼了。
”\par
凤姐儿知道邢夫人禀性愚犟,只知承顺贾赦以自保,次则婪聚财货为自得,家下一应大小事务,俱由贾赦摆布。
凡出入银钱事务,一经他手,便克啬异常,\zhu{克啬:刻薄、吝啬。
克,同刻。
}以贾赦浪费为名,“须得我就中俭省,方可偿补”,儿女奴仆,一人不靠,一言不听的。
\ping{邢夫人一方面娘家背景薄弱,一方面不是原配是填房,邢夫人在贾赦面前很弱势,而且促成鸳鸯进门也能在其中再捞一笔,邢夫人没什么不同意的动力。
}如今又听邢夫人如此的话,便知他又弄左性,\zhu{左性:性情固执,遇事不肯变通。
}劝了不中用,连忙陪笑说道:“太太这话说的极是。
我能活了多大,知道什么轻重?想来父母跟前,别说一个丫头,就是那么大的活宝贝,不给老爷给谁?背地里的话那里信得?我竟是个呆子。
琏二爷或有日得了不是,老爷太太恨的那样,恨不得立刻拿来一下子打死;及至见了面,也罢了,依旧拿着老爷太太心爱的东西赏他。
如今老太太待老爷,自然也是那样了。
依我说,老太太今儿喜欢,要讨今儿就讨去。
\ping{因为这个事情目前没有第三个人知道,婆婆只跟她一个人讲了,如果隔几天再要,婆婆就会怀疑是不是她把风声走漏了。}
我先过去哄着老太太发笑,等太太过去了,我搭讪着走开,把屋子里的人我也带开,太太好和老太太说的。
给了更好,不给也没妨碍,众人也不知道。
”邢夫人见他这般说,便又喜欢起来,又告诉他道:“我的主意先不和老太太要。
老太太要说不给,这事便死了。
我心里想着先悄悄的和鸳鸯说。
他虽害臊,我细细的告诉了他,他自然不言语,就妥了。
那时再和老太太说,老太太虽不依,搁不住他愿意,常言‘人去不中留’,自然这就妥了。
”\ping{直接突破贾母可能不太现实,所以先寻找容易突破的,再以既成事实促使贾母不得不承认。
}凤儿姐笑道:“到底是太太有智谋,这是千妥万妥的。
别说是鸳鸯,凭他是谁,那一个不想巴高望上,不想出头的?这半个主子不做,倒愿意做个丫头,将来配个小子就完了。
”邢夫人笑道:“正是这个话了。
别说鸳鸯,就是那些执事的大丫头,谁不愿意这样呢。
你先过去,别露一点风声,我吃了晚饭就过来。
”\par
凤姐儿暗想:“鸳鸯素习是个可恶的,虽如此说,保不严他就愿意。
我先过去了,太太后过去,若他依了便没话说;倘或不依,太太是多疑的人,只怕就疑我走了风声,使他拿腔作势的。
那时太太又见了应了我的话,羞恼变成怒,拿我出起气来,倒没意思。
不如同着一齐过去了,他依也罢,不依也罢,就疑不到我身上了。
”想毕,因笑道:“方才临来,舅母那边送了两笼子鹌鹑,我吩咐他们炸了,原要赶太太晚饭上送过来的。
我才进大门时,见小子们抬车,说太太的车拔了缝,
\zhu{
拔了缝:我国古代木工制作多采用榫卯连接各个部件。
经过长时间使用,榫卯结合可能松脱,致使部件之间出现缝隙,这一现象清代称“拔了缝”。
}
拿去收拾去了。
不如这会子坐了我的车一齐过去倒好。
”邢夫人听了,便命人来换衣服。
凤姐忙着伏侍了一回,娘儿两个坐车过来。
凤姐儿又说道:“太太过老太太那里去,我若跟了去,老太太若问起我过去作什么的,倒不好。
不如太太先去,我脱了衣裳再来。
”\ping{凤姐绝不亲口向老太太提及贾赦纳妾,甚至也不要出现在现场。
}\par
邢夫人听了有理,便自往贾母处,和贾母说了一回闲话,便出来假托往王夫人房里去,从后门出去,打鸳鸯的卧房前过。
只见鸳鸯正然坐在那里做针线,\zhu{正然:正在。
}见了邢夫人,忙站起来。
邢夫人笑道:“做什么呢?我瞧瞧,你扎的花儿越发好了。
”一面说,一面便接他手内的针线瞧了一瞧,只管赞好。
放下针线,又浑身打量。
只见他穿着半新的藕合色的绫袄,\zhu{藕合:也作“藕荷”,形容颜色浅紫而微微发红。
}青缎掐牙背心,\zhu{
掐牙:在衣服的滾边上再镶一条极细的滚边。
滚边:在衣服、布鞋等的边缘缝制的一种圆棱的边。
}下面水绿裙子。
蜂腰削背,
\zhu{
蜂腰:蜂体中部细狭的部分。比喻物体狭窄的中间部位或特别狭窄的通道;也比喻人的细腰。
削背:形容背部瘦削,体态优美。
}
鸭蛋脸面,乌油头发,高高的鼻子,两边腮上微微的几点雀斑。
鸳鸯见这般看他,自己倒不好意思起来,心里便觉诧异,因笑问道:“太太,这会子不早不晚的,过来做什么?”邢夫人使个眼色儿,跟的人退出。
邢夫人便坐下,拉着鸳鸯的手笑道:“我特来给你道喜来了。
”鸳鸯听了,心中已猜着三分,不觉红了脸,低了头不发一言。
听邢夫人道:“你知道你老爷跟前竟没有个可靠的人,\geng{说得得体。
我正想开口一句不知如何说,如此则妙极是极,如闻如见。
}心里再要买一个,又怕那些人牙子家出来的不干不净,\zhu{人牙子:即人贩子。
旧时称买卖的中间经纪人为“牙子”,即掮客。
}也不知道毛病儿,买了来家,三日两日,又要肏鬼吊猴的。
\zhu{肏鬼吊猴:捣鬼调皮,节外生枝,招惹是非。}
因满府里要挑一个家生女儿收了,又没个好的:不是模样儿不好,就是性子不好,有了这个好处,没了那个好处。
因此冷眼选了半年,这些女孩子里头,就只你是个尖儿,模样儿,行事作人,温柔可靠,一概是齐全的。
意思要和老太太讨了你去,收在屋里。
你比不得外头新买的,你这一进去了,进门就开了脸,
\zhu{开脸:旧俗女子出嫁时用线绞净脸上的汗毛,修齐鬓角,叫作“开脸”。}
就封你姨娘,又体面,又尊贵。
你又是个要强的人,俗语说的,‘金子终得金子换’,谁知竟被老爷看重了你。
如今这一来,你可遂了素日志大心高的愿了,也堵一堵那些嫌你的人的嘴。
跟了我回老太太去!”说着拉了他的手就要走。
鸳鸯红了脸,夺手不行。
邢夫人知他害臊,因又说道:“这有什么臊处?你又不用说话,只跟着我就是了。
”鸳鸯只低了头不动身。
邢夫人见他这般,便又说道:“难道你不愿意不成?若果然不愿意,可真是个傻丫头了。
放着主子奶奶不作,倒愿意作丫头!三年二年,不过配上个小子,还是奴才。
你跟了我们去,你知道我的性子又好,又不是那不容人的人。
老爷待你们又好。
过一年半载,生下个一男半女,你就和我并肩了。
家里的人你要使唤谁,谁还不动?现成主子不做去,错过这个机会,后悔就迟了。
”鸳鸯只管低了头,仍是不语。
邢夫人又道:“你这么个响快人,怎么又这样积粘起来?\zhu{积粘:同滞粘。
扭扭捏捏,不干脆,不爽快。
}
有什么不称心之处,只管说与我,我管你遂心如意就是了。
”鸳鸯仍不语。
邢夫人又笑道:“想必你有老子娘,你自己不肯说话,怕臊。
你等他们问你,这也是理。
让我问他们去,叫他们来问你,有话只管告诉他们。
”说毕,便往凤姐儿房中来。
\ping{邢夫人在鸳鸯那里碰了钉子,所以寻找更容易的突破口,即鸳鸯的家人。
}\par
凤姐儿早换了衣服,因房内无人,便将此话告诉了平儿。
平儿也摇头笑道:“据我看,此事未必妥。
平常我们背着人说起话来,听他那主意,未必是肯的。
也只说着瞧罢了。
”凤姐儿道:“太太必来这屋里商议。
依了还可,若不依,白讨个臊,当着你们,岂不脸上不好看。
你说给他们炸鹌鹑,再有什么配几样,预备吃饭。
你且别处逛逛去,估量着去了再来。
”平儿听说,照样传给婆子们,便逍遥自在的往园子里来。
\par
这里鸳鸯见邢夫人去了,必在凤姐儿房里商议去了,必定有人来问他的,不如躲了这里,\geng{终不免女儿气,不知躲在那里方无人来罗唣,\zhu{罗唣:即“啰唣”,音“罗造”,骚扰,吵闹。
}写得可怜可爱。
}因找了琥珀说道:“老太太要问我,只说我病了,没吃早饭,往园子里逛逛就来。
”琥珀答应了。
鸳鸯也往园子里来,各处游玩,不想正遇见平儿。
平儿因见无人,便笑道:“新姨娘来了!”鸳鸯听了,便红了脸,说道:“怪道你们串通一气来算计我!等着我和你主子闹去就是了。
”平儿听了,自悔失言,便拉他到枫树底下,\geng{随笔带出妙景,正愁园中草木黄落,不想看此一句,便恍如置身于千霞万锦、绛雪红霜之中矣。
}坐在一块石上,越性把方才凤姐过去回来所有的形景言词、始末原由告诉与他。
鸳鸯红了脸,向平儿冷笑道:“这是咱们好,比如袭人、琥珀、素云、紫鹃、彩霞、玉钏儿、麝月、翠墨,跟了史姑娘去的翠缕,死了的可人和金钏,去了的茜雪,\geng{余按此一算,亦是十二钗,真镜中花,水中月,云中豹,林中之鸟,穴中之鼠,无数可考,无人可指,有迹可追,有形可据,九曲八折,远响近影、迷离烟灼,纵横隐现,千奇百怪,眩目移神,现千手千眼大游戏法也。
脂砚斋。
}连上你我,这十来个人,从小儿什么话儿不说?什么事儿不作?这如今因都大了,各自干各自的去了,\geng{此语已可伤,犹未“各自干各自去”,后日更有各自之处也,知之乎!}然我心里仍是照旧,有话有事,并不瞒你们。
这话我且放在你心里,且别和二奶奶说:别说大老爷要我做小老婆,就是太太这会子死了,他三媒六聘的娶我去作大老婆,我也不能去。
”\par
平儿方欲笑答,只听山石背后哈哈的笑道:“好个没脸的丫头,亏你不怕牙碜。
”\zhu{牙碜[yáchen]:
食物中夹杂砂石,咀嚼起来硌牙,皮肤起栗,叫牙碜。
这里引伸为说肉麻话,令人难受。
}二人听了不免吃了一惊,忙起身向山石背后找寻,不是别个,却是袭人笑着走了出来问:“什么事情?告诉我。
”说着,三人坐在石上。
平儿又把方才的话说与袭人听,袭人道:“真真这话论理不该我们说,这个大老爷太好色了,略平头正脸的,他就不放手了。
”平儿道:“你既不愿意,我教你个法子,不用费事就完了。
”鸳鸯道:“什么法子?你说来我听。
”平儿笑道:“你只和老太太说,就说已经给了琏二爷了,大老爷就不好要了。
”鸳鸯啐道:“什么东西!你还说呢!前儿你主子不是这么混说的?谁知应到今儿了!”\zhu{第三十八回,凤姐开玩笑说贾琏要讨鸳鸯做小老婆。
}袭人笑道:“他们两个都不愿意,我就和老太太说,叫老太太说把你已经许了宝玉了,大老爷也就死了心了。
”鸳鸯又是气,又是臊,又是急,因骂道:“两个蹄子不得好死的!人家有为难的事,拿着你们当正经人,告诉你们与我排解排解,你们倒替换着取笑儿。
你们自为都有了结果了,将来都是做姨娘的。
据我看,天下的事未必都遂心如意。
你们且收着些儿,别忒乐过了头儿!”二人见他急了,忙陪笑央告道:“好姐姐,别多心,咱们从小儿都是亲姊妹一般,不过无人处偶然取个笑儿。
你的主意告诉我们知道,也好放心。
”鸳鸯道:“什么主意!我只不去就完了。
”平儿摇头道:“你不去未必得干休。
大老爷的性子你是知道的。
虽然你是老太太房里的人,此刻不敢把你怎么样,将来难道你跟老太太一辈子不成?也要出去的。
那时落了他的手,倒不好了。
”鸳鸯冷笑道:“老太太在一日,我一日不离这里;若是老太太归西去了,他横竖还有三年的孝呢,没个娘才死了他先纳小老婆的!等过三年,知道又是怎么个光景,那时再说。
纵到了至急为难,我剪了头发作姑子去;不然,还有一死。
一辈子不嫁男人,又怎么样?乐得干净呢!”平儿袭人笑道:“真这蹄子没了脸,越发信口儿都说出来了。
”鸳鸯道:“事到如此,臊一会怎么样!你们不信,慢慢的看着就是了。
太太才说了,找我老子娘去。
我看他南京找去!”平儿道:“你的父母都在南京看房子,\ping{贾府旧宅?可能暗示了作品原型真实发生在南京,贾府对应着把持江宁织造的曹家。
}没上来,终久也寻的着。
现在还有你哥哥嫂子在这里。
可惜你是这里的家生女儿,不如我们两个人是单在这里。
”鸳鸯道:“家生女儿怎么样?‘牛不吃水强按头’?我不愿意,难道杀我的老子娘不成?”\par
正说着,只见他嫂子从那边走来。
袭人道:“当时找不着你的爹娘,一定和你嫂子说了。
”鸳鸯道:“这个娼妇专管是个‘九国贩骆驼的’,\zhu{九国贩骆驼的:比喻巧言善辩、钻营图利的人。
亦作“六国贩骆驼的”。
}
听了这话,他有个不奉承去的!”说话之间,已来到跟前。
他嫂子笑道:“那里没找到,姑娘跑了这里来!你跟了我来,我和你说话。
”平儿袭人都忙让坐。
他嫂子说:“姑娘们请坐,我找我们姑娘说句话。
”袭人平儿都装不知道,笑道:“什么话这样忙?我们这里猜谜儿赢手批子打呢,\zhu{赢手批子打:赢家打输家的手心。
}等猜了这个再去。
”鸳鸯道:“什么话?你说罢。
”他嫂子笑道:“你跟我来,到那里我告诉你,横竖有好话儿。
”鸳鸯道:“可是大太太和你说的那话?”他嫂子笑道:“姑娘既知道,还奈何我!
\zhu{奈何:惩治,对付。}
快来,我细细的告诉你可是天大的喜事。
”鸳鸯听说,立起身来,照他嫂子脸上下死劲啐了一口,指着他骂道:“你快夹着屄嘴离了这里,好多着呢!什么‘好话’!宋徽宗的鹰,赵子昂的马,都是好画儿。
\zhu{宋徽宗的鹰,赵子昂的马,都是好画儿:歇后语,意即“都是好话儿”。
“画儿”与“话儿”谐音。
宋徽宗的鹰:宋徽宗赵估,工于花鸟,以画鹰著称。
赵子昂的马:赵孟頫字子昂,元代书画家,擅长画马。
}什么‘喜事’!状元痘儿灌的浆儿又满是喜事。
\zhu{状元痘儿灌的浆儿又满是喜事:歇后语,意即“喜事”。
状元痘,是天花痘疹的讳称。
痘疹发出灌浆饱满,生命即可保无虞,故称“喜事”。
这里是对“天大喜事”一语的嘲弄。
}怪道成日家羡慕人家女儿作了小老婆了,一家子都仗着他横行霸道的,一家子都成了小老婆了!看的眼热了,也把我送在火坑里去。
我若得脸呢,你们外头横行霸道,自己就封自己是舅爷了。
我若不得脸败了时,你们把忘八脖子一缩,生死由我。
”\ping{元妃这个贵妃也不过是帝王家的小老婆,而尊贵如贾府,放在皇家面前也不过是赵姨娘的兄弟罢了。
这里可能也是借着鸳鸯的口说出了贾元春的心声,自己成为了家人追求富贵的牺牲品。
}一面说,一面哭,平儿袭人拦着劝。
他嫂子脸上下不来,因说道:“愿意不愿意,你也好说,不犯着牵三挂四的。
俗语说,‘当着矮人,别说矮话’。
姑奶奶骂我,我不敢还言;这二位姑娘并没惹着你,小老婆长小老婆短,大家脸上怎么过得去?”袭人平儿忙道:“你倒别这么说,他也并不是说我们,你倒别牵三挂四的。
你听见那位太太、太爷们封我们做小老婆?况且我们两个也没有爹娘、哥哥兄弟在这门子里仗着我们横行霸道的。
他骂的人自有他骂的,我们犯不着多心。
”鸳鸯道:“他见我骂了他,他臊了,没的盖脸,\zhu{盖脸:北京一带的方言,遮盖,掩饰羞容。
}又拿话挑唆你们两个,幸亏你们两个明白。
原是我急了,也没分别出来,他就挑出这个空儿来。
”他嫂子自觉没趣,赌气去了。
鸳鸯气得还骂,平儿袭人劝他一回,方才罢了。
\par
平儿因问袭人道:“你在那里藏着做甚么的?我们竟没看见你。
”袭人道:“我因为往四姑娘房里瞧我们宝二爷去的,谁知迟了一步,说是来家里来了。
我疑惑怎么不遇见呢,想要往林姑娘家里找去,又遇见他的人说也没去。
我这里正疑惑是出园子去了,可巧你从那里来了,我一闪,你也没看见。
后来他又来了。
我从这树后头走到山子石后,我却见你两个说话来了,谁知你们四个眼睛没见我。
”\par
一语未了,又听身后笑道:“四个眼睛没见你?你们六个眼睛竟没见我!”三人唬了一跳,回身一看,不是别个,正是宝玉走来。
\geng{通部情案,皆必从石兄挂号,然各有各稿,穿插神妙。
}袭人先笑道:“叫我好找,你那里来?”宝玉笑道:“我从四妹妹那里出来,迎头看见你来了,我就知道是找我去的,我就藏了起来哄你。
看你低着头过去了,进了院子就出来了,逢人就问。
我在那里好笑,只等你到了跟前唬你一跳的,后来见你也藏藏躲躲的,我就知道也是要哄人了。
我探头往前看了一看,却是他两个,所以我就绕到你身后。
你出去,我就躲在你躲的那里了。
”平儿笑道:“咱们再往后找找去,只怕还找出两个人来也未可知。
”宝玉笑道:“这可再没了。
”鸳鸯已知话俱被宝玉听了,只伏在石头上装睡。
宝玉推他笑道:“这石头上冷,咱们回房里去睡,岂不好?”说着拉起鸳鸯来,又忙让平儿来家坐吃茶。
平儿和袭人都劝鸳鸯走,鸳鸯方立起身来,四人竟往怡红院来。
宝玉将方才的话俱已听见,心中自然不快,只默默的歪在床上,任他三人在外间说笑。
\par
外边邢夫人因问凤姐儿鸳鸯的父母,凤姐因回说:“他爹的名字叫金彩,\geng{姓金名彩,由“鸳鸯”二字化出,因文而生文也。
}两口子都在南京看房子,从不大上京。
他哥哥金文翔,\geng{更妙!}现在是老太太那边的买办。
他嫂子也是老太太那边浆洗的头儿。
”\geng{只鸳鸯一家,写得荣府中人各有各职,如目已睹。
}邢夫人便令人叫了他嫂子金文翔媳妇来,细细说与他。
金家媳妇自是喜欢,兴兴头头找鸳鸯,只望一说必妥,不想被鸳鸯抢白一顿,又被袭人平儿说了几句,羞恼回来,便对邢夫人说:“不中用,他倒骂了我一场。
”因凤姐儿在旁,不敢提平儿,只说:“袭人也帮着他抢白我,也说了许多不知好歹的话,回不得主子的。
太太和老爷商议再买罢。
谅那小蹄子也没有这么大福,我们也没有这么大造化。
”邢夫人听了,因说道:“又与袭人什么相干?他们如何知道的?”又问:“还有谁在跟前?”金家的道:“还有平姑娘。
”凤姐儿忙道:“你不该拿嘴巴子打他回来?我一出了门,他就逛去了;回家来连一个影儿也摸不着他!他必定也帮着说什么呢!”金家的道:“平姑娘没在跟前,远远的看着倒像是他,可也不真切,不过是我白忖度。
”凤姐便命人去:“快打了他来,告诉他我来家了,太太也在这里,请他来帮个忙儿。
”丰儿忙上来回道:“林姑娘打发了人下请字请了三四次,他才去了。
\ping{丰儿逢场作戏,配合凤姐。}
奶奶一进门我就叫他去的。
林姑娘说:‘告诉你奶奶,我烦他有事呢。
’”凤姐儿听了方罢,故意的还说:“天天烦他,有些什么事!”\par
邢夫人无计,吃了饭回家,晚间告诉了贾赦。
贾赦想了一想,即刻叫贾琏来说:“南京的房子还有人看着,不止一家,即刻叫上金彩来。
”贾琏回道:“上次南京信来,金彩已经得了痰迷心窍,那边连棺材银子都赏了,不知如今是死是活,便是活着,人事不知,叫来也无用。
他老婆子又是个聋子。
”贾赦听了,喝了一声,又骂:“下流囚攮的,
\zhu{
攮:骂人糊涂愚笨。例如“狗攮的”。 
囚攮的:骂人的话。意指囚犯的子女。
}
偏你这么知道,还不离了我这里!”唬得贾琏退出,一时又叫传金文翔。
贾琏在外书房伺候着,又不敢家去,又不敢见他父亲,只得听着。
一时金文翔来了,小幺儿们直带入二门里去,隔了五六顿饭的工夫才出来去了。
贾琏暂且不敢打听,隔了一会,又打听贾赦睡了,方才过来。
至晚间凤姐儿告诉他,方才明白。
\par
鸳鸯一夜没睡,至次日,他哥哥回贾母接他家去逛逛,贾母允了,命他出去。
鸳鸯意欲不去,只怕贾母疑心,只得勉强出来。
他哥哥只得将贾赦的话说与他,又许他怎么体面,又怎么当家作姨娘。
鸳鸯只咬定牙不愿意。
他哥哥无法,少不得去回覆了贾赦。
贾赦怒起来,因说道:“我这话告诉你,叫你女人向他说去,就说我的话:‘自古嫦娥爱少年’,他必定嫌我老了,大约他恋着少爷们,多半是看上了宝玉,只怕也有贾琏。
果有此心,叫他早早歇了心,我要他不来,此后谁还敢收?此是一件。
第二件,想着老太太疼他,将来自然往外聘作正头夫妻去。
叫他细想,凭他嫁到谁家去,也难出我的手心。
除非他死了,或是终身不嫁男人,我就伏了他!若不然时,叫他趁早回心转意,有多少好处。
”贾赦说一句,金文翔应一声“是”。
贾赦道:“你别哄我,我明儿还打发你太太过去问鸳鸯,你们说了,他不依,便没你们的不是。
若问他,他再依了,仔细你的脑袋!”\zhu{若问他,他再依了:如果你们不去劝鸳鸯,或者谎称鸳鸯不愿意,最后发现鸳鸯本人其实愿意。
}\par
金文翔忙应了又应,退出回家,也不等得告诉他女人转说,竟自己对面说了这话。
把个鸳鸯气的无话可回,想了一想,便说道:“便愿意去,也须得你们带了我回声老太太去。
”他哥嫂听了,只当回想过来,都喜之不胜。
他嫂子即刻带了他上来见贾母。
\par
可巧王夫人、薛姨妈、李纨、凤姐儿、宝钗等姊妹并外头的几个执事有头脸的媳妇,都在贾母跟前凑趣儿呢。
鸳鸯喜之不尽,拉了他嫂子,到贾母跟前跪下,一行哭,一行说,把邢夫人怎么来说,园子里他嫂子又如何说,今儿他哥哥又如何说,“因为不依,方才大老爷越性说我恋着宝玉,不然要等着往外聘,我到天上,这一辈子也跳不出他的手心去,终久要报仇。
我是横了心的,当着众人在这里,我这一辈子莫说是‘宝玉’,便是‘宝金’‘宝银’‘宝天王’‘宝皇帝’,横竖不嫁人就完了!就是老太太逼着我,我一刀子抹死了,也不能从命!若有造化,我死在老太太之先;若没造化,该讨吃的命,\zhu{讨吃:讨饭,要饭。
}伏侍老太太归了西,我也不跟着我老子娘哥哥去,我或是寻死,或是剪了头发当尼姑去!若说我不是真心,暂且拿话来支吾,日后再图别的,天地鬼神,日头月亮照着嗓子,从嗓子里头长疔烂了出来,
\zhu{疔:音“丁”,一种毒疮,形小根深,坚硬如钉。}
烂化成酱在这里!”原来他一进来时,便袖了一把剪子,一面说着,一面左手打开头发,右手便铰。
众婆娘丫鬟忙来拉住,已剪下半绺来了。
众人看时,幸而他的头发极多,铰的不透,连忙替他挽上。
贾母听了,气的浑身乱战,口内只说:“我通共剩了这么一个可靠的人,他们还要来算计!”因见王夫人在旁,便向王夫人道:“你们原来都是哄我的!外头孝敬,暗地里盘算我。
有好东西也来要,有好人也要,剩了这么个毛丫头,见我待他好了,你们自然气不过,弄开了他,好摆弄我!”王夫人忙站起来,不敢还一言。
\geng{千奇百怪,王夫人亦有罪乎?老人家迁怒之言必应如此。
}\ping{贾母对王夫人也有旧账可算。
袭人本是从贾母那里借调到宝玉那里的丫鬟,人事关系归于贾母。
王夫人在第三十四回和袭人密谈之后,在第三十六回给袭人提高了待遇,从自己的月钱里给袭人发钱,并把袭人从贾母那里除名,虽然给贾母补了个丫头使,但是连招呼都没打(“又叫他与王夫人叩头,且不必去见贾母”)。
王夫人暗地挖走袭人,和邢夫人企图暗地挖走鸳鸯,事情性质是一样的,都是擅动贾母自己身边的丫鬟,分明是两个儿媳妇对婆婆贾母的藐视和宣战。
袭人被挖走,当时没有明写贾母的反应,但是可以推测贾母心里暗藏的对王夫人的不满,邢夫人要挖鸳鸯成为了贾母爆发的导火索:大儿媳的非分之举,促使贾母联想到二儿媳的非分之举,所以贾母会以此事为契机对王夫人发火算旧账。
}薛姨妈见连王夫人怪上,反不好劝的了。
李纨一听见鸳鸯的话,早带了姊妹们出去。
\par
探春有心的人,想王夫人虽有委曲,如何敢辩;薛姨妈也是亲姊妹,自然也不好辩的;宝钗也不便为姨母辩;李纨、凤姐、宝玉一概不敢辩;这正用着女孩儿之时,迎春老实,惜春小,因此窗外听了一听,便走进来陪笑向贾母道:“这事与太太什么相干?老太太想一想,也有大伯子要收屋里的人,小婶子如何知道?便知道,也推不知道。
”犹未说完,贾母笑道:“可是我老糊涂了!姨太太别笑话我。
你这个姐姐他极孝顺我,不像我那大太太一味怕老爷,婆婆跟前不过应景儿。
可是委屈了他。
”薛姨妈只答应“是”,又说:“老太太偏心,多疼小儿子媳妇,也是有的。
”贾母道:“不偏心!”因又说道:“宝玉,我错怪了你娘,你怎么也不提我,看着你娘受委屈?”宝玉笑道:“我偏着娘说大爷大娘不成?通共一个不是,我娘在这里不认,却推谁去?我倒要认是我的不是,老太太又不信。
”贾母笑道:“这也有理。
你快给你娘跪下,你说太太别委屈了,老太太有年纪了,看着宝玉罢。
”宝玉听了,忙走过去,便跪下要说;王夫人忙笑着拉他起来,说:“快起来,快起来,断乎使不得。
终不成你替老太太给我赔不是不成?”宝玉听说,忙站起来。
\geng{宝玉亦有罪了。
}贾母又笑道:“凤姐儿也不提我。
”\geng{阿凤也有了罪。
}
\geng{奇奇怪怪之文,所谓《石头记》不是作出来的。
}凤姐儿笑道:“我倒不派老太太的不是,老太太倒寻上我了?”贾母听了,与众人都笑道:“这可奇了!倒要听听这不是。
”凤姐儿道:“谁教老太太会调理人,调理的水葱儿似的,怎么怨得人要?我幸亏是孙子媳妇,若是孙子,我早要了,还等到这会子呢。
”贾母笑道:“这倒是我的不是了?”凤姐儿笑道:“自然是老太太的不是了。
”贾母笑道:“这样,我也不要了,你带了去罢!”凤姐儿道:“等着修了这辈子,来生托生男人,我再要罢。
”贾母笑道:“你带了去,给琏儿放在屋里,看你那没脸的公公还要不要了!”凤姐儿道:“琏儿不配,就只配我和平儿这一对烧糊了的卷子和他混罢。
”\zhu{烧糊了的卷子:喻貌丑。
糊:烤焦。
卷子:即蒸卷,一种面食。
}说的众人都笑起来了。
\par
丫鬟回说:“大太太来了。
”王夫人忙迎了出去。
要知端的——\par
\qi{总评:鸳鸯女从热闹中别具一副肠胃,“不轻许人”一事,是宦途中药石仙方。
\zhu{这条批语的意思是,以鸳鸯不轻易许诺嫁人,联系到官场上的“药石仙方”即处事法则,也是不要轻易给人明确的承诺。
}}
\dai{091}{嫂子劝鸳鸯,鸳鸯骂嫂子}
\dai{092}{鸳鸯割发誓不嫁人}
\sun{p46-1}{嫂子苦劝鸳鸯,鸳鸯誓不嫁人}{图右侧:贾赦欲纳鸳鸯为妾,让邢夫人出面说亲。
鸳鸯不从, 正与平儿袭人诉苦,见其嫂前来道喜,怨怒忿起,照她嫂子脸上下死劲啐了一口,指着鼻子痛骂。
藏于假山后的宝玉见状,便领其到怡红院吃茶消气。
图左侧:鸳鸯假意同意,和她嫂子来见贾母。
鸳鸯哭诉原委,誓死不嫁。
贾母听了, 气得浑身乱颤。
李纨一听见鸳鸯的话,早带了姊妹们出去。
}