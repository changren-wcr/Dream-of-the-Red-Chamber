\chapter{王熙凤正言弹妒意\quad 林黛玉俏语谑娇音}
\qi{智慧生魔多象,魔生智慧方深。
智慧寂灭万缘根,不解智魔作甚。
\hang
\zhu{
智慧:梵语音译,也作“般若”[bōrě],佛教谓超越世俗虚幻的认识,达到把握真理的能力。
魔:“魔罗”的略称。意为“扰乱”、“破坏”、“障碍”等。古印度神话传说中,魔王魔波旬常率魔众进行破坏善事的活动。
后佛教采用其说,将能扰乱身心、破坏好事、障碍善法者称之为魔。魔指障碍善法的人和事,
也包括一切烦恼、疑惑、迷恋等妨碍修行的心理活动。
受此影响,中国民间常将宗教及迷信中的妖孽、鬼怪等与之联系起来,有“妖魔鬼怪”、“魔鬼”等说法,并以此形容坏人恶人等。
这首诗令人费解,可能通过智慧和魔相反相成的辩证关系,体现出作者对于秉正邪两气所生的人蕴含的复杂性的感叹。
第二十回写宝玉的奶妈骂袭人,贾环赌输了赖钱,王熙凤弹压赵姨娘,宝玉和黛玉、湘云、宝钗之间的微妙感情纠缠等,
其中有爱有恨,有天真有计较,有光明正大有阴暗龌龊,难以非黑即白的做出善恶好恶判断。
}
}\par
话说宝玉在林黛玉房中说“耗子精”,宝钗撞来,讽刺宝玉元宵不知“绿蜡”之典,三人正在房中互相讥刺取笑。
那宝玉正恐黛玉饭后贪眠,一时存了食,或夜间走了困,皆非保养身体之法;\ji{云宝玉亦知医理,却只是在颦、钗等人前方露,亦如后回许多明理之语,只在闺前现露三分,越在雨村等经济人前如痴如呆,实令人可恨。
但雨村等视宝玉不是人物,岂知宝玉视彼等更不是人物,故不与接谈也。
宝玉之情痴,真乎?假乎?看官细评。
}幸而宝钗走来,大家谈笑,那林黛玉方不欲睡,自己才放了心。
忽听他房中嚷起来,大家侧耳听了一听,林黛玉先笑道:“这是你妈妈和袭人叫嚷呢。
那袭人也罢了,你妈妈再要认真排场他,\zhu{排场:在这里义同下文的“排揎”(揎音“宣”),数落、责难的意思。
}可见老背晦了。
”\ji{袭卿能使颦卿一赞,愈见彼之为人矣,观者诸公以为如何?}\par
宝玉忙要赶过来,宝钗忙一把拉住道:\geng{的是宝钗行事。
}“你别和你妈妈吵才是,他老糊涂了,倒要让他一步为是。
”\ji{宝钗如何?观者思之。
}宝玉道:“我知道了。
”说毕走来,只见李嬷嬷拄着拐棍,在当地骂袭人:\geng{活像过时奶妈骂丫头。
}“忘了本的小娼妇!\geng{在袭卿身上却叫下撞天屈来。
}我抬举起你来,这会子我来了,你大模大样的躺在炕上,见我来也不理一理。
一心只想妆狐媚子哄宝玉,\geng{看这句几把批书人吓杀了。
}哄的宝玉不理我,听你们的话。
\geng{幸有此二句,不然我石兄袭卿扫地矣。
}\ping{李奶妈骂袭人“小娼妇”、“狐媚子”,似乎要泄露怡红院偷试云雨之事,还好最后说出骂人的理由,原来是宝玉不吃奶之后,宝玉天天和姑娘们厮混,忽视了奶妈,导致奶妈心生妒意怨愤。
}你不过是几两臭银子买来的毛丫头,这屋里你就作耗,\zhu{作耗:捣乱生事。
}如何使得!好不好拉出去配一个小子,\geng{虽写得酷肖,然唐突我袭卿,实难为情。
}
\ping{“配一个小子”暗示袭人最终婚姻结局。}
看你还妖精似的哄宝玉不哄!”\geng{若知“好事多魔”,
\zhu{
好事多魔:喜庆美好的事,往往要经过很多波折才能如愿。常用来指男女佳期不顺。
也作「好事多磨」、「好事多妨」。
}
方会作者这意。
}袭人先只道李嬷嬷不过为他躺着生气,少不得分辨说“病了,才出汗,蒙着头,原没看见你老人家”等语。
后来只管听他说“哄宝玉”、“妆狐媚”,又说“配小子”等,由不得又愧又委屈,禁不住哭起来。
\par
宝玉虽听了这些话,也不好怎样,少不得替袭人分辨病了吃药等话,又说:“你不信,只问别的丫头们。
”李嬷嬷听了这话,益发气起来了,说道:“你只护着那起狐狸,那里认得我了!叫我问谁去?\geng{真有是语。
}谁不帮着你呢,\geng{真有是事。
}谁不是袭人拿下马来的!\zhu{拿下马:降伏。
}\geng{冤枉冤哉!}
我都知道那些事。
\geng{囫囵语,难解。
}我只和你在老太太、太太跟前去讲了。
\ping{云山雾绕,不知是否是说宝玉和袭人偷试云雨之事,还要报告老太太、太太,让人捏一把汗。
}把你奶了这么大,\geng{奶妈拿手话。
}到如今吃不着奶了,把我丢在一旁,逞着丫头们要我的强。
”\geng{特为乳母传照,暗伏后文倚势奶娘线脉。
\zhu{第七十三回,迎春奶娘用迎春的金丝凤典当,聚赌犯事,还求迎春去说情。}
《石头记》无闲文并虚字在此。
壬午孟夏。
畸笏老人。
}一面说,一面也哭起来。
彼时黛玉、宝钗等也走过来劝说:“妈妈你老人家担待他们一点子就完了。
”李嬷嬷见他二人\geng{四字,嬷嬷是看重二人身份。
}来了,便拉住诉委屈,将当日吃茶,茜雪出去,与昨日酥酪等事,唠唠叨叨说个不清。
\geng{好极,妙极,毕肖极!}\geng{茜雪至“狱神庙”方呈正文。
袭人正文标目曰“花袭人有始有终”,余只见有一次誊清时,与“狱神庙慰宝玉”等五六稿,被借阅者迷失,叹叹!丁亥夏。
畸笏叟。
\zhu{这条批语暗示了佚失后文的诸多情节。}
}\par
可巧凤姐正在上房算完输赢账,听得后面声嚷动,便知是李嬷嬷老病发了,排揎宝玉的人。
\zhu{排揎(揎音“宣”):数落、责难的意思。}
——正值他今儿输了钱,\geng{找上文。
\zhu{
耍牌赌博是贾府常事。
第七回:尤氏、凤姐、秦氏等抹骨牌;
第十九回:宝玉的小厮们偷空有去会赌的;宝玉出了门后房中的丫鬟们也有掷骰抹牌的;
本回后文:贾母要和老管家嬷嬷斗牌解闷;晴雯等丫鬟赌博。
}
}迁怒于人。
\geng{有是争竞事。
}便连忙赶过来,拉了李嬷嬷,笑道:“好妈妈,别生气。
大节下老太太才喜欢了一日,你是个老人家,别人高声,你还要管他们呢,难道你反不知道规矩,在这里嚷起来,叫老太太生气不成?\geng{阿凤两提“老太太”,是叫老妪想袭卿是老太太的人,况又双关大体,勿泛泛看去。
}你只说谁不好,我替你打他。
我家里烧的滚热的野鸡,快来跟我吃酒去。
”\geng{何等现成,何等自然,的是凤卿笔法。
}一面说,一面拉着走,又叫:“丰儿,替你李奶奶拿着拐棍子,擦眼泪的手帕子。
”\geng{一丝不漏。
}那李嬷嬷脚不沾地跟了凤姐走了,一面还说:“我也不要这老命了,越性今儿没了规矩,闹一场子,讨个没脸,强如受那娼妇蹄子的气!”\zhu{蹄子:骂女人的话。
}后面宝钗、黛玉随着,见凤姐儿这般,都拍手笑道:“亏这一阵风来,把个老婆子撮了去了。
”\geng{批书人也是这样说。
看官将一部书中人一一想来,收拾文字非阿凤俱有琐细引迹事。
\zhu{
“引迹”一词或是“隐迹”。整句的意思是“收拾文字非阿凤俱,有琐细隐迹事”。
“琐细隐迹事”应是指文中用幻笔隐含真事,具体何真事待考。
}
《石头记》得力处俱在此。
}\ping{王熙凤正言弹妒意之一:李奶妈对得宠丫鬟的嫉妒。
}\par
宝玉点头叹道:“这又不知是那里的帐,只拣软的排揎。
昨儿又不知是那个姑娘得罪了,上在他帐上。
”一句未了,晴雯在旁笑道:“谁又不疯了,得罪他作什么。
便得罪了他,就有本事承任,不犯着带累别人!”\zhu{不犯:犯不着,不值得。
}袭人一面哭,一面拉宝玉道:“为我得罪了一个老奶奶,你这会子又为我得罪这些人,这还不够我受的,还只是拉别人。
”宝玉见他这般病势,又添了这些烦恼,连忙忍气吞声,安慰他仍旧睡下出汗。
又见他汤烧火热,自己守着他,歪在旁边,劝他只养着病,别想着些没要紧的事生气。
袭人冷笑道:“要为这些事生气,这屋里一刻还站不得了。
\geng{实言,非谬语也。
}但只是天长日久,只管这样,可叫人怎么样才好呢?时常我劝你,别为我们得罪人,你只顾一时为我们那样,他们都记在心里,遇着坎儿,\zhu{坎儿:地上的坡埂,走路时易绊。
遇着坎儿,喻碰在当口上。
这里应该是指第七十四回,王善保家因对丫鬟们不满,向王夫人进谗言,导致王夫人对怡红院的丫鬟进行了大清洗。
}说的好说不好听,大家什么意思。
”\geng{从“狐媚子”等语来,实实好语,的是袭卿。
}\ping{李奶妈早就不在怡红院当差了,对于袭人和宝玉偷试云雨这种非常私密的事情,应该不太可能知道。
袭人如果得罪了她,最坏的情况是,李奶妈去向贾母王夫人告状,说自己受到了冷落,结果不会太糟糕。
但是袭人如果得罪了同为丫鬟而知道底细的晴雯等人,把她们逼急了很可能去举报自己偷试云雨之事,结果很可能让自己扫地出门。
}一面说,一面禁不住流泪,又怕宝玉烦恼,只得又勉强忍着。
\geng{一段特为怡红袭人、晴雯、茜雪三鬟之性情见识身份而写。
己卯冬夜。
}\par
一时杂使的老婆子煎了二和药来。
\zhu{二和(和音“获”)药:即二煎药,指煎了第二次的中药汤剂。
}宝玉见他才有汗意,不肯叫他起来,自己便端着就枕与他吃了,即令小丫头子们铺炕。
袭人道:“你吃饭不吃饭,到底老太太、太太跟前坐一会子,\geng{心中时时刻刻正意语也。
}
和姑娘们顽一会子再回来。
我就静静的躺一躺也好。
”宝玉听说,只得替他去了簪环,看他躺下,自往上房来。
同贾母吃毕饭,贾母犹欲同那几个老管家嬷嬷斗牌解闷,宝玉记着袭人,便回至房中,见袭人朦朦睡去。
自己要睡,天气尚早。
彼时晴雯、绮霰、秋纹、碧痕都寻热闹,找鸳鸯、琥珀等耍戏去了,独见麝月一个人在外间房里灯下抹骨牌。
宝玉笑问道:“你怎么不同他们顽去?”麝月道:“没有钱。
”宝玉道:“床底下堆着那么些,还不够你输的?”麝月道:“都顽去了,这屋里交给谁呢?\geng{正文。
}那一个又病了。
满屋里上头是灯,地下是火。
\geng{灯节。
}
那些老妈妈子们,老天拔地,
\zhu{老天拔地:形容老年人的行动不灵活。}
伏侍一天,也该叫他们歇歇,小丫头子们也是伏侍了一天,这会子还不叫他们顽顽去。
所以让他们都去罢,我在这里看着。
”\geng{麝月闲闲无语,令余酸鼻,正所谓对景伤情。
丁亥夏。
畸笏。
}\par
宝玉听了这话,公然又是一个袭人。
\geng{岂敢。
}因笑道:“我在这里坐着,你放心去罢。
”\geng{每于如此等处,石兄何尝轻轻放过不介意来?亦作者欲瞒看官,又被批书人看出,呵呵。
\zhu{
石兄:本文以青埂峰下化为玉的顽石到人间风流繁华之地的所见所闻为内容。
“轻轻放过不介意”指的是如果麝月听宝玉的话去了,那么就放过了后面一段好看的篦头文字,这令人遗憾感到介意。
作者欲说还休,先让读者以后麝月要去了,使得读者产生小遗憾,再想个理由使麝月无法去,用宝玉和麝月的闺中秘事满足读者的好奇心。
}
}麝月道:“你既在这里,越发不用去了,咱们两个说话顽笑岂不好?”\geng{全是袭人口气,所以后来代任。
}宝玉笑道:“两个作什么呢?怪没意思的,也罢了,早上你说头痒,这会子没什么事,我替你篦头罢。
”\zhu{篦:音“必”,篦子,一种比梳子密的梳头用具。
这里用作动词,用篦子梳头。
}麝月听了便道:“就是这样。
”说着,将文具镜匣搬来,卸去钗钏,打开头发,宝玉拿了篦子替他一一的梳篦。
\geng{金闺细事如此写。
}只篦了三五下,只见晴雯忙忙走进来取钱。
一见了他两个,便冷笑道:“哦,交杯盏还没吃,倒上头了!”\zhu{交杯盏:旧时婚礼,用两杯酒以彩线连之,新婚夫妇换杯饮酒,叫吃“交杯盏”。
上头:旧时女子出嫁始梳发髻叫上头,表示由姑娘变成了媳妇。
}\geng{虽谑语,亦少露怡红细事。
}\ping{可能是指袭人和宝玉偷试云雨之事。
}宝玉笑道:“你来,我也替你篦一篦。
”晴雯道:“我没那么大福。
”说着,拿了钱,便摔帘子出去了。
\par
宝玉在麝月身后,麝月对镜,二人在镜内相视。
\geng{此系石兄得意处。
}
宝玉便向镜内笑道:“满屋里就只是他磨牙。
\zhu{磨牙:比喻毫无意义地争辩;白费口舌。}
”麝月听说,忙向镜中摆手,\geng{好看,趣。
}宝玉会意。
忽听唿一声帘子响,晴雯又跑进来,问道:\geng{麝月摇手为此,可儿可儿!}“我怎么磨牙了?\geng{好看煞!}咱们倒得说说。
”\geng{娇憨满纸,令人叫绝。
壬午九月。
}麝月笑道:“你去你的罢,又来问人了。
”晴雯笑道:“你又护着。
你们那瞒神弄鬼的,\geng{找上文。
}\ping{可能暗示麝月和宝玉也有类似于袭人和宝玉的亲密关系。
}
我都知道。
等我捞回本儿来再说话。
”说着,一径出去了。
\ji{闲闲一段儿女口舌,却写麝月一人。
袭人出嫁之后,宝玉、宝钗身边还有一人,虽不及袭人周到,亦可免微嫌小弊等患,方不负宝钗之为人也。
故袭人出嫁后云“好歹留着麝月”一语,宝玉便依从此话。
可见袭人虽去实未去也。
写晴雯之疑忌,亦为下文跌扇角口等文伏脉,却又轻轻抹去。
正见此时都在幼时,虽微露其疑忌,见得人各禀天真之性,善恶不一,往后渐大渐生心矣。
但观者凡见晴雯诸人则恶之,何愚也哉!要知自古及今,愈是尤物,其猜忌愈甚。
若一味浑厚大量涵养,则有何可令人怜爱护惜哉?然后知宝钗、袭人等行为,并非一味蠢拙古板以女夫子自居,当绣幕灯前、绿窗月下,亦颇有或调或妒、轻俏艳丽等说,不过一时取乐买笑耳,非切切一味妒才嫉贤也,是以高诸人百倍。
不然,宝玉何甘心受屈于二女夫子哉?看过后文则知矣。
故观书诸君子不必恶晴雯,正该感晴雯金闺绣阁中生色方是。
}这里宝玉通了头,
\zhu{通头:梳、篦头发。}
命麝月悄悄的伏侍他睡下,不肯惊动袭人。
一宿无话。
\par
至次日清晨起来,袭人已是夜间发了汗,觉得轻省了些,\zhu{轻省[sheng]:这里指病情减轻,身体变好。
}只吃些米汤静养。
宝玉放了心,因饭后走到薛姨妈这边来闲逛。
彼时正月内,学房中放年学,
\zhu{放年学:旧时私塾在春节前放假,犹今之放寒假。}
闺阁中忌针,却都是闲时。
因贾环也过来顽,正遇见宝钗、香菱、莺儿三个赶围棋作耍,贾环见了也要顽。
宝钗素习看他亦如宝玉,并没他意,今儿听他要顽,让他上来坐了一处顽。
一磊十个钱,\zhu{磊:音“洛”,叠,摞。
}头一回自己赢了,心中十分欢喜。
\geng{写环兄先赢,亦是天生地设现成文字。
己卯冬夜。
}后来接连输了几盘,便有些着急。
赶着这盘正该自己掷骰子,若掷个七点便赢,若掷个六点,下该莺儿掷三点就赢了。
因拿起骰子来,狠命一掷,一个作定了五,那一个乱转。
莺儿拍着手只叫“幺”, \ji{娇憨如此。
}\geng{好看煞。
}贾环便瞪着眼,“六——七——八”混叫。
那骰子偏生转出幺来。
贾环急了,伸手便抓起骰子来,然后就拿钱,\geng{更也好看。
}说是个六点。
莺儿便说:“分明是个幺!”宝钗见贾环急了,便瞅莺儿说道:“越大越没规矩,难道爷们还赖你?\meng{酷肖。
}还不放下钱来呢!”莺儿满心委屈,见宝钗说,不敢则声,
\zhu{
则:做。
则声:开口发言、出声。
}
只得放下钱来,口内嘟囔说:“一个作爷的,还赖我们这几个钱,\geng{酷肖。
}连我也不放在眼里。
\zhu{连我也不放在眼里:我连这几个钱也不放在眼里,何况是作爷的,更不该放在眼里耍赖不给。}
前儿和宝玉顽,他输了那些,也没着急。
\geng{倒卷帘法,实写幼时往事。
可伤。
}下剩的钱,还是几个小丫头子们一抢,他一笑就罢了。
”宝钗不等说完,连忙断喝。
贾环道:“我拿什么比宝玉呢。
你们怕他,都和他好,\geng{蠢驴!}都欺负我不是太太养的。
\zhu{养:这里指生孩子。}
”\geng{观者至此,有不卷帘厌看者乎?\ping{“卷帘”令人费解。
这句话的意思大概是,贾环的话令人厌烦。
}余替宝卿实难为情。
}\ping{可怜可恨。
}说着,便哭了。
宝钗忙劝他:“好兄弟,快别说这话,人家笑话你。
”又骂莺儿。
\par
正值宝玉走来,见了这般形况,问是怎么了。
贾环不敢则声。
宝钗素知他家规矩,凡作兄弟的,都怕哥哥,\ji{大族规矩原是如此,一丝儿不错。
}却不知那宝玉是不要人怕他的。
他想着:“兄弟们一并都有父母教训,何必我多事,反生疏了。
况且我是正出,他是庶出,\zhu{正出、庶出:封建宗法制度下,正室(妻)所生的子女为“正出”,称为“嫡”;侧室(妾)所生的子女为“庶出”,称为“庶”。
}饶这样还有人背后谈论,\geng{此意不呆。
}还禁得辖治他了。
”\zhu{禁:音“斤”,禁得起,受得住,例如成语“弱不禁风”中的“禁”字即是此意。
}更有个呆意思存在心里。
\geng{又用讳人语瞒着看官。
\zhu{
批语认为曹雪芹总喜欢用表面文字掩盖作者的某种意图而瞒过看官。
叙述贾宝玉论兄弟之间的关系的一段话,批书人唯恐读者误认为这是作者真意,
所以特在这段话上面加了这条眉批。很显然,曹雪芹在这里是采用欲抑故扬的手法,抨击封建统治阶级鼓吹的“孝梯之道”。
}
己卯冬夜。
}——你道是何呆意?因他自幼姊妹丛中长大,亲姊妹有元春、探春,伯叔的有迎春、惜春,亲戚中又有史湘云、林黛玉、薛宝钗等诸人。
他便料定,原来天生人为万物之灵,凡山川日月之精秀,只钟于女儿,须眉男子不过是些渣滓浊沫而已。
因有这个呆念在心,把一切男子都看成混沌浊物,可有可无。
只是父亲叔伯兄弟中,因孔子是亘古第一人说下的,不可忤慢,只得要听他这句话。
\zhu{程乙本将“只是父亲……要听他这句话”改为“只是父亲、伯叔、兄弟之伦,因是圣人遗训,不敢违忤”,将宝玉从对三纲五常十分勉强,敷衍了事的叛逆形象,改为对孔子圣训服服贴贴的信奉者。}
\geng{听了这一个人之话,岂是呆子?由你自己说罢。
我把你作极乖的人看。
}所以,弟兄之间不过尽其大概的情理就罢了,并不想自己是丈夫,须要为子弟之表率。
是以贾环等都不怕他,却怕贾母,才让他三分。
如今宝钗恐怕宝玉教训他,倒没意思,便连忙替贾环掩饰。
宝玉道:“大正月里哭什么?这里不好,你别处顽去。
你天天念书,倒念糊涂了。
比如这件东西不好,横竖那一件好,就弃了这件取那个。
难道你守着这个东西哭一会子就好了不成?你原是来取乐顽的,既不能取乐,就往别处去寻乐顽去。
哭一会子,难道算取乐顽了不成?倒招自己烦恼,不如快去为是。
”\geng{呆子都会立这样意,说这样话?}贾环听了,只得回来。
\par
赵姨娘见他这般,因问:“又是那里垫了踹窝来了?”\zhu{垫踹窝:垫平路面,引申为供人践踏、代人受过。
踹窝:路面上践踏成的坑窝。
}\geng{多事人等口角谈吐。
}一问不答,\geng{毕肖。
}再问时,贾环便说:“同宝姐姐顽的,莺儿欺负我,赖我的钱,宝玉哥哥撵我来了。
”赵姨娘啐道:“谁叫你上高台盘去了?下流没脸的东西!那里顽不得?谁叫你跑了去讨没意思!”\ping{贾环本来就因为自己的庶出身份而自卑,赵姨娘作为亲生母亲,反而不断加强儿子的心理扭曲。
}\par
正说着,可巧凤姐在窗外过,都听在耳内,便隔窗说道:“大正月又怎么了?环兄弟小孩子家,一半点儿错了,你只教导他,说这些淡话作什么!\zhu{淡:没有意味的,无关紧要的。
“淡话”、“淡事”、“扯淡”中的“淡”字都是这个意思。
}凭他怎么去,还有太太老爷管他呢,就大口啐他!\geng{反得了理了,所谓贬中褒,想赵姨即不畏阿凤,亦无可回答。
}他现是主子,不好了,横竖有教导他的人,与你什么相干!\ping{亲生母亲反不能教导自己的儿子,赵姨娘的儿子贾环的地位是主子,但是自己的地位还是奴才,自己只是贡献卵子和子宫的生育工具。
}环兄弟,出来,跟我顽去。
”\geng{嫡嫡是彼亲生,句句竟成正中贬,赵姨实难答言。
到此方知题标用“弹”字甚妥协。
己卯冬夜。
}\ping{王熙凤正言弹妒意之二,赵姨娘对于宝玉、王夫人、凤姐这些得势风光之人的嫉妒怨愤。
}贾环素日怕凤姐比怕王夫人更甚,听见叫他,忙唯唯的出来。
\zhu{唯唯:音“委委”,恭敬应诺之词。
}
赵姨娘也不敢则声。
\geng{“弹妒意”正文。
}凤姐向贾环道:“你也是个没气性的!时常说给你:要吃,要喝,要顽,要笑,只爱同那一个姐姐妹妹哥哥嫂子顽,就同那个顽。
你不听我的话,反叫这些人教的歪心邪意,\geng{借人发脱,好阿凤!好口齿!句句正言正理,赵姨安得不抿翅低头,静听发挥?批至此,不禁一大白又[一]大白矣!\zhu{
白:酒杯。
}}狐媚子霸道的。
自己不尊重,要往下流走,安着坏心,还只管怨人家偏心。
输了几个钱?\geng{转得好。
}就这么个样儿!”贾环见问,只得诺诺的回说:“输了一二百。
”凤姐道:“亏你还是爷,输了一二百钱就这样!”\geng{作者\sout{当}[尚]记一大百乎?\sout{笑笑}[叹叹]。
}\ping{亲母还不如凤姐,亲生母亲打压本来就自卑的儿子的自尊心,而凤姐则是让贾环认识到自己毕竟是贾政仅有的两个儿子之一,不应该为了一二百这样的蝇头小利斤斤计较,而是要拿出小主人的派头,增加贾环的自信自尊。
凤姐这里简直可称光风霁月。
}回头叫丰儿:“去取一吊钱来,姑娘们都在后头顽呢,把他送了顽去。
\geng{收拾得好。
}你明儿再这么下流狐媚子,我先打了你,打发人告诉学里,皮不揭了你的!为你这个不尊重,\geng{又一折笔,更觉有味。
}恨的你哥哥牙痒,不是我拦着,窝心脚把你的肠子窝出来了。
”喝命:“去罢!”\geng{本来面目,断不可少。
}贾环诺诺的跟了丰儿,得了钱,\geng{三字写着环哥。
}自己和迎春等顽去。
不在话下。
\ji{一段大家子奴妾吆吻,如见如闻,正为下文五鬼作引也。
\zhu{第二十五回:魇魔法叔嫂逢五鬼。赵姨娘找马道婆施法害宝玉和凤姐。}
余谓宝玉肯效凤姐一点馀风,亦可继荣、宁之盛,诸公当为如何?}\par
且说宝玉正和宝钗顽笑,忽见人说:“史大姑娘来了。
”\ji{妙极!凡宝玉、宝钗正闲相遇时,非黛玉来,即湘云来,是恐泄漏文章之精华也。
若不如此,则宝玉久坐忘情,\ping{可能暗指第二十八回,薛宝钗羞笼红麝串,宝玉见之忘情。
}必被宝卿见弃,杜绝后文成其夫妇时无可谈旧之情,
\zhu{“杜绝……无可……”否定层数过多,应该删去一层否定。}
有何趣味哉?}
宝玉听了,抬身就走。
宝钗笑道:“等着,\geng{“等着”二字大有神情。
看官闭目熟思,方知趣味。
非批书人漫拟也。
\zhu{漫:随便。
}己卯冬夜。
}咱们两个一齐走,瞧瞧他去。
”说着,下了炕,同宝玉一齐来至贾母这边。
只见史湘云大笑大说的,见他两个来,忙问好厮见。
\zhu{厮:互相。
}\ji{写湘云又一笔法,特犯不犯。
}正值林黛玉在旁,因问宝玉:“在那里的?”宝玉便说:“在宝姐姐家的。
”黛玉冷笑道:“我说呢,亏在那里绊住,不然早就飞了来了。
”\geng{总是心中事语,故机括一动,随机而出。
\zhu{机括:弩上控制箭矢发射的机件。
泛指机械发动、开启的部分。
这里比喻心机、计谋。
}}宝玉笑道:“只许同你顽,替你解闷儿。
不过偶然去他那里一趟,就说这话。
”林黛玉道:“好没意思的话!去不去管我什么事,我又没叫你替我解闷儿。
可许你从此不理我呢!”说着,便赌气回房去了。
\par
宝玉忙跟了来,问道:“好好的又生气了?就是我说错了,你到底也还坐在那里,和别人说笑一会子。
又来自己纳闷。
”林黛玉道:“你管我呢!”宝玉笑道:“我自然不敢管你,只没有个看着你自己作践了身子呢。
”林黛玉道:“我作践坏了身子,我死,与你何干!”宝玉道:“何苦来,大正月里,死了活了的。
”林黛玉道:“偏说死!我这会子就死!你怕死,你长命百岁的,如何?”宝玉笑道:“要像只管这样闹,我还怕死呢?倒不如死了干净。
”黛玉忙道:“正是了,要是这样闹,不如死了干净。
”宝玉道:“我说我自己死了干净,别听错了话赖人。
”正说着,宝钗走来道:“史大妹妹等你呢。
”说着,便推宝玉走了。
\ji{此时宝钗尚未知他二人心性,故来劝,后文察其心性,故掷之不闻矣。
}这里林黛玉越发气闷,只向窗前流泪。
没两盏茶的工夫,宝玉仍来了。
\ji{盖宝玉亦是心中只有黛玉,见宝钗难却其意,故暂随彼去,以完宝钗之情,故少坐仍来也。
}
林黛玉见了,越发抽抽噎噎的哭个不住。
宝玉见了这样,知难挽回,打叠起千百样的款语温言来劝慰。
不料自己未张口,\geng{石头惯用如此笔仗。
}
只见黛玉先说道:“你又来作什么?横竖如今有人和你顽,比我又会念,又会作,又会写,又会说笑,又怕你生气拉了你去,你又作什么来?死活凭我去罢了!”宝玉听了忙上来悄悄的说道:“你这么个明白人,难道连‘亲不间疏,先不僭后’\geng{八字足可消气。
}也不知道?\zhu{亲不间疏,先不僭后:亲密者不被疏远者所离间,先到者不被后来者所超越。
间:离间。
僭:音“贱”,超越本分。
}我虽糊涂,却明白这两句话。
头一件,咱们是姑舅姊妹,宝姐姐是两姨姊妹,论亲戚,他比你疏。
\zhu{在古代,男性的亲缘关系要强于女性。
堂兄弟姐妹由于父亲和父亲是兄弟,距离最近;姑舅表亲由于母亲和父亲是兄妹或者姐弟,距离次之;姨表亲由于母亲和母亲是姐妹,距离最远。
}第二件,你先来,咱们两个一桌吃,一床睡,长的这么大了,他是才来的,岂有个为他疏你的?”林黛玉啐道:“我难道为叫你疏他?我成了个什么人了呢!我为的是我的心。
”宝玉道:“我也为的是你的心。
难道你就知你的心,不知我的心不成?”\ji{此二语不独观者不解,料作者亦未必解;不但作者未必解,想石头亦不解;不过述宝、林二人之语耳。
石头既未必解,宝、林此刻更自己亦不解,皆随口说出耳。
若观者必欲要解,须自揣自身是宝、林之流,则洞然可解;若自料不是宝、林之流,则不必求解矣。
万不可\sout{记}[借]此二句不解,错谤宝、林及石头、作者等人。
}林黛玉听了,低头一语不发,半日说道:“你只怨人行动嗔怪了你,\zhu{嗔:音“沉”一声,怒,生气。
}你再不知道你自己怄人难受。
\zhu{怄[òu]:使生气。}
就拿今日天气比,分明今儿冷的这样,你怎么倒反把个青肷披风脱了呢?”\zhu{青肷(肷音“遣”):指青狐皮的腋部。
}\ji{真正奇绝妙文,真如羚羊挂角,无迹可求。
\zhu{羚羊挂角,无迹可求:传说中羚羊晚上睡觉的时候,跟普通的牲口野兽不同,它会寻找一棵树,看准了位置就奋力一跳,用它的角挂在树杈上,这样可以保证整个身体是悬空的,别的野兽够不着它。
旧时多比喻诗的意境超脱,不着形迹。
}此等奇妙,非口中笔下可形容出者。
}宝玉笑道:“何尝不穿着,见你一恼,我一炮燥就脱了。
”\zhu{炮燥:由于心中烦躁而感到身上燥热的意思。
炮:音“袍”,裹物而烧谓之炮。
} 黛玉叹道:“回来伤了风,又该饿着吵吃的了。
”\ji{一语仍归儿女本传,却又轻轻抹去也。
}\geng{明明写湘云来是正文,只用二三答言,反写玉、林小角口,又用宝钗岔开,仍不了局。
再用千句柔言百般温态,正在情完未完之时,湘云突至,“谑娇音”之文终见。
真正“卖弄有家私”之笔也。
\ping{“卖弄有家私”可能是展现写作技巧的意思。
}丁亥夏。
畸笏叟。
}\par
二人正说着,只见湘云走来,笑道:“二哥哥,林姐姐,你们天天一处顽,我好容易来了,也不理我一理儿。
”林黛玉笑道:“偏是咬舌子爱说话,连个‘二’哥哥也叫不出来,只是‘爱’哥哥‘爱’哥哥的。
回来赶围棋儿,又该你闹‘幺爱三四五’了。
”宝玉笑道:“你学惯了他,明儿连你还咬起来呢。
”\ji{可笑近之野史中,满纸羞花闭月、莺啼燕语。
殊不知真正美人方有一陋处,如太真之肥、飞燕之瘦、西子之病,若施于别个,不美矣。
今见“咬舌”二字加之湘云,是何大法手眼敢用此二字哉?不独不见其陋,且更觉轻巧娇媚,俨然一娇憨湘云立于纸上,掩卷合目思之,其“爱”“厄”娇音如入耳内。
然后将满纸莺啼燕语之字样填粪窖可也。
}史湘云道:“他再不放人一点儿,专挑人的不好。
你自己便比世人好,也不犯着见一个打趣一个。
指出一个人来,你敢挑他,我就伏你。
”黛玉忙问是谁。
湘云道:“你敢挑宝姐姐的短处,就算你是好的。
我算不如你,他怎么不及你呢。
”林黛玉听了,冷笑道:“我当是谁,原来是他!我那里敢挑他呢。
”\geng{此作者放笔写,非褒钗贬颦也。
己卯冬夜。
}宝玉不等说完,忙用话岔开。
湘云笑道:“这一辈子我自然比不上你。
我只保佑着明儿得一个咬舌的林姐夫,\ping{宝玉不咬舌,湘云所言的这个林姐夫不是宝玉。
}
时时刻刻你可听‘爱’‘厄’去。
阿弥陀佛,那才现在我眼里!”说的众人一笑,湘云忙回身跑了。
要知端详,下回分解。
\par
\ji{此回文字重作轻抹。
\zhu{
重作:重要的创作需要重笔着力描写。轻抹:这里指用轻描淡写。
重作轻抹表现了作者写作功力之深,举重若轻。
}
得力处是凤姐拉李嬷嬷去,借环哥弹压赵姨。
细致处宝钗为李嬷劝宝玉,安慰环哥,断喝莺儿。
至急为难处是宝、颦论心。
无可奈何处是“就拿今日天气比”,“黛玉冷笑道:‘我当谁,原来是他!’”冷眼最好看处是宝钗、黛玉看凤姐拉李嬷云“这一阵风”;玉、麝一节;
\zhu{玉、麝一节:宝玉给麝月篦头一段文字。}
湘云到,宝玉就走,宝钗笑说“等着”;湘云大笑大说;颦儿学咬舌;湘云念佛跑了数节,可使看官于纸上能耳闻目睹其音其形之文。
}
\dai{039}{宝玉给麝月篦头晴雯吃醋,丫鬟们耍戏}
\dai{040}{凤姐训斥贾环赵姨娘}
\sun{p20-1}{王熙凤正言弹妒意}{图右侧:袭人病了,才吃了药,蒙着头发汗,宝玉奶妈李嬷嬷进来时没有见到,李嬷嬷挑剔袭人不理她,吵闹起来。
宝玉赶过来替袭人分辩了几句,李嬷嬷越发倚老卖老不依不饶。
黛玉宝钗也劝不住, 凤姐在上房听到了,忙过来连哄带劝将李嬷嬷拉走了。
图上侧中部:贾环因掷骰子输了几个钱赖帐被莺儿说了几句便哭了起来。
当下,宝钗骂了莺儿,劝了贾环。
回家后,赵姨娘啐道:“下流没脸的东西!那里顽不得?谁叫你跑了去讨没意思!”正巧凤姐在窗外过,都听在耳内,便隔窗说道:“说这些淡话作什么!”凤姐回头叫丰儿取一吊钱来送贾环和姐妹玩耍。
}
