\chapter{撕扇子作千金一笑\quad 因麒麟伏白首双星}
\zhu{因麒麟伏白首双星:因:凭;借。
伏:隐伏,伏笔。
白首:指老年人。
双星:牛郎、织女二星。
据脂批:“后数十回若兰在射圃所佩之麒麟,正此麒麟也。
提纲伏于此回中。
”这个回目可能是八十回以后原稿中有关史湘云命运的伏笔。
史湘云可能和同样有麒麟的卫若兰结为夫妻,但是好景不长,卫若兰可能卷入到了四大家族倒台的事件之中,夫妻离散,从此和史湘云如天上的牛郎星和织女星那样天各一方,直到两人在孤独中步入老年,头发斑白,依旧不得团聚。
}\par
\ji{“撕扇子”是以不知情之物,供娇嗔不知情事之人一笑,所谓“情不情”。
\hang
“金玉姻缘”已定,又写一金麒麟,是间色法也。
何颦儿为其所惑?故颦儿谓“情情”。
}\par
话说袭人见了自己吐的鲜血在地,也就冷了半截,想着往日常听人说:“少年吐血,年月不保,纵然命长,终是废人了。
”想起此言,不觉将素日想着后来争荣夸耀之心尽皆灰了,眼中不觉滴下泪来。
\ping{人都是在自己身体健康的时候名利心旺盛,甚至不惜牺牲自己的身体。
但是当失去健康的时候,就会觉得名利如过眼云烟,没有健康就没有了一切。
}宝玉见他哭了,也不觉心酸起来,因问道:“你心里觉的怎么样?”袭人勉强笑道:“好好的,觉怎么呢。
”宝玉的意思即刻便要叫人烫黄酒,要山羊血黎洞丸来。
\zhu{山羊血黎洞丸:黎洞丸:中成药名,由血竭、三七、儿茶、雄黄、牛黄等十馀味中药组成。
治金疮出血,跌打损伤,淤血奔心、头昏不省及痛肿等症。
因为配方用山羊血,故称“山羊血黎洞丸”。
}袭人拉了他的手,笑道:“你这一闹不打紧,闹起多少人来,倒抱怨我轻狂。
分明人不知道,倒闹的人知道了,你也不好,我也不好。
正经明儿你打发小子问问王太医去,\ping{袭人作为一个丫鬟,都知道看病要找王太医,而不是去找鲍太医。
王夫人给黛玉找的鲍太医,估计水平实在不敢恭维。
}\ping{宝玉心里袭人珍贵,不过旁人眼里还是主是主,仆是仆,宝玉为仆人如此紧张,对袭人来说,也是把她放在火上烤,惹得嫉妒和议论。
}弄点子药吃吃就好了。
人不知鬼不觉的可不好?”宝玉听了有理,也只得罢了,向案上斟了茶来,给袭人漱了口。
袭人知宝玉心内是不安稳的,待要不叫他伏侍,他又必不依;二则定要惊动别人,不如由他去罢:因此只在榻上由宝玉去伏侍。
\par
一交五更,宝玉也顾不的梳洗,忙穿衣出来,将王济仁叫来,亲自确问。
王济仁问其原故,不过是伤损,便说了个丸药的名字,怎么服,怎么敷。
宝玉记了,回园依方调治。
不在话下。
\par
这日正是端阳佳节,蒲艾簪门,虎符系臂。
\zhu{蒲艾簪门,虎符系臂:蒲、艾:都是香草。
簪:插。
虎符:这里指用绫罗制成的小老虎。
旧俗每逢端午节,将蒲艾插在门上,把虎符系在儿童的臂上,认为可以“避邪”。
}午间,王夫人治了酒席,请薛家母女等赏午。
\zhu{赏午:凡端午节吃午饭,饮雄黄酒,吃樱桃、桑葚等时鲜果品,以及赏石榴花等等节日活动,统称赏午。
}宝玉见宝钗淡淡的,也不和他说话,自知是昨儿的原故。
王夫人见宝玉没精打彩,也只当是金钏儿昨日之事,他没好意思的,越发不理他。
林黛玉见宝玉懒懒的,只当是他因为得罪了宝钗的原故,心中不自在,形容也就懒懒的。
凤姐昨日晚间王夫人就告诉了他宝玉金钏的事,知道王夫人不自在,自己如何敢说笑,也就随着王夫人的气色行事,更觉淡淡的。
贾迎春姊妹见众人无意思,也都无意思了。
因此,大家坐了一坐就散了。
\par
林黛玉天性喜散不喜聚。
他想的也有个道理,他说,“人有聚就有散,聚时欢喜,到散时岂不清冷?既清冷则生伤感,所以不如倒是不聚的好。
比如那花开时令人爱慕,谢时则增惆怅,所以倒是不开的好。
”故此人以为喜之时,他反以为悲。
那宝玉的情性只愿常聚,生怕一时散了添悲;那花只愿常开,生怕一时谢了没趣;只到筵散花谢,虽有万种悲伤,也就无可如何了。
因此,今日之筵,大家无兴散了,林黛玉倒不觉得,倒是宝玉心中闷闷不乐,回至自己房中长吁短叹。
偏生晴雯上来换衣服,不防又把扇子失了手跌在地下,将股子跌折。
宝玉因叹道:“蠢才,蠢才!将来怎么样?明日你自己当家立事,难道也是这么顾前不顾后的?”晴雯冷笑道:“二爷近来气大的很,行动就给脸子瞧。
前儿连袭人都打了,今儿又来寻我们的不是。
要踢要打凭爷去。
就是跌了扇子,也是平常的事。
先时连那么样的玻璃缸、玛瑙碗不知弄坏了多少,也没见个大气儿,这会子一把扇子就这么着了。
何苦来!要嫌我们就打发我们,再挑好的使。
好离好散的,倒不好?”宝玉听了这些话,气的浑身乱战,
\zhu{战:通“颤”,发抖。}
因说道:“你不用忙,将来有散的日子!”\par
袭人在那边早已听见,忙赶过来向宝玉道:“好好的,又怎么了?可是我说的:‘一时我不到,就有事故儿。
’”\ping{袭人自大,俨然大管家的心态。
}晴雯听了冷笑道:“姐姐既会说,就该早来,也省了爷生气。
自古以来,就是你一个人伏侍爷的,我们原没伏侍过。
因为你伏侍的好,昨日才挨窝心脚;我们不会伏侍的,到明儿还不知是个什么罪呢!”袭人听了这话,又是恼,又是愧,待要说几句话,又见宝玉已经气的黄了脸,少不得自己忍了性子,推晴雯道:“好妹妹,你出去逛逛,原是我们的不是。
”晴雯听他说“我们”两个字,自然是他和宝玉了,不觉又添了酸意,冷笑几声,道:“我倒不知道你们是谁,别教我替你们害臊了!便是你们鬼鬼祟祟干的那事儿,也瞒不过我去,\ping{可能暗指偷试云雨之事。
}那里就称起‘我们’来了。
明公正道,连个姑娘还没挣上去呢,\zhu{姑娘:这里指通房丫头。
贴身侍婢收纳为妾,称“通房丫头”。
其地位低于姨娘。
}也不过和我似的,那里就称上‘我们’了!”袭人羞的脸紫胀起来,想一想,原来是自己把话说错了。
宝玉一面说:“你们气不忿,我明儿偏抬举他。
”袭人忙拉了宝玉的手道:“他一个糊涂人,你和他分证什么?况且你素日又是有担待的,比这大的过去了多少,今儿是怎么了?”晴雯冷笑道:“我原是糊涂人,那里配和我说话呢!”袭人听说道:“姑娘倒是和我拌嘴呢,是和二爷拌嘴呢?要是心里恼我,你只和我说,不犯着当着二爷吵;要是恼二爷,不该这么吵的万人知道。
\ping{袭人主动把事态从丫鬟之间的吵架升级为丫鬟和主人的吵架。
}我才也不过为了事,进来劝开了,大家保重。
姑娘倒寻上我的晦气。
又不像是恼我,又不像是恼二爷,夹枪带棒,终久是个什么主意?我就不多说,让你说去。
”说着便往外走。
\par
宝玉向晴雯道:“你也不用生气,我也猜着你的心事了。
我回太太去,你也大了,打发你出去好不好?”晴雯听了这话,不觉又伤起心来,含恨说道:“为什么我出去?要嫌我,变着法儿打发我出去,也不能够。
”宝玉道:“我何曾经过这个吵闹?一定是你要出去了。
不如回太太,打发你去吧。
”说着,站起来就要走。
袭人忙回身拦住,笑道:“往那里去?”宝玉道:“回太太去。
”袭人笑道:“好没意思!真个的去回,你也不怕臊了?便是他认真的要去,也等把这气下去了,等无事中说话儿回了太太也不迟。
这会子急急的当作一件正经事去回,岂不叫太太犯疑?”\ping{这时候如果直接把晴雯撵了出去,可能导致晴雯撕破脸皮,毫无顾忌地说出自己知道的袭人和宝玉偷试云雨之事,袭人受到的责罚肯定比晴雯要大了。
在前一回金钏只是因为和宝玉调笑就被赶走。
}宝玉道:“太太必不犯疑,我只明说是他闹着要去的。
”晴雯哭道:“我多早晚闹着要去了?饶生了气,\zhu{饶:即使,尽管,表示让步关系。
}还拿话压派我。
只管去回,我一头碰死了也不出这门儿。
”\ping{晴雯高傲自我,不受委屈,时常被当作反封建的代表,但是晴雯并没有独立抗争的资本,当自己要被撵出去的时候,也不得不服软低头求饶。
}宝玉道:“这也奇了。
你又不去,你又闹些什么?我经不起这吵,不如去了倒干净。
”说着一定要去回。
袭人见拦不住,只得跪下了。
碧痕、秋纹、麝月等众丫鬟见吵闹,都鸦雀无闻的在外头听消息,这会子听见袭人跪下央求,便一齐进来都跪下了。
宝玉忙把袭人扶起来,叹了一声,在床上坐下,叫众人起去,向袭人道:“叫我怎么样才好!这个心使碎了也没人知道。
”说着不觉滴下泪来。
袭人见宝玉流下泪来,自己也就哭了。
\par
晴雯在旁哭着,方欲说话,只见林黛玉进来,便出去了。
林黛玉笑道:“大节下怎么好好的哭起来?难道是为争粽子吃争恼了不成?”宝玉和袭人嗤的一笑。
黛玉道:“二哥哥不告诉我,我问你就知道了。
”一面说,一面拍着袭人的肩,笑道:“好嫂子,你告诉我。
必定是你两个拌了嘴了。
告诉妹妹,替你们和劝和劝。
”袭人推他道:“林姑娘你闹什么?我们一个丫头,姑娘只是混说。
”黛玉笑道:“你说你是丫头,我只拿你当嫂子待。
”宝玉道:“你何苦来替他招骂名儿。
饶这么着,\zhu{饶:即使,尽管,表示让步关系。
}
还有人说闲话,还搁的住你来说他。
”\ping{黛玉称袭人为嫂子,明面上是捧袭人,实际上是压袭人,让袭人和宝玉醒悟过来,袭人只是个丫鬟,还没有正式娶亲,宝玉的偏爱、袭人的自大,都将招来风言风语,并不得体。
}袭人笑道:“林姑娘,你不知道我的心事,除非一口气不来死了倒也罢了。
”\ping{袭人说自己不被黛玉理解,只有死才能不因此而伤。
}林黛玉笑道:“你死了,别人不知怎么样,我先就哭死了。
”宝玉笑道:“你死了,我作和尚去。
”袭人笑道:“你老实些罢,何苦还说这些话。
”林黛玉将两个指头一伸,抿嘴笑道:“作了两个和尚了。
我从今以后都记着你作和尚的遭数儿。
”\ping{又埋下宝玉结局的伏笔。
}
宝玉听得,知道是他点前儿的话,自己一笑也就罢了。
\ping{黛玉压制把自己看作女主人的袭人,并巧妙地岔开话题,让宝玉息怒。}
\par
一时黛玉去后,就有人说“薛大爷请”,宝玉只得去了。
原来是吃酒,不能推辞,只得尽席而散。
\par
晚间回来,已带了几分酒,踉跄来至自己院内,只见院中早把乘凉枕榻设下,榻上有个人睡着。
宝玉只当是袭人,一面在榻沿上坐下,一面推他,问道:“疼的好些了?”只见那人翻身起来说:“何苦来,又招我!”宝玉一看,原来不是袭人,却是晴雯。
宝玉将他一拉,拉在身旁坐下,笑道:“你的性子越发惯娇了。
早起就是跌了扇子,我不过说了那两句,你就说上那些话。
说我也罢了,袭人好意来劝,你又括上他,你自己想想,该不该?”晴雯道:“怪热的,拉拉扯扯作什么!叫人来看见像什么!我这身子也不配坐在这里。
”宝玉笑道:“你既知道不配,为什么睡着呢?”晴雯没的话,嗤的又笑了,说:“你不来便使得,你来了就不配了。
起来,让我洗澡去。
袭人麝月都洗了澡,我叫了他们来。
”宝玉笑道:“我才又吃了好些酒,还得洗一洗。
你既没有洗,拿了水来咱们两个洗。
”\par
晴雯摇手笑道:“罢,罢,我不敢惹爷。
还记得碧痕打发你洗澡,足有两三个时辰,也不知道作什么呢。
我们也不好进去的。
后来洗完了,进去瞧瞧,地下的水淹着床腿,连席子上都汪着水,也不知是怎么洗了,笑了几天。
我也没那工夫收拾,也不用同我洗去。
今儿也凉快,那会子洗了,可以不用再洗。
\ping{碧痕可能也和宝玉偷试云雨了,但是晴雯在宝玉发出一起洗澡的邀约后,依旧洁身自好,果断决绝了,不愿意为了争得宝玉的宠爱而无所不为。
}我倒舀一盆水来,你洗洗脸通通头。
\zhu{通头:梳、篦头发。
}才刚鸳鸯送了好些果子来,都湃在那水晶缸里呢,\zhu{湃:音“拔”,用冰或凉水浸泡果品或饮料等使之变冷。
}叫他们打发你吃。
”宝玉笑道:“既这么着,你也不许洗去,只洗洗手来拿果子来吃罢。
”晴雯笑道:“我慌张的很,连扇子还跌折了,那里还配打发吃果子。
倘或再打破了盘子,还更了不得呢。
”宝玉笑道:“你爱打就打,这些东西原不过是借人所用,你爱这样,我爱那样,各自性情不同。
比如那扇子原是扇的,你要撕着玩也可以使得,只是不可生气时拿他出气。
就如杯盘,原是盛东西的,你喜听那一声响,就故意的碎了也可以使得,只是别在生气时拿他出气。
这就是爱物了。
”晴雯听了,笑道:“既这么说,你就拿了扇子来我撕。
我最喜欢撕的。
”宝玉听了,便笑着递与他。
晴雯果然接过来,嗤的一声,撕了两半,接着嗤嗤又听几声。
宝玉在旁笑着说:“响的好,再撕响些!”\par
正说着,只见麝月走过来,笑道:“少作些孽罢。
”宝玉赶上来,一把将他手里的扇子也夺了递与晴雯。
晴雯接了,也撕了几半子,二人都大笑。
麝月道:“这是怎么说,拿我的东西开心儿?”宝玉笑道:“打开扇子匣子你拣去,什么好东西!”麝月道:“既这么说,就把匣子搬了出来,让他尽力的撕,岂不好?”宝玉笑道:“你就搬去。
”麝月道:“我可不造这孽。
他也没折了手,叫他自己搬去。
”晴雯笑着,倚在床上说道:“我也乏了,明儿再撕罢。
”宝玉笑道:“古人云:‘千金难买一笑。
’\zhu{“千金难买一笑”:形容博得美人一笑之不易。
}几把扇子能值几何!”一面说着,一面叫袭人。
袭人才换了衣服走出来,小丫头佳蕙过来拾去破扇,大家乘凉,不消细说。
\par
至次日午间,王夫人、薛宝钗、林黛玉众姊妹正在贾母房内坐着,就有人回:“史大姑娘来了。
”一时果见史湘云带领众多丫鬟媳妇走进院来。
宝钗黛玉等忙迎至阶下相见。
青年姊妹间经月不见,一旦相逢,其亲密自不必细说。
\par
一时进入房中,请安问好,都见过了。
贾母因说:“天热,把外头的衣服脱脱罢。
”史湘云忙起身宽衣。
王夫人因笑道:“也没见穿上这些作什么?”史湘云笑道:“都是二婶婶叫穿的,谁愿意穿这些。
”宝钗一旁笑道:“姨娘不知道,他穿衣裳还更爱穿别人的衣裳。
可记得旧年三四月里,他在这里住着,把宝兄弟的袍子穿上,靴子也穿上,额子也勒上,猛一瞧倒像是宝兄弟,就是多两个坠子。
\zhu{坠子:耳坠。}
他站在那椅子后边,哄的老太太只是叫‘宝玉,你过来,仔细那上头挂的灯穗子招下灰来迷了眼’。
他只是笑,也不过去。
后来大家撑不住笑了,老太太才笑了,说:‘倒扮上男人好看了。
’”林黛玉道:“这算什么。
惟有前年正月里接了他来,住了没两日就下起雪来,老太太和舅母那日想是才拜了影回来,\zhu{影:指旧时供奉的祖先画像。
逢年过节或祭祀时子孙叩拜祖先画像称“拜影”。
}老太太的一个新新的大红猩猩毡斗蓬放在那里,\zhu{猩猩毡:猩红色毛毡。
}谁知眼错不见他就披了,
\zhu{眼错不见:眨眼之间没注意到,或未觉察。}
又大又长,他就拿了个汗巾子拦腰系上,和丫头们在后院子扑雪人儿去,一跤栽到沟跟前,弄了一身泥水。
”说着,大家想着前情,都笑了。
\par
宝钗笑向那周奶妈道:“周妈,你们姑娘还是那么淘气不淘气了?”周奶娘也笑了。
迎春笑道:“淘气也罢了,我就嫌他爱说话。
也没见睡在那里还是咭咭呱呱,\zhu{
咭咭呱呱[jījīguāguā]:又说又笑的声音。
}笑一阵,说一阵,也不知那里来的那些话。
”王夫人道:“只怕如今好了。
前日有人家来相看,眼见有婆婆家了,还是那么着。
”贾母因问:“今儿还是住着,还是家去呢?”周奶娘笑道:“老太太没有看见衣服都带了来,可不住两天?”史湘云问道:“宝玉哥哥不在家么?”宝钗笑道:“他再不想着别人,只想宝兄弟,两个人好憨的。
这可见还没改了淘气。
”贾母道:“如今你们大了,别提小名儿了。
”\par
刚只说着,只见宝玉来了,笑道:“云妹妹来了。
怎么前儿打发人接你去,怎么不来?”王夫人道:“这里老太太才说这一个,他又来提名道姓的了。
”林黛玉道:“你哥哥得了好东西,等着你呢。
”史湘云道:“什么好东西?”宝玉笑道:“你信他呢!几日不见,越发高了。
”湘云笑道:“袭人姐姐好?”宝玉道:“多谢你记挂。
”湘云道:“我给他带了好东西来了。
”说着,拿出手帕子来,挽着一个疙瘩。
宝玉道:“什么好的?你倒不如把前儿送来的那种绛纹石的戒指儿带两个给他。
”湘云笑道:“这是什么?”说着便打开。
众人看时,果然就是上次送来的那绛纹戒指,一包四个。
林黛玉笑道:“你们瞧瞧他这主意。
前儿一般的打发人给我们送了来,你就把他的带来岂不省事?今儿巴巴的自己带了来,我当又是什么新奇东西,原来还是他。
真真你是糊涂人。
”史湘云笑道:“你才糊涂呢!我把这理说出来,大家评一评谁糊涂。
给你们送东西,就是使来的不用说话,拿进来一看,自然就知是送姑娘们的了;若带他们的东西,这得我先告诉来人,这是那一个丫头的,那是那一个丫头的,那使来的人明白还好,再糊涂些,丫头的名字他也不记得,混闹胡说的,反连你们的东西都搅糊涂了。
若是打发个女人素日知道的还罢了,偏生前儿又打发小子来,可怎么说丫头们的名字呢?横竖我来给他们带来,岂不清白。
”说着,把四个戒指放下,说道:“袭人姐姐一个,鸳鸯姐姐一个,金钏儿姐姐一个,平儿姐姐一个:这倒是四个人的,难道小子们也记得这么清白?”众人听了都笑道:“果然明白。
”宝玉笑道:“还是这么会说话,不让人。
”林黛玉听了,冷笑道:“他不会说话,他的金麒麟会说话。
”一面说着,便起身走了。
幸而诸人都不曾听见,只有薛宝钗抿嘴一笑。
宝玉听见了,倒自己后悔又说错了话,忽见宝钗一笑,由不得也笑了。
宝钗见宝玉笑了,忙起身走开,找了林黛玉去说话。
\ping{黛玉占下风,湘云占上风,宝钗此时抿嘴一笑,被宝玉看到后,害怕得意之心和金配玉之意泄露,所以害羞而离开,去慰问有点小失落的黛玉。
}\par
贾母向湘云道:“吃了茶歇一歇,瞧瞧你的嫂子们去。
园里也凉快,同你姐姐们去逛逛。
”湘云答应了,将三个戒指儿包上,歇了一歇,便起身要瞧凤姐等人去。
众奶娘丫头跟着,到了凤姐那里,说笑了一回,出来便往大观园来,见过了李宫裁,少坐片时,便往怡红院来找袭人。
因回头说道:“你们不必跟着,只管瞧你们的朋友亲戚去,留下翠缕伏侍就是了。
”众人听了,自去寻姑觅嫂,早剩下湘云翠缕两个人。
\par
翠缕道:“这荷花怎么还不开?”史湘云道:“时候没到。
”翠缕道:“这也和咱们家池子里的一样,也是楼子花?”\zhu{楼子花:在花蕊里又开出一层花,叫楼子花,又叫重台,俗称起楼子。
}湘云道:“他们这个还不如咱们的。
”翠缕道:“他们那边有棵石榴,接连四五枝,真是楼子上起楼子,这也难为他长。
”史湘云道:“花草也是同人一样,气脉充足,长的就好。
”翠缕把脸一扭,说道:“我不信这话。
若说同人一样,我怎么不见头上又长出一个头来的人?”湘云听了由不得一笑,说道:“我说你不用说话,你偏好说。
这叫人怎么好答言?天地间都赋阴阳二气所生,或正或邪,或奇或怪,千变万化,都是阴阳顺逆多少,一生出来,人罕见的就奇,究竟理还是一样。
”翠缕道:“这么说起来,从古至今,开天辟地,都是些阴阳了?”湘云笑道:“糊涂东西,越说越放屁。
什么‘都是些阴阳’,难道还有个‘阴阳’不成!‘阴’‘阳’两个字还只是一字,阳尽了就成阴,阴尽了就成阳,不是阴尽了又有个阳生出来,阳尽了又有个阴生出来。
”翠缕道:“这糊涂死了我!什么是个阴阳,没影没形的。
我只问姑娘,这阴阳是怎么个样儿?”湘云道:“阴阳可有什么样儿,不过是个气,器物赋了成形。
比如天是阳,地就是阴;水是阴,火就是阳;日是阳,月就是阴。
”\par
翠缕听了,笑道:“是了,是了,我今儿可明白了。
怪道人都管着日头叫‘太阳’呢,算命的管着月亮叫什么‘太阴星’,就是这个理了。
”湘云笑道:“阿弥陀佛!刚刚的明白了。
”翠缕道:“这些大东西有阴阳也罢了,难道那些蚊子、虼蚤、\zhu{虼:音“各”。
虼蚤:跳蚤。
}蠓虫儿、\zhu{蠓(音“蒙”)虫儿:蠓科昆虫,体呈褐色或黑色,翅短而宽,甚小,雌蠓吸食人畜血液。
}花儿、草儿、瓦片儿、砖头儿也有阴阳不成?”湘云道:“怎么有没阴阳的呢?比如那一个树叶儿还分阴阳呢,那边向上朝阳的便是阳,这边背阴覆下的便是阴。
”翠缕听了,点头笑道:“原来这样,我可明白了。
只是咱们这手里的扇子,怎么是阳,怎么是阴呢?”湘云道:“这边正面就是阳,那边反面就为阴。
”翠缕又点头笑了,还要拿几件东西问,因想不起个什么来,猛低头就看见湘云宫绦上系的金麒麟,便提起来问道:“姑娘,这个难道也有阴阳?”湘云道:“走兽飞禽,雄为阳,雌为阴;牝为阴,牡为阳。
\zhu{牝[pìn]牡[mǔ]:鸟兽雌者叫牝,雄者叫牡。
}怎么没有呢!”翠缕道:“这是公的,到底是母的呢?”湘云道:“这连我也不知道。
”翠缕道:“这也罢了,怎么东西都有阴阳,咱们人倒没有阴阳呢?”湘云照脸啐了一口道:“下流东西,好生走罢!越问越问出好的来了!”\zhu{湘云前日有人家来相看,眼见有婆婆家了,对阴阳男女之事感到害羞。
}翠缕笑道:“这有什么不告诉我的呢?我也知道了,不用难我。
”湘云笑道:“你知道什么?”翠缕道:“姑娘是阳,我就是阴。
”说着,湘云拿手帕子握着嘴,呵呵的笑起来。
翠缕道:“说是了,就笑的这样了。
”湘云道:“很是,很是。
”翠缕道:“人规矩主子为阳,奴才为阴。
我连这个大道理也不懂得?”湘云笑道:“你很懂得。
”\par
一面说,一面走,刚到蔷薇架下,湘云道:“你瞧那是谁掉的首饰,金晃晃在那里。
”翠缕听了,忙赶上拾在手里攥着,笑道:“可分出阴阳来了。
”说着,先拿史湘云的麒麟瞧。
湘云要他拣的瞧,翠缕只管不放手,笑道:“是件宝贝,姑娘瞧不得。
这是从那里来的?好奇怪!我从来在这里没见有人有这个。
”湘云笑道:“拿来我看。
”翠缕将手一撒,笑道:“请看。
”湘云举目一验,却是文彩辉煌的一个金麒麟,比自己佩的又大又有文彩。
湘云伸手擎在掌上,只是默默不语,正自出神,忽见宝玉从那边来了,笑问道:“你两个在这日头底下作什么呢?怎么不找袭人去?”湘云连忙将那麒麟藏起道:“正要去呢。
咱们一处走。
”说着,大家进入怡红院来。
\par
袭人正在阶下倚槛追风,\zhu{槛:音“剑”,栏杆。
}忽见湘云来了,连忙迎下来,携手笑说一向久别情况。
一时进来归坐,宝玉因笑道:“你该早来,我得了一件好东西,专等你呢。
”说着,一面在身上摸掏,掏了半天,呵呀了一声,便问袭人“那个东西你收起来了么?”袭人道:“什么东西?”宝玉道:“前儿得的麒麟。
”袭人道:“你天天带在身上的,怎么问我?”宝玉听了,将手一拍说道:“这可丢了,往那里找去!”就要起身自己寻去。
湘云听了,方知是他遗落的,便笑问道:“你几时又有了麒麟了?”宝玉道:“前儿好容易得的呢,\zhu{好容易:好不容易。
}不知多早晚丢了,我也糊涂了。
”湘云笑道:“幸而是顽的东西,还是这么慌张。
”说着,将手一撒,“你瞧瞧,是这个不是?”宝玉一见,由不得欢喜非常,因说道……不知是如何,且听下回分解。
\par
\ji{后数十回若兰在射圃所佩之麒麟,正此麒麟也。
提纲伏于此回中,所谓“草蛇灰线,在千里之外”。
\ping{卫若兰和史湘云的姻缘因为宝玉的金麒麟而牵线,正如蒋玉函和袭人的姻缘因为宝玉的汗巾子而牵线,宝玉把自己的汗巾子送给了袭人,类似地,可能宝玉把自己的金麒麟送给了卫若兰。
}}
\dai{061}{撕扇子作千金一笑}
\dai{062}{湘云翠缕论阴阳}
\sun{p31-1}{龄官划蔷痴及局外,宝玉淋雨错踢袭人,撕扇子作千金一笑}{图右上:宝玉走到蔷薇架下,见一个女孩子蹲在花下用簪子写字,一面悄悄流泪。
仔细看了, 那女孩原来写了一个个“蔷”宇。
正痴想时,落下一阵雨来。
图左下:宝玉淋了雨,回到怡红院,叩门许久方开,还只当是那些小丫头,踢了一脚,谁知错踢了袭人。
图左上:端阳节这日,晴雯失手跌断了折扇,宝玉说了她两句,引出一番口角。
晚间宝玉见园内乘凉的枕榻上睡着晴雯,为博得晴雯的欢心,任其撕扇取乐。
}
