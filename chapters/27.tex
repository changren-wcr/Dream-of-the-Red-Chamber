\chapter{滴翠亭杨妃戏彩蝶\quad 埋香冢飞燕泣残红}
\zhu{杨妃、飞燕:即唐玄宗妃杨玉环和汉成帝妃赵飞燕,均为古代著名美人。
环肥燕瘦,可堪对举,在此分别喻宝钗黛玉。
}
\par
\geng{《葬花吟》是大观园诸艳之归源小引,故用在饯花日诸艳毕集之期。
饯花日不论其典与不典,
\zhu{饯花:送别残花。}
只取其韵耳。
}\par
话说林黛玉正自悲泣,忽听院门响处,只见宝钗出来了,宝玉、袭人一群人送了出来。
待要上去问着宝玉,又恐当着众人问,羞了他倒不便,因而闪过一旁,让宝钗去了,宝玉等进去关了门,方转过来,犹望着门洒了几点泪。
自觉无味,
\geng{四字闪煞颦儿也。
}
便转身回来,无精打彩的卸了残妆。
\par
紫鹃、雪雁素日知道他的情性:无事闷坐,不是愁眉,\geng{画美人之秘诀。
}便是长叹,且好端端的不知为了什么,便常常的就自泪自干。
\geng{补写,却是避繁文法。
}先时还解劝,怕他思父母,想家乡,受了委曲,用话来宽慰解劝。
谁知后来一年一月竟常常的如此,\jia{补潇湘馆常文也。
}
把这个样儿看惯了,也都不理论了。
\zhu{理论:理会。
}所以没人去理,由他去闷坐,\geng{所谓“久病床前少孝子”是也。
}只管睡觉去了。
那林黛玉倚着床栏杆,两手抱着膝,\jia{画美人秘诀。
}眼睛含着泪,\geng{前批的画美人秘诀,今竟画出《金闺夜坐图》来了。
}好似木雕泥塑\jia{木是旃檀,\zhu{旃檀:音“沾谈”,梵语音译,即檀香。
}泥是金沙方可。
}的一般,直坐到三更多天方才睡了。
一宿无话。
\par
至次日乃是四月二十六日,原来这日未时交芒种节。
尚古风俗:凡交芒种节的这日,都要设摆各色礼物,祭饯花神,言芒种一过,便是夏日了,众花皆卸,花神退位,\geng{无论事之有无,看去有理。
}须要饯行。
然闺中更兴这件风俗,所以大观园中之人都早起来了。
那些女孩子,或用花瓣柳枝编成轿马的,或用绫锦纱罗叠成干旄旌幢的,\zhu{干旄(旄音“毛”)旌(旌音“京”)幢(幢音“床”):干通“竿”。
旄:牦牛尾。
干旄:古代饰牦牛尾于旗竿,以示威仪。
旌:与旄相似,另有五彩鸟羽装饰。
幢:形状像伞。
}都用彩线系了。
每一颗树每一枝花上,都系上了这些物事。
\zhu{物事:泛指东西。
}满园中绣带飘飖,
\zhu{飘飖:现在一般写作“飘摇”。}
花枝招展,\jia{数句大观园景倍胜省亲一回,在一园人俱得闲闲寻乐上看,彼时只有元春一人闲耳。
} \geng{数句抵省亲一回文字,反觉闲闲有趣有味的领略。
}更又兼这些人打扮得桃羞杏让,燕妒莺惭,\jia{桃、杏、燕、莺是这样用法。
}一时也道不尽。
\par
且说宝钗、迎春、探春、惜春、李纨、凤姐\geng{写凤姐随大众一笔,不见红玉一段则认为泛文矣。
何一丝不漏若此。
畸笏。
}等并巧姐、大姐、香菱与众丫鬟们都在园内顽耍,独不见林黛玉。
迎春因说道:“林妹妹怎么不见?好个懒丫头!这会子还睡觉不成?”宝钗道:“你们等着,我去闹了他来。
”说着便丢下众人,一直的往潇湘馆来。
正走着,只见文官等十二个女孩子也来了,\geng{一人不漏。
}见宝钗问了好,说了一回闲话。
宝钗回身指道:“他们都在那里呢,你们找去罢。
我叫林姑娘去就来。
”说着便往潇湘馆来。
\jia{安插一处,好写一处,正一张口难说两家话也。
}忽然抬头见宝玉进去了,宝钗便站住,低头想了一想:宝玉和林黛玉是从小一处长大,他二人间多有不避嫌疑之处,嘲笑喜怒无常;\geng{道尽二玉连日事。
}况且黛玉素习猜忌,好弄小性儿。
此刻自己也进去,一则宝玉不便,二则黛玉嫌疑,\jia{道尽黛玉每每小性,全不在宝钗\sout{身}[心]上。
}倒是回来的妙。
\zhu{
第八回,宝玉在梨香院闻到宝钗吃的冷香丸的香气时,黛玉忽然到来了。
第十九回,宝玉在潇湘馆编故事逗黛玉时,宝钗忽然来了。
这次宝钗反而不进去了。
}
\par
想毕,抽身要寻别的姊妹去,忽见前面一双玉色蝴蝶,大如团扇,一上一下的迎风翩跹,十分有趣。
宝钗意欲扑了来顽耍,遂向袖中取出扇子来,向草地下来扑。
\jia{可是一味知书识礼女夫子行止?写宝钗无不相宜。
}
只见那一双蝴蝶忽起忽落,来来往往,穿花度柳,将欲过河。
倒引的宝钗蹑手蹑脚的,一直跟到池中的滴翠亭,香汗淋漓,娇喘细细,\geng{若玉兄在,必有许多张罗。
}也无心扑了。
\geng{原是无可无不可。
}刚欲回来,只听亭子里面嘁嘁喳喳有人说话。
\jia{无闲纸闲笔之文如此。
}原来这亭子四面俱是游廊曲桥,盖在池中,周围都是雕镂槅子糊着纸。
\par
\chai{baochai}{宝钗捕蝶}
宝钗在亭外听见说话,便站住往里细听,\geng{这桩风流案,又一体写法,甚当。
己卯冬夜。
}只听说道:“你瞧瞧这手帕子,果然是你丢的那块,你就拿着;要不是,就还芸二爷去。
”又有一人道:“可不是那块!拿来给我罢。
”又听说道:“你拿什么谢我呢?难道白寻了来不成。
”又答道:“我既许了谢你,自然不哄你。
”又听说道:“我寻了来给你,自然谢我;但只是拣的人,你就不拿什么谢他?”又回道:“你别胡说。
他是个爷们家,拣了我们的东西,自然该还的。
叫我拿什么给他呢?”又听说道:“你不谢他,我怎么回他呢?况且他再三再四的和我说了,若没谢的,不许给你呢。
”半晌,又听答道:“也罢,拿我这个给他,就算谢他的罢。
——你要告诉别人呢?须说个誓来。
”又听说道:“我要告诉一个人,就长一个疔,
\zhu{
疔[dīng]:中医指病理变化快并引起全身症状的一种毒疮。形小根深,坚硬如钉,多长在颜面和四肢末梢。也说疔疮。
}
日后不得好死!”又听说道:“嗳呀!咱们只顾说话,看有人来悄悄的在外头听见。
\geng{岂敢。
}\geng{这是自难自法,好极好极!惯用险笔如此。
壬午夏,雨窗。
}不如把这槅子都推开了,\geng{“贼起飞智”,\zhu{贼起飞智:指贼人往往会急中生智。
}不假。
}便是有人见咱们在这里,他们只当我们说顽话呢。
若走到跟前,咱们也看的见,就别说了。
”\par
宝钗在外面听见这话,心中吃惊,\jia{四字写宝钗守身如此。
}想道:“怪道从古至今那些奸淫狗盗的人,心机都不错。
\geng{道尽矣。
}这一开了,见我在这里,他们岂不臊了。
况才说话的语音儿,大似宝玉房里的红儿。
他素习眼空心大,最是个头等刁钻古怪的东西。
今儿我听了他的短儿,一时人急造反,狗急跳墙,不但生事,而且我还没趣。
如今便赶着躲了,料也躲不及,少不得要使个‘金蝉脱壳’的法子。
”\zhu{金蝉脱壳:蝉由幼虫变为成虫时,要脱掉外壳(蝉蜕)。
喻以假象作掩蔽暗中溜走。
“金蝉脱壳计”是古代“三十六计”第二十一计。
}犹未想完,只听“咯吱”一声,宝钗便故意放重了脚步,\geng{闺中弱女机变,如此之便,如此之急。
}笑着叫道:“颦儿,我看你往那里藏!”一面说,一面故意往前赶。
那亭子里的红玉、坠儿刚一推窗,只见宝钗如此说着往前赶,\geng{此句实借红玉反写宝钗也,勿得认错作者章法。
}两个人都唬怔了。
宝钗反向他二人笑道:“你们把林姑娘藏在那里了?”\geng{像极!好煞,妙煞!焉的不拍案叫绝!}坠儿道:“何曾见林姑娘了。
”宝钗道:“我才在河边看着他在这里蹲着弄水儿的。
我要悄悄的唬他一跳,还没走到跟前,他倒看见我了,朝东一绕就不见了。
必是藏在这里头了。
”\geng{像极!是极!}一面说,一面故意进去寻了一寻,抽身就走,口里说道:“一定又是在那山子洞里去。
遇见蛇,咬一口也罢了。
”一面说一面走,心里又好笑:\jia{真弄婴儿,
\zhu{弄婴儿:形容成熟的宝钗对付尚且稚嫩的红玉和坠儿,手段十分老练。}
轻便如此,即余至此亦要发笑。
}这件事算遮过去了,不知他二人是怎么样。
\ping{宝钗在“紧要”关头“嫁祸”黛玉,这属于女孩子玩闹,可能从后果上来说没有严重后果,但是也体现了此时宝钗自然流露的对于黛玉的嫌隙。
}\par
谁知红玉听见了宝钗的话,便信以为真,\jia{宝钗身份。
} \geng{实有这一句的。
}让宝钗去远,便拉坠儿道:“了不得了!林姑娘蹲在这里,一定听了话去了!”\geng{移东挪西,任意写去,却是真有的。
}坠儿听说,也半日不言语。
红玉又道:“这可怎么样呢?”\jia{二句系黛玉身份。
}坠儿道:“便听见了,管谁筋疼,各人干各人的就完了。
”\geng{勉强话。
}红玉道:“若是宝姑娘听见,还倒罢了。
林姑娘嘴里又爱刻薄人,心里又细,他一听见了,倘或走露了,怎么样呢?”二人正说着,只见文官、香菱、司棋、待书等上亭子来了。
二人只得掩住这话,且和他们顽笑。
\par
只见凤姐儿站在山坡上招手叫红玉,红玉连忙弃了众人,跑至凤姐前,笑问:“奶奶使唤作什么?”凤姐打量了一打量,见他生的干净俏丽,说话知趣,因说道:“我的丫头今儿没跟进来。
我这会子想起一件事来,要使唤个人出去,可不知你能干不能干,说的齐全不齐全?”红玉道:“奶奶有什么话,只管吩咐我说去。
若说不齐全,误了奶奶的事,凭奶奶责罚罢了。
”\jia{操必胜之券。
红儿机括志量,自知能应阿凤使令意。
}凤姐笑道:“你是谁房里的?\geng{反如此问。
}我使你出去,他回来找你,我好替你答应。
”\geng{问那小姐为此。
}红玉道:“我是宝二爷房里的。
”凤姐听了笑道:“嗳哟!你原来是宝玉房里的,怪道呢。
\jia{“哎哟”“怪道”四字,一是玉兄手下无能为者。
\ping{凤姐认为宝玉房里没有能干的人,所以会对宝玉房里能干的小红感觉诧异。
}前文打量生的“干净俏丽”四字,合而观之,小红则活现于纸上矣。
}\geng{夸赞语也。
}也罢了,你到我家,告诉你平姐姐:外头屋里桌子上汝窑盘子架儿底下放着一卷银子,那是一百二十两,给绣匠的工价,等张材家的来要,当面称给他瞧了,再给他拿去。
\geng{一件。
}再里头屋里床上有个小荷包,拿了来给我。
”\geng{二件。
}\par
红玉听了,撤身去了。
回来只见凤姐不在这山坡上了。
因见司棋从山洞里出来,站着系裙子,\geng{小点缀。
}一笑。
便上来问道:“姐姐,不知道二奶奶往那去了?”司棋道:“没理论。
”\geng{妙极!}红玉听了,又往四下里看,只见那边探春、宝钗在池边看鱼。
红玉便走来陪笑问道:“姑娘们可看见二奶奶没有?”探春道:“往大奶奶院里找去。
”红玉听了,才往稻香村来,顶头只见\geng{又一折。
}晴雯、绮霰、碧痕、紫绡、麝月、待书、入画、莺儿等一群人来了。
晴雯一见了红玉,便说道:“你只是疯罢!花儿也不浇,雀儿也不喂,茶炉子也不爖,\zhu{爖:音“龙”,即爖火,生火。
}就在外头逛。
”\geng{必有此数句,方引出称心得意之语来。
再不用本院人见小红,此差只几分遂心。
\zhu{差[chāi]:被委派做的事;公务。这里指凤姐让红玉传话事。}
}
红玉道:“昨儿二爷说了,今儿不用浇花,过一日再浇罢。
我喂雀儿的时候,姐姐还睡觉呢。
”碧痕道:“茶炉子呢?”\jia{岔一人问,俱是不受用意。
}红玉道:“今儿不是我爖的班儿,有茶没茶别问我。
”绮霰道:“你听听他的嘴!你们别说了,让他逛去罢。
”红玉道:“你们再问问我逛了没有。
二奶奶才使唤我说话取东西去的。
”\jia{非小红夸耀,系尔等逼出来的,离怡红意已定矣。
}说着将荷包举给他们看,\geng{得意!称心如意,在此一举荷包。
}方没言语了,\jia{众女儿何苦自讨之。
}大家分路走开。
晴雯冷笑道:“怪道呢!原来爬上高枝儿去了,把我们不放在眼里。
不知说了一句半句话,名儿姓儿知道了不曾呢,就把他兴的这样!这一遭儿半遭儿的算不得什么,过了后儿还得听呵!\zhu{呵:大声责备。
}有本事的从今儿出了这园子,长长远远的在高枝儿上才算得。
”\geng{虽是醋语,却\sout{与}[过]下无痕。
}一面说着走了。
\par
这里红玉听说,也不便分证,只得忍着气来找凤姐。
到了李氏房中,果见凤姐在那里说话儿呢。
红玉便上来回道:“平姐姐说,奶奶刚出来了,他就把银子收起来了,\jia{交代不在盘架下了。
}才张材家的来取,\zhu{才:刚刚。
}当面称了给他拿去了。
”说着将荷包递了上去,\geng{两件完了。
}又道:“平姐姐叫我回奶奶:旺儿进来讨奶奶的示下,好往那家子去的。
平姐姐就把这话按着奶奶的主意打发他去了。
”凤姐笑道:“他怎么按我的主意打发去了?”\jia{可知前红玉云“就把那按奶奶的主意”是欲俭,但恐累赘耳,故阿凤有是问,彼能细答。
}红玉道:“平姐姐说:我们奶奶问这里奶奶好。
原是我们二爷不在家,虽然迟了两天,只管请奶奶放心。
等五奶奶\jia{又一门。
}好些,我们奶奶还会了五奶奶来瞧奶奶呢。
五奶奶前儿打发人来说,舅奶奶\jia{又一门。
}带了信来了,问奶奶好,还要和这里的姑奶奶\jia{又一门。
}寻两丸延年神验万全丹。
若有了,奶奶打发人来,只管送在我们奶奶这里。
明儿有人去,就顺路给那边舅奶奶带去的。
”\ping{小红话传的好,间接也是平儿思维清楚利落,把这些事儿整理好了。
明写小红能干,暗写平儿能干,一笔写两人。
}\par
话未说完,\geng{又一润色。
}李纨笑道:“嗳哟哟!\jia{红玉今日方遂心如意,却为宝玉后伏线。
}这话我就不懂了。
什么‘奶奶’‘爷爷’的一大堆。
”凤姐笑道:“怨不得你不懂,这是四五门子的话呢。
”说着又向红玉笑道:“好孩子,倒难为你说的齐全。
别像他们扭扭捏捏蚊子似的。
\geng{写死假斯文。
}嫂子不知道,如今除了我随手使的这几个人之外,我就怕和别人说话。
他们必定把一句话拉长了作两三截儿,咬文咬字,拿着腔,哼哼唧唧的,急的我冒火。
先时我们平儿也是这么着,我就问着他:必定装蚊子哼哼,难道就是美人了?\geng{贬杀,骂杀。
}说了几遭才好些了。
”李宫裁笑道:“都像你破落户才好。
”凤姐又道:“这个丫头就好。
\jia{红玉听见了吗?}方才说话虽不多,听那口气就简断。
”\jia{红玉此刻心内想:可惜晴雯等不在傍。
}说着又向红玉笑道:“你明儿伏侍我去罢。
我认你作女儿,我再调理调理,你就出息了。
”\geng{不假。
}\par
红玉听了,扑哧一笑。
凤姐道:“你怎么笑?你说我年轻,比你能大几岁,就作你的妈了?你别作春梦呢!你打听打听,这些人都比你大的大的,赶着我叫妈,我还不理呢!”红玉笑道:“我不是笑这个,我笑奶奶认错了辈数了。
我妈是奶奶的女儿,\geng{所以说“比你大的大的”。
}
这会子又认我作女儿。
”凤姐道:“谁是你妈?”\geng{晴雯说过。
}李宫裁道:“你原来不认得他?他就是林之孝之女。
”\jia{管家之女,而晴卿辈挤之,招祸之媒也。
}
\ping{
李纨在丈夫贾珠去世前,作为王夫人的大儿媳,应该也像凤姐一样是管家,所以会对下人的情况很熟悉。
无奈贾珠去世后,李纨只能守寡,从权力中心淡出。
前文李纨说听不懂四五门奶奶。其实作为前任管家,李纨并非真不懂,可能是在新管家凤姐面前装糊涂。
}
凤姐听了十分诧异,因笑道:“哦!原来是他的丫头。
”\jia{传神。
}又笑道:“林之孝两口子都是锥子扎不出一声儿来的。
我成日家说,他们倒是配就了的一对,夫妻一双天聋地哑。
\jia{用的是阿凤口角。
}那里承望养出这么个伶俐丫头来!你十几岁了?”红玉道:“十七岁了。
”又问名字,\jia{真真不知名,可叹!}红玉道:“原叫红玉的,因为重了宝二爷,如今叫红儿了。
”\par
凤姐听了将眉一皱,把头一回,说道:“讨人嫌的很!\geng{又一下针。
}得了玉的益似的,你也玉,我也玉。
”\ping{这里可能暗中透露凤姐并不喜欢宝玉和黛玉,故借题发挥。
}因说道:“既这么着,上月\foot{“上月”,原作“肯跟”,庚本同。
此处作“肯跟”不通,“肯”疑为“上”、“月”连写之误,依列本、杨本改。
}我还和他妈说,‘赖大家的如今事多,也不知这府里谁是谁,你替我好好的挑两个丫头我使’,他一般答应着。
他饶不挑,\zhu{饶:即使,尽管,表示让步关系。
}倒把他这女孩子送了别处去。
难道跟我必定不好?”李氏笑道:“你可是又多心了。
他进来在先,你说话在后,怎么怨得他妈呢!”凤姐道:“既这么着,明儿我和宝玉说,叫他再要人,\jia{有悌弟之心。
}叫这丫头跟我去。
可不知本人愿意不愿意?”\jia{总是追写红玉十分心事。
}红玉笑道:“愿意不愿意,我们不敢说。
\jia{好答!可知两处俱是主儿。
}只是跟着奶奶,我们也学些眉眼高低,\geng{千愿意万愿意之言。
}出入上下,大小的事也得见识见识。
”\jia{且系本心本意,“狱神庙”回内方见。
}\geng{奸邪婢岂是怡红应答者,故即逐之。
前良儿,后篆儿,便是确证。
作者又不得可也。
\zhu{
可:表示同意。
这句话的意思是,批书人自认为发现了离开宝玉的丫鬟的特点是“奸邪”,看透了作者的小心思,会使作者嘴硬拒绝承认自己被看透。
}
己卯冬夜。
}\geng{此系未见“抄没”、“狱神庙”诸事,故有是批。
\zhu{是批:指的是前面把红玉当作“奸邪婢”的批语。
}丁亥夏。
畸笏。
}\ping{仆人没有自由,需主人裁决,不能自己擅作主张决定自己主人是谁。
所以小红虽然愿意换个主人,但是不能以自己主动的姿态提出,所以她说“我们不敢说”。
}刚说着,只见王夫人的丫头来请,\geng{截得真好。
}凤姐便辞了李宫裁去了。
红玉回怡红院,\geng{好,接得更好。
}
不在话下。
\par
如今且说林黛玉因夜间失寐,次日起迟了,闻得众姊妹都在园中作饯花会,恐人笑他痴懒,连忙梳洗了出来。
刚到了院中,只见宝玉进门来了,笑道:“好妹妹,昨儿可告我不曾?\jia{明知无是事,不得不作开谈。
}
叫我悬了一夜心。
”\geng{并不为告悬心。
}林黛玉便回头叫紫鹃道:\jia{不见宝玉,阿颦断无此一段闲言,总在欲言不言难禁之意,了却“情情”之正文也。
}\geng{倒像不曾听见的。
}“把屋子收拾了,下一扇纱屉子;
\zhu{
纱屉子:旧时的窗户分两层,里面一层是用纱绷上的,透明、通气,称“纱屉子”。
外面一层是用纸糊或木板装的,白天可以卸下来或支起,晚间再安上或放下。
}
看那大燕子回来,把帘子放下来,拿狮子倚住;\zhu{狮子:这里是一种压帘用的带座的小石狮子。
}烧了香,就把炉罩上。
”一面说一面仍往外走。
宝玉见他这样,还认作是昨日中晌的事,\jia{毕真不错。
}那知晚间的这段公案,还打恭作揖的。
黛玉正眼也不看,各自出了院门,一直找别的姊妹去了。
宝玉心中纳闷,自己猜疑:看起这个光景来,不像是为昨日的事;但只昨日我回来的晚了,又没见他,再没有冲撞了他的去处。
\geng{毕真不错。
}一面想,一面走,又由不得从后面追了来。
\par
只见宝钗、探春正在那边看仙鹤,\geng{二玉文字岂是容易写的,故有此截。
} \geng{《石头记》用截法、岔法、突然法、伏线法、由近渐远法、将繁改简法、重作轻抹法、虚敲实应法种种诸法,总在人意料之外,且不曾见一丝牵强,所谓“信手拈来无不是”是也。
\zhu{
重作:重要的创作需要重笔着力描写。轻抹:这里指用轻描淡写。
重作轻抹表现了作者写作功力之深,举重若轻。
}
}见黛玉来了,三个一同站着说话儿。
又见宝玉来了,探春便笑道:“宝哥哥,身上好?整整三天没见了。
”\jia{横云截岭,好极,妙极!二玉文原不易写,《石头记》得力处在兹。
}宝玉笑道:“妹妹身上好?我前儿还在大嫂子跟前问你呢。
”探春道:“哥哥往这里来,我和你说话。
”\geng{是移一处语。
}宝玉听说,便跟了他,来到一棵石榴树下。
探春因说道:“这几天老爷可叫你没有?”\jia{老爷叫宝玉再无喜事,故园中合宅皆知。
}宝玉道:“没有叫。
”探春道:“昨儿我恍惚听见说老爷叫你出去的。
”宝玉笑道:“那想是别人听错了,并没叫的。
”\jia{非谎也,避繁也。
} \geng{怕文繁。
}探春又笑道:“这几个月,我又攒下有十来吊钱了。
你还拿去,明儿逛去的时候,或是好字画、书籍卷册、轻巧顽意儿,给我带些来。
”\geng{若无此一岔,二玉和合则成嚼蜡文字。
《石头记》得力处正此。
丁亥夏。
畸笏叟。
}宝玉道:“我这么城里城外、大廊小庙的逛,也没见个新奇精致东西,左不过是金玉铜器、\zhu{左不过:反正,只不过,无非。
}没处撂的古董,再就是绸缎、吃食、衣服了。
”探春道:“谁要那些。
像你上回买的那柳枝儿编的小篮子,整竹子根抠的香盒儿,胶泥垛的风炉儿,这就好。
把我喜欢的什么似的,\ping{这真是看惯金玉不稀罕,反稀罕民间小机巧物件。
}谁知他们都爱上了,都当宝贝似的抢了去了。
”宝玉笑道:“原来要这个。
这不值什么,拿五百钱出去给小子们,管拉两车来。
”\geng{不知物理艰难,公子口气也。
}探春道:“小厮们知道什么。
你拣那朴而不俗、直而不拙者,\jia{是论物?是论人?看官着眼。
}这些东西,你多多的替我带了来。
我还像上回的鞋作一双你穿,比那双还加工夫,如何呢?”\par
宝玉笑道:“你提起鞋来,我想起个故事来:那一回我穿着,可巧遇见了老爷,\geng{补遗法。
}老爷就不受用,问是谁作的。
我那里敢提‘三妹妹’三个字,我就回说是前儿我的生日,是舅母给的。
老爷听了是舅母给的,才不好说什么,半日还说:‘何苦来!虚耗人力,作践绫罗,作这样的东西。
’因而我回来告诉袭人,袭人说这还罢了,赵姨娘气的抱怨的了不得:‘正经兄弟,\geng{指环哥。
}鞋搭拉袜搭拉的\jia{何至如此,写妒妇信口逗。
\zhu{逗:引逗,撩拨。
}}没人看见,且作这些东西!’”探春听说,登时沉下脸来,道:“你说,这话糊涂到什么田地!怎么我是该做鞋的人么?环儿难道没有分例的,没有人的?衣裳是衣裳,鞋袜是鞋袜,丫头、老婆一屋子,怎么抱怨这些话!给谁听呢!我不过闲着没有事,作一双半双的,爱给那个哥哥兄弟,随我的心。
谁敢管我不成!这也是他气的?”宝玉听了,点头笑道:“你不知道,他心里自然又有个想头了。
”探春听说,一发动了气,将头一扭,说道:“连你也糊涂了!他那想头自然有的,不过是那阴微鄙贱的见识。
他只管这么想,我只管认得老爷、太太两个人,别人我一概不管。
就是姊妹弟兄跟前,谁和我好,我就和谁好,什么偏的庶的,我也不知道。
论理我不该说他,但他特昏愦的不像了!还有笑话儿呢:\jia{开一步,妙妙!}就是上回我给你那钱,替我带那顽的东西。
过了两天,他见了我,也是说没钱使,怎么难,我也不理论。
谁知后来丫头们出去了,他就抱怨起我来,说我攒了钱为什么给你使,倒不给环儿使了。
我听见这话,又好笑又好气,我就出来往太太屋里去了。
”\geng{这一节特为“兴利除弊”一回伏线。
\zhu{第六十五回。
}}\ping{前文探春求宝玉带点东西,宝玉是有点敷衍的,而且探春还得想着做双鞋来报答。
对比起薛蟠和宝钗这一对兄妹,异母兄妹还是差了些。
而且探春一出生就被告知自己为主母为奴,母亲还是个见识一般的,探春此时也是还小,说出不在意母亲话刺耳,可是十一二岁的探春可能还觉着自己和母亲弟弟划清界限就能被自己的父亲家族的人同等对待,简直就像现代香蕉人。
这里对探春,觉得可恨但更多觉得可怜了。
}正说着,只见宝钗那边笑道:\geng{截得好。
}“说完了,来罢。
显见的是哥哥妹妹了,丢下别人,且说梯己去。
\zhu{梯己:意即私人的、贴心的。
这里是指私下里的知心话。
}
我们听一句儿就使不得了!”说着,探春、宝玉二人方笑着来了。
\par
宝玉因不见林黛玉,\jia{兄妹话虽久长,心事总未少歇,接得好。
}便知他是躲了别处去了,想了一想,越性迟两日,\jia{作书人调侃耶?}等他的气消一消再去也罢了。
因低头看见许多凤仙、石榴等各色落花,锦重重的落了一地,\geng{不因见落花,宝玉如何突至埋香冢?不至埋香冢,如何写《葬花吟》?《石头记》无闲文闲字正此。
丁亥夏。
畸笏叟。
}因叹道:“这是他心里生了气,也不收拾这花儿了。
待我送了去,明儿再问他。
”\jia{至埋香冢方不牵强,好情理。
}说着,只见宝钗约着他们往外头去。
\jia{收拾的干净。
}宝玉道:“我就来。
”说毕,等他二人去远了,\jia{怕人笑说。
}便把那花兜了起来,登山渡水,过树穿花,一直奔了那日同林黛玉葬桃花的去处。
将已到了花冢\foot{此句原无,据诸本补。
},\geng{新鲜。
}犹未转过山坡,只听山坡那边有呜咽之声,一行数落着,\zhu{一行[xíng]:一边。
}
哭的好不伤感。
\jia{奇文异文,俱出《石头记》上,且愈出愈奇文。
}宝玉心中想道:“这不知是那房里的丫头,受了委曲,\jia{岔开线络,活泼之至!}
跑到这个地方来哭。
”一面想,一面煞住脚步,听他哭道是:\jia{诗词歌赋,如此章法写于书上者乎?}
\geng{诗词文章,试问有如此行笔者乎?}
\par
\jia{“开生面”、“立新场”,是书多多矣,惟此回\sout{处}[更]生更新。
非颦儿断无是佳吟,非石兄断无是情聆。
难为了作者了,故留数字以慰之。
}\geng{“开生面”、“立新场”是书不止“红楼梦”一回,惟是回更生更新,且读去非阿颦无是佳吟,非石兄断无是章法行文,愧杀古今小说家也。
畸笏。
}\par
\hop
花谢花飞飞满天,红消香断有谁怜?\par
游丝软系飘春榭,落絮轻沾扑绣帘。
\zhu{游丝:飘荡在半空中的断掉的蜘蛛丝。
}\par
闺中女儿惜春暮,愁绪满怀无释处,\par
手把花锄出绣帘,忍踏落花来复去。
\par
柳丝榆荚自芳菲,不管桃飘与李飞。
\par
桃李明年能再发,明年闺中知有谁?\ping{花在人亡。
}\par
三月香巢已垒成,梁间燕子太无情!\par
明年花发虽可啄,却不道人去梁空巢也倾。
\par
一年三百六十日,风刀霜剑严相逼,\par
明媚鲜妍能几时,一朝飘泊难寻觅。
\par
花开易见落难寻,阶前闷杀葬花人,\par
独倚花锄泪暗洒,洒上空枝见血痕。
\par
杜鹃无语正黄昏,荷锄归去掩重门。
\par
青灯照壁人初睡,冷雨敲窗被未温。
\par
怪奴底事倍伤神,半为怜春半恼春:\zhu{奴:古人谦称自己(原来男女都可以用,后多用于年轻女子)。底事:甚么事。
底:何?}\par
怜春忽至恼忽去,至又无言去不闻。
\par
昨宵庭外悲歌发,知是花魂与鸟魂?\par
花魂鸟魂总难留,鸟自无言花自羞。
\par
愿奴胁下生双翼,随花飞到天尽头。
\zhu{胁:胸部的两侧。
}\par
天尽头,何处有香丘?\par
未若锦囊收艳骨,一抔净土掩风流。
\zhu{一抔净土:抔[póu]:掬。
一抔:一捧,双手捧物。
《史记·张释之列传》:“取长陵一抔土”,比喻盗开坟墓。
后人就以“一抔土”代指坟墓。
这里“一抔净土”指花冢。
}\ping{葬花亦是葬人。
}\par
质本洁来还洁去,强于污淖陷渠沟。
\zhu{淖[nào]:烂泥;泥沼。}
\par
尔今死去侬收葬,未卜侬身何日丧?\zhu{侬:我。
}\par
侬今葬花人笑痴,他年葬侬知是谁?\par
试看春残花渐落,便是红颜老死时。
\par
一朝春尽红颜老,花落人亡两不知!\par
\jia{余读《葬花吟》至再至三四,其凄楚感慨,令人身世两忘,举笔再四不能加批。
有客曰:“先生身非宝玉,何能下笔?即字字双圈,批词通仙,料难遂颦儿之意。
俟看过玉兄后文再批。
”噫嘻!阻余者想亦《石头记》来的?故停笔以待。
}\geng{余读《葬花吟》凡三阅,其凄楚感慨,令人身世两忘,举笔再四不能加批。
}\geng{先生想身非宝玉,何得而下笔?即字字双圈,料难遂颦儿之意。
俟看过玉兄后文再批。
}\geng{噫嘻!客亦《石头记》化来之人!故掷笔以待。
}\par
\hop
宝玉听了,不觉痴倒。
要知端底,再看下回。
\par


 \jia{饯花辰不论典与不典,只取其韵致生趣耳。
\hang
池边戏蝶,偶尔适兴;亭外急智脱壳。
明写宝钗非拘拘然一女夫子。
\hang
凤姐用小红,可知晴雯等埋没其人久矣,无怪有私心私情。
且红玉后有宝玉大得力处,此于千里外伏线也。
\hang
《石头记》用截法、岔法、突然法、伏线法、由近渐远法、将繁改简法、重作轻抹法、虚敲实应法种种诸法,总在人意料之外,且不曾见一丝牵强,所谓“信手拈来无不是”是也。
\hang
不因见落花,宝玉如何突至埋香冢;不至埋香冢又如何写《葬花吟》。
\hang
埋香冢葬花乃诸艳归源,《葬花吟》又系诸艳一偈也。
}\par
\qi{总评:幸逢知己无回避,密语隔窗怕有人。
\zhu{这两句是指小红和坠儿在滴翠亭谈论与贾芸交换手帕,“知己”指小红和坠儿。}
总是关心浑不了,叮咛嘱咐为轻春。
\zhu{
这两句当是指宝玉和探春的谈话。因为探春是庶出,这是探春的一块心病,时刻警惕因此被人看轻,
这就是“叮咛嘱咐为轻春”的含意所在,“轻春”即探春怕被人轻看、小瞧。
}
\ping{“轻春”令人费解。
}\hang
心事将谁告,花飞动我悲。
埋香吟哭后,日日敛双眉。
}
\dai{053}{滴翠亭杨妃戏彩蝶}
\dai{054}{埋香冢飞燕泣残红}
\sun{p27-1}{杨妃戏蝶金蝉脱壳}{图左侧:四月二十六日芒种节这一天,民俗要祭饯花神,大观园里更是绣带飘摇、花枝招展。
图上侧:宝钗忽见了一对玉色蝴蝶, 遂取出扇子来扑,追到滴翠亭边,忽听亭里有人叽叽喳喳说话,细听原来是两个女孩在说男女情事。
“咱们只顾说,要有人悄悄在外头听呢,不如把这窗子推开吧。
”宝钗怕惹出是非,故意放重脚步, 笑着叫道:“颦儿,看你往哪里藏!”小红、坠儿刚推开窗,只听宝钗说着往前赶,两人都吓怔了。
}
\sun{p27-2}{小红传话凤姐赞叹,探春观舞邀兄私语,埋香冢飞燕泣残红}{图右上:见凤姐招手,小红赶过去问,凤姐要找人传话,小红笑道;“奶奶只管吩咐, 误了奶奶的事责罚就是了。
”凤姐见她干净俏丽,说话利落,很是喜欢,要她到屋里做丫头。
图下侧:宝钗探春正在看鹤舞。
探春缠着要宝玉给她买轻巧雅致的玩意。
图左上:宝玉去寻黛玉,犹未转过山坡,只听那边有呜咽之声。
细细听来,原来是黛玉在亦吟亦哭道:“花谢花飞飞满天,红消香断有谁怜……”宝玉听了,不觉痴倒。
}