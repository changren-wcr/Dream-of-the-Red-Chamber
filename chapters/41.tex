\chapter{栊翠庵茶品梅花雪 \quad 怡红院劫遇母蝗虫}
\geng{此回栊翠品茶,怡红遇劫。
盖妙玉虽以清静无为自守,而怪洁之癖未免有过,老妪只污得一杯,见而勿用,岂似玉兄日享洪福,竟至无以复加而不自知。
故老妪眠其床,卧其席,酒屁熏其屋,却被袭人遮过,则仍用其床其席其屋。
亦作者特为转眼不知身后事写来作戒,纨绔公子可不慎哉?}\par
\qi{任呼牛马从来乐,随分清高方可安。
\zhu{前句指刘姥姥甘心受作弄出洋相,以讨贾母等人欢心;后句指妙玉过于清高,需要随便一点,不那么狷傲,才能保得平安。}
自古世情难意拟,淡妆浓抹有千般。
立松轩。
}\par
话说刘姥姥两只手比着说道:“花儿落了结个大倭瓜。
”众人听了哄堂大笑起来。
于是吃过门杯,因又逗趣笑道:“实告诉说罢,我的手脚子粗笨,又喝了酒,仔细失手打了这磁杯。
有木头的杯取个子来,我便失了手,掉了地下也无碍。
”众人听了,又笑起来。
凤姐儿听如此说,便忙笑道:“果真要木头的,我就取了来。
可有一句话先说下:这木头的可比不得磁的,他都是一套,定要吃遍一套方使得。
”刘姥姥听了心下敁敠道:\zhu{敁敠:音“颠多”,也写作“掂掇”,估量、盘算、斟酌的意思。
}“我方才不过是趣话取笑儿,谁知他果真竟有!我时常在村庄乡绅大家也赴过席,金杯银杯倒都也见过,从来没见有木头杯之说。
哦,是了,想必是小孩子们使的木碗儿,不过诓我多喝两碗。
别管他,横竖这酒蜜水儿似的,多喝点子也无妨。
”\geng{为登厕伏脉。
}想毕,便说:“取来再商量。
”凤姐乃命丰儿:“到前面里间屋,书架子上有十个竹根套杯取来。
”丰儿听了答应,才要去,鸳鸯笑道:“我知道你这十个杯还小。
况且你才说是木头的,这会子又拿了竹根子的来,倒不好看。
不如把我们那里的黄杨根整抠的十个大套杯拿来,灌他十下子。
”凤姐儿笑道:“更好了。
”鸳鸯果命人取来。
刘姥姥一看,又惊又喜:惊的是一连十个挨次大小分下来,那大的足似个小盆子,第十个极小的还有手里的杯子两个大;喜的是雕镂奇绝,一色山水、树木、人物,并有草字以及图印。
因忙说道:“拿了那小的来就是了,怎么这样多?”凤姐儿笑道:“这个杯没有喝一个的理。
我们家因没有这大量的,所以没人敢使他。
姥姥既要,好容易寻了出来,必定要挨次吃一遍才使得。
”刘姥姥唬的忙道:“这个不敢。
好姑奶奶,饶了我罢。
”\meng{挟炎的苦恼。
}
贾母、薛姨妈、王夫人知道他上了年纪的人,禁不起,忙笑道:“说是说,笑是笑,不可多吃了,只吃这头一杯罢。
”刘姥姥道:“阿弥陀佛!我还是小杯吃罢。
把这大杯收着,我带了家去慢慢的吃罢。
”说的众人又笑起来。
鸳鸯无法,只得命人满斟了一大杯,刘姥姥两手捧着喝。
贾母薛姨妈都道:“慢些,不要呛了。
”\par
薛姨妈又命凤姐儿布了菜。
凤姐笑道:“姥姥要吃什么,说出名儿来,我搛了喂你。
”\zhu{搛:音“兼”,夹持,此指用筷子夹菜。
}刘姥姥道:“我知什么名儿,样样都是好的。
”贾母笑道:“你把茄鲞搛些喂他。
”\zhu{茄鲞:茄干。
鲞:音“享”,原指干鱼、腊鱼,亦泛指成片或成丁的腌腊食品。
}凤姐儿听说,依言搛些茄鲞送入刘姥姥口中,因笑道:“你们天天吃茄子,也尝尝我们的茄子弄的可口不可口。
”刘姥姥笑道:“别哄我了,茄子跑出这个味儿来了,我们也不用种粮食,只种茄子了。
”众人笑道:“真是茄子,我们再不哄你。
”刘姥姥诧异道:“真是茄子?我白吃了半日。
姑奶奶再喂我些,这一口细嚼嚼。
”凤姐果又搛了些放入口内。
刘姥姥细嚼了半日,笑道:“虽有一点茄子香,只是还不像是茄子。
告诉我是个什么法子弄的,我也弄着吃去。
”凤姐儿笑道:“这也不难。
你把才下来的茄子把皮\qian 了,\zhu{\qian :音“千”,削。
}只要净肉,切成碎钉子,用鸡油炸了,再用鸡脯子肉并香菌、新笋、蘑菇、五香腐干、\zhu{腐干:豆腐干。
}各色干果子,都切成钉子,拿鸡汤煨干,将香油一收,外加糟油一拌,\zhu{糟油:用酒糟调制的油,用来浇拌凉菜。
}盛在磁罐子里封严,要吃时拿出来,用炒的鸡瓜一拌就是。
”\zhu{鸡瓜:鸡的腱子肉或胸脯肉。
因其长圆如瓜形,故称。
一说即鸡丁。
}
刘姥姥听了,摇头吐舌说道:“我的佛祖!倒得十来只鸡来配他,怪道这个味儿!”\par
一面说笑,一面慢慢的吃完了酒,还只管细玩那杯。
凤姐笑道:“还是不足兴,再吃一杯罢!”刘姥姥忙道:“了不得,那就醉死了。
我因为爱这样范,\zhu{样范:模型,榜样。
这里是模样的意思。
}亏他怎么作了。
”鸳鸯笑道:“酒吃完了,到底这杯子是什么木的?”刘姥姥笑道:“怨不得姑娘不认得,你们在这金门绣户的,如何认得木头!我们成日家和树林子作街坊,困了枕着他睡,乏了靠着他坐,荒年间饿了还吃他,眼睛里天天见他,耳朵里天天听他,口儿里天天讲他,所以好歹真假,我是认得的。
让我认一认。
”\meng{好充懂的来看。
}一面说,一面细细端详了半日,道:“你们这样人家断没有那贱东西,那容易得的木头,你们也不收着了。
我掂着这杯体重,断乎不是杨木,这一定是黄松做的。
”众人听了,哄堂大笑起来。
\par
只见一个婆子走来请问贾母,说:“姑娘们都到了藕香榭,请示下,就演罢还是再等一会子?”贾母忙笑道:“可是倒忘了他们,就叫他们演罢。
”那个婆子答应去了。
不一时,只听得箫管悠扬,笙笛并发。
正值风清气爽之时,那乐声穿林度水而来,自然使人神怡心旷。
宝玉先禁不住,拿起壶来斟了一杯,一口饮尽。
\meng{作者似曾在座。
}复又斟上,才要饮,只见王夫人也要饮,命人换暖酒,宝玉连忙将自己的杯捧了过来,送到王夫人口边,\geng{妙极!忽写宝玉如此,便是天地间母子之至情至性。
献芹之民之意,\zhu{献芹:古时有人认为芹菜的味道很美,就向乡豪称赞,乡豪尝后,却觉得很难吃。
见《列子·杨朱篇》。
后比喻自己欣赏的事物推荐给别人,却无法获得认同。
亦用为人对所献东西或意见的自谦之词。
后人常用“献芹”、“芹意”等作为送礼或请客的谦词。
}令人酸鼻。
}王夫人便就他手内吃了两口。
一时暖酒来了,宝玉仍归旧坐,王夫人提了暖壶下席来,众人皆都出了席,薛姨妈也立起来,贾母忙命李、凤二人接过壶来:“让你姑妈坐了,大家才便。
”王夫人见如此说,方将壶递与凤姐,自己归坐。
贾母笑道:“大家吃上两杯,今日着实有趣。
”说着擎杯让薛姨妈,又向湘云宝钗道:“你姐妹两个也吃一杯。
你妹妹虽不大会吃,也别饶他。
”说着自己已干了。
湘云、宝钗、黛玉也都干了。
当下刘姥姥听见这般音乐,且又有了酒,越发喜的手舞足蹈起来。
宝玉因下席过来向黛玉笑道:“你瞧刘姥姥的样子。
”黛玉笑道:“当日圣乐一奏,百兽率舞,\zhu{圣乐一奏,百兽率舞:语出《尚书·虞书·益稷》,意思是舜时乐器一响,百兽全都随乐起舞。
}如今才一牛耳。
”\meng{随笔写来,趣极。
}众姐妹都笑了。
\par
须臾乐止,薛姨妈出席笑道:“大家的酒想也都有了,且出去散散再坐罢。
”贾母也正要散散,于是大家出席,都随着贾母游玩。
贾母因要带着刘姥姥散闷,遂携了刘姥姥至山前树下盘桓了半晌,又说与他这是什么树,这是什么石,这是什么花。
刘姥姥一一的领会,又向贾母道:“谁知城里不但人尊贵,连雀儿也是尊贵的。
偏这雀儿到了你们这里,他也变俊了,也会说话了。
”众人不解,因问什么雀儿变俊了,会讲话。
刘姥姥道:“那廊下金架子上站的绿毛红嘴是鹦哥儿,我是认得的。
那笼子里的黑老鸹子怎么又长出凤头来,\zhu{黑老鸹子怎么又长出凤头来:这里实指八哥。
黑老鸹(鸹音“瓜”)子,即乌鸦。
八哥与乌鸦形近,喙部上端多一撮凤毛(即凤头)。
}也会说话呢。
”众人听了都笑将起来。
\par
一时只见丫鬟们来请用点心。
贾母道:“吃了两杯酒,倒也不饿。
也罢,就拿了这里来,大家随便吃些罢。
”丫鬟便去抬了两张几来,又端了两个小捧盒。
揭开看时,每个盒内两样:这盒内一样是藕粉桂糖糕,一样是松穰鹅油卷;\zhu{松:松子仁。
穰:同“瓤”。
}那盒内一样是一寸来大的小饺儿,贾母因问什么馅儿,婆子们忙回是螃蟹的。
贾母听了,皱眉说:“这油腻腻的,谁吃这个!”那一样是奶油炸的各色小面果,也不喜欢。
因让薛姨妈吃,薛姨妈只拣了一块糕;贾母拣了一个卷子,只尝了一尝,剩的半个递与丫鬟了。
刘姥姥因见那小面果子都玲珑剔透,便拣了一朵牡丹花样的笑道:“我们那里最巧的姐儿们,也不能铰出这么个纸的来。
我又爱吃,又舍不得吃,包些家去给他们做花样子去倒好。
”\meng{世上竟有这样人。
}众人都笑了。
贾母道:“家去我送你一坛子。
你先趁热吃这个罢。
”别人不过拣各人爱吃的一两点就罢了;刘姥姥原不曾吃过这些东西,且都作的小巧,不显盘堆的,\zhu{不显盘堆:点心没有摆得很多、堆得很高,只有少量几个,所以刘姥姥和板儿吃了一些就没了半盘子,因为本来就没多少。
}他和板儿每样吃了些,就去了半盘子。
剩的,凤姐又命攒了两盘并一个攒盒,与文官等吃去。
忽见奶子抱了大姐儿来,大家哄他顽了一会。
那大姐儿因抱着一个大柚子玩的,忽见板儿抱着一个佛手,便也要佛手。
\geng{小儿常情遂成千里伏线。
}丫鬟哄他取去,大姐儿等不得,便哭了。
众人忙把柚子与了板儿,\meng{伏线千里。
}将板儿的佛手哄过来与他才罢。
那板儿因顽了半日佛手,此刻又两手抓着些果子吃,又忽见这柚子又香又圆,更觉好顽,且当球踢着玩去,也就不要佛手了。
\geng{柚子即今香\sout{团}[橼]之属也,应与“缘”通。
佛手者,正指迷津者也。
\ping{
可能是说佛手指点迷津,出现佛手的位置就是指点读者,透露结局的地方。
神佛的手在困境中起援助,暗示刘姥姥救助落难的大姐。
}以小儿之戏暗透前后通部脉络,隐隐约约,毫无一丝漏泄,岂独为刘姥姥之俚言博笑而有此一大回文字哉?}\meng{画工。
\zhu{工:精巧。
}}\ping{板儿和大姐交换手里玩的东西,可能暗伏贾府败落之后,刘姥姥搭救大姐,最后和板儿成婚,即所谓的“缘”。
}\par
当下贾母等吃过茶,又带了刘姥姥至栊翠庵来。
妙玉忙接了进去。
至院中见花木繁盛,贾母笑道:“到底是他们修行的人,没事常常修理,比别处越发好看。
”一面说,一面便往东禅堂来。
\zhu{禅堂:犹言佛堂,僧尼参禅礼佛的地方。
}妙玉笑往里让,贾母道:“我们才都吃了酒肉,你这里头有菩萨,冲了罪过。
我们这里坐坐,把你的好茶拿来,我们吃一杯就去了。
”妙玉听了,忙去烹了茶来。
宝玉留神看他是怎么行事。
只见妙玉亲自捧了一个海棠花式雕漆填金云龙献寿的小茶盘,\zhu{雕漆:将涂上许多层漆的铜胎或木胎烘干、磨光后,再雕出立体花纹的技术。
 填金:指在雕漆及花纹的凹处,用金漆填满。
云龙献寿:指漆器的花纹,为云纹和龙纹衬托着寿字,呈群星拱月状。
应该是以雕漆的方式呈现的花纹。
}里面放一个成窑五彩小盖钟,\zhu{成窑:指明代成化年间官窑所出的瓷器,以五彩者为上。
盖钟:有盖的小杯。
钟:同“盅”。
盅[zhōng]:没有柄的小杯子。
}捧与贾母。
贾母道:“我不吃六安茶。
”\zhu{六安茶:产于安徽省六安县。
}妙玉笑说:“知道。
这是老君眉。
”\zhu{老君眉:湖南洞庭湖君山所产的银针茶,形如长眉,故名“老君眉”。
历代沿作贡品。
一说六安银针即老君眉。
}贾母接了,又问是什么水。
妙玉笑回:“是旧年蠲的雨水。
”\zhu{蠲:通“涓”,清洁。
这里是密闭封存使之澄清的意思。
}贾母便吃了半盏,便笑着递与刘姥姥说:“你尝尝这个茶。
”刘姥姥便一口吃尽,笑道:“好是好,就是淡些,再熬浓些更好了。
”贾母众人都笑起来。
然后众人都是一色官窑脱胎填白盖碗。
\zhu{
官窑:专为供应宫廷所需而设的瓷窑,始于北宋大观、政和年间。
脱胎:指薄胎瓷器,因极薄,映光可以透见指纹,似乎釉层之内已经脱去胎骨,故名。
填白:即甜白,是在有暗花刻纹的薄胎器面上挂一层透明釉,温润如玉,若无胎骨,给人以“甜润”的感受。
另一种说法,填白即可填画彩的填白器。清代蓝浦《景德镇陶录》:“所谓填白,盖纯白器可填画彩者。”
盖碗:一种上有盖、下有托,中有碗的汉族茶具。
又称“三才碗”、“三才杯”,盖为天、托为地、碗为人,暗含天地人和之意。
}\par
那妙玉便把宝钗和黛玉的衣襟一拉,二人随他出去,宝玉悄悄的随后跟了来。
只见妙玉让他二人在耳房内,
\zhu{耳房:像耳朵一样位置在正房两侧的小房子。}
宝钗坐在榻上,黛玉便坐在妙玉的蒲团上。
\zhu{蒲团:用蒲草、高粱叶或玉米皮等编成的圆形垫子。
是农村中常见的坐具,也是僧尼、道士坐禅或跪拜的用具。
}妙玉自向风炉上扇滚了水,另泡一壶茶。
宝玉便走了进来,笑道:“偏你们吃梯己茶呢。
”二人都笑道:“你又赶了来飺茶吃。
\zhu{飺:音“雌”,窃视之意,引申为伺机讨要,沾光、揩油。
}这里并没你的。
”妙玉刚要去取杯,只见道婆收了上面的茶盏来。
妙玉忙命:“将那成窑的茶杯别收了,搁在外头去罢。
”宝玉会意,知为刘姥姥吃了,他嫌脏不要了。
\ping{妙玉表面是出家人,但俗世的事样样在乎,吃茶杯盏讲究,在意茶客身份,这种洁癖就是所谓的“欲洁何曾洁”里的“欲洁”。
}又见妙玉另拿出两只杯来。
一个旁边有一耳,
\zhu{耳:像耳朵一样位置在两侧的,这里指杯子把手。}
杯上镌着“\gua 瓟斝”三个隶字,\zhu{\gua 瓟斝:音“班袍甲”,\gua 瓟:均葫芦类。
斝:古代酒器,圆口平底。
\gua 瓟斝:用一斝形模子套在小\gua 瓟上,使之按模子的形状成长,成型后去子风干做饮器。
一说是一种特制的饮器,状似\gua 瓟,故名。
}后有一行小真字是“晋王恺珍玩”,\zhu{真字:楷书。
}又有“宋元丰五年四月眉山苏轼见于秘府”一行小字。
\zhu{王恺:晋代著名的富豪,喜蓄珍奇宝物。
这里所谓“王恺珍玩”和“苏轼见于秘府”等语,乃小说家言,意在写其珍贵。
秘府:又称秘阁,古代宫廷中藏图书秘珍的地方。
}妙玉便斟了一斝,递与宝钗。
那一只形似钵而小,\zhu{钵:音“波”,泛称可盛酒、装东西或洗涤东西的圆形金属或陶瓷器具。
}也有三个垂珠篆字,\zhu{垂珠篆字:或即垂露篆字。
相传为汉郎中曹喜所创,笔划断续成小点,犹如串串垂珠或点点轻露,故名。
}镌着“点犀\qiao ”。
\zhu{点犀\qiao:犀牛角做成的饮器。
\qiao :碗类器皿。
\qiao 以“点犀”取名,似借李商隐《无题》诗“心有灵犀一点通”诗意,极言此\qiao 之珍贵。
一说“点犀”应作“杏犀”:一般犀角制成的器皿,不论白天或灯光下,都呈不透明的灰褐色,只有上好的犀角制成的器皿,对着光看,呈半透明的杏黄色,但极罕见。
}
妙玉斟了一\qiao 与黛玉。
仍将前番自己常日吃茶的那只绿玉斗来斟与宝玉。
\ping{妙玉洁癖,刘姥姥用过的杯子都不要,这里却把自己用的给宝玉,可能暗示了妙玉对宝玉的情愫。
}宝玉笑道:“常言‘世法平等’,\zhu{世法平等:佛家语。
即平等地对待世间的一切事物。
《金刚经》:“是法平等,无有高下。
”法:梵文“达磨”的意译,指信条、规范等,有时亦通指一切事物。
}
\ping{宝玉暗中讽刺,佛教说世法平等,修行的妙玉却看不起贫婆子刘姥姥。}
他两个就用那样古玩奇珍,我就是个俗器了。
”\ping{宝玉用的“俗器”是妙玉吃茶用的,可能借助宝玉的口讽刺待人不平等的妙玉也是“俗”的,因为妙玉对刘姥姥十分嫌弃。
}妙玉道:“这是俗器?不是我说狂话,只怕你家里未必找的出这么一个俗器来呢。
”宝玉笑道:“俗说‘随乡入乡’,\zhu{随乡入乡:即“入乡随俗”,到什么地方就顺从那个地方的风俗习惯。
语本《庄子·山木》:“入其俗,从其俗。
”比喻能适应环境,随遇而安。
}到了你这里,自然把那金玉珠宝一概贬为俗器了。
”
\ping{宝玉巧妙解释了为什么自己把妙玉用的绿玉斗叫做“俗器”的原因,为自己解围。}
妙玉听如此说,十分欢喜,遂又寻出一只九曲十环一百二十节蟠虬整雕竹根的一个大\hai 出来,\zhu{
九曲十环一百二十节:竹根器以盘曲多节者为贵。
蟠:盘曲。
虬:音“求”,《离骚》王逸注:“有角曰龙,无角曰虬”;但是另一种说法是龙子之有角者曰虬。
整雕竹根:用整块竹根雕刻。
\hai :音“海”,大杯。
}笑道:“就剩了这一个,你可吃的了这一海?”\zhu{海:这里指容量大的器皿。
今犹称大碗为海碗。
}宝玉喜的忙道:“吃的了。
”妙玉笑道:“你虽吃的了,也没这些茶糟蹋。
\geng{茶下“糟蹋”二字,成窑杯已不屑再要,妙玉真清洁高雅,然亦怪谲孤僻甚矣。
实有此等人物,但罕耳。
}岂不闻‘一杯为品,二杯即是解渴的蠢物,三杯便是饮牛饮骡了’。
你吃这一海便成什么?”说的宝钗、黛玉、宝玉都笑了。
妙玉执壶,只向海内斟了约有一杯。
宝玉细细吃了,果觉轻浮无比,\zhu{轻浮:言茶味不凡。
}赏赞不绝。
妙玉正色道:“你这遭吃的茶是托他两个福,独你来了,我是不给你吃的。
”宝玉笑道:“我深知道的,我也不领你的情,只谢他二人便是了。
”妙玉听了,方说:“这话明白。
”\ping{妙玉可能在掩饰自己对于宝玉的独特情感。
}黛玉因问:“这也是旧年的雨水?”妙玉冷笑道:“你这么个人,竟是大俗人,连水也尝不出来。
这是五年前我在玄墓蟠香寺住着,\zhu{玄墓:山名,在今江苏吴县。
相传东晋郁泰玄葬此,故名。
}收的梅花上的雪,共得了那一鬼脸青的花瓮一瓮,\zhu{
鬼脸:指戏剧表演时所用的鬼神面具。鬼脸青:一种深蓝色。这里指一种釉色深青的瓷。
}
总舍不得吃,埋在地下,\meng{妙手。
层层叠起,竟能以他人所画之天王作众神矣。
\zhu{妙玉出现,使得本来是“天王”的黛玉“泯然众人”成为“众神”。
}}今年夏天才开了。
我只吃过一回,这是第二回了。
你怎么尝不出来?隔年蠲的雨水那有这样轻浮,如何吃得。
”黛玉知他天性怪僻,不好多话,亦不好多坐,吃过茶,便约着宝钗走了出来。
\ping{黛玉也是遇到对手了。
}\ping{妙玉很矫情,不过也是反映了她“云空未必空”。
}\par
\chai{miaoyu}{妙玉奉茶}
宝玉和妙玉陪笑道:“那茶杯虽然脏了,白撂了岂不可惜?依我说,不如就给那贫婆子罢,他卖了也可以度日。
你道可使得。
”妙玉听了,想了一想,点头说道:“这也罢了。
幸而那杯子是我没吃过的,若我使过,我就砸碎了也不能给他。
\meng{更奇!世上我也见过此等人。
}你要给他,我也不管你,只交给你,快拿了去罢。
”宝玉道:“自然如此,你那里和他说话授受去,越发连你也脏了。
\meng{人若忘形,最喜此等言语。
}只交与我就是了。
”\ping{灵石虽是蠢物,好歹是真仙带入红尘,秉性里仍存有慈悲。
假尼姑就算了,不过都是十几岁的孩子,矫情也是正常的。
}妙玉便命人拿来递与宝玉。
宝玉接了,又道:“等我们出去了,我叫几个小幺儿来河里打几桶水来洗地如何?”\zhu{小幺儿:身边使唤的小仆人。
幺(音“妖”):幼小。
}妙玉笑道:“这更好了,只是你嘱咐他们,抬了水只搁在山门外头墙根下,别进门来。
”\meng{偏于无可写处,深入一层。
}宝玉道:“这是自然的。
”说着,便袖着那杯,递与贾母房中小丫头拿着,说:“明日刘姥姥家去,给他带去罢。
”交代明白,贾母已经出来要回去。
妙玉亦不甚留,送出山门,回身便将门闭了。
不在话下。
\par
且说贾母因觉身上乏倦,便命王夫人和迎春姊妹陪了薛姨妈去吃酒,自己便往稻香村来歇息。
凤姐忙命人将小竹椅抬来,贾母坐上,两个婆子抬起,凤姐李纨和众丫鬟婆子围随去了,不在话下。
这里薛姨妈也就辞出。
王夫人打发文官等出去,将攒盒散与众丫鬟们吃去,自己便也乘空歇着,随便歪在方才贾母坐的榻上,命一个小丫头放下帘子来,又命他捶着腿,吩咐他:“老太太那里有信,你就叫我。
”说着也歪着睡着了。
\par
宝玉湘云等看着丫鬟们将攒盒搁在山石上,也有坐在山石上的,也有坐在草地下的,也有靠着树的,也有傍着水的,倒也十分热闹。
一时又见鸳鸯来了,要带着刘姥姥各处去逛,\meng{又另是一番气象。
}众人也都赶着取笑。
一时来至“省亲别墅”的牌坊底下,刘姥姥道:“嗳呀!这里还有个大庙呢。
”说着,便爬下磕头。
众人笑弯了腰。
刘姥姥道:“笑什么?这牌楼上字我都认得。
我们那里这样的庙宇最多,都是这样的牌坊,那字就是庙的名字。
”众人笑道:“你认得这是什么庙?”刘姥姥便抬头指那字道:“这不是‘玉皇宝殿’四字?”众人笑的拍手打脚,还要拿他取笑。
刘姥姥觉得腹内一阵乱响,忙的拉着一个小丫头,要了两张纸就解衣。
众人又是笑,又忙喝他“这里使不得!”忙命一个婆子带了东北上去了。
那婆子指与地方,便乐得走开去歇息。
\par
那刘姥姥因喝了些酒,他脾气不与黄酒相宜,\zhu{脾气:这里犹言脾胃。
}且吃了许多油腻饮食,发渴多喝了几碗茶,不免通泻起来,蹲了半日方完。
及出厕来,酒被风禁,\zhu{禁:克制,抑制。
酒被风禁:风把酒约束在体内散不出来,所以刘姥姥会醉。
}且年迈之人,蹲了半天,忽一起身,只觉得眼花头眩,辨不出路径。
四顾一望,皆是树木山石楼台房舍,却不知那一处是往那里去的了,只得认着一条石子路慢慢的走来。
及至到了房舍跟前,又找不着门,再找了半日,忽见一带竹篱,刘姥姥心中自忖道:“这里也有扁豆架子。
”一面想,一面顺着花障走了来,得了一个月洞门进去。
\zhu{花障:有花草攀附的篱笆。
月洞门:圆形的门洞。
}只见迎面忽有一带水池,只有七八尺宽,石头砌岸,里边碧浏清水流往那边去了,\zhu{浏:音“刘”,水流清亮的样子。
}
\meng{借刘姥姥醉中,写境中景。
}上面有一块白石横架在上面。
刘姥姥便度石过去,顺着石子甬路走去,转了两个弯子,只见有一房门。
于是进了房门,只见迎面一个女孩儿,满面含笑迎了出来。
刘姥姥忙笑道:“姑娘们把我丢下来了,要我碰头碰到这里来。
”说了,只觉那女孩儿不答。
刘姥姥便赶来拉他的手,“咕咚”一声,便撞到板壁上,\zhu{板壁:分隔房间的木板墙。
}把头碰的生疼。
细瞧了一瞧,原来是一幅画儿。
刘姥姥自忖道:“原来画儿有这样活凸出来的。
”一面想,一面看,一面又用手摸去,却是一色平的,点头叹了两声。
一转身方得了一个小门,门上挂着葱绿撒花软帘。
刘姥姥掀帘进去,抬头一看,只见四面墙壁玲珑剔透,琴剑瓶炉皆贴在墙上,
\zhu{第十七回,贾政等人到怡红院看见“满墙满壁,皆系随依古董玩器之形抠成的槽子。
诸如琴、剑、悬瓶、桌屏之类,虽悬于壁,却都是与壁相平的。”}
锦笼纱罩,金彩珠光,连地下踩的砖,皆是碧绿凿花,竟越发把眼花了,找门出去,那里有门?左一架书,右一架屏。
刚从屏后得了一门转去,只见他亲家母也从外面迎了进来。
刘姥姥诧异,忙问道:“你想是见我这几日没家去,亏你找我来。
那一位姑娘带你进来的?”他亲家只是笑,不还言。
刘姥姥笑道:“你好没见世面,见这园里的花好,你就没死活戴了一头。
”他亲家也不答。
便心下忽然想起:“常听大富贵人家有一种穿衣镜,这别是我在镜子里头呢罢。
”说毕伸手一摸,再细一看,可不是,四面雕空紫檀板壁将镜子嵌在中间。
因说:“这已经拦住,如何走出去呢?”一面说,一面只管用手摸。
这镜子原是西洋机括,\zhu{机括:旧称弩的发箭器叫“机”,矢末扣弦之处叫“括”。
这里指一触即动的开关装置,也叫“消息”或“机关”。
}可以开合。
不意刘姥姥乱摸之间,其力巧合,便撞开消息,掩过镜子,露出门来。
刘姥姥又惊又喜,迈步出来,忽见有一副最精致的床帐。
他此时又带了七八分醉,又走乏了,便一屁股坐在床上,只说歇歇,不承望身不由己,前仰后合的,朦胧着两眼,一歪身就睡熟在床上。
\par
且说众人等他不见,板儿见没了他姥姥,急的哭了。
众人都笑道:“别是掉在茅厕里了?快叫人去瞧瞧。
”因命两个婆子去找,回来说没有。
众人各处搜寻不见。
袭人敁敠其道路:“是他醉了迷了路,顺着这一条路往我们后院子里去了。
若进了花障子到后房门进去,虽然碰头,还有小丫头们知道;若不进花障子再往西南上去,若绕出去还好,若绕不出去,可够他绕回子好的。
我且瞧瞧去。
”一面想,一面回来,进了怡红院便叫人,谁知那几个房子里小丫头已偷空顽去了。
\par
袭人一直进了房门,转过集锦槅子,\zhu{槅子:架子,放置器物的木器。
木架上分不同形状的许多层小格,格内可放入各种器皿、用具。
也作“格子”。
集锦槅(槅音“隔”)子:又称“多宝塔”、“博古架”,多以贵重木料制成各种形状的槅子,可摆设各种珍奇古物,是我国古代建筑内檐装修隔断的一种。
}就听的鼾齁如雷。
忙进来,只闻见酒屁臭气,满屋一瞧,只见刘姥姥扎手舞脚的仰卧在床上。
\zhu{扎手舞脚:形容伸开手、蹬着腿的样子。}
\ping{竟然是宝玉的床,当然是宝玉的床,只能是宝玉的床,写这书的时候,作者已经不是沉溺皮肉滥淫的蠢物,这一场胡闹欢笑说不定都是为了让刘姥姥胡乱躺倒在怡红院里打鼾放屁,写书时作者也是有点矫枉过正的意思了,红楼里外表极丑恶肮脏往往代表性灵及其通透高尚,宝玉十几岁睡在雕梁画栋玉石床上,可这床刘姥姥也睡得,待红楼旧梦逝去,宝玉也是个跛脚道人癞头和尚。
}袭人这一惊不小,慌忙赶上来将他没死活的推醒。
那刘姥姥惊醒,睁眼见了袭人,连忙爬起来道:“姑娘,我失错了!并没弄脏了床帐。
”一面说,一面用手去掸。
袭人恐惊动了人,被宝玉知道了,只向他摇手,不叫他说话。
忙将鼎内贮了三四把百合香,仍用罩子罩上。
些须收拾收拾,所喜不曾呕吐,忙悄悄的笑道:“不相干,有我呢。
你随我出来。
”\meng{这方是袭人的平素,笔至此不得不屈,再增支派则累[赘]矣。
}刘姥姥跟了袭人,出至小丫头们房中,命他坐了,向他说道:“你就说醉倒在山子石上打了个盹儿。
”刘姥姥答应知道。
\meng{总是恰好便住。
}又与他两碗茶吃,方觉酒醒了,因问道:“这是那个小姐的绣房,这样精致?我就像到了天宫里的一样。
”袭人微微笑道:“这个么,是宝二爷的卧室。
”那刘姥姥吓的不敢作声。
袭人带他从前面出去,见了众人,只说他在草地下睡着了,带了他来的。
众人都不理会,也就罢了。
\par
一时贾母醒了,就在稻香村摆晚饭。
贾母因觉懒懒的,也不吃饭,便坐了竹椅小敞轿,回至房中歇息,命凤姐儿等去吃饭。
他姊妹方复进园来。
要知端的——\par
\qi{总评:刘姥姥之憨从利,妙玉尼之怪图名,宝玉之奇、黛玉之妖亦自敛迹。
是何等画工,能将他人之天王,作我卫护之神祗?\zhu{黛玉宝玉是本书的主人公,是被作者浓墨重彩描写的人物,故称“天王”,但是在本回,刘姥姥和妙玉的独特行为性格,盖过了这两个主角的光芒,昔日的主角“天王”成为了配角“卫护之神祗”。
}文技至此,可为至矣!}
\dai{081}{栊翠庵茶品梅花雪}
\dai{082}{刘姥姥醉卧怡红院}
\sun{p41-1}{栊翠庵品茶,刘姥姥拜庙}{贾母带刘姥姥到栊翠庵来,妙玉相迎进去,亲自捧茶与贾母。
贾母吃了半盏,递与刘姥姥,刘姥姥一口吃进,笑道:“好是好,就是淡些。
”贾母众人都笑起来。
图右上:妙玉又单独领宝钗、黛玉去耳房,宝玉悄悄跟进。
妙玉自向风炉上扇滚了水,另泡了一壶茶。
图右下:吃了茶,贾母要回去,妙玉亦不甚留,送出山门。
图左侧:鸳鸯等人继续带刘姥姥闲逛,来到“省亲别墅”的牌坊下,刘姥姥以为是庙,便跪地磕头,众人皆笑。
}