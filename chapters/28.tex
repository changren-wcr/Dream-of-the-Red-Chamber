\chapter{蒋玉菡情赠茜香罗\quad 薛宝钗羞笼红麝串}
\geng{茜香罗、红麝串写于一回,盖琪官虽系优人,后回与袭人供奉玉兄宝卿得同终始者,非泛泛之文也。
\hang
自“闻曲”回以后,回回写药方,是白描颦儿添病也。
}\par
话说林黛玉只因昨夜晴雯不开门一事,错疑在宝玉身上。
至次日,又可巧遇见饯花之期,正是一腔无明正未发泄,\zhu{无明:佛教用语。
意译为“痴”,即“没有智慧”。
佛家认为,人的种种烦恼痛苦,是由“无明”引起的。
后也称发火动怒为“无明火起”,无明便成为怒火的代称。
}又勾起伤春愁思,因把些残花落瓣去掩埋,由不得感花伤己,哭了几声,便随口念了几句。
不想宝玉在山坡上,听见是黛玉之声,先不过是点头感叹;次后听到“侬今葬花人笑痴,他年葬侬知是谁”,“一朝春尽红颜老,花落人亡两不知”等句,不觉恸倒山坡之上,怀里兜的落花撒了一地。
试想林黛玉的花颜月貌,将来亦到无可寻觅之时,宁不心碎肠断!\ping{伏黛玉之死。
}既黛玉终归无可寻觅之时,推之于他人,如宝钗、香菱、袭人等,亦可到无可寻觅之时矣。
宝钗等终归无可寻觅之时,则自己又安在哉?且自身尚不知何在何往,则斯处、斯园、斯花、斯柳,又不知当属谁姓矣!\ping{伏抄家之事。
}因此一而二,二而三,反复推求了去,\jia{不言炼句炼字辞藻工拙,只想景想情想事想理,反复推求悲伤感慨,乃玉兄一生之天性。
真颦儿之知己,玉兄外实无一人。
想昨阻批《葬花吟》之客,嫡是玉兄之化身无疑。
余几作点金为铁之人,\sout{笨}[幸]甚\sout{笨}[幸]甚!\foot{庚辰本末句作“幸甚幸甚”;甲戌本末句作“笨甚笨甚”。
末句应为“幸甚幸甚”为是。
甲戌本致误原因是:传抄过程中某本误“幸”为“本”(此本已有第一回“是书何本”、第二十五回“看书人亦要如是看为本”两误例),后之传抄者以“本”字不通,据文意改为音同形近的“笨”字。
}}
\geng{百转千回矣。
}真不知此时此际欲为何等蠢物,杳无所知,逃大造,出尘网,\zhu{大造、尘网:皆泛指人间。
大造:大自然创造、化育万物,指宇宙。
尘网:比喻人在世间被名利声色束缚,如在网中不得解脱。
}使可解释这段悲伤。
\zhu{解释:消除,消释。
}\jia{非大善知识,说不出这句话来。
}正是:\par
花影不离身左右,鸟声只在耳东西。
\jia{二句作禅语参。
}
\jia{一大篇《葬花吟》却如此收拾,真好机杼笔法,\zhu{杼:音“住”,织布用的梭子。
机杼:指织布机,这里比喻创作诗文的巧思、布局与结构。
}令人焉得不叫绝称奇!}\par
那黛玉正自悲伤,忽听山坡上也有悲声,心下想道:“人人都笑我有些痴病,难道还有一个痴子不成?”\jia{岂敢岂敢。
}想着,抬头一看,见是宝玉。
林黛玉看见,便道:“啐!我当是谁,原来是这个狠心短命的……”刚说到“短命”二字,又把口掩住,\jia{“情情”,不忍道出“的”字来。
}长叹了一声,\geng{不忍也。
}自己抽身便走了。
\par
这里宝玉悲恸了一回,见黛玉去了,便知黛玉看见他躲开了,自己也觉无味,抖抖土起来,下山寻归旧路,\jia{折得好,誓不写开门见山文字。
}
往怡红院来。
可巧\geng{哄人字眼。
}看见林黛玉在前头走,连忙赶上去,说道:“你且站住。
我知你不理我,我只说一句话,从今以后撂开手。
”\jia{非此三字难留莲步,玉兄之机变如此。
}林黛玉回头见是宝玉,待要不理他,听他说“只说一句话,从此撂开手”,这话里有文章,少不得站住说道:“有一句话,请说来。
”宝玉笑道:“两句话,说了你听不听?” 
\jia{相离尚远,用此句补空,好近阿颦。
}黛玉听说,回头就走。
\geng{走得是。
}宝玉在身后面叹道:“既有今日,何必当初!”\jia{自言自语,真是一句话。
}林黛玉听见这话,由不得站住,回头道:“当初怎么样?今日怎么样?”宝玉叹道:\jia{以下乃答言,非一句话也。
}“当初姑娘来了,那不是我陪着顽笑?\jia{我阿颦之恼,玉兄实摸[头]不着,不得不将自幼之苦心实事一诉,方可明心,以白今日之故,勿作闲文看。
}凭我心爱的,姑娘要,就拿去;我爱吃的,听见姑娘也爱吃,连忙干干净净收着等姑娘吃。
一桌子吃饭,一床上睡觉。
丫头们想不到的,我怕姑娘生气,我替丫头们想的到。
我心里想着:姊妹们从小儿长大,亲也罢,热也罢,和气到了头,才见得比人好。
\geng{要紧语。
}如今谁承望姑娘人大心大,\geng{反派不是。
}不把我放在眼里,倒把外四路的什么宝姐姐\zhu{外四路:指关系疏远。
}\geng{心事。
}凤姐姐\jia{用此人瞒看官也,瞒颦儿也。
心动阿颦在此数句也。
一节颇似说辞,玉兄口中却是衷肠话。
}的放在心坎儿上,倒把我三日不理四日不见的。
我又没个亲兄弟亲姊妹——虽然有两个,你难道不知道是和我隔母的?我也和你是独出,只怕同我的心一样。
谁知我是白操了这个心,弄的我有冤无处诉!”说着不觉滴下泪来。
\jia{玉兄泪非容易有的。
}\par
黛玉耳内听了这话,眼内见了这形景,心内不觉灰了大半,也不觉滴下泪来,低头不语。
宝玉见他这般形景,遂又说道:“我也知道我如今不好了,但只凭着怎么不好,万不敢在妹妹跟前有错处。
\geng{有是语。
}
便有一二分错处,你倒是或教导我,戒我下次,\geng{可怜语。
}或骂我两句,打我两下,我都不灰心。
谁知你总不理我,\geng{实难为情。
}叫我摸不着头脑,少魂失魄,不知怎么样才是。
\geng{真有是事。
}就便死了,也是个屈死鬼,任凭高僧高道忏悔也不能超升,\geng{又瞒看官及批书人。
}还得你申明了缘故,我才得托生呢!”\par
黛玉听了这个话,不觉将昨晚的事都忘在九霄云外了,\jia{“情情”本来面目也。
} \geng{“情情”衷肠。
}便说道:“你既这么说,昨儿为什么我去了,你不叫丫头开门?”\geng{正文,该问。
}宝玉诧异道:“这话从那里说起?\geng{实实不知。
}我要是这么样,立刻就死了!”\jia{急了。
}林黛玉啐道:\geng{如闻。
}“大清早死呀活的,也不忌讳。
你说有呢就有,没有就没有,起什么誓呢。
”宝玉道:“实在没有见你去。
就是宝姐姐坐了一坐,\geng{不要兄言,彼已亲睹。
}就出来了。
”林黛玉想了一想,笑道:“想必是你的丫头们懒怠动,丧声歪气的也是有的。
”\ping{黛玉实在好哄,怎么得了尖酸多事的名声?}宝玉道:“想必是这个原故。
等我回去问了是谁,教训教训他们就好了。
”\geng{玉兄口气毕真。
}
黛玉道:“你的那些姑娘们\geng{不快活之称。
}也该教训教训,\geng{照样的妙!}
只是论理我不该说。
今儿得罪了我的事小,倘或明儿宝姑娘来,\geng{也还一句,的是心坎上人。
}什么贝姑娘来,也得罪了,事情岂不大了。
” 
\jia{至此心事全无矣。
}说着抿着嘴笑。
宝玉听了,又是咬牙,又是笑。
\par
二人正说话,只见丫头来请吃饭,\jia{收拾得干净。
}遂都往前头来了。
王夫人见了林黛玉,因问道:“大姑娘,你吃那鲍太医的药可好些?”\geng{是新换了的口气。
}林黛玉道:“也不过这么着。
老太太还叫我吃王大夫的药呢。
”\geng{何如?}\ping{王夫人要黛玉吃鲍太医的药,而贾母要黛玉吃王大夫的药,两人在黛玉吃药上的分歧,折射出两人对于黛玉态度的分歧。
}\ping{黛玉忽然情商下线,直着说不好,王夫人会很尴尬吧。
}宝玉道:“太太不知道,林妹妹是内症,先天生的弱,所以禁不住一点风寒,不过吃两剂煎药,疏散了风寒,还是吃丸药\jia{引下文。
}的好。
”王夫人道:“前儿大夫说了个丸药的名字,我也忘了。
”宝玉道:“我知道那些丸药,不过叫他吃什么人参养荣丸。
”王夫人道:“不是。
”宝玉又道:“八珍益母丸?左归?右归?再不,就是麦味地黄丸。
”\zhu{这几个方剂皆出自明代张景岳的《景岳全书》。
八珍益母丸由益母草、当归、熟地黄、人参、白术等九味药配成,治妇女气血亏损等症。
左归即左归丸,以熟地为君药,配以枸杞子、鹿角胶、龟板胶等共八味药,能滋补肾阴。
右归即右归丸,亦以熟地为君药,再加川附子、肉桂、杜仲等共十味药,功效是温补肾阳。
麦味地黄丸是以六味地黄丸的配方为基础而加以化裁的药丸,君药为熟地黄,加入麦冬、五味子,能滋阴补肝肾,治疗虚劳咳嗽。
}王夫人道:“都不是。
我只记得有个‘金刚’两个字的。
”\jia{奇文奇语。
}宝玉扎手笑道:\zhu{扎手:两手摊开,是一种比较放肆、不大礼貌的姿势。
}\jia{慈母前放肆了。
}  \geng{此写玉兄,亦是释却心中一夜半日要事,故大大一泄。
己卯冬夜。
}“从来也没听见有个什么‘金刚丸’。
若有了‘金刚丸’,自然有‘菩萨散’了!”\jia{宝玉因黛玉事完,一心无挂碍,故不知不觉手之舞之,足之蹈之。
}说的满屋里人都笑了。
宝钗笑道:“想是天王补心丹。
”\zhu{天王补心丹:由酸枣仁、柏子仁、当归、生地黄、人参等十三味药配制成的丸药,补养心神。
出自《世医得效方》。
}\jia{慧心人自应知之。
}王夫人笑道:“是这个名儿。
如今我也糊涂了。
”宝玉道:“太太倒不糊涂,都是叫‘金刚’‘菩萨’支使糊涂了。
”\jia{是语甚对,余幼时所闻之语合符,哀哉伤哉!}\ping{讽刺王夫人盲目信佛导致自己糊涂。
}
王夫人道:“扯你娘的臊!又欠你老子捶你了。
”\geng{伏线。
}宝玉笑道:“我老子再不为这个捶我的。
”\jia{此语亦不假。
}\par
王夫人又道:“既有这个名儿,明日就叫人买些来。
” \geng{写药案是暗度颦卿病势渐加之笔,非泛泛闲文也。
丁亥夏。
畸笏叟。
}宝玉道:“这些药都是不中用的。
太太给我三百六十两银子,我给妹妹配一料丸药,包管一料不完就好了。
”王夫人道:“放屁!什么药就这么贵?”宝玉道:“当真的呢,我这方子比别个不同。
这个药名儿也古怪,一时也说不清。
只讲那头胎紫河车,\geng{只闻名。
}人形带叶参,\yang{三百六十两不足\foot{“三百六十两不足”:此句列藏本缺,杨本为旁添;其馀各本都混入正文(“不足”,戚本蒙本作“还不够”)。
从上下文语气连贯看,此句应为批语。
}。
}龟大何首乌,\geng{听也不曾听过。
}千年松根茯苓胆,\zhu{紫河车即胎盘,功能为补气养血,旧时认为以头胎(初产)者为佳。
人形带叶参即人参,旧时以似人形者为佳,但不能带叶入药,此处云带叶系防止作假,求其货真质佳。
何首乌:补肝肾,益精血,《本草纲目》:“即大而佳”。
茯苓:音“伏灵”,健脾和胃,寄生在松树根上的菌类植物。
《淮南子》:“千年之松,下有茯苓”。
茯苓胆:中药名,即茯神,为寄生于松树根部且抱根生长的茯苓,品质优于茯苓。
《本草纲目·茯苓》:“茯神:〔气味〕甘,平,无毒。〔主治〕辟不祥,疗风眩风虚、五劳口干,止惊悸、多恚怒、善忘,开心益智,安魂魄,养精神。”
为什么可称茯神为茯苓胆呢?似无明确典籍依据,但是我们可以从茯神的形态特征和“胆”的字义上来综合分析。
汉字“胆”,可以指装在器物内部,可以容纳水、空气等的东西,如暖水瓶的瓶胆。
茯神有黑褐色外皮,中药名为茯苓皮;中间为白色菌核组织体包裹着的松根,中药名为茯神木;
若取出中间的松根,茯神在外部形态上就是一个中空的不规则块状体,似中间被掏空的甘薯的形状,极似“胆”的样子,故茯神可以形象地称为茯苓胆。
茯苓和茯神的药物功能相近,但茯神长于安神养心,开心益智,止惊悸、多恚怒,安魂魄,养精神,不但品质胜于茯苓,而且更适合治疗林黛玉的病症。
} \geng{写得不犯冷香丸方子。
前“玉生香”回中颦云“他有金你有玉;他有冷香你岂不该有暖香?”是宝玉无药可配矣。
今颦儿之剂,若许材料皆系滋补热性之药,兼有许多奇物,而尚未拟名,何不竟以“暖香”名之?以代补宝玉之不足,岂不三人一体矣。
己卯冬夜。
}\ping{第十九回:黛玉冷笑道:“难道我也有什么‘罗汉’‘真人’给我些香不成?便是得了奇香,也没有亲哥哥亲兄弟弄了花儿、朵儿、霜儿、雪儿替我炮制。
我有的是那些俗香罢了!”宝钗虽然失去了父亲,但是还有亲哥哥和妈妈,而黛玉伶仃孤苦,孑然一身。
这里宝玉亲自给黛玉调配药,算是了却了黛玉的这个心愿吧。
}诸如此类的药都不算为奇,\geng{还有奇的。
}只在群药里算。
那为君的药,\zhu{为君的药:中医处方中的各味药,根据不同的作用、药量,分为君、臣、佐、使。
在群药中,有一种起主要作用的药,叫“君药”。
君臣佐使:出自《神农本草经》:“上药一百二十种为君,主养命;中药一百二十种为臣,主养性;下药一百二十种为佐使,主治病;用药须合君臣佐使。
”原指君主、臣僚、僚佐、使者四种人分别起着不同的作用,后指中药处方中的各味药的不同作用。
}说起来唬人一跳。
前儿薛大哥求了我有一二年,我才给了他这个方子。
他拿了方子去又寻了二三年,花了有上千的银子,才配成了。
太太不信,只问宝姐姐。
”宝钗听说,笑着摇手儿道:“我不知道,也没听见。
你别叫姨娘问我。
”王夫人笑道:“到底是宝丫头,好孩子,不撒谎。
”宝玉站在当地,听见如此说,一回身把手一拍,说道:“我说的倒是真话呢,倒说我撒谎。
”说着一回身,只见黛玉坐在宝钗身后抿着嘴笑,用手指在脸上画着羞他。
\geng{好看煞,在颦儿必有之。
}\par
\zhu{“人形带叶参三百六十两不足龟大何首乌”如何断句,时有争议。
上海红学会的《红楼梦鉴赏辞典》,把“不足龟”作为一个词条。
说是“不足龟”即“无足龟”即“玳瑁”。
周汝昌在《红楼夺目红》中则认为,“不”乃“六”字之误抄,此处应作:“人形带叶参三百六十两,六足龟,大何首乌,……”\hang
这样的猜测是否有根据呢?\hang
首先我们来看原文。
除己卯本缺此回外,其他各本原文如下:\hang
甲戌本:人形带叶参三百六十两不足龟大何首乌\hang
舒序本,程甲、乙本同。
\hang
庚本同。
“不足龟大何”旁朱批“听也不曾听过”。
\hang
戚本:人形带叶参三百六十两还不够龟大的何首乌(俞平伯校本从。
蔡义江校本据前句作“还不够”,刘世德校本据后句添“的”字),蒙古王府本同。
\hang
列藏本:无“三百六十两不足”\hang
杨本:人形带叶参(旁添“三百六十两不够还有”)龟大的何首乌\hang
甲辰本:人形带叶参三百六十两也不足龟大何首乌(甲戌邓校本据添“也”字。
)\hang
各新校本标点情况:\hang
红研所 1982 版《红楼梦》:“人形带叶参,三百六十两不足。
龟大何首乌”,这是最流行的标法。
红研所 1996 版作“人形带叶参——三百六十两不足——龟大何首乌”。
\hang
其他排印本大多数在不足后断句,也有作“ \; ;”或“ \; ,”的。
\hang
人文社旧版:“人形带叶参,三百六十两不足,龟,大何首乌,”\hang
中华书局启功等校本:“人形带叶参,三百六十两不足龟,大何首乌,”\hang
仔细考察以上各本异文,提出几点看法:\hang
“三百六十两不足”,是解答王夫人“什么药就这么贵?”说的:前面宝玉向母亲要这三百六十两银子,只用来购买“人形带叶参”(或加上“头胎紫河车”)就不够用;戚本、甲辰本说的更清楚(其异文未必是原有,但可以证明向来读者的理解都是一致的)。
列藏本及杨本原文无“三百六十两不足”。
所以,在“足龟”二字中间应该断句。
\hang 
不考虑前人意见,硬把“不足”和“龟”连在一起,作“三百六十两不足龟”也是经不起推敲的。
红研所三版注释引顾玠《海槎录》云:“(玳瑁)大者难得,小者时时有之。
”玳瑁是海龟的一种,体型较大,“平均体重一般可达45到80千克,历史上曾经捕获的最重的玳瑁达到210千克。
”(据维基百科)而三百六十两的玳瑁不到20千克,是小中尤小者,既时时可得,何足以和其他药材相提并论?\hang
周校把“三百六十两”当作“人形带叶参”的剂量,有悖常理,而“六足龟”改动原文,兹不讨论。
\hang
“龟大何首乌”的意思当是“像龟一样大的何首乌”,以龟来状物大小,固然新奇,却也非绝无仅有,现在闽南方言中就有“龟大鳖小”一语,只不过它侧重在“鳖小”,用于形容物品之小而已。
\hang 
另外,宋《开宝重定本草》称何首乌“根大如拳”,明《五杂俎》卷十一谓“何首乌,五十年大如掌……百年大如腕……百五十年大如盆……”。
和“龟”一样,“拳”、“掌”、“腕”也是没有固定大小的,但只要不是故意抬杠,它们的大小还是有形可循的。
综上,“龟大何首乌”一语是可通的。
\hang 
列藏本及杨本原文无“三百六十两不足”,这七个字或是评语混入正文。
红研所1996年二版《红楼梦》的标点:“人形带叶参——三百六十两不足——龟大何首乌,”说明校注者也认为“三百六十两不足”是注释性文字,但格于体例,没有把它删去。
正文中如删去“三百六十两不足”一句,馀下几种药材都带修饰词形容其难得,比较整齐,而语气似觉更流畅些。
}
\par
凤姐因在里间屋里看着人放桌子,\geng{且不接宝玉文字,妙!}听如此说,便走来笑道:“宝兄弟不是撒谎,这倒是有的。
上月薛大哥亲自和我寻珍珠,\zhu{珍珠:产自蚌类壳中,可以入药,有泻热、定惊、镇心、下痰、安魂魄等功能,外用可拔毒生肌,内服时均研粉吞服或与其它药物共配成丸。
珍珠粉要研得极细,否则伤胃,故下文说要“隔面子”。
珍珠入药,不可用曾作首饰及经尸气者,下文宝玉说要古坟里的或作过头面的不过是笑谈,不足为凭。
}我问他作什么,他说是配药。
他还抱怨说,不配也罢了,如今那里知道这么费事。
我问他什么药,他说是宝兄弟的方子,说了多少药,我也没工夫听。
他说:‘不然我也买几颗珍珠了,只是定要头上戴过的,所以来和你寻。
’他说:‘妹妹若没散的,花儿上也得,掐下来,过后儿我拣好的再给妹妹穿了来。
’我没法儿,把两枝珠花现拆了给他。
还要了一块三尺大红库纱去,乳钵乳了隔面子呢。
”\zhu{乳钵乳了隔面子:用乳钵把药研成碎末,再筛出细面。
乳钵:一种研磨药面用的小臼;把药研细叫乳。
隔面子:筛药面子,这里用大红纱筛药面子。
}凤姐说一句,宝玉念一句佛,说:“太阳在屋里呢!”\ping{宝玉的药方应该不是谎话,宝钗可能也知道。
王夫人听到宝玉的药方,已经说了“放屁!什么药就这么贵?”,表示了对于药方的质疑否定,薛宝钗紧跟王夫人,违心的说自己不知道。
从这一段可以明显地看出来两种派系,一是王夫人和支持她的薛宝钗,二是贾宝玉和支持他的王熙凤。
这也可能是暗示了针对金玉良缘和木石前盟的两种不同阵营。
王夫人更喜欢和自己更近的外甥女薛宝钗,而不喜欢相对较远的丈夫的外甥女林黛玉。
王熙凤不和王夫人在一个阵营,可能原因在于如果贾宝玉娶了薛宝钗的话,薛宝钗既善于管理(第五十五、五十六回,宝钗参与大观园管理),也识文断字,还是王夫人的亲儿媳,处处比自己强,自己肯定会失去权势;而如果娶的是林黛玉,林黛玉在家务管理上并不热衷,也没有施展过能力,自己可能会保住现有的权势。
}凤姐说完了,宝玉又道:“太太想,这不过是将就呢。
正经按那方子,这珍珠宝石定要古坟里的,有那古时富贵人家装裹的头面,\zhu{装裹的头面:指死人戴的珠宝。
装裹:装殓。
头面:这里指死人头上戴的首饰。
}拿了来才好。
如今那里为这个去刨坟掘墓,所以只要活人戴过的,也可以使得。
”王夫人随念:“阿弥陀佛,不当家花花的!\zhu{不当家花花的:“不当家”亦作“不当价”。
意即不应该、当不起、罪过。
“价”为助词,“花花的”是词尾,无义。
}就是坟里有这个,人家死了几百年,如今翻尸盗骨的,作了药也不灵!”\jia{不止阿凤圆谎,今作者亦为圆谎了,看此数句则知矣。
}\ping{这条评语的意思是,宝玉和黛玉和好后高兴,胡扯了一味药,宝钗不想帮宝玉圆谎,而凤姐出于哄宝玉开玩笑或者支持黛玉的原因,帮宝玉圆了谎。
谎话尽量不要给出特别具体的细节,否则很容易被证伪推翻。
从凤姐绘声绘色地细致叙述看,不像是谎话。这条评语是批书人一家之词,从后文能感觉到宝玉说的是真的,而宝钗装作不知道。
}\par
宝玉向林黛玉说道:“你听见了没有,难道二姐姐也跟着我撒谎不成?”脸望着黛玉说话,却拿眼睛飘着宝钗。
黛玉便拉王夫人道:“舅母听听,宝姐姐不替他圆谎,他直问着我。
”王夫人也道:“宝玉很会欺负你妹妹。
”宝玉笑道:“太太不知道原故。
宝姐姐先在家里住着,那薛大哥的事,他就不知道,何况如今在里头住着呢,自然是越发不知道了。
\geng{分析得是,不敢正犯。
}林妹妹才在背后,以为是我撒谎,就羞我。
”\par
说着,只见贾母房里的丫头找宝玉、黛玉吃饭。
林黛玉也不叫宝玉,便起身拉了那丫头就走。
那丫头说等着宝玉一块儿走。
林黛玉道:“他不吃饭了,咱们走吧\foot{列、杨本此处多29 字,作“‘(咱们走)吧。
’那丫头道:‘吃不吃,等他一块儿去。
老太太问,让他说去。
’黛玉道:‘你就等着。
(我先走了。
)’”似觉语气更衔接自然。
}。
我先走了。
”\ping{宝钗撒谎说自己不知道,宝玉替宝钗打圆场,惹怒了黛玉。
}说着便出去了。
宝玉道:“我今儿还跟着太太吃罢。
”王夫人道:“罢,罢,我今儿吃斋,你正经吃去罢。
”宝玉道:“我也跟着吃斋。
”说着便叫那丫头“去罢”,自己先跑到炕上坐了。
王夫人向宝钗道:“你们只管吃你们的去,由他罢。
”宝钗因笑道:“你正经去罢。
吃不吃,陪着林妹妹走一趟,他心里打紧的不自在呢。
”宝玉道:“理他呢,过一会子就好了。
”\geng{后文方知。
}\ping{宝玉留下来而不是去陪生气的黛玉,实际上还是觉得要先陪谎言被戳穿而尴尬的宝钗,因为觉得黛玉和自己更近,安慰黛玉可以等会在私下里而不是在公开场合,这里并不是对黛玉不理不睬。
}\par
一时吃过饭,宝玉一则怕贾母记挂,二则也记挂着黛玉,忙忙的要茶漱口。
探春、惜春都笑道:“二哥哥,你成日家忙些什么?\zhu{家:一作“价”,语尾助词,无义。
成日家:一天到晚,终日里。
}\jia{冷眼人自然了了。
\zhu{了:前一个“了”是明白,懂得的意思。
}}吃饭吃茶也是这么忙碌碌的。
”宝钗笑道:“你叫他快吃了瞧林妹妹去罢,叫他在这里胡羼些什么。
\zhu{
羼:音“颤”。《说文》:“羼,羊相厕(厕:置身于)也。
”引申为搀杂。
胡羼:犹言“鬼混”。
}
”宝玉吃了茶便出来,直往西院走。
可巧走到凤姐院前,只见凤姐蹬着门槛子拿耳挖子剔牙,\geng{也才吃了饭。
}看着小子们挪花盆呢。
\geng{是阿凤身段。
}见宝玉来了,笑道:“你来的正好。
进来,进来,替我写几个字儿。
”宝玉只得跟了进来。
到了房里,凤姐命人取过笔砚纸来,向宝玉道:“大红妆缎四十匹,蟒缎四十匹,上用纱各色一百匹,金项圈四个。
”宝玉道:“这算什么?又不是帐,又不是礼物,怎么个写法?”凤姐道:“你只管写上,横竖我自己明白就罢了。
”\geng{有是语,有是事。
}\ping{可能是权力寻租,收受贿赂。
}宝玉听说,只得写了。
凤姐收起来,笑道:“还有句话告诉你,不知你依不依?你屋里有个丫头叫红玉,我和你说说,要叫了来使唤,也总没得说,今儿见你才想起来。
” 
\jia{字眼。
}
\ping{凤姐有意趁宝玉着急去见黛玉的时机说调走红玉的事情,宝玉来不及细想斟酌,一说必应。}
宝玉道:“我屋里的人也多的很,姐姐喜欢谁,只管叫了来,何必问我。
”\jia{红玉接杯倒茶,自纱屉内觅至回廊下,
\zhu{
纱屉子:旧时的窗户分两层,里面一层是用纱绷上的,透明、通气,称“纱屉子”。
外面一层是用纸糊或木板装的,白天可以卸下来或支起,晚间再安上或放下。
}
再见此处如此写来,可知玉兄除颦外,俱是行云流水。
\ping{偶然注意,过后就忘。
}}凤姐笑道:“既这么着,我就叫人带他去了。
”\jia{又了却怡红一冤孽,一叹!}宝玉道:“只管带去。
”说着便要走。
\jia{忙极!}凤姐道:“你回来,我还有句话说。
”宝玉道:“老太太叫我呢,\jia{非也,林妹妹叫我呢。
一笑。
}有话等我回来罢。
”说着,便来至贾母这边,已经都吃完了饭。
贾母因问他:“跟着你母亲吃什么好的了?”宝玉笑道:“也没什么好的,我倒多吃了一碗饭。
”\jia{安慰祖母之心也。
}
因问:“林妹妹在那里呢?”\jia{何如?余言不谬。
}贾母道:“里头屋里呢。
”\par
宝玉进来,只见地下一个丫头吹熨斗,
\zhu{
熨斗:熨平衣物的金属器具。形状像斗,有长柄,斗中烧木炭,用斗底的热量烫平衣物。
吹熨斗:实指吹斗中烧的木炭使之更旺。
}
炕上两个丫头打粉线,黛玉弯着腰,拿着剪子裁什么呢。
宝玉走进来笑道:“哦,这是作什么呢?才吃了饭,这么空着头,\zhu{空着头:俯身低头。
空即“控”。
}一会子又头疼了。
”黛玉并不理,只管裁他的。
有一个丫头道:“这块绸子角儿还不好呢,再熨他一熨。
”黛玉把剪子一撂,说道:“理他呢,过一会子就好了。
”\jia{有意无意,暗合针对,无怪玉兄纳闷。
}宝玉听了,只是纳闷。
只见宝钗、探春也来了,和贾母说了一会话。
宝钗也进来问:“林妹妹作什么呢?”见黛玉裁剪,因笑道:“越发能干了,连裁剪都会了。
”黛玉笑道:“这也不过是撒谎哄人罢了。
”\ping{讽刺宝钗撒谎说自己不知道,站在对自己有意见的王夫人阵营。
}宝钗笑道:“我告诉你个笑话儿,才刚为那个药,我说了个不知道,宝玉心里不受用了。
”
\ping{这里宝钗含蓄地承认说谎。宝玉心里不受用,黛玉心里更不受用。}
林黛玉道:“理他呢,过一会子就好了。
”\jia{连重二次前言,是颦、宝气味暗合,勿认作有小人过言也。
\zhu{过言:意思大概是传话。
}}\geng{连重两遍前言,是颦、玉气味相仿,无非偶然暗合相符,勿认作有过言小人也。
}宝玉又向宝钗道:“老太太要抹骨牌,正没人,你抹骨牌去。
”\ping{刚才宝玉陪尴尬的宝钗吃饭,那可能是公开场合的礼数,私下里却打发宝钗离开,好自己宽慰黛玉。
}宝钗听说,便笑道:“我是为抹骨牌才来了?”说着便走了。
林黛玉道:“你倒是去罢,这里有老虎,看吃了你!”说着又裁。
宝玉见他不理,只得还陪笑说道:“你也去逛逛再裁不迟。
”黛玉总不理。
宝玉便问丫头们:“这是谁叫裁的?”黛玉见问丫头们,便说道:“凭他谁叫裁,不管二爷的事!”宝玉听了,方欲说话,只见有人进来回说“外头有人请你呢”。
宝玉听说,忙撤身出来。
黛玉向外说道:\jia{仍丢不下,叹叹!}“阿弥陀佛!赶你回来,我死了也罢了。
”\jia{何苦来?余不忍听。
}\ping{红楼梦87版电视剧里,宝玉从军西海沿子,回来后黛玉已经死了。
}\par
宝玉出来,到外头,只见茗烟说道:“冯大爷家请。
”宝玉听了,知道是昨日的话,便说:“要衣裳去。
”自己便往书房里来。
茗烟一直到了二门前等人,\jia{此门请出玉兄来,故信步又至书房,文人弄墨,虚点缀也。
}
只见出来个老婆子,茗烟上去说道:“宝二爷在书房里等出门的衣裳,你老人家进去带个信儿。
”那婆子说:“你妈的屄倒好!\geng{活现活跳。
}
宝二爷如今在园子里住着,\jia{与夜间叫人对看。
\zhu{第七十一回。
}}跟他的人都在园子里,你又跑了这里来带信儿!”茗烟听了,笑道:“骂的是,我也糊涂了。
”说着一径往东边二门上来。
可巧门上小厮在甬路底下踢球,茗烟将原故说了。
有个小厮跑了进去,半日才抱了一个包袱出来,递与茗烟。
\ping{茗烟打发老婆子去干活被老婆子回怼,碰壁之后又打发小厮去。
茗烟和老婆子都不肯多干一点活。
}回到书房里,宝玉换了,命人备马,只带着茗烟、锄药、双瑞、双寿四个小厮,一径来到冯紫英门口。
有人报与冯紫英,出来迎接进去。
只见薛蟠早已在那里久候,还有许多唱曲儿的小厮并唱小旦的蒋玉菡、锦香院的妓女云儿。
\zhu{小旦:戏曲中旦角的一种,多扮演年轻女子。}
大家都见过了,然后吃茶。
\par
宝玉擎茶笑道:“前儿所言幸与不幸之事,我昼悬夜想,今日一闻呼唤即至。
”冯紫英笑道:“你们令表兄弟倒都心实。
前日不过是我的设辞,诚心请你们一饮,恐又推托,故说下这句话。
\jia{若真有一事,则不成《石头记》文字矣。
作者的三昧在兹,\zhu{三昧:佛教用语。
本意是心神专一,杂念止息,是佛家修持的重要方法之一。
后借指事物的奥秘和精义。
}批书人得书中三昧亦在兹。
壬午孟夏。
}
今日一邀即至,谁知都信真了。
”说毕大家一笑,然后摆上酒来,依次坐定。
冯紫英先命唱曲儿的小厮过来让酒,然后命云儿也来敬。
\par
那薛蟠三杯下肚,不觉忘了情,拉着云儿的手笑道:“你把那梯己新样儿的曲子唱个我听,\zhu{梯己:意即私人的、贴心的。
这里是指私下里的知心话。
}我吃一坛如何?”云儿听说,只得拿起琵琶来,唱道:\par
\hop
两个冤家,都难丢下,想着你来又记挂着他。
两个人形容俊俏,都难描画。
想昨宵幽期私订在荼蘼架,一个偷情,一个寻拿,拿住了三曹对案,\zhu{三曹对案:指诉讼案件中的原告、被告和证人。
审案件时,这三方面的人同时到场,进行对证,叫做“三曹对案”。
这里指妓女和两个相好的嫖客一共三个人争风吃醋。
}我也无回话。
\jia{此唱一曲为直刺宝玉。
\ping{可能暗指宝玉亦曾眠花卧柳,和别的嫖客争抢一个妓女的欢心。
}}\par
\hop
唱毕笑道:“你喝一坛子罢了。
”薛蟠听说,笑道:“不值一坛,再唱好的来。
”\par
宝玉笑道:“听我说来:如此滥饮,易醉而无味。
我先吃一大海,\geng{大海饮酒,西堂产九台灵芝日也,批书至此,宁不悲乎?壬午重阳日。
}发一新令,有不遵者,连罚十大海,逐出席外与人斟酒。
”\jia{谁曾经过?叹叹!西堂故事。
\zhu{
西堂:曹雪芹祖父曹寅任江宁织造时的斋名,曹寅的《楝亭集》中有不少有关西堂的诗题及词题,曹寅常常在此堂邀同好题诗饮酒。另外,曹寅又自号西堂、西堂扫花行者。
九台灵芝:《楝亭集》中未见记述。但是在诗钞卷七中有《粟花歌》一首,诗题下有自注:“粟花,粟树所产菌,其大逾常。不时见偃,盖七重,色绀赤。友人云:即紫芝,因戏为此歌。”脂批所言“西堂产九台灵芝之日”,曹寅曾在此堂邀请同好有过一番燕集,来庆祝此等盛事。
}}冯紫英、蒋玉菡等都道:“有理,有理。
”宝玉拿起海来,一气饮尽,说道:“如今要说悲、愁、喜、乐四字,都要说出女儿来,还要注明这四字的原故。
\zhu{原故:现在规范词形写作“缘故”。}
说完了,饮门杯。
酒面要唱一个新鲜时样的曲子;酒底要席上生风一样东西,\zhu{门杯:对公杯而言,酒宴时用以敬酒、罚酒等公用的酒杯叫公杯,放在各人面前的酒杯叫门杯,也叫门前杯。
酒面:斟满一杯酒,不饮,先行酒令,叫酒面。
“酒面”的本义是满杯的样子。
酒底:每行完一个酒令时,饮干一杯酒,叫“酒底”。
席上生风:借酒席上的食品或装饰等现成东西,说一句与此有关的古诗或古文。
}或古诗旧对、《四书》《五经》成语。
”薛蟠未等说完,先站起来拦住道:“我不来,别算我。
\jia{爽人爽语。
}这竟是捉弄我呢!”\geng{岂敢?}\ping{不学无术纨绔子弟,形容毕肖。
}云儿便站起来,推他坐下,笑道:“怕什么?这还亏你天天吃酒呢,难道连我也不如!我回来还说呢。
说是了,罢;不是了,不过罚上几杯,那里就醉死了。
你如今一乱令,倒喝十大杯,下去给人斟酒不成?”\geng{有理。
}众人都拍手道妙。
薛蟠听说,无法可治,只得坐下。
听宝玉先说,宝玉便道:\par
\hop
女儿悲,青春已大守空闺。
\ping{暗示宝钗结局:宝玉出家,宝钗独守空闺。
}\par
女儿愁,悔教夫婿觅封侯。
\ping{暗示宝钗悲剧原因:频频劝导宝玉考科举走仕途。
}\par
女儿喜,对镜晨妆颜色美。
\par
女儿乐,秋千架上春衫薄。
\par
\hop
众人听了,都道:“说得有理。
”薛蟠独扬着脸摇头说:“不好,该罚!”众人问:“如何该罚?”薛蟠道:“他说的我都不懂,怎么不该罚?”云儿便拧他一把,笑道:“你悄悄的想你的罢。
回来说不出,才是该罚呢。
”于是拿琵琶听宝玉唱道:\par
\hop
滴不尽相思血泪抛红豆,\zhu{红豆:又名相思子,大如豌豆,色鲜红。
这里用以代指眼泪。
}开不完春柳春花满画楼。
睡不稳纱窗风雨黄昏后,忘不了新愁与旧愁,咽不下玉粒金莼噎满喉,\zhu{玉粒金莼:玉粒:喻上好的米饭。
莼:音“纯”,我国江南生长的一种睡莲科水生植物,夏天开赤褐色小花,嫩叶是一种名菜。
金莼:泛指美味的菜肴。
}\ping{“愁”可能是指宝玉思念黛玉而愁苦,“忘不了”刻画出宝玉对黛玉难以忘怀,抛舍不下;玉粒金莼:暗含“金玉良缘”的“金玉”二字,指的是贾宝玉和薛宝钗的婚姻在外人看来是如此的幸福美满,而又用“咽不下”、“噎满喉”,刻画出贾宝玉对此的难以接受,黯然神伤。
这两句写出了宝玉婚姻的悲剧。
}照不见菱花镜里形容瘦。
\zhu{菱花镜:古代铜镜。
六角形或背面刻有菱花者名菱花镜。
}展不开的眉头,捱不明的更漏。
\zhu{捱:同“挨”,熬,撑。
}呀!恰便似遮不住的青山隐隐,流不断的绿水悠悠。
\par
\hop
唱完,大家齐声喝彩,独薛蟠说无板。
\zhu{板:乐器中打节拍的板,这里指音乐的节拍、节奏。
}宝玉饮了门杯,便拈起一片梨来,说道:“雨打梨花深闭门。
”
\zhu{雨打梨花深闭门:出自宋代秦观《鹧鸪天》:
枝上流莺和泪闻,新啼痕间旧啼痕。
一春鱼鸟无消息,千里关山劳梦魂。
无一语,对芳尊。
安排肠断到黄昏。
甫能炙得灯儿了,雨打梨花深闭门。
原词写怀人不归。这里用此句作宝玉的酒底,当有寓意。
雨打梨花深闭门:描写闺中独居的寂寞凄凉。梨:谐音离。这里似用以暗示宝钗婚后的生活情境。
}
完了令。
\par
下该冯紫英。
听冯紫英说道:\par
\hop
女儿悲,儿夫染病在垂危。
\par
女儿愁,大风吹倒梳妆楼。
\par
女儿喜,头胎养了双生子。
\zhu{养:生孩子。
}\par
女儿乐,私向花园掏蟋蟀。
\jia{紫英口中应当如是。
}\par
\hop
说毕,端起酒来,唱道:\par
\hop
你是个可人,你是个多情,你是个刁钻古怪鬼灵精,你是个神仙也不灵。
我说的话儿你全不信,只叫你去背地里细打听,才知道我疼你不疼!\par
\hop
唱完,饮了门杯,说道:“鸡鸣茅店月。
”令完,下该云儿。
云儿便说道:\par
\hop
女儿悲,将来终身指靠谁?” \jia{道着了。
}\ping{契合妓女身份。
}\par
\hop
薛蟠叹道:“我的儿,有你薛大爷在,你怕什么!”众人都道:“别混他,别混他!”云儿又道:\par
\hop
女儿愁,妈妈打骂何时休!\zhu{妈妈:这里指妓女的养母,即“鸨(音“保”)儿”。
}\par
\hop
薛蟠道:“前儿我见了你妈,还吩咐他不叫他打你呢。
”众人都道:“再多言者罚酒十杯。
”薛蟠连忙自己打了一个嘴巴子,说道:“没耳性,\zhu{没耳性:没记性。
}
再不许说了。
”云儿又道:\par
\hop
女儿喜,情郎不舍还家里。
\par
女儿乐,住了箫管弄弦索。
\par
\hop
说完,便唱道:\par
\hop
豆蔻开花三月三,一个虫儿往里钻。
钻了半日不得进去,爬到花儿上打秋千。
肉儿小心肝,我不开了你怎么钻?\jia{双关,妙!}\zhu{豆蔻\foot{
\footPic{豆蔻}{doukou.jpg}{0.5}
}:暗指女性生殖器。
虫儿:暗指男性生殖器。
}\par
\hop
唱毕,饮了门杯,说道:“桃之夭夭。
”\zhu{桃之夭夭:《诗·周南·桃夭》:“桃之夭夭,灼灼其华(花)”。
夭夭:美丽茂盛的样子。
}令完了,下该薛蟠。
\par
薛蟠道:“我可要说了:女儿悲——”说了半日,不见说底下的。
冯紫英笑道:“悲什么?快说来。
”薛蟠登时急的眼睛铃铛一般,瞪了半日,才说道:“女儿悲——”又咳嗽了两声,\jia{受过此急者,大都不止呆兄一人耳。
}说道:“女儿悲,嫁了个男人是乌龟。
”\zhu{乌龟:类似于“王八”,骂人的话,指妻子有外遇的男人。
}众人听了都大笑起来。
\jia{此段与《金瓶梅》内西门庆、应伯爵在李桂姐家饮酒一回对看,未知孰家生动活泼?
\zhu{指《金瓶梅》第十六回,“潘金莲激打孙雪娥,西门庆梳笼李桂姐”,西门庆、应伯爵、谢希大在李桂姐家饮酒唱曲一事。}
}薛蟠道:“笑什么,难道我说的不是?一个女儿嫁了汉子,要当忘八,\zhu{忘八:即“王八”,乌龟或鳖的俗称,骂人的话,指妻子有外遇的男人。
}他怎么不伤心呢?”众人笑的弯腰,说道:“你说的很是,快说底下的。
”薛蟠瞪了瞪眼,又说道:“女儿愁——”说了这句,又不言语了。
众人道:“怎么愁?”薛蟠道:“女儿愁,绣房撺出个大马猴。
”\zhu{撺:可能是“蹿”或者“窜”的错讹。
大马猴:一种说法是指吓唬别人时常说的一种想象中的形如猴的动物。
另一种说法是指像猴子一样淘气的小孩子,“嫁了个男人是乌龟”指妻子有外遇,“绣房撺出个大马猴”指妻子私通别的男人并且生下孩子。
}众人呵呵笑道:“该罚,该罚!这句更不通,先还可恕。
” 
\jia{不愁,一笑。
}说着便要筛酒。
\zhu{筛酒:斟酒。
}宝玉笑道:“押韵就好。
”薛蟠道:“令官都准了,你们闹什么?”众人听说,方罢了。
云儿笑道:“下两句越发难说了,我替你说罢。
”薛蟠道:“胡说!当真的我就没好的了!听我说罢:女儿喜,洞房花烛朝慵起。
”众人听了,都诧异道:“这句何其太韵?”薛蟠又道:“女儿乐,一根\jiji \baba 往里戳。
”\zhu{ \jiji \baba :男性生殖器。
}\jia{有前韵句,故有是句。
}众人听了,都扭着脸说道:“该死,该死!快唱了罢。
”薛蟠便唱道:“一个蚊子哼哼哼。
”众人都怔了,说“这是个什么曲儿?”薛蟠还唱道:“两个苍蝇嗡嗡嗡。
”众人都道:“罢,罢,罢!”薛蟠道:“爱听不听!这个新鲜曲儿,叫作哼哼韵。
你们要懒待听,连酒底都免了,我就不唱。
\jia{何尝呆?}”众人都道:“免了罢,倒别耽误了别人家。
”于是蒋玉菡说道:\par
\hop
女儿悲,丈夫一去不回归。
\ping{可能暗指宝玉出家不回归。
}\par
女儿愁,无钱去打桂花油。
\zhu{桂花油:一种有桂花香气的发油。
}\par
女儿喜,灯花并头结双蕊。
\zhu{灯花结双蕊:蜡烛芯点燃后呈穗状,叫“烛花”。
“双蕊”即两个“烛花”。
旧时认为它象征吉祥或夫妻久别相会。
}\jia{佳谶也。
}\ping{伏蒋玉函和袭人的婚姻。
}\par
女儿乐,夫唱妇随真和合。
\par
\hop
说毕,唱道:\par
\hop
可喜你天生成百媚娇,恰便似活神仙离云霄。
度青春,年正小;配鸾凤,\zhu{鸾凤:类似于“鸾俦[chóu]”,比喻夫妻。
俦:伴侣;同类。
}真也着。
呀!看天河正高,听谯楼鼓敲,\zhu{谯(音“瞧”)楼:即鼓楼。
古代击鼓报时之楼。
}剔银灯同入鸳帏悄。
\zhu{剔灯:挑起灯芯,剔除馀烬,使灯更亮。
}\par
\hop
唱毕,饮了门杯,笑道:“这诗词上我倒有限。
幸而昨日见了一副对子,可巧\jia{真巧!}只记得这句,幸而席上还有这件东西。
”\jia{瞒过众人。
}
说毕,便饮干了酒,拿起一朵木樨来,\zhu{木樨(樨音“西”):即桂花。
}念道:“花气袭人知昼暖。
”\ping{内含袭人的名字“花袭人”。
}\par
众人倒都依了,完令。
薛蟠又跳了起来,喧嚷道:“了不得,了不得!该罚,该罚!这席上并没有宝贝,\jia{奇谈。
}你怎么念起宝贝来?”蒋玉菡怔了,说道:“何曾有宝贝?”薛蟠道:“你还赖呢!你再念来。
”蒋玉菡只得又念了一遍。
薛蟠道:“袭人可不是宝贝是什么!你们不信,只问他。
”说着,指着宝玉。
宝玉没好意思起来,说道:“薛大哥,你该罚多少?”薛蟠道:“该罚,该罚!”说着拿起酒来,一饮而尽。
冯紫英与蒋玉菡等不知原故,犹问原故,云儿便告诉了出来。
\jia{用云儿细说,的是章法。
}\geng{云儿知怡红细事,可想玉兄之风情月意也。
壬午重阳。
}\ping{可能宝玉也和云儿厮混过,所以云儿对宝玉家里的事很了解。
}蒋玉菡忙起身陪罪。
众人都道:“不知者不作罪。
”\par
少刻,宝玉席外解手,\zhu{解手:上厕所。
}蒋玉菡便随了出来。
二人站在廊檐底下,蒋玉菡又陪不是。
宝玉见他妩媚温柔,心中十分留恋,\ping{回想起同样温柔的秦钟。
}便紧紧的搭着他的手,叫他:“闲了往我们这里来。
还有一句话借问,也是你们贵班中,有一个叫琪官的,他在那里?如今名驰天下,我独无缘一见。
”蒋玉菡笑道:“就是我的小名儿。
”宝玉听说,不觉欣然跌足笑道:\zhu{跌足:跺脚。
}“有幸,有幸!果然名不虚传。
今儿初会,便怎么样呢?”想了一想,向袖中取出扇子,将一个玉玦扇坠解下来,\zhu{
玉玦(玦音“决”):古玉器名,环状,有缺口。
玉玦扇坠:系在扇轴上的饰物。
}递与琪官道:“微物不堪,略表初见之谊。
”琪官接了,笑道:“无功受禄,何以克当!也罢,我这里也得了一件奇物,今日早起方系上,还是簇新的,聊可表我一点亲热之意。
”说着,将系小衣儿一条大红汗巾子解下来,递与宝玉,道:“这汗巾是茜香国女国王进贡来的,夏天系着,肌肤生香,不生汗渍。
昨日北静王给我的,今日才上身。
若是别人,我断不肯相赠。
二爷请把自己系的给我系着。
”宝玉听说,喜不自禁,连忙接了,将自己一条松花汗巾解了下来,\zhu{松花:偏黑的深绿色。
}递与琪官。
\jia{红绿\sout{牵}[汗]巾是这样用法。
一笑。
}二人方束好,只听一声大叫:“我可拿住了!”只见薛蟠跳了出来,拉着二人道:“放着酒不吃,两个人逃席出来干什么?快拿出来我瞧瞧。
”二人都道:“没什么。
”薛蟠那里肯依,还是冯紫英出来才解开了。
于是复又归座饮酒,至晚方散。
\par
宝玉回至园中,宽衣吃茶。
袭人见扇子上的扇坠儿没了,便问他:“往那里去了?”宝玉道:“马上丢了。
”\geng{随口谎言。
}睡觉时只见腰里一条血点似的大红汗巾子,袭人便猜了八九分,因说道:“你有了好的系裤子,把我那条还我罢。
”宝玉听说,方想起那条汗巾子原是袭人的,不该给人才是,心里后悔,口里说不出来,只得笑道:“我赔你一条罢。
”袭人听了,点头叹道:“我就知道又干这些事!也不该拿着我的东西给那起混帐人去。
也难为你心里没个算计儿。
”再要说上几句,又恐怕怄上他的酒来,
\zhu{怄[òu]:引逗,招惹,引人发笑或使人生气。}
少不得睡了,一宿无话。
\par
至次日天明起来,只见宝玉笑道:“夜里失了盗也不晓得,你瞧瞧裤子上。
”袭人低头一看,只见昨日宝玉系的那条汗巾子系在自己腰里,便知是宝玉夜间换了,\ping{袭人和蒋玉函通过宝玉间接交换信物,互换汗巾子,伏两人的婚姻。
}忙一顿把解下来,
\zhu{一顿把:这是一个不可分拆的词语,犹言“一下子”、“一鼓作气”。}
说道:“我不希罕这行子,\zhu{行(音“杭”)子:贬称自己所不喜爱的东西或人。
}趁早儿拿了去!”宝玉见他如此,只得委婉解劝了一回。
袭人无法,只得系上。
过后宝玉出去,终久解下来,掷在个空箱子里,自己又换了一条系着。
\par
宝玉并不理论,因问起昨日可有什么事情。
袭人便回说道:“二奶奶打发了人叫了红玉去了。
他原要等你来,我想什么要紧,我就作了主,打发他去了。
”宝玉道:“很是。
我已知道了,不必等我罢了。
”袭人又道:“昨儿贵妃差了夏太监出来,送了一百二十两银子,叫在清虚观初一到初三打三天平安醮,\zhu{打平安醮:旧时因病或因丧事延请僧、道诵经叫“打醮”。
为一般祈福消灾举行的“打醮”仪式,叫“打平安醮”。
}唱戏献供,叫珍大爷领着众位爷们跪香拜佛呢。
还有端午儿的节礼也赏了。
”说着命小丫头来,将昨日的所赐之物取了出来,只见上等宫扇两柄,红麝香珠二串,\zhu{红麝香珠:又叫“红麝串”、“红麝串子”。
用麝香加上其它配料做成的红色念珠儿,穿成串子,戴在手腕上作装饰。
麝香为雄麝之麝香腺中分泌物,干燥后成红棕至暗棕色颗粒。
}凤尾罗二端,\zhu{罗:稀疏而轻软的丝织品。
端:量词,布帛的长度单位。
}芙蓉簟一领。
\zhu{簟:音“电”,竹席。
领:量词。
}宝玉见了,喜不自胜,问道:“别人的也都是这个么?”袭人道:“老太太的多着一个香如意,
\zhu{如意:一种用玉、象牙等制成的象征吉祥的物品,长约一二市尺,头为灵芝形或云朵形,柄略呈波浪形。}
一个玛瑙枕。
老爷、太太、姨太太的只多着一柄如意。
你的同宝姑娘的一样。
\jia{金娃玉郎是这样写法。
}林姑娘同二姑娘、三姑娘、四姑娘只单有扇子同数珠儿,\zhu{数珠:佛教徒诵经时用以计算诵经次数的串珠,也称“念珠”、“佛珠”。
}别人都没了。
大奶奶、二奶奶他两个是每人两匹纱、两匹罗、两个香袋儿、两个锭子药。
”\zhu{锭子药:把药制成坚硬的小块叫“锭子药”,亦称“药锭子”,常做成各种花样。
}宝玉听了,笑道:“这是怎么个原故?怎么林姑娘的倒不同我的一样,倒是宝姐姐的同我一样!别是传错了罢?”\ping{元春可能受到母亲王夫人的影响,通过赏赐礼物,表明了她在宝玉婚姻问题上的态度是支持王夫人的金玉良缘。
但是在宝玉心中的排序,宝钗还是排在了黛玉的后面。
}袭人道:“昨儿拿出来,都是一份一份的写着签子,怎么就错了!你的是在老太太屋里来着,我去拿了来了。
老太太说,明儿叫你一个五更天进去谢恩呢。
”宝玉道:“自然要走一趟。
”说着便叫紫绡:“来,拿了这个到林姑娘那里去,就说是昨儿我得的,爱什么留下什么。
”紫绡答应了,便拿了去,不一时回来说:“林姑娘说了,昨儿也得了,二爷留着罢。
”\par
宝玉听说,便命人收了。
刚洗了脸出来,要往贾母那边请安去,只见林黛玉顶头来了。
宝玉赶上去,笑道:“我的东西叫你拣,你怎么不拣?”林黛玉昨日所恼宝玉的心事早又丢开,只顾今日的事了,因说道:“我没这么大福禁受,比不得宝姑娘,什么金什么玉的,我们不过是草木之人!”\jia{自道本是绛珠草也。
}宝玉听他提出“金玉”二字来,不觉心动疑猜,便说道:“除了别人说什么金什么玉,我心里要有这个想头,天诛地灭,万世不得人身!”\ping{咽不下玉粒金莼噎满喉。
}林黛玉听他这话,便知他心里动了疑,忙又笑道:“好没意思,白白的说什么誓?管你什么金什么玉的呢!”宝玉道:“我心里的事也难对你们说,日后自然明白。
除了老太太、老爷、太太这三个人,第四个就是妹妹了。
\ping{这里大概是宝玉对黛玉的表白。
}要有第五个人,我就说个誓。
”黛玉道:“你也不用说誓,我很知道你心里有‘妹妹’,但只是见了‘姐姐’,就把‘妹妹’忘了。
”宝玉道:“那是你多心,我再不的。
”黛玉道:“昨儿宝丫头不替你圆谎,为什么问着我呢?那要是我,你又不知怎么样了。
”\par
正说着,只见宝钗从那边来了,二人便走开了。
宝钗分明看见,只装看不见,低着头过去了,到了王夫人那里,坐了一回,然后到了贾母这边,只见宝玉在这里呢。
\jia{宝钗往王夫人处去,故宝玉先在贾母处,一丝不乱。
}宝钗因往日母亲对王夫人等曾提过“金锁是个和尚给的,等日后有玉的方可结为婚姻”等语,\jia{此处表明以后二宝文章,宜换眼看。
}\ping{这里明出“金玉良缘”。
}所以总远着宝玉。
\jia{峰峦全露,又用烟云截断,好文字。
}\ping{外力强行撮合,宝钗心里觉得没意思。
}昨日见了元春所赐的东西,独他与宝玉一样,心里越发没意思起来。
\ping{金玉良缘开始只是王夫人和薛姨妈的撮合,宝钗已经觉得没意思,所以总远着宝玉。
如今甚至贵妃也被说动而加入到母亲和姨妈的阵营,宝钗更觉得没意思。
就算宝钗喜欢宝玉,作为一个自尊自爱的女孩子,也是希望靠着自己的人格魅力达成心愿,和黛玉公平竞争,而非依靠外在强力撮合。
宝钗进京也是因为被家人安排待选进宫,被寄予得宠而使得家族中兴的任务,就像表姐贾元春那样。
本书到此,基本可以肯定,宝钗待选进宫应该是失败了,因为如果宝钗依旧是候选人的话,那么王夫人和薛姨妈不可能提出金玉良缘撮合宝玉宝钗。
失败的原因可能是进京路上自己的哥哥背上了人命官司,这个案底可能成为薛家的道德污点。
宝钗一直在家人的安排中挣扎,进宫的安排失败了,退而求其次,又有了和宝玉结婚的安排,“没意思”反映了宝钗始终不得自由的怅惘。
另外,贾元春被家人安排进宫,肩负起中兴家族的责任,从省亲的话可以看出,她在宫里并不幸福,从自身经历出发,她应该深知这种被安排的婚姻的悲剧。
元春自己被剥夺了婚姻自由,但是通过赏赐礼物,还是不可避免地要介入别人的婚姻,剥夺弟弟和表妹的婚姻自由,从受害人变成了加害人,宝钗可能也觉得自己的表姐“没意思”。
}幸亏宝玉被一个黛玉缠绵住了,心心念念只记挂着黛玉,并不理论这事。
此刻忽见宝玉笑问道:“宝姐姐,我瞧瞧你的那红麝串子。
”可巧宝钗左腕上笼着一串,见宝玉问他,少不得褪了下来。
宝钗原生的肌肤丰泽,容易褪不下来。
宝玉在旁边看着雪白一段酥臂,不觉动了羡慕之心,暗暗想道:“这个膀子要长在林妹妹身上,或者还得摸一摸,偏生长在他身上。
”正是恨没福得摸,\ping{宝玉的矛盾心理,出于生理冲动,想去摸宝钗的雪白酥臂;出于自己对黛玉的承诺,又不能去摸。
}忽然想起“金玉”一事来,再看看宝钗形容,只见脸若银盆,眼似水杏,唇不点而红,眉不画而翠,\zhu{翠:青绿色,这里可能是青黑色的意思。
古代女子用青黛(青黑色颜料)画眉。
“青”在古汉语里有黑色的意思,如“朝如青丝暮成雪”。
}\jia{太白所谓“清水出芙蓉”。
}比黛玉另具一种妩媚风流,不觉就呆了,\jia{忘情,非呆也。
}宝钗褪了串子来递与他也忘了接。
宝钗见他怔了,自己倒不好意思的,丢下串子,回身才要走,只见黛玉蹬着门槛子,嘴里咬着手帕子笑呢。
宝钗道:“你又禁不得风儿吹,怎么又站在那风口里呢?”黛玉笑道:“何曾不是在屋里呢。
只因听见天上一声叫,出来瞧了一瞧,原来是个呆雁。
”宝钗道:“呆雁在那里呢?我也瞧瞧。
”黛玉道:“我才出来,他就‘忒儿’一声飞了。
”口里说着,将手里的帕子一甩,向宝玉脸上甩来。
不防正打在眼上,“嗳哟”了一声。
\ping{黛玉吃醋。
}再看下回分明。
\par
 \jia{总评:茜香罗、红麝串写于一回,盖琪官虽系优人,后回与袭人供奉玉兄宝卿得同终始者,非泛泛之文也。
\hang
自“闻曲”回以后,回回写药方,是白描颦儿添病也。
\hang
前“玉生香”回中颦云“他有金你有玉;他有冷香你岂不该有暖香?”是宝玉无药可配矣。
今颦儿之剂若许材料皆系滋补热性之药,兼有许多奇物,而尚未拟名,何不竟以“暖香”名之?以代补宝玉之不足,岂不三人一体矣。
\hang
宝玉忘情,露于宝钗,是后回累累忘情之引。
\hang
茜香罗暗系于袭人腰中,系伏线之文。
}\par
\qi{总评:世间最苦是痴情,不遇知音休应声。
盟誓已成了,莫迟误今生。
}
\dai{055}{宝玉给袭人换上自己从蒋玉函那里得到的汗巾子}
\dai{056}{薛宝钗羞笼红麝串}
\sun{p28-1}{宝玉说奇方配药丸,林黛玉裁衣谈笑语}{图右侧:宝玉黛玉一起来到王夫人处,说起黛玉吃药的事, 宝玉说了一味药,还说:“太太不信,只问宝姐姐。
”宝钗笑着摇手儿道:“我不知道,也没听见。
你别叫姨娘问我。
”黛玉在宝钗身后抿嘴笑, 用手指在脸上羞他。
图上侧:宝玉吃过饭便出来找黛玉,凤姐见了笑道:“你来得正好,进来进来,替我写几个字。
”图左上:待写完字见了贾母,来到里屋,只见黛玉正在裁剪衣料。
一会儿宝钗探春也来了。
}
\sun{p28-2}{行酒令低唱红豆曲,蒋玉菡情赠茜香罗}{图右下:宝玉被茗烟传话叫到冯紫英家里,薛蟠和唱小旦的蒋玉函、锦香院的妓女云儿也在。
大家行酒令。
图上侧:宝玉出席解手,蒋玉函跟出来。
宝玉见他妩媚温柔,并得知他原来就是无缘相见的琪官儿,于是摘下玉玦扇坠送给他。
蒋玉函也将北静王送他的茜香国女国王的贡物大红汗巾回赠宝玉。
}