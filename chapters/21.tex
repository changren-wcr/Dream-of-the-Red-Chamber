%\chapter{贤袭人娇嗔箴宝玉\quad 俏平儿软语救贾琏}
\chapter[贤袭人娇嗔箴宝玉\quad 俏平儿软语救贾琏]{贤袭人\geng{当得起。
}娇嗔箴宝玉\quad 俏平儿软语救贾琏}
\geng{有客题《红楼梦》一律,失其姓氏,惟见其诗意骇警,故录于斯:\hang
自执金矛又执戈,自相戕戮自张罗。
\zhu{
戕[qiāng]:杀害;摧残。
张罗:张网。
这两句诗对家族内部的“自杀自灭”痛心疾首。
小说中最主要的“内斗”,一是荣府赵姨娘和贾环的庶子派对垒受贾母宠爱的宝玉和凤姐代表的嫡子派,
二是贾赦、邢夫人的荣府大房对垒贾政、王夫人的荣府二房。
正如第七十五回探春所说: “咱们倒是一家子亲骨肉呢,一个个不像乌眼鸡,恨不得你吃了我,我吃了你!”
}\hang
茜纱公子情无限,脂砚先生恨几多。
\zhu{
茜纱公子:指贾宝玉。
第五十八回所谓“茜纱窗真情揆痴理”,
第七十九回宝玉和黛玉修改《芙蓉诔》,黛玉说:“咱们如今都系霞影纱糊的窗槅,何不说‘茜纱窗下,公子多情’呢?”
}
\hang
是幻是真空历遍,闲风闲月枉吟哦。
\hang
情机转得情天破,情不情兮奈我何?
\zhu{
第十九回脂批:《情榜》评曰“宝玉情不情”,“黛玉情情”。
“情不情”是作者对贾宝玉的一种“谥法”。
}
\hang
凡是书题者,不可[不以]此为绝调。
诗句警拔,且深知拟书底里,惜乎失名矣!按此回之文固妙,然未见后之卅回,
\zhu{卅[sà]:三十。}
犹不见此之妙。
此回“娇嗔箴宝玉”、“软语救贾琏”,后回“薛宝钗借词含讽谏,王熙凤知命强英雄”。
\zhu{强:音“抢”,竭力,尽力,勉强。
强英雄:强撑着做英雄。
}今只从二婢说起,后则直指其主。
然今日之袭人、之宝玉,亦他日之袭人、他日之宝玉也。
今日之平儿、之贾琏,亦他日之平儿、他日之贾琏也。
何今日之玉犹可箴,他日之玉已不可箴耶?今日之琏犹可救,他日之琏已不可救耶?箴与谏无异也,而袭人安在哉?宁不悲乎!救与强无别也,今因平儿救,此日阿凤英气何如是也?他日之强,何身微运蹇,
\zhu{蹇:音“捡”,困苦,不顺利。}
展眼何如彼耶?甚矣,人世之变迁如此,光阴倏尔如此!\hang
今日写袭人,后文写宝钗;今日写平儿,后文写阿凤。
文是一样情理,景况光阴,事却天壤矣!多少恨泪洒出此两回书。
\hang
此回袭人三大功,直与宝玉一生三大病映射。
\zhu{
三大功、三大病:见本回后文脂评。
}
}\par
话说史湘云跑了出来,怕林黛玉赶上,宝玉在后忙说:“仔细绊跌了!那里就赶上了?”林黛玉赶到门前,被宝玉叉手在门框上拦住,笑劝道:“饶他这一遭罢。
”林黛玉搬着手说道:“我若饶过云儿,再不活着!”湘云见宝玉拦住门,料黛玉不能出来,\geng{写得湘云与宝玉又亲厚之极,却不见疏远黛玉,是何情思耶?}便立住脚笑道:“好姐姐,饶我这一遭罢。
”恰值宝钗来在湘云身后,也笑道:“我劝你两个看宝兄弟分上,都丢开手罢。
”\geng{好极,妙极!玉、颦、云三人已难解难分,插入宝钗云“我劝你两个看宝玉兄弟分上”,话只一句,便将四人一齐笼住,不知孰远孰近,孰亲孰疏,真好文字!}黛玉道:“我不依。
你们是一气的,都戏弄我不成!”\geng{话是颦儿口吻,虽属尖利,真实堪爱堪怜。
}宝玉劝道:“谁敢打趣你!你不打趣他,他焉敢说你?”\geng{好!二“你”字连二“他”字,华灼之至!}四人正难分解,\geng{好!前三人,今忽四人,俱是书中正眼,不可少矣。
}有人来请吃饭,方往前边来。
\geng{好文章!正是闺中女儿口角之事。
若只管谆谆不已,则成何文矣!}那天早又掌灯时分,\zhu{天早:天黑得早。
}王夫人、李纨、凤姐、迎、探、惜等都往贾母这边来,大家闲话了一回,各自归寝。
湘云仍往黛玉房中安歇。
\geng{前文黛玉未来时,湘云、宝玉则随贾母。
今湘云已去,黛玉既来,年岁渐成,宝玉各自有房,黛玉亦各有房,故湘云自应同黛玉一处也。
}\ping{黛玉虽追打湘云,两人却是真好,少客气少设防,谁能想象黛玉和宝钗追打玩闹?小孩子也不记仇,刚拌嘴打闹之后又和好如初。
}\par
宝玉送他二人到房,那天已二更多时,袭人来催了几次,方回自己房中来睡。
次日天明时,便披衣靸鞋往黛玉房中来,\zhu{靸:音“洒”,穿鞋时把鞋后帮踩在脚后跟下,拖着走。
}不见紫鹃、翠缕二人,只见他姊妹两个尚卧在衾内。
那林黛玉\geng{写黛玉身分。
}严严密密裹着一幅杏子红绫被,安稳合目而睡。
\geng{一个睡态。
}那史湘云却一把青丝拖于枕畔,
\zhu{青丝:比喻黑发(多用于女子的头发)。}
被只齐胸,一弯雪白的膀子掠于被外,又带着两个金镯子。
\geng{又一个睡态。
写黛玉之睡态,俨然就是娇弱女子,可怜。
湘云之态,则俨然是个娇态女儿,可爱。
真是人人俱尽,个个活跳,吾不知作者胸中埋伏多少裙钗。
}宝玉见了,叹道:\geng{“叹”字奇!除玉卿外,世人见之自曰喜也。
}“睡觉还是不老实!回来风吹了,又嚷肩窝疼了。
”一面说,一面轻轻的替他盖上。
林黛玉早已醒了,\geng{不醒不是黛玉了。
}觉得有人,就猜着定是宝玉,因翻身一看,果中其料。
因说道:“这早晚就跑过来作什么?”宝玉笑道:“这天还早呢!你起来瞧瞧。
”黛玉道:“你先出去,让我们起来。
”\geng{一丝不乱。
}宝玉听了,转身出至外边。
\par
黛玉起来叫醒湘云,二人都穿了衣服。
宝玉复又进来,坐在镜台旁边,只见紫鹃、雪雁进来伏侍梳洗。
湘云洗了面,翠缕便拿残水要泼,宝玉道:“站着,我趁势洗了就完了,省得又过去费事。
”说着便走过来,弯腰洗了两把。
\geng{妙在两把。
}紫鹃递过香皂去,宝玉道:“这盆里的就不少,不用搓了。
”\meng{此等用心淫极,请看却自不淫,非世之凡夫俗子得梦见者,真雅极趣极。
}再洗了两把,便要手巾。
\geng{在怡红何其费事多多。
}翠缕道:“还是这个毛病儿,多早晚才改。
”\geng{冷眼人旁点,一丝不漏。
}宝玉也不理,忙忙的要过青盐擦了牙,
\zhu{
青盐:出自盐井或盐湖的盐。主要产于大陆西南、西北等地区,以纯净色青者为佳,故称。可食用,也可入药。凉血明目,可抹牙、漱口、洗眼。
}
漱了口,完毕,见湘云已梳完了头,便走过来笑道:“好妹妹,替我梳上头罢。
”湘云道:“这可不能了。
”宝玉笑道:“好妹妹,你先时怎么替我梳了呢?”湘云道:“如今我忘了,\geng{“忘了”二字在娇憨口中自是应声而出,捉笔人却从何处设想而来,成此天然对答。
壬午九月。
}怎么梳呢?”宝玉道:“横竖我不出门,又不带冠子勒子,不过打几根散辫子就完了。
”说着,又千妹妹万妹妹的央告。
\meng{逼近情态。
}湘云只得扶过他的头来,一一梳篦。
在家不戴冠,并不总角,\zhu{总角:古代未成年人把头发扎成向上分开的两个发髻,形状像角。
}只将四围短发编成小辫,往顶心发上归了总,编一根大辫,红绦结住。
自发顶至辫梢,一路四颗珍珠,下面有金坠脚。
湘云一面编着,一面说道:“这珠子只三颗了,这一颗不是的。
\geng{梳头亦有文字,前已叙过,今将珠子一穿插,却天生有是事。
}我记得是一样的,怎么少了一颗?”\ping{闪回宝玉湘云往事。
}宝玉道:“丢了一颗。
”湘云道:“必定是外头去掉下来,不防被人拣了去,倒便宜他。
”\geng{妙谈!道“倒便宜他”四字,是大家千金口吻。
近日多用“可惜了的”四字。
今失一珠,不闻此四字。
妙极!是极!}
\geng{“倒便宜他”四字与“忘了”二字是一气而来,将一侯府千金白描矣。
畸笏。
}\meng{是湘云口气。
}黛玉一旁盥手,冷笑道:\geng{纯用画家烘染法。
}“也不知是真丢了,也不知是给了人镶什么戴去了!”\meng{是黛玉口气。
}宝玉不答,\geng{有神理,有文章。
}因镜台两边俱是妆奁等物,顺手拿起来赏玩,\geng{何赏玩也?写来奇特。
}不觉又顺手拈了胭脂,意欲要往口边送,\geng{是袭人劝后馀文。
}因又怕史湘云说。
\geng{好极!的是宝玉也。
}正犹豫间,湘云果在身后看见,一手掠着辫子,便伸手来“拍”的一下,从手中将胭脂打落,说道:“这不长进的毛病儿,多早晚才改过!”\geng{前翠缕之言并非白写。
}\par
一语未了,只见袭人进来,看见这般光景,知是梳洗过了,只得回来自己梳洗。
忽见宝钗走来,因问道:“宝兄弟那去了?”袭人含笑道:“宝兄弟那里还有在家的工夫!”宝钗听说,心中明白。
又听袭人叹道:“姊妹们和气,也有个分寸礼节,也没个黑家白日闹的!
\zhu{
没个黑家白日:不分白天夜晚。
}
凭人怎么劝,都是耳旁风。
”\ping{是为了礼节,还是受冷落而吃醋,抑或两者兼而有之?}宝钗听了,心中暗忖道:“倒别看错了这个丫头,听他说话,倒有些识见。
”\geng{此是宝卿初试,以下渐成知已,盖宝卿从此心察得袭人果贤女子也。
}宝钗便在炕上坐了,\geng{好!逐回细看,宝卿待人接物,不疏不亲,不远不近。
可厌之人,亦未见冷淡之态,形诸声色;可喜之人,亦未见醴密之情,\zhu{醴:音“礼”,甜酒,也指甜美的泉水。
}
形诸声色。
今日“便在炕上坐了”,盖深取袭卿矣。
二人文字,此回为始。
详批于此,诸公请记之。
}慢慢的闲言中套问他年纪家乡等语,留神窥察,其言语志量深可敬爱。
\geng{四字包罗许多文章笔墨,不似近之开口便云“非诸女子之可比者”,此句大坏。
然袭人故佳矣,不书此句是大手眼。
}\par
一时宝玉来了,宝钗方出去。
\geng{奇文!写得钗、玉二人形景较诸人皆近,何也?宝玉之心,凡女子前不论贵贱,皆亲密之至,岂于宝钗前反生远心哉?盖宝钗之行止端肃恭严,不可轻犯,宝玉欲近之,而恐一时有渎,故不敢狎犯也。
宝钗待下愚尚且和平亲密,何反于兄弟前有远心哉?盖宝玉之形景已泥于闺阁,
\zhu{泥[nì]:受限制而不知变通。如“泥古不化”。}
近之则恐不逊,反成远离之端也。
故二人之远,实相近之至也。
至颦儿于宝玉实近之至矣,却远之至也。
不然,后文如何凡较胜角口诸事皆出于颦哉?以及宝玉砸玉,颦儿之泪枯,种种孽障,种种忧忿,皆情之所陷,更何辩哉?\ping{这个批书人,大概是反对宝玉和黛玉的木石前盟,支持宝玉和宝钗的金玉良缘。
作为宝钗的粉丝,可以称之为“金玉党”。
}}\geng{此一回将宝玉、袭人、钗、颦、云等行止大概一描,已启后大观园中文字也。
今详批于此,后久不忘矣。
}\geng{钗与玉远中近,颦与玉近中远,是要紧两大股,不可粗心看过。
}宝玉便问袭人道:“怎么宝姐姐和你说的这么热闹,见我进来就跑了?”\geng{此问必有。
}\meng{我则以宝钗之去、因袭人之言,不得不去。
}问一声不答,再问时,袭人方道:“你问我么?我那里知道你们的原故。
”宝玉听了这话,见他脸上气色非往日可比,便笑道:“怎么动了真气?”\geng{宝玉如此。
}袭人冷笑道:“我那里敢动气!只是从今以后别再进这屋子了。
横竖有人伏侍你,再别来支使我。
我仍旧还伏侍老太太去。
”一面说,一面便在炕上合眼倒下。
\geng{醋妒娇憨假态,至矣尽矣!观者但莫认真此态为幸。
}
\meng{是醋?是谏?不敢拟定,似在可否之间!}宝玉见了这般景况,深为骇异,\geng{好!可知未尝见袭人之如此技艺也!}禁不住赶来劝慰。
那袭人只管合了眼不理。
\geng{与颦儿前番娇态如何?愈觉可爱犹甚。
}宝玉无了主意,因见麝月进来,\geng{偏麝月来,好文章!}便问道:“你姐姐怎么了?”\geng{如见如闻。
}麝月道:“我知道么?问你自己便明白了。
”\geng{又好麝月!}\meng{溺入者每受侮谩而不顾。
}宝玉听说,呆了一回,自觉无趣,便起身叹道:“不理我?罢!我也睡去。
”说着,便起身下炕,到自己床上歪下。
袭人听他半日无动静,微微的打鼾,\geng{真乎?诈乎?}料他睡着,便起身拿一领斗蓬来,替他刚压上,只听“忽”的一声,\geng{文是好文,唐突我袭卿,吾不忍也。
}
\meng{不可少。
}宝玉便掀过去,也仍合目装睡。
\geng{写得烂熳。
\zhu{烂熳:同“烂漫”。天真自然,毫不做作。
}}袭人明知其意,便点头冷笑道:“你也不用生气,从此后我只当哑子,再不说你一声儿,如何?”宝玉禁不住起身问道:“我又怎么了?你又劝我。
你劝我也罢了,才刚又没见你劝我,一进来你就不理我,赌气睡了。
我还摸不着是为什么,这会子你又说我恼了。
\geng{这是委屈了石兄。
}\meng{是神理。
}我何尝听见你劝我什么话了。
”袭人道:“你心里还不明白,还等我说呢!”\geng{亦是囫囵语,却从有生以来肺腑中出,千斤重。
}\geng{《石头记》每用囫囵语处,无不精绝奇绝,且总不觉相犯。
壬午九月。
畸笏。
}\par
正闹着,贾母遣人来叫他吃饭,方往前边来,胡乱吃了半碗,仍回自己房中。
只见袭人睡在外头炕上,麝月在旁边抹骨牌。
宝玉素知麝月与袭人亲厚,一并连麝月也不理,揭起软帘自往里间来。
麝月只得跟进来。
宝玉便推他出去,说:“不敢惊动你们。
”麝月只得笑着出来,唤了两个小丫头进来。
宝玉拿一本书,歪着看了半天,\meng{斗凑得巧。
\zhu{斗凑:凑合;连接,拼合。}
}因要茶,抬头只见两个小丫头在地下站着。
一个大些儿的生得十分水秀,\geng{二字奇绝!多少娇态包括一尽。
今古野史中无有此文也。
}宝玉便问:“你叫什么名字?”那丫头便说:“叫蕙香。
”\geng{也好。
}宝玉便问:“是谁起的?”蕙香道:“我原叫芸香的,\geng{原俗。
}是花大姐姐改了蕙香。
”宝玉道:“正经该叫‘晦气’罢了,什么蕙香呢!”\geng{好极!趣极!}又问:“你姊妹几个?”蕙香道:“四个。
”宝玉道:“你第几?”蕙香道:“第四。
”宝玉道:“明儿就叫‘四儿’,不必什么‘蕙香’‘兰气’的。
那一个配比这些花,没的玷辱了好名好姓。
”\zhu{没的:犹休要。
}\geng{“花袭人”三字在内,说的有趣。
}\ping{宝玉贬低以花为人命名的行为,实际上是表达对以花为姓的袭人的不满。
}一面说,一面命他倒了茶来吃。
袭人和麝月在外间听了抿嘴而笑。
\geng{一丝不漏,好精神!}\par
这一日,宝玉也不大出房,\geng{此是袭卿第一功劳也。
}\meng{“不大出房”四字,见宝玉是真情种。
}也不和姊妹丫头等厮闹,\geng{此是袭卿第二功劳也。
}
自己闷闷的,只不过拿着书解闷,或弄笔墨,\geng{此虽未必成功,较往日终有微补小益,所谓袭卿有三大功劳也。
}\meng{可怜可爱。
}也不使唤众人,只叫四儿答应。
谁知四儿是个聪敏乖巧不过的丫头,\geng{又是一个有害无益者。
作者一生为此所误,批者一生亦为此所误,于开卷凡见如此人,世人故为喜,余反抱恨,盖四字误人甚矣。
被误者深感此批。
}见宝玉用他,他变尽方法笼络宝玉。
\geng{也好,但不知袭卿之心思何如?}至晚饭后,宝玉因吃了两杯酒,眼饧耳热之际,\zhu{
饧:音“行”,眼睛半睁半闭或呆滞无神。
眼饧:眼皮滞涩、朦胧欲睡。
}若往日则有袭人等大家喜笑有兴,今日却冷清清的一人对灯,好没兴趣。
待要赶了他们去,
\zhu{赶:追。}
又怕他们得了意,以后越发来劝,\geng{宝玉恶劝,此是第一大病也。
}若拿出做上的规矩来镇唬,似乎无情太甚。
\geng{宝玉重情不重礼,此是第二大病也。
}说不得横心只当他们死了,横竖自然也要过的。
便权当他们死了,毫无牵挂,反能怡然自悦。
\geng{此意却好,但袭卿辈不应如此弃也。
宝玉之情,今古无人可比,固矣。
然宝玉有情极之毒,
\zhu{毒:指宝玉悬崖撒手、出家为僧弃钗麝的决绝。}
亦世人莫忍为者,看至后半部则洞明矣。
此是宝玉第三大病也。
宝玉有此世人莫忍为之毒,故后文方有“悬崖撒手”一回。
若他人得宝钗之妻、麝月之婢,岂能弃而为僧哉?此宝玉一生偏僻处。
}\meng{此是宝玉大智慧、大力量处,别个不能,我也不能。
}因命四儿剪灯烹茶,自己看一回《南华经》。
\zhu{《南华经》即《庄子》。
}正看至《外篇·胠箧》一则,\zhu{胠:音“区”,从旁打开(器物)。
箧:音“妾”,小箱子。
}其文曰:\par
\hop
故绝圣弃知,\zhu{绝圣弃知:抛弃聪明智巧的意思。
语出《老子》第十九章:“绝圣弃知,民利百倍。
”“圣”在这里是睿智、于事无所不通的意思。
}大盗乃止,擿玉毁珠,\zhu{擿:音“至”,同“掷”,扔掉。
}小盗不起,焚符破玺,\zhu{焚符破玺:烧毁信符,砸碎印玺。
符:古时用竹、木、金、玉等制成,上刻文字,剖成两半,以相契合为凭证。
玺:印章。
}而民朴鄙,
\zhu{
朴鄙:朴质鄙陋。也作「朴野」。
欧阳修《谢知制诰表》:「志欲去于雕华,文反成于朴鄙。」
}
掊斗折衡,\zhu{
掊:音“剖”,击。
衡:秤。
掊斗折衡:把斗击破,把秤折断。
}而民不争,殚残天下之圣法,\zhu{
殚:音“单”,尽。
殚残:毁灭。
}而民始可与论议。
擢乱六律,\zhu{擢(擢音“浊”)乱:搅乱。
六律:音乐术语,我国古代律制将一个八度分为十二个音阶,从低到高逢奇数的合称“六律”,逢偶数的合称“六吕”。
在这里“六律”泛指音律。
}铄绝竽瑟,\zhu{
铄:音“硕”,销熔。
这里竽瑟泛指乐器。
铄绝竽瑟:销毁乐器。
}塞瞽旷之耳,\zhu{
瞽:音“古”,瞎眼。
先秦时以盲人为乐官,故“瞽”又为乐官的代称。
瞽旷:即师旷,春秋时代晋国乐师,目盲。
相传他善于审音辨律。
}而天下始人含其聪矣;\zhu{聪:灵敏的听觉。
}灭文章,
\zhu{文章:斑斓美丽的花纹。}
散五采,胶离朱之目,\zhu{离朱:亦作“离娄”,古代传说中视力最强的人。
}而天下始人含其明矣,毁绝钩绳而弃规矩,\zhu{钩:定曲线的工具。
绳:定直线的工具。
规:画圆形的工具。
矩:画方形的工具。
}
攦工倕之指,\zhu{攦:音“立”,折断。
工倕(倕音“垂”):相传为尧时巧匠。
}而天下始人有其巧矣。
\geng{此上语本《庄子》。
}\par
\hop
看至此,意趣洋洋,趁着酒兴,不禁提笔续曰:\geng{趁着酒兴不禁而续,是作者自站地步处,谓余何人耶,敢续《庄子》?然奇极怪极之笔,从何设想,怎不令人叫绝?己卯冬夜。
}\geng{这亦暗露玉兄闲窗净几、不\sout{寂}[即]不离之\sout{工}[功]业。
\zhu{这条评语是说从宝玉续《庄子》可以看出宝玉读书用功的效果。}
壬午孟夏。
}\meng{敢续!}\par
\hop
焚花散麝,而闺阁始人含其劝矣,\geng{奇。
}\zhu{劝:劝勉,箴规。
在这里作名词用。
}\ping{花暗指花袭人,麝暗指麝月。
}戕宝钗之仙姿,
\zhu{戕[qiāng]:杀害;摧残。}
灰黛玉之灵窍,丧减情意,而闺阁之美恶始相类矣。
彼含其劝,则无参商之虞矣,戕其仙姿,无恋爱之心矣,灰其灵窍,无才思之情矣。
彼钗、玉、花、麝者,皆张其罗而穴其隧,\zhu{张其罗:张网。
穴其隧:挖好了陷阱。
穴:洞。
此处用作动词,义同挖。
隧:地道。
此处引申作陷阱。
}所以迷眩缠陷天下者也。
\zhu{迷眩:指用声色迷惑人。
缠陷:指用罗网陷阱捕捉人。
眩:昏花惑乱。
}
\geng{直似庄老,奇甚怪甚!}\geng{赵香梗先生《秋树根偶谭》内,兖州少陵台有子美祠为郡守毁为己祠。
先生叹子美生遭丧乱,奔走无家,孰料千百年后数椽片瓦犹遭贪吏之毒手。
甚矣,才人之厄也!因改公《茅屋为秋风所破歌》数句,为少陵解嘲:“少陵遗像太守欺无力,忍能对面为盗贼,公然\sout{折克非}[拆去作]
己祠,旁人有口呼不得,梦归来兮闻叹息,白日无光天地黑。
安得旷宅千万间,太守取之不尽生欢颜,公祠免毁安如山。
”读之令人感慨悲愤,心常耿耿。
壬午九月。
——因索书甚迫,姑志于此,非批《石头记》也。
}\geng{为续《庄子因》数句,
\zhu{《庄子因》:一部阐释《庄子》的书,清代康熙时林云铭著。}
真是打破胭脂阵,坐透红粉关,另开生面之文,无可评处。
}\meng{见得透彻,恨不守此,人人同病。
}\par
续毕,掷笔就寝。
头刚着枕便忽睡去,一夜竟不知所之,直至天明方醒。
\geng{此犹是袭人馀功也。
想每日每夜,宝玉自是心忙身忙口忙之极,今则怡然自适。
虽此一刻,于身心无所补益,能有一时之闲闲自若,亦岂非袭卿之所使然耶?}翻身看时,只见袭人和衣睡在衾上。
\geng{神极之笔!试思袭人不来同卧亦不成文字,来同卧更不成文字。
却云“和衣衾上”,正是来同卧不来同卧之间。
何神奇又妙绝矣!好袭人,真好!石头记得真,真好!述者述得不错,真好!批者批得出。}宝玉将昨日的事已付与度外,\geng{更好!可见玉卿的是天真烂漫之人也!近之所谓呆公子,又曰“老好人”、又曰“无心道人”是也!殊不知尚古淳风。
}便推他说道:“起来好生睡,看冻着了。
”\par
原来袭人见他无晓夜和姊妹们厮闹,若直劝他,料不能改,故用柔情以警之,料他不过半日片刻仍复好了。
不想宝玉一日一夜竟不回转,自己反不得主意,直一夜没好生睡得。
今忽见宝玉如此,料他心意回转,便越性不睬他。
宝玉见他不应,便伸手替他解衣,刚解开了钮子,被袭人将手推开,\geng{好看煞!}又自扣了。
宝玉无法,只得拉他的手笑道:“你到底怎么了?”连问几声,袭人睁眼说道:“我也不怎么。
你睡醒了,你自过那边房里去梳洗,再迟了就赶不上。
”\geng{说得好痛快。
}宝玉道:“我过那里去?”\geng{问得更好。
}袭人冷笑道:“你问我,\geng{三字如闻。
}
我知道?你爱往那里去,就往那里去。
从今咱们两个丢开手,省得鸡声鹅斗,叫别人笑。
横竖那边腻了过来,这边又有个什么‘四儿’‘五儿’伏侍。
我们这起东西,可是‘白玷辱了好名好姓’的。
”宝玉笑道:“你今儿还记着呢!”\geng{非浑一纯粹,那能至此!}袭人道:“一百年还记着呢!比不得你,拿着我的话当耳旁风,夜里说了,早起就忘了。
”\geng{这方是正文,直勾起“花解语”一回文字。
}宝玉见他娇嗔满面,情不可禁,\geng{又用幻笔瞒过看官。
}便向枕边拿起一根玉簪来,一跌两段,说道:“我再不听你说,就同这个一样。
”\meng{迎头一棒!}袭人忙的拾了簪子,说道:“大清早起,这是何苦来!\meng{撞心儿盟誓,教人听了折柔肠,好些不忍。
}听不听什么要紧,\geng{已留后文地步。
}也值得这种样子。
”宝玉道:“你那里知道我心里急!”袭人笑道:\geng{自此方笑。
}“你也知道着急么!可知我心里怎么着?快起来洗脸去罢。
”\geng{结得一星渣滓全无,且合怡红常事。
}说着,二人方起来梳洗。
\par
宝玉往上房去后,谁知黛玉走来,见宝玉不在房中,因翻弄案上书看,可巧翻出昨儿的《庄子》来。
看至所续之处,不觉又气又笑,不禁也提笔续书一绝云:\par
\hop
无端弄笔是何人?作践南华《庄子因》。
\zhu{《庄子因》:一部阐释《庄子》的书,清代康熙时林云铭著。
}\par
不悔自己无见识,却将丑语怪他人。
\geng{骂得痛快,非颦儿不可。
真好颦儿,真好颦儿!好诗!若云知音者,颦儿也。
至此方完“箴玉”半回。
} 
\geng{不用宝玉见此诗,若长若短,亦是大手法。
}\geng{又借阿颦诗自相鄙驳,可见余前批不谬。
己卯冬夜。
}\geng{宝玉不见诗,是后文馀步也,《石头记》得力所在。
\zhu{本回和下回的正文中没有再提及黛玉写的这首诗,宝玉依旧执迷不悟,所以下回宝玉又写了一偈及一支《寄生草》。}
丁亥夏。
畸笏叟。
}\par
\hop
写毕,也往上房来见贾母,后往王夫人处来。
\par
谁知凤姐之女大姐病了,正乱着请大夫来诊脉。
大夫便说:“替夫人奶奶们道喜,姐儿发热是见喜了,\zhu{见喜:旧时以小儿出痘疹(天花)为险症,忌讳直说,又因痘疹发出后可望平安,所以称为“见喜”。
下文“痘疹娘娘”是迷信传说中专管小儿痘疹的神。
}并非别病。
”王夫人凤姐听了,忙遣人问:“可好不好?”医生回道:“病虽险,却顺,\geng{在“子嗣艰难”化出。
\zhu{
子嗣艰难:没有儿子的委婉说法。古人认为天花为“先天之毒(又说,胎毒)”。
凤姐没有儿子,唯一的女儿还得天花,膝下寥落,难有后代。
}
}倒还不妨。
预备桑虫猪尾要紧。
\zhu{桑虫猪尾:按照中医理论,桑虫有驱风之功,猪尾有除毒之效。}
”凤姐听了,登时忙将起来:一面打扫房屋供奉痘疹娘娘,一面传与家人忌煎炒等物,一面命平儿打点铺盖衣服与贾琏隔房,一面又拿大红尺头与奶子丫头亲近人等裁衣。
\geng{几个“一面”,写得如见其景。
}外面又打扫净室,款留两个医生,轮流斟酌诊脉下药,十二日不放家去。
贾琏只得搬出外书房来斋戒,\geng{此二字内生出许多事来。
}凤姐与平儿都随着王夫人日日供奉娘娘。
\meng{写尽母氏为子之心。
}\par
那个贾琏,只离了凤姐便要寻事,独寝了两夜,便十分难熬,便暂将小厮们内有清俊的选来出火。
\zhu{
出火:泄欲或泄愤之意。这里指发泄性欲。
}
不想荣国府内有一个极不成器破烂酒头厨子,
\zhu{
酒头:好酒糊涂,被欺而不自知的人;指容易上当受骗而无自知之明的人。
}
名唤多官,\geng{今是多多也,妙名!}人见他懦弱无能,都唤他作“多浑虫”。
\geng{更好!今之浑虫更多也。
}因他自小父母替他在外娶了一个媳妇,今年方二十来往年纪,生得有几分人才,见者无不羡爱。
他生性轻浮,最喜拈花惹草,多浑虫又不理论,只是有酒有肉有钱,便诸事不管了,所以荣宁二府之人都得入手。
因这个媳妇美貌异常,轻浮无比,众人都呼他作“多姑娘儿”。
\geng{更妙!}如今贾琏在外熬煎,往日也曾见过这媳妇,失过魂魄,只是内惧娇妻,外惧娈宠,\zhu{娈(娈音“峦”)宠:即男宠。
}不曾下得手。
那多姑娘儿也曾有意于贾琏,只恨没空。
今闻贾琏挪在外书房来,他便没事也要走两趟去招惹。
惹的贾琏似饥鼠一般,少不得和心腹的小厮们计议,合同遮掩谋求,多以金帛相许。
小厮们焉有不允之理,况都和这媳妇是好友,一说便成。
是夜二鼓人定,
\zhu{二鼓:晚九点到十一点。}
多浑虫醉昏在炕,贾琏便溜了来相会。
进门一见其态,早已魄飞魂散,也不用情谈款叙,便宽衣动作起来。
谁知这媳妇有天生的奇趣,一经男子挨身,便觉遍身筋骨瘫软,\geng{淫极!亏想的出!}使男子如卧棉上,\geng{如此境界,自胜西方、蓬莱等处。
}更兼淫态\geng{总为后文宝玉一篇作引。
\zhu{第七十七回,宝玉去探视病卧在嫂子多姑娘家的晴雯,受到了多姑娘的调戏。}
}浪言,压倒娼妓,诸男子至此岂有惜命者哉。
\geng{凉水灌顶之句。
}那贾琏恨不得连身子化在他身上。
\geng{亲极之语,趣极之语。
}
那媳妇故作浪语,在下说道:“你家女儿出花儿,供着娘娘,你也该忌两日,倒为我脏了身子。
快离了我这里罢。
”\geng{淫妇勾人,惯加反语,看官着眼。
}贾琏一面大动,一面喘吁吁答道:“你就是娘娘!我那里管什么娘娘!”\geng{乱语不伦,的是有之。
}那媳妇越浪,贾琏越丑态毕露。
\geng{可以喷饭!}一时事毕,两个又海誓山盟,难分难舍,\geng{着眼,再从前看如何光景。
}\meng{此种文字亦不可少,请看者自度。
}此后遂成相契。
\zhu{相契:互相投合。}
\geng{趣闻!“相契”作如此用,“相契”扫地矣。
}\geng{一部书中,只有此一段丑极太露之文,写于贾琏身上,恰极当极!己卯冬夜。
}\geng{看官熟思:写珍、琏辈当以何等文方妥方恰也?壬午孟夏。
}\geng{此段系书中情之瘕疵,写为阿凤生日泼醋回及“夭风流”宝玉悄看晴雯回作引,伏线千里外之笔也。
丁亥夏。
畸笏。
\zhu{泼醋回:第四十四回。晴雯回:第七十七回。}
}\par
一日大姐毒尽癍回,
\zhu{癍[bān]:皮肤生斑点的病;也指皮肤生的斑点。}
\geng{好快日子吓!
\zhu{吓[hè]:叹词,表示不满意,认为不该如此。}
}十二日后送了娘娘,合家祭天祀祖,还愿焚香,\zhu{还愿:求神保佑的人实践对神许下的报酬,如祭祀、慈善、捐献。
}庆贺放赏已毕,贾琏仍复搬进卧室。
见了凤姐,正是俗语云“新婚不如远别”,更有无限恩爱,自不必烦絮。
\zhu{烦絮:言语噜苏不简要。}
\geng{隐得好。
}\par
次日早起,凤姐往上屋去后,平儿收拾贾琏在外的衣服铺盖,不承望枕套中抖出一绺青丝来。
\zhu{绺[liǔ]:量词,用于聚集成束的细丝状的东西。}
平儿会意,忙拽在袖内,\geng{好极!不料平儿大有袭卿之身分,可谓何地无材,盖遭际有别耳。
}便走至这边房内来,拿出头发来,向贾琏笑道:“这是什么?”\geng{好看之极!}贾琏看见着了忙,\geng{也有今日。
}抢上来要夺。
平儿便跑,被贾琏一把揪住,按在炕上,掰手要夺,口内笑道:“小蹄子,\zhu{小蹄子:骂年轻女孩或婢女的话。
}你不趁早拿出来,我把你膀子撅折了。
”\geng{无情太甚!}\meng{此等人口中只好说此等话。
}平儿笑道:“你就是没良心的。
我好意瞒着他来问,你倒赌狠!
\zhu{赌狠:发狠、逞凶。}
你只赌狠,等他回来我告诉他,\geng{有是语,恐卿口不应[心]。
}看你怎么着。
”\ping{宝玉和姐妹嬉笑,预备姨娘袭人已微含酸,而贾琏胡闹,平儿这正式的通房却毫无怒意。
}贾琏听说,忙陪笑央求道:“好人,赏我罢,\meng{彼此用强用霸。
}我再不赌狠了。
”\geng{好听好看之极,迥不犯袭卿。
}\par
一语未了,只听凤姐声音进来。
\geng{惊天骇地之文!如何?不知下文怎样了结,使贾琏及观者一齐丧胆。
}\geng{《石头记》大法小法累累如是,并不为厌。
}
贾琏听见,松了手不是,还要抢又不是,只叫:“好人,别叫他知道。
”平儿刚起身,凤姐已走进来,命平儿“快开匣子,替太太找样子”。
平儿忙答应了找时,凤姐见了贾琏,忽然想起来,便问平儿:“拿出去的东西都收进来了么?”平儿道:“收进来了。
”凤姐道:“可少什么没有?”平儿道:“我也怕丢下一两件,细细的查了查,也不少。
”凤姐道:“不少就好,只是别多出来罢?” \geng{奇!}\geng{看至此,宁不拍案叫绝?}平儿笑道:“不丢万幸,谁还添出来呢?”\geng{可儿可儿,卿亦明知故说耳。
}凤姐冷笑道:“这半个月难保干净,或者有相厚的丢下的东西:戒指、汗巾、\zhu{汗巾:系腰用的长巾。
}香袋儿,再至于头发、指甲,都是东西。
”\geng{好阿凤,令人胆寒。
}\meng{行文故犯,反觉别致。
}一席话,说的贾琏脸都黄了。
贾琏在凤姐身后,只望着平儿杀鸡抹脖使眼色儿。
\meng{作丈夫者,要当自重!}平儿只装着看不见,\geng{余自有三分主意。
}因笑道:“怎么我的心就和奶奶的心一样!我就怕有这些个,留神搜了一搜,竟一点破绽也没有。
奶奶不信时,那些东西我还没收呢,奶奶亲自翻寻一遍去。
”\geng{好平儿!遍天下惧内者来感谢。
}凤姐笑道:“傻丫头,\geng{可叹可笑,竟不知谁傻。
}他便有这些东西,那里就叫咱们翻着了!”\geng{好阿凤,好文字,虽系闺中女儿口角小事,读之不无聪明、得失、痴心、真假之感。
}说着,寻了样子又上去了。
\par
平儿指着鼻子,\geng{好看煞。
}晃着头笑道:\geng{可儿,可儿。
}“这件事怎么回谢我呢?”\geng{姣俏如见,迥不犯袭卿麝月一笔。
}喜的个贾琏身痒难挠,\geng{不但贾兄痒痒,即批书人此刻几乎落笔。
试问看官此际若何光景?}跑上来搂着,“心肝肠肉”乱叫乱谢。
平儿仍拿了头发笑道:“这是我一生的把柄了。
好就好,不好就抖露出这事来。
”贾琏笑道:“你只好生收着罢,千万别叫他知道。
”口里说着,瞅他不防,便抢了过来,\geng{毕肖。
琏兄不分玉石,但负我平姐。
奈何,奈何!}笑道:“你拿着终是祸患,不如我烧了他完事了。
”\geng{妙!设使平儿[收了,]再不致泄露,故仍用贾琏抢回,后文遗失,[方能穿插]过脉也。
\zhu{平儿细心,不会遗失。
而贾琏粗心,抢回来之后遗失自己出轨的罪证,可能后来被凤姐发现。
可惜这个情节大概在本书遗失的章回里。
}}一面说着,一面便塞于靴掖内。
平儿咬牙道:“没良心的东西,过了河就拆桥,明儿还想我替你撒谎!”贾琏见他娇俏动情,便搂着求欢,被平儿夺手跑了,急的贾琏弯着腰恨道:“死促狭小淫妇!\zhu{促狭:刁钻机灵,爱捉弄人。
}一定浪上人的火来,他又跑了。
”\geng{丑态如见,淫声如闻,今古淫书未有之章法。
}
平儿在窗外笑道:“我浪我的,谁叫你动火了?\geng{妙极之谈。
直是理学工夫,
\zhu{
直是:竟是;真是。
理学:性理之学。宋代儒家学者以传道为使命,注重阐释经义,兼谈性命,并转化禅、道思想精华和修养方式所产生的学派。
元代时衰落,明朝又再度兴起。王阳明继承陆九渊的学说,并发扬光大。但后来只知言心言性,而欠缺实践,流于空谈。
}
所谓不可正照风月鉴也。
}难道图你\geng{阿平“你”字作牵强,余不画押。
一笑。
\zhu{
贾琏欲搂着平儿求欢,平儿跑了。整个过程,是贾琏主动,平儿被动,借用批者的话来说,应是贾琏“自作牵强”、平儿不愿“画押”。
现在批者看此产生联想,进入角色,把贾琏与平儿的关系当成了自己与平儿的关系,自作多情地反说平儿“自作牵强,余不画押”。
这虽是戏语,但也未免趣味低下,毫不足取。
}
}受用一回,叫他知道了,又不待见我。
”\zhu{不待见:不喜欢、讨厌的意思。
俗有“人嫌狗不待见”的话。
}\geng{凤姐醋妒,于平儿前犹如是,况他人乎!余谓凤姐必是甚于诸人。
观者不信,今平儿说出,然乎?否乎?}贾琏道:“你不用怕他,等我性子上来,把这醋罐打个稀烂,他才认得我呢!他防我像防贼的,只许他同男人说话,不许我和女人说话,我和女人略近些,他就疑惑,他不论小叔子侄儿,大的小的,说说笑笑,就不怕我吃醋了。
\meng{作者又何必如此想?亦犯此病也!}以后我也不许他见人!”\geng{无理之甚,却是妙极趣谈,天下惧内者背后之谈皆如此。
}平儿道:“他醋你使得,你醋他使不得。
他原行的正走的正,你行动便有个坏心,连我也不放心,别说他了。
”贾琏道:“你两个一口贼气。
都是你们行的是,我凡行动都存坏心。
\meng{一片俗气!}多早晚都死在我手里!”\ping{贾琏首次暴露狠象,说的话可能是谶语。
}\par
一句未了,凤姐走进院来,因见平儿在窗外,就问道:“要说话两个人不在屋里说,怎么跑出一个来,隔着窗子,是什么意思?”贾琏在窗内接道:“你可问他,倒像屋里有老虎吃他呢。
”\geng{好!}\geng{此等章法是在戏场上得来,一笑。
畸笏。
}平儿道:“屋里一个人没有,我在他跟前作什么?”凤姐儿笑道:“正是没人才好呢。
”平儿听说,便说道:“这话是说我呢?”凤姐笑道:\geng{“笑”字妙!平儿反正色,凤姐反陪笑,奇极意外之文。
}“不说你说谁?”平儿道:“别叫我说出好话来了。
”说着,也不打帘子让凤姐,自己先摔帘子进来,\geng{若在屋里,何敢如此形景,不要加上许多小心?平儿平儿,有你说嘴的。
\zhu{平儿如果在屋内,那么在吃醋的凤姐面前将无法自证清白;平儿在屋外行得端坐得正,面对吃醋的凤姐有底气顶嘴。}
}\ping{平儿作为贾琏的妾,在凤姐的淫威下,不敢于和贾琏亲近,刻意保持距离,不过是有名无实的夫妻。
就这样凤姐还是吃醋,怀疑自己,所以平儿才会感觉自己委屈冤枉。
}往那边去了。
凤姐自掀帘子进来,说道:“平儿疯魔了。
这蹄子认真要降伏我,仔细你的皮要紧!”贾琏听了,已绝倒在炕上,\zhu{绝倒:大笑不能自持。
}\geng{惧内形景写尽了。
}拍手笑道:“我竟不知平儿这么利害,从此倒伏他了。
”凤姐道:“都是你惯的他,我只和你说!”贾琏听说忙道:“你两个不卯,\zhu{不卯:不投合的意思。
卯:即卯眼,器物上安榫头的孔眼。
}又拿我来作人。
\zhu{作人:这里是作践人、拿人出气的意思。
}我躲开你们。
”凤姐道:“我看你躲到那里去。
”\meng{世俗之态熏人。
}贾琏道:“我就来。
”凤姐道:“我有话和你商量。
”不知商量何事,且听下回分解。
\geng{收\sout{后}[得]
淡雅之至!}正是:\par
\hop
淑女从来多抱怨,娇妻自古便含酸。
\geng{二语包尽古今万万世裙钗。
}\par
\hop
\qi{总评:不惜恩爱为良人,\zhu{良人:古代女子对丈夫的称呼。}方是温存一脉真。\zhu{这两句咏叹袭人谏劝宝玉的情节。}
俗子妒妇浑可笑,\zhu{俗子妒妇:偷情的贾琏和醋妒的凤姐。}语言偏自涉风尘。
}
\dai{041}{黛玉追打湘云,宝玉拦在门框,宝钗前来解劝}
\dai{042}{贾琏和平儿隔窗对话,凤姐吃醋}
\sun{p21-1}{林黛玉俏语谑娇音,宝袭冷战四儿得宠,隔窗调笑凤姐吃醋}{图右上:黛玉学湘云咬舌说话,湘云回敬道: “明儿得一个咬舌林姐夫,时刻可听‘爱’呀‘厄’的。
”说完返身就跑。
黛玉听湘云刺她便追,被宝玉拦在门口。
正在闹着,宝钗在湘云身后笑道:“我劝你们两个看宝兄弟面上,都撂开手吧。
”图左侧:袭人见宝玉终日与姐妹们厮混,将她昔日的规劝忘在脑后,故意不理他。
宝玉自觉无趣,拿了本书独自解闷,也不使唤众人, 只叫四儿答应。
图右下:凤姐之女大姐儿出痘疹,为供奉痘疹娘娘,凤姐与贾琏隔房。
贾琏离了凤姐,与多姑娘鬼混。
送走痘疹娘娘后,平儿收拾外边拿进来的被褥发现了一缕青丝,和贾琏隔着窗户调笑,可巧凤姐这时走进院来。
}