\chapter{送宫花周瑞叹英莲\quad 谈肄业秦钟结宝玉}
\zhu{肄:音“亿”。肄业:练习,学习。
}
\par
\qi{苦尽甘来递转,正强忽弱谁明?惺惺自古惜惺惺,世运文章操劲。
无缝机关难见,多才笔墨偏精。
有情情处特无情,何是人人不醒?}\par
题曰:\par
十二花容色最新,不知谁是惜花人?\par
相逢若问名何氏?家住江南姓本秦。
\par
\hop
话说周瑞家的送了刘姥姥去后,便上来回王夫人话。
\jia{不回凤姐,却回王夫人,不交代处,正交代得清楚。
}谁知王夫人不在上房,问丫鬟们时,方知往薛姨妈那边闲话去了。
\jia{文章只是随笔写来,便有流离生动之妙。
}周瑞家的听说,便转东角门出至东院,往梨香院来。
刚至院门前,只见王夫人的丫鬟名金钏儿\jia{金钏、宝钗互相映射。
妙!}者,和一个才留了头的小女孩儿\jia{莲卿别来无恙否?}站立台矶上顽。
\zhu{留头:又叫“留满头”。
旧时女子幼年剃发,随着年事增长,先留顶心头发,再留全发,叫做“留头”。
}见周瑞家的来了,便知有话回,因向内努嘴儿。
\jia{画。
}周瑞家的轻轻掀帘进去,只见王夫人和薛姨妈长篇大套的说些家务人情等语。
\meng{非此等事,不能长篇大套。
}\par
周瑞家的不敢惊动,遂进里间来。
\jia{总用双歧岔路之笔,令人估料不到之文。
}只见薛宝钗\jia{自入梨香院,至此方写。
}穿着家常衣服,\jia{好!写一人换一副笔墨,另出一花样。
} \jia{“家常爱着旧衣裳”是也。
}头上只散挽着䰖儿,\zhu{䰖:音“钻”三声。䰖儿:也写作“纂儿”。
妇女的发髻。
}坐在炕里边,伏在小炕几上,同丫鬟莺儿正描花样子呢。
\jia{一幅《绣窗仕女图》,亏想得周到。
}见他进来,宝钗便放下笔,转过身来,满面堆笑让:“周姐姐坐。
”周瑞家的也忙陪笑问:“姑娘好?”一面炕沿边坐了,因说:“这有两三天也没见姑娘到那边逛逛去,只怕是你宝玉兄弟冲撞了你不成?”\jia{一人不漏,一笔不板。
}宝钗笑道:“那里的话。
只因我那种病又发了两天,\jia{“那种病”。
“那”字与前二玉“不知因何”二“又”字,
\zhu{二“又”字:第五回:“黛玉又气的独在房中垂泪,宝玉又自悔语言冒撞,前去俯就”。}
皆得天成地设之体;且省却多少闲文,所谓“惜墨如金”是也。
}所以且静养两日。
”\jia{得空便入。
}\ping{宝钗确实妥帖,如沐春风。
}周瑞家的道:“正是呢,姑娘到底有什么病根儿,也该趁早儿请了大夫来,好生开个方子,认真吃几剂药,一势除了根才好。
小小的年纪倒坐下个病根,也不是顽的。
”宝钗听说,便笑道:“再不要提吃药,为这病请大夫、吃药,也不知白花了多少银子钱呢。
凭你什么名医仙药,总不见一点儿效。
后来还亏了一个秃头和尚,\jia{奇奇怪怪,真如云龙作雨,忽隐忽见,使人逆料不到。
\zhu{逆:事先。
}}说专治无名之症,因请他看了。
他说我这是从胎里带来的一股热毒,\jia{凡心偶炽,是以孽火齐攻。
}\qi{“热毒”二字画出富家夫妇,图一时遗害于子女,而可不谨慎!
}幸而我先天结壮,\zhu{结壮:结实强壮。
}还不相干。
\jia{浑厚故也,假使颦、凤辈,不知又何如治之。
}若吃凡药,是不中用的。
他就说了一个海上方,\zhu{海上方:旧时传说,东海(一说在渤海)之中的蓬莱、方丈、瀛洲三神山上,有不死之药。
后人遂称民间验方,秘方为“海上方”,意谓从东海神仙处求得的灵验药方。
}又给了一包末药作引,\zhu{末药:药末子。引:即“药引”,指处方中能引药力达到病变部位的药物,是中医方剂中“君、臣、佐、使”四个部分的“使”的俗称,也叫“引经报使”或“引经药”。
}异香异气的。
不知是那里弄来的。
他说发了时吃一丸就好。
倒也奇怪,这倒效验些。
”\jia{卿不知从那里弄来,余则深知是从放春山采来,以灌愁海水和成,烦广寒玉兔捣碎,在太虚幻境空灵殿上炮制配合者也。
}\par
周瑞家的因问道:“不知是个什么海上方儿?姑娘说了,我们也记着,说与人知道,倘遇见这样的病,也是行好的事。
”宝钗见问,乃笑道:“不问这方儿还好,若问起这方儿,真真把人琐碎坏了。
东西药料一概都有,现易得的,只难得‘可巧’二字:要春天开的白牡丹花蕊十二两,\jia{凡用“十二”字样,皆照应十二钗。
}\meng{周岁十二月之象。
}夏天开的白荷花蕊十二两,秋天开的白芙蓉花蕊十二两,冬天开的白梅花蕊十二两。
将这四样花蕊,于次年春分这日晒干,和在末药一处,一齐研好。
又要雨水这日的雨水十二钱,……”周瑞家的忙道:“嗳哟!这样说来,这就得一二年的工夫。
倘或雨水这日不下雨水,又怎处呢?”宝钗笑道:“所以了,那里有这样可巧的雨,便没雨也只好再等罢了。
白露这日的露水十二钱,霜降这日的霜十二钱,小雪这日的雪十二钱。
把这四样水调匀,和了丸药,再加蜂蜜十二钱,白糖十二钱,丸了龙眼大的丸子,
\zhu{
龙眼:植物名。无患子科龙眼属,常绿乔木。
果实称为「龙眼」,呈球形,黄褐色,
外具不规则而不显明的浅龟甲纹,假种皮味甜,供食用。
也称为「骊珠」、「荔枝奴」、「龙目」、「桂圆」。
}
盛在旧磁罐内,
\zhu{磁:通「瓷」。以瓷土烧制成的器物。}
埋在花根底下。
若发了病时,拿出来吃一丸,用十二分黄柏\qi{历着炎凉,知着甘苦,虽离别亦自能安,故名曰冷香丸。
又以谓香可冷得,天下一切无不可冷者。
}煎汤送下。
”\zhu{黄柏:中药名,性寒味苦,清热解毒。
}\jia{末用黄柏更妙。
可知“甘苦”二字,不独十二钗,世皆同有者。
}\ping{奇方幻药,风雅至极。
但是可能只有黄柏才是有效成分。
}\par
周瑞家的听了,笑道:“阿弥陀佛,真坑死了人\foot{原作“真巧死了人”,己、庚本作“真坑死人的事儿”,“巧”当系“坑”字之讹,据改。
按此处口语以“坑死人”为传神,且下一句又有一个“巧”字,也以不重复为佳。
}!等十年未必都这样巧呢。
”宝钗道:“竟好,自他说了去后,一二年间可巧都得了,好容易配成一料。
如今从南带至北,现就埋在梨花树下。
”\jia{“梨香”二字有着落,并未白白虚设。
}周瑞家的又道:“这药可有名字没有呢?”宝钗道:“有。
\jia{一字句。
}这也是癞和尚说下的,叫作‘冷香丸’。
\ping{
宝钗在前文说的是“秃头和尚”,这里又变成了“癞和尚”,可能是作者的笔误,
或者抑或是作者在暗示和尚送药之事子虚乌有,宝钗杜撰的故事自相矛盾。
}
”\jia{新雅,奇甚。
}周瑞家的听了点头儿,因又说:“这病发了时到底觉怎样?”宝钗道:“也不觉什么,只不过喘嗽些,吃一丸也就罢了。
”\jia{以花为药,可是吃烟火人想得出者?诸公且不必问其事之有无,只据此新奇妙文悦我等心目,便当浮一大白。
\zhu{浮:罚人饮酒,这里应该是饮酒的意思。
白:酒杯。
}}\par
周瑞家的还欲说话时,忽听王夫人问:\meng{了结得齐整。
}“是谁在里头?”周瑞家的忙出去答应了,趁便回了刘姥姥之事。
略待半刻,见王夫人无话,方欲退出,\jia{行文原只在一二字,便有许多省力处。
不得此窍者,便在窗下百般扭捏。
}薛姨妈忽又笑道:\jia{“忽”字“又”字与“方欲”二字对射。
}“你且站住。
我有一宗东西,你带了去罢。
”说着便叫香菱。
\jia{二字仍从“莲”上起来。
盖“英莲”者,“应怜”也,“香菱”者亦“相怜”之意。
此是改名之“英莲”也。
}帘栊响处,方才和金钏儿顽的那个小女孩子进来了,问:“奶奶叫我作什么?”\jia{这是英莲天生成的口气,妙甚!}薛姨妈道:“把那匣子里的花儿拿来。
”香菱答应了,向那边捧了个小锦匣来。
薛姨妈乃道:“这是宫里头作的新鲜样法堆纱花十二枝。
\zhu{堆:累叠。堆纱:以纱制作人物花鸟的一种工艺。}
昨儿我想起来,白放着可惜旧了,何不给他们姊妹们戴去。
昨儿要送去,偏又忘了。
你今儿来的巧,就带了去罢。
你家的三位姑娘,每人两枝,下剩六枝,送林姑娘两枝,那四枝给了凤哥儿罢。
”\jia{妙文!今古小说中可有如此口吻者?}王夫人道:“留着给宝丫头戴罢了,又想着他们。
”薛姨妈道:“姨妈不知道,宝丫头古怪\jia{“古怪”二字,正是宝卿身份。
}呢,他从来不爱这些花儿粉儿的。
”\jia{可知周瑞一回,正为宝菱二人所有,正《石头记》得力处也。
}\par
说着,周瑞家的拿了匣子走出房门,见金钏儿仍在那里晒日阳。
周瑞家的因问他道:“那香菱小丫头子,可就是时常说临上京时买的、为他打人命官司的那个小丫头子?”\meng{点醒从来。
}金钏道:“可不就是。
”\jia{出明英莲。
}正说着,只见香菱笑嘻嘻的走来。
周瑞家的便拉了他的手,细细的看了一回,因向金钏儿笑道:“倒好个模样儿,竟有些像咱们东府里蓉大奶奶的品格。
”\jia{一击两鸣法,二人之美,并可知矣。
再忽然想到秦可卿,何玄幻之极。
假使说像荣府中所有之人,则死板之至,故远远以可卿之貌为譬,似极扯淡,然却是天下必有之情事。
}金钏儿笑道:“我也是这么说呢。
”周瑞家的又问香菱:“你几岁投身到这里?”又问:“你父母今在何处?今年十几岁了?本处是那里人?”香菱听问,摇头说:“不记得了。
”\jia{伤痛之极,亦必如此收住方妙。
不然,则又将作出香菱思乡一段文字矣。
}
周瑞家的和金钏儿听了,倒反为他叹息伤感一回。
\meng{西施心痛之态,其时自己也还耐得,倒是旁人替伊为多少思虑,不禁无穷痛楚之香菱,其是乎,否乎?}\par
一时周瑞家的携花至王夫人正房后来。
原来近日贾母说孙女们太多了,一处挤着倒不便,只留宝玉、黛玉二人在这边解闷,\ping{孙女太多,真是幸福的烦恼。
只留宝黛二人,可见贾母之偏爱。
}却将迎、探、惜三人移到王夫人这边房后三间小抱厦内居住,
\zhu{抱厦:原建筑之前或之后接建出来的小房子。\foot{\footPic{抱厦(在建筑之前接建)}{baoxia.jpg}{0.6}}}
令李纨陪伴照管。
\jia{不作一笔安逸之笔矣。
}如今周瑞家的故顺路先往这里来,只见几个小丫头子都在抱厦内听呼唤默坐。
迎春的丫头司棋与探春的丫鬟待书\jia{妙名。
贾家四钗之鬟,暗以琴、棋、书、画四字列名,省力之甚,醒目之甚,却是俗中不俗处。
}
\zhu{
“待书”:己、戚、列、杨本同,庚、蒙、辰、舒本则作“侍书”。
“侍”字,查《辞源》,“侍”字的解释有两个义项:①陪从尊长身旁。
②进言,进谏。
按,“侍”字的第一义项完全符合《红楼梦》中“侍书”的身份和地位。
《辞源》甚至还列有一个专门的名词“侍书”:侍书:官名。《后汉书》六十下《蔡邕传》:“举高第,补侍御史,又转侍书御史。”明代翰林院有侍书二人,见《明史·职官志》二,翰林院。
“待”字,查《辞源》,“待”字有六个义项:①等待。②准备。③对待,款待。④宽容。⑤须。⑥将要。
从这六个义项来看,无一可以与“书”字造句合拍者。
元春、迎春、探春、惜春四姐妹的丫鬟所设计的名字分别是“抱琴”“司棋”“侍书”和“入画”,巧妙地嵌入“琴”“棋”“书”“画”四字。
《辞源》对“琴棋书画”的解释:琴棋书画:弹琴、下棋、写字、绘画,皆为旧时文士引为风雅之事,故常四字连称。
“侍书”中的“书”的意思是“写字”,符合探春喜欢书法的特点:
第二十七回:探春又笑道:“……明儿逛去的时候,或是好字画、书籍卷册、轻巧顽意儿,给我带些来”;
第四十回:探春……案上磊着各种名人法帖,并数十方宝砚,各色笔筒,笔海内插的笔如树林一般……左右挂着一副对联,乃是颜鲁公墨迹;
除侍书之外,探春还有一个丫鬟叫作“翠墨”,从侧面反映了探春对“写字”“书法”的钟爱。
}
二人正掀帘出来,手里都捧着茶盘茶钟,周瑞家的便知他姊妹在一处坐着,遂进入内房,只见迎春、探春二人正在窗下围棋。
周瑞家的将花送上,说明原故。
他二人忙住了棋,都欠身道谢,命丫鬟们收了。
周瑞家的答应了,因说:“四姑娘不在房里?只怕在老太太那边呢。
”丫鬟们道:“在这屋里不是?”\jia{用画家三五聚散法写来,方不死板。
}周瑞家的听了,便往这边屋内来。
只见惜春正同水月庵\lie{即馒头庵。
}
的小姑子智能儿,两个一处顽笑,\jia{总是得空便入。
百忙中又带出王夫人喜施舍等事,可知一支笔作千百支用。
}\jia{又伏后文。
}\jia{闲闲一笔,却将后半部线索提动。
}见周瑞家的进来,惜春便问他何事。
周瑞家的便将花匣打开,说明原故。
惜春笑道:“我这里正和智能儿说,我明儿也剃了头同他作姑子去呢,可巧又送了花儿来。
若剃了头,把这花可戴在那里?”\meng{触景生情,透漏身分。
}\ping{伏惜春出家为尼的结局。
}说着,大家取笑一回,惜春命丫鬟入画来收了。
\jia{曰司棋,曰待书,曰入画,后文补抱琴。
}\jia{琴、棋、书、画四字最俗,上添一虚字则觉新雅。
}\par
周瑞家的因问智能儿:“你是什么时候来的?你师傅那秃歪剌往那里去了?”\zhu{
秃:指光头。
歪剌:也作歪辣,又叫歪剌骨,歪剌货,意谓不正当的女人。
秃歪剌:骂尼姑的话。
}\ping{在当时,僧尼地位并不高,故惜春因结局是出家为尼而入薄命司。
}智能儿道:“我们一早就来了,我师傅见过太太,就往于老爷府里去了,叫我在这里等他呢。
”\jia{又虚贴一个于老爷,可知尚僧尼者,悉愚人也。
}周瑞家的又道:“十五的月例香供银子可得了没有?”\zhu{香供:香和供品。
}智能儿摇头儿说:“不知道。
”\jia{妙!年轻未任事也。
一应骗布施、哄斋供诸恶,皆是老秃贼设局。
写一种人,一种人活像。
}惜春听了,便问周瑞家的:“如今各庙月例银子是谁管着?”周瑞家的道:“是余信\jia{明点“愚性”二字。
}
管着。
”\meng{写家奴每相妒毒,人前有意倾陷。
}惜春听了笑道:“这就是了。
他师傅一来了,余信家的就赶上来,和他师傅咕唧了半日,想是就为这事了。
”\jia{一人不落,一事不忽,伏下多少后文,岂真为送花哉!}\par
那周瑞家的又和智能儿唠叨了一回,便往凤姐处来。
穿夹道从李纨后窗下过,\jia{细极!李纨虽无花,岂可失而不写者?故用此顺笔便墨,间三带四,使观者不忽。
}越西花墙,出西角门,进入凤姐院中。
\ping{李纨青春守寡,如同槁木死灰一般,得不到宫花。
}走至堂屋,只见小丫头丰儿坐在凤姐房门槛上,见周瑞家的来了,连忙\jia{二字着紧。
}摆手儿,叫他往东屋里去。
周瑞家的会意,慌的蹑手蹑脚的往东边房里来,只见奶子正拍着大姐儿睡觉呢。
\jia{总不重犯,写一次有一次的新样文法。
}周瑞家的悄问奶子道:“奶奶睡中觉呢?也该请醒了。
”奶子摇头儿。
\jia{有神理。
}正问着,只听那边一阵笑声,却有贾琏的声音。
接着房门响处,平儿拿着大铜盆出来,叫丰儿舀水进去。
\jia{妙文奇想!阿凤之为人,岂有不着意于“风月”二字之理哉?若直以明笔写之,不但唐突阿凤声价,亦且无妙文可赏。
若不写之,又万万不可。
故只用“柳藏鹦鹉语方知”之法,略一皴染,不独文字有隐微,亦且不至污渎阿凤之英风俊骨。
所谓此书无一不妙。
}\jia{余素所藏仇十洲《幽窗听莺暗春图》,其心思笔墨,已是无双,今见此阿凤一传,则觉画工太板。
}\ping{庚辰本此回回目为“送宫花贾琏戏熙凤\quad 宴宁府宝玉会秦钟”,“贾琏戏熙凤”即是指此处情节。
}平儿便进这边来,一见了周瑞家的便问:“你老人家又跑了来作什么?”周瑞家的忙起身,拿匣子与他,说送花一事。
平儿听了,便打开匣子,拿出四枝,转身去了。
半刻工夫,手里又拿出两枝来,\jia{攒花簇锦文字,故使人耳目眩乱。
}先叫彩明来,吩咐他“送到那边府里,给小蓉大奶奶戴去。
”\jia{忙中更忙,又曰“密处不容针”,此等处是也。
}\ping{王熙凤与秦可卿交好,为王熙凤探视卧病秦可卿和秦可卿死前托梦王熙凤做铺垫。
}次后方命周瑞家的回去道谢。
\par
周瑞家的这才往贾母这边来。
穿过了穿堂,顶头忽见他女儿打扮着才从他婆家来。
周瑞家的忙问:“你这会子跑来作什么?”他女儿笑道:“妈一向身上好?我在家里等了这半日,妈竟不出去,什么事情这样忙的不回家?我等烦了,自己先到了老太太跟前请了安了,这会子请太太安去。
妈还有什么不了的差事?手里是什么东西?”周瑞家的笑道:“嗳!今儿偏偏的来了个刘姥姥,我自己多事,为他跑了半日,这会子又被姨太太看见了,送这几枝花儿与姑娘奶奶们。
\ping{周瑞家的并看不上刘姥姥,周旋忙活的原因是卖弄夸耀自己在豪门的地位。
}这会子还没送清白呢。
你这会子跑来,一定有什么事情的。
”他女儿笑道:“你老人家倒会猜。
实对你老人家说,你女婿前儿因多吃了两杯酒,和人纷争起来,不知怎的被人放了一把邪火,\zhu{放了一把邪火:即造谣中伤。
}说他来历不明,告到衙门里,要递解还乡。
\zhu{递解:旧时把解往别处的犯人由所经各地派人一站转一站地押送叫“递解”。
}所以我来和你老人家商议商议,这个情分,求那一个可了事?”周瑞家的听了道:“我就知道的。
这有什么大不了的!你且家去等我,我送林姑娘的花儿去了就回家来。
此时太太、二奶奶都不得闲儿,你回去等我。
这没有什么忙的。
”他女儿听如此说,便回去了。
还说:“妈,你好歹快来!”周瑞家的道:“是了。
小人家没经过什么事情,就急的你这样子。
”\ping{仗势豪奴,情态逼人。
}
说着。
便到黛玉房中去了。
\jia{又生出一小段来,是荣、宁中常事,亦是阿凤正文,若不如此穿插,直用一送花到底,亦太死板,不是《石头记》笔墨矣。
}\par
谁知此时黛玉不在自己房中,却在宝玉房中大家解九连环作戏。
\zhu{九连环:一种智力玩具,用金属丝制成一狭长的方圈,上套九个圆环,可解下套上,手续极繁,玩时以能全部解下圆环者为胜。
}
\jia{妙极!又一花样。
此时二玉已隔房矣。
}周瑞家的进来笑道:“林姑娘,姨太太着我送花来与姑娘戴。
”宝玉听说,便先说:“什么花?拿来给我。
”一面早伸手接过来了。
\jia{瞧他夹写宝玉。
}开匣看时,原来是两枝宫制堆纱新巧的假花。
\jia{此处方一细写花形。
}黛玉只就宝玉手中看了一看,\jia{妙!看他写黛玉。
}便问道:“还是单送我一个人的,还是别的姑娘们都有?”\jia{在黛玉心中,不知有何丘壑。
}周瑞家的道:“各位都有了,这两枝是姑娘的了。
”\ping{是真不会说话,还是故意呢?}黛玉再看了一看,\jia{“再看一看”,传神。
}冷笑道:“我就知道,别人不挑剩下的也不给我。
替我道谢罢!”\jia{吾实不知黛卿胸中有何丘壑。
}周瑞家的听了,一声儿不言语。
\jia{余\sout{问}[阅]送花一回,薛姨妈云“宝丫头不喜这些花儿粉儿的”,则谓是宝钗正传;又\sout{主}[至]阿凤\sout{惜}[嬉]
春一段,则又知是阿凤正传;今又到颦儿一段,却又将阿颦之天性从骨中一写,方知亦系颦儿正传。
小说中一笔作两三笔者有之,一事启两事者有之,未有如此恒河沙数之笔也。
\zhu{恒河沙数:比喻数量多得像恒河里的沙子一样无法计数。}
}宝玉便问道:“周姐姐,你作什么到那边去了。
”\ping{贾母和王夫人对待黛玉是两种态度:“如今且说林黛玉自在荣府以来,贾母万般怜爱,寝食起居,一如宝玉,迎春、探春、惜春三个亲孙女倒且靠后。
”,而王夫人的态度可以从陪房周瑞家的看出来,不仅最后才送花给黛玉,而且在黛玉发小脾气的时候,“一声儿不言语”,并不去哄劝宽慰,还需要宝玉及时岔开话题。
黛玉客居外祖母家,敏感自尊,惟恐他人轻视,此时年纪尚小,尚不善于掩饰自己的情绪,小孩子发小脾气也是无可厚非的,而仆人周瑞家的在惹主子生气后,对黛玉态度倨傲,撑腰的很可能是也对黛玉有意见的王夫人。
}周瑞家的因说:“太太在那里,因回话去了,姨太太就顺便叫我带来了。
”宝玉道:“宝姐姐在家作什么呢?怎么这几日也不过来?”周瑞家的道:“身上不大好呢。
”宝玉听了,便和丫头们说:“谁去瞧瞧?就说我和林姑娘\jia{“和林姑娘”四字着眼。
}打发来问姨娘、姐姐安,问姐姐是什么病,吃什么药。
论理我该亲自来的,就说才从学里来的,也着了些凉,异日再亲来。
”\jia{余观“才从学里来”几句,忽追思昔日情景,可叹!想纨绔小儿,自开口云“学里”,亦如市俗人开口便云“有些小事”,然何尝真有事哉!此掩饰推托之词耳。
宝玉若不云“从学房里来凉着”,然则便云“因憨顽时凉着”者哉?写来一笑,继之一叹。
}说着,茜雪便答应去了。
周瑞家的自去,无话。
\par
原来这周瑞的女婿,便是雨村的好友冷子兴,\jia{着眼。
}近因卖古董和人打官司,故遣女人来讨情分。
周瑞家的仗着主子的势利,把这些事也不放在心上,晚间只求求凤姐儿便完了。
\ping{“便完了”三字写尽轻蔑不屑。
}\par
至掌灯时分,凤姐已卸了妆,来见王夫人回话:“今儿甄家\jia{又提甄家。
}送了来的东西,我已收了。
\jia{不必细说方妙。
}咱们送他的,趁着他家有年下进鲜的船回去,\zhu{进鲜:封建时代官僚贵族向皇帝进献水果鱼虾等时鲜物品。
}一并都交给他们带去了。
”王夫人点头。
凤姐又道:“临安伯老太太千秋的礼已经打点了,
\zhu{千秋:尊称别人的生日。有恭维的意味。}
太太派谁送去?”\jia{阿凤一生尖处。
}王夫人道:“你瞧谁闲着,不管打发那两个女人去就完了,又来当什么正经事问我。
”\jia{虚描二事,真真千头万绪,纸上虽一回两回中或有不能写到阿凤之事,然亦有阿凤在彼处手忙心忙矣,观此回可知。
}\meng{各自各自心计,在问答之间渺茫欲露。
}\ping{在大事上,王熙凤自己做主,如“弄权铁槛寺”;反而在小事上,请示王夫人,以显示出自己的尊重服从,令自己的姑姑放心把管家的权力交给自己。
}凤姐又笑道:“今儿珍大嫂子来,请我明儿过去逛逛,明儿倒没有什么事。
”王夫人道:“有事没事都害不着什么。
每常他来请,有我们,你自然不便意,他既不请我们,单请你,可知是他诚心叫你散淡散淡,
\zhu{散淡:也作散旦、散诞、散荡。舒散、优闲的意思。}
别辜负了他的心,便是有事,也该过去才是。
”\meng{用人力者当有此段心想。
}凤姐答应了。
当下李纨、迎春等姐妹们亦曾定省毕,\zhu{定省:定,侍候就寝;省,探望问候。
定省指子女早晚向父母请安问好的礼节。
}各自归房无话。
\par
次日,凤姐儿梳洗了,先回王夫人毕,方来辞贾母。
宝玉听了,也要逛去。
凤姐只得答应着,立等换了衣服,姐儿两个坐了车,一时进入宁府。
早有贾珍之妻尤氏与贾蓉之妻秦氏,婆媳两个引了多少姬妾丫鬟媳妇等接出仪门。
那尤氏一见了凤姐,必先笑嘲一阵,一手携了宝玉,入上房来归坐。
秦氏献茶毕,凤姐因说:“你们请我来作什么?有什么东西来孝敬就献上来,我还有事呢。
”\meng{口头心头,惟恐人不知。
}尤氏秦氏未及答话,地下几个姬妾先就笑说:“二奶奶今儿不来就罢,既来了,就依不得二奶奶了。
”\meng{非把世态熟于胸中者,不能有如此妙文。
}正说着,只见贾蓉进来请安。
宝玉因问:“大哥哥今日不在家?”尤氏道:“出城请老爷安去了。
”又道:“可是你怪闷的,也坐在这里作什么?何不去逛逛?”\par
秦氏笑道:“今日巧,上回宝叔立刻要见见我兄弟,他今儿也在这里,\jia{欲出鲸卿,
\zhu{鲸卿:秦钟的字。}
却先小妯娌闲闲一聚,
\zhu{妯娌:音“轴里”,兄弟之妻相互的称呼。}
随笔带出,不见一丝造作。
}想在书房里,宝叔何不去瞧一瞧?”宝玉听了,即便下炕要走。
尤氏、凤姐都忙说:“好生着,忙什么?”一面便吩咐人,“好生小心跟着,别委屈着他,倒比不得跟了老太太来,就罢了。
”\jia{“委屈”二字极不通,却是至情,写愚妇至矣!}凤姐儿道:“既这么着,何不请进这秦小爷来,我也瞧瞧。
难道我就见不得他不成?”尤氏笑道:“罢,罢!可以不必见他,比不得咱们家的孩子们,胡打海摔的惯了。
\jia{卿家“胡打海摔”,不知谁家方珍怜珠惜?此极相矛盾却极入情,盖大家妇人口吻如此。
}\meng{偏会反衬,方显尊重。
}人家的孩子都是斯斯文文惯了的,乍见了你这破落户,还被人笑话死了呢。
”凤姐笑道:“普天下的人,我不笑话就罢,\meng{自负得起。
}
竟叫这小孩子笑话我不成?”贾蓉笑道:“不是这话,他生的腼腆,没见过大阵仗儿,婶子见了,没的生气。
”凤姐啐道:“他是哪吒,我也要见一见!别放你娘的屁了。
再不带去,看给你一顿好嘴巴子。
”\jia{此等处写阿凤之放纵,是为后回伏线。
}贾蓉笑嘻嘻的说:“我不敢强,
\zhu{强:音“酱“,固执、不柔顺。}
就带他来。
”\par
说着,果然出去带进一个小后生来,较宝玉略瘦巧些,清眉秀目,粉面朱唇,身材俊俏,举止风流,似在宝玉之上,只是怯怯羞羞,有女儿之态,腼腆含糊的向凤姐作揖问好。
凤姐喜的先推宝玉,笑道:“比下去了!”\jia{不知从何处想来。
}便探身一把携了这孩子的手,就命他身旁坐下,慢慢问他年纪、读书等事,\jia{分明写宝玉,却先偏写阿凤。
}方知他学名唤秦钟。
\jia{设云“情种”。
古诗云:“未嫁先名玉,来时本姓秦。
”二语便是此书大纲目、大比托、大讽刺处。
\ping{
“未嫁先名玉,来时本姓秦”出自于《玉台新咏》南朝梁刘缓的《敬酬刘长史咏名士悦倾城》
全诗如下:不信巫山女,不信洛川神。何关别有物,还是倾城人。经共陈王戏,曾与宋家邻。未嫁先名玉,来时本姓秦。粉光犹似面,朱色不胜唇。遥见疑花发,闻香知异春。钗长逐鬟发,袜小称腰身。夜夜言娇尽,日日态还新。工倾荀奉倩,能迷石季伦。上客徒留目,不见正横陈。
这首诗铺陈了多个关于美人的典故,批语中引用的两句,分别指的是吴王夫差的小女儿小玉和《陌上桑》中的秦罗敷。
批语说这两句诗是“大纲目、大比托、大讽刺”,可能是因为“秦”谐音“情”,“玉”指“宝玉”。
全书第一回开宗明义用“大旨谈情”概括了全书围绕宝玉“情痴抱恨长”展开的故事。
注:脂砚斋重评石头记凡例中有诗“谩言红袖啼痕重,更有情痴抱恨长”,前一句对应的是黛玉,后一句对应的是宝玉。
}}早有凤姐的丫鬟媳妇们见凤姐初会秦钟,并未备得表礼来,\zhu{表礼:旧日赠送或赏赐的礼物。
}遂忙过那边去告诉平儿。
平儿素知凤姐与秦氏厚密,虽是小后生家,亦不可太俭,遂自作了主意,拿了一匹尺头,\zhu{尺头:衣料。
}两个“状元及第”的小金锞子,\zhu{小金锞(锞音“课”)子:金子铸成的小锭。
状元及第:这里指金锭上的一种“吉祥图案”,作考中的状元戴金花骑马的形状(也有只用“状元及第”四字的)。
此外的“吉祥图案”还有“岁岁平安”、“事事如意”、“云龙捧寿”、“玉堂富贵”等等。
见本书后文。
}交付与来人送过去。
\ping{如此周全妥帖,礼数尽至。
}凤姐犹笑说“太简薄”等语。
秦氏等谢毕。
一时吃过饭,尤氏、凤姐、秦氏等抹骨牌,\zhu{抹骨牌:即打骨牌。
骨牌:又名“牙牌”或“牌九”。
一种用兽骨或竹、木、象牙等制的娱乐品,也用作赌具。
牌作长方形,一面雕圆点,其数多寡不一,又分红绿二色,象天地及星辰布列之形。
后世之打麻将由此演变而来。
}不在话下。
\jia{一人不落,又带出“强将手下无弱兵”。
}\par
宝玉、秦钟二人随便起坐说话。
\jia{淡淡写来。
}那宝玉只一见秦钟人品,心中便有所失,痴了半日,自己心中又起了呆意,乃自思道:“天下竟有这等人物!如今看来,我竟成了泥猪癞狗了。
可恨我为什么生在这侯门公府之家,若也生在寒儒薄宦之家,早得与他交结,也不枉生了一世。
我虽如此比他尊贵,\jia{这一句不是宝玉本意中语,却是古今历来膏粱纨绔之意。
}可知绫锦纱罗,也不过裹了我这根死木;美酒羊羔,也只不过填了我这粪窟泥沟。
‘富贵’二字,不料遭我涂毒了!”\jia{一段痴情,翻“贤贤易色”一句筋斗,
使此后朋友中无复再敢假谈道义,虚论情常。
\zhu{贤贤易色:出自《论语·学而》。
第一个贤是动词,意思是崇尚贤者。易:换取,代替。谓以崇尚贤者之心来替代好色之心。
翻……筋斗:指做翻案文章。
这条批语的意思是,宝玉以貌取人,与秦钟结为好友,和“贤贤易色”所倡导的重视贤德,不看重姿色正好相反。
本回结尾有“不因俊俏难为友,正为风流始读书”的诗句可以作证。
}
}
\zhu{涂毒:即“荼毒”,毁坏。}
\meng{此是作者一大发泄处。
}秦钟自见了宝玉形容出众,举止不浮,\jia{“不浮”二字妙,秦卿目中所取正在此。
}更兼金冠绣服,骄婢侈童,\jia{这二句是贬,不是奖。
此八字遮饰过多少魑魅纨绔,秦卿目中所鄙者。
}秦钟心中亦自思道:“果然这宝玉怨不得人人溺爱他。
可恨我偏生于清寒之家,不能与他耳鬓交结,可知‘贫富’二字限人,亦世间之大不快事。
”\jia{“贫富”二字中,失却多少英雄朋友!}\meng{总是作者大发泄处,借此以伸多少不乐。
}二人一样的胡思乱想。
\jia{作者又欲瞒过众人。
}忽又\jia{二字写小儿得神。
}有宝玉问他读什么书。
\jia{宝玉问读书,亦想不到之大奇事。
}秦钟见问,便因实而答。
\jia{四字普天下朋友来看。
}二人你言我语,十来句后,越觉亲密起来。
\par
一时摆上茶果吃茶,宝玉便说:“我两个又不吃酒,把果子摆在里间小炕上,我们那里坐去,省得闹你们。
”\jia{眼见得二人一身一体矣。
}于是二人进里间来吃茶。
秦氏一面张罗与凤姐摆酒果,一面忙进来嘱宝玉道:“宝叔,你侄儿年小,倘或言语不防头,\zhu{不防头:冒失,不留神、不经意。
}你千万看着我,不要理他。
他虽腼腆,却性子左强,\zhu{
“强”(音“匠”)即“犟”。
左强:执拗倔强。
}不大随和些是有的。
”\jia{实写秦钟,双映宝玉。
}
\meng{伏后文。
}宝玉笑道:“你去罢,我知道了。
”秦氏又嘱了他兄弟一回,方去陪凤姐。
\par
一时凤姐尤氏又打发人来问宝玉:“要吃什么,外面有,只管要去。
”宝玉只答应着,也无心在饮食,只问秦钟近日家务等事。
\jia{宝玉问读书已奇,今又问家务,岂不更奇?}秦钟因说:“业师于去年病故,\zhu{业师:旧时称给本人授业的老师。
}家父又年纪老迈,贱疾在身,公务繁冗,因此尚未议及再延师一事,目下不过在家温习旧课而已。
再读书一事,也必须有一二知己\jia{眼。
}为伴,\meng{伏线。
}时常大家讨论,才能进益。
”\jia{真是可儿之弟。
}宝玉不待说完,便答道:“正是呢,我们家却有个家塾,合族中有不能延师的,便可入塾读书,子弟们中亦有亲戚在内,可以附读。
我因上年业师回家去了,也现荒废着。
家父之意,亦欲暂送我去,且温习着旧书,待明年业师上来,再各自在家亦可。
家祖母因说:一则家学里子弟太多,生恐大家淘气,反不好,二则也因我病了几天,遂暂且耽搁着。
如此说来,尊翁如今也为此事悬心。
今日回去何不禀明,就在我们这敝塾中来,我亦相伴,彼此有益,岂不是好事?”秦钟笑道:“家父前日在家提及延师一事,也曾提起这里的义学倒好,\zhu{义学:也叫“义塾”。
中国封建时代的一种免费学校。
有宗族办的,也有私人集资或用地方公费办的。
一般招收主办者的族人、亲友或乡里子弟。
}原要来和这里的亲翁商议引荐。
因这里事忙,不便为这点小事来聒絮的。
\zhu{聒絮:唠叨。
}宝叔果然度小侄或可磨墨涤砚,何不速速作成,\jia{真是可卿之弟。
}又彼此不致荒废,又可以常相谈聚,又可以慰父母之心,又可以得朋友之乐,岂不是美事?”\meng{痛快淋漓以至于此。
}宝玉笑道:“放心,放心。
咱们回来先告诉你姐夫、姐姐和琏二嫂子。
你今日回家就禀明令尊,我回去再回明家祖母,再无不速成之理的。
”二人计议一定。
那天气已是掌灯时候,出来又看他们顽了一回牌。
算帐时,却又是秦氏、尤氏二人输了戏酒的东道,\jia{自然是二人输。
}
\zhu{东道:请客的事或义务。}
言定后日吃这东道,一面又说了回话。
\par
晚饭毕,因天黑了,尤氏因说:“先派两个小子送了这秦相公去。
”媳妇们传出去半日,秦钟告辞起身。
尤氏问:“派了谁送去?”媳妇们回说:“外头派了焦大,谁知焦大醉了,又骂呢。
”\jia{可见骂非一次矣。
}
\meng{恶恶而不能去,善善而不能用,所以流毒无穷,可胜叹哉。
}尤氏、秦氏都说道:“偏又派他作什么!放着这些小子们,那一个派不得?偏要惹他去。
”\jia{便奇。
}凤姐道:“我成日家说你太软弱了,\zhu{家:一作“价”,语尾助词,无义。
成日家:一天到晚,终日里。
}纵的家里人这样,还了得呢!”尤氏叹道:“你难道不知这焦大的?连太爷都不理他的,你珍哥哥也不理他。
只因他从小儿跟着太爷们出过三四回兵,从死人堆里把太爷背了出来,得了命,自己挨着饿,却偷了东西来给主子吃。
两日没得水,得了半碗水,给主子吃,他自喝马溺。
不过仗着这些功劳情分,有祖宗时都另眼相待,如今谁肯难为他去?他自己又老了,又不顾体面,一味的噇酒,\zhu{噇:音“床”,无节制地狂吃狂喝。
}一吃醉了,无人不骂。
我常说给管事的,不要派他事,全当一个死的就完了。
今儿又派了他!”\meng{有此功劳,实不可轻易摧折,亦当处之以道,厚其赡养,尊其等次。
送人回家,原非酬功之事。
所谓汉之功臣不得保其首领者,
\zhu{首领:头与脖子。引申为性命。}
我知之矣。
}凤姐道:“我何曾不知这焦大。
倒是你们没主意,有这样,何不打发他远远的庄子上去就完了。
”\jia{这是为后协理宁国[府]伏线。
}说着,因问:“我们的车可齐备了?”地下众人都应:“伺候齐了。
”\par
凤姐亦起身告辞,和宝玉携手同行。
尤氏等送至大厅,只见灯烛辉煌,众小厮都在丹墀侍立。
\zhu{丹:红色。
墀:音“迟”,台阶;也称阶面。
丹墀:古代宫殿台阶上的地面涂成红色,叫“丹墀”。
这里泛指台阶。
}那焦大又恃贾珍不在家——即在家亦不好怎样——更可以恣意的洒落洒落。
\zhu{洒落洒落:数说人家的不是。
}因趁着酒兴,先骂大总管赖二,\jia{记清,荣府中则是赖大,又故意综错的妙。
}说他“不公道,欺软怕硬,有了好差事就派别人,像这样黑更半夜送人的事就派我。
没良心的忘八羔子!瞎充管家!你也不想想,焦大太爷跷起一只脚,比你的头还高呢。
二十年头里的焦大太爷眼里有谁?别说你们这把子的杂种忘八羔子们!”\par
正骂的兴头上,贾蓉送凤姐的车出去,众人喝他不听,贾蓉忍不得,便骂了他两句:“使人捆起来!等明日酒醒了,问他还寻死不寻死了!”\meng{可怜天下每每如此。
}那焦大那里把贾蓉放在眼里,反大叫起来,赶着贾蓉叫:“蓉哥儿,你别在焦大跟前使主子性儿。
别说你这样儿的,就是你爹、你爷爷,也不敢和焦大挺腰子呢!\zhu{挺腰子:犹言硬抗、耍威风。
}不是焦大一个人,你们做官儿,享荣华,受富贵?你祖宗九死一生挣下这个家业,到如今不报我的恩,反和我充起主子来了。
\jia{忽接此焦大一段,真可惊心骇目,一字化一泪,一泪化一血珠。
}不和我说别的还可,若再说别的,咱们‘白刀子进去红刀子出来’!”\jia{是醉人口中文法。
\zhu{白刀子进去红刀子出来:己、庚、杨本作“红刀子进去白刀子出来”,
因批语“是醉人口中文法”,“红刀子进去白刀子出来”才是醉人颠倒口吻。但是焦大虽醉,其大篇骂人话文通理顺,不当独此句颠倒。
按:此处情节与《金瓶梅词话》第二十五回,来旺儿醉酒恨骂西门庆一段相似,连引用俗语也一字不差:
一日,来旺儿吃醉了,和一般家人小厮在前边恨骂西门庆,说怎的我不在家,耍了我老婆,使玉筲丫头,拿一疋蓝段子,别房里啜他,把他吊在花园里奸耍;后来怎的停眠整宿,潘金莲怎做窝主。「由他,只休要撞到我手里;我教他白刀子进去,红刀子出来。好不好把潘家那淫妇也杀了,我也只是个死;你看我说出来,做的出来!……」
考虑到批语中多次将本书与《金》书对举,此批或可理解为,来旺儿醉骂主子说这句俗语,焦大醉骂主子也说这句俗语,这是醉人通用的说话方法。
}
}\jia{一段借醉奴口角闲闲补出宁荣往事近故,特为天下世家一\sout{笑}[哭]。
}凤姐在车上说与贾蓉:“以后还不早打发了这没王法的东西!留在这里岂不是祸害?倘或亲友知道了,岂不笑话咱们这样的人家,连个王法规矩都没有。
”贾蓉答应“是”。
\par
众小厮见他太撒野不堪了,只得上来几个,揪翻捆倒,拖往马圈里去。
焦大益发连贾珍都说出来,乱嚷乱叫:“我要往祠堂里哭太爷去。
那里承望到如今生下这些畜牲来!每日家偷狗戏鸡,\zhu{家:一作“价”,语尾助词,无义。
}爬灰的爬灰,\zhu{爬灰:公公与儿媳妇私通。
}\ping{暗指秦可卿和贾珍。
}养小叔子的养小叔子,\ping{焦大身在宁国府不熟悉荣国府,又因为当时是夜里,他还吃醉了,很可能误以为“和宝玉携手同行”的王熙凤和她的小叔子贾宝玉有私情。
另一种可能是暗指王熙凤和侄儿贾蓉关系暧昧,但是他俩人之间的关系是婶子和婆家侄儿的关系,并不是嫂子和小叔子的关系。
}我什么不知道?咱们‘胳膊折了往袖子里藏’!”\zhu{胳膊折了往袖子里藏:俗语。
意思是家丑不可外扬,不好的事不要张扬出去。
}
\jia{“不如意事常八九,可与人言无二三。
”以二句批是段,聊慰石兄。
}
\meng{放笔痛骂一回,富贵之家,每罹此祸。
}众小厮听他说出这些没天日的话来,唬的魂飞魄散,也不顾别的了,便把他捆起来,用土和马粪满满的填了他一嘴。
\par
凤姐和贾蓉等也遥遥的闻得,便都装作听不见。
\ping{这里可能在暗示凤姐和贾蓉的暧昧私情。}
宝玉在车上见这般醉闹,倒也有趣,因问凤姐儿道:“姐姐,你听他说‘爬灰的爬灰’,什么是‘爬灰’?”\meng{暗伏后来史湘云之问。
\zhu{第五十七回:忽见湘云走来,手里拿着一张当票,口内笑道:“这是个帐篇子?”……湘云道:“什么是当票子?”众人都笑道:“真真是个呆子,连个当票子也不知道。
”}}\ping{孩子的天真之问。
}凤姐听了,连忙立眉嗔目断喝道:“少胡说!那是醉汉嘴里混唚。
\zhu{混唚(音“沁”):骂人的话。
牲畜呕吐叫“唚”。
混唚,把别人说话比作牲畜呕吐,较骂人“胡说”更甚。
}你是什么样的人,不说不听见,还倒细问!等我回去回了太太,仔细捶你不捶你!”\meng{熙凤能事。
}唬的宝玉连忙央告:“好姐姐,我再不敢说这话了。
”凤姐亦忙回色哄道:“好兄弟,这才是。
等回去咱们回了老太太,打发人往家学里说明白了,请了秦钟家学里念书去要紧。
”说着,自回荣府而来。
要知端的,且听下回分解。
\zhu{分解:说明,交代。
旧时章回小说每回结尾时常作为习用之语。
}正是:\par
不因俊俏难为友,正为风流始读书。
\jia{原来不读书即蠢物矣。
}\par
\qi{总评:焦大之醉,伏可卿之病至死。
周妇之谈,势利之害真凶。
作者具菩提心,于世人说法。
}
\dai{013}{周瑞家的给黛玉送宫花}
\dai{014}{焦大醉骂爬灰被捆}
\sun{p7-1}{薛宝钗谈冷香丸,送宫花周瑞叹英莲}{图上侧中部:送走刘姥姥,周瑞家的便来梨香院寻王夫人回话,见王夫人正和薛姨妈说家务话,遂进里间问候宝钗,谈及冷香丸的做法。
图右上:待到王夫人问:“谁在里头?”周瑞家的忙出来回了刘姥姥之事。
方欲退出,薛姨妈令香菱捧出小锦匣来,薛姨妈道“这里有十二支宫里做的新鲜堆纱花,给家里的姐妹们带去。
”图右下:周瑞家的出门时,碰到金钏儿和香菱,听金钏说,香菱就是那个被拐卖的英莲,周瑞家的拉着香菱的手仔细端详,夸她模样长得俊俏。
}
\sun{p7-2}{周瑞家的送各姊妹宫花}{图右下:周瑞家的先给迎春、探春送了花,只见迎春、探春二人正在窗下围棋。
图左侧:惜春在同水月庵的姑子智能儿一处玩耍,周瑞家的将花匣打开,说明原故,惜春笑道:“我正在和智能儿说做姑子去呢,要剃了头,可把花儿戴在哪儿呢?”图上侧中部:接着周瑞家的来到凤姐院中,小丫头连忙向她摆手,只听房里贾琏嬉笑声。
图右上:待到了黛玉处,周瑞家的将花儿递上,黛玉问“是单送我一人的,还是别的姑娘们都有呢?”周瑞家的道:“各位都有了,这两枝是姑娘的。
”黛玉冷笑道:“我就知道,别人不挑剩下的也不给我。
”宝玉连忙将话岔开。
}
\sun{p7-3}{宴宁府宝玉会秦钟}{图右侧:凤姐带着宝玉来到宁府,与贾珍之妻尤氏,贾蓉媳妇秦氏聚会。
说笑间秦氏道:“上回宝叔要见我兄弟,今儿他正在书房里呢!”贾蓉带了秦钟过来,只见秦钟眉清目秀,粉面朱唇,举止风流,怯怯羞羞。
图左侧:二人进里间去吃茶。
宝玉邀秦钟到家塾来与自己结伴读书,秦钟当下答应。
}
\sun{p7-4}{凤姐宝玉坐车闻焦大醉骂}{图上侧:老佣焦大趁贾珍不在家,又在耍酒疯。
众人见他撒野,只得上来几个,掀翻捆倒。
焦大大骂:“你们这些畜生,每日家偷狗戏鸡,爬灰的爬灰,养小叔子的养小叔子,我什么不知道!”图右侧:凤姐宝玉起身告辞,尤氏等送到大厅前,凤姐和贾蓉听见了醉骂,都装作没听见。
宝玉听到了,问凤姐什么是“爬灰”。
}