\chapter{享福人福深还祷福\quad 痴情女情重愈斟情}
\geng{清虚观,贾母、凤姐原意大适意大快乐,偏写出多少不适意事来,此亦天然至情至理必有之事。
\hang
二玉心事,此回大书,是难了割,却用太君一言以定,是道悉通部书之大旨。
}\par
话说宝玉正自发怔,不想黛玉将手帕子甩了来,正碰在眼睛上,倒唬了一跳,问是谁。
林黛玉摇着头儿笑道:“不敢,是我失了手。
因为宝姐姐要看呆雁,我比给他看,不想失了手。
”宝玉揉着眼睛,待要说什么,又不好说的。
\par
一时,凤姐儿来了,因说起初一日在清虚观打醮的事来,
\zhu{打醮(醮音“叫”):旧时请僧道设坛念经,祈福消灾,超度亡魂的一种宗教仪式。}
遂约着宝钗、宝玉、黛玉等看戏去。
宝钗笑道:“罢,罢,怪热的。
什么没看过的戏,我就不去了。
”凤姐儿道:“他们那里凉快,两边又有楼。
咱们要去,我头几天打发人去,把那些道士都赶出去,把楼打扫干净,挂起帘子来,一个闲人不许放进庙去,才是好呢。
我已经回了太太了,你们不去我去。
这些日子也闷的很了。
家里唱动戏,我又不得舒舒服服的看。
”\ping{从下一段可见,不得舒服的原因是需要到贾母旁边“立规矩”。
}\par
贾母听说,笑道:“既这么着,我同你去。
”凤姐听说,笑道:“老祖宗也去,敢情好了!就只是我又不得受用了。
”贾母道:“到明儿,我在正面楼上,你在旁边楼上,你也不用到我这边来立规矩,可好不好?”凤姐儿笑道:“这就是老祖宗疼我了。
”贾母因又向宝钗道:“你也去,连你母亲也去。
长天老日的,在家里也是睡觉。
”宝钗只得答应着。
\par
贾母又打发人去请了薛姨妈,顺路告诉王夫人,要带了他们姊妹去。
王夫人因一则身上不好,二则预备着元春有人出来,早已回了不去的;听贾母如今这样说,笑道:“还是这么高兴。
”\ping{王夫人这句话弦外之音是贾母不应该这么高兴。
}因打发人去到园里告诉:“有要逛的,只管初一跟了老太太逛去。
”这个话一传开了,别人都还可以,只是那些丫头们天天不得出门槛子,听了这话,谁不要去。
便是各人的主子懒怠去,他也百般撺掇了去,因此李宫裁等都说去。
贾母越发心中喜欢,早已吩咐人去打扫安置,都不必细说。
\par
单表到了初一这一日,荣国府门前车辆纷纷,人马簇簇。
那底下凡执事人等,闻得是贵妃作好事,贾母亲去拈香,正是初一日乃月之首日,况是端阳节间,
\zhu{端阳节:端午节。}
因此凡动用的什物,一色都是齐全的,不同往日。
少时,贾母等出来。
贾母坐一乘八人大轿,李氏、凤姐儿、薛姨妈每人一乘四人轿,宝钗、黛玉二人共坐一辆翠盖珠缨八宝车,
\zhu{八宝车:镶饰华丽的车。}
迎春、探春、惜春三人共坐一辆朱轮华盖车。
\zhu{华盖:古代帝王将相车上的伞盖;借指帝王将相所乘的车。}
然后贾母的丫头鸳鸯、鹦鹉、琥珀、珍珠,林黛玉的丫头紫鹃、雪雁、春纤,宝钗的丫头莺儿、文杏,迎春的丫头司棋、绣橘,探春的丫头待书、翠墨,惜春的丫头入画、彩屏,薛姨妈的丫头同喜、同贵,外带着香菱,香菱的丫头臻儿,李氏的丫头素云、碧月,凤姐儿的丫头平儿、丰儿、小红,并王夫人两个丫头也要跟了凤姐儿去的金钏、彩云,奶子抱着大姐儿带着巧姐儿另在一车,\ping{大姐儿、巧姐儿应该是同一个人,即王熙凤的女儿,出现两个名字的原因是第四十二回,刘姥姥给大姐儿取名巧姐儿,这里同时出现两个名字,应该是书稿尚未修改完成的证据。
}还有两个丫头,一共又连上各房的老嬷嬷奶娘并跟出门的家人媳妇子,
\zhu{家人:仆役。}
乌压压的占了一街的车。
\ping{唯独王夫人不去。
}贾母等已经坐轿去了多远,这门前尚未坐完。
这个说“我不同你在一处”,那个说“你压了我们奶奶的包袱”,那边车上又说“蹭了我的花儿”,这边又说“碰折了我的扇子”,咭咭呱呱,
\zhu{咭咭呱呱[jījīguāguā]:又说又笑的声音。}
说笑不绝。
周瑞家的走来过去的说道:“姑娘们,这是街上,看人笑话。
”说了两遍,方觉好了。
前头的全副执事摆开,早已到了清虚观了。
\ping{车队的前头已经到了目的地,而后头还没出发。
}宝玉骑着马,在贾母轿前。
街上人都站在两边。
\par
将至观前,只听钟鸣鼓响,早有张法官执香披衣,\zhu{法官:这里是对有职位的道士的尊称。
}带领众道士在路旁迎接。
贾母的轿刚至山门以内,\zhu{山门:佛寺的外门,后亦泛称佛寺的二道门为“山门”。
一说:“山门”应是“三门”,指佛寺外面的三座门,象征“三解脱门”,“寺宇开三门者,谓空门、无相门、无作门,故谓三门。
”}贾母在轿内因看见有守门大帅并千里眼、顺风耳、当方土地、
\zhu{土地:迷信指管一个小地区的土神。也说土地爷。}
本境城隍各位泥胎圣像,
\zhu{城隍[huáng]:传说中指守护城池的神。道教尊奉为护国保邦的神。}
便命住轿。
贾珍带领各子弟上来迎接。
凤姐儿知道鸳鸯等在后面,赶不上来搀贾母,自己下了轿,忙要上来搀。
可巧有个十二三岁的小道士儿,拿着剪筒,\zhu{
蜡花:旧时以蜡烛照明,蜡烛点燃时烛芯灰烬结成的花状物,亦称“蜡花”,影响照明的亮度,须用剪子剪除。
剪筒:专门用来收纳剪下来的蜡花的小筒子。
}照管剪各处蜡花,正欲得便且藏出去,不想一头撞在凤姐儿怀里。
凤姐便一扬手,照脸一下,把那小孩子打了一个筋斗,骂道:“野牛肏的,胡朝那里跑!”那小道士也不顾拾烛剪,爬起来往外还要跑。
正值宝钗等下车,众婆娘媳妇正围随的风雨不透,但见一个小道士滚了出来,都喝声叫“拿,拿,拿!打,打,打!”\par
贾母听了忙问:“是怎么了?”贾珍忙出来问。
凤姐上去搀住贾母,就回说:“一个小道士儿,剪灯花的,没躲出去,这会子混钻呢。
”贾母听说,忙道:“快带了那孩子来,别唬着他。
小门小户的孩子,都是娇生惯养的,那里见的这个势派。
倘或唬着他,倒怪可怜见的,他老子娘岂不疼的慌?”\ping{凤姐与贾母对待一个小道士的态度大相径庭,作者可能是想要表现贾府权势的一个缩影。
贾母来清虚观打醮,为家族祈福,这一番对小道士的安慰可能是想为儿孙积德吧。
}说着,便叫贾珍去好生带了来。
贾珍只得去拉了那孩子来。
那孩子还一手拿着蜡剪,跪在地下乱战。
\zhu{战:通“颤”,发抖。
}贾母命贾珍拉起来,叫他别怕,问他几岁了。
那孩子通说不出话来。
\zhu{通:全部,整个。
这里形容程度之深。
}贾母还说“可怜见的”,又向贾珍道:“珍哥儿,带他去罢。
给他些钱买果子吃,别叫人难为了他。
”贾珍答应,领他去了。
这里贾母带着众人,一层一层的瞻拜观玩。
外面小厮们见贾母等进入二层山门,忽见贾珍领了一个小道士出来,叫人来带去,给他几百钱,不要难为了他。
家人听说,忙上来领了下去。
\par
贾珍站在阶矶上,因问:“管家在那里?”底下站的小厮们见问,都一齐喝声说:“叫管家!”登时林之孝一手整理着帽子跑了来,到贾珍跟前。
贾珍道:“虽说这里地方大,今儿不承望来这么些人。
你使的人,你就带了往你的那院里去;使不着的,打发到那院里去。
把小幺儿们多挑几个在这二层门上同两边的角门上,伺候着要东西传话。
你可知道不知道,今儿小姐奶奶们都出来,一个闲人也到不了这里。
”林之孝忙答应“晓得”,又说了几个“是”。
贾珍道:“去罢。
”又问:“怎么不见蓉儿?”一声未了,只见贾蓉从钟楼里跑了出来。
贾珍道:“你瞧瞧他,我这里也还没敢说热,他倒乘凉去了!”喝命家人啐他。
那小厮们都知道贾珍素日的性子,违拗不得,有个小厮便上来向贾蓉脸上啐了一口。
贾珍又道:“问着他!”那小厮便问贾蓉道:“爷还不怕热,哥儿怎么先乘凉去了?”贾蓉垂着手,一声不敢说。
那贾芸、贾萍、贾芹等听见了,不但他们慌了,亦且连贾璜、贾㻞、贾琼等也都忙了,一个一个从墙根下慢慢的溜上来。
贾珍又向贾蓉道:“你站着作什么?还不骑了马跑到家里,告诉你娘母子去!老太太同姑娘们都来了,叫他们快来伺候。
”贾蓉听说,忙跑了出来,一叠声要马,一面抱怨道:“早都不知作什么的,这会子寻趁我。
”\zhu{寻趁:本意是寻找。
这里是故意找茬的意思。
}一面又骂小子:“捆着手呢?马也拉不来。
”待要打发小子去,又恐后来对出来,说不得亲自走一趟,骑马去了,不在话下。
\par
且说贾珍方要抽身进去,只见张道士站在旁边陪笑说道:“论理我不比别人,应该里头伺候。
只因天气炎热,众位千金都出来了,法官不敢擅入,请爷的示下。
恐老太太问,或要随喜那里,
\zhu{
随喜:佛教术语。
谓见人作善事而随之生欢喜心。
后游览参观寺院,亦称随喜。
}
我只在这里伺候罢了。
”贾珍知道这张道士虽然是当日荣国府国公的替身,\zhu{替身:旧时,王公贵族有寄名为僧、道的,本人不在寺、观,而由别人代替,这种代人为僧、道者,称为“替身”。
}曾经先皇御口亲呼为“大幻仙人”,如今现掌“道录司”印,\zhu{道录司:明洪武十三年设,清沿之。
管理道教事务,发给道士“度牒”(取得道士资格的身份证明)。
掌道录司印指任道录司长官。
}又是当今封为“终了真人”,现今王公藩镇都称他为“神仙”,所以不敢轻慢。
二则他又常往两个府里去,凡夫人小姐都是见的。
今见他如此说,便笑道:“咱们自己,你又说起这话来。
再多说,我把你这胡子还挦了呢!\zhu{挦:音“闲”,拔(毛发)。
}还不跟我进来。
”那张道士呵呵大笑,跟了贾珍进来。
\par
贾珍到贾母跟前,控身陪笑说:\zhu{控身:半弯腰的姿势,表示恭敬。
}“这张爷爷进来请安。
”贾母听了,忙道:“搀他来。
”贾珍忙去搀了过来。
那张道士先哈哈笑道:“无量寿佛!老祖宗一向福寿安康?众位奶奶小姐纳福?一向没到府里请安,老太太气色越发好了。
”贾母笑道:“老神仙,你好?”张道士笑道:“托老太太万福万寿,小道也还康健。
别的倒罢,只记挂着哥儿,一向身上好?前日四月二十六日,我这里做遮天大王的圣诞,
\zhu{遮天大王:神名。未详。
第二十七回中写道四月二十六日是芒种节,这一天民俗要祭饯花神。不知和遮天大王的圣诞有何关系。
}
人也来的少,东西也很干净,我说请哥儿来逛逛,怎么说不在家?”贾母说道:“果真不在家。
”一面回头叫宝玉。
谁知宝玉解手去了才来,忙上前问:“张爷爷好?”张道士忙抱住问了好,又向贾母笑道:“哥儿越发发福了。
”
\zhu{发福:过去人们以为发胖是有福气。}
贾母道:“他外头好,里头弱。
又搭着他老子逼着他念书,生生的把个孩子逼出病来了。
”张道士道:“前日我在好几处看见哥儿写的字、作的诗,都好的了不得,怎么老爷还抱怨说哥儿不大喜欢念书呢?依小道看来,也就罢了。
”又叹道:“我看见哥儿的这个形容身段,言谈举动,怎么就同当日国公爷一个稿子!”说着两眼流下泪来。
贾母听说,也由不得满脸泪痕,说道:“正是呢,我养这些儿子孙子,也没一个像他爷爷的,就只这玉儿像他爷爷。
”\par
那张道士又向贾珍道:“当日国公爷的模样儿,爷们一辈的不用说,自然没赶上,大约连大老爷、二老爷也记不清楚了。
”说毕呵呵又一大笑,道:“前日在一个人家看见一位小姐,今年十五岁了,\ping{第二十二回,宝钗过十五岁生日。
}生的倒也好个模样儿。
我想着哥儿也该寻亲事了。
若论这个小姐模样儿,聪明智慧,根基家当,倒也配的过。
但不知老太太怎么样,小道也不敢造次。
等请了老太太的示下,才敢向人去说。
”贾母道:“上回有和尚说了,这孩子命里不该早娶,等再大一大儿再定罢。
你可如今打听着,不管他根基富贵,只要模样配的上就好,来告诉我。
便是那家子穷,不过给他几两银子罢了。
\ping{贾母贬低富有在宝玉婚姻选择中的重要性,而薛家乃皇商,“珍珠如土金如铁”。
}只是模样性格儿难得好的。
”\par
说毕,只见凤姐儿笑道:“张爷爷,我们丫头的寄名符儿你也不换去。
前儿亏你还有那么大脸,打发人和我要鹅黄缎子去!要不给你,又恐怕你那老脸上过不去。
”张道士呵呵大笑道:“你瞧,我眼花了,也没看见奶奶在这里,也没道多谢。
符早已有了,前日原要送去的,不指望娘娘来作好事,就混忘了,还在佛前镇着。
待我取来。
”说着跑到大殿上去,一时拿了一个茶盘,搭着大红蟒缎经袱子,\zhu{经袱子:过去称包裹书卷的布、帛为“袱子”;僧道用以包裹经卷的叫“经袱子”。
}托出符来。
大姐儿的奶子接了符。
张道士方欲抱过大姐儿来,只见凤姐笑道:“你就手里拿出来罢了,又用个盘子托着。
”
\ping{凤姐岔开话题,委婉的阻拦了张道士抱自己的女儿。}
张道士道:“手里不干不净的,怎么拿?用盘子洁净些。
”凤姐儿笑道:“你只顾拿出盘子来,倒唬我一跳。
我不说你是为送符,倒像是和我们化布施来了。
”众人听说,哄然一笑,连贾珍也撑不住笑了。
贾母回头道:“猴儿猴儿,你不怕下割舌头地狱?”凤姐儿笑道:“我们爷儿们不相干。
\zhu{我们爷儿们:我们(娘儿们)和爷儿们。
}他怎么常常的说我该积阴骘,\zhu{阴骘(音“治”):阴德。
}
迟了就短命呢!”\par
张道士也笑道:“我拿出盘子来一举两用,却不为化布施,倒要将哥儿的这玉请了下来,托出去给那些远来的道友并徒子徒孙们见识见识。
”贾母道:“既这么着,你老人家老天拔地的跑什么,
\zhu{老天拔地:形容老年人的行动不灵活。}
就带他去瞧了,叫他进来,岂不省事?”张道士道:“老太太不知道,看着小道是八十多岁的人,托老太太的福倒也健壮;
\zhu{一则:张道士说自己虽然老,但是还很健壮,费事跑跑没有问题。}
二则外面的人多,气味难闻,况是个暑热的天,哥儿受不惯,倘或哥儿受了腌臜气味,倒值多了。
”贾母听说,便命宝玉摘下通灵玉来,放在盘内。
那张道士兢兢业业的用蟒袱子垫着,捧了出去。
\par
这里贾母与众人各处游玩了一回,方去上楼。
只见贾珍回说:“张爷爷送了玉来了。
”刚说着,只见张道士捧了盘子,走到跟前笑道:“众人托小道的福,见了哥儿的玉,实在可罕。
都没什么敬贺之物,这是他们各人传道的法器,\zhu{法器:道士传道诵经使用的器具。
}都愿意为敬贺之礼。
哥儿便不希罕,只留着在房里顽耍赏人罢。
”贾母听说,向盘内看时,只见也有金璜,\zhu{璜:音“黄”,玉器,形状像半块璧。(璧:扁平圆形玉器,中有圆孔)。
金璜:仿璜制成的金饰。
}也有玉玦,
\zhu{玦[jué]:有缺口的环形佩玉。}
或有事事如意,
\zhu{
事事如意:寓吉祥意之金、玉佩饰。图案为柿子、如意。“柿”与“事”同音,加之如意,即成吉利语“事事如意”。
如意:一种用玉、象牙等制成的象征吉祥的物品,长约一二市尺,头为灵芝形或云朵形,柄略呈波浪形。
}
或有岁岁平安,
\zhu{
岁岁平安:寓吉祥意之金、玉佩饰。图案为麦穗、瓶、鹌鹑。
“穗”谐音“岁”,“瓶”谐音“平”,“鹌”谐音“安”,组成吉利语“岁岁平安”。
}
皆是珠穿宝贯,玉琢金镂,共有三五十件。
因说道:“你也胡闹。
他们出家人是那里来的,何必这样,这不能收。
”张道士笑道:“这是他们一点敬心,小道也不能阻挡。
老太太若不留下,岂不叫他们看着小道微薄,不像是门下出身了。
”\zhu{门下出身:门下:门庭之下。
张道士是荣国公的替身,故云。
}贾母听如此说,方命人接了。
宝玉笑道:“老太太,张爷爷既这么说,又推辞不得,我要这个也无用,不如叫小子们捧了这个,跟着我出去散给穷人罢。
”贾母笑道:“这倒说的是。
”张道士又忙拦道:“哥儿虽要行好,但这些东西虽说不甚希奇,到底也是几件器皿。
若给了乞丐,一则与他们无益,二则反倒遭塌了这些东西。
要舍给穷人,何不就散钱与他们。
”宝玉听说,便命收下,等晚间拿钱施舍罢了。
说毕,张道士方退出去。
\par
这里贾母与众人上了楼,在正面楼上归坐。
凤姐等占了东楼。
众丫头等在西楼,轮流伺候。
贾珍一时来回:“神前拈了戏,\zhu{神前拈了戏:打醮演戏是给“神”看的,不能由人指定戏目,而要用抽签、拈阉一类的方式,由“神”选出要看的戏。
}头一本《白蛇记》。
”\zhu{《白蛇记》:或指明代无名氏弋阳腔剧本。
演刘邦斩白蛇起义的故事。
}贾母问:“《白蛇记》是什么故事?”贾珍道:“是汉高祖斩蛇方起首的故事。
\zhu{起首:开始,这里是开始造反、开创基业的意思。
}第二本是《满床笏》。
”\zhu{《满床笏》:清代传奇剧,一名《十醋记》,演唐郭子仪“七子八婿,富贵寿考”的故事。
\zhu{寿考:长寿。
}}贾母笑道:“这倒是第二本上?也罢了。
神佛要这样,也只得罢了。
”又问第三本,贾珍道:“第三本是《南柯梦》。
”\zhu{《南柯梦》:明代汤显祖著传奇剧,名《南柯记》。
演淳于棼[fén]梦至大槐安国,拜驸马,当太守,显赫一时,而终于失宠见逐的故事。
《红楼梦》作者通过《白蛇记》、《满床笏》和《南柯记》三个戏,暗示了贾府由兴起至极盛而终于败落的过程。
}贾母听了便不言语。
贾珍退了下来,至外边预备着申表、焚钱粮、开戏,\zhu{申表:斋醮时道士恭读表章向神奏告的仪式。
焚钱粮:又名“烧包袱”,用纸糊的口袋,内装金、银箔纸折叠成的元宝。
祭神时与“申表”同时焚烧。
}不在话下。
\par
且说宝玉在楼上,坐在贾母旁边,因叫个小丫头子捧着方才那一盘子贺物,将自己的玉带上,用手翻弄寻拨,一件一件的挑与贾母看。
贾母因看见有个赤金点翠的麒麟,\zhu{赤金点翠的麒麟:麒麟状的嵌有翠鸟羽毛的纯金佩饰。
点翠是中国羽毛传统工艺之一,以翠鸟之蓝紫色羽毛巧妙地粘贴而成,色彩鲜艳,永不褪色。
麒麟为传说中的瑞兽。
}便伸手拿了起来,笑道:“这件东西好像我看见谁家的孩子也带着这么一个的。
”宝钗笑道:“史大妹妹有一个,比这个小些。
”贾母道:“是云儿有这个。
”宝玉道:“他这么往我们家去住着,我也没看见。
”探春笑道:“宝姐姐有心,不管什么他都记得。
”林黛玉冷笑道:“他在别的上还有限,惟有这些人带的东西上越发留心。
”\ping{
前一回已明出金玉良缘:“(宝钗的)金锁是个和尚给的,等日后有玉的方可结为婚姻”,这里又来一金麒麟。
}宝钗听说,便回头装没听见。
宝玉听见史湘云有这件东西,自己便将那麒麟忙拿起来揣在怀里。
一面心里又想到怕人看见他听见史湘云有了,他就留这件,因此手里揣着,却拿眼睛瞟人。
只见众人都倒不大理论,惟有林黛玉瞅着他点头儿,似有赞叹之意。
宝玉不觉心里没好意思起来,又掏了出来,向黛玉笑道:“这个东西倒好顽,我替你留着,到了家穿上你带。
”林黛玉将头一扭,说道:“我不希罕。
”宝玉笑道:“你果然不希罕,我少不得就拿着。
”说着又揣了起来。
\par
刚要说话,只见贾珍、贾蓉的妻子婆媳两个来了,彼此见过,贾母方说:“你们又来做什么,我不过没事来逛逛。
”一句话没说了,只见人报:“冯将军家有人来了。
”原来冯紫英家听见贾府在庙里打醮,连忙预备了猪羊香烛茶银之类的东西送礼。
凤姐儿听了,忙赶过正楼来,拍手笑道:“嗳呀!我就不防这个。
只说咱们娘儿们来闲逛逛,人家只当咱们大摆斋坛的来送礼。
\zhu{斋坛:僧道诵经的场所。
}都是老太太闹的。
这又不得不预备赏封儿。
\zhu{赏封:赏金。多用红封套装或用红纸包。}
”刚说了,只见冯家的两个管家娘子上楼来了。
冯家两个未去,接着赵侍郎也有礼来了。
于是接二连三,都听见贾府打醮,女眷都在庙里,凡一应远亲近友,世家相与都来送礼。
\zhu{相与:彼此交往的人。
}贾母才后悔起来,说:“又不是什么正经斋事,我们不过闲逛逛,就想不到这礼上,没的惊动了人。
”因此虽看了一天戏,至下午便回来了,次日便懒怠去。
凤姐又说:“打墙也是动土,\zhu{打墙也是动土:旧时迷信,盖房或筑墙都须先祭土神,然后“破土”,叫“动土”。
这句意谓反正要动手干,大干小干都一样。
}已经惊动了人,今儿乐得还去逛逛。
”那贾母因昨日张道士提起宝玉说亲的事来,谁知宝玉一日心中不自在,回家来生气,嗔着张道士与他说了亲,口口声声说从今以后不再见张道士了,别人也并不知为什么原故;二则林黛玉昨日回家又中了暑:因此二事,贾母便执意不去了。
\ping{贾母坚定支持宝玉和黛玉,抗议张道士的提亲。
}凤姐见不去,自己带了人去,也不在话下。
\par
且说宝玉因见林黛玉又病了,心里放不下,饭也懒去吃,不时来问。
林黛玉又怕他有个好歹,因说道:“你只管看你的戏去,在家里作什么?”宝玉因昨日张道士提亲,心中大不受用,今听见林黛玉如此说,心里因想道:“别人不知道我的心还可恕,连他也奚落起我来。
”因此心中更比往日的烦恼加了百倍。
若是别人跟前,断不能动这肝火,只是林黛玉说了这话,倒比往日别人说这话不同,由不得立刻沉下脸来,说道:“我白认得了你。
罢了,罢了!”林黛玉听说,便冷笑了两声:“我也知道白认得了我,那里像人家有什么配的上呢。
”宝玉听了,便向前来直问到脸上:“你这么说,是安心咒我天诛地灭?”林黛玉一时解不过这个话来。
宝玉又道:“昨儿还为这个赌了几回咒,今儿你到底又准我一句。
\zhu{准:揣测,揣度。}
我便天诛地灭,你又有什么益处?”林黛玉一闻此言,方想起上日的话来。
\ping{第二十八回,宝玉说道:“除了老太太、老爷、太太这三个人,第四个就是妹妹了。
要有第五个人,我就说个誓。
”}今日原是自己说错了,又是着急,又是羞愧,便颤颤兢兢的说道:“我要安心咒你,我也天诛地灭。
何苦来!我知道,昨日张道士说亲,你怕阻了你的好姻缘,你心里生气,来拿我煞性子。
”\par
原来那宝玉自幼生成有一种下流痴病,况从幼时和黛玉耳鬓厮磨,心情相对;及如今稍明时事,又看了那些邪书僻传,凡远亲近友之家所见的那些闺英闱秀,皆未有稍及林黛玉者,所以早存了一段心事,只不好说出来,故每每或喜或怒,变尽法子暗中试探。
那林黛玉偏生也是个有些痴病的,也每用假情试探。
因你也将真心真意瞒了起来,只用假意,我也将真心真意瞒了起来,只用假意,如此两假相逢,终有一真。
其间琐琐碎碎,难保不有口角之争。
\ping{痴男怨女。
}即如此刻,宝玉的心内想的是:“别人不知我的心,还有可恕,难道你就不想我的心里眼里只有你!你不能为我烦恼,反来以这话奚落堵我。
可见我心里一时一刻白有你,你竟心里没我。
”心里这意思,只是口里说不出来。
那林黛玉心里想着:“你心里自然有我,虽有‘金玉相对’之说,你岂是重这邪说不重我的?我便时常提这‘金玉’,你只管了然自若无闻的,方见得是待我重,而毫无此心了。
如何我只一提‘金玉’的事,你就着急,可知你心里时时有‘金玉’,见我一提,你又怕我多心,故意着急,安心哄我。
”\par
看来两个人原本是一个心,但都多生了枝叶,反弄成两个心了。
那宝玉心中又想着:“我不管怎么样都好,只要你随意,我便立刻因你死了也情愿。
你知也罢,不知也罢,只由我的心,可见你方和我近,不和我远。
”那林黛玉心里又想着:“你只管你,你好我自好,你何必为我而自失。
殊不知你失我自失。
可见是你不叫我近你,有意叫我远你了。
”如此看来,却都是求近之心,反弄成疏远之意。
如此之话,皆他二人素习所存私心,也难备述。
\par
如今只述他们外面的形容。
那宝玉又听见他说“好姻缘”三个字,越发逆了己意,心里干噎,口里说不出话来,便赌气向颈上抓下通灵宝玉,咬牙恨命往地下一摔,道:“什么捞什骨子,
\zhu{捞什骨子:即“劳什子”,如同说“东西”、“玩意”,含有厌恶之意。}
我砸了你完事!”偏生那玉坚硬非常,摔了一下,竟文风没动。
\zhu{文风没动:纹丝不动。
}宝玉见没摔碎,便回身找东西来砸。
\ping{宝玉坚决拒绝金玉良缘。
}林黛玉见他如此,早已哭起来,说道:“何苦来,你摔砸那哑吧物件。
有砸他的,不如来砸我。
”二人闹着,紫鹃雪雁等忙来解劝。
后来见宝玉下死力砸玉,忙上来夺,又夺不下来,见比往日闹的大了,少不得去叫袭人。
袭人忙赶了来,才夺了下来。
宝玉冷笑道:“我砸我的东西,与你们什么相干!”\par
袭人见他脸都气黄了,眼眉都变了,从来没气的这样,便拉着他的手,笑道:“你同妹妹拌嘴,不犯着砸他,倘或砸坏了,叫他心里脸上怎么过的去?”林黛玉一行哭着,一行听了这话说到自己心坎儿上来,可见宝玉连袭人不如,越发伤心大哭起来。
心里一烦恼,方才吃的香薷饮解暑汤便承受不住,\zhu{香薷(薷音“如”)饮:香薷:植物名,叶茎可入药。
香薷饮:是由香薷、厚朴、扁豆制成的一种药剂。
治伤暑感冒。
}“哇”的一声都吐了出来。
紫鹃忙上来用手帕子接住,登时一口一口的把一块手帕子吐湿。
雪雁忙上来捶。
紫鹃道:“虽然生气,姑娘到底也该保重着些。
才吃了药好些,这会子因和宝二爷拌嘴,又吐出来。
倘或犯了病,宝二爷怎么过的去呢?”宝玉听了这话说到自己心坎儿上来,可见黛玉不如一紫鹃。
又见林黛玉脸红头胀,一行啼哭,
\zhu{一行[xíng]:一边。}
一行气凑,
\zhu{气凑:气喘。}
一行是泪,一行是汗,不胜怯弱。
宝玉见了这般,又自己后悔方才不该同他较证,\zhu{较证:辩驳是非。
}这会子他这样光景,我又替不了他。
心里想着,也由不的滴下泪来了。
袭人见他两个哭,由不得守着宝玉也心酸起来,又摸着宝玉的手冰凉,待要劝宝玉不哭罢,一则又恐宝玉有什么委曲闷在心里,二则又恐薄了林黛玉。
不如大家一哭,就丢开手了,因此也流下泪来。
紫鹃一面收拾了吐的药,一面拿扇子替林黛玉轻轻的扇着,见三个人都鸦雀无声,各人哭各人的,也由不得伤心起来,也拿手帕子擦泪。
四个人都无言对泣。
\par
一时,袭人勉强笑向宝玉道:“你不看别的,你看看这玉上穿的穗子,也不该同林姑娘拌嘴。
”林黛玉听了,也不顾病,赶来夺过去,顺手抓起一把剪子来要剪。
袭人紫鹃刚要夺,已经剪了几段。
林黛玉哭道:“我也是白效力。
他也不希罕,自有别人替他再穿好的去。
”袭人忙接了玉道:“何苦来,这是我才多嘴的不是了。
”宝玉向林黛玉道:“你只管剪,我横竖不带他,也没什么。
”\par
只顾里头闹,谁知那些老婆子们见林黛玉大哭大吐,宝玉又砸玉,不知道要闹到什么田地,倘或连累了他们,便一齐往前头回贾母王夫人知道,好不干连了他们。
那贾母王夫人见他们忙忙的作一件正经事来告诉,也都不知有了什么大祸,便一齐进园来瞧他兄妹。
急的袭人抱怨紫鹃为什么惊动了老太太、太太,紫鹃又只当是袭人去告诉的,也抱怨袭人。
那贾母,王夫人进来,见宝玉也无言,林黛玉也无话,问起来又没为什么事,便将这祸移到袭人紫鹃两个人身上,说:“为什么你们不小心伏侍,这会子闹起来都不管了!”因此将他二人连骂带说教训了一顿。
二人都没话,只得听着。
还是贾母带出宝玉去了,方才平服。
\par
过了一日,至初三日,乃是薛蟠生日,家里摆酒唱戏,来请贾府诸人。
宝玉因得罪了林黛玉,二人总未见面,心中正自后悔,无精打采的,那里还有心肠去看戏,因而推病不去。
林黛玉不过前日中了些暑溽之气,\zhu{溽:音“入”,湿。
}
本无甚大病,听见他不去,心里想:“他是好吃酒看戏的,今日反不去,自然是因为昨儿气着了。
再不然,他见我不去,他也没心肠去。
只是昨儿千不该万不该剪了那玉上的穗子。
管定他再不带了,
\zhu{管定:肯定。}
还得我穿了他才带。
”因而心中十分后悔。
\par
那贾母见他两个都生了气,只说趁今儿那边看戏,他两个见了也就完了,不想又都不去。
老人家急的抱怨说:“我这老冤家是那世里的孽障,偏生遇见了这么两个不省事的小冤家,\zhu{
省[xǐng]:明白;觉悟。
冤家:对所爱的人的昵称。
}没有一天不叫我操心。
真是俗语说的,‘不是冤家不聚头’。
几时我闭了这眼,断了这口气,凭着这两个冤家闹上天去,我眼不见心不烦,也就罢了。
\ping{贾母此时坚定支持宝黛结合,但是正如贾母自己所言,最大的变数可能是贾母自己的身体状况能否支撑到那一天,倘若贾母“闭了这眼,断了这口气”,纵使宝黛“两个冤家闹上天去”,最后还是被拆散。
}偏又不咽这口气。
”自己抱怨着也哭了。
\par
这话传入宝林二人耳内。
原来他二人竟是从未听见过“不是冤家不聚头”的这句俗语,如今忽然得了这句话,好似参禅的一般,都低头细嚼此话的滋味,都不觉潸然泣下。
虽不曾会面,然一个在潇湘馆临风洒泪,一个在怡红院对月长吁,却不是人居两地,情发一心!\zhu{却不是:可能多了“不”字。
}\par
袭人因劝宝玉道:“千万不是,都是你的不是。
往日家里小厮们和他们的姊妹拌嘴,或是两口子分争,你听见了,你还骂小厮们蠢,不能体贴女孩儿们的心。
今儿你也这么着了。
明儿初五,大节下,你们两个再这么仇人似的,老太太越发要生气,一定弄的大家不安生。
依我劝,你正经下个气,陪个不是,大家还是照常一样,这么也好,那么也好。
”那宝玉听见了不知依与不依,要知端详,且听下回分解。
\par
\qi{总评:一片哭声,总因情重;金玉无言,何可为证?\ping{宝玉宝钗两人虽然有金锁和通灵宝玉这样的信物,但是冰冷的不会说话的“金玉”并不能验证两人的感情。
宝玉黛玉两人为对方垂泪,虽无信物,但是感情之重无需多言。
}}
\dai{057}{清虚观打醮}
\dai{058}{痴情女情重愈斟情}
\sun{p29-1}{贾母亲拈香清虚观}{元妃派夏太监送来一百二十两恨子,叫在清虚观初一到初三打三天平安醮。
初一这一日,荣国府门前车辆纷纷,人马簇簇,街上的人见是贾府去烧香,都站在两边观看。
}
\sun{p29-2}{清虚观张道士迎接}{宝玉骑马在贾母轿前,众车轿人马在后,浩浩荡荡,直奔清虚观而来。
不多时到了清虚观门口,只听钟鸣鼓响,早有张法官执香披衣,带领众道士在路旁迎候。
宝玉下了马, 贾母的轿抬进山门以内,见了本境城隍土地各位泥塑圣像,便命住轿。
}
\sun{p29-3}{小道误撞凤姐,贾母慈悲饶恕}{贾珍带领子弟上来迎接,凤姐的轿子却赶在头里先到了,见贾母下了轿,忙要搀扶。
可巧有一个十二三岁的小道士,拿着个剪筒,照管各处剪蜡花儿,正欲得便钻出去,不想一头撞在凤姐怀里,凤姐便一扬手,照脸打了个嘴巴,把那小孩子打了一个跟头,骂道“野牛肏的,胡朝哪里跑?”正值宝钗等下车,众婆娘媳妇围随得风雨不透,但见一个小道士滚了出来,皆喝拿喊打。
贾母见小道士可怜,命贾珍拉起来,叫他别怕,又令贾珍赏了几百小钱。
}
\sun{p29-4}{张道士托送寄名符}{张道士前来给贾母问安, 又向贾府奶奶姑娘们道了万福。
张道士问起宝玉,宝玉忙上前来。
张道士夸奖了宝玉一番。
这时,只见凤姐因问张道士要巧姐儿的寄名符, 张道士跑到大殿上,拿了个茶盘,搭着大红蟒缎经袱子,托出符来。
此时,贾珍之妻尤氏与贾蓉新娶媳妇胡氏也赶到了。
}