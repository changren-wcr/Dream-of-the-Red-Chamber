\chapter{贾宝玉初试云雨情\quad 刘姥姥一进荣国府}
\jia{宝玉、袭人亦大家常事耳,写得是已全领警幻意淫之训。
此回借刘妪,却是写阿凤正传,并非泛文,且伏“二进”“三进”及巧姐之归着。
\zhu{巧姐之归着:根据脂评,巧姐在贾府败落后最终被刘姥姥所救。}
\hang
此回刘妪一进荣国府,用周瑞家的,又过下回无痕,
\zhu{又过下回无痕:下一回“送宫花周瑞叹英莲”周瑞家的再次登场。}
是无一笔写一人文字之笔。
}
\par
\qi{风流真假一般看,借贷亲疏触眼酸。
总是幻情无了处,银灯挑尽泪漫漫。
}\par
题曰:\par
朝扣富儿门,富儿犹未足。
\zhu{足:足够,充足。
}虽无千金酬,嗟彼胜骨肉。
\par
\hop
却说秦氏因听见宝玉从梦中唤他的乳名,心中自是纳闷,又不好细问。
彼时宝玉迷迷惑惑,若有所失。
众人忙端上桂圆汤来,呷了两口,遂起身整衣。
袭人伸手与他系裤带时,不觉伸手至大腿处,只觉冰凉一片粘湿。
\ping{贾宝玉春梦导致的梦遗。
}唬的忙退出手来,问是怎么了。
宝玉红涨了脸,把他手一捻。
袭人本是个聪明女子,年纪本又比宝玉大两岁,近来也渐通人事,今见宝玉如此光景,心中便觉察了一半,不觉也羞的红涨了脸面,\meng{存身分。
}遂不敢再问。
\meng{既少通人事,无心者则再不复问矣;既问,则无限幽思,皆在于伏身之一笑,
\zhu{本回后文:“羞的袭人掩面伏身而笑“。}
所以必当有偷试之一番。
行文轻巧,皆出于自然,毫无一些勉强。
妙极!}仍旧理好衣裳,遂至贾母处来,胡乱吃毕晚饭,过这边来。
\par
袭人忙趁众奶娘丫鬟不在旁时,另取出一件中衣来与宝玉换上。
\zhu{中衣:贴身衬裤。
}宝玉含羞央告道:“好姐姐,千万别告诉别人,要紧!”袭人亦含羞笑问道:“你梦见什么故事了?\meng{是必当问者。
若不问则下文涉于唐突。
}是那里流出来的些脏东西?”宝玉道:“一言难尽。
”说着,便把梦中之事细说与袭人听了,然后说至警幻所授云雨之情,羞的袭人掩面伏身而笑。
\meng{试想。
}宝玉亦素喜袭人柔媚姣俏,遂强袭人同领警幻所训云雨之事。
\jia{数句文完一回提纲文字。
}袭人素知贾母已将自己与了宝玉的,今便如此,亦不为越礼,\jia{写出袭人身份。
}遂和宝玉偷试一番,幸得无人撞见。
\ping{既然说偷试云雨“亦不为越礼”,又何必有“幸得无人撞见”的侥幸?从后文王夫人的态度可知,偷试云雨并非能被她接受,袭人的“亦不为越礼”实际上是在采取冒险行动时安慰自己、给自己壮胆的借口。
}自此宝玉视袭人更与别个不同,\jia{伏下晴雯。
}袭人侍宝玉更为尽职。
\jia{一段小儿女之态,可谓追魂摄魄之笔。
}暂且别无话说。
\jia{一句\sout{接}[结]住上回“红楼梦”大篇文字,另起本回正文。
}\par
按荣府中一宅中合算起来,人口虽不多,从上至下也有三四百丁;事虽不多,一天也有一二十件,竟如乱麻一般,并没个头绪可作纲领。
正寻思从那一件事、自那一个人写起方妙,恰好忽从千里之外、芥豆之微、小小一个人家,因与荣府略有些瓜葛,\jia{略有些瓜葛,是数十回后之正脉也。
真千里伏线。
}这日正往荣府中来,因此便就此一家说来,倒还是头绪。
你道这一家姓甚名谁,又与荣府有甚瓜葛?诸公若嫌琐碎粗鄙呢,则快掷下此书,另觅好书去醒目;\meng{\sout{加}[夹]杂世态,巧伏下文。
}若谓聊可破闷时,待蠢物\jia{妙谦,是石头口角。
}逐细言来。
\par
方才所说的这小小一家,乃本地人氏,姓王,祖上曾作过小小的一个京官,昔年曾与凤姐之祖、王夫人之父识认,因贪王家的势利,便连了宗认作侄儿。
\zhu{连了宗:亦作“联宗”。
旧时为拉关系把同姓而本非一个宗族的人认了本家,叫作“联宗”。
}\jia{与贾雨村遥遥相对。
}\meng{可怜。
}那时只有王夫人之大兄、凤姐之父\jia{两呼两起,不过欲观者自醒。
}与王夫人随在京中的,知有此一门远族,馀者皆不识认。
\meng{强认亲的榜样。
}目今其祖已故,
\zhu{目今:现在,当前。}
只有一个儿子,名唤王成,因家业萧条,仍搬出城外原乡中住去了。
王成新近亦因病故,只有其子,小名狗儿,亦生一子,小名板儿,嫡妻刘氏,又生一女,名唤青儿。
\jia{《石头记》中公勋世宦之家以及草莽庸俗之族,无所不有,自能各得其妙。
}一家四口,仍以务农为业,\ping{阶级下降,从京官跌落到农民,甚至还需要去“打秋风”维持生计。
}因狗儿白日间又作些生计,刘氏又操井臼等事,
\zhu{
井:汲水。
臼:音“旧”,舂米的器具(舂:音“充“,把谷物的壳捣掉),用石头凿成,这里代指舂米。
井臼:指汲水、舂米,泛指家务劳作。
}
青板姊妹\foot{原作“青板姊弟”,据己、庚本改。
按:据前文“又生一女”,可知青儿年龄小于板儿,青儿、板儿并非姐弟关系,而“姊妹”则可泛指姐妹、兄妹、姐弟等关系。
}两个无人看管,狗儿遂将岳母刘姥姥\jia{音老,出《谐声字笺》。
称呼毕肖。
}接来一处过活。
\meng{总是用\sout{过}[逼]近法。
}这刘姥姥乃是个久经世代的老寡妇,膝下又无儿女,只靠两亩薄田地度日。
如今女婿接来养活,岂不愿意,遂一心一计,帮趁着女儿女婿过活起来。
\par
因这年秋尽冬初,天气冷将上来,家中冬事未办,狗儿未免心中烦虑,吃了几杯闷酒,在家闲寻气恼,\jia{病此病人不少,请来看狗儿。
}
\meng{贫苦人多有此等景象。
}刘氏不敢顶撞。
\jia{自“红楼梦”一回至此,则珍馐中之虀耳,\zhu{虀:音“基”,“齑”的繁体字,切碎的姜、葱、蒜等。
}好看煞!}因此刘姥姥看不过,乃劝道:“姑夫,你别嗔着我多嘴。
咱们村庄人,那一个不是老老诚诚的,多大碗吃多大的饭?\jia{能两亩薄田度日,方说的出来。
}你皆因年小时,托着你那老的福,\jia{妙称,何肖之至!}吃喝惯了,如今所以把持不住。
有了钱就顾头不顾尾,没了钱就瞎生气,\ping{昔盛今衰,从俭入奢易,从奢入俭难。
}成个什么男子汉大丈夫了!\jia{为纨绔下针,却先从此等小处写来。
}
\jia{此口气自何处得来?}\meng{英雄失足千古同慨,笑煞天下一切。
}如今咱们虽离城住着,终是天子脚下。
这长安城中,\zhu{长安城:在今陕西西安市西北。
汉、唐等朝曾建都于此。
这里借指京都。
}遍地都是钱,只可惜没人会拿去罢了。
在家跳蹋也没中用的。
”\zhu{跳蹋:也作“跳跶”。
急得顿足。
}狗儿听说,便急道:“你老只会炕头儿上混说,难道叫我打劫偷去不成?”\meng{古人有错用盗字之说,的是此句章本。
\zhu{
“错用盗字”令人费解。
章本:即“张本”,为了事情的发展而于预先所做的安排;
为作伏笔而预先说的话或写的文章;
根据、理由。
这里取“根据、理由”之意。
}
}刘姥姥道:“谁叫你偷去呢。
到底大家想方法儿裁度,不然,那银子钱自己跑到咱家来不成?”狗儿冷笑道:“有法儿还等到这会子呢!我又没有收税的亲戚,\jia{骂死。
}作官的朋友,\jia{骂死。
}有什么法子可想的?便有,也只怕他们未必来理我们呢!”\par
刘姥姥道:“这倒不然。
谋事在人,成事在天。
咱们谋到了,靠菩萨的保佑,有些机会,也未可知。
我倒替你们想出一个机会来。
当日你们原是和金陵王家\jia{四字便抵一篇世家传。
}连过宗的,二十年前,他们看承你们还好,如今自然是你们拉硬屎,\zhu{拉硬屎:装作硬气。
俗谓瘦驴拉硬屎——瞎逞能。
瘦弱的毛驴拉出来的却是硬屎。比喻没有本事硬逞能或没有实力硬摆架子。
}不肯去俯就他,\meng{天下事无有不可为者。
总因打不破,若打破时何事不能?请看刘姥姥一篇议论,便应解得些个才是。
}
故疏远起来。
想当初我和女儿还去过一遭。
\jia{补前文之未到处。
}他家的二小姐着实响快会待人的,倒不拿大。
\zhu{拿大:摆架子,瞧不起人。
}如今现是荣国府贾二老爷的夫人。
听得说,如今上了年纪,越发怜贫恤老,最爱斋僧敬道,舍米舍钱的。
如今王府虽升了边任,\zhu{
边任:防守边疆的重任。
这里指的是王子腾升了九省统制,奉旨出都查边。
}只怕这二姑太太还认得咱们。
你何不去走动走动,或者他念旧,有些好处,也未可定。
只要他发一点好心,拔一根寒毛比咱们的腰还粗呢!”刘氏一旁接口道:“你老虽说得是,但只你我这样个嘴脸,怎么好到他门上去的?先不先,他们那些门上人也未必肯去通报。
没的去打嘴现世。
”\zhu{没的:无端,平白无故。
打嘴现世:说嘴打嘴,丢人现眼。
说嘴打嘴:才夸口就出丑。
}\meng{“打嘴现世”等字,误尽许多苍生,也能成全多少事体。
}\ping{上顿不接下顿还死要面子,未尝试便自觉放弃。
}\par
谁知狗儿名利心甚重,\jia{调侃语。
}听如此一说,心下便有些活动起来。
又听他妻子这番话,便笑接道:“姥姥既如此说,况且当年你又见过这姑太太一次,何不你老人家明日就走一趟,先试试风头再说。
”刘姥姥道:“嗳哟哟!\jia{口声如闻。
}可是说的,‘侯门深似海’,\zhu{侯门深似海:形容官僚贵族之家宅大院深、门禁森严,难以进入。
语本唐代崔郊《赠去婢》诗:“侯门一入深如海,从此萧郎是路人。
”后常以“侯门深似海”,喻故友旧识因地位悬殊而隔绝。
}我是个什么东西,他家人又不认得我,我去了也是白去的。
”狗儿笑道:“不妨,我教你老一个法子:你竟带了外孙子小板儿,先去找陪房周瑞,\zhu{陪房:旧时富家女子的随嫁仆人。
}若见了他,就有些意思了。
这周瑞先时曾和我父亲交过一桩事,我们极好的。
”\jia{欲赴豪门,必先交其仆。
写来一叹。
}\meng{画出当日品行。
}刘姥姥道:“我也知道他的。
只是许多时不走动,知道他如今是怎么样?这也说不得了,你又是个男人,又这样个嘴脸,自然去不得,我们姑娘年轻媳妇子,也难卖头卖脚去,倒还是舍着我这付老脸去碰一碰。
果然有些好处,大家都有益,便是没银子来,我也到那公府侯门见一见世面,也不枉我一生。
”说毕,大家笑了一回。
当晚计议已定。
\par
次日天未明,刘姥姥便起来梳洗了,又将板儿教训几句。
那板儿才亦五六岁的孩子,一无所知,听见带他进城逛\jia{音光,去声。
游也。
出《谐声字笺》。
}去,便喜的无不应承。
于是刘姥姥带他进城,找至宁荣街。
\jia{街名。
本地风光,妙!}来至荣府大门石狮子前,只见簇簇的轿马,刘姥姥便不敢过去,且弹弹衣服,又教了板儿几句话,然后\ceng\jia{“\ceng ”字神理。
}到角门前。
\zhu{\ceng:也作蹭,这里是行动缓慢、欲行又止的样子。
}只见几个挺胸叠肚、指手画脚的人,坐在大凳上说东谈西呢。
\jia{不知如何想来,又为侯门三等豪奴写照。
}\meng{世家奴仆个个皆然,形容逼真。
}刘姥姥只得上来问:“太爷们纳福。
”\zhu{纳福:受福。
旧时见面常用的客套话。
}众人打量了他一会,便问是那里来的。
刘姥姥陪笑道:“我找太太的陪房周大爷的,烦那位太爷替我请他出来。
”那些人听了,都不瞅睬,\zhu{瞅睬:理睬。
}半日方说道:“你远远的那墙角下等着,\meng{故套。
}一会子他们家有人就出来的。
”内中有一年老的说道:“不要误他的事,何苦耍他。
”因向刘姥姥道:“那周大爷已往南边去了。
他在后一带住着,他娘子却在家。
你要找时,从这边绕到后街上后门上问就是了。
”\jia{有年纪人诚厚,亦是自然之理。
}
\meng{转换法。
写门上豪奴不能尽是规矩,故用转换法则不强硬,而笔气自顺。
\zhu{转换法:类似于“横云断岭法”,行文曲折生动、情节跌宕起伏。}
}\par
刘姥姥听了谢过,遂携了板儿,绕到后门上。
只见门前歇着些生意担子,也有卖吃的,也有卖顽意物件的,闹烘烘三二十个孩子在那里厮闹。
\jia{如何想来?合眼如见。
}刘姥姥便拉住了一个道:“我问哥儿一声,有个周大娘可在家么?”孩子道:“那个周大娘?我们这里周大娘有三个呢,还有两个周奶奶,不知是那一行当上的?”\zhu{行(音“杭”)当:本指戏曲中角色的分类。
这里指职务的类别。
}刘姥姥道:“是太太的陪房周瑞。
”孩子道:“这个容易,你跟我来。
”说着,跳跳蹿蹿的引着刘姥姥进了后门,\jia{因女眷,又是后门,故容易引入。
}至一院墙边,指与刘姥姥道:“这就是他家。
”又叫道:“周大娘,有个老奶奶来找你呢。
”\par
周瑞家的在内听说,忙迎了出来,问:“是那位?”刘姥姥忙迎上来问道:“好呀,周嫂子!”周瑞家的认了半日,方笑道:“刘姥姥,你好呀!你说说,能几年,我就忘了。
\jia{如此口角,从何处出来?}请家里来坐罢。
”刘姥姥一壁走,
\zhu{一壁:即“一边”。
}
一壁笑说道:“你老是贵人多忘事,那里还记得我们了。
”\ping{有奉承,有嗔怪。
}说着,来至房中。
周瑞家的命雇的小丫头倒上茶来吃着,周瑞家的又问板儿“长的这么大了”,又问些别后闲语,再问刘姥姥:“今日还是路过,还是特来的?”\jia{问的有情理。
}\meng{刘姥姥此是一团要紧事在心,有问不得不答,递转递进,不敢陟然看之,\zhu{陟:音“志”,登,上;提升,提拔。
“陟然看之”在这里大概是居高临下地看,鄙视轻蔑的意思。
}令人可怜。
而大英雄亦有若此者,所谓“欲图大事,不拘小节。
”}刘姥姥便说:“原是特来瞧瞧你嫂子,二则也请请姑太太的安。
若可以领我见一见更好,若不能,便借重嫂子转致意罢了。
”\jia{刘婆亦善于权变应酬矣。
}\par
周瑞家的听了,便猜着几分意思。
只因昔年他丈夫周瑞争买田地一事,其中多得狗儿之力,今见刘姥姥如此而来,心中难却其意,\jia{在今世,周瑞妇算是个怀情不忘的正人。
}二则也要显弄自己体面。
\jia{“也要显弄”句为后文作地步,也陪房本心本意实事。
}\meng{实有此等情理。
}听如此说,便笑说:“姥姥你放心,\jia{自是有宠人声口。
}大远的诚心诚意的来了,岂有个不教你见个真佛去的?\jia{好口角。
}\zhu{真佛:佛教术语。
佛教徒谓佛有报、应、化三身,“报身佛”相对于“化身佛”称为真佛,又名“无相之法身”,即难以见到之意。
因此世俗借此喻难以见到的人物。
}论理,人来客至回话,却不与我们相干。
我们这里都是各占一枝儿:\jia{略将荣府中带一带。
}我们男的只管春秋两季地租子,闲时只带着小爷们出门就完了,我只管跟太太奶奶们出门的事。
皆因你原是太太的亲戚,又拿我当个人,投奔了我来,我竟破个例,给你通个信去。
\ping{要情的姿态做足了。
}但只一件,姥姥有所不知,我们这里又比不得五年前了。
如今太太竟不大管事了,都是琏二奶奶当家。
你道这琏二奶奶是谁?就是太太的内侄女,当日大舅爷的女儿,小名凤哥的。
”刘姥姥听了,罕问道:“原来是他!怪道呢,我当日就说他不错呢。
\jia{我亦说不错。
}这等说来,我今儿还得见他了。
”周瑞家的道:“这个自然的。
如今太太事多心烦,有客来了,略可推得去的也就推过去了,都是这凤姑娘周旋迎待。
今儿宁可不见太太,倒要见他一面,\meng{礼势必然。
}
才不枉这里来一遭。
”刘姥姥道:“阿弥陀佛!这全仗嫂子方便了。
”周瑞家的道:“说那里话。
俗语说的:‘与人方便,自己方便。
’不过用我说一句话罢了,害着我什么。
”\ping{示宠炫能。
}说着,便唤小丫头子到倒厅上\jia{一丝不乱。
}悄悄的打听打听,\zhu{倒厅:古代建筑,大厅多数是坐北向南,坐南向北的厅房以及大厅后面向后院开门的附属部分,均称“倒厅”。
}老太太屋里摆了饭了没有。
小丫头去了。
这里二人又说些闲话。
\meng{急忙中偏不就进去,又添一番议论,从中又伏下多少线索,方见得大家势派,出入不易,方见得周瑞家的处事详细,即至后文,放笔写凤姐,亦不唐突,仍用冷子兴说荣、宁旧笔法。
}\par
刘姥姥因说:“这位凤姑娘今年大不过二十岁罢了,就这等有本事,当这样的家,可是难得的。
”周瑞家的听了道:“嗐!我的姥姥,告诉不得你呢。
这位凤姑娘年纪虽小,行事却比世人都大呢。
如今出挑的美人一样的模样儿,少说些有一万个心眼子。
再要赌口齿,十个会说话的男人也说他不过。
回来你见了就信了。
就只一件,待下人未免太严了些。
”\jia{略点一句,伏下后文。
}说着,只见小丫头回来说:“老太太屋里已摆完了饭,二奶奶在太太屋里呢。
”周瑞家的听了,连忙起身,催着刘姥姥说:“快走,快走!这一下来他吃饭是一个空子,\meng{非身临其境者不知。
}咱们先等着去。
若迟一步,回事的人也多了,难说话。
再歇了中觉,越发没了时候了。
”\jia{写出阿凤勤劳冗杂,并骄矜珍贵等事来。
}
\jia{写阿凤勤劳等事,然却是虚笔,故于后文不犯。
}\meng{有曰:富贵不还乡,如衣锦夜行。
今日周瑞家的得遇刘姥姥,实可谓锦衣不夜行者。
}说着一齐下了炕,打扫打扫衣服,又教了板儿几句话,随着周瑞家的,逶迤往贾琏的住宅来。
\par
先到了倒厅,周瑞家的将刘姥姥安插在那里略等一等。
自己先过影壁,进了院门,知凤姐未下来,先找着了凤姐的一个心腹通房大丫头,\zhu{通房大丫头:贴身侍婢收纳为妾,称“通房丫头”。
其地位低于姨娘。
通房又称“收房”。
}
\jia{着眼。
这也是书中一要紧人。
《红楼梦》内虽未见有名,想亦在副册内者也。
}
名唤平儿的。
\jia{名字真极,文雅则假。
}\meng{三等奴仆,第次不乱。
\ping{平儿是王熙凤的心腹,这里被称为“三等奴仆”,令人摸不到头脑。}
}周瑞家的先将刘姥姥起初来历说明,\jia{细!盖平儿原不知有此一人耳。
}又说:“今日大远的特来请安。
当日太太是常会的,今儿不可不见,\ping{捧刘姥姥,言必见之理。
}所以我带了他进来了。
等奶奶下来,我细细回明,奶奶想也不责备我莽撞的。
”平儿听了,便作了主意:\meng{各有各自的身分。
}“叫他们进来,先在这里坐着就是了。
”\jia{暗透平儿身份。
}周瑞家的听了,忙出去领他两个进入院来。
上了正房台矶,小丫头子打起了猩红毡帘,\jia{是冬日。
}才入堂屋,只闻一阵香扑了脸来,\jia{是刘姥姥鼻中。
}竟不辨是何香味,身子如在云端里一般。
\jia{是刘姥姥身子。
}满屋里之物都是耀眼争光,使人头悬目眩。
\jia{是刘姥姥头目。
}\meng{是写府第奢华,还是写刘姥姥粗夯?大抵村舍人家见此等气象,未有不破胆惊心,迷魄醉魂者。
}刘姥姥斯时惟点头咂嘴念佛而已。
\jia{六字尽矣,如何想来。
}\meng{刘姥姥犹能念佛,已自出人头地矣。
}于是来至东边这间屋内,乃是贾琏的女儿大姐儿睡觉之所。
\jia{记清。
}\meng{不知不觉先到大姐寝室,岂非有缘?
\zhu{有缘:指刘姥姥给大姐取名、在贾府败落后救出大姐。}
}平儿站在炕沿边,打量了刘姥姥两眼,\jia{写豪门侍儿。
}只得\jia{字法。
\zhu{此处甲戌本脂评的意思大概是赞叹用字精妙恰当。
}}问个好让坐。
刘姥姥见平儿遍身绫罗,插金带银,花容玉貌的,\jia{从刘姥姥心中目中略一写,非平儿正传。
}便当是凤姐儿了。
\jia{毕肖。
}\meng{的真有是情理。
}才要称姑奶奶,忽听周瑞家的称他是平姑娘,又见平儿赶着周瑞家的称周大嫂,方知不过是个有些体面的丫头。
于是让刘姥姥和板儿上了炕,平儿和周瑞家的对面坐在炕沿上,小丫头子斟上茶来吃茶。
\par
刘姥姥只听见“咯当”“咯当”的响声,大有似乎打箩柜筛面的一般,\zhu{打箩柜筛面:箩柜:装有筛面箩的木柜。
筛面时用脚不断踩踏机关,发出“咯当咯当”的声音。
}\jia{从刘姥姥心中意中幻拟出奇怪文字。
}\yang{小家气象。
}不免东瞧西望的。
忽见堂屋中柱子上挂着一个匣子,底下又坠着一个秤砣般的一物,却不住的乱晃。
\jia{从刘姥姥心中目中设譬拟想,真是镜花水月。
\zhu{镜花水月:镜中的花,水里的月。比喻空幻不实在。}
}刘姥姥心中想着:“这是个什么爱物儿?\zhu{爱物儿:玩意儿。
}有煞用呢?”正呆时,\jia{三字有劲。
}陡听得“当”地一声,又若金钟铜磬一般,不防倒唬的一展眼。
接着又是一连八九下。
\jia{细!是巳时。
}\jia{写得出。
}方欲问时,\meng{刘姥姥不认得,偏不令问明。
}只见小丫头子们一齐乱跑,说:“奶奶下来了。
”\meng{即以“奶奶下来了”结局,是画云龙妙手。
}平儿与周瑞家的忙起身,命刘姥姥:“只管坐着等,是时候我们来请你呢。
”说着,都迎出去了。
\par
刘姥姥屏声侧耳默候。
只听远远有人笑声,\jia{写得侍仆妇。
}约有一二十妇人,衣裙悉率,
\zhu{悉率:形容物体摩擦的声音。}
渐入堂屋,往那边屋内去了。
又见两三个妇人,都捧着大漆捧盒,进这东边来等候。
听见那边说了一声“摆饭”,渐渐人才都散出,只有伺候端菜的几个人。
半日鸦雀不闻之后,忽见二个人抬了一张炕桌来,放在这边炕上,桌上碗盘森列,仍是满满的鱼肉在内,不过略动了几样。
\meng{白描入神。
}
\ping{这是凤姐吃剩的,但是并非是给刘姥姥吃的,因为下文叙述了专门的“客馔”。
}
板儿一见了,便吵着要肉吃,刘姥姥一巴掌打下他去。
忽见周瑞家的笑嘻嘻走过来,招手儿叫他。
刘姥姥会意,于是携了板儿,下炕至堂屋中,周瑞家的又和他唧咕了一会,方\ceng       到这边屋里来。
\par
只见门外錾铜钩上悬着大红撒花软帘,\jia{从门外写来。
}南窗下是炕,炕上大红毡条,靠东边板壁立着一个锁子锦靠背与一个引枕,\zhu{板壁:分隔房间的木板墙。
锁子锦:用金色丝线织成锁链形图案的锦缎。
}铺着金心绿闪缎大坐褥,
\zhu{
闪缎:缎子的一种。一般缎子正面亮,背面暗,
但闪缎织时经纬线交叉,缎面在光线投照下,有暗处,有亮处,
闪烁光芒,故名。
金心绿闪缎:绿色地子上闪烁着金黄色纹样。
}
旁边有银唾沫盒。
那凤姐儿家常带着紫貂昭君套,\zhu{貂:鼬属的一种小型动物。
貂皮是贵重的短毛细皮,以紫貂最贵。
昭君套:没有顶的女用皮帽罩,因形同戏曲、绘画中昭君出塞所戴之罩,故名。
}围着攒珠勒子,\zhu{勒子:帽箍,用珠玉穿缀或以绒缎做成,套于额上,掩及耳间。
}穿着桃红撒花袄,石青刻丝灰鼠披风,\zhu{石青:淡灰青色。
刻丝:在丝织品上用丝平织成的图案,与凸出的绣花不同。
灰鼠:即松鼠。体毛灰色、暗褐色或赤褐色,腹面白色。毛皮可制衣,尾毛可制笔。
}大红洋绉银鼠皮裙,\zhu{
洋:指舶来品。
绉:音“宙”,皱纹。
洋绉:极薄而软的平纹春绸,微带自然皱纹。
银鼠:状颇类鼬,耳小毛短,其色洁白,皮可御轻寒,极贵重。
}
粉光脂艳,端端正正坐在那里,\jia{一段阿凤房室起居器皿家常正传,奢侈珍贵好奇货注脚,写来真是好看。
}手内拿着小铜火箸儿拨手炉内的灰。
\zhu{
手炉:暖手用的小炉子,铜制,内中燃炭,有多孔的盖子和提梁。
火箸:铜制的小火筷子,用于拨手炉内的灰、炭。
}
\jia{这一句是天然地设,非别文杜撰妄拟者。
}\jia{至平,实至奇,稗官中未见此笔。
}平儿站在炕沿边,捧着一个小小的填漆茶盘,
\zhu{
填漆:漆器制作工艺的一种。即在漆器上雕刻花纹,在刻纹处填以彩漆。
填漆有两种工艺:一是填彩和漆面相平;二是雕填后花纹凹陷,不与漆面平,显出刀刻味。
}盘内一小盖钟。
凤姐儿也不接茶,也不抬头,\jia{神情宛肖。
}只管拨手炉内的灰,慢慢地问道:“怎么还不请进来?”\jia{此等笔墨,真可谓追魂摄魄。
}\meng{“还不请进来”五字,写尽天下富贵人待穷亲戚的态度。
}一面说,一面抬身要茶时,只见周瑞家的已带了两个人在地下站着了。
这才忙欲起身,犹未起身,满面春风的问好,又嗔周瑞家的不早说。
\ping{先冷后热。
}刘姥姥在地下已是拜了数拜,问姑奶奶安。
凤姐忙说:“周姐姐,快搀住不拜罢,请坐。
我年轻,不大认得,可也不知是什么辈数,不敢称呼。
”周瑞家的忙回道:“这就是我才回的那个姥姥了。
”\jia{凤姐云“不敢称呼”,周瑞家的云“那个姥姥”。
凡三四句一气读下,方是凤姐声口。
}凤姐点头。
刘姥姥已在炕沿上坐下,板儿便躲在背后,百般的哄他出来作揖,他死也不肯。
\par
凤姐笑\jia{二笑。
}道:“亲戚们不大走动,都疏远了。
知道的呢,说你们弃厌我们,不肯常来;\jia{阿凤真真可畏可恶。
}\meng{偏会如此写来,教人爱煞!}不知道的那起小人,还只当我们眼里没人似的。
”刘姥姥忙念佛\jia{如闻。
}道:“我们家道艰难,走不起,来了这里,没的给姑奶奶打嘴,\zhu{打嘴:丢脸、出丑。
}就是管家爷们看着也不像。
\zhu{不像:不像样儿,不体面。}
”凤姐笑\jia{三笑。
}道:“这话叫人没的恶心。
不过借赖着祖父虚名,作个穷官儿罢了,谁家有什么,不过是个旧日的空架子。
\ping{此时实话听起来也像是客套。
}俗语说,‘朝廷还有三门子穷亲’呢,何况你我。
”\meng{点醒多少势利鬼。
}说着,又问周瑞家的回了太太了没有。
\jia{一笔不肯落空,的是阿凤。
}周瑞家的道:“如今等奶奶的示下。
”凤姐儿道:“你去瞧瞧,要是有人有事就罢,得闲呢就回,看怎么说。
”\meng{“看”之一字细极。
\ping{王熙凤让王夫人定夺如何打发刘姥姥。}
}周瑞家的答应着去了。
\par
这里凤姐叫人抓些果子与板儿吃,刚问些闲话时,就有家下许多媳妇管事的来回话。
\jia{不落空家务事,却不实写。
妙极!妙极!}平儿回了,凤姐道:“我这里陪客呢,晚上再回。
若有很要紧的,你就带进来现办。
”平儿出去一会,进来说:“我都问了,没有什么紧事,我就叫他们散了。
”\meng{能事者故自不凡。
}凤姐儿点头。
只见周瑞家的回来,向凤姐道:“太太说了,今日不得闲,二奶奶陪着便是一样。
多谢费心想着。
白来逛逛呢便罢,若有甚说的,只管告诉二奶奶,都是一样。
”刘姥姥道:“也没甚说的,不过是来瞧姑太太、姑奶奶,也是亲戚们的情分。
”周瑞家的道:“没甚说的便罢,若有话,回二奶奶,是和太太一样的。
”\jia{周妇系真心为老妪也,可谓得方便。
}一面说,一面递眼色儿与刘姥姥。
\jia{何如?余批不谬。
}刘姥姥会意,未语先飞红的脸,\meng{开口告人难。
}欲待不说,今日又所为何来?只得忍耻\jia{老妪有忍耻之心,故后有招大姐之事。
作者并非泛写,且为求亲靠友下一棒喝。
}说道:“论理今儿初次见姑奶奶,却不该说的,只是大远的奔了你老这里来,也少不的说了。
”刚说到这里,只听得二门上小厮们回说:“东府里小大爷进来了。
”凤姐忙止刘姥姥:“不必说了。
”一面便问:“你蓉大爷在那里呢?”\jia{惯用此等横云断山法。
}只听一路靴子脚响,进来了一个十七八岁的少年,面目清秀,身材夭娇,轻裘宝带,美服华冠。
\jia{如纨绔写照。
}刘姥姥此时坐不是,立不是,藏没处藏。
凤姐笑道:“你只管坐着,这是我侄儿。
”刘姥姥方扭扭捏捏在炕沿上坐了。
\par
贾蓉笑道:“我父亲打发我来求婶子,说上回老舅太太给婶子的那架玻璃炕屏,\zhu{炕屏:陈设在炕上的一种小屏风。
}明日请一个要紧的客,借了略摆一摆就送过来的。
”\jia{夹写凤姐好奖誉。
}凤姐道:“说迟了一日,昨儿已经给了人了。
”贾蓉听说,嘻嘻的笑着,在炕沿下半跪道:“婶子若不借,又说我不会说话了,又挨一顿好打呢。
婶子只当可怜侄儿罢。
”\ping{不知是婶慈侄孝还是打情骂俏。
}凤姐笑\jia{又一笑,凡五。
}道:“也没见我们王家的东西都是好的不成?一般你们那里放着那些东西,只是看不见我的才罢。
”贾蓉笑道:“那里如这个好呢!只求开恩罢。
”凤姐道:“碰一点儿,你可仔细你的皮!”因命平儿拿了楼门钥匙,传几个妥当人来抬去。
贾蓉喜的眉开眼笑,忙说:“我亲自带了人拿去,别由他们乱碰。
”说着便起身出去了。
\par
这里凤姐忽又想起一事来,便向窗外叫:“蓉儿回来。
”外面几个人接声说:“蓉大爷快回来。
”贾蓉忙复身转来,垂手侍立,听何指示。
\jia{传神之笔,写阿凤跃跃纸上。
}那凤姐只管慢慢的吃茶,出了半日神,方笑道:“罢了,你且去罢。
\meng{试想“且去”以前的丰态,其心思用意,作者无一笔不巧,无一事不丽。
}晚饭后你来再说罢。
这会子有人,我也没精神了。
”\ping{暧昧。
}贾蓉应了,方慢慢的退去。
\jia{妙!却是从刘姥姥身边目中写来。
}\jia{度至下回。
}\par
这里刘姥姥心神方安,方又说道:“今日我带了你侄儿来,也不为别的,只因为他老子娘在家里,连吃的都没有。
如今天又冷了,越想没个派头儿,\zhu{派头儿:这里是“盼头儿”的衍音。
}只得带了你侄儿奔了你老来。
”说着又推板儿道:“你那爹在家怎么教你了?打发咱们作煞事来?\zhu{作煞事:作啥事。
}只顾吃果子咧。
”凤姐早已明白了,听他不会说话,因笑止道:\jia{又一笑,凡六。
自刘姥姥来凡笑五次,写得阿凤乖滑伶俐,\zhu{乖滑:机灵圆滑。
}合眼如立在前。
}\jia{若会说话之人,便听他说了,阿凤厉害处正在此。
}
\jia{问看官:常有将挪移借贷已说明白了,彼仍推聋装哑,这人\sout{为}[比]阿凤若何?呵呵,一叹!}“不必说了,我知道了。
”因问周瑞家的道:“这刘姥姥不知可用过饭没有呢?”刘姥姥忙道:“一早就往这里赶咧,那里还有吃饭的工夫咧。
”凤姐听说,忙命快传饭来。
一时周瑞家的传了一桌客馔来,摆在东边屋内,过来带了刘姥姥和板儿过去吃饭。
凤姐说道:“周姐姐,好生让着些儿,我不能陪了。
”于是过东边房里来。
凤姐又叫过周瑞家的去,问他方才回了太太,说了些什么?周瑞家的道:“太太说,他们家原不是一家子,不过因出一姓,当年又与太老爷在一处作官,偶然连了宗的。
这几年来也不大走动。
当时他们来一遭,却也没空了他们。
今儿既来了瞧瞧我们,是他的好意思,\jia{穷亲戚来看是“好意思”,余又自《石头记》中见了,叹叹!}也不可简慢了。
他便是有什么说的,叫二奶奶裁度着就是了。
”\jia{王夫人数语令余几哭出。
}\ping{周瑞家的帮刘姥姥疏通关系讨要救济,也是为了向刘姥姥彰显自己在豪门的影响力。
周瑞家的作为传话人,可以掺入自己的主观想法,从而借助他人之手达到自己的目的。
}凤姐听了说道:“我说呢,既是一家子,我如何连影儿也不知道。
”\par
说话时,刘姥姥已吃毕饭,拉了板儿过来,舔唇抹嘴的道谢。
凤姐笑道:“且请坐下,听我告诉你老人家。
方才意思我已知道了。
若论亲戚之间,原该不待上门来就该有照应才是。
但如今家里杂事太烦,太太渐上了年纪,一时想不到也是有的。
\jia{点“不待上门就该有照应”数语,此亦于《石头记》再见话头。
}况是我近来接着管些事,都不大知道这些个亲戚们。
二则外头看着这里烈烈轰轰的,殊不知大有大的艰难去处,说与人也未必信罢了。
\ping{的确实话,但在别人耳中却是客套哭穷。
}今儿你既老远的来了,又是头一次见我张口,怎好叫你空回去呢。
\jia{也是《石头记》再见了,叹叹!}可巧昨儿太太给我的丫头们作衣裳的二十两银子,我还没动呢,你们不嫌少,就暂且拿了去罢。
”\meng{凤姐能事,在能体王夫人的心,托故周全,无过不及之弊。
}那刘姥姥先听见告艰难,只当是没有,心里便突突的,\jia{可怜可叹!}后来听见给他二十两,喜的浑身发痒起来,\jia{可怜可叹!}说道:“嗳,我也是知道艰难的。
但俗语说,‘瘦死的骆驼比马还大’,凭的怎么样,你老拔根寒毛比我们的腰还粗呢!”周瑞家的在旁听他说的粗鄙,只管使眼色止他。
凤姐听了,笑而不睬,只命平儿把昨儿那包银子拿来,再拿一串钱来,\jia{这样常例亦再见。
}都送至刘姥姥跟前。
凤姐乃道:“这是二十两银子,暂且给这孩子做件冬衣罢。
若不拿着,可真是怪我了。
这串钱雇了车子坐罢。
改日无事,只管来逛逛,方是亲戚间的意思。
天也晚了,也不虚留你们了,到家里该问好的问个好儿罢。
”\meng{口角春风,如闻其声。
}一面说,一面就站起来了。
\par
刘姥姥只管千恩万谢,拿了银钱,随周瑞家的出来。
至外厢房,
\zhu{厢房:指四合院中东西两边的房子。}
周瑞家的方道:“我的娘!你见了他怎么倒不会说话了?开口就是‘你侄儿’。
我说句不怕你恼的话,便是亲侄儿,也要说和柔些。
那蓉大爷才是他的正经侄儿呢,他怎么又跑出这么个侄儿来了。
”\jia{与前“眼色”针对,可见文章中无一个闲字。
}\jia{为财势一哭。
}\meng{不自量者每每有之,而能不露圭角,\zhu{圭:音“归”,上圆或上尖下方的玉器。
圭角:圭的棱角,比喻锋芒,也比喻迹象。
}形诸无事,
\zhu{诸:文言「之于」、「之乎」的合音字。}
凤姐亦可谓人豪矣。
}刘姥姥笑道:“我的嫂子,\jia{赧颜如见。
\zhu{赧:音“难”三声,(因羞愧等)脸色泛红。
}}我见了他,心眼里爱还爱不过来,那里还说上话了。
”二人说着,又至周瑞家。
坐了片时,刘姥姥便要留下一块银,与周瑞家的儿女买果子吃,周瑞家的如何放在眼里,执意不肯。
\ping{豪奴夸富炫能,情态逼人。
}刘姥姥感谢不尽,仍从后门去了。
正是:\par
得意浓时易接济,受恩深处胜亲朋。
\par
\jia{“一进荣府”一回,曲折顿挫,笔如游龙,且将豪华举止令观者已得大概,想作者应是心花欲开之候。
\hang
借刘妪入阿凤正文,“送宫花”写“金玉初聚”为引,作者真笔似游龙,变幻难测,非细究至再三再四不记数,那能领会也?叹叹!}\par
\qi{总评:梦里风流,醒后风流,试问何真何假?刘姆乞谋,蓉儿借求,多少颠倒相酬。
\zhu{相酬:答谢恩情。颠倒相酬:凤姐对贾蓉的帮助很多,本应他对凤姐的回报要大,然而他还不如刘姥姥懂得感恩。}
英雄反正用机筹,不是死生看守。
\zhu{死生看守:对刘姆和蓉儿慷慨大方,不是吝啬的守财奴。
}}
\dai{011}{贾宝玉初试云雨情}
\dai{012}{刘姥姥一进荣国府}
\sun{p6-1}{宝玉袭人初试云雨,刘姥姥见周瑞家的}{图上侧:宝玉从梦中惊醒,袭人过来给他系裤带时,伸手至大腿处,只觉冰冷精湿一片,不觉把个粉脸羞得通红。
图左下:一日, 王夫人一个几十年前连过宗的远亲,如今家道败落,其岳母刘姥姥在女婿的怂恿下,带着外孙板儿来到荣府,托王夫人的陪房周瑞家的引见。
右下角的场景尚不清楚。
}
\sun{p6-2}{刘姥姥初会王熙凤,贾蓉借物言谈隐情}{图右侧:周瑞家的带刘姥姥来到凤姐房中,只见一位丽人端端正正坐在那里,手内拿着小铜火箸儿拨手炉内的灰。
凤姐儿也不抬头,慢慢地问道:“怎么还不请进来?”一面说,一面抬身要茶时,只见周瑞家的已带了两个人在地下站着了。
这才忙欲起身,犹未起身,满面春风的问好,又嗔周瑞家的不早说。
刘姥姥在地下已是拜了数拜,问姑奶奶安。
图左侧:贾蓉前来索借炕屏,和婶子王熙凤调笑,刘姥姥此时坐不是,立不是,藏没处藏。
凤姐笑道:“你只管坐着,这是我侄儿。
”刘姥姥方扭扭捏捏在炕沿上坐了。
}