\chapter{蘅芜君兰言解疑癖 \quad 潇湘子雅谑补馀音}
\zhu{兰言:知心话。
《易·系辞上》:“	二人同心,其利断金;同心之言,其臭(嗅)如兰。
”}
\par
\geng{钗、玉名虽二个,人却一身,此幻笔也。
今书至三十八回时,已过三分之一有馀,
\ping{从这条批语可以推测,本书最多有114回。}
故写是回,使二人合而为一。
请看黛玉逝后宝钗之文字,便知余言不谬矣。
\ping{钗黛合一。
}\ping{所以最起码脂批人看到了黛玉去世的情节。
}}\par
\qi{谁说诗书解误人,\zhu{解:能。
}豪华相尚失天真。
\zhu{豪华相尚:推崇追求豪华奢侈。
}见得古人原立意,不正心身总莫论。
}\par
话说他姊妹复进园来,吃过饭,大家散出,都无别话。
\par
且说刘姥姥带着板儿,先来见凤姐儿,说:“明日一早定要家去了。
虽住了两三天,日子却不多,把古往今来没见过的,没吃过的,没听见过的,都经验了。
难得老太太和姑奶奶并那些小姐们,连各房里的姑娘们,都这样怜贫惜老照看我。
我这一回去后没别的报答,惟有请些高香天天给你们念佛,\zhu{请高香:即烧高香,虔敬祈福之意。
}保佑你们长命百岁的,就算我的心了。
”凤姐儿笑道:“你别喜欢。
都是为你,老太太也被风吹病了,睡着说不好过;我们大姐儿也着了凉,在那里发热呢。
”刘姥姥听了,忙叹道:“老太太有年纪的人,不惯十分劳乏的。
”凤姐儿道:“从来没像昨儿高兴。
往常也进园子逛去,不过到一二处坐坐就回来了。
昨儿因为你在这里,要叫你逛逛,一个园子倒走了多半个。
大姐儿因为找我去,太太递了一块糕给他,谁知风地里吃了,就发起热来。
”刘姥姥道:“小姐儿只怕不大进园子,生地方儿,小人儿家原不该去。
比不得我们的孩子,会走了,那个坟圈子里不跑去。
一则风扑了也是有的;二则只怕他身上干净,眼睛又净,或是遇见什么神了。
依我说,给他瞧瞧祟书本子,\zhu{祟书本子:讲论鬼神星命、吉凶祸福的迷信书籍。
下文《玉匣记》即其一。
祟:音“岁”,鬼怪或鬼怪的祸害。
}仔细撞客着了。
”\zhu{撞客:旧时迷信用语,人遇鬼神为其所附以致生病招灾,叫“撞客”。
也叫“撞克”或“克碰”。
}
一语提醒了凤姐儿,便叫平儿拿出《玉匣记》着彩明来念。
彩明翻了一回念道:“八月二十五日,病者在东南方得遇花神。
用五色纸钱四十张,向东南方四十步送之,大吉。
”凤姐儿笑道:“果然不错,园子里头可不是花神!只怕老太太也是遇见了。
”一面命人请两分纸钱来,着两个人来,一个与贾母送祟,一个与大姐儿送祟。
果见大姐儿安稳睡了。
\geng{岂真送了就安稳哉?盖妇人之心意皆如此,即不送岂有一夜不睡之理?作者正描愚人之见耳。
}\par
凤姐儿笑道:“到底是你们有年纪的人经历的多。
我这大姐儿时常肯病,也不知是个什么原故。
”刘姥姥道:“这也有的事。
富贵人家养的孩子多太娇嫩,自然禁不得一些儿委曲;再他小人儿家,过于尊贵了,也禁不起。
以后姑奶奶少疼他些就好了。
”凤姐儿道:“这也有理。
我想起来,他还没个名字,你就给他起个名字。
一则借借你的寿;二则你们是庄家人,不怕你恼,到底贫苦些,你贫苦人起个名字,只怕压的住他。
”
\ping{
在富贵人家看来,贫贱人所起名字就是贱名,就能避邪祛病。起了贱名的孩子,例如“牛儿”“小猪”,以为孩子就真的像牛、猪这类低贱而易养活的畜类一样,吃粗劣的食物、不必怎样精心看护他也能养大。
}
\geng{一篇愚妇无理之谈,实是世间必有之事。
}刘姥姥听说,便想了一想,笑道:“不知他几时生的?”凤姐儿道:“正是生日的日子不好呢,可巧是七月初七日。
”
\zhu{
七月初七:即七夕。相传每年此夜,牛郎、织女二星在天河相会。旧俗妇女多进行乞巧活动,即在院里供奉瓜果,请求织女帮助自己提高纺织刺绣技巧。因此也称“乞巧节”或“女儿节”。
古人认为七不吉利,跟死亡有关。女子逢七就更加不吉利,凤姐才说“日子不好”。另外牛郎织女被拆散不吉利。
且七夕牛郎织女相遇,人间的喜鹊都飞去给牛郎织女搭桥去了,而喜鹊历来是报喜的鸟儿,所以这一天出生的女孩,因为人间没了喜鹊,被认为是天生没有喜气的,所以不吉利。
巧姐出生在女性乞巧的日子,伏后文巧姐一生以纺绩为生,终生劳碌。
}
刘姥姥忙笑道:“这个正好,就叫他是巧哥儿。
这叫作‘以毒攻毒,以火攻火’的法子。
姑奶奶定要依我这名字,他必长命百岁。
日后大了,各人成家立业,或一时有不遂心的事,必然是遇难成祥,逢凶化吉,却从这‘巧’字上来。
”\meng{作谶语以影射后文。
\ping{巧姐判词:“偶因济刘氏,巧得遇恩人”,巧姐为刘姥姥所救。
}}\par
凤姐儿听了,自是欢喜,忙道谢,又笑道:“只保佑他应了你的话就好了。
”说着叫平儿来吩咐道:“明儿咱们有事,恐怕不得闲儿。
你这空儿把送姥姥的东西打点了,他明儿一早就好走的便宜了。
”刘姥姥忙说:“不敢多破费了。
已经遭扰了几日,\zhu{遭扰:犹言打扰。
客人对主人道谢的客套话。
}又拿着走,越发心里不安起来。
”\meng{世俗常态,逼真。
}凤姐儿道:“也没有什么,不过随常的东西。
好也罢,歹也罢,带了去,你们街坊邻舍看着也热闹些,也是上城一次。
”\ping{这是给刘姥姥社会地位的背书,可能也使得刘姥姥的家庭经济状况改善,考虑得不能说不周全,后文巧姐落入妓院,要被赎出来必然要一大笔钱,如果没有前面这一点因果,怕是刘姥姥即使有心也无力拿出这么多钱。
}
只见平儿走来说:“姥姥过这边瞧瞧。
”\par
刘姥姥忙赶了平儿到那边屋里,只见堆着半炕东西。
平儿一一的拿与他瞧着,说道:“这是昨日你要的青纱一匹,奶奶另外送你一个实地子月白纱做里子。
\zhu{实地子纱:纱中最厚密者。
月白:一种接近于白的浅蓝色。
}这是两个茧绸,\zhu{茧绸:以柞[zuò]蚕丝(蚕丝有柞蚕丝和桑蚕丝两类)织成的绸子。
}作袄儿裙子都好。
这包袱里是两匹绸子,年下做件衣裳穿。
这是一盒子各样内造点心,\zhu{内造点心:宫内制作的点心。
其制作的方法叫“内法”。
这里似指照内法仿制的点心。
}也有你吃过的,也有你没吃过的,拿去摆碟子请客,比你们买的强些。
这两条口袋是你昨日装瓜果子来的,如今这一个里头装了两斗御田粳米,熬粥是难得的;这一条里头是园子里果子和各样干果子。
这一包是八两银子。
这都是我们奶奶的。
这两包每包里头五十两,共是一百两,是太太给的,叫你拿去或者作个小本买卖,或者置几亩地,以后再别求亲靠友的。
”说着又悄悄笑道:“这两件袄儿和两条裙子,还有四块包头,\zhu{包头:缠裹在头上的布帛。
}一包绒线,可是我送姥姥的。
衣裳虽是旧的,我也没大狠穿,你要弃嫌,我就不敢说了。
”平儿说一样,刘姥姥就念一句佛,已经念了几千声佛了,又见平儿也送他这些东西,又如此谦逊,忙念佛道:“姑娘说那里话?这样好东西我还弃嫌!我便有银子也没处去买这样的呢。
只是我怪臊的,收了又不好,不收又辜负了姑娘的心。
”平儿笑道:“休说外话,咱们都是自己,我才这样。
你放心收了罢,我还和你要东西呢。
到年下,你只把你们晒的那个灰条菜干子和豇豆、
\zhu{
灰条菜干子:将长帮白菜在滚水中抄过,然后晾晒成干菜,颜色灰白,故称。
豇[jiāng]豆:一年生草本植物,荚果长条形,种子肾形。
}
扁豆、茄子、葫芦条儿各样干菜带些来,我们这里上上下下都爱吃。
这个就算了,别的一概不要,别罔费了心。
”刘姥姥千恩万谢答应了。
平儿道:“你只管睡你的去。
我替你收拾妥当了就放在这里,明儿一早打发小厮们雇辆车装上,不用你费一点心的。
”\par
刘姥姥越发感激不尽,过来又千恩万谢的辞了凤姐儿,过贾母这一边睡了一夜,次早梳洗了就要告辞。
因贾母欠安,众人都过来请安,出去传请大夫。
一时婆子回大夫来了,老妈妈请贾母进幔子去坐。
\zhu{幔子:帐幕。
这里指坐帐。
旧时贵妇人起居之处设此,作回避男宾等用。
}贾母道:“我也老了,那里养不出那阿物儿来,
\zhu{养……阿物儿:生孩子。}
还怕他不成!不要放幔子,就这样瞧罢。
”众婆子听了,便拿过一张小桌来,放下一个小枕头,便命人请。
\par
一时只见贾珍、贾琏、贾蓉三个人将王太医领来。
王太医不敢走甬路,只走旁阶,跟着贾珍到了阶矶上。
早有两个婆子在两边打起帘子,两个婆子在前导引进去,又见宝玉迎了出来。
只见贾母穿着青皱绸一斗珠的羊皮褂子,\zhu{一斗珠的羊皮褂子:即用未出生的胎羊皮做成的皮褂子。
这种羊皮,卷毛如一粒粒珠子,故又名“珍珠毛”。
一斗珠,又作“一斛(斛音“胡”)珠”。
}端坐在榻上,两边四个未留头的小丫鬟都拿着蝇帚漱盂等物;\zhu{留头:又叫“留满头”。
旧时女子幼年剃发,随着年事增长,先留顶心头发,再留全发,叫做“留头”。
}又有五六个老嬷嬷雁翅摆在两旁,\zhu{雁翅:雁群飞行时,排列有序,故用以比喻队列整齐。
}碧纱橱后隐隐约约有许多穿红着绿戴宝簪珠的人。
\zhu{
碧纱橱:装在房内起隔开作用的一扇一扇的木板墙,也称“隔扇”、“槅扇”。中间两扇平日可以开关,或加挂帘子帷帐,又叫“纱橱”、“纱厨”。
槅心部分常糊以绿纱,故称碧纱橱。
}
王太医便不敢抬头,忙上来请了安。
贾母见他穿着六品服色,便知御医了,也便含笑问:“供奉好?”\zhu{供奉:旧时以各种专长在宫廷内供职的人统称“供奉”。
这里是对王太医的尊称。
}因问贾珍:“这位供奉贵姓?”贾珍等忙回:“姓王。
”贾母道:“当日太医院正堂王君效,好脉息。
”\zhu{好脉息:指切脉的本领很高。
}王太医忙躬身低头,含笑回说:“那是晚晚生家叔祖。
”贾母听了,笑道:“原来这样,也是世交了。
”一面说,一面慢慢的伸手放在小枕头上。
老嬷嬷端着一张小杌,\zhu{杌:音“误”,小方凳。
}连忙放在小桌前,略偏些。
王太医便屈一膝坐下,歪着头诊了半日,又诊了那只手,忙欠身低头退出。
贾母笑说:“劳动了。
珍儿让出去好生看茶。
”\par
贾珍贾琏等忙答了几个“是”,复领王太医出到外书房中。
王太医说:“太夫人并无别症,偶感一点风凉,究竟不用吃药,不过略清淡些,暖着一点儿,就好了。
如今写个方子在这里,若老人家爱吃,便按方煎一剂吃,若懒待吃,也就罢了。
”说着吃过茶写了方子。
刚要告辞,只见奶子抱了大姐儿出来,笑说:“王老爷也瞧瞧我们。
”王太医听说忙起身,就奶子怀中,左手托着大姐儿的手,右手诊了一诊,又摸了一摸头,又叫伸出舌头来瞧瞧,笑道:“我说姐儿又骂我了,只是要清清净净的饿两顿就好了,不必吃煎药,我送丸药来,临睡时用姜汤研开,吃下去就是了。
”说毕作辞而去。
\par
贾珍等拿了药方来,回明贾母原故,将药方放在桌上出去,不在话下。
这里王夫人和李纨、凤姐儿、宝钗姊妹等见大夫出去,方从橱后出来。
王夫人略坐一坐,也回房去了。
\par
刘姥姥见无事,方上来和贾母告辞。
贾母说:“闲了再来。
”又命鸳鸯来,“好生打发刘姥姥出去。
我身上不好,不能送你。
”刘姥姥道了谢,又作辞,方同鸳鸯出来。
到了下房,鸳鸯指炕上一个包袱说道:“这是老太太的几件衣服,都是往年间生日节下众人孝敬的,老太太从不穿人家做的,收着也可惜,却是一次也没穿过的。
\meng{写富贵常态,一笔作三五笔用,妙文。
}昨日叫我拿出两套儿送你带去,或是送人,或是自己家里穿罢,别见笑。
这盒子里是你要的面果子。
这包子里是你前儿说的药:梅花点舌丹也有,紫金锭也有,活络丹也有,催生保命丹也有,\zhu{梅花点舌丹:消肿止痛,主治疗毒恶疮,口舌糜烂。
紫金锭:避秽解毒,主治由湿温时邪引起的呕恶泄泻,以及小儿痰壅惊闭(壅[yōng]:堵塞;堆积)。
活络丹:活血祛瘀,主治半身不遂,风湿疼痛。
催生保命丹:安胎镇痉(痉:痉挛[jìngluán],肌肉紧张,不由自主地收缩),主治难产。
}
每一样是一张方子包着,总包在里头了。
这是两个荷包,带着顽罢。
”说着便抽系子,掏出两个笔锭如意的锞子来给他瞧,\zhu{锞子(锞音“课”):旧时做货币用的小金锭或银锭。
笔锭如意的锞子:上面铸有一只如意和一枝笔的金银小元宝。
“笔锭如意”谐音“必定如意”,以讨吉利口彩。
}又笑道:“荷包拿去,这个留下给我罢。
”刘姥姥已喜出望外,早又念了几千声佛,听鸳鸯如此说,便说道:“姑娘只管留下罢。
”鸳鸯见他信以为真,仍与他装上,笑道:“哄你顽呢,我有好些呢。
留着年下给小孩子们罢。
”\meng{逼真。
}说着,只见一个小丫头拿了个成窑钟子来递与刘姥姥,“这是宝二爷给你的。
”刘姥姥道:“这是那里说起。
我那一世修了来的,今儿这样。
”说着便接了过来。
鸳鸯道:“前儿我叫你洗澡,换的衣裳是我的,你不弃嫌,我还有几件,也送你罢。
”刘姥姥又忙道谢。
鸳鸯果然又拿出两件来与他包好。
刘姥姥又要到园中辞谢宝玉和众姊妹、王夫人等去。
鸳鸯道:“不用去了。
他们这会子也不见人,回来我替你说罢。
闲了再来。
”又命了一个老婆子,吩咐他:“二门上叫两个小厮来,帮着姥姥拿了东西送出去。
”婆子答应了,又和刘姥姥到了凤姐儿那边一并拿了东西,在角门上命小厮们搬了出去,直送刘姥姥上车去了。
不在话下。
\par
且说宝钗等吃过早饭,又往贾母处问过安,回园至分路之处,宝钗便叫黛玉道:“颦儿跟我来,有一句话问你。
”黛玉便同了宝钗,来至蘅芜院中。
进了房,宝钗便坐了笑道:“你跪下,我要审你。
”\meng{严整。
}黛玉不解何故,因笑道:“你瞧宝丫头疯了!审问我什么?”宝钗冷笑道:“好个千金小姐!好个不出闺门的女孩儿!满嘴说的是什么?你只实说便罢。
”黛玉不解,只管发笑,心里也不免疑惑起来,口里只说:“我何曾说什么?你不过要捏我的错儿罢了。
你倒说出来我听听。
”宝钗笑道:“你还装憨儿。
昨儿行酒令你说的是什么?我竟不知那里来的。
”\meng{何等爱惜。
}\par
黛玉一想,方想起来昨儿失于检点,那《牡丹亭》、《西厢记》说了两句,不觉红了脸,便上来搂着宝钗,笑道:“好姐姐,原是我不知道随口说的。
你教给我,再不说了。
”\meng{真能受教。
尊重之态,姣痴之情,令人爱煞!}宝钗笑道:“我也不知道,听你说的怪生的,所以请教你。
”黛玉道:“好姐姐,你别说与别人,我以后再不说了。
”宝钗见他羞得满脸飞红,满口央告,便不肯再往下追问,因拉他坐下吃茶,\meng{若无下文,自己何由而知?笔下一丝不露痕迹中补足,存小姐身分,颦儿不得反问。
}款款的告诉他道:\zhu{款款:徐缓的,慢慢的。
}“你当我是谁,我也是个淘气的。
从小七八岁上也够个人缠的。
我们家也算是个读书人家,祖父手里也爱藏书。
先时人口多,姊妹弟兄都在一处,都怕看正经书。
弟兄们也有爱诗的,也有爱词的,诸如这些《西厢》《琵琶》以及‘元人百种’,\zhu{《琵琶》:即《琵琶记》,南戏剧本,元末高则诚作。
写蔡伯喈[jiē]负义再娶,其妻赵五娘卖发葬亲、求乞寻夫的故事。
“元人百种”:即《元曲选》,元代杂剧选集。
}无所不有。
\meng{藏书家当留意。
}他们是偷背着我们看,我们却也偷背着他们看。
后来大人知道了,打的打,骂的骂,烧的烧,才丢开了。
所以咱们女孩儿家不认得字的倒好。
男人们读书不明理,尚且不如不读书的好,何况你我。
就连作诗写字等事,原不是你我分内之事,究竟也不是男人分内之事。
男人们读书明理,辅国治民,这便好了。
\meng{作者一片苦心,代佛说法,代圣讲道,看书者不可轻忽。
}只是如今并不听见有这样的人,读了书倒更坏了。
这是书误了他,可惜他也把书遭塌了,所以竟不如耕种买卖,倒没有什么大害处。
你我只该做些针黹纺织的事才是,偏又认得了字,既认得了字,不过拣那正经的看也罢了,最怕见了些杂书,移了性情,就不可救了。
”一席话,说的黛玉垂头吃茶,心下暗伏,只有答应“是”的一字。
\meng{结得妙。
}\ping{宝姐姐是真正的推心置腹,其实想想之前的那些不愉快,不过是犹豫走向成人世界的黛玉和欣然拥抱成人世界的宝钗之间一点龃龉罢了,黛玉从这一刻,又向成人世界迈进一步啊。
}\par
忽见素云进来说:“我们奶奶请二位姑娘商议要紧的事呢。
二姑娘、三姑娘、四姑娘、史姑娘、宝二爷都在那里等着呢。
”宝钗道:“又是什么事?”黛玉道:“咱们到了那里就知道了。
”说着便和宝钗往稻香村来,果见众人都在那里。
\par
李纨见了他两个,笑道:“社还没起,就有脱滑的了,\zhu{脱滑:溜走,躲懒的意思。
}四丫头要告一年的假呢。
”黛玉笑道:“都是老太太昨儿一句话,又叫他画什么园子图儿,惹得他乐得告假了。
”探春笑道:“也别要怪老太太,都是刘姥姥一句话。
”林黛玉忙笑道:“可是呢,都是他一句话。
他是那一门子的姥姥,直叫他是个‘母蝗虫’就是了。
”说着大家都笑起来。
宝钗笑道:“世上的话,到了凤丫头嘴里也就尽了。
幸而凤丫头不认得字,不大通,不过一概是市俗取笑。
更有颦儿这促狭嘴,他用‘春秋’的法子,\zhu{“春秋”的法子:又称“春秋笔法”。
《春秋》是孔子根据鲁史撰修的编年体史书。
古代学者说它“以一字为褒贬”,含有“微言大义”。
后来就把文笔深隐曲折、意含褒贬叫“春秋笔法”。
}将市俗的粗话,撮其要,删其繁,再加润色比方出来,一句是一句。
\meng{触目惊心,请自回思。
}这‘母蝗虫’三字,把昨儿那些形景都现出来了。
亏他想的倒也快。
”众人听了,都笑道:“你这一注解,也就不在他两个以下。
”\par
李纨道:“我请你们大家商议,给他多少日子的假。
我给了他一个月他嫌少,你们怎么说?”黛玉道:“论理一年也不多。
这园子盖才盖了一年,如今要画自然得二年工夫呢。
又要研墨,又要蘸笔,又要铺纸,又要着颜色,又要……”刚说到这里,众人知道他是取笑惜春,便都笑问说:“还要怎样?”黛玉也自己撑不住笑道:“又要照着这样儿慢慢的画,可不得二年的工夫!”众人听了,都拍手笑个不住。
\zhu{
当借春嫌一个月的假少,黛玉指出这园子只盖了一年,画它何尝要用二年?黛玉欲贬故褒,反话正说,诙谐地为惜春嫌假日少编造出一连串站不住脚的、甚至荒唐的理由,
例如研墨、蘸笔、铺纸、着颜色,入木三分地凸出了惜春“照着样儿慢慢的画”的笨拙而偷懒的情状,给人一种滑稽可笑的印象。
}
宝钗笑道:“‘又要照着这个慢慢的画’,这落后一句最妙。
所以昨儿那些笑话儿虽然可笑,回想是没味的。
你们细想颦儿这几句话虽是淡的,回想却有滋味。
我倒笑的动不得了。
”\geng{看他刘姥姥笑后复一笑,亦想不到之文也。
听宝卿之评亦千古定论。
}惜春道:“都是宝姐姐赞的他越发逞强,这会子拿我也取笑儿。
”黛玉忙拉他笑道:“我且问你,还是单画这园子呢,还是连我们众人都画在上头呢?”惜春道:“原说只画这园子的,昨儿老太太又说,单画了园子成个房样子了,叫连人都画上,就像‘行乐’似的才好。
\zhu{行乐:行乐图的简称。
我国传统写真画的一种,要求人物神态毕肖,有一定的情节,有的还带有背景和次要人物。
}我又不会这工细楼台,\zhu{工细:工整而细致。
}又不会画人物,又不好驳回,正为这个为难呢。
”黛玉道:“人物还容易,你草虫上不能。
”李纨道:“你又说不通的话了,这个上头那里又用的着草虫?或者翎毛倒要点缀一两样。
”
\zhu{翎毛[língmáo]:羽毛;国画的一种,以鸟类为主要题材,也说翎毛画。}
黛玉笑道:“别的草虫不画罢了,昨儿‘母蝗虫’不画上,岂不缺了典!”众人听了,又都笑起来。
黛玉一面笑的两手捧着胸口,一面说道:“你快画罢,我连题跋都有了,\zhu{题跋:写于书籍、碑帖、字画等前面的文字叫“题”,后面的文字叫“跋”。
内容多为评介、考订、记事等。
}起个名字,就叫作《携蝗大嚼图》。
”\meng{愈出愈奇。
}\par
\chai{xichun}{惜春作画}
众人听了,越发哄然大笑,前仰后合。
只听“咕咚”一声响,不知什么倒了,急忙看时,原来是湘云伏在椅子背儿上,那椅子原不曾放稳,被他全身伏着背子大笑,他又不提防,两下里错了劲,向东一歪,连人带椅都歪倒了,幸有板壁挡住,不曾落地。
众人一见,越发笑个不住。
宝玉忙赶上去扶了起来,方渐渐止了笑。
宝玉和黛玉使个眼色儿,黛玉会意,\meng{何等妙文心,故意唐突。
}便走至里间将镜袱揭起,照了一照,只见两鬓略松了些,忙开了李纨的妆奁,拿出抿子来,\zhu{抿子:梳头时抹发油的小刷子。
}对镜抿了两抿,仍旧收拾好了,方出来,指着李纨道:“这是叫你带着我们作针线教道理呢,你反招我们来大顽大笑的。
”李纨笑道:“你们听他这刁话。
他领着头儿闹,引着人笑了,倒赖我的不是。
真真恨的我只保佑明儿你得一个利害婆婆,再得几个千刁万恶的大姑子小姑子,试试你那会子还这么刁不刁了。
”\meng{收结转折,处处情趣。
}\par
林黛玉早红了脸,拉着宝钗说:“咱们放他一年的假罢。
”宝钗道:“我有一句公道话,你们听听。
藕丫头虽会画,不过是几笔写意。
\zhu{写意:国画中属疏放类画法,与工笔画相对。
要求通过极简练疏放的笔墨写出对象的神态和意趣,借以抒发作者的胸怀。
}如今画这园子,非离了肚子里头有几幅丘壑的才能成画。
这园子却是像画儿一般,山石树木,楼阁房屋,远近疏密,也不多,也不少,恰恰的是这样。
你就照样儿往纸上一画,是必不能讨好的。
这要看纸的地步远近,
\zhu{地步:距离和规格。}
该多该少,分主分宾,该添的要添,该减的要减,该藏的要藏,该露的要露。
这一起了稿子,再端详斟酌,方成一幅图样。
第二件,这些楼台房舍,是必要用界划的。
\zhu{界划:即“界画”,国画术语。
指画家用界尺作线,精细地画出以宫室楼台为主体的画。
宋元时画分十三科,第十科为“界画楼台”。
见元代陶宗仪《辍耕录》。
后专指亭台楼阁一类的画为“界画”。
}一点不留神,栏杆也歪了,柱子也塌了,门窗也倒竖过来,阶矶也离了缝,甚至于桌子挤到墙里去,花盆放在帘子上来,岂不倒成了一张笑‘话’儿了。
第三,要插人物,也要有疏密,有高低。
衣折裙带,手指足步,最是要紧;一笔不细,不是肿了手就是跏了腿。\zhu{
跏[jiā]:行走时脚向内拐。如:跏子(瘸子,跛子)。
}染脸撕发倒是小事。
\zhu{
染脸:指人物面部施色。
撕发:一作“丝发”,用浓墨,笔尖散开,画出丝丝缕缕的细线,是为头发,这是工笔人物常用的技法。写意笔就不是撕发而用渲晕了。
}
依我看来竟难的很。
如今一年的假也太多,一月的假也太少,竟给他半年的假,再派了宝兄弟帮着他。
并不是为宝兄弟知道教着他画,那就更误了事;为的是有不知道的,或难安插的,宝兄弟好拿出去问问那会画的相公,就容易了。
”\par
宝玉听了,先喜的说:“这话极是。
詹子亮的工细楼台就极好,程日兴的美人是绝技,如今就问他们去。
”宝钗道:“我说你是无事忙,说了一声你就问去。
等着商议定了再去。
如今且拿什么画?”宝玉道:“家里有雪浪纸,\zhu{雪浪纸:一种优质的宣纸,适于画山水、树石。
}又大又托墨。
”\zhu{托墨:纸张不涩不滑,写字作画易于着墨渗附,谓之托墨。
下言“托色”,意同。
}宝钗冷笑道:“我说你不中用!那雪浪纸写字画写意画儿,或是会山水的画南宗山水,\zhu{南宗山水:指一种注重笔墨意趣的文人山水画。
明代董其昌等效禅学分南、北宗之意,将唐以来的山水画分为南北两大派系,名之为南北宗。
认为南宗的画注重水墨气韵,风格飘逸,重皴染,画得比较简洁,以王维为代表;北宗的画注重色彩工力,风格刚劲,重钩勒,画得比较工细,以李思训为代表。
关于绘画南北宗的此种界说相沿至今,当代书画家启功曾予驳正,参见《启功丛稿·山水画南北宗说辩》。
}托墨,禁得皴搜。
\zhu{皴(音“村”)搜:疑为“皴擦”之误。
皴擦又叫皴染,国画的一种技法,多用以表现山石、峰峦及树身的纹理。
特点是先勾出山石等轮廓再蘸水墨擦染出层次向背和质感来,常需多次加工。
}拿了画这个,又不托色,又难滃,\zhu{滃:音“翁”,三声,云气腾涌,水大的样子。
这里应该是“滃染”,即“烘染”。
}画也不好,纸也可惜。
我教你一个法子。
原先盖这园子,就有一张细致图样,虽是匠人描的,那地步方向是不错的。
\zhu{地步:距离和规格。}
你和太太要了出来,也比着那纸大小,和凤丫头要一块重绢,\zhu{重绢:厚重的好绢。
作画的画绢,以厚重细密均匀者为佳品。
}叫相公矾了,\zhu{矾:即明矾。
这里作动词。
用胶矾水浸刷生纸生绢,使之变得吸水适度,叫做矾。
没有上过矾的叫生纸,写书法或画画的时候,墨一上去就会渗开,因为它追求的是墨韵。熟纸则是矾过的纸,上过矾以后墨就不会被吸收,不会产生毛细现象,就可以在纸上慢慢地皴跟染,凡是草虫画、仕女画、界画都是用熟纸。
胶矾水:胶为黄明胶,矾为明矾。
}叫他照着这图样删补着立了稿子,添了人物就是了。
就是配这些青绿颜色并泥金泥银,\zhu{
泥金泥银:一种用金属粉末制成的颜料,呈金色(泥金)或银色(泥银),用来涂饰笺纸,或调和在油漆里涂饰器物。
}也得他们配去。
你们也得另爖上风炉子,\zhu{爖:音“龙”,点燃。
}预备化胶、出胶、洗笔。
\zhu{
化胶、出胶:中国画颜料,凡矿物质的朱砂、赭石、泥金、石青、石绿以及铝粉,
植物颜料的花青、胭脂之类,用时必须加胶,才能增强其黏附力。胶有以树脂,有以动物皮角制成,平时胶质坚韧呈块状或片状,加热始化成水,因之施色前必先“化胶”,使能与色调合。
但用后须将上浮的胶水撇净,否则有损色泽,这叫“出胶”。下次用时再加胶。
}
还得一张粉油大案,
\zhu{粉油大案:一种油漆大画案。}
铺上毡子。
你们那些碟子也不全,笔也不全,都得从新再置一分儿才好。
”惜春道:“我何曾有这些画器?不过随手写字的笔画画罢了。
就是颜色,只有赭石、广花、藤黄、胭脂这四样。
\zhu{
赭[zhě]石:中国画矿物颜料,即红土,但以黄赤鲜明为上,杵细,漂净杂质,加胶。
广花:广东出产的花青颜料,亦称“广青”、“靛花”,国画常用原色,加黄成绿,加红成紫。其原料以靛青(即靛蓝)植物如蓼蓝、马蓝等加工而成。
藤黄:中国画常用原色之一。藤黄乃树皮损伤后所渗出之树脂,故含有树胶,用时不必胶。
胭脂:一作“燕支”、“燕脂”,植物名。中国画红色颜料名。用红蓝花或苏木煎汁制成。亦可制为妇女化妆品,用以涂擦唇颊。
古时常以“胭脂”代红色,其色略带紫色。
}
再有,不过是两支着色笔就完了。
”宝钗道:“你不该早说。
这些东西我却还有,只是你也用不着,给你也白放着。
如今我且替你收着,等你用着这个时候我送你些,也只可留着画扇子,若画这大幅的也就可惜了的。
今儿替你开个单子,照着单子和老太太要去。
你们也未必知道的全,我说着,宝兄弟写。
”\par
宝玉早已预备下笔砚了,原怕记不清白,要写了记着,听宝钗如此说,喜的提起笔来静听。
宝钗说道:“头号排笔四支,
\zhu{排笔:染色、作画或油漆、粉刷用的一种笔,由平列的一排笔毛或几支毛笔连成一排做成。}
二号排笔四支,三号排笔四支,大染四支,
\zhu{染:专用于绘画中着色的笔。}
中染四支,小染四支,大南蟹爪十支,
\zhu{
蟹爪:绘画用的笔,以其形似蟹爪,故名。
产杭州、湖州、苏州者为佳,故名“南蟹爪”。
}
小蟹爪十支,须眉十支,
\zhu{须眉:小画笔,笔力强劲,用于细部勾勒,挺拔有力。}
大著色二十支,
\zhu{著色:上色的画笔。}
小著色二十支,开面十支,
\zhu{开面:一种很细的硬毫画笔,用以绘画人物面部的细部。}
柳条二十支,
\zhu{柳条:用来起画稿的细柳枝。用前略烧一下,吹灭后即可用。}
箭头朱四两,
\zhu{箭头朱:红色颜料,做成似小箭镞形的朱砂,供大量使用红色时用。用时将它放在乳钵中研细再对胶调匀。}
南赭四两,
\zhu{南赭[zhě]:南方产的赭石,是赭黄色的颜料。}
石黄四两,
\zhu{石黄:即雄黄。一种矿物,成分是硫化砷,桔黄色,有光泽。可用来制作黄色颜料。}
石青四两,
\zhu{石青:蓝铜矿的蓝色颜料,色青翠,颜色经久不变,多用于国画。}
石绿四两,
\zhu{
石绿:用孔雀石制成的绿色颜料,多用于国画。
孔雀石:一种矿物,是铜的碱式碳酸盐,翠绿色,有的像孔雀尾羽的花纹,多作葡萄状。
}
管黄四两,
\zhu{管黄:藤黄精制的管状绘画颜料。}
广花八两,
\zhu{
广花:广东出产的花青颜料,亦称“广青”、“靛花”,国画常用原色,加黄成绿,加红成紫。其原料以靛青(即靛蓝)植物如蓼蓝、马蓝等加工而成。
}
蛤粉四匣,
\zhu{蛤[gé]粉:中国颜料,以蛤蚌壳经煅研细,水飞用之,其色纯白细净而有光泽。}
胭脂十片,大赤飞金二百帖,青金二百帖,
\zhu{大赤飞金、青金:中国颜料,参见“泥金泥银”,一种用金属粉末制成的颜料,呈金色(泥金)或银色(泥银),用来涂饰笺纸,或调和在油漆里涂饰器物。}
广匀胶四两,
\zhu{
胶乃黏性物质,有取之动物皮角如牛皮胶,有取之于植物者为树胶。用于绘画颜料,可增强颜色的稳固性。
广匀胶:一种透明无臭气的胶,原产广东。
}
净矾四两。
\zhu{
净矾:即澄净明矾,为无色透明或白色半透明结晶体,故称“明矾”。
矾易溶解于水,水干又复呈粒状块状。画绢须先反复以矾水矾过,至绢不透光止。
画成后亦宜拂以轻微矾水,裱托时乃不易脱色。
}
矾绢的胶矾在外,
\zhu{
胶矾:胶为黄明胶,矾为明矾。
}
别管他们,你只把绢交出去叫他们矾去。
这些颜色,咱们淘澄飞跌着,\zhu{淘澄飞跌:调治国画颜料的四个步骤。
淘:把颜料研碎,洗去泥土。
澄:音“邓”,用乳钵研细淘过的颜料,兑入胶水澄清。
飞:澄清后淡色上浮,将其吹去。
跌:飞后留下中色和重色,再跌荡碗盏留下重色。
“淘”是淘洗,“澄”是沉淀,“飞”是吹走不要的东西,“跌”是剩下最后沉淀的最好的那个颜料。
}又顽了,又使了,包你一辈子都够使了。
再要顶细绢箩四个,
\zhu{细绢箩:以细绢制成的箩,用以淘洗或筛颜料用。}
粗绢箩四个,担笔四支,
\zhu{
“担”应写作“掸”,一种羊毫笔,多干用,如掸灰、掸细粉等。
有时沾湿用,如烘染底色时,为加速烘开,则用掸笔蘸水来帮助烘匀。
}
大小乳钵四个,
\zhu{乳钵:研细颜料的用器,形似臼,将颜料粉末或碎屑倾入,以杵研之使细。或干研,或带水研磨。}
大粗碗二十个,五寸粗碟十个,三寸粗白碟二十个,风炉两个,沙锅大小四个,
\zhu{沙锅:现在一般写作“砂锅”。}
新磁罐二口,
\zhu{磁:通「瓷」。以瓷土烧制成的器物。}
新水桶四只,一尺长白布口袋四条,浮炭二十斤,
\zhu{浮炭:同“麸炭”,轻而易燃的木炭。}
柳木炭一斤,
\zhu{柳木炭:用柳树木头烧制的炭。}
三屉木箱一个,实地纱一丈,
\zhu{实地纱:实地子纱,纱里面最厚密的。}
生姜二两,酱半斤。
”黛玉忙道:“铁锅一口,锅铲一个。
”宝钗道:“这作什么?”黛玉笑道:“你要生姜和酱这些作料,我替你要铁锅来,好炒颜色吃的。
”众人都笑起来。
宝钗笑道:“你那里知道。
那粗色碟子保不住不上火烤,不拿姜汁子和酱预先抹在底子上烤过了,一经了火是要炸的。
”众人听说,都道:“原来如此。
”\par
黛玉又看了一回单子,笑着拉探春悄悄的道:“你瞧瞧,画个画儿又要这些水缸箱子来了。
想必他糊涂了,把他的嫁妆单子也写上了。
”探春“嗳”了一声,笑个不住,说道:“宝姐姐,你还不拧他的嘴?你问问他编排你的话。
”宝钗笑道:“不用问,狗嘴里还有象牙不成!”一面说,一面走上来,把黛玉按在炕上,便要拧他的脸。
黛玉笑着忙央告:“好姐姐,饶了我罢!颦儿年纪小,只知说,不知道轻重,作姐姐的教导我。
姐姐不饶我,还求谁去?”众人不知话内有因,都笑道:“说的好可怜见的,连我们也软了,饶了他罢。
”宝钗原是和他顽,忽听他又拉扯前番说他胡看杂书的话,便不好再和他厮闹,放起他来。
黛玉笑道:“到底是姐姐,要是我,再不饶人的。
”
\ping{黛玉真心感激宝钗抓住自己的错却不大张声势,却一语而过。}
宝钗笑指他道:“怪不得老太太疼你,众人爱你伶俐,今儿我也怪疼你的了。
过来,我替你把头发拢一拢。
”黛玉果然转过身来,宝钗用手拢上去。
宝玉在旁看着,只觉更好,不觉后悔不该令他抿上鬓去,也该留着,此时叫他替他抿去。
\meng{又一点。
作者可称无漏子。
}正自胡思,只见宝钗说道:“写完了,明儿回老太太去。
若家里有的就罢,若没有的,就拿些钱去买了来,我帮着你们配。
”宝玉忙收了单子。
\par
大家又说了一回闲话。
至晚饭后又往贾母处来请安。
贾母原没有大病,不过是劳乏了,兼着了些凉,温存了一日,\zhu{温存:殷勤抚慰的意思,这里引申为休养、休息。
}又吃了一剂药疏散一疏散,\zhu{疏散:疏通、发散。
中医认为病有表里之分,病初起时邪在表,停留于肌肤腠理之间(腠理:音“凑理”,肌肉的纹理)。这时用发表药疏散一下即愈。
}至晚也就好了。
不知次日又有何话,且听下回分解。
\par
\qi{总评:摹写富贵,至于家人女子无不妆点,论诗书,讲画法,皆尽其妙,而其中隐语,惊人教人,不一而足,作者之用心,诚佛菩萨之用心也。
读者不可因其浅近而渺忽之。
}
\dai{083}{蘅芜君兰言解疑癖}
\dai{084}{潇湘子雅谑补馀音}
\sun{p42-1}{刘姥姥醉卧怡红院,蘅芜君兰言解疑癖}{图右侧:刘姥姥因喝了酒, 且吃了许多油腻,不免通泻。
入厕出来, 便觉头昏眼花,迷了来路。
恍惚中来到一间富丽房舍,忽见一副最精致的床帐,一歪身,就睡倒在床上。
袭人来找,见状忙让其喝茶醒酒。
图左侧:次日,宝钗将黛玉招至舍下,让她如实说来,昨日行酒令之词是从哪里来的?黛玉自知理亏,不觉红了脸。
}
\sun{p42-2}{潇湘子雅谑补馀音}{李纨派人请大家到稻香村去议事。
李纨笑道:“社还没起,就有脱滑的了,四丫头要告一年的假呢。
我请你们大家商议,给她多少日子的假。
”此事是因为贾母昨日要惜春画园子所引起。
黛玉却借机揶揄打诨,引得大家哄笑。
大家同意宝钗的意见,给了半年的假,并由宝钗口述,宝玉笔录,列了一张所需画具颜料等物清单。
}