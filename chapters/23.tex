\chapter{西厢记妙词通戏语\quad 牡丹亭艳曲警芳心}
\qi{群艳大观中,柳弱系轻风。
惜花与度曲,笑看利名空。
\zhu{
“柳弱系轻风”既是描写大观园中的自然风景,也是比喻众女儿的娇态旖旎,柳弱即柳嫩,
古代常用娇花嫩柳喻指少女。脂批即云宝钗和黛玉“一如娇花,一似纤柳”。后两句则是宝玉在大观园的生活状态。
}
}\par
话说贾元春自那日幸大观园回宫去后,便命将那日所有的题咏,命探春依次抄录妥协,自己编次,叙其优劣,又命在大观园勒石,\zhu{勒石:在石碑上刻字。
}为千古风流雅事。
因此,贾政命人各处选拔精工名匠,在大观园磨石镌字,贾珍率领蓉、萍等监工。
因贾蔷又管理着文官等十二个女戏并行头等事,不大得便,因此贾珍又将贾菖、贾菱唤来监工。
一日,烫蜡钉朱,\zhu{烫蜡钉朱:刻碑时的两道工序。
将熔化了的白蜡涂在已经用朱色写好文字的石碑面上,保护朱书,以免擦掉,叫做“烫蜡”。
烫蜡后石工按朱书镌刻,叫做“钉朱”。
}动起手来。
这也不在话下。
\par
且说那个玉皇庙并达摩庵两处,
\zhu{达摩:据传,南天竺人菩提达摩于五世纪初来中国传播禅法,建立了早期禅宗,被推为禅宗东土始祖。}
一班的十二个小沙弥并十二个小道士,如今挪出大观园来,贾政正想发到各庙去分住。
不想后街上住的贾芹之母周氏,正盘算着也要到贾政这边谋一个大小事务与儿子管管,也好弄些银钱使用,可巧听见这件事出来,便坐轿子来求凤姐。
凤姐因见他素日不大拿班作势的,
\zhu{拿班作势:装模作样,装腔作势。}
便依允了,想了几句话\geng{一派心机。
}便回王夫人说:“这些小和尚道士万不可打发到别处去,一时娘娘出来就要承应。
倘或散了,若再用时,可是又费事。
依我的主意,不如将他们竟送到咱们家庙里铁槛寺去,月间不过派一个人拿几两银子去买柴米就完了。
\ping{本回后文贾芹从银库领出三个月的供给总计“白花花二三百两”来,区区几两银子显然不够。}
说声用,走去叫来,一点儿不费事呢。
”王夫人听了,便商之于贾政。
贾政听了笑道:“倒是提醒了我,就是这样。
”即时唤贾琏来。
\par
当下贾琏正同凤姐吃饭,一闻呼唤,不知何事,放下饭便走。
凤姐一把拉住,笑道:“你且站住,听我说话。
若是别的事我不管,若是为小和尚们的事,好歹依我这么着。
”如此这般教了一套话。
贾琏笑道:“我不知道,你有本事你说去。
”凤姐听了,把头一梗,把筷子一放,\meng{活跳。
}腮上似笑不笑的瞅着贾琏道:“你当真的,是玩话?”贾琏笑道:“西廊下五嫂子的儿子芸儿来求了我两三遭,\zhu{廊下:建筑术语。
中国建筑史上自隋唐以来,府第、衙署、寺庙这一类多院落的大建筑群,四周皆以廊庑围绕,沿回廊两侧之街巷则称东西廊下(正房皆座北朝南,所以不会有南北廊下)。
宋以后回廊多以廊屋和围墙代替,形成今天四合院的面貌,但东西廊下之旧称仍沿用不变。
}\meng{发人一笑。
}要个事情管管。
我依了,叫他等着。
好容易出来这件事,你又夺了去。
”凤姐儿笑道:“你放心。
园子东北角子上,娘娘说了,还叫多多的种松柏树,楼底下还叫种些花草。
等这件事出来,我管保叫芸儿管这件工程。
”贾琏道:“果这样也罢了。
只是昨儿晚上,我不过是要改个样儿,你就扭手扭脚的。
”\geng{写凤姐风月之文如此,总不脱漏。
}凤姐儿听了,嗤的一声笑了,\geng{好章法!}\meng{粗蠢,情景可笑。
后将有大观园中一段奇情韵[事],不得不先为此等丑语一跌,以作未火先烟之象。
}向贾琏啐了一口,低下头便吃饭。
\par
贾琏已经笑着去了,到了前面见了贾政,果然是小和尚一事。
贾琏便依了凤姐主意,说道:“如今看来,芹儿倒大大的出息了,这件事竟交予他去管办。
横竖照在里头的规例,每月叫芹儿支领就是了。
”贾政原不大理论这些事,听贾琏如此说,便如此依了。
贾琏回到房中告诉凤姐儿,凤姐即命人去告诉了周氏。
贾芹便来见贾琏夫妻两个,感谢不尽。
凤姐又作情央贾琏先支三个月的,叫他写了领字,贾琏批票画了押,登时发了对牌出去。
银库上按数发出三个月的供给来,白花花二三百两。
贾芹随手拈一块,撂与掌平的人,\zhu{
清时用戥子(少数也有用天平的)称银子的重量,掌握戥子(或天平)的人即是掌平的。 
戥:音“等”,一种称量金银、药品等所用的小秤。
}叫他们吃了茶罢。
于是命小厮拿回家,与母亲商议。
登时雇了大叫驴,自己骑上,又雇了几辆车,至荣国府角门,唤出二十四个人来,坐上车,一径往城外铁槛寺去了。
当下无话。
\par
如今且说贾元春,因在宫中自编大观园题咏之后,忽想起那大观园中景致,自己幸过之后,贾政必定敬谨封锁,不敢使人进去骚扰,岂不寥落。
况家中现有几个能诗会赋的姊妹,何不命他们进去居住,也不使佳人落魄,花柳无颜。
\geng{韵人行韵事。
}却又想到宝玉自幼在姊妹丛中长大,\meng{何等精细!}不比别的兄弟,若不命他进去,只怕他冷清了,一时不大畅快,未免贾母、王夫人愁虑,须得也命他进园居住方妙。
\geng{大观园原系十二钗栖止之所,然工程浩大,故借元春之名而起,再用元春之命以安诸艳,不见一丝扭捻。
\zhu{扭捻:谓生硬编造。}
己卯冬夜。
}想毕,遂命太监夏守忠到荣国府来下一道谕,命宝钗等只管在园中居住,不可禁约封锢,命宝玉仍随进去读书。
\par
贾政、王夫人接了这谕,待夏守忠去后,便来回明贾母,遣人进去各处收拾打扫,安设帘幔床帐。
别人听了还自犹可,惟宝玉听了这谕,喜的无可不可。
\zhu{无可不可:犹言不知如何是好。
形容情绪激动至极。
形容感激、喜悦的样子。
}正和贾母盘算,要这个,弄那个,忽见丫鬟来说:“老爷叫宝玉。
”\geng{多大力量写此句。
余亦惊骇,况宝玉乎!回思十二三时,亦曾有是病来。
想时不再至,不禁泪下。
}宝玉听了,\meng{大家风范!}好似打了个焦雷,登时扫去兴头,脸上转了颜色,便拉着贾母扭的好似扭股儿糖,杀死不敢去。
贾母只得安慰他道:“好宝贝,你只管去,有我呢,他不敢委屈了你。
\meng{写尽祖母溺爱,作后文之本!}况且你又作了那篇好文章。
想是娘娘叫你进去住,他吩咐你几句,不过不教你在里头淘气。
他说什么,你只好生答应着就是了。
”一面安慰,一面唤了两个老嬷嬷来,吩咐:“好生带了宝玉去,别叫他老子唬着他。
”老嬷嬷答应了。
\par
宝玉只得前去,一步挪不了三寸,\ceng \zhu{\ceng:也作蹭,这里是行动缓慢、欲行又止的样子。
}\geng{\ceng,撑去声。
}到这边来。
可巧贾政在王夫人房中商议事情,金钏儿、彩云、彩霞、绣鸾、绣凤等众丫鬟都在廊檐底下站着呢,一见宝玉来,都抿着嘴笑。
金钏一把拉住宝玉,\geng{有是事,有是人。
}悄悄的笑道:“我这嘴上是才擦的香浸胭脂,\geng{活像活现。
}你这会子可吃不吃了?”\ping{暗透往事。
}彩云一把推开金钏,笑道:“人家正心里不自在,你还奚落他。
趁这会子喜欢,快进去罢。
”宝玉只得挨进门去。
原来贾政和王夫人都在里间呢。
赵姨娘打起帘子,宝玉躬身进去。
只见贾政和王夫人对面坐在炕上说话,地下一溜椅子,迎春、探春、惜春、贾环四个人都坐在那里。
一见他进来,惟有探春和惜春、贾环站了起来。
\ping{迎春是姐姐,见宝玉不需要站起来,而探春、惜春和贾环是宝玉的弟弟,则需要站起来。
}\par
贾政一举目,见宝玉站在跟前,神彩飘逸,秀色夺人,\geng{“消气散”用的好。
}看看贾环,人物委琐,
\zhu{委琐:现在一般写作“猥琐”。}
举止荒疏,忽又想起贾珠来,\geng{批至此,几乎失声哭出。
}再看看王夫人只有这一个亲生的儿子,素爱如珍,自己的胡须将已苍白:因这几件上,把素日嫌恶处分宝玉之心不觉减了八九。
\meng{为天下年老父母一哭!}半晌说道:“娘娘吩咐说,你日日外头嬉游,渐次疏懒,如今叫禁管,\geng{写宝玉可入园,用“禁管”二字,得体理之至。
壬午九月。
}同你姊妹在园里读书写字。
你可好生用心习学,再如不守分安常,你可仔细!”宝玉连连的答应了几个“是”。
王夫人便拉他在身旁坐下。
\meng{活现!}他姊弟三人依旧坐下。
\par
王夫人摸挲着宝玉的脖项说道:“前儿的丸药都吃完了?”宝玉答道:“还有一丸。
”王夫人道:“明儿再取十丸来,天天临睡的时候,叫袭人伏侍你吃了再睡。
”宝玉道:“只从太太吩咐了,
\zhu{只从:自从。}
袭人天天晚上想着,打发我吃。
”\geng{大家细细听去,活似小儿口气。
}贾政问道:“袭人是何人?”王夫人道:“是个丫头。
”贾政道:“丫头不管叫个什么罢了,是谁这样刁钻,起这样的名字?”王夫人见贾政不自在了,便替宝玉掩饰道:“是老太太起的。
”贾政道:“老太太如何知道这话,一定是宝玉。
”\ping{贾母不识字,证据在三十八回“(贾母)一面说,一面又看见柱上挂的黑漆嵌蚌的对子,命人念。
湘云念道……”}宝玉见瞒不过,只得起身回道:“因素日读诗,曾记古人有一句诗云:‘花气袭人知昼暖\foot{出宋陆游《村居书喜》诗。
“昼”,原作“骤”。
}’。
因这个丫头姓花,便随口起了这个名字。
”王夫人忙又道:“宝玉,你回去改了罢。
老爷也不用为这小事动气。
”贾政道:“究竟也无碍,又何用改。
\geng{几乎改去好名。
}只是可见宝玉不务正,专在这些秾词艳赋上作工夫。
”说毕,断喝一声:\geng{好收拾。
}“作业的畜生,\zhu{作业:作孽。
}还不出去!”王夫人也忙道:“去罢,只怕老太太等你吃饭呢。
”\meng{严父慈母,其事异,其行则一。
}宝玉答应了,慢慢的退出去,向金钏儿笑着伸伸舌头,带着两个嬷嬷一溜烟去了。
\par
刚至穿堂门前,\geng{妙!这便是凤姐扫雪拾玉之处,一丝不乱。
\ping{凤姐此时已不再有当年的威风,而是沦落到在园子里执帚“扫雪”的地步。}
}只见袭人倚门立在那里,\meng{何等牵连!}一见宝玉平安回来,堆下笑来问\geng{等坏了,愁坏了。
所以有“堆下笑来问”之话。
}道:“叫你作什么?”宝玉告诉他:“没有什么,不过怕我进园去淘气,吩咐吩咐。
”\geng{就说大话,毕肖之至!}一面说,一面回至贾母跟前,回明原委。
只见林黛玉正在那里,宝玉便问他:“你住那一处好?”林黛玉正心里盘算这事,\geng{颦儿亦有盘算事,拣择清幽处耳,未知择邻否?一笑。
}忽见宝玉问他,便笑道:“我心里想着潇湘馆好,爱那几竿竹子隐着一道曲栏,比别处更觉幽静。
”宝玉听了拍手笑道:“正和我的主意一样,我也要叫你住这里呢。
我就住怡红院,咱们两个又近,又都清幽。
”\geng{择邻出于玉兄,所谓真知己。
}
\meng{作后文无限张本。
\zhu{张本:预先为事态的发展作铺垫和准备;在前面为后文埋下的伏笔。}
}\par
二人正计较,就有贾政遣人来回贾母说:“二月二十二,日子好,哥儿姐儿们好搬进去的。
这几日内遣人进去分派收拾。
”薛宝钗住了蘅芜苑,林黛玉住了潇湘馆,贾迎春住了缀锦楼,探春住了秋爽斋,惜春住了蓼风轩,李氏住了稻香村,宝玉住了怡红院。
每一处添两个老嬷嬷,四个丫头,除各人奶娘亲随丫鬟不算外,另有专管收拾打扫的。
至二十二日,一齐进去,登时园内花招绣带,柳拂香风,\geng{八字写得满园之内处处有人,无一处不到。
}不似前番那等寂寞了。
\par
闲言少叙。
且说宝玉自进花园以来,心满意足,再无别项可生贪求之心。
每日只和姊妹丫头们一处,或读书,\geng{未必。
}或写字,或弹琴下棋,作画吟诗,以至描鸾刺凤,\geng{有之。
}斗草簪花,\zhu{斗草:又称“斗百草”,一种起源很古的民俗游戏,春夏花草繁茂之期,闺中多喜此戏,参加者各采花草竹木,举其名称作对,以吉祥而少见者为胜。
}低吟悄唱,拆字猜枚,\zhu{拆字:拆字是指将一文字,以笔画、字形等基本组成单位分解成多个文字,如“弓长张”、“木子李”等。
猜枚:又称猜拳、划拳、豁拳、拇战或酒拳,是一种酒席上的游戏,饮酒时常以此助兴,属酒令的一种。
最初是把席上的果品或棋子握在拳中,让人猜测其数目之多寡、单双或颜色,输了的要罚酒,是谓猜枚。
后来演化成一种猜手指数目的游戏,是谓豁拳。
豁字有张开义,那么字面上,豁拳就是张手出指的意思。
而“划拳”等都是豁拳的讹写。
传统豁拳的方法是:参与的两个人同时各出一手(一般为右手),靠握紧拳头或伸出不同数目的手指比出从零到五六种不同的数目,在出手的同时喊出一个口令,对应最小为〇最大为十的一个数。
若喊出的口令所代表的数目与两人所出的手指数的总和相同,即为猜中。
若两人皆猜错或皆猜中,则视为平手,继续出拳呼令,直到一方猜中一方猜错时,决出胜负。
猜错者要罚酒。
豁拳所呼的口令皆是一些以数字开头或跟数字有关的,象征友谊和讨吉利的熟语,例如“一条龙”、“哥俩好”、“五魁首”等。
}无所不至,倒也十分快乐。
他曾有几首即事诗,\zhu{即事诗:以眼前事物为题材的诗。
}虽不算好,却倒是真情真景,略记几首云:\par
\hop
春夜即事\par
霞绡云幄任铺陈,隔巷蟆更听未真。
\zhu{
绡:音“消”,轻软的丝织品。
幄:音“握”,帐幕。
霞绡云幄:彩色丝衾,轻纱帷帐。
蟆[má]更:即虾蟆[háma]更。明代郎瑛《七修续稿》辩证类“六更鼓”条引《蟫精隽》:“宋内五鼓绝,梆鼓遍作,谓之虾蟆更。其时禁门开而百官入,所谓六更也。”蟆更大作正是天将破晓的时候。虾蟆:即“蛤蟆”,青蛙和蟾蜍的统称。
“隔巷蟆更听未真”意谓更鼓之声从隔巷传来,听不真切。
}\par
枕上轻寒窗外雨,眼前春色梦中人。
\par
盈盈烛泪因谁泣,默默花愁为我嗔。
\par
自是小鬟娇懒惯,拥衾不耐笑言频。
\zhu{不耐:(诗人)拥衾不耐(小鬟)笑言频。
}\par
\hop
夏夜即事\par
倦绣佳人幽梦长,金笼鹦鹉唤茶汤。
\par
窗明麝月开宫镜,室霭檀云品御香。
\zhu{
麝月:月亮。
檀云:香云。
品:鉴赏。
本句说打开镜匣,好像明月映窗;御香弥漫,好像檀云绕室。
}\par
琥珀杯倾荷露滑,玻璃槛纳柳风凉。
\zhu{琥珀:黄褐色透明松脂化石,可作器皿饰物。
荷露:指酒,以花露为名。
滑:酒味醇美。
玻璃:一种石英类透明晶体,不同于今之玻璃。
槛[jiàn]:栏杆。
这里“荷露”“柳风”又都是夏天实景,可以引起荷翻露珠似倾杯、垂柳成行如栏杆的联想。
一说:这里的“麝月”、“檀云”、“琥珀”、“玻璃”又指贾府的四个丫头。
}\par
水亭处处齐纨动,帘卷朱楼罢晚妆。
\zhu{
齐纨:古代齐国出产的细绢。
齐纨动:意谓夏日女子所穿绢绸衫裙随风飘动。
又团扇多以细绢制成,也可解作纨扇摇动。
}\par
\hop
秋夜即事\par
绛芸轩里绝喧哗,桂魄流光浸茜纱。
\zhu{桂魄:月亮,传说月中有桂。
茜:音“欠”,草名,根可做红色染料,这里指大红色。
}\par
苔锁石纹容睡鹤,井飘桐露湿栖鸦。
\zhu{本句说纹痕上布满青苔的岩石可容仙鹤憩息,井边飘落沾满秋露的桐叶,沾湿了栖止在树上的乌鸦。
}\par
抱衾婢至舒金凤,倚槛人归落翠花。
\zhu{本句上句用《西厢记》第四本第一折红娘抱衾而至的故事。
舒:展。
金凤:绣有金凤图案的被褥。
下句写贵族女子兴尽人归卸下头饰。
槛[jiàn]:栏杆。
翠花:饰有翡翠珠玉的簪花。
}\par
静夜不眠因酒渴,沉烟重拨索烹茶。
\par
\hop
冬夜即事\par
梅魂竹梦已三更,锦罽鹴衾睡未成。
\zhu{
罽[jì]:毡子之类的毛织品。
锦罽:织有文彩的毛毯。
鹴:音“双”,即鹔鹴,雁的一种。
也是传说中五方神鸟之一。
鹴衾:指绣有鹔鹴图案花纹的被子。
}\par
松影一庭惟见鹤,梨花满地不闻莺。
\zhu{梨花:喻雪。
}\par
女儿翠袖诗怀冷,公子金貂酒力轻。
\zhu{
女儿翠袖诗怀冷:写冬夜严寒。意谓穿着翠袖衣衫、吟着诗句的女儿已觉怀冷。
金貂:黄色貂皮。
貂皮轻暖,十分珍贵。
公子金貂酒力轻:公子穿着貂裘还嫌酒力不足以御寒。
}\par
却喜侍儿知试茗,扫将新雪及时烹。
\zhu{试:尝。
}\par
\hop
\geng{四诗作尽安福尊荣之贵介公子也。
\zhu{贵介:尊贵,高贵。
}壬午孟夏。
}\par
\hop
因这几首诗,当时有一等势利人,见是荣国府十二三岁的公子作的,抄录出来各处称颂,再有一等轻浮子弟,爱上那风骚妖艳之句,也写在扇头壁上,\zhu{壁上:墙壁上。
}不时吟哦赏赞。
因此竟有人来寻诗觅字,倩画求题的。
\zhu{倩:音“庆”,请别人代自己做事。
}宝玉亦发得了意,镇日家作这些外务。
\zhu{家:一作“价”,语尾助词,无义。
}\par
谁想静中生烦恼,忽一日不自在起来,这也不好,那也不好,出来进去只是闷闷的。
园中那些人多半是女孩儿,正在混沌世界,天真烂漫之时,坐卧不避,嘻笑无心,那里知宝玉此时的心事。
那宝玉心内不自在,便懒在园内,只在外头鬼混,却又痴痴的。
\geng{不进园去,真不知何心事。
}茗烟见他这样,因想与他开心,左思右想,皆是宝玉顽奈烦了的,
\zhu{奈:通“耐”。}
不能开心,惟有这件,宝玉不曾看见过。
\geng{书房伴读累累如是,余至今痛恨。
}想毕,便走去到书坊内,把那古今小说并那飞燕、合德、武则天、杨贵妃的外传与那传奇角本买了许多来,
\zhu{
飞燕:赵飞燕,汉成帝的皇后。合德:汉成帝的皇后赵飞燕之妹。
角本:脚本,戏剧表演或电影、电视摄制等所依据的底本。
}
引宝玉看。
宝玉何曾见过这些书,一看见了便如得了珍宝。
茗烟嘱咐他不可拿进园去,\meng{自古恶奴坏事。
}“若叫人知道了,我就吃不了兜着走呢。
”宝玉那里舍的不拿进去,踟蹰再三,
\zhu{踟蹰[chíchú]:徘徊;来回走动。形容犹豫不定,要走又不走的样子。}
单把那文理细密的拣了几套进去,放在床顶上,无人时自己密看。
那粗俗过露的,都藏在外面书房里。
\ping{宝玉偷看那个时代的淫秽读物。
}\par
那一日,正当三月中浣,\zhu{中浣:指每月的中旬。
浣:音“幻”,洗涤。
唐代规定官吏们一个月中每十日休假一天,用来沐浴、洗涤。
一个月分为上浣、中浣、下浣。
后借作上旬、中旬、下旬的别称。
}早饭后,宝玉携了一套《会真记》,\zhu{《会真记》:即唐代元稹作的传奇小说《莺莺传》。
因文中有“会真”诗三十韵故又称《会真记》。
金、元人把其中的故事演为诸宫调和杂剧,名为《西厢记》。
这里是指元代王实甫的杂剧《西厢记》。
}走到沁芳闸桥边桃花底下一块石上坐着,展开《会真记》,从头细玩。
正看到“落红成阵”,只见一阵风过,把树头上桃花吹下一大半来,\geng{好一阵凑趣风。
}落的满身满书满地皆是。
宝玉要抖将下来,恐怕脚步践踏了,\geng{情不情。
}只得兜了那花瓣,来至池边,抖在池内。
那花瓣浮在水面,飘飘荡荡,竟流出沁芳闸去了。
\par
回来只见地下还有许多,宝玉正踟蹰间,只听背后有人说道:“你在这里作什么?”宝玉一回头,却是林黛玉来了,肩上担着花锄,锄上挂着花囊,手内拿着花帚。
\geng{一幅采芝图,非葬花图也。
}\geng{此图欲画之心久矣,誓不遇仙笔不写,恐亵我颦卿故也。
己卯冬。
}\geng{丁亥春间,偶识一浙省[新]发,\zhu{
解释一:笔误。靖本将“浙省发”更改为“浙省客”。
解释二:“一浙省发”即“一位浙江的举人”。
解释三:“省发”乃官员委任的一种制度,从下文“彼因官缘所缠”看,此人是往浙江赴任的官员。
}其白描美人,真神品物,甚合余意。
奈彼因宦缘所缠无暇,且不能久留都下,未几南行矣。
余至今耿耿,怅然之至。
恨与阿颦结一笔墨缘之难若此!叹叹!丁亥夏。
畸笏叟。
}\meng{真是韵人韵事!}\chen{写出扫花仙女。
}宝玉笑道:“好,好,来把这个花扫起来,\geng{如见如闻。
}撂在那水里。
我才撂了好些在那里呢。
”林黛玉道:“撂在水里不好。
你看这里的水干净,只一流出去,有人家的地方脏的臭的混倒,仍旧把花遭塌了。
\zhu{遭塌:同“糟蹋”。}
那畸角上我有一个花冢,
\zhu{畸角:即犄角,角落。}
\geng{好名色!新奇!葬花亭里埋花人。
}如今把他扫了,装在这绢袋里,拿土埋上,日久不过随土化了,\geng{宁使香魂随土化。
}岂不干净。
”\geng{写黛玉又胜宝玉十倍痴情。
}\ping{大观园内干净,园外脏臭,宁可葬于园内,也不陷入尘网。
葬花亦是葬人。
}\par
\chai{daiyu}{黛玉葬花}
宝玉听了喜不自禁,笑道:“待我放下书,帮你来收拾。
”\geng{顾了这头,忘却那头。
}黛玉道:“什么书?”宝玉见问,慌的藏之不迭,便说道:“不过是《中庸》、《大学》。
”黛玉笑道:“你又在我跟前弄鬼。
趁早儿给我瞧,好多着呢。
”宝玉道:“好妹妹,若论你,我是不怕的。
你看了,好歹别告诉别人去。
真真这是好书!你要看了,连饭也不想吃呢。
”一面说,一面递了过去。
林黛玉把花具且都放下,接书来瞧,从头看去,越看越爱看,不到一顿饭工夫,将十六出俱已看完,自觉词藻警人,馀香满口。
虽看完了书,却只管出神,心内还默默记诵。
\par
宝玉笑道:“妹妹,你说好不好?”林黛玉笑道:“果然有趣。
”宝玉笑道:“我就是个‘多愁多病身’,你就是那‘倾国倾城貌’。
”\zhu{倾国倾城貌:
倾:倾覆。
《汉书·孝武李夫人传》:“延年侍上起舞,歌曰:‘北方有佳人,绝世而独立。
一顾倾人城,再顾倾人国。
’”后常用“倾国倾城”形容女子的美貌。
《西厢记》第一本第四折,张生称自己是“多愁多病身”,莺莺是“倾国倾城貌”。
西厢记中张生和莺莺经历波折有情人终成眷属,宝玉自比张生,比黛玉为莺莺,将戏剧中的爱侣关系投射到现实中。
}
\geng{看官说宝玉忘情有之,若认作有心取笑,则看不得《石头记》。
}\chen{借用得妙!}
林黛玉听了,不觉带腮连耳通红,登时直竖起两道似蹙非蹙的眉,瞪了两只似睁非睁的眼,微腮带怒,薄面含嗔,指宝玉道:“你这该死的胡说!好好的把这淫词艳曲弄了来,还学了这些混话来欺负我。
我告诉舅舅舅母去。
”说到“欺负”两个字上,早又把眼睛圈儿红了,转身就走。
宝玉着了急,\geng{唬杀!急杀!}向前拦住说道:“好妹妹,千万饶我这一遭,原是我说错了。
若有心欺负你,明儿我掉在池子里,教个癞头鼋吞了去,变个大忘八,\zhu{癞头鼋、大忘八:鼋:音“元”,大鳖。
这里的大忘八指俗传能驮碑的大乌龟。
实为赑屃(音“币戏”),是传说中龙所生的怪物,似龟,好负重。
见《升庵外集》。
}等你明儿做了‘一品夫人’病老归西的时候,\geng{虽是混话一串,却成了最新最奇的妙文。
}我往你坟上替你驮一辈子的碑去。
”\chen{此誓新鲜。
}\ping{这可能也是宝黛结局的暗示吧。
}说的林黛玉嗤的一声笑了,\geng{看官想用何等话,令黛玉一笑收科?
\zhu{收科:收场;圆场。}
}揉着眼睛,一面笑道:“一般也唬的这个调儿,还只管胡说。
‘呸,原来是苗而不秀,是个银样鑞枪头’。
”\zhu{
镴[là]:锡和铅的合金,通称焊锡。
原来是苗而不秀,是个银样鑞枪头:即中看不中用的意思,语出《西厢记》第四本第二折。
苗而不秀:语见《论语·子罕》:“子曰:‘苗而不秀者有矣夫!’”意谓庄稼苗长了,却不秀穗。
喻才质秀美而早夭,没有什么成就。
后亦用以比喻虚有其表,其实无能。
银样鑞枪头,与此义近。
鑞是一种铅锡合金,色似银,亮而软。
}\chen{更借得妙!}宝玉听了,笑道:“你这个呢?我也告诉去。
”林黛玉笑道:“你说你会过目成诵,难道我就不能一目十行么?”\meng{儿女情态,毫无淫念,韵雅之至!}\par
宝玉一面收书,一面笑道:“正经快把花埋了罢,别提那个了。
”二人便收拾落花,正才掩埋妥协,只见袭人走来,说道:“那里没找到,摸在这里来。
那边大老爷身上不好,姑娘们都过去请安,老太太叫打发你去呢。
快回去换衣裳去罢。
”宝玉听了,忙拿了书,别了黛玉,同袭人回房换衣不提。
\chen{一语度下。
}\par


这里林黛玉见宝玉去了,又听见众姊妹也不在房,自己闷闷的。
\geng{有原故。
}正欲回房,刚走到梨香院墙角上,只听墙内笛韵悠扬,歌声婉转。
\geng{入正文方不牵强。
}林黛玉便知是那十二个女孩子演习戏文呢。
只是林黛玉素习不大喜看戏文,\geng{妙法!必云“不大喜看”。
}便不留心,只管往前走。
偶然两句吹到耳内,明明白白,一字不落,唱\geng{却一喜便总不忘,方见楔得紧。
\zhu{楔[xiē]:把楔形物插入或捶打到物体里面。楔得紧:用在这里应该是指情节安排紧凑,黛玉素来不喜戏文,却在此时因戏词而触动转变。}
}道是:“原来姹紫嫣红开遍,似这般都付与断井颓垣。
”\geng{情小姐故以情小姐词曲警之,恰极当极!己卯冬。
}林黛玉听了,倒也十分感慨缠绵,便止住步侧耳细听,又听唱道是:“良辰美景奈何天,赏心乐事谁家院。
”听了这两句,不觉点头自叹,心下自思道:“原来戏上也有好文章。
\geng{非不及钗,系不曾于杂学上用意也。
}可惜世人只知看戏,未必能领略这其中的趣味。
”\geng{将进门便是知音。
}想毕,又后悔不该胡想,耽误了听曲子。
又侧耳时,只听唱道:“则为你如花美眷,似水流年……”林黛玉听了这两句,不觉心动神摇。
又听道“你在幽闺自怜”等句,\zhu{“原来姹紫嫣红开遍……你在幽闺自怜”等句:见《牡丹亭·惊梦》。
前四句是杜丽娘唱词;后几句是柳梦梅唱词。
《牡丹亭》描写了大家闺秀杜丽娘和书生柳梦梅的生死之恋。
姹紫嫣红:指各色娇艳的花朵。
良辰美景:美好的时光和景物,这里是指美好的春光、春景。
赏心乐事:称心如意的事,这里是指向往爱情和美满婚姻的心事。
南朝谢灵运《拟魏太子邺中集诗序》:“天下良辰美景赏心乐事,四者难并”。
谁家:一说作“哪一家”解,“赏心乐事谁家院”意为自己家的庭院花园里没有赏心乐事;另一说作“甚么”解,“家”与“价”通,语尾助词,无义。“谁家院”即“谁价院”,与“奈何天”作对,意谓还成什么院落。
}亦发如醉如痴,站立不住,便一蹲身坐在一块山子石上,细嚼“如花美眷,似水流年”八个字的滋味。
忽又想起前日见古人诗中有“水流花谢两无情”之句,
\zhu{水流花谢两无情:出自唐代崔涂的《春夕旅怀》。}
再又有词中有“流水落花春去也,天上人间”之句,\zhu{“流水”二句:见南唐李煜《浪淘沙》词。
原词说好时光已如花落春去,相见之难犹如天上与人间之隔。
}
又兼方才所见《西厢记》中“花落水流红,闲愁万种”之句,都一时想起来,凑聚在一处。
仔细忖度,不觉心痛神痴,眼中落泪。
正没个开交,忽觉背上击了一下,及回头看时,原来是……且听下回分解。
正是:\par
妆晨绣夜心无矣,对月临风恨有之。
\par
\geng{前以《会真记》文,后以《牡丹亭》曲,加以有情有景消魂落魄诗词,总是急于令颦儿种病根也。
看其一路不即不离,曲曲折折写来,令观者亦自难持,况瘦怯怯之弱女乎!}\par
\qi{总评:诗童才女,添大观之颜色;埋花听曲,写灵慧之悠闲。
妒妇主谋,愚夫听命,
\zhu{这两句指贾芹和贾芸分别走凤姐和贾琏的后门,而凤姐把管小和尚的事给了贾芹,虽然也答应了将来让贾芸管种树种花的事,毕竟是贾琏作了让步。}
恶仆殷勤,淫词胎邪。
\zhu{这两句指宝玉的书童茗烟“把那古今小说并那飞燕、合德、武则天、杨贵妃的外传与那传奇脚本买了许多来,引宝玉看”。}
开楞严之密语,\zhu{楞严:即《楞严经》,阐明心性本体,属大乘秘密部,是佛家重要经典之一。
}阐法戒之真宗,\zhu{
“法”即梵语达摩,意思是通于一切。
“戒”则是梵语尸罗,意为防禁身心的放逸。
}以撞心之言,与石头讲道,悲夫!}
\dai{045}{宝黛共读西厢记}
\dai{046}{梨香院十二个女孩子演习戏文}
\sun{p23-1}{宝玉被召金钏调笑,共读西厢同葬桃花}{图左侧:
贾政在王夫人房中商议事情,叫宝玉来。众丫鬟都在廊檐底下站着呢,一见宝玉来,都抿着嘴笑。
金钏一把拉住宝玉,悄悄的笑道:“我这嘴上是才擦的香浸胭脂,你这会子可吃不吃了?”彩云一把推开金钏,笑道:“人家正心里不自在,你还奚落他。
趁这会子喜欢,快进去罢。
”宝玉只得挨进门去。
图右上:一日宝玉携了一套《西厢记》,来到沁芳闸旁桃花底下细看,一阵风吹过,桃花满地。
黛玉来了,肩上担着花锄,锄上挂着花囊,手内拿着花帚,道:“那我有一个花冢,如今把它扫了,装在绢袋里埋了,岂不干净。
”}