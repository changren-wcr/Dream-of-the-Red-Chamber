\chapter{情小妹耻情归地府 \quad 冷二郎一冷入空门}
\qi{余叹世人不识“情”字,常把“淫”字当作“情”字。
殊不知淫里无情,情里无淫,淫必伤情,情必戒淫,情断处淫生,淫断处情生。
三姐项下一横,是绝情,乃是正情;湘莲万根皆削,是无情,乃是至情。
生为情人,死为情鬼。
故结句曰“来自情天,去自情地”,岂非一篇尽情文字?再看他书,则全是“淫”不是“情”了。
}\par
话说鲍二家的走来打了兴儿一下子,笑道:“原有些真的,叫你又编了这混话,越发没了捆儿了。
\zhu{没了捆儿:没有拘束,信口乱说。
}你倒不像跟二爷的人,这些混话倒像是宝玉那边的了。
”\ji{好极之文,将茗烟等已全写出,可谓一击两鸣法,不写之写也。
}尤二姐才要又问,忽见尤三姐笑问道:“可是你们家那宝玉,除了上学,他作些什么?”\ji{拍案叫绝!此处方问,是何文情!\zhu{文情:文辞和情思。
}}\par
兴儿笑道:“姨娘别问他,说起来姨娘也未必信。
他长了这么大,独他没有上过正经学堂。
我们家从祖宗直到二爷,谁不是寒窗十载,偏他不喜读书。
老太太的宝贝,老爷先还管,如今也不敢管了。
成天家疯疯颠颠的,说的话人也不懂,干的事人也不知。
外头人人看着好清俊模样儿,心里自然是聪明的,谁知是外清而内浊,见了人,一句话也没有。
所有的好处,虽没上过学,倒难为他认得几个字。
每日也不习文,也不学武,又怕见人,只爱在丫头群里闹。
再者也没刚柔,\zhu{刚柔:偏指“刚”。
}有时见了我们,喜欢时没上没下,大家乱顽一阵;不喜欢各自走了,他也不理人。
我们坐着卧着,见了他也不理,他也不责备。
因此没人怕他,只管随便,都过的去。
”\par
尤三姐笑道:“主子宽了,你们又这样;严了,又抱怨。
可知难缠。
”\ji{情语,情文至语。
}尤二姐道:“我们看他倒好,原来这样。
可惜了一个好胎子。
”尤三姐道:“姐姐信他胡说,咱们也不是见一面两面的,行事言谈吃喝,原有些女儿气,那是只在里头惯了的。
若说糊涂,那些儿糊涂?姐姐记得,穿孝时咱们同在一处,那日正是和尚们进来绕棺,\zhu{绕棺:迷信习俗,人死后请和尚或道士绕着棺材念经以超度亡魂。
}
咱们都在那里站着,他只站在头里挡着人。
人说他不知礼,又没眼色。
过后他没悄悄的告诉咱们说:‘姐姐不知道,我并不是没眼色。
想和尚们脏,恐怕气味熏了姐姐们。
”
\ping{作者并不认为修行就一定是洁净的。}
接着他吃茶,姐姐又要茶,那个老婆子就拿了他的碗倒。
他赶忙说:“我吃脏了的,另洗了再拿来。
’这两件上,我冷眼看去,原来他在女孩子们前不管怎样都过的去,只不大合外人的式,所以他们不知道。
”尤二姐听说,笑道:“依你说,你两个已是情投意合了。
竟把你许了他,岂不好?”三姐见有兴儿,不便说话,只低头嗑瓜子。
\ping{为何尤三姐不予以反驳,反而害羞的低头,是不是真的对宝玉有意?尤三姐在上文确实也替宝玉说话。
}兴儿笑道:“若论模样儿行事为人,倒是一对好的。
只是他已有了,只未露形。
将来准是林姑娘定了的。
因林姑娘多病,二则都还小,故尚未及此。
再过三二年,老太太便一开言,那是再无不准的了。
”大家正说话,只见隆儿又来了,说:“老爷有事,是件机密大事,要遣二爷往平安州去。
不过三五日就起身,来回也得半月工夫。
今日不能来了。
请老奶奶早和二姨定了那事,
\zhu{那事:聘嫁尤三姐之事。}
明日爷来,好作定夺。
”说着,带了兴儿回去了。
\par
这里尤二姐命掩了门早睡,盘问他妹子一夜。
至次日午后,贾琏方来了。
尤二姐因劝他说:“既有正事,何必忙忙又来,千万别为我误事。
”贾琏道:“也没甚事,只是偏偏的又出来了一件远差。
出了月就起身,得半月工夫才来。
”尤二姐道:“既如此,你只管放心前去,这里一应不用你记挂。
三妹子他从不会朝更暮改的。
他已说了改悔,必是改悔的。
他已择定了人,你只要依他就是了。
”贾琏问是谁,尤二姐笑道:“这人此刻不在这里,不知多早才来,也难为他眼力。
自己说了,这人一年不来,他等一年;十年不来,等十年;若这人死了再不来了,他情愿剃了头当姑子去,吃长斋念佛,以了今生。
”贾琏问:“到底是谁,这样动他的心?”二姐笑道:“说来话长。
五年前我们老娘家里做生日,\zhu{老娘:外祖母。
}妈和我们到那里与老娘拜寿。
他家请了一起串客,\zhu{串客:也叫“票友”。
旧时戏曲、曲艺的业馀演员、乐师等的通称。
串:表演。
}里头有个作小生的叫作柳湘莲,\ji{千奇百怪之文何至于此!}他看上了,如今要是他才嫁。
旧年我们闻得柳湘莲惹了一个祸逃走了,不知可有来了不曾?”贾琏听了道:“怪道呢!我说是个什么样人,原来是他!果然眼力不错。
你不知道这柳二郎,那样一个标致人,最是冷面冷心的,差不多的人,都无情无义。
他最和宝玉合的来。
去年因打了薛呆子,他不好意思见我们的,不知那里去了一向。
\zhu{一向:一段时间。
}后来听见有人说来了,不知是真是假,一问宝玉的小子们就知道了。
倘或不来,他萍踪浪迹,知道几年才来,岂不白耽搁了?”尤二姐道:“我们这三丫头说的出来,干的出来,他怎样说,只依他便了。
”\par
二人正说之间,只见尤三姐走来说道:“姐夫,你只放心。
我们不是那心口两样人,说什么是什么。
若有了姓柳的来,我便嫁他。
从今日起,我吃斋念佛,只伏侍母亲,等他来了,嫁了他去,若一百年不来,我自己修行去了。
”说着,将一根玉簪,击作两段,“一句不真,就如这簪子!”说着,回房去了,真个竟非礼不动,非礼不言起来。
贾琏无了法,只得和二姐商议了一回家务,复回家与凤姐商议起身之事。
一面着人问茗烟,茗烟说:“竟不知道。
大约未来;若来了,必是我知道的。
”一面又问他的街坊,也说未来。
贾琏只得回复了二姐。
至起身之日已近,前两天便说起身,却先往二姐这边来住两夜,从这里再悄悄长行。
果见小妹竟又换了一个人,又见二姐持家勤慎,自是不消记挂。
\par
是日一早出城,就奔平安州大道,晓行夜住,渴饮饥餐。
方走了三日,那日正走之间,顶头来了一群驮子,内中一夥,\zhu{夥:同“伙”。
}主仆十来骑马,走的近来一看,不是别人,竟是薛蟠和柳湘莲来了。
贾琏深为奇怪,\ji{余亦为怪。
}忙伸马迎了上来,大家一齐相见,说些别后寒温,大家便入酒店歇下,叙谈叙谈。
贾琏因笑说:“闹过之后,我们忙着请你两个和解,谁知柳兄踪迹全无。
怎么你两个今日倒在一处了?”薛蟠笑道:“天下竟有这样奇事。
我同伙计贩了货物,自春天起身,往回里走,一路平安。
谁知前日到了平安州界,遇一伙强盗,已将东西劫去。
不想柳二弟从那边来了,方把贼人赶散,夺回货物,还救了我们的性命。
我谢他又不受,所以我们结拜了生死弟兄,如今一路进京。
从此后我们是亲弟亲兄一般。
到前面岔口上分路,他就分路往南二百里有他一个姑妈,他去望候望候。
我先进京去安置了我的事,然后给他寻一所宅子,寻一门好亲事,大家过起来。
”贾琏听了道:“原来如此,倒教我们悬了几日心。
”因又听道寻亲,又忙说道:“我正有一门好亲事堪配二弟。
”说着,便将自己娶尤氏,如今又要发嫁小姨一节说了出来,只不说尤三姐自择之语。
又嘱薛蟠且不可告诉家里,等生了儿子,自然是知道的。
\par
薛蟠听了大喜,说:“早该如此,这都是舍表妹之过。
”\zhu{舍表妹:指凤姐。
舍:对人谦称自己卑幼的亲属或亲戚。
}湘莲忙笑说:“你又忘情了,还不住口。
”薛蟠忙止住不语,便说:“既是这等,这门亲事定要做的。
”湘莲道:“我本有愿,定要一个绝色的女子。
如今既是贵昆仲高谊,\zhu{昆仲:对他人兄弟的敬称。
昆:兄。
仲:弟;第二。
}顾不得许多了,任凭裁夺,我无不从命。
”贾琏笑道:“如今口说无凭,等柳兄一见,便知我这内娣的品貌是古今有一无二的了。
”\zhu{内娣:妻妹。
}湘莲听了大喜,说:“既如此说,等弟探过姑娘,\zhu{姑娘:姑姑。
}不过月中就进京的,那时再定如何?”贾琏笑道:“你我一言为定,只是我信不过柳兄。
你乃是萍踪浪迹,倘然淹滞不归,岂不误了人家。
须得留一定礼。
”湘莲道:“大丈夫岂有失信之理。
小弟素系寒贫,况且客中,何能有定礼。
”薛蟠道:“我这里现成,就备一分二哥带去。
”贾琏笑道:“也不用金帛之礼,须是柳兄亲身自有之物,不论物之贵贱,不过我带去取信耳。
”湘莲道:“既如此说,弟无别物,此剑防身,不能解下。
囊中尚有一把鸳鸯剑,乃吾家传代之宝,弟也不敢擅用,只随身收藏而已。
贾兄请拿去为定。
\ping{
剑本是用于斩断东西的,曹雪芹却以利器为信物并冠以鸳鸯之名,其中的反讽意味十分明显。
}
弟纵系水流花落之性,然亦断不舍此剑者。
”说毕,解囊出剑捧与贾琏,贾琏命人收了。
大家又饮了几杯,方各自上马。
作别起程。
 正是:\par
\hop
将军不下马,各自奔前程。
\par
\hop
且说贾琏一日到了平安州,见了节度,完了公事。
因又嘱他十月前后务要还来一次,贾琏领命。
次日连忙取路回家,先到尤二姐处探望。
谁知贾琏出门之后,尤二姐操持家务十分谨肃,每日关门閤户,\zhu{閤:通“合”,闭。
}一点外事不闻。
他小妹子果是个斩钉截铁之人,每日侍奉母姊之馀,只安分守已,随分过活。
虽是夜晚间孤衾独枕,不惯寂寞,奈一心丢了众人,\zhu{奈:怎奈的省文。
}\ping{由此可见尤三姐之前确实有很多相好的情人,突然洗心革面,还有点不适应。
}只念柳湘莲早早回来完了终身大事。
这日贾琏进门,见了这般景况,喜之不尽,深念二姐之德。
大家叙些寒温之后,贾琏便将路上相遇湘莲一事说了出来,又将鸳鸯剑取出,递与三姐。
三姐看时,上面龙吞夔护,\zhu{
夔[kuí]:传说中像龙而只有一足的神兽,古代器物常雕其形状作文饰。
龙吞夔护:夔龙环抱的花纹。
这里用以形容剑柄和剑鞘上图案的古雅。
}珠宝晶莹,将靶一掣,\zhu{靶:柄。
}里面却是两把合体的。
一把上面錾着一“鸳”字,\zhu{錾(音“赞”):雕刻。
}
一把上面錾着一“鸯”字,冷飕飕,明亮亮,如两痕秋水一般。
三姐喜出望外,连忙收了,挂在自己绣房床上,每日望着剑,自笑终身有靠。
\ping{“每日望着剑”看起来就不祥。
}
贾琏住了两天,回去复了父命,回家合宅相见。
那时凤姐已大愈,出来理事行走了。
贾琏又将此事告诉了贾珍。
贾珍因近日又遇了新友,\ping{贾珍另有新欢,丢下了尤三姐。
}将这事丢过,不在心上,任凭贾琏裁夺,只怕贾琏独力不加,\zhu{不加:不能支撑。
}少不得又给了他三十两银子。
贾琏拿来交与二姐预备妆奁。
\par
谁知八月内湘莲方进了京,先来拜见薛姨妈,又遇见薛蝌,方知薛蟠不惯风霜,不服水土,一进京时便病倒在家,请医调治。
听见湘莲来了,请入卧室相见。
薛姨妈也不念旧事,只感新恩,母子们十分称谢。
又说起亲事一节,凡一应东西皆已妥当,只等择日。
柳湘莲也感激不尽。
\par
次日又来见宝玉,二人相会,如鱼得水。
湘莲因问贾琏偷娶二房之事,宝玉笑道:“我听见茗烟一干人说,我却未见,我也不敢多管。
我又听见茗烟说,琏二哥哥着实问你,不知有何话说?”湘莲就将路上所有之事一概告诉宝玉,宝玉笑道:“大喜,大喜!难得这个标致人,果然是个古今绝色,堪配你之为人。
”湘莲道:“既是这样,他那里少了人物,如何只想到我。
况且我又素日不甚和他厚,也关切不至此。
\zhu{关切:关系密切;关心,亲切。
}路上工夫忙忙的就那样再三要来定,难道女家反赶着男家不成。
\ping{在当时的时代背景下,女孩子要矜持。}
我自己疑惑起来,后悔不该留下这剑作定。
所以后来想起你来,可以细细问个底里才好。
”宝玉道:“你原是个精细人,如何既许了定礼又疑惑起来?你原说只要一个绝色的,如今既得了个绝色便罢了,何必再疑?”湘莲道:“你既不知他娶,如何又知是绝色?”宝玉道:“他是珍大嫂子的继母带来的两位小姨。
我在那里和他们混了一个月,怎么不知?真真一对尤物,\zhu{尤物:特异的人物,多指美女。
}\ji{可巧。
}他又姓尤。
”湘莲听了,跌足道:\zhu{跌足:跺脚。
}“这事不好,断乎做不得了。
你们东府里除了那两个石头狮子干净,只怕连猫儿狗儿都不干净。
我不做这剩忘八。
”\zhu{忘八:即“王八”,乌龟或鳖的俗称,骂人的话,指妻子有外遇的男人。
}\ji{极奇之文!极趣之文!《金瓶梅》中有云“把忘八的脸打绿了”,已奇之至,此云“剩忘八”,岂不更奇!}
\ping{
第四十七回:“那柳湘莲原是世家子弟,读书不成,父母早丧,素性爽侠,不拘细事,酷好耍枪舞剑,赌博吃酒,以至眠花卧柳,吹笛弹筝,无所不为。”
柳湘莲自己眠花卧柳,但是要求尤三姐历史清白,属于双重标准。
}
宝玉听说,红了脸。
湘莲自惭失言,连忙作揖说:“我该死胡说。
\ji{忽用湘莲提东府之事骂及宝玉,可是人想得到的?所谓“一个人不曾放过”。
}你好歹告诉我,他品行如何?”宝玉笑道:“你既深知,又来问我作甚么?连我也未必干净了。
”湘莲笑道:“原是我自己一时忘情,好歹别多心。
”宝玉笑道:“何必再提,这倒似有心了。
”湘莲作揖告辞出来,若去找薛蟠,一则他现卧病,二则他又浮躁,不如去索回定礼。
主意已定,便一径来找贾琏。
\par
贾琏正在新房中,闻得湘莲来了,喜之不禁,忙迎了出来,让到内室与尤老相见。
湘莲只作揖称老伯母,\zhu{伯母:用以称同学或朋友的妈妈。
}自称晚生,贾琏听了诧异。
吃茶之间,湘莲便说:“客中偶然忙促,谁知家姑母于四月间订了弟妇,使弟无言可回。
若从了老兄背了姑母,似非合理。
若系金帛之订,弟不敢索取,但此剑系祖父所遗,请仍赐回为幸。
”贾琏听了,便不自在,还说:“定者,定也。
原怕反悔所以为定。
岂有婚姻之事,出入随意的?还要斟酌。
”湘莲笑道:“虽如此说,弟愿领责领罚,然此事断不敢从命。
”贾琏还要饶舌,湘莲便起身说:“请兄外坐一叙,此处不便。
”那尤三姐在房明明听见。
好容易等了他来,今忽见反悔,便知他在贾府中得了消息,自然是嫌自己淫奔无耻之流,不屑为妻。
今若容他出去和贾琏说退亲,料那贾琏必无法可处,自己岂不无趣。
一听贾琏要同他出去,连忙摘下剑来,将一股雌锋隐在肘内,出来便说:“你们不必出去再议,还你的定礼。
”一面泪如雨下,左手将剑并鞘送与湘莲,右手回肘只往项上一横。
可怜——\par
\hop
揉碎桃花红满地,玉山倾倒再难扶。
\par
\zhu{红:比喻血。
玉山倾倒:语本《世说新语·容止》,这里用作身死倒地的婉辞。
玉山:形容仪容之美好。
}\par
\hop
芳灵蕙性,渺渺冥冥,不知那边去了。
当下唬得众人急救不迭。
尤老一面嚎哭,一面又骂湘莲。
贾琏忙揪住湘莲,命人捆了送官。
尤二姐忙止泪反劝贾琏:“你太多事,人家并没威逼他死,是他自寻短见。
你便送他到官,又有何益,反觉生事出丑。
不如放他去罢,岂不省事。
”贾琏此时也没了主意,便放了手命湘莲快去。
湘莲反不动身,泣道:“我并不知是这等刚烈贤妻,\ping{到底是“刚烈”还是绝望后的冲动呢?}可敬,可敬。
”湘莲反扶尸大哭一场。
等买了棺木,眼见入殓,又俯棺大哭一场,方告辞而去。
\par
出门无所之,昏昏默默,自想方才之事。
原来尤三姐这样标致,又这等刚烈,自悔不及。
正走之间,只见薛蟠的小厮寻他家去,那湘莲只管出神。
那小厮带他到新房之中,十分齐整。
忽听环珮叮当,尤三姐从外而入,一手捧着鸳鸯剑,一手捧着一卷册子,向柳湘莲泣道:“妾痴情待君五年矣,不期君果冷心冷面,妾以死报此痴情。
妾今奉警幻之命,前往太虚幻境修注案中所有一干情鬼。
妾不忍一别,故来一会,从此再不能相见矣。
”说着便走。
湘莲不舍,忙欲上来拉住问时,那尤三姐便说:“来自情天,去由情地。
前生误被情惑,今既耻情而觉,与君两无干涉。
”说毕,一阵香风,无踪无影去了。
\ping{尤三姐既然已经认识柳湘莲五年了,而且誓要嫁给他,为何还要和贾珍厮混呢?是玩累了打算让老实人接盘,抑或是洗心革面痛改前非,但是并没有得到宽恕容忍。
}\par
湘莲警觉,似梦非梦,睁眼看时,那里有薛家小童,也非新室,竟是一座破庙,旁边坐着一个跏腿道士捕虱。
\zhu{跏[jiā]:行走时脚向内拐。
如:跏子(瘸子,跛子)。
}湘莲便起身稽首相问:\zhu{稽首:音“起首”,一种俯首至地的最敬礼。
}“此系何方?仙师仙名法号?”道士笑道:“连我也不知道此系何方,我系何人,不过暂来歇足而已。
”柳湘莲听了,不觉冷然如寒冰侵骨,掣出那股雄剑,将万根烦恼丝一挥而尽,便随那道士,不知往那里去了。
后回便见。
\par
\ping{
脂本故事中尤三姐由淫转贞的形象前后反差过大,对此没有转变过程的描写。程本把尤三姐形象修改为从头到尾全贞全烈的女性,尤三姐自择柳湘莲,但柳湘莲以为她是不干净的女人,脾气刚烈的尤三姐无法容忍自己被柳湘莲冤枉误会,因此拔剑自刎。但是程本三姐从一开始就是无淫正情,恪守礼法,却在这样的情况下选择以淫戏的方式报复贾家兄弟,并且做出自择夫婿的选择,是存在矛盾的。
从《红楼梦》对“情”这一叙事主题来看,曹雪芹与程高对于“情”在态度上的区别对于异文走向产生影响。曹雪芹称此书“大旨谈情”,情是小说的叙事中心。而曹雪芹对过去才子佳人故事中错误情描写颇为不满,在《红楼梦》中他有意打破兼情兼淫的美学范式强调礼仪主义,以尤三姐表达作者对从良者的刚强态度以及对浊世之情的悲悯。相较之下,程本尤三姐故事以妇女贞操为叙事主题,展现出作者注重教化的封建道德观念,其思想高度远低于脂本。
}
\par
\qi{总评:尤三姐失身时,浓妆艳抹凌辱群凶;择夫后,念佛吃斋敬奉老母;能辨宝玉能识湘莲,活是红拂文君一流人物。
\zhu{红拂:唐代杜光庭《虬髯客传》(虬髯:音“求然”,蜷曲的须髯。
)的女主人公,姓张,初为隋朝大臣杨素的侍女,后私奔李靖。
她在杨家时手执红拂(掸灰尘的用具),见李靖时又自称“红拂妓”。
文君:汉代卓王孙的女儿.新寡后“私奔”文学家司马相如,结为夫妇。
}\hang
鸳鸯剑能斩鸳鸯,鸳鸯人能破鸳鸯,\zhu{鸳鸯人能破鸳鸯:应该是指第七十一回,鸳鸯偶然发现司棋和潘又安偷情。
}岂有此理?鸳鸯剑梦里不会杀奸妇,\zhu{鸳鸯剑梦里不会杀奸妇:第六十九回,尤二姐做梦,梦中死去的尤三姐劝二姐用鸳鸯剑斩妒妇王熙凤。
}鸳鸯人白日偏要助淫夫,\ping{此句令人费解。
可能指的是第七十二回,贾琏请求鸳鸯偷出老太太的金银用来抵押,获得维持贾府开销的周转资金,鸳鸯后来帮助了贾琏。
}焉有此情?真天地间不测的怪事!}
\dai{131}{尤三姐挂鸳鸯剑于床上,期盼嫁给柳湘莲}
\dai{132}{尤三姐自刎}
\sun{p65-1}{贾琏和尤二姐调情,尤三姐酒席斥骂贾琏贾珍}{图右侧:尤三姐和尤老娘后头去了,炕上只有尤二姐,贾琏忙上前问好相见。
贾琏讨尤二姐的槟榔吃,暗将自己的一个九龙佩让尤二姐收下。
图上侧:尤三姐站在炕上,指着贾琏嘲笑讥讽,淫词浪语一闹,二人反倒被镇住了,不知所措。
庚辰本中尤老娘和尤二姐都不在现场,而程甲本中却都在。
图左侧上部贾珍和贾琏的两马同槽。图左侧下部尤三姐表达誓嫁柳湘莲的志向。但是书中贾琏、尤三姐、尤二姐三人商议,而图中却有四个人。
}
\sun{p66-1}{贾琏路中为柳郎定亲}{图右侧:贾琏行到半路遇到了薛蟠和柳湘莲,便入一酒店叙谈。
贾琏夸赞尤三姐品貌,要介绍给柳湘莲为妻。
图左侧:可能是虚陪另一桌酒席,并没有在文中有实际的对应。
}
\sun{p66-2}{情小妹耻情归地府,冷二郎一冷入空门}{图右侧:柳湘莲想要退回订婚信物鸳鸯剑,尤三姐羞愤难当,自刎而死。
图中部:柳湘莲梦境中恍惚见尤三姐一手捧着鸳鸯剑,一手捧着一卷册子,向柳湘莲泣道。
图左侧:柳湘莲醒来发现自己在一座破庙,旁边坐着一个跏腿道士捕虱。
柳湘莲听他数句冷言,万念俱灰,断发出家。
}