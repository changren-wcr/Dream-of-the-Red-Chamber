\chapter{老学士闲征姽婳词 \quad 痴公子杜撰芙蓉诔}
\zhu{姽婳:音“鬼画”,形容女子娴静美好。
诔:音“累”三声,原为表彰死者德行、寄托生者哀思的文辞,仅能用于上对下。
后变成哀祭文体的一种。
}
\par
\qi{文有宾主,不可误。
此文以《芙蓉诔》为主,以《姽婳词》为宾,以宝玉古歌为主,以贾环贾兰诗绝为宾。
文有宾中宾,不可误。
以清客作序为宾,以宝玉出游作诗为宾中宾。
\zhu{在上一回有人请贾政寻秋赏桂花,贾政带宝玉、贾环、贾兰出游,本回才回来。}
由虚入实,可歌可咏。
}\par
话说两个尼姑领了芳官等去后,王夫人便往贾母处来省晨,见贾母喜欢,便趁便回道:“宝玉屋里有个晴雯,那个丫头也大了,而且一年之间,病不离身;我常见他比别人分外淘气,也懒;前日又病倒了十几天,叫大夫瞧,说是女儿痨,\zhu{痨:一种慢性消耗性传染病,类今之肺结核病,称“肺痨”,亦称“传尸痨”。
年轻女子患此病的叫“女儿痨”。
}所以我就赶着叫他下去了。
若养好了也不用叫他进来,就赏他家配人去也罢了。
再那几个学戏的女孩子,我也作主放出去了。
一则他们都会戏,口里没轻没重,只会混说,女孩儿们听了如何使得?二则他们既唱了会子戏,白放了他们,也是应该的。
况丫头们也太多,若说不够使,再挑上几个来也是一样。
”贾母听了,点头道:“这倒是正理,我也正想着如此呢。
但晴雯那丫头我看他甚好,怎么就这样起来。
我的意思,这些丫头的模样、爽利、言谈、针线多不及他,将来只他还可以给宝玉使唤得。
谁知变了。
”\par
王夫人笑道:“老太太挑中的人原不错。
只怕他命里没造化,所以得了这个病。
俗语又说:‘女大十八变。
’况且有本事的人,未免就有些调歪。
\zhu{调歪:依恃才能故意作难。
}老太太还有什么不曾经验过的。
三年前我也就留心这件事。
先只取中了他,我便留心。
冷眼看去,他色色虽比人强,只是不大沉重。
若说沉重知大礼,莫若袭人第一。
虽说贤妻美妾,然也要性情和顺、举止沉重的更好些。
就是袭人模样虽比晴雯略次一等,然放在房里,也算得一二等的了。
况且行事大方,心地老实,这几年来,从未逢迎着宝玉淘气。
凡宝玉十分胡闹的事,他只有死劝的。
因此品择了二年,一点不错了,我就悄悄的把他丫头的月分钱止住,我的月分银子里批出二两银子来给他。
不过使他自己知道,越发小心效好之意。
且不明说者,一则宝玉年纪尚小,老爷知道了又恐说耽误了书;二则宝玉再自为已是跟前的人不敢劝他说他,反倒纵性起来。
所以直到今日才回明老太太。
”\par
贾母听了,笑道:“原来这样,如此更好了。
袭人本来从小儿不言不语,我只说他是没嘴的葫芦。
既是你深知,岂有大错误的。
而且你这不明说与宝玉的主意更好。
且大家别提这事,只是心里知道罢了。
我深知宝玉将来也是个不听妻妾劝的。
我也解不过来,也从未见过这样的孩子。
别的淘气都是应该的,只他这种和丫头们好却是难懂。
我为此也耽心,每每的冷眼查看他。
只和丫头们闹,必是人大心大,知道男女的事了,所以爱亲近他们。
既细细查试,究竟不是为此。
岂不奇怪。
想必原是个丫头错投了胎不成。
”说着,大家笑了。
王夫人又回今日贾政如何夸奖,又如何带他们逛去,贾母听了,更加喜悦。
\par
一时,只见迎春妆扮了前来告辞过去。
凤姐也来省晨,伺候过早饭,又说笑了一回。
贾母歇晌后,王夫人便唤了凤姐,问他丸药可曾配来。
凤姐儿道:“还不曾呢,如今还是吃汤药。
太太只管放心,我已大好了。
”\geng{总是勉强。
}王夫人见他精神复初,也就信了。
\geng{只用此一句,便入后文。
}
因告诉撵逐晴雯等事,又说:“怎么宝丫头私自回家睡了,你们都不知道?我前儿顺路都查了一查。
谁知兰小子这一个新进来的奶子也十分的妖乔,\zhu{妖乔:妖冶轻佻的样子。也作“娇趫”
趫:音“乔”,行动轻捷,这里有举止轻浮的意思。
}我也不喜欢他。
我也说与你嫂子了,好不好叫他各自去罢。
况且兰小子也大了,用不着奶子了。
我因问你大嫂子:‘宝丫头出去难道你也不知道不成?’他说是告诉了他的,不过住两三日,等你姨妈好了就进来。
姨妈究竟没甚大病,不过还是咳嗽腰疼,年年是如此的。
他这去必有原故,敢是有人得罪了他不成?那孩子心重,亲戚们住一场,别得罪了人,反不好了。
”凤姐笑道:“谁可好好的得罪着他?况且他天天在园里,左不过是他们姊妹那一群人。
”王夫人道:“别是宝玉有嘴无心,傻子似的从没个忌讳,高兴了信嘴胡说也是有的。
”凤姐笑道:“这可是太太过于操心了。
若说他出去干正经事、说正经话去,却像个傻子;若只叫进来,在这些姊妹跟前,以至于大小的丫头们跟前,他最有尽让,\zhu{尽让:使别人占先,谦让。
}
又恐怕得罪了人,那是再不得有人恼他的。
我想薛妹妹此去,想必为着前时搜检众丫头的东西的原故。
他自然为信不及园里的人才搜检,他又是亲戚,现也有丫头老婆在内,我们又不好去搜检,恐我们疑他,所以多了这个心,自己回避了。
也是应该避嫌疑的。
”\par
王夫人听了这话不错,自己遂低头想了一想,便命人请了宝钗来分晰前日的事以解他疑心,又仍命他进来照旧居住。
宝钗陪笑道:“我原要早出去的,只是姨娘有许多的大事,所以不便来说。
可巧前日妈又不好了,家里两个靠得的女人也病着,我所以趁便出去了。
姨娘今日既已知道了,我正好明讲出情理来,就从今日辞了好搬东西的。
”王夫人凤姐都笑着:“你太固执了。
正经再搬进来为是,休为没要紧的事反疏远了亲戚。
”宝钗笑道:“这话说的太不解了,并没为什么事我出去。
我为的是妈近来神思比先大减,而且夜间晚上没有得靠的人,通共只我一个。
二则如今我哥哥眼看要娶嫂子,多少针线活计并家里一切动用的器皿,尚有未齐备的,我也须得帮着妈去料理料理。
姨妈和凤姐姐都知道我们家的事,不是我撒谎。
三则自我在园里,东南上小角门子就常开着,原是为我走的,保不住出入的人就图省路也从那里走,又没人盘查,设若从那里生出一件事来,岂不两碍脸面。
而且我进园里来住,原不是什么大事,因前几年年纪皆小,且家里没事,有在外头的,不如进来姊妹相共,或作针线,或顽笑,皆比在外头闷坐着好,如今彼此都大了,也彼此皆有事。
况姨娘这边历年皆遇不遂心的事故,那园子也太大,一时照顾不到,皆有关系,惟有少几个人,就可以少操些心。
所以今日不但我致意辞去之外,还要劝姨娘如今该减些的就减些,也不为失了大家的体统。
据我看,园里这一项费用也竟可以免的,说不得当日的话。
姨娘深知我家的,难道我们当日也是这样冷落不成。
”\ping{薛家已经力有不逮,无力支持,家业萧条。
}凤姐听了这篇话,便向王夫人笑道:“这话竟是,不必强了。
”王夫人点头道:“我也无可回答,只好随你便罢了。
”\par
话说之间,只见宝玉等已回来,因说他父亲还未散,“恐天黑了,所以先叫我们回来了。
”王夫人忙问:“今日可有丢了丑?”宝玉笑道:“不但不丢丑,倒拐了许多东西来。
”接着,就有老婆子们从二门上小厮手内接了东西来。
王夫人一看时,只见扇子三把,扇坠三个,笔墨共六匣,香珠三串,玉绦环三个。
宝玉说道:“这是梅翰林送的,那是杨侍郎送的,这是李员外送的,每人一分。
”说着,又向怀中取出一个旃檀香小护身佛来,\zhu{旃檀:音“沾谈”,梵语音译,即檀香。
护身佛:迷信的人带在身边的佛像,认为可以保佑自己。
}说:“这是庆国公单给我的。
”王夫人又问在席何人,作何诗词等语毕,只将宝玉一分令人拿着,同宝玉兰环前来见过贾母。
贾母看了,喜欢不尽,不免又问些话。
无奈宝玉一心记着晴雯,答应完了话时,便说骑马颠了,骨头疼。
贾母便说:“快回房去换了衣服,疏散疏散就好了,不许睡倒。
”宝玉听了,便忙入园来。
\par
当下麝月秋纹已带了两个丫头来等候,见宝玉辞了贾母出来,秋纹便将笔墨拿起来,一同随宝玉进园来。
宝玉满口里说“好热”,一壁走,一壁便摘冠解带,将外面的大衣服都脱下来麝月拿着,\geng{看他用智之处。
}
只穿着一件松花绫子夹袄,\zhu{松花:偏黑的深绿色。
夹袄:有面有里,中间不衬垫絮类的袄。
}袄内露出血点般大红裤子来。
秋纹见这条红裤是晴雯手内针线,因叹道:“这条裤子以后收了罢,真是物件在人去了。
”麝月忙也笑道:“这是晴雯的针线。
”又叹道:“真真物在人亡了!”秋纹将麝月拉了一把,笑道:“这裤子配着松花色袄儿、石青靴子,\zhu{石青:淡灰青色。
}越显出这靛青的头,\zhu{靛青:一种深蓝色的有机染料;深蓝色。
也称为“靛蓝”。
“靛青的头”:清代男子刚剃去周围头发的那种青头皮,证明贾宝玉是剃发垂辫,地道的满族发式。
}雪白的脸来了。
”宝玉在前只装听不见,又走了两步,便止步道:“我要走一走,这怎么好?”麝月道:“大白日里,还怕什么?还怕丢了你不成!”因命两个小丫头跟着,“我们送了这些东西去再来。
”宝玉道:“好姐姐,等一等我再去。
”麝月道:“我们去了就来。
两个人手里都有东西,倒像摆执事的,\zhu{执事:这里指仪仗,有时也指差事或当差的人。
}一个捧着文房四宝,一个捧着冠袍带履,成个什么样子。
”宝玉听见,正中心怀,便让他两个去了。
\ping{宝玉心中怀疑晴雯是在大丫鬟们之间的斗争中出局的,所以不想和她们一起。}
\par
他便带了两个小丫头到一石后,也不怎么样,只问他二人道:“自我去了,你袭人姐姐打发人瞧晴雯姐姐去了不曾?”这一个答道:“打发宋妈妈瞧去了。
”宝玉道:“回来说什么?”小丫头道:“回来说晴雯姐姐直着脖子叫了一夜,今日早起就闭了眼,住了口,世事不知,也出不得一声儿,只有倒气的分儿了。
”宝玉忙道:“一夜叫的是谁?”小丫头子说:“一夜叫的是娘。
”宝玉拭泪道:“还叫谁?”小丫头子道:“没有听见叫别人了。
”宝玉道:“你糊涂,想必没有听真。
”\par
旁边那一个小丫头最伶俐,听宝玉如此说,便上来说:“真个他糊涂。
”又向宝玉道:“不但我听得真切,我还亲自偷着看去的。
”宝玉听说,忙问:“你怎么又亲自看去?”小丫头道:“我因想晴雯姐姐素日与别人不同,待我们极好。
如今他虽受了委屈出去,我们不能别的法子救他,只亲去瞧瞧,也不枉素日疼我们一场。
就是人知道了回了太太,打我们一顿,也是愿受的。
所以我拚着挨一顿打,\zhu{拚:同“拼”,舍弃,不顾惜一切,豁出去。
}偷着下去瞧了一瞧。
谁知他平生为人聪明,至死不变。
他因想着那起俗人不可说话,所以只闭眼养神,见我去了便睁开眼,拉我的手问:‘宝玉那去了?’我告诉他实情。
他叹了一口气说:‘不能见了。
’我就说:‘姐姐何不等一等他回来见一面,岂不两完心愿?’他就笑道:‘你们还不知道。
我不是死,如今天上少了一位花神,玉皇敕命我去司主。
我如今在未正二刻到任司花,宝玉须待未正三刻才到家,
\zhu{未正二刻:下午两点半。未正三刻:下午两点四十五。}
只少得一刻的工夫,不能见面。
世上凡该死之人阎王勾取了过去,是差些小鬼来捉人魂魄。
若要迟延一时半刻,不过烧些纸钱浇些浆饭,那鬼只顾抢钱去了,该死的人就可多待些个工夫。
\geng{好,奇之至!又从来皆说“阎王注定三更死,谁敢留人至五更”之语,今忽借此小女儿一篇无稽之谈,反成无人敢翻之案,且又寓意调侃,骂尽世态。
岂非“[奇]之至”文章耶?寄语观者:至此\sout{一}[不]浮一大白者,\zhu{浮:罚人饮酒。
白:酒杯。
}已后不必看书也。
}我这如今是有天上的神仙来召请,岂可捱得时刻!’我听了这话,竟不大信,及进来到房里留神看时辰表时,果然是未正二刻他咽了气,正三刻上就有人来叫我们,说你来了。
这时候倒都对合。
”\par
宝玉忙道:“你不识字看书,所以不知道。
这原是有的,不但花有一个神,一样花有一位神之外还有总花神。
但他不知是作总花神去了,还是单管一样花的神?”这丫头听了,一时诌不出来。
恰好这是八月时节,园中池上芙蓉正开。
这丫头便见景生情,忙答道:“我也曾问他是管什么花的神,告诉我们日后也好供养的。
他说:‘天机不可泄漏。
你既这样虔诚,我只告诉你,你只可告诉宝玉一人。
除他之外若泄了天机,五雷就来轰顶的。
’他就告诉我说,他就是专管这芙蓉花的。
”
\ping{第六十三回怡红夜宴,黛玉抽到的花签恰好是芙蓉。}
宝玉听了这话,不但不为怪,亦且去悲而生喜,乃指芙蓉笑道:“此花也须得这样一个人去司掌。
我就料定他那样的人必有一番事业做的。
虽然超出苦海,从此不能相见,也免不得伤感思念。
”因又想:“虽然临终未见,如今且去灵前一拜,也算尽这五六年的情常。
”\zhu{情常:情分。
}\par
想毕忙至房中,又另穿戴了,只说去看黛玉,遂一人出园来,往前次之处去,意为停柩在内。
谁知他哥嫂见他一咽气便回了进去,希图早些得几两发送例银。
\zhu{发送:殡葬死者。
}王夫人闻知,便命赏了十两烧埋银子。
又命:“即刻送到外头焚化了罢。
女儿痨死的,断不可留!”他哥嫂听了这话,一面得银,一面就雇了人来入殓,抬往城外化人场上去了。
剩的衣履簪环,约有三四百金之数,他兄嫂自收了为后日之计。
二人将门锁上,一同送殡去未回。
宝玉走来扑了个空。
\geng{收拾晴雯,\sout{故}[固]为红颜一哭。
然亦大令人不堪。
}\geng{上云王夫人怕女儿痨不祥,今则忽从宝玉心中道其苦。
}\geng{又非模拟出,是己悒郁其词,
\zhu{悒[yì]郁:忧愁;愁闷。}
其母子之心中体贴眷爱之情,曲委已尽。
}
\ping{王夫人谎称晴雯死于“女儿痨”,为的是有理由作速处理晴雯丧事, 以免宝玉为之牵缠,所以有“母子之心中体贴眷爱之情”的批语。
而宝玉果不出其母所料,竟找了借口一个人跑出来想悼念晴雯。
}\par
宝玉自立了半天,别无法儿,只得复身进入园中。
待回至房中,甚觉无味,因乃顺路来找黛玉。
偏黛玉不在房中,问其何往,丫鬟们回说:“往宝姑娘那里去了。
”宝玉又至蘅芜苑中,只见寂静无人,房内搬的空空落落的,不觉吃一大惊。
忽见个老婆子走来,宝玉忙问这是什么原故。
老婆子道:“宝姑娘出去了。
这里交我们看着,还没有搬清楚。
我们帮着送了些东西去,这也就完了。
你老人家请出去罢,让我们扫扫灰尘。
也好,从此你老人家省跑这一处的腿子了。
”宝玉听了,怔了半天,因看着那院中的香藤异蔓,仍是翠翠青青,忽比昨日好似改作凄凉了一般,更又添了伤感。
默默出来,又见门外的一条翠樾埭上也半日无人来往,\zhu{樾:音“越”,树阴。
埭:音“代”,堤坝。
}不似当日各处房中丫鬟不约而来者络绎不绝。
又俯身看那埭下之水,仍是溶溶脉脉的流将过去。
心下因想:“天地间竟有这样无情的事!”悲感一番,忽又想到去了司棋、入画、芳官等五个;死了晴雯;今又去了宝钗等一处;迎春虽尚未去,然连日也不见回来。
且接连有媒人来求亲,大约园中之人不久都要散的了。
纵生烦恼,也无济于事。
不如还是找黛玉去相伴一日,回来还是和袭人厮混,只这两三个人,只怕还是同死同归的。
想毕,仍往潇湘馆来,偏黛玉尚未回来。
宝玉想亦当出去候送才是,无奈不忍悲感,还是不去的是,遂又垂头丧气的回来。
\par
正在不知所以之际,忽见王夫人的丫头进来找他说:“老爷回来了,找你呢,又得了好题目来了。
快走,快走。
”宝玉听了,只得跟了出来。
到王夫人房中,他父亲已出去了。
王夫人命人送宝玉至书房中。
\par
彼时贾政正与众幕友们谈论寻秋之胜,又说:“快散时忽然谈及一事,最是千古佳谈,‘风流隽逸,忠义慷慨’八字皆备,倒是个好题目,大家要作一首挽词。
”众幕宾听了,都忙请教系何等妙事。
贾政乃道:“当日曾有一位王,封曰恒王,出镇青州。
这恒王最喜女色,且公馀好武,因选了许多美女,日习武事。
每公馀辄开宴连日,令众美女习战斗攻拔之事。
其姬中有姓林行四者,姿色既冠,且武艺更精,皆呼为林四娘。
恒王最得意,遂超拔林四娘统辖诸姬,\zhu{林四娘:据清代陈维崧《妇人集》、王士祯《池北偶谈》和蒲松龄《聊斋志异》记载,她本是明代青州衡王府宫人。
}又呼为‘姽婳将军’。
”\zhu{姽婳:音“鬼画”,形容女子娴静美好。
}
\ping{“姽婳”谐音“鬼话”,可能是暗示贾政等人歌颂林四娘的虚伪可笑。}
众清客都称“妙极神奇。
竟以‘姽婳’下加‘将军’二字,反更觉妩媚风流,真绝世奇文也。
想这恒王也是千古第一风流人物了。
”贾政笑道:“这话自然是如此,但更有可奇可叹之事。
”众清客都愕然惊问道:“不知底下有何奇事?”贾政道:“谁知次年便有‘黄巾’‘赤眉’一干流贼馀党复又乌合,\zhu{黄巾:指东汉末年张角领导的农民起义军。
他们以黄巾裹头,故称“黄巾军”。
赤眉:指西汉末年樊崇领导的农民起义军。
他们以赤色染眉,因称“赤眉军”。
这里泛指农民起义军。
}抢掠山左一带。
\zhu{左:地理位置上,指面向南时的东边。
山左:太行山以东,即山东。
}\geng{妙!“赤眉”“黄巾”两时之事,今合而为一,盖云不过是此等众类,非特历历指明某赤某黄。
若云不合两用便呆矣。
此书全是如此,为混人也。
\zhu{为混人也:目的是为了使读者混淆年代。
即本书第一回:“然朝代年纪,地舆邦国,却反失落无考。
”}}恒王意为犬羊之恶,不足大举,因轻骑前剿。
不意贼众颇有诡谲智术,两战不胜,恒王遂为众贼所戮。
于是青州城内文武官员,各各皆谓:‘王尚不胜,你我何为!’遂将有献城之举。
林四娘得闻凶报,遂集聚众女将,发令说道:‘你我皆向蒙王恩,\zhu{向:从前,往昔。
}戴天履地,不能报其万一。
今王既殒身国事,我意亦当殒身于王。
尔等有愿随者,即时同我前往;有不愿者,亦早各散。
’众女将听他这样,都一齐说愿意。
于是林四娘带领众人连夜出城,直杀至贼营里头。
众贼不防,也被斩戮了几员首贼。
然后大家见是不过几个女人,料不能济事,遂回戈倒兵,奋力一阵,把林四娘等一个不曾留下,倒作成了这林四娘的一片忠义之志。
后来报至中都,自天子以至百官,无不惊骇道奇。
其后朝中自然又有人去剿灭,天兵一到,化为乌有,不必深论。
只就林四娘一节,众位听了,可羡不可羡呢?”众幕友都叹道:“实在可羡可奇,实是个妙题,原该大家挽一挽才是。
”说着,早有人取了笔砚,按贾政口中之言稍加改易了几个字,便成了一篇短序,递与贾政看了。
贾政道:“不过如此。
他们那里已有原序。
昨日因又奉恩旨,着察核前代以来应加褒奖而遗落未经请奏各项人等,无论僧尼乞丐与女妇人等,有一事可嘉,即行汇送履历至礼部备请恩奖。
所以他这原序也送往礼部去了。
大家听见这新闻,所以都要作一首《姽婳词》,\ping{宝玉奉父命作《姽婳词》纪念与自己毫无关系的人物,和宝玉纪念朝夕相伴的晴雯所写的《芙蓉女儿诔》,形成了对比.姽婳谐音“鬼话”,可能也是暗示了这是一篇言不由衷的应试作品。
}以志其忠义。
”众人听了,都又笑道:“这原该如此。
只是更可羡者,本朝皆系千古未有之旷典隆恩,实历代所不及处,可谓‘圣朝无阙事’,\zhu{圣朝无阙事:唐代岑参《寄左省杜拾遗》中的诗句。
阙:同“缺”,过失、错误。
这句是说贤明的朝廷是没有什么过失的。
}唐朝人预先竟说了,竟应在本朝。
如今年代方不虚此一句。
”贾政点头道:“正是。
”\par
说话间,贾环叔侄亦到。
贾政命他们看了题目。
他两个虽能诗,较腹中之虚实虽也去宝玉不远,但第一件他两个终是别路,若论举业一道,似高过宝玉,若论杂学,则远不能及;第二件他二人才思滞钝,不及宝玉空灵娟逸,每作诗亦如八股之法,未免拘板庸涩。
那宝玉虽不算是个读书人,然亏他天性聪敏,且素喜好些杂书,他自为古人中也有杜撰的,也有误失之处,拘较不得许多;若只管怕前怕后起来,纵堆砌成一篇,也觉得甚无趣味。
因心里怀着这个念头,每见一题,不拘难易,他便毫无费力之处,就如世上的流嘴滑舌之人,无风作有,信着伶口俐舌,长篇大论,胡扳乱扯,敷演出一篇话来。
虽无稽考,却都说得四座春风。
虽有正言厉语之人,亦不得压倒这一种风流去。
近日贾政年迈,名利大灰,然起初天性也是个诗酒放诞之人,因在子侄辈中,少不得规以正路。
近见宝玉虽不读书,竟颇能解此,细评起来,也还不算十分玷辱了祖宗。
就思及祖宗们,各各亦皆如此,虽有深精举业的,也不曾发迹过一个,看来此亦贾门之数。
况母亲溺爱,遂也不强以举业逼他了。
所以近日是这等待他。
又要环兰二人举业之馀,怎得亦同宝玉才好,所以每欲作诗,必将三人一齐唤来对作。
\geng{妙!世事皆不可无足餍,\zhu{餍:音“厌”,饱;满足。
}只有“读书”二字是万不可足餍的。
父母之心可不甚哉!近之父母只怕儿子不能名利,岂不可叹乎?}
\zhu{因程高本续书中贾宝玉学习时文八股求取功名,歪曲了曹雪芹的本意,为了不前后矛盾,这一段在程高本被删去。}
\par
闲言少述。
且说贾政又命他三人各吊一首,谁先成者赏,佳者额外加赏。
贾环贾兰二人近日当着多人皆作过几首了,胆量愈壮,今看了题,遂自去思索。
一时,贾兰先有了。
贾环生恐落后也就有了。
二人皆已录出,宝玉尚出神。
\geng{妙,偏写出钝态来。
}贾政与众人且看他二人的二首。
贾兰的是一首七言绝,写道是:\par
\hop
姽婳将军林四娘,玉为肌骨铁为肠,\par
捐躯自报恒王后,此日青州土亦香。
\par
\hop
众幕宾看了,便皆大赞:“小哥儿十三岁的人就如此,可知家学渊源,真不诬矣。
”贾政笑道:“稚子口角,也还难为他。
”又看贾环的,是首五言律,写道是:\par
\hop
红粉不知愁,将军意未休。
\par
掩啼离绣幕,抱恨出青州。
\par
自谓酬王德,讵能复寇仇。
\zhu{讵:音“句”,表示反问,相当于“难道”、“哪里”。
}\par
谁题忠义墓,千古独风流。
\par
\hop
众人道:“更佳。
倒是大几岁年纪,立意又自不同。
”贾政道:“还不甚大错,终不恳切。
”众人道:“这就罢了。
三爷才大不多两岁,在未冠之时如此,\zhu{未冠:古礼男子年二十而加冠。
故未满二十岁为“未冠”。
}用了工夫,再过几年,怕不是大阮小阮了。
”\zhu{大阮指三国时魏诗人阮籍,小阮指阮籍的侄子阮咸,均为“竹林七贤”之一。
}贾政笑道:“过奖了。
只是不肯读书过失。
”\par
因又问宝玉怎样。
众人道:“二爷细心镂刻,定又是风流悲感,不同此等的了。
”宝玉笑道:“这个题目似不称近体,须得古体,或歌或行,\zhu{近体:即近体诗,又名今体诗,律诗和绝句的通称。
近体诗在句数、字数和平仄、用韵等方面都有严格的规定。
古体:即古体诗,也称“古诗”、“古风”,在对仗、平仄、用韵方面较自由。
唐时律诗、绝句被称为“近体”以后,把唐以前的诗歌称为“古体”,并把采用这种古体写成的诗歌,也称为“古诗”或“古风”。
歌、行(行音“形”):都是乐府诗的体裁,或连称“歌行”。
}长篇一首,方能恳切。
”众人听了,都立身点头拍手道:“我说他立意不同!每一题到手,必先度其体格宜与不宜,这便是老手妙法。
就如裁衣一般,未下剪时,须度其身量。
这题目名曰《姽婳词》,且既有了序,此必是长篇歌行方合体的。
或拟白乐天《长恨歌》,\zhu{白乐天:白居易,字乐天。
}或拟咏古词,半叙半咏,流利飘逸,始能尽妙。
”贾政听说,也合了主意,遂自提笔向纸上要写,又向宝玉笑道:“如此,你念我写。
若不好了,我捶你那肉。
谁许你先大言不惭了!”宝玉只得念了一句,道是:\par
\hop
恒王好武兼好色,\par
\hop
贾政写了看时,摇头道:“粗鄙。
”一幕宾道:“要这样方古,究竟不粗。
且看他底下的。
”贾政道:“姑存之。
”宝玉又道:\par
\hop
遂教美女习骑射。
\par
秾歌艳舞不成欢,列阵挽戈为自得。
\par
\hop
贾政写出,众人都道:“只这第三句便古朴老健,极妙。
这四句平叙出,也最得体。
”贾政道:“休谬加奖誉,且看转的如何。
”宝玉念道:\par
\hop
眼前不见尘沙起,将军俏影红灯里。
\par
\hop
众人听了这两句,便都叫:“妙!好个‘不见尘沙起’!又承了一句‘俏影红灯里’,用字用句,皆入神化了。
”\zhu{神化:犹言出神入化,变化神妙。
}宝玉道:\par
\hop
叱吒时闻口舌香,\zhu{叱吒:音“斥乍”,指操练时的呼喊。
}霜矛雪剑娇难举。
\par
\hop
众人听了,便拍手笑道:“益发画出来了。
当日敢是宝公也在座,见其娇且闻其香否?不然,何体贴至此。
”宝玉笑道:“闺阁习武,任其勇悍,怎似男人。
\geng{贾老在座,故不便出“浊物”二字,妙甚细甚!}不待问而可知娇怯之形的了。
”贾政道:“还不快续,这又有你说嘴的了。
”宝玉只得又想了一想,念道:\par
\hop
丁香结子芙蓉绦,\zhu{本句意谓绣有芙蓉花样的绦带,结扎着丁香花式的结子。
}\par
\hop
众人都道:“转‘绦’,萧韵,\zhu{萧韵:即二萧韵。
诗韵中下平声第二部以“萧”字开头的韵目,有“萧”、“箫”、“挑”、“貂”等字。
}更妙,这才流利飘荡。
而且这一句也绮靡秀媚的妙。
”贾政写了,看道:“这一句不好。
已写过‘口舌香’‘娇难举’,何必又如此。
这是力量不加,故又用这些堆砌货来搪塞。
”宝玉笑道:“长歌也须得要些词藻点缀点缀,不然便觉萧索。
”贾政道:“你只顾用这些,但这一句底下如何能转至武事?若再多说两句,岂不蛇足了。
”\zhu{蛇足:比喻多余无用。
画蛇添足:战国时有楚人以比赛画蛇决定谁可以喝酒,有人画好后又添画四只脚,此时第二个人也已画成,第一个人则因为多画了根本不存在的蛇足,反而失去本已赢得的酒。
典出《战国策·齐策二》。
后比喻多此一举而于事无补。
}宝玉道:“如此,底下一句转煞住,想亦可矣。
”贾政冷笑道:“你有多大本领?上头说了一句大开门的散话,如今又要一句连转带煞,岂不心有馀而力不足些。
”宝玉听了,垂头想了一想,说了一句道:\par
\hop
不系明珠系宝刀。
\par
\hop
忙问:“这一句可还使得?”众人拍案叫绝。
贾政写了,看着笑道:“且放着,再续。
”宝玉道:“若使得,我便要一气下去了。
若使不得,越性涂了,我再想别的意思出来,再另措词。
”贾政听了,便喝道:“多话!不好了再作,便作十篇百篇,还怕辛苦了不成!”宝玉听说,只得想了一会,便念道:\par
\hop
战罢夜阑心力怯,脂痕粉渍污鲛鮹。
\zhu{鲛鮹:音“交消”,传说南海中有鲛人,即人鱼,能织绡,后用以泛称薄纱。
}\par
\hop
贾政道:“又一段。
底下怎样?”宝玉道:\par
\hop
明年流寇走山东,强吞虎豹势如蜂。
\par
\hop
众人道:“好个‘走’字!便见得高低了。
且通句转的也不板。
”宝玉又念道:\par
\hop
王率天兵思剿灭,一战再战不成功。
\par
腥风吹折陇头麦,日照旌旗虎帐空。
\zhu{虎帐:古代元帅发号施令的营帐。
}\par
青山寂寂水澌澌,正是恒王战死时。
\par
雨淋白骨血染草,月冷黄沙鬼守尸。
\par
\hop
众人都道:“妙极,妙极!布置、叙事、词藻,无不尽美。
且看如何至四娘,必另有妙转奇句。
”宝玉又念道:\par
\hop
纷纷将士只保身,青州眼见皆灰尘,\par
不期忠义明闺阁,愤起恒王得意人。
\par
\hop
众人都道:“铺叙得委婉。
”贾政道:“太多了,底下只怕累赘呢。
”宝玉乃又念道:\par
\hop
恒王得意数谁行,\zhu{行:音“航”,次第、等辈。
这句的意思是:恒王最宠爱的人数谁呢?}就死将军林四娘,\par
号令秦姬驱赵女,\zhu{秦姬赵女:秦和赵是战国时代的两个国家,相传这两地多出美女,后用“秦姬赵女”作为美貌女子的代称,这里泛指恒王的姬妾。
}艳李秾桃临战场。
\par
绣鞍有泪春愁重,铁甲无声夜气凉。
\par
胜负自然难预定,誓盟生死报前王。
\par
贼势猖獗不可敌,柳折花残实可伤,\par
魂依城郭家乡近,马践胭脂骨髓香。
\par
星驰时报入京师,谁家儿女不伤悲!\par
天子惊慌恨失守,此时文武皆垂首。
\par
何事文武立朝纲,不及闺中林四娘!\par
我为四娘长太息,歌成馀意尚徬徨。
\par
\hop
念毕,众人都大赞不止,又都从头看了一遍。
贾政笑道:“虽然说了几句,到底不大恳切。
”因说:“去罢。
”三人如得了赦的一般,一齐出来,各自回房。
\par
\ping{
在《姽婳词》中他以当今皇帝褒奖前代所遗落的可嘉人事为名,指桑骂槐,揭露和嘲笑当朝统治者的昏庸腐朽和外强中干的虚弱本质:“天子惊慌恨失守,此时文武皆垂首。何事文武立朝纲,不及闺中林四娘!”这无疑是大胆的。但是,把封建王朝在农民起义风暴的猛烈扫荡下的土崩瓦解看成是一场灾难,这又说明曹雪芹并没有完全背叛自己的阶级。
《姽婳词》这段情节,在小说描述晴雯之死的过程中是强行插入的,给人以一种节外生枝的感觉。宝玉吊晴雯扑了空回来,就被叫去做吊林四娘的诗,紧接着就让他撰写《芙蓉女儿诔》,这一切都显然是有用意的,把一个以生命去酬答平日恩宠的贵族姬妾与一个遭封建势力迫害而死的女奴放在一起写,表明了曹雪芹思想中所存在的深刻矛盾。
}
\par
众人皆无别话,不过至晚安歇而已。
独有宝玉一心凄楚,回至园中,猛然见池上芙蓉,想起小丫鬟说晴雯作了芙蓉之神,不觉又喜欢起来,乃看着芙蓉嗟叹了一会。
忽又想起死后并未到灵前一祭,如今何不在芙蓉前一祭,岂不尽了礼,比俗人去灵前祭吊又更觉别致。
想毕,便欲行礼。
忽又止住道:“虽如此,亦不可太草率,也须得衣冠整齐,奠仪周备,方为诚敬。
”想了一想,“如今若学那世俗之奠礼,断然不可;竟也还别开生面,另立排场,风流奇异,于世无涉,方不负我二人之为人。
况且古人有云:‘潢污行潦,\zhu{潢(音“黄”)污:坑中的死水。
行潦(潦音“老”):车辙中的流水。
}蘋蘩薀藻之贱,\zhu{蘋:音“贫”,浮萍。
蘩:音“烦”,白蒿。
薀藻:音“运藻”,水草。
}可以羞王公,荐鬼神。
\zhu{羞:奉献。
荐:呈献。
}’\zhu{潢污……鬼神:句出《左传》隐公三年。
意谓只要胸怀诚意,即使是坑中的积水和野生的水草,也可以奉献王公,祭奠鬼神。
}
原不在物之贵贱,全在心之诚敬而已。
此其一也。
二则诔文挽词也须另出己见,\zhu{诔:音“累”三声,原为表彰死者德行、寄托生者哀思的文辞,仅能用于上对下。
后变成哀祭文体的一种。
}自放手眼,亦不可蹈袭前人的套头,填写几字搪塞耳目之文,亦必须洒泪泣血,一字一咽,一句一啼,宁使文不足悲有馀,万不可尚文藻而反失悲戚。
况且古人多有微词,\zhu{微词:真意隐微不显,另有寄托之词语。
}非自我今作俑也。
\zhu{俑:音“永”,古代陪葬用的偶人。
《孟子·梁惠王上》:“始作俑者,其无后乎”。
意谓首开以俑陪葬先例的人,一定断子绝孙。
后以“作俑”为首创先例的意思。
}奈今人全惑于功名二字,尚古之风一洗皆尽,恐不合时宜,于功名有碍之故。
我又不希罕那功名,不为世人观阅称赞,何必不远师楚人之《大言》、《招魂》、《离骚》、《九辩》、《枯树》、《问难》、《秋水》、《大人先生传》等法,\zhu{《大言》即《大言赋》,它和《招魂》、《九辩》均为楚国诗人宋玉所作(一说《招魂》为屈原作)。
《离骚》为楚国诗人屈原的作品。
《枯树》即《枯树赋》,作者北周诗人庾信。
《问难》:不详,疑指汉代东方朔的《答客难》或扬雄的《解难》。
《秋水》:《庄子》中的篇名。
《大人先生传》:三国魏诗人阮籍的作品。
}或杂参单句,\zhu{杂参:音“杂餐”,即“参杂”,混合杂入。
}或偶成短联,或用实典,或设譬寓,随意所之,\zhu{之:到……去。
}信笔而去,喜则以文为戏,悲则以言志痛,辞达意尽为止,何必若世俗之拘拘于方寸之间哉。
”\zhu{拘拘于方寸之间:指拘于旧的格式,不敢纵情抒写。
方寸:一寸见方。
这里喻旧格式。
}宝玉本是个不读书之人,再心中有了这篇歪意,怎得有好诗文作出来。
他自己却任意纂著,并不为人知慕,所以大肆妄诞,竟杜撰成一篇长文,用晴雯素日所喜之冰鲛縠一幅楷字写成,\zhu{冰鲛[jiāo]:传说鲛人居南海中,如鱼,滴泪成珠,善机织。所织之绡,明洁如冰,暑天令人凉快,以此命名。
縠:音“胡”,有皱纹的纱。
}名曰《芙蓉女儿诔》,\zhu{《芙蓉女儿诔》:为芙蓉女儿写的祭文。
诔:音“累”三声,原为表彰死者德行、寄托生者哀思的文辞,仅能用于上对下。
后变成哀祭文体的一种。
}前序后歌。
又备了四样晴雯所喜之物,于是夜月下,命那小丫头捧至芙蓉花前。
先行礼毕,将那诔文即挂于芙蓉枝上,乃泣涕念曰:\geng{诸君阅至此,只当一笑话看去,便可醒倦\foot{按:“诸君阅至此,只当一笑话看去,便可醒倦。
”此评原误入正文。
}。
\zhu{
批书人是封建等级制度卫道士,他认为偌大一个贾府,一个小小的丫头病死了,一位堂堂少爷竟然如此郑重其事写出一篇如此庄严的祭文,把一个丫头美化成天神,岂非大笑话吗?
}
}\par
\hop
维太平不易之元,\zhu{维太平不易之元:维:语助词,无义,常用于语首。
太平不易:本为“永远太平”的意思,在这里意含讥贬。
元:纪年。
旧时的诔、祭、哀、书等文体,往往用“维年月日”这种固定格式作为开头。
本书作者曾托言故事“无朝代年纪可考”,所以用了这样一句既符合诔文格式,又似乎不“干涉时世”的话。
}\geng{年便奇。
}蓉桂竞芳之月,\geng{是八月。
}无可奈何之日,\geng{日更奇。
细思日何难于直说某某,今偏用如此说,则可知矣。
}
怡红院浊玉,\geng{自谦得更奇。
盖常以“浊”字评天下之男子,竟自谓,所谓“以责人之心责己”矣。
}谨以群花之蕊,\geng{奇香。
}冰鲛之縠,\zhu{
冰鲛[jiāo]:传说鲛人居南海中,如鱼,滴泪成珠,善机织。所织之绡,明洁如冰,暑天令人凉快,以此命名。
縠:音“胡”,有皱纹的纱。
}\geng{奇帛。
}沁芳之泉,\geng{奇奠。
}\zhu{沁芳:第十七回,大观园试才题对额,宝玉对亭子的命名。
}枫露之茗:\geng{奇茗。
}\zhu{枫露茶:第八回,宝玉因奶妈喝了枫露茶,发脾气打碎茶杯。
}四者虽微,聊以达诚申信,乃致祭于白帝宫中抚司秋艳芙蓉女儿\geng{奇称。
}之前曰:\zhu{白帝:古人以百物配五行(金、木、水、火、土)。
如春天属木,其味为酸,其色为青,司时之神就叫青帝;秋天属金,其味为辛,其色为白,司时之神就叫白帝,等等。
}\par
窃思女儿自临浊世,\geng{世不浊,因物所混而浊也,前后便有照应。
\zhu{这里的“物”应该是照应前文宝玉对男子“浊物”的评价,浊世是由于“浊物”即男子导致的。
}
}\geng{“女儿”称妙!盖思普天下之称断不能有如此二字之清洁者。
亦是宝玉之真心。
}
迄今凡十有六载。
\geng{方十六岁而夭,亦伤矣。
}其先之乡籍姓氏,湮沦而莫能考者久矣。
\zhu{湮:音“烟”,埋没,不被人所知道。
}\geng{忽又有此文。
不可\sout{后}[考何]来,亦可伤矣。
}而玉得于衾枕栉沐之间,\zhu{栉:音“志”,梳子、篦子的通称,引申为梳头。
栉沐:梳洗。
}栖息宴游之夕,亲昵狎亵,相与共处者,仅五年八月有奇。
\zhu{奇:音“机”,零数。
}
\geng{相共不足六载,一旦夭别,岂不可伤!}忆女儿曩生之昔,\zhu{曩:音“攮”,从前;过去。
}其为质则金玉不足喻其贵,其为性则冰雪不足喻其洁,其为神则星日不足喻其精,其为貌则花月不足喻其色。
姊妹悉慕媖娴,\zhu{媖娴:音“英闲”,美好;文静。
}妪媪咸仰惠德。
\zhu{媪:音“奥”三声,妪音“玉”,都是对年老妇女的称呼,也作为妇女的通称。
}孰料鸠鸩恶其高,\zhu{
鸠[jiū]:斑鸠,爱鸣叫,这里用来比喻多嘴多舌好进谗言的人。
鸩[zhèn]:传说中的一种恶鸟,羽毛有毒,能致人死命。
}鹰鸷翻遭罦罬;\zhu{鹰鸷:鸷音“至”,指鹰鹞等飞翔高空的猛禽。
罦罬[fúzhuó]:一种装有机关能捕捉鸟兽的网,又叫复车网。
}\zhu{孰料……罦罬:意谓因鸠鸩一般的恶人讨厌晴雯的高洁,遂使鹰鸷一样的晴雯反而遭到陷害。
屈原《离骚》:“鸷鸟之不群兮,自前世而固然。
何方圜之能周兮,夫孰异道而相安。
……吾令鸠为媒兮,鸠告余以不好。
雄鸠之鸣逝兮,余又恶其佻巧。
”}\geng{《离骚》:“鸷鸟之不群兮。
”又:“吾令鸩为媒兮,鸩告余以不好。
雄鸠之鸣逝兮,余犹恶其佻巧。
”注:鸷特立不群,故不群,故不于。
\zhu{于:犹“与”。与:交往。}
鸩羽毒杀人。
鸠多声,有如人之多言不实。
罦罬,音孚拙。
翻车网。
《诗经》:“雉离于罦。
”《尔雅》:“罬谓之罦。
”}\ping{宝玉对侍妾的爱也是有数的,身边的人为了追求这有限的爱也要你死我活,宝玉大概是永远无法理解的。
但是他因晴雯之事开始提防、愤怒,一定程度上也是回到现实。
}薋葹妒其臭,\zhu{薋:音“瓷”,蒺藜[jílí]。
葹:音“施”,苍耳。
古人认为这两种都是恶草,常常拿它们比喻坏人。
这里比喻嫉妒晴雯的恶人。
臭:音“秀”,气味,这里指香气。
}茝兰竟被芟鉏!\zhu{茝(音“柴”三声)兰:两种香草,多用以喻贤人,这里喻晴雯。
芟鉏:音“杉除”,除掉。
芟:用镰刀割草。
鉏:同“锄”。
}\geng{《离骚》:薋、葹皆恶草,以辨邪佞。
茝兰,芳草,以别君子。
}花原自怯,岂奈狂飙;柳本多愁,何禁骤雨。
偶遭蛊虿之谗,\zhu{蛊:音“古”,毒虫。
《本草纲目·
虫部四》李时珍集解引陈藏器曰:“取百虫入瓮中.经年开之,必有一虫食尽诸虫,即此名为蛊”。
虿[chài]:蝎子一类的毒虫。
蛊虿:都是害人的毒虫。
}遂抱膏肓之疚。
\zhu{疚:久病;忧苦,心内痛苦。
}故尔樱唇红褪,韵吐呻吟;杏脸香枯,色陈顑颔。
\zhu{顑颔:音“喊旱”,因饥饿而面黄肌瘦。
这里形容晴雯因疾病而面色憔悴。
}\geng{《离骚》:“长顑颔亦何伤。
”面黄色。
}诼谣謑诟,\zhu{诼:音“浊”,谗言、谣言。
谣:凭空虚构没有根据的话或传闻。
謑诟:音“希够”,嘲讽辱骂。
}出自屏帏;\zhu{屏帏:用布做成的帐幕,可以分隔内外。
}
荆棘蓬榛,蔓延户牖。
\zhu{户:门。
牖:音“友”,窗户。
}岂招尤则替,\zhu{尤:罪过,过错;指责,归罪。
招尤则替:自招过失而受损害。
}实攘诟而终。
\zhu{攘[rǎng]:容忍。
诟[gòu]:耻辱。
攘诟:蒙受耻辱。
}\geng{《离骚》:“謇朝谇而夕替。
”废也。
\zhu{“废也”是对于“替”的解释。
}“忍尤而攘诟。
”攘,取也。
}既忳幽沉于不尽,\zhu{忳:音“屯”,忧郁。
《离骚》:“忳郁邑余侘傺兮”。
幽沉:指隐藏在内心深处的怨恨。
}复含罔屈于无穷。
\zhu{罔屈:冤屈。
不直叫罔。
}高标见嫉,闺帏恨比长沙;\zhu{长沙:指贾谊,西汉洛阳人,汉文帝时官至大中大夫,遭谗被贬为长沙王太傅,故称贾长沙。
后又迁梁怀王太傅。
梁怀王坠马死,贾谊认为自己未能尽到职责,常常哭泣,年馀亦死。
见《史记·屈原贾生列传》。
这里借贾谊受屈遭贬喻晴雯因诬被逐。
}\geng{汲黯辈嫉贾谊之才,谪贬长沙。
}直烈遭危,巾帼惨于羽野。
\zhu{直烈……羽野:意谓晴雯同鲧(音“滚”)一样正直刚烈,结局却比鲧更惨。
《离骚》:“鲧婞直以亡身兮,\zhu{婞[xìng]:倔强。}终然夭乎羽之野。
”直烈:正直刚烈。
羽野:传说中的羽山的荒野。
《山海经·海内经》:“洪水滔天。
鲧窃帝之息壤以堙洪水,不待帝命。
帝令祝融杀鲧于羽郊。
”禹的父亲鲧没有天帝的命令,就擅自拿息壤(一种可以生长不息的神土,能堵塞洪水)治洪水,天帝就叫祝融将他杀死在羽山的荒野。
}\geng{鲧刚直自命,舜殛于羽山。
\zhu{殛:音“集”,杀死,惩罚。
舜杀死鲧在羽山。
}《离骚》曰:“鲧婞直以亡身兮,终然殀乎羽之野。
”
\zhu{殀:同「夭」,少壮而死。}
}
\ping{
贾宝玉列举了作为男性的鲧、贾谊,和作为女性的晴雯对比,以反衬“巾帼”、“闺帏”遭遇之惨甚于男子。
程高本改“直烈遭危,巾帼惨于羽野”为“贞烈遭危,巾帼惨于雁塞”,强调了晴雯的贞洁,换成王昭君出塞和亲事。
这一改,其中的女德教化意味极强,不仅有碍文理,且在思想性上也削弱了原稿中的斗争反抗和叛逆精神。
}
自蓄辛酸,谁怜夭折!仙云既散,芳趾难寻。
\zhu{趾:踪迹,例如“庶追芳趾”。
}洲迷聚窟,何来却死之香?\zhu{洲迷……香:意谓找不到去聚窟洲的道路,上哪儿找起死回生的神香呢?南朝梁任昉《述异记》卷上:“聚窟洲有返魂树,伐其根心,于玉釜中煮,取汁又熬之,令可丸,名曰惊精香,或名震灵丸,或名反生香,或名却死香。
死尸在地,闻气即活。
”}海失灵槎,不获回生之药。
\zhu{海失……药:意谓没有去海上的仙筏,无法获得回生之药。
灵槎(槎音“茶”):神仙的木筏。
回生之药:传说渤海中有蓬莱、方丈、瀛洲三神山,山上有不死之药,秦始皇使人求之,因船不能至而未得。
见《史记·封禅书》。
}眉黛烟青,昨犹我画;指环玉冷,今倩谁温?\zhu{倩:音“庆”,请别人代自己做事。
}鼎炉之剩药犹存,襟泪之馀痕尚渍。
镜分鸾别,\zhu{镜分鸾别:本指夫妻的分离,这里借喻和晴雯的永别。
鸾镜:即古时妆镜。
镜分:南朝陈亡时,徐德言与其妻乐昌公主各持破镜之半,并约定正月十五日卖镜于市,后因此而重圆,见唐代孟棨《本事诗》。
}愁开麝月之奁;\zhu{奁:音“连”,古代妇女梳妆用的镜匣和盛其他化妆品的器皿。
这一句写宝玉回忆起当年他对镜替麝月篦头,遭晴雯取笑的往事(见本书第二十回),而今晴雯已死,麝月镜奁还在,睹物伤情,所以不忍再开。
}梳化龙飞,哀折檀云之齿。
\zhu{这一句从上文看,这里似和“愁开麝月之奁”一样,应是一段与宝玉、檀云及晴雯有关的故事。
可能是由于原稿遗失或作者在修改时删去,其人其事已无法确知。
}委金钿于草莽,\zhu{委:丢弃。
金钿:镶嵌着金花的首饰。
草莽:野草;杂草。
}拾翠㔩于尘埃。
\zhu{翠㔩(㔩音“饿”):装潢着翠羽的妇女发饰。
}\zhu{“委……尘埃”:从唐代陈鸿《长恨歌传》中“坠金钿翠羽于地,上自取之”之语化出,以唐玄宗与杨贵妃的爱情悲剧,比喻晴雯被逐致死的惨悲遭遇和宝玉依依难舍的哀痛心情。
}楼空鳷鹊,\zhu{鳷:音“支”。
鳷鹊楼:西汉上林苑楼观名,后因“鹊”字与七夕故事相关,渐把鳷鹊楼同七夕的乞巧的事联系起来。
}徒悬七夕之针;\zhu{七夕之针:晋代宗懔《荆楚岁时记》:“七月七日,为牵牛织女聚会之夜。
是夕,人家妇女结彩缕,穿七孔针,陈瓜果于庭中以乞巧(乞巧:请求织女帮助自己提高纺织刺绣技巧)。
”}\zhu{“楼空……针”:意思是,晴雯死后,鳷鹊楼空,七夕乞巧之针无人去穿。
}带断鸳鸯,谁续五丝之缕?\zhu{“带断……缕”:带断鸳鸯即鸳鸯带断,这里比喻同晴雯的永诀。
五丝之缕:五色的丝缕。
传说汉高祖宠姬戚夫人,七月七日以五色缕相系,谓之相连爱。
见《西京杂记》。
}况乃金天属节,白帝司时,\zhu{古人以百物配五行(金、木、水、火、土)。
如春天属木,其味为酸,其色为青,司时之神就叫青帝;秋天属金,其味为辛,其色为白,司时之神就叫白帝。
}孤衾有梦,空室无人。
桐阶月暗,芳魂与倩影同销;蓉帐香残,娇喘共细言皆绝。
连天衰草,岂独蒹葭;\zhu{蒹葭:芦苇。
《诗·秦风·蒹葭》:“蒹葭苍苍,白露为霜。
所谓伊人,在水一方。
溯洄从之,道阻且长,溯游从之,宛在水中央。
”这里借以抒发对晴雯的怀念之情。
}匝地悲声,\zhu{匝:周;遍。
}无非蟋蟀。
露苔晚砌,穿帘不度寒砧;\zhu{度:传。
砧:音“针”,捣衣石,引申为捣衣声,古时妇女为远人作寒衣多于秋夜将衣捣平。
这句话的意思是说人已死去,不再有捣衣的砧声传来。
}雨荔秋垣,\zhu{垣:音“元”,矮墙。
}隔院希闻怨笛。
\zhu{希:稀疏;稀少。
这一句意谓断断续续听到邻家哀怨的笛声。
晋代嵇康、吕安被杀,其挚友向秀从他们的旧居经过,听到邻人吹笛,感音而叹,作《思旧赋》以寄哀思。
见《晋书·向秀传》。
}芳名未泯,檐前鹦鹉犹呼;艳质将亡,槛外海棠预老。
\zhu{槛:音“剑”,栏杆。
}\geng{恰极!}捉迷屏后,\zhu{捉迷:捉迷藏。
}莲瓣无声;\zhu{莲瓣:指旧时女子缠得很小的脚;指绣鞋。
}
\geng{元微之诗:“小楼深迷藏。
”}斗草庭前,\zhu{斗草:又称“斗百草”,一种起源很古的民俗游戏,春夏花草繁茂之期,闺中多喜此戏,参加者各采花草竹木,举其名称作对,以吉祥而少见者为胜。
}兰芽枉待。
\zhu{兰芽:兰的嫩芽。
}抛残绣线,银笺彩缕谁裁?\zhu{
银笺[jiān]:白纸。
与上句“抛残绣线”联系起来,当指刺绣所用的纸样。
}折断冰丝,\zhu{折:折迭,有皱纹的意思。
冰丝:传说冰蚕所吐之丝,洁白清凉如冰,这里代指素绢所制的衣服。
}金斗御香未熨。
\zhu{金斗:熨斗。
}\zhu{“抛残……未熨”:意谓晴雯死后,连刺绣的线都已丢掉,还有谁来剪花样裁衣裳呢?洁白的锦衣已现皱折,却没有人用烧御香的熨斗将它熨平。
}昨承严命,\zhu{严:严父。
}既趋车而远陟芳园;\zhu{陟:音“志”,登高、爬上。
}今犯慈威,\zhu{慈:慈母。
}复拄杖而近抛孤匶。
\geng{
柩本字。
}
\zhu{匶:同“柩”。
}
\zhu{这一句意谓我含悲拄杖亲来吊祭,没想到你孤独的灵柩竟被人仓促地抛掉。
}及闻槥棺被燹,\zhu{槥:音“彗”,小而薄的棺材。
燹:音“显”,火;野火。
这里作焚烧解。
}惭违共穴之盟;\zhu{穴:墓穴。
共穴之盟:死当同葬的盟约。
}石椁成灾,\zhu{椁:音“果”,棺外的套棺。
灾:火灾。
}愧迨同灰之诮。
\zhu{迨[dài]:及。
同灰:李白《长干行》:“十五始展眉,愿同尘与灰。
”本谓夫妇爱情之坚贞。
宝玉曾说过将来要和大观园里的女孩子们一同化烟化灰。
诮:音“翘”,责备,讥讽。
}\zhu{意谓宝玉不能与芙蓉女儿化烟化灰,对因此而将受到讥诮和非议感到惭愧。
}\geng{唐诗云:“光开石棺,木可为棺。
”晋杨公回诗云:“生为并身杨,死作同棺灰。
”}尔乃西风古寺,\zhu{尔乃:发语词。
}
淹滞青燐;\zhu{淹滞青燐:青色的燐火缓缓飘动。
骨中磷质遇到空气燃烧而发的光,从前人们误以为鬼火。
}落日荒丘,零星白骨。
楸榆飒飒,\zhu{楸:音“秋”,楸树。
}蓬艾萧萧。
隔雾圹以啼猿,\zhu{圹:音“框”,坟墓。
}绕烟塍而泣鬼。
\zhu{塍:音“成”,田间的土埂。
}自为红绡帐里,公子情深;始信黄土垄中,女儿命薄!汝南泪血,\zhu{汝南:指南朝宋汝南王。
贾宝玉在这里借汝南王同刘碧玉的故事来比喻自己同晴雯的亲密感情。
宋代郭茂倩《乐府诗集》卷四十五《碧玉歌》,题注引《乐苑》:“《碧玉歌》者,宋汝南王所作也。
碧玉,汝南王妾名,以宠爱之甚,所以歌之。
”}斑斑洒向西风;梓泽馀衷,\zhu{梓泽:石崇的别馆名。
梓泽馀衷:用的是石崇和绿珠的故事。
绿珠是晋代石崇的侍妾,姓梁,善吹笛。
孙秀想要绿珠,石崇不给,孙遂假传皇帝诏令逮捕石崇,绿珠跳楼自杀,石崇也被处死。
见《晋书·石崇传》及宋代乐史撰《绿珠传》。
}默默诉凭冷月。
呜呼!固鬼蜮之为灾,\zhu{蜮:音“育”,传说中水边的一种害人虫,能含了沙射人的影子,人被射后要害病毒。
“鬼蜮”用《诗·小雅·何人斯》“为鬼为蜮”,指用阴谋诡计暗害人的人。
}岂神灵而亦妒。
箝诐奴之口,\zhu{箝:同“钳”,钳制;约束。
诐:音“币”,邪恶。
箝诐奴之口:封住那邪恶奴才的嘴巴。
}讨岂从宽;剖悍妇之心,
\ping{这里的“悍妇”,是指王善保家的等拱火的仆妇,还是也暗指坐镇指挥的王夫人?}
忿犹未释!\geng{《庄子》:“箝杨墨之口。
”《孟子》谓:“诐辞知其所蔽。
”
\zhu{诐辞知其所蔽:偏颇的话知道它所隐瞒之处。}
}在君之尘缘虽浅,然玉之鄙意岂终。
因蓄惓惓之思,\zhu{惓惓:同“拳拳”,情意深厚,恳切。
}
不禁谆谆之问。
始知上帝垂旌,\zhu{垂旌:用竿挑看旌旗,作为使者征召的信号。
}花宫待诏,\zhu{待诏:本汉代官职名。
这里是等待上帝的诏命,即供职的意思。
}生侪兰蕙,\zhu{侪:音“柴”,共同,一起。
}死辖芙蓉。
听小婢之言,似涉无稽;以浊玉之思,则深为有据。
何也?昔叶法善摄魂以撰碑,\zhu{叶法善摄魂以撰碑:传说唐开元间松阳道士叶法善,曾求当时以文章和书法著称的处州刺史李邕为其祖父撰述碑文;文成,再求书写,李邕未允,叶遂用法术摄李魂于梦中写之。
见《处州府志》卷十六。
}李长吉被诏而为记,\zhu{李长吉被诏而为记:据唐代李商隐在《李长吉小传》中说:长吉将死时,忽见一穿红衣、骑赤虬、手持诏书的人,从天上下来对他说,天帝建成了白玉楼,召他前去作文记述其事。
不久,长吉便死去。
}事虽殊,其理则一也。
故相物以配才,苟非其人,恶乃滥乎?\zhu{滥乎:滥乎其位。
“其位”二字,各本俱无,文义不完,据“乾隆抄”一百二十回稿本补。
}始信上帝委托权衡,可谓至洽至协,庶不负其所秉赋也。
因希其不昧之灵,\zhu{昧:隐藏。
如“拾金不昧”。
}或陟降于兹;\zhu{陟降:陟(音“志”)是上升,降是下降。
古籍里“陟降”一词往往只用偏义,或谓上升或谓下降。
这里是降临的意思。
}特不揣鄙俗之词,有污慧听。
乃歌而招之曰:\par
天何如是之苍苍兮,乘玉虬以游乎穹窿耶?\zhu{
玉虬[qiú]:玉色的无角龙。
穹窿:音“穷隆”,天看上去中间高,四方下垂像蓬帐,所以称穹窿。
}\geng{《楚辞》:“驷玉虬以乘鹥兮。
”}\par
地何如是之茫茫兮,驾瑶象以降乎泉壤耶?\zhu{瑶象:指用美玉和象牙制成的车子。
《离骚》:“为余驾飞龙兮,杂瑶象以为车”。
}\geng{《楚辞》:“杂瑶象以为车。
”}\par
望伞盖之陆离兮,抑箕尾之光耶?\zhu{箕尾:星宿名,即箕星和尾星。
传说殷高宗之相傅说(读作“悦”)死后,精魂化为傅说星,在箕星和尾星之间(见《庄子·大宗师》),后因称人死叫“骑箕尾”。
}\par
列羽葆而为前导兮,\zhu{羽葆:音“羽保”,仪仗中用鸟羽联缀装饰的华盖。
}卫危虚于旁耶?\zhu{危虚:星宿名,即危星和虚星,均属二十八宿。
}\par
驱丰隆以为比从兮,\zhu{
丰隆:古代神话中的云神或雷神。
《离骚》:“吾令丰隆乘云兮。”王逸注:“丰隆,云师,一曰雷师。”
比:并列,挨着。
}望舒月以离耶?\zhu{
望舒:古代神话中给月亮赶车的神,后也用作月亮的代称。
一说,此处“望”或作动词。
本句的意思是“你望着那赶月车的神来送你走吗?”
}\geng{危、虚二星为卫护星。
丰隆,雷师。
望舒,月御也。
}\par
听车轨而伊轧兮,\zhu{伊轧:象声词。
船桨、轮轴等发出的声响。
}御鸾鷖以征耶?\zhu{鷖:音“医”。
《离骚》:“驷玉虬以乘鷖兮。
” 王逸注:“鷖,凤凰别名也。
《山海经》云:‘鷖,身有五彩而文如凤凰类也’。
”征:出征,远行。
}\par
闻馥郁而薆然兮,\zhu{薆:音“爱”,香气。
薆然:形容香气浓郁。
}纫蘅杜以为纕耶?\zhu{纫[rèn]:穿联。
蘅、杜:都是香草。
纕:音“香”,佩带。
纫蘅杜以为纕:把蘅杜穿成串当作佩带。
}\par
炫裙裾之烁烁兮,\zhu{裾:音“居”,衣服的后襟。
}镂明月以为珰耶?\zhu{珰:音“当”,耳坠子。
}\par
籍葳蕤而成坛畤兮,\zhu{籍:通“藉”,凭借。
葳蕤:草木茂盛枝叶下垂貌。
坛:古代用来举行祭祀的高台。
畤:音“至”,古时帝王祭天地五帝之所。
}檠莲焰以烛兰膏耶?\zhu{檠:音“情”,灯台。
檠莲焰:在灯台里点燃起莲花似的灯焰。
烛兰膏:烧香油。
}\par
文瓟匏以为觯斝兮,\zhu{文:花纹,引申为刺花纹。
瓟匏:音“袍袍”,葫芦类,可作盛水的器具。
瓟:似瓠(瓠音“户”,葫芦的一种)。
匏:即瓠。
觯斝:音“至假”,古代两种酒器名。
}漉醽醁以浮桂醑耶?\zhu{漉:音“路”,渗出,过滤。
醽醁:音“灵录”,美酒名。
浮:罚人饮酒,这里应该是饮酒的意思。
醑:音“许”,美酒。
桂醑:桂花酒,亦泛指美酒。
}\par
瞻云气而凝盼兮,仿佛有所觇耶?\zhu{觇:音“沾”,窥视、观察。
}\par
俯窈窕而属耳兮,\zhu{窈窕:深远貌;秘奥貌。
属耳:注意倾听。
}恍惚有所闻耶?\par
期汗漫而无夭阏兮,\zhu{期:约会;希望。
汗漫:广漠而无边际。
又为虚拟的神仙名,寓有渺茫不可知的意思。
两说在此皆可通。
无夭阏(阏音“饿”):畅通无阻。
《庄子·逍遥游》:“背负青天而莫之夭阏者。
”}忍捐弃余于尘埃耶?\geng{《逍遥游》:“夭阏”,止也。
}\par
倩风廉之为余驱车兮,\zhu{倩:音“庆”,请别人代自己做事。
风廉:应为“飞廉”,古代神话中的风神。
《离骚》:“后飞廉使奔属。
”}冀联辔而携归耶?\zhu{冀:希望。
辔:音“配”,驾驭牲口的缰绳。
}\par
余中心为之慨然兮,\geng{《庄子·至乐篇》:“我独何能无慨然?”}
徒噭噭而何为耶?\zhu{噭噭:音“叫叫”,状声词,形容哭泣声。
}\geng{《庄子》:“噭噭然随而哭之。
”}\par
君偃然而长寝兮,岂天运之变于斯耶?\geng{《庄子》:“偃然寝于巨室”,谓人死也。
}\geng{又“变而有气,气变而有形,形变而有生,今又变而之死,是相与为春秋冬夏四时行也。
”
\zhu{
    《庄子·至乐》:庄子妻死,惠子吊之,庄子则方箕踞鼓盆而歌。惠子曰:“与人居长子,老身死,不哭亦足矣,又鼓盆而歌,不亦甚乎!”庄子曰:“不然。是其始死也,我独何能无概然!察其始而本无生,非徒无生也,而本无形,非徒无形也,而本无气。杂乎芒芴之间,变而有气,气变而有形,形变而有生,今又变而之死,是相与为春秋冬夏四时行也。人且偃然寝于巨室,而我噭噭然随而哭之,自以为不通乎命,故止也。”
    这段的大致意思是,庄子说在妻子刚死的时候怎能不哀伤呢?可是观察她起初本来是没有生命的,不仅没有生命而且还没有形体,不仅没有形体而且还没有气息。在若有若无之间,变而成气,气变而成形,形变而成生命,现在又变而为死,这样生来死往的变化就好像春夏秋冬四季的运行一样。人家静静地安息在天地之间,而我还在啼啼哭哭,我以为这样是不通达生命的道理,所以才不哭。
}
}\geng{《天道篇》:“其死也物化。”
\zhu{
《庄子·天道》:其生也天行,其死也物化。大致意思是,他存在时便顺自然而行,他死亡时便和外物融合。
}
}\par
既窀穸且安稳兮,\zhu{
窀穸[zhūnxī]:墓穴。
}反其真而复奚化耶?\zhu{反其真:指死。
意谓反本归源。
奚:疑问代词。
什么,哪里。
化:表示死的一种委婉说法。
}\geng{窀音肫。
《左传》:“窀穸之事”,墓穴幽堂也。
左贵嫔《杨后诔》:“早即窀穸。
”《庄子·大宗师》:“而已反其真。
”注:以死为真。
}\par
余犹桎梏而悬附兮,\zhu{悬附:“附赘悬疣”(疣:音“油”)的省略语,比喻累赘。
赘:瘤肿。
疣:生在皮肤上的肉赘,通称瘊子。
}灵格余以嗟来耶?\zhu{灵:灵魂,指晴雯的灵魂。
格:感通。
嗟来:招唤灵魂到来的话。
《庄子.大宗师》:“嗟来桑户乎!磋来桑户乎!”桑户,人名。
他的朋友孟子反和子琴张招他的魂这样说。
}\geng{《庄子·大宗师》:桎梏之名。
}\geng{“彼以生为悬疣附赘,以死为决疣溃痈。”
\zhu{
   “彼方且与造物者为人,而游乎天地之一气,彼以生为悬疣附赘,以死为决疣溃痈。夫若然者,又恶知死生先后之所在!” 
   这段的大致意思是,他们正和造物者为友伴,而遨游于天地之间。他们把生命看作是气的凝结,像身上的赘瘤一般,把死亡看作(是气的消散,)像脓疮溃破了一样,像这样子,又哪里知道死生先后的分别呢!
}
}\geng{“嗟来桑户乎!嗟来桑户乎!”注:桑户,人名。
孟子反、琴张二人,招其魂而语之也。
}\geng{“方将不化,恶知已化哉!”言人死犹如化去。
《法华经》云:“法华道师多殊方便,于险道中化一城,疲极之众,入城皆生已度想,安稳想。”
\zhu{《法华经》:《法华经·化城喻品第七》。}
}\par
来兮止兮,君其来耶!\par
若夫鸿蒙而居,\zhu{鸿蒙:旧指宇宙形成以前的原始浑沌状态。
}寂静以处,虽临于兹,余亦莫睹。
搴烟萝而为步幛,\zhu{搴:音“千”,拔取。
幛:音“帐”,遮蔽、遮挡。
步幛:同“步障”,古代显贵者出游时,于道旁设下遮蔽风寒尘土或禁人窥视的帐幕,长者可达数十里。
}
列枪蒲而森行伍。
\zhu{枪:形状像枪的器物。
蒲:音“仆”,蒲柳,在群树中最早凋落,故用来比喻女子体质衰弱或身分低微的女子。
如:“蒲柳之姿”。
森:整肃、不可侵犯的,这里应该是使动用法,使……整肃。
}警柳眼之贪眠,\zhu{柳眼:柳叶细长如眼,所以这样说。
}释莲心之味苦。
\zhu{莲心:莲心味苦,古乐府中常喻男女思念之苦,井用“莲心”谐音“怜心”。
}素女约于桂岩,\zhu{素女:古代神话中善鼓瑟的神女(见《史记·封禅书》)。
在这里,素女即月中素娥。
唐代柳宗元《龙城录·明皇梦游广寒宫》载:唐明皇游月宫,见素娥十馀人,舞于桂树之下。
桂岩:长有桂树的山崖,与下句“兰渚”相对。
}
宓妃迎于兰渚。
\zhu{宓(音“伏”)妃:传说是宓羲之女,洛水之神。
}弄玉吹笙,\zhu{弄玉吹笙:据明代董斯张《广博物志》载:春秋时秦穆公之女弄玉善吹笙,能招来凤凰,嫁于善吹箫的萧史。
汉代刘向《列仙传》等称后来二人成仙飞去。
}寒簧击敔。
\zhu{寒簧:仙女名。
传说她曾作过西王母的散花女史,后来又当月宫的侍书,向嫦娥学紫云之歌、霓裳之舞。
见明代叶绍袁《续窈闻记》及清代尤侗《钧天乐》。
敔:音“与”,打击乐器。
}征嵩岳之妃,\zhu{征:召,征召。
特指君召臣。
嵩岳之妃:指嵩山神的夫人灵妃。
《旧唐书·礼仪志》:武则天证圣元年,“下制,号嵩山为神岳,尊嵩山神为天中王,夫人为灵妃。
”}启骊山之姥。
\zhu{启:陈述、告诉。
如“启禀”、“启奏”。
骊山之姥(姥音“母”):女仙名,也称骊山老母。
《太平广记》卷六十三引《集仙传》,有骊山老母为李筌讲解黄帝《阴符经》的故事。
又,《搜神记》中说有个神妪叫成夫人,好音乐,每听到有人奏乐歌唱,便跳起舞来。
所以李贺《李凭箜篌引》中有「梦入神山教神妪」的诗句。
}
龟呈洛浦之灵,\zhu{龟呈洛浦之灵:传说夏禹治水时,洛水里曾有神龟背着文书来献给他。
《尚书·洪范》:“天乃赐禹洪范九畴。
”汉代孔安国传:“天与禹,洛出书,神龟负文而出,列于背,有数至九。
”}兽作咸池之舞。
\zhu{咸池:乐名,也叫《大咸》。
《礼记·乐记》:“咸池备矣。
”东汉郑玄注:“黄帝所作乐名也,尧增修而用之。
咸,皆也。
池,施也。
”}潜赤水兮龙吟,集珠林兮凤翥。
\zhu{珠林:也称珠树林。
《山海经·海外南经》:“三株(一作“珠”)树在厌火北,生赤水上,其为树如柏。
叶皆为珠。
”翥:音“助”,鸟向上飞。
}爰格爰诚,\zhu{爰格爰诚:这种句法,在《诗经》等古籍中屡见,在多数情况下,“爰”(音“元”)只能作联接两个意义相近的词的语助词。
格,在这里是感动的意思,如“格于皇天”。
}匪簠匪筥。
\zhu{匪:通“非”。
簠:音“府”,古代祭祀或宴会时,用来盛谷物的长方形器皿。
铜制或木制。
筥:音“举”,圆形的竹筐。
本句意谓祭在心诚,不在供品。
}发轫乎霞城,\zhu{轫:音“刃”,阻碍车轮转动的木棍,车发动时须抽去。
发轫:启程,出发。
霞城:相传是神仙居住的地方。
《太平御览》引《上清经》曰:“元始天尊居紫云之阙,碧霞为城。
”}返旌乎玄圃。
\zhu{旌:古时一种用五色羽毛装饰的旗子。
又作旗子的通称。
玄圃:亦作悬圃,神仙居住的地方。
《离骚》:“朝发轫于苍梧兮,夕余至乎悬圃。
”王逸注:“悬圃,神山,在昆仑之上。
”}既显微而若通,复氤氲而倏阻。
\zhu{氤氲:音“因晕”,烟云笼罩。
倏:音“述”,急速。
}离合兮烟云,空蒙兮雾雨。
尘霾敛兮星高,溪山丽兮月午。
\zhu{午:白天或夜晚中间时段的。
如“午饭”、“午夜”。
}何心意之忡忡,\zhu{忡忡:音“冲冲”,忧愁的样子。
}若寤寐之栩栩。
\zhu{
寤[wù],睡醒。
寐[mèi],就寝。
寤寐表示无时无刻。
栩栩:喜悦自得,活泼生动的样子。
《庄子·齐物论》:“栩栩然胡蝶也。
”成语有“栩栩如生”。
何心意之忡忡,若寤寐之栩栩:晴雯仿佛无时无刻还生动活泼地活在我身边,而实际上却已经死去了,这使我忧愁。
}余乃欷歔怅望,泣涕徬徨。
人语兮寂历,天籁兮筼筜。
\zhu{
天籁:发自自然界的声音,如风声、鸟声、流水声等。
筼筜:音“云当”,长节的大竹。
天籁兮筼筜:竹林里发出天然的音响。
}鸟惊散而飞,鱼唼喋以响。
\zhu{唼喋:音“霎闸”,鱼或水鸟聚食声。
这里指鱼嘴开合,咂水吞食。
}志哀兮是祷,成礼兮期祥。
\zhu{期:期望。
}呜呼哀哉!尚飨!\zhu{飨:音“想”。
尚飨:古时祭文中的固定词,意谓望死者前来享用祭品。
}\par
读毕,遂焚帛奠茗,犹依依不舍。
小鬟催至再四,方才回身。
忽听山石之后有一人笑道:“且请留步。
”二人听了,不免一惊。
那小鬟回头一看,却是个人影从芙蓉花中走出来,他便大叫:“不好,有鬼。
晴雯真来显魂了!”唬得宝玉也忙看时,——且听下回分解。
\par

\qi{总评:前文入一院,必叙一番养竹种花,为诸婆争利渲染。
此文入一院,必叙一番树枯香老,为亲眷凋零凄楚。
字字实境,字字奇情,令我把玩不释。
\hang
《姽婳词》一段,与前后文似断似连,如罗浮二山,烟雨为连合,时有精气来往。
}
\dai{155}{老学士闲征姽婳词}
\dai{156}{痴公子杜撰芙蓉诔}
\sun{p78-1}{美优伶斩情归水月,晴雯死辖芙蓉}{图右侧:芳官、蕊官、藕官被逐后,哭闹着要削发为尼。
正巧两个尼姑来送供尖,听说后,巴不得又拐两个女孩子去作活使唤,劝王夫人不要限了善念,三人遂出家为尼。
图左侧:宝玉向丫头打探晴雯消息,得知她已经死去。
其中一个小丫头哄宝玉说晴雯做了芙蓉花神。
}
\sun{p78-2}{老学士闲征姽婳词}{贾政正与幕僚们讲起当年女杰姽婳将军林四娘报国捐躯轶事,欲征《姽婳词》一首,以志其忠义。
遂命宝玉、贾环、贾兰三人各写一首。
众幕宾看了称赞一番。
}
\sun{p78-3}{痴公子杜撰芙蓉诔,黛玉评点暗藏谶语}{宝玉用晴雯素日所喜之冰鲛縠写成《芙蓉女儿诔》,又备了四样晴雯所喜之物,于是夜月下,命那小丫头捧至芙蓉花前,将那诔文挂于芙蓉枝上,泣涕读之。
读毕,遂焚帛奠茗。
刚欲回身,猛见黛玉自石后转出,吓了一跳。
}
