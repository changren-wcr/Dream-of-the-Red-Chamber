\chapter{诉肺腑心迷活宝玉 \quad 含耻辱情烈死金钏}
\ji{前明显祖汤先生有《怀人》诗一绝,堪合此回,故录之以待知音:\hang
无情无尽却情多,情到无多得尽么?\hang
解到多情情尽处,月中无树影无波。\hang
\zhu{
这首诗是对情的认同和赞美。全诗以绕口令式的句法将“情”反复渲染,无非是说“情”乃人生的一种悖论,无可奈何而又心甘情愿地沉溺其中。
三、四句说当人悟彻了“情”之悖论后,即使面对月中的桂花树和水中的影像也视而不见、波澜不起了。这其实是一种“反弹琵琶”的写法,骨子里还是在说“情”的无限魔力。
}
}\par
话说宝玉见那麒麟,心中甚是欢喜,便伸手来拿,笑道:“亏你拣着了。
你是那里拣的?”史湘云笑道:“幸而是这个,明儿倘或把印也丢了,\zhu{印:指官印。
丢印意味着丢官。
}难道也就罢了不成?”宝玉笑道:“倒是丢了印平常,若丢了这个,我就该死了。
”袭人斟了茶来与史湘云吃,一面笑道:“大姑娘,听见前儿你大喜了。
”史湘云红了脸,吃茶不答。
\zhu{湘云前日有人家来相看,眼见有婆婆家了,所以害羞了。
}袭人道:“这会子又害臊了。
你还记得十年前,咱们在西边暖阁住着,晚上你同我说的话儿?那会子不害臊,这会子怎么又害臊了?”史湘云笑道:“你还说呢。
那会子咱们那么好。
后来我们太太没了,我家去住了一程子,\zhu{一程子:最近一些天,这些日子。
}怎么就把你派了跟二哥哥,我来了,你就不像先待我了。
”袭人笑道:“你还说呢。
先姐姐长姐姐短哄着我替你梳头洗脸,作这个弄那个,\meng{大家风范,情景逼真。
}如今大了,就拿出小姐的款来。
\zhu{拿……款:摆架子。
}你既拿小姐的款,我怎敢亲近呢?”史湘云道:“阿弥陀佛,冤枉冤哉!我要这样,就立刻死了。
你瞧瞧,这么大热天,我来了,必定赶来先瞧瞧你。
不信你问问缕儿,我在家时时刻刻那一回不念你几声。
”话未了,忙的袭人和宝玉都劝道:“顽话你又认真了。
还是这么性急。
”史湘云道:“你不说你的话噎人,倒说人性急。
”一面说,一面打开手帕子,将戒指递与袭人。
\meng{心中意中,多少情致。
}\par
袭人感谢不尽,因笑道:“你前儿送你姐姐们的,我已得了;今儿你亲自又送来,可见是没忘了我。
只这个就试出你来了。
戒指儿能值多少,可见你的心真。
”史湘云道:“是谁给你的?”袭人道:“是宝姑娘给我的。
”\ping{袭为钗副,袭人是宝钗的影像。
}湘云笑道:“我只当是林姐姐给你的,原来是宝钗姐姐给了你。
我天天在家里想着,这些姐姐们再没一个比宝姐姐好的。
可惜我们不是一个娘养的。
\meng{感知己之一叹。
}我但凡有这么个亲姐姐,就是没了父母,也是没妨碍的。
”说着,眼睛圈儿就红了。
\meng{千古同慨。
}宝玉道:“罢,罢,罢!不用提这个话。
”史湘云道:“提这个便怎么?我知道你的心病,恐怕你的林妹妹听见,又怪嗔我赞了宝姐姐。
可是为这个不是?”袭人在旁嗤的一笑,说道:“云姑娘,你如今大了,越发心直口快了。
”宝玉笑道:“我说你们这几个人难说话,果然不错。
”史湘云道:“好哥哥,你不必说话教我恶心。
只会在我们跟前说话,见了你林妹妹,又不知怎么了。
”\meng{豪爽情形如画。
}\par
袭人道:“且别说顽话,正有一件事还要求你呢。
”史湘云便问:“什么事?”袭人道:“有一双鞋,抠了垫心子。
\zhu{抠了垫心子:抠:挖;镂。
意谓将鞋面用剪刀挖铰出各种花样图案,从背面再衬上别种颜色的料子。
}我这两日身上不好,不得做,你可有工夫替我做做?”史湘云笑道:“这又奇了,你家放着这些巧人不算,还有什么针线上的,裁剪上的,怎么教我做起来?你的活计叫谁做,谁好意思不做呢。
”袭人笑道:“你又糊涂了。
你难道不知道,我们这屋里的针线,\meng{“我们这屋里”等字,精神活跳。
}\ping{袭人俨然把自己当成了怡红院的女主人,和宝玉并列,难怪上回晴雯会挑刺,说道:“那里就称起‘我们’来了。
明公正道,连个姑娘还没挣上去呢,也不过和我似的,那里就称上‘我们’了!”}是不要那些针线上的人做的。
”史湘云听了,便知是宝玉的鞋了,因笑道:“既这么说,我就替你做了罢。
只是一件,你的我才作,别人的我可不能。
”袭人笑道:“又来了,我是个什么,就烦你做鞋了。
实告诉你,可不是我的。
你别管是谁的,横竖我领情就是了。
”史湘云道:“论理,你的东西也不知烦我做了多少了,今儿我倒不做了的原故,你必定也知道。
”\zhu{湘云已经开始相亲,要有婆家了,所以会为给宝玉做鞋而感到害羞。
}
袭人道:“倒也不知道。
\meng{反衬迭起,灵活之至。
}”史湘云冷笑道:“前儿我听见把我做的扇套子拿着和人家比,赌气又铰了。
我早就听见了,你还瞒我。
这会子又叫我做,我成了你们的奴才了。
”宝玉忙笑道:“前儿的那事,本不知是你做的。
”袭人也笑道:“他本不知是你做的。
是我哄他的话,说是新近外头有个会做活的女孩子,说扎的出奇的花,我叫他拿了一个扇套子试试看好不好。
他就信了,拿出去给这个瞧给那个看的。
不知怎么又惹恼了林姑娘,铰了两段。
\ping{第十八回,黛玉误以为宝玉把自己给他做的荷包送给了别人,赌气回房,将前日宝玉所烦她作的那个香袋儿,赌气拿过来就铰。
黛玉的剪刀好快,一生气就铰。
}回来他还叫赶着做去,我才说了是你作的,他后悔的什么似的。
\meng{描神!}”史湘云道:“越发奇了。
林姑娘他也犯不上生气,他既会剪,就叫他做。
”袭人道:“他可不作呢。
饶这么着,老太太还怕他劳碌着了。
大夫又说好生静养才好,谁还烦他做?旧年好一年的工夫,做了个香袋儿;今年半年,还没见拿针线呢。
”\ping{阴阳怪气。
}\par
正说着,有人来回说:“兴隆街的大爷来了,老爷叫二爷出去会。
”宝玉听了,便知是贾雨村来了,心中好不自在。
袭人忙去拿衣服。
宝玉一面蹬着靴子,一面抱怨道:“有老爷和他坐着就罢了,\meng{原本烦俗。
}
回回定要见我。
”史湘云一边摇着扇子,笑道:“自然你能会宾接客,老爷才叫你出去呢。
”宝玉道:“那里是老爷,都是他自己要请我去见的。
”湘云笑道:“主雅客来勤,自然你有些警他的好处,他才只要会你。
”宝玉道:“罢,罢,我也不敢称雅,俗中又俗的一个俗人,并不愿同这些人往来。
”\meng{我也不知宝玉是雅是俗,请诸同类一拟。
}\par
湘云笑道:“还是这个情性不改。
如今大了,你就不愿读书去考举人进士的,也该常常的会会这些为官做宰的人们,谈谈讲讲些仕途经济的学问,也好将来应酬世务,日后也有个朋友。
没见你成年家只在我们队里搅些什么!”宝玉听了道:“姑娘请别的姊妹屋里坐坐,我这里仔细污了你知经济学问的。
”袭人道:“云姑娘快别说这话。
\meng{此际不同湘云一语,湘云也实难出一语。
}上回也是宝姑娘也说过一回,他也不管人脸上过的去过不去,他就咳了一声,
\zhu{咳[hāi]:叹词,表示惊异。}
拿起脚来走了。
这里宝姑娘的话也没说完,见他走了,登时羞的脸通红,说又不是,不说又不是。
幸而是宝姑娘,那要是林姑娘,不知又闹到怎么样,哭的怎么样呢。
提起这个话来,真真的宝姑娘叫人敬重,自己讪了一会子去了。
\zhu{讪:音“善”,羞惭,难为情。}
我倒过不去,\meng{袭人善解忿。
}
只当他恼了。
谁知过后还是照旧一样,真真有涵养,心地宽大。
谁知这一个反倒同他生分了。
那林姑娘见你赌气不理他,你得赔多少不是呢。
”\ping{宝钗和黛玉之间,袭人更喜欢宝钗。
}宝玉道:“林姑娘从来说过这些混帐话不曾?若他也说过这些混帐话,我早和他生分了。
”\meng{花爱水清明,水怜花色鲜。
浮落虽同流,空惹鱼龙涎。
\zhu{
涎[xián]:口水;唾液。
第八回黛玉给宝玉整冠戴笠处,有批语“知己最难逢,相逢意自同。花新水上香,花下水含红”。
这两条批语都是赞美宝黛心心相印,似水花相映。
“空惹”二字,何其悲怆,那种可望不可及的心怀昭然若揭,黛玉正是死于这种可望不可及的焦灼等待与期盼。
}
}袭人和湘云都点头笑道:“这原是混帐话。
”\chen{写足!憨宝玉殊可发一大笑。
\zhu{殊:很,非常。
}}\ping{宝玉和未出场的黛玉是一个阵营,袭人湘云宝钗是另一个阵营。
}\par
原来林黛玉知道史湘云在这里,宝玉又赶来,一定说麒麟的原故。
因此心下忖度着,
\zhu{忖:音“村”三声,思量,揣度。}
近日宝玉弄来的外传野史,多半才子佳人都因小巧玩物上撮合,或有鸳鸯,或有凤凰,或玉环金珮,或鲛帕鸾绦,\zhu{鲛帕鸾绦:鲛:音“交”,这里指鲛绡(绡音“消”)纱。
传说南海中有鲛人,即人鱼,能织绡,后用以泛称薄纱。
鸾:传说中凤凰一类的鸟。
鸾绦:指上面织有凤鸾一类图案的丝带。
}皆由小物而遂终身。
今忽见宝玉亦有麒麟,便恐借此生隙,
\zhu{生隙:产生嫌隙或事端。}
同史湘云也做出那些风流佳事来。
\ping{明面上在意湘云的金麒麟,暗地里在意宝钗的金锁。
}因而悄悄走来,见机行事,以察二人之意。
不想刚走来,正听见史湘云说经济一事,宝玉又说:“林妹妹不说这样混帐话,若说这话,我也和他生分了。
”林黛玉听了这话,不觉又喜又惊,又悲又叹。
\ping{这样背地里听到的话对于安抚黛玉可能更有用,比当面更好。
}
所喜者,果然自己眼力不错,素日认他是个知己,果然是个知己。
所惊者,他在人前一片私心称扬于我,其亲热厚密,竟不避嫌疑。
所叹者,你既为我之知己,自然我亦可为你之知己矣;既你我为知己,则又何必有金玉之论哉;既有金玉之论,亦该你我有之,则又何必来一宝钗哉!所悲者,父母早逝,虽有铭心刻骨之言,无人为我主张。
\ping{可能暗示了黛玉悲剧的原因,黛玉父母双亡,自己对于婚姻的想法,没法给父母述说,让父母为自己主张,好在贾母支持自己的想法,可惜贾母去世之后,就真的无人为黛玉主张她想要的婚姻。
}况近日每觉神思恍惚,病已渐成,医者更云气弱血亏,恐致劳怯之症。
\zhu{劳怯之症:劳:即痨,一种消耗性疾病。
怯:身体怯弱,也指气血不足。
“劳”病包括现代的结核、严重贫血等病。
}你我虽为知己,但恐自不能久待;你纵为我知己,奈我薄命何!\ping{可能暗示黛玉的结局,等不到宝玉的婚约,就因病死去。
}想到此间,不禁滚下泪来。
\meng{普天下才子佳人、英雄侠[士]都同来一哭!我虽愚浊,也愿同声一哭。
}待进去相见,自觉无味,便一面拭泪,一面抽身回去了。
\par
这里宝玉忙忙的穿了衣裳出来,忽见林黛玉在前面慢慢的走着,似有拭泪之状,便忙赶上来,\meng{关心情致。
}笑道:“妹妹往那里去?怎么又哭了?又是谁得罪了你?”林黛玉回头见是宝玉,便勉强笑道:“好好的,我何曾哭了。
”宝玉笑道:“你瞧瞧,眼睛上的泪珠儿未干,还撒谎呢。
”一面说,一面禁不住抬起手来替他拭泪。
林黛玉忙向后退了几步,说道:“你又要死了!\meng{娇羞态!}作什么这么动手动脚的!”宝玉笑道:“说话忘了情,不觉的动了手,也就顾不的死活。
”林黛玉道:“你死了倒不值什么,只是丢下了什么金,又是什么麒麟,可怎么样呢?”一句话又把宝玉说急了,赶上来问道:“你还说这话,到底是咒我还是气我呢?”林黛玉见问,方想起前日的事来,\zhu{第二十九回,黛玉提及金玉良缘,怀疑宝玉的真心,惹得宝玉砸玉证明自己的真心。
}遂自悔自己又说造次了,忙笑道:“你别着急,我原说错了。
这有什么的,筋都暴起来,急的一脸汗。
”一面说,一面禁不住近前伸手替他拭面上的汗\meng{痴情态。
}。
\par
宝玉瞅了半天,方说道“你放心”三个字。
\meng{连我今日看之,也不懂是何等文章。
}林黛玉听了,怔了半天,方说道:“我有什么不放心的?我不明白这话。
你倒说说怎么放心不放心?”宝玉叹了一口气,问道:“你果不明白这话?难道我素日在你身上的心都用错了?连你的意思若体贴不着,就难怪你天天为我生气了。
”林黛玉道:“果然我不明白放心不放心的话。
”宝玉点头叹道:“好妹妹,你别哄我。
果然不明白这话,不但我素日之意白用了,且连你素日待我之意也都辜负了。
\meng{第二层。
}你皆因总是不放心的原故,\ping{这里的“不放心”,大概就是没有安全感的意思吧。
}才弄了一身病。
但凡宽慰些,\meng{真疼真爱、真怜真惜中,每每生出此等心病来。
}这病也不得一日重似一日。
”\ping{黛玉的病也是心病,不放心导致的,宽慰一些,病就好一些,可能最后黛玉受不了巨大的精神打击,加速了本来就有的病情,最后凄惨的死去。
}林黛玉听了这话,如轰雷掣电,细细思之,竟比自己肺腑中掏出来的还觉恳切,\meng{何等神佛开慧眼,照见众生孽障,为现此锦绣文章,说此上乘功德法。
}
竟有万句言语,满心要说,只是半个字也不能吐,却怔怔的望着他。
此时宝玉心中也有万句言语,不知从那一句上说起,却也怔怔的望着黛玉。
两个人怔了半天,林黛玉只咳了一声,两眼不觉滚下泪来,回身便要走。
\meng{下笔时用一“走”,文之大力,孟贲不若也。
\zhu{孟贲[bēn]:中国战国时期卫国的勇士,相传他力大无穷。
}}宝玉忙上前拉住,说道:“好妹妹,且略站住,我说一句话再走。
”林黛玉一面拭泪,一面将手推开,说道:“有什么可说的。
你的话我早知道了!”口里说着,却头也不回竟去了。
\par
宝玉站着,只管发起呆来。
\chen{儿女之情毕露,至此极矣!}原来方才出来慌忙,不曾带得扇子,袭人怕他热,忙拿了扇子赶来送与他,忽抬头见了林黛玉和他站着。
一时黛玉走了,他还站着不动,因而赶上来说道:“你也不带了扇子去,亏我看见,赶了送来。
”宝玉出了神,见袭人和他说话,并未看出是何人来,便一把拉住,说道:“好妹妹,我的这心事,从来也不敢说,今儿我大胆说出来,死也甘心!我为你也弄了一身的病在这里,又不敢告诉人,只好掩着。
只等你的病好了,只怕我的病才得好呢。
睡里梦里也忘不了你!”\ping{宝玉的表白。
}袭人听了这话,吓得魄消魂散,只叫“神天菩萨,坑死我了!”便推他道:“这是那里的话!敢是中了邪?还不快去?”宝玉一时醒过来,方知是袭人送扇子来,羞的满面紫涨,夺了扇子,便忙忙的抽身跑了。
\par
这里袭人见他去了,自思方才之言,一定是因黛玉而起,如此看来,将来难免不才之事,\zhu{不才之事:没出息的事。
这里指男女间的丑事。
}令人可惊可畏。
想到此间,也不觉怔怔的滴下泪来,心下暗度如何处治方免此丑祸。
正裁疑间,\zhu{裁:裁断,衡量。
另一种解释,“裁疑”义同“猜疑”。
}\ping{伏第三十四回,袭人向王夫人进言,要分开宝玉和姐妹,避免不才之事。
}忽有宝钗从那边走来,笑道:“大毒日头地下,出什么神呢?”袭人见问,忙笑道:“那边两个雀儿打架,倒也好玩,我就看住了。
”宝钗道:“宝兄弟这会子穿了衣服,忙忙的那去了?我才看见走过去,倒要叫住问他呢。
他如今说话越发没了经纬,\zhu{经纬:织机上的直线叫经,横线叫纬,这里引伸为道理、规矩的意思。
}\ping{可能是指本回袭人所说的,宝钗向宝玉讲仕途经济学问,宝玉咳了一声,抬脚就走,一点也没给宝钗留面子,羞得宝钗脸通红。
}我故此没叫他了,由他过去罢。
”袭人道:“老爷叫他出去。
”宝钗听了,忙道:“嗳哟!这么黄天暑热的,\zhu{黄天:即农历六月,亦称“长夏”。
按五行之说:夏,色赤;长夏,色黄;故大暑天又称黄天。
}叫他做什么!别是想起什么来生了气,\meng{偏是近。
}叫出去教训一场。
”袭人笑道:“不是这个,想是有客要会。
”宝钗笑道:“这个客也没意思,这么热天,不在家里凉快,还跑些什么!”\ping{
宝钗批评别人大热天还到处乱跑,全忘了自己也正在乱跑。
作者此处可能暗含细微的讽刺。
}袭人笑道:“倒是你说说罢。
”\par
宝钗因而问道:“云丫头在你们家做什么呢?”\ping{
宝钗其实也是为麒麟的事来的,第二十九回说出湘云有个金麒麟的就是宝钗。
}袭人笑道:“才说了一会子闲话。
你瞧,我前儿粘的那双鞋,明儿叫他做去。
”宝钗听见这话,便两边回头,看无人来往,便笑道:“你这么个明白人,怎么一时半刻的就不会体谅人情。
我近来看着云丫头神情,再风里言风里语的听起来,那云丫头在家里竟一点儿作不得主。
他们家嫌费用大,竟不用那些针线上的人,差不多的东西多是他们娘儿们动手。
为什么这几次他来了,他和我说话儿,见没人在跟前,他就说家里累的很。
我再问他两句家常过日子的话,他就连眼圈儿都红了,口里含含糊糊待说不说的。
想其形景来,自然从小儿没爹娘的苦。
\meng{真是知己,不枉湘云前言。
}我看着他,也不觉的伤起心来。
”\ping{本回前文,湘云说道:“我天天在家里想着,这些姐姐们再没一个比宝姐姐好的。
可惜我们不是一个娘养的。
我但凡有这么个亲姐姐,就是没了父母,也是没妨碍的。
”这里补足湘云赞叹宝钗的原因。
}袭人见说这话,将手一拍,说:“是了,是了。
怪道上月我烦他打十根蝴蝶结子,过了那些日子才打发人送来,还说‘打的粗,且在别处能着使罢;
\zhu{能着:犹言“耐着”、“忍着”,引申为“将就着”。}
要匀净的,等明儿来住着再好生打罢’。
如今听宝姑娘这话,想来我们烦他他不好推辞,不知他在家里怎么三更半夜的做呢。
可是我也糊涂了,早知是这样,我也不烦他了。
”宝钗道:“上次他就告诉我,在家里做活做到三更天,若是替别人做一点半点,他家的那些奶奶太太们还不受用呢。
”\ping{湘云外面看来十分豪爽,光风霁月,但是因为自己也是父母双亡,寄人篱下,不得不隐忍委屈自己。
}袭人道:“偏生我们那个牛心左性的小爷,\zhu{牛心:犟,死心眼儿。
左性:性情固执,遇事不肯变通。
}\meng{多情的当有这样“牛心左性”之癖。
}凭着小的大的活计,一概不要家里这些活计上的人作。
我又弄不开这些。
”宝钗笑道:“你理他呢!只管叫人做去,只说是你做的就是了。
”袭人笑道:“那里哄的信他,他才是认得出来呢。
说不得我只好慢慢的累去罢了。
\meng{痴心的情愿。
}”宝钗笑道:“你不必忙,我替你作些如何?”袭人笑道:“当真的这样,就是我的福了。
晚上我亲自送过来。
”\ping{本回前文,袭人说黛玉一年拿不了几次针,而这里却让宝钗帮自己做针线,可见袭人对于宝钗的偏爱,袭人宝钗两人逐渐走到了一起,正是“袭为钗副”。
}\ping{
宝钗到底有没有夸张,史湘云在家里面的遭遇是不是真的有那么严重。
如此轻描淡写,宝钗就把宝玉的东西接过去了,这就是她想要的结果。
}\par
一句话未了,忽见一个老婆子忙忙走来,说道:“这是那里说起!金钏儿姑娘好好的投井死了!”袭人唬了一跳,忙问:“那个金钏儿?”那老婆子道:“那里还有两个金钏儿呢?就是太太屋里的。
前儿不知为什么撵他出去,在家里哭天哭地的,也都不理会他,谁知找他不见了。
刚才打水的人在那东南角上井里打水,见一个尸首,赶着叫人打捞起来,谁知是他。
他们家里还只管乱着要救活,那里中用了!”宝钗道:“这也奇了。
”袭人听说,点头赞叹,\zhu{赞:金钏个性刚烈,在蒙受不白之冤后,死也不受这种羞辱,这种骨气令袭人赞赏。
}想素日同气之情,不觉流下泪来。
\meng{又一哭法。
}宝钗听见这话,忙向王夫人处来道安慰。
这里袭人回去不提。
\par
却说宝钗来至王夫人处,只见鸦雀无闻,独有王夫人在里间房内坐着垂泪。
\meng{又一哭法。
}宝钗便不好提这事,只得一旁坐了。
王夫人便问:“你从那里来?”宝钗道:“从园里来。
”王夫人道:“你从园里来,可见你宝兄弟?”\meng{世人多是凡事欲瞒人,偏不意中将要着逗露,
\zhu{要着:重要之事。}
理之所无,事则多有,何也?}宝钗道:“才倒看见了。
他穿了衣服出去了,不知那里去。
”\par
王夫人点头哭道:“你可知道一桩奇事?金钏儿忽然投井死了!”宝钗见说,道:“怎么好好的投井?这也奇了。
”王夫人道:“原是前儿他把我一件东西弄坏了,我一时生气,打了他几下,撵了他下去。
我只说气他两天,还叫他上来,谁知他这么气性大,就投井死了。
岂不是我的罪过。
”\ping{王夫人遮掩真相,一方面因为很小的事情,钓鱼执法撵走金钏,自觉理亏;另一方面,金钏因为和宝玉过度亲密而被撵走导致她走投无路投井自尽,这也是宝玉的丑事,如果公开真相,经过添油加醋,反而会产生更多的不必要的流言蜚语。
}宝钗叹道:“姨娘是慈善人,固然这么想。
据我看来,他并不是赌气投井。
多半他下去住着,或是在井跟前憨顽,失了脚掉下去的。
他在上头拘束惯了,这一出去,自然要到各处去顽顽逛逛,岂有这样大气的理!纵然有这样大气,也不过是个糊涂人,也不为可惜。
\meng{善劝人,大见解!惜乎不知其情,\zhu{情:隐情。
}虽精[金]美玉之言,不中奈何!\zhu{中:符合。
}}”王夫人点头叹道:“这话虽然如此说,到底我心不安。
”宝钗叹道:“姨娘也不必念念于兹,十分过不去,不过多赏他几两银子发送他,\zhu{发送:殡葬死者。
}也就尽主仆之情了。
”\ping{宝钗在面对生死的时候,显示出特别的冷静,甚至比王夫人都要镇定,可能也是因为宝钗并不知道事情的真相。
}\par
王夫人道:“刚才我赏了他娘五十两银子,原要还把你妹妹们的新衣服拿两套给他妆裹。
谁知凤丫头说可巧都没什么新做的衣服,只有你林妹妹作生日的两套。
我想你林妹妹那个孩子素日是个有心的,况且他也三灾八难的,既说了给他过生日,这会子又给人妆裹去,岂不忌讳。
因为这么样,我现叫裁缝赶两套给他。
要是别的丫头,赏他几两银子也就完了,只是金钏儿虽然是个丫头,素日在我跟前比我的女儿也差不多。
”口里说着,不觉泪下。
\ping{王夫人虚伪的一面展露无遗,如果是自己的女儿,会舍得因为一件小事就逼死她嘛?}宝钗忙道:“姨娘这会子又何用叫裁缝赶去,我前儿倒做了两套,拿来给他岂不省事。
况且他活着的时候也穿过我的旧衣服,身量又相对。
”王夫人道:“虽然这样,难道你不忌讳?”宝钗笑道:“姨娘放心,我从来不计较这些。
”一面说,一面起身就走。
王夫人忙叫了两个人来跟宝姑娘去。
\ping{在王夫人面前,宝钗又加分了,而黛玉又扣分了。
}\par
一时宝钗取了衣服回来,只见宝玉在王夫人旁边坐着垂泪。
王夫人正才说他,因宝钗来了,却掩了口不说了。
\meng{云龙现影法,可爱煞人。
}宝钗见此光景,察言观色,早知觉了八分,于是将衣服交割明白。
王夫人将他母亲叫来拿了去。
再看下回便知。
\par
\qi{总评:世上无情空大地,人间少爱景何穷。
其中世界其中了,\zhu{了:了结,结束。
}含笑同归造化功。
\zhu{
全诗在赞美“情”,说如果“世上无情”、“人间少爱”,那就大地也要“空”,景物也要“穷”了,
所以有情人无论为情遭受多少痛苦,也甘心“其中世界其中了”,并“含笑”承受,认为“情”是“造化”的体现,要与情同在,在情中获得归宿感。
}
\hang
袭人、湘云、黛玉、宝钗等之爱之哭,各具一心,各具一见。
而宝玉、黛玉之痴情痴性,行文如绘,真是现身说法,岂三家村老学究之可能梦见者!
\zhu{三家村:人烟稀少,偏僻的小乡村。}
不禁炷香再拜!}
\dai{063}{林黛玉听到宝玉和湘云谈论仕途经济}
\dai{064}{含耻辱情烈死金钏}
\sun{p32-1}{史湘云翠缕论阴阳,湘云偶拾金麒麟,诉肺腑心迷活宝玉}{图右侧:史湘云一路上给翠缕解说阴阳,正说着,拾到一个金麒麟。
忽见宝玉来了,湘云忙将其藏起。
图左侧:黛玉听见湘云说仕途经济之事,宝玉却说:“林妹妹从不说这些混账话。
”黛玉听了又喜又惊又悲又叹,待要进去,又抽身回来,热泪滚滚。
宝玉追来,倾诉肺腑,安慰黛玉。
袭人前来送扇,袭人听了个正着,惊疑不止。
}