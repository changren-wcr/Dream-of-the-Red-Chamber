\chapter{惑奸谗抄检大观园 \quad 矢孤介杜绝宁国府}
\zhu{矢孤介:誓守孤介之志。
矢:誓,用作动词。
孤介:孤高耿介,不喜与世俗之人交往。
}
\par
\qi{司棋一事,在七十一回叙明,暗用山石伏线,七十三回用绣春囊在山石上一逗便住,至此回可直叙去,又用无数曲折渐渐逼来,及至司棋,忽然顿住,结到入画,文气如黄河出昆仑,横流数万里,九曲至龙门,又有孟门、吕梁峡束,不得入海。
是何等奇险怪特文字,令我拜服!}\par
话说平儿听迎春说了正自好笑,忽见宝玉也来了。
原来管厨房柳家媳妇之妹,也因放头开赌得了不是。
\zhu{放头:这里是聚赌、作头家的意思。
抽头:向赢钱的赌徒抽取一部分的利益给提供赌博场所的人。
也称为“拈头”。
头家:聚赌抽头的人。
聚赌抽头所得的钱叫头儿钱。
}这园中有素与柳家不睦的,\geng{前文已卯之伏线。
\zhu{令人费解,“卯”可能是“有”。
可能的情况是,早期抄录者的行草字迹“有”和“卯”形状相似,其后的抄录者,在抄写早期抄录者的过录本时,因文化修养不高,不认识草书,也不深究批语文意,就依样画葫芦,又画走了样,字体形状更像“卯”。
存世版本的抄录者看到这个字之后,理解为“卯”字。
虽然上下文不通顺,但是还是忠实保留了原貌,未加以修改,工整地写为“卯”字。
}}便又告出柳家来,说他和他妹子是伙计,虽然他妹子出名,其实赚了钱两个人平分。
因此凤姐要治柳家之罪。
那柳家的因得此信,便慌了手脚,因思素与怡红院人最为深厚,故走来悄悄地央求晴雯、金星玻璃等人。
金星玻璃告诉了宝玉。
\zhu{金星玻璃:芳官在第六十三回由宝玉改名为金星玻璃,简称玻璃。
}宝玉因思内中迎春之乳母也现有此罪,不若来约同迎春讨情,比自己独去单为柳家说情又更妥当,故此前来。
忽见许多人在此,见他来时,都问:“你的病可好了?跑来作什么?”宝玉不便说出讨情一事,只说:“来看二姐姐。
”当下众人也不在意,且说些闲话。
平儿便出去办累丝金凤一事。
那王住儿媳妇紧跟在后,口内百般央求,只说:“姑娘好歹口内超生,\zhu{超生:宽宥其生命。
常用于祈求他人怜悯救助。
}我横竖去赎了来。
”平儿笑道:“你迟也赎,早也赎,既有今日,何必当初。
你的意思得过去就过去了。
既是这样,我也不好意思告人,趁早去赎了来交与我送去,我一字不提。
”王住儿媳妇听说,方放下心来,就拜谢,又说:“姑娘自去贵干,\zhu{贵干:称人所作之事的敬词。
}我赶晚拿了来,先回了姑娘,再送去,如何?”平儿道:“赶晚不来,可别怨我。
”说毕,二人方分路各自散了。
\par
平儿到房,凤姐问他:“三姑娘叫你作什么?”平儿笑道:“三姑娘怕奶奶生气,叫我劝着奶奶些,问奶奶这两天可吃些什么。
”凤姐笑道:“倒是他还记挂着我。
刚才又出来了一件事:有人来告柳二媳妇和他妹子通同开局,凡妹子所为,都是他作主。
我想,你素日肯劝我‘多一事不如省一事’,就可闲一时心,自己保养保养也是好的。
我因听不进去,果然应了些,先把太太得罪了,而且自己反赚了一场病。
如今我也看破了,随他们闹去罢,横竖还有许多人呢。
我白操一会子心,倒惹的万人咒骂。
我且养病要紧,便是好了,我也作个好好先生,得乐且乐,得笑且笑,一概是非都凭他们去罢。
\geng{历来世人到此作此想,但悔不及矣。
可伤可叹。
\zhu{悔不及:应该是“追悔莫及”的意思。
}}所以我只答应着知道了,白不在我心上。
”\zhu{白:竟。
与“不”连用。
}平儿笑道:“奶奶果然如此,便是我们的造化。
”\par
一语未了,只见贾琏进来,拍手叹气道:“好好的又生事。
前儿我和鸳鸯借当,那边太太怎么知道了。
才刚太太叫过我去,叫我不管那里先迁挪二百银子,做八月十五日节间使用。
我回没处迁挪。
太太就说:‘你没有钱就有地方迁挪,我白和你商量,\zhu{白:单单,只是。
}你就搪塞我,你就说没地方。
前儿一千银子的当是那里的?连老太太的东西你都有神通弄出来,这会子二百银子,你就这样。
幸亏我没和别人说去。
’我想太太分明不短,何苦来要寻事奈何人。
”凤姐儿道:“那日并没一个外人,谁走了这个消息?”平儿听了,也细想那日有谁在此,想了半日,笑道:“是了。
那日说话时没一个外人,但晚上送东西来的时节,老太太那边傻大姐的娘也可巧来送浆洗衣服。
他在下房里坐了一会子,见一大箱子东西,自然要问,必是小丫头们不知道,说了出来,也未可知。
”\geng{奇奇怪怪,从何处转至素日\sout{成}[来],真如常山之蛇。
}因此便唤了几个小丫头来问,那日谁告诉呆大姐的娘。
众小丫头慌了,都跪下赌咒发誓,说:“自来也不敢多说一句话。
有人凡问什么,都答应不知道。
这事如何敢多说。
”凤姐详情说:\zhu{详情:审察情理。
}“他们必不敢,倒别委屈了他们。
如今且把这事靠后,且把太太打发了去要紧。
宁可咱们短些,又别讨没意思。
”因叫平儿:“把我的金项圈拿来,且去暂押二百银子来送去完事。
”贾琏道:“越性多押二百,咱们也要使呢。
”凤姐道:“很不必,我没处使钱。
这一去还不知指那一项赎呢。
”平儿拿去,吩咐一个人唤了旺儿媳妇来领去,不一时拿了银子来。
贾琏亲自送去,不在话下。
\par
这里凤姐和平儿猜疑,终是谁人走的风声,竟拟不出人来。
凤姐儿又道:“知道这事还是小事,怕的是小人趁便又造非言,生出别的事来。
当紧那边正和鸳鸯结下仇了,如今听得他私自借给琏二爷东西,那起小人眼馋肚饱,连没缝儿的鸡蛋还要下蛆呢,如今有了这个因由,恐怕又造出些没天理的话来也定不得。
在你琏二爷还无妨,只是鸳鸯正经女儿,带累了他受屈,岂不是咱们的过失。
”平儿笑道:“这也无妨。
鸳鸯借东西看的是奶奶,并不为的是二爷。
一则鸳鸯虽应名是他私情,其实他是回过老太太的。
老太太因怕孙男弟女多,这个也借,那个也要,到跟前撒个娇儿,和谁要去,因此只装不知道。
\geng{奇文神文!岂世人\sout{余相}[意想]得出者?前文云“一箱子”若是私拿出,贾母其睡梦中之人矣。
盖此等事作者曾经,批者曾经,实系一写往事,非特造出,故弄新笔,究竟不记不神也。
\zhu{不记不神:意思是夸赞作者写的都是神文。
}}\geng{鸳鸯借物一回于此便结了。
}纵闹了出来,究竟那也无碍。
”凤姐儿道:“理固如此。
只是你我是知道的,那不知道的,焉得不生疑呢。
”\par
一语未了,人报:“太太来了。
”凤姐听了诧异,不知为何事亲来,与平儿等忙迎出来。
只见王夫人气色更变,\geng{奇。
}只带一个贴己的小丫头走来,一语不发,走至里间坐下。
凤姐忙奉茶,因陪笑问道:“太太今日高兴,到这里逛逛。
”王夫人喝命:“平儿出去!”平儿见了这般,着慌不知怎么样了,忙应了一声,带着众小丫头一齐出去,在房门外站住,越性将房门掩了,自己坐在台矶上,所有的人,一个不许进去。
凤姐也着了慌,不知有何等事。
只见王夫人含着泪,从袖内掷出一个香袋子来,说:“你瞧。
”凤姐忙拾起一看,见是十锦春意香袋,\zhu{十锦:即“什锦”,由多种原料制成或多种花样拼成的。
春意:两性爱恋的情意。
}也吓了一跳,忙问:“太太从那里得来?”王夫人见问,越发泪如雨下,颤声说道:“我从那里得来!我天天坐在井里,拿你当个细心人,所以我才偷个空儿。
谁知你也和我一样。
这样的东西大天白日明摆在园里山石上,被老太太的丫头拾着,不亏你婆婆遇见,早已送到老太太跟前去了。
\ping{王夫人也是个老实人,她根本没想到邢夫人是趁机来整她的儿媳妇的,如果婆婆真的关心儿媳妇,私下里直接给儿媳妇就好了,干吗要弄到她姑妈那里去?}
我且问你,这个东西如何遗在那里来?”\geng{奇问。
}凤姐听得,也更了颜色,忙问:“太太怎知是我的?”\geng{问的是。
}王夫人又哭又叹说道:“你反问我!你想,一家子除了你们小夫小妻,馀者老婆子们,要这个何用?再女孩子们是从那里得来?自然是那琏儿不长进下流种子那里弄来。
你们又和气。
当作一件顽意儿,年轻人儿女闺房私意是有的,你还和我赖!幸而园内上下人还不解事,尚未拣得。
倘或丫头们拣着,你姊妹看见,这还了得。
不然有那小丫头们拣着,出去说是园内拣着的,外人知道,这性命脸面要也不要?”\par
凤姐听说,又急又愧,登时紫涨了面皮,便依炕沿双膝跪下,也含泪诉道:“太太说的固然有理,我也不敢辩我并无这样的东西。
但其中还要求太太细详其理:那香袋是外头雇工仿着内工绣的,带子穗子一概是市卖货。
我便年轻不尊重些,也不要这劳什子,自然都是好的,此其一。
二者这东西也不是常带着的,我纵有,也只好在家里,焉肯带在身上各处去?况且又在园里去,个个姊妹我们都肯拉拉扯扯,倘或露出来,不但在姊妹前,就是奴才看见,我有什么意思?我虽年轻不尊重,亦不能糊涂至此。
三则论主子内我是年轻媳妇,算起奴才来,比我更年轻的又不止一个人了。
况且他们也常进园,晚间各人家去,焉知不是他们身上的?四则除我常在园里之外,还有那边太太常带过几个小姨娘来,如嫣红翠云等人,皆系年轻侍妾,他们更该有这个了。
还有那边珍大嫂子,他不算甚老外,他也常带过配凤等人来,焉知又不是他们的?五则园内丫头太多,保的住个个都是正经的不成?也有年纪大些的知道了人事,或者一时半刻人查问不到偷着出去,或借着因由同二门上小幺儿们打牙犯嘴,\zhu{
小幺儿:身边使唤的小仆人。 
打牙犯嘴:打牙:说闲话。
犯嘴:用冷言冷语讥笑、逗弄他人。
打牙犯嘴指说闲话,互相嘲弄戏骂。
也作“打牙撩嘴”、“打牙撂嘴”。
}外头得了来的,也未可知。
如今不但我没此事,就连平儿我也可以下保的。
太太请细想。
”\par
王夫人听了这一席话大近情理,因叹道:“你起来。
我也知道你是大家小姐出身,焉得轻薄至此,不过我气急了,拿了话激你。
但如今却怎么处?你婆婆才打发人封了这个给我瞧,说是前日从傻大姐手里得的,把我气了个死。
”凤姐道:“太太快别生气。
若被众人觉察了,保不定老太太不知道。
且平心静气暗暗访察,才得确实,纵然访不着,外人也不能知道。
这叫作‘胳膊折在袖内’。
如今惟有趁着赌钱的因由革了许多的人这空儿,把周瑞媳妇旺儿媳妇等四五个贴近不能走话的人安插在园里,以查赌为由。
再如今他们的丫头也太多了,保不住人大心大,生事作耗,等闹出事来,反悔之不及。
如今若无故裁革,不但姑娘们委屈烦恼,就连太太和我也过不去。
不如趁此机会,以后凡年纪大些的,或有些咬牙难缠的,拿个错儿撵出去配了人。
一则保得住没有别的事,二则也可省些用度。
太太想我这话如何?”王夫人叹道:“你说的何尝不是,但从公细想,你这几个姊妹也甚可怜了。
\geng{犹云“可怜”,妙!又在别人视之,今古无比矣;若在荣府论,实不能比先矣。
}也不用远比,只说如今你林妹妹的母亲,未出阁时,是何等的娇生惯养,是何等的金尊玉贵,那才像个千金小姐的体统。
如今这几个姊妹,不过比人家的丫头略强些罢了。
\geng{所谓“观于海者难为水”,俗子谓王夫人不知足,是不可矣,又认作太过,真蟪蛄、学鸠之见也。
\zhu{蟪蛄[huìgū]:蝉的一种。
学:通“鷽”(“鸴”),鸟名。
鸠[jiū]:鸟名。
庄子《逍遥游》:蜩与学鸠笑之曰:“我决起而飞,抢榆枋而止,时则不至,而控于地而已矣,奚以之九万里而南为?”适莽苍者,三餐而反,腹犹果然;适百里者,宿舂粮,适千里者,三月聚粮。之二虫又何知?小知不及大知,小年不及大年。奚以知其然也?朝菌不知晦朔,蟪蛄不知春秋,此小年也。楚之南有冥灵者,以五百岁为春,五百岁为秋。上古有大椿者,以八千岁为春,八千岁为秋。此大年也。
}}通共每人只有两三个丫头像个人样,馀者纵有四五个小丫头子,竟是庙里的小鬼。
如今还要裁革了去,不但于我心不忍,只怕老太太未必就依。
虽然艰难,难不至此。
我虽没受过大荣华富贵,比你们是强的。
如今我宁可省些,别委曲了他们。
以后要省俭先从我来倒使的。
如今且叫人传了周瑞家的等人进来,就吩咐他们快快暗地访拿这事要紧。
”凤姐听了,即唤平儿进来吩咐出去。
\par
一时,周瑞家的与吴兴家的、郑华家的、来旺家的、来喜家的现在五家陪房进来,馀者皆在南方各有执事。
\geng{又伏一笔。
}王夫人正嫌人少不能勘察,忽见邢夫人的陪房王善保家的走来,方才正是他送香囊来的。
王夫人向来看视邢夫人之得力心腹人等原无二意,\geng{大书看下人犹如此,可知待邢夫人矣。
}今见他来打听此事,十分关切,\geng{小人外是内非,\sout{委}[悉]皆如此。
}便向他说:“你去回了太太,也进园内照管照管,不比别人又强些。
”\ping{王夫人以此堵邢夫人的嘴,邢夫人是长媳但是却无管家地位,王夫人处理的不好,邢夫人更有的说了,王善保家的充当邢夫人监督代言人。
}\par
这王善保家正因素日进园去那些丫鬟们不大趋奉他,他心里大不自在,要寻他们的故事又寻不着,恰好生出这事来,以为得了把柄。
又听王夫人委托,正撞在心坎上,说:“这个容易。
不是奴才多话,论理这事该早严紧的。
太太也不大往园里去,这些女孩子们一个个倒像受了封诰似的,他们就成了千金小姐了。
闹下天来,谁敢哼一声儿。
不然,就调唆姑娘的丫头们,说欺负了姑娘们了,谁还耽得起。
”王夫人道:“这也有的常情,跟姑娘的丫头原比别的娇贵些。
你们该劝他们。
连主子们的姑娘不教导尚且不堪,何况他们。
”王善保家的道:“别的都还罢了。
太太不知道,一个宝玉屋里的晴雯,那丫头仗着他生的模样儿比别人标致些,又生了一张巧嘴,天天打扮的像个西施的样子,在人跟前能说惯道,掐尖要强。
一句话不投机,他就立起两个骚眼睛来骂人,妖妖趫趫,\zhu{妖妖趫趫:妖冶轻佻的样子。
趫:音“乔”,行动轻捷,这里有举止轻浮的意思。
}
大不成个体统。
”\geng{活画出晴雯来。
可知已前知晴雯必应遭妒者,可怜可伤,竟死矣。
}\par
王夫人听了这话,猛然触动往事,\ping{往事可能是曾经王夫人就觉得晴雯有不妥的地方,现在被人指出来,更合了过去心上的疑虑。
这里的“往事”也可能是指王夫人的丈夫贾政和赵姨娘的往事,这里王夫人把晴雯和赵姨娘做类比,赵姨娘依靠美色分走了贾政的爱,所以王夫人不想让晴雯也分走自己儿子贾宝玉对自己未来心仪的儿媳妇的爱。
}便问凤姐道:“上次我们跟了老太太进园逛去,有一个水蛇腰,\geng{妙妙,好腰!}削肩膀,\geng{妙妙,好肩!俗云:“水蛇腰则游曲小也。
”又云:“美人无肩。
”又曰:“肩若削成。
”皆是美之形也。
凡写美人皆用俗笔反笔,与他书不同也。
}眉眼又有些像你林妹妹的,\geng{更好,形容尽矣。
}
\ping{晴为黛影。}
正在那里骂小丫头。
我的心里很看不上那狂样子,因同老太太走,我不曾说得。
后来要问是谁,又偏忘了。
今日对了坎儿,这丫头想必就是他了。
”\ping{王夫人不喜欢像林妹妹的人,其实也不喜欢林妹妹。
}凤姐道:“若论这些丫头们,共总比起来,都没晴雯生得好。
论举止言语,他原有些轻薄。
方才太太说的倒很像他,我也忘了那日的事,不敢乱说。
”王善保家的便道:“不用这样,此刻不难叫了他来太太瞧瞧。
”王夫人道:“宝玉房里常见我的只有袭人麝月,这两个笨笨的倒好。
若有这个,他自不敢来见我的。
我一生最嫌这样人,况且又出来这个事。
好好的宝玉,倘或叫这蹄子勾引坏了,那还了得。
”因叫自己的丫头来,吩咐他:“到园里去,只说我说有话问他们,留下袭人麝月伏侍宝玉不必来,有一个晴雯最伶俐,叫他即刻快来。
你不许和他说什么。
”\par
小丫头子答应了,走入怡红院,正值晴雯身上不自在,睡中觉才起来,正发闷,听如此说,只得随了他来。
素日这些丫鬟皆知王夫人最嫌趫妆艳饰语薄言轻者,故晴雯不敢出头。
今因连日不自在,\geng{\sout{音}[摹]神之至!所谓“魂早离舍”矣,将死之兆也。
若俗笔必云十分妆饰,今云不自在,想无挂心之态,\zhu{挂心:挂念,牵念。
}更不入王夫人之眼也。
\ping{为何“更不入王夫人之眼”?挂心是指对于王夫人要见自己这件事情上心,晴雯因为身体不适所以难以聚集精神,面对王夫人显得心不在焉不够郑重,更让王夫人厌恶。
}}并没十分妆饰,自为无碍。
\geng{好!可知天生美人原不在妆饰,使人一见便觉心惊目骇。
可恨世之涂脂抹粉,真同鬼魅而不见觉。
}
及到了凤姐房中,王夫人一见他钗軃鬓松,\zhu{
軃:音“朵”,下垂的样子。
钗軃:发髻上的钗饰将要脱落。
}衫垂带褪,有春睡捧心之遗风,\zhu{春睡捧心之遗风:春睡:本喻杨贵妃之醉态,《明皇杂录》:“上尝登沉香亭,召妃子。
妃子时卯酒未醒,高力士从侍儿扶掖而至。
上皇笑曰:岂是妃子醉耶?海棠睡未足耳。
”。
捧心:相传春秋时越国美女西施因病捧心皱眉,显得更美。
遗风:即馀风,前人遗留下来的风韵、风致。
这里讥刺女子的娇慵病弱。
}而且形容面貌恰是上月的那人,不觉勾起方才的火来。
\par
王夫人原是天真烂漫之人,喜怒出于心臆,不比那些饰词掩意之人,今既真怒攻心,又勾起往事,便冷笑道:“好个美人!真像个病西施了。
你天天作这轻狂样儿给谁看?你干的事,打量我不知道呢!我且放着你,自然明儿揭你的皮!宝玉今日可好些?”晴雯一听如此说,心内大异,便知有人暗算了他。
虽然着恼,只不敢作声。
他本是个聪敏过顶的人,\geng{深罪聪明,到底不错一笔。
}见问宝玉可好些,他便不肯以实话对,只说:“我不大到宝玉房里去,又不常和宝玉在一处,好歹我不能知道,只问袭人麝月两个。
”王夫人道:“这就该打嘴!你难道是死人,要你们作什么!”晴雯道:“我原是跟老太太的人。
因老太太说园里空大人少,宝玉害怕,所以拨了我去外间屋里上夜,不过填屋子。
我原回过我笨,不能伏侍。
老太太骂了我,说:‘又不叫你管他的事,要伶俐的作什么。
’我听了这话才去的。
不过十天半个月之内,宝玉闷了大家顽一会子就散了。
至于宝玉饮食起坐,上一层有老奶奶老妈妈们,下一层又有袭人、麝月、秋纹几个人。
我闲着还要作老太太屋里的针线,所以宝玉的事竟不曾留心。
太太既怪,从此后我留心就是了。
”
\ping{晴雯在回答王夫人时,通过点出贾母,暗示自己的背景,希望王夫人能投鼠忌器。}
王夫人信以为实了,忙说:“阿弥陀佛!你不近宝玉是我的造化,竟不劳你费心。
既是老太太给宝玉的,我明儿回了老太太,再撵你。
”因向王善保家的道:“你们进去,好生防他几日,不许他在宝玉房里睡觉。
等我回过老太太,再处治他。
”
\ping{后文可知,晴雯被赶走后,王夫人才趁贾母喜欢以晴雯得痨病为借口告诉了贾母。王夫人没有提前回过贾母,而是先斩后奏。}
喝声“去!站在这里,我看不上这浪样儿!谁许你这样花红柳绿的妆扮!”晴雯只得出来,这气非同小可,一出门便拿手帕子握着脸,一头走,一头哭,直哭到园门内去。
\par
这里王夫人向凤姐等自怨道:“这几年我越发精神短了,照顾不到。
这样妖精似的东西竟没看见。
只怕这样的还有,明日倒得查查。
”凤姐见王夫人盛怒之际,又因王善保家的是邢夫人的耳目,常调唆着邢夫人生事,纵有千百样言词,此刻也不敢说,只低头答应着。
王善保家的道:“太太请养息身体要紧,这些小事只交与奴才。
如今要查这个主儿也极容易,等到晚上园门关了的时节,内外不通风,我们竟给他们个猛不防,带着人到各处丫头们房里搜寻。
想来谁有这个,断不单只有这个,自然还有别的东西。
那时翻出别的来,自然这个也是他的。
”王夫人道:“这话倒是。
若不如此,断不能清的清白的白。
”因问凤姐如何。
凤姐只得答应说:“太太说的是,就行罢了。
”王夫人道:“这主意很是,不然一年也查不出来。
”于是大家商议已定。
\par
至晚饭后,待贾母安寝了,宝钗等入园时,王善保家的便请了凤姐一并入园,喝命将角门皆上锁,便从上夜的婆子处抄检起,不过抄检出些多馀攒下蜡烛灯油等物。
\geng{毕真。
}王善保家的道:“这也是赃,不许动,等明儿回过太太再动。
”于是先就到怡红院中,喝命关门。
当下宝玉正因晴雯不自在,忽见这一干人来,不知为何直扑了丫头们的房门去,因迎出凤姐来,问是何故。
凤姐道:“丢了一件要紧的东西,因大家混赖,恐怕有丫头们偷了,所以大家都查一查去疑。
”一面说,一面坐下吃茶。
王善保家的等搜了一回,又细问这几个箱子是谁的,都叫本人来亲自打开。
袭人因见晴雯这样,知道必有异事,又见这番抄检,只得自己先出来打开了箱子并匣子,任其搜检一番,不过是平常动用之物。
随放下又搜别人的,挨次都一一搜过。
到了晴雯的箱子,因问:“是谁的,怎不开了让搜?”袭人等方欲代晴雯开时,只见晴雯挽着头发闯进来,豁一声将箱子掀开,两手捉着,底子朝天往地下尽情一倒,将所有之物尽都倒出。
王善保家的也觉没趣\foot{按:此处程本比诸脂本多了以下一段文字:便紫涨了脸,说道:“姑娘你别生气。
我们并非私自就来的,原是奉太太的命来搜察。
你们叫翻呢,我们就翻一翻,不叫翻,我们还许回太太去呢!那用急的这个样子。
”晴雯听了这话,越发火上浇油,便指着他的脸说道:“你说你是太太打发来的,我还是老太太打发来的呢!太太那边的人我也都见过,就只没看见你这么个有头有脸大管事的奶奶。
”凤姐见晴雯说话锋利尖酸,心中甚喜,却碍着邢夫人的脸,忙喝住晴雯。
那王善保家的又羞又气,刚要还言,凤姐道:“妈妈,你也不必合他们一般见识,你且细细搜你的。
咱们还到各处走走呢!再迟了走了风,我可担不起。
”王善保家的只得咬咬牙,且忍了这口气,细细的……一般来说,程本对脂本所作删改,及个别补缀缺文,均乏善可陈。
此处多出的二百馀字与前后文比较连贯,描写也还生动,所以有人认为当另有所据,或者就是曹雪芹原有文字。
其实,这段文字虽然读来很解气,但对表现晴雯的性格显得过火了。
它就算是曹雪芹的文字,也只能是被删去的初稿文字。
},看了一看,也无甚私弊之物。
回了凤姐,要往别处去。
凤姐儿道:“你们可细细的查,若这一番查不出来,难回话的。
”众人都道:“都细翻看了,没什么差错东西。
虽有几样男人物件,都是小孩子的东西,想是宝玉的旧物件,没甚关系的。
”凤姐听了,笑道:“既如此咱们就走,再瞧别处去。
”\par
说着,一径出来,因向王善保家的道:“我有一句话,不知是不是。
要抄检只抄检咱们家的人,薛大姑娘屋里,断乎检抄不得的。
”王善保家的笑道:“这个自然。
岂有抄起亲戚家来。
”凤姐点头道:“我也这样说呢。
”\geng{写阿凤心灰意懒,且避祸从时,迥又是一个人矣。
}一头说,一头到了潇湘馆内。
黛玉已睡了,忽报这些人来,也不知为甚事。
才要起来,只见凤姐已走进来,忙按住他不许起来,只说:“睡罢,我们就走。
”这边且说些闲话。
那个王善保家的带了众人到丫鬟房中,也一一开箱倒笼抄检了一番。
\ping{从这里也可以看出,薛宝钗被认为是亲戚,而林黛玉被认为是自己家的人。
}因从紫鹃房中抄出两副宝玉常换下来的寄名符儿,\zhu{常:通“尝”,曾经。
}一副束带上的披带,两个荷包并扇套,套内有扇子。
打开看时皆是宝玉往年往日手内曾拿过的。
王善保家的自为得了意,遂忙请凤姐过来验视,又说:“这些东西从那里来的?”凤姐笑道:“宝玉和他们从小儿在一处混了几年,这自然是宝玉的旧东西。
这也不算什么罕事,撂下再往别处去是正经。
”紫鹃笑道:“直到如今,我们两下里的东西也算不清。
要问这一个,连我也忘了是那年月日有的了。
”王善保家的听凤姐如此说,也只得罢了。
\geng{一处一样。
}\par
又到探春院内,谁知早有人报与探春了。
探春也就猜着必有原故,所以引出这等丑态来,\geng{实注一笔。
}遂命众丫鬟秉烛开门而待。
一时,众人来了。
探春故问何事。
凤姐笑道:“因丢了一件东西,连日访察不出人来,恐怕旁人赖这些女孩子们,所以越性大家搜一搜,使人去疑,倒是洗净他们的好法子。
”探春冷笑道:“我们的丫头自然都是些贼,我就是头一个窝主。
既如此,先来搜我的箱柜,他们所有偷了来的都交给我藏着呢。
”说着便命丫头们把箱柜一齐打开,将镜奁、妆盒、衾袱、衣包若大若小之物一齐打开,请凤姐去抄阅。
凤姐陪笑道:“我不过是奉太太的命来,妹妹别错怪我。
何必生气。
”因命丫鬟们快快关上。
\par
平儿丰儿等忙着替待书等关的关,收的收。
探春道:“我的东西倒许你们搜阅,要想搜我的丫头,这却不能。
我原比众人歹毒,凡丫头所有的东西我都知道,都在我这里间收着,一针一线他们也没的收藏,要搜所以只来搜我。
你们不依,只管去回太太,只说我违背了太太,该怎么处治,我去自领。
你们别忙,自然连你们抄的日子有呢!你们今日早起不曾议论甄家,自己家里好好的抄家,果然今日真抄了。
\geng{奇极!此曰甄家事。
}咱们也渐渐的来了。
可知这样大族人家,若从外头杀来,一时是杀不死的,这是古人曾说的‘百足之虫,死而不僵’,必须先从家里自杀自灭起来,才能一败涂地!”说着,不觉流下泪来。
\par
凤姐只看着众媳妇们。
周瑞家的便道:“既是女孩子的东西全在这里,奶奶且请到别处去罢,也让姑娘好安寝。
”凤姐便起身告辞。
探春道:“可细细的搜明白了?若明日再来,我就不依了。
”凤姐笑道:“既然丫头们的东西都在这里,就不必搜了。
”探春冷笑道:“你果然倒乖。
连我的包袱都打开了,还说没翻。
明日敢说我护着丫头们,不许你们翻了。
你趁早说明,若还要翻,不妨再翻一遍。
”凤姐知道探春素日与众不同的,只得陪笑道:“我已经连你的东西都搜查明白了。
”探春又问众人:“你们也都搜明白了不曾?”周瑞家的等都陪笑说:“都翻明白了。
”\par
那王善保家的本是个心内没成算的人,素日虽闻探春的名,那是为众人没眼力没胆量罢了,\zhu{为:因为。
}那里一个姑娘家就这样起来,况且又是庶出,他敢怎么。
他自恃是邢夫人陪房,连王夫人尚另眼相看,何况别个。
今见探春如此,他只当是探春认真单恼凤姐,与他们无干。
他便要趁势作脸献好,\zhu{作脸:争光,争气,争脸面,出风头。
}因越众向前拉起探春的衣襟,故意一掀,嘻嘻笑道:“连姑娘身上我都翻了,果然没有什么。
”凤姐见他这样,忙说:“妈妈走罢,别疯疯颠颠的。
”一语未了,只听“拍”的一声,王家的脸上早着了探春一掌。
探春登时大怒,指着王家的问道:“你是什么东西,敢来拉扯我的衣裳!我不过看着太太的面上,你又有年纪,叫你一声妈妈,你就狗仗人势,天天作耗,专管生事。
如今越性了不得了。
你打量我是同你们姑娘那样好性儿,由着你们欺负他,就错了主意!你搜检东西我不恼,你不该拿我取笑。
”说着,便亲自解衣卸裙,拉着凤姐儿细细的翻。
又说:“省得叫奴才来翻我身上。
”凤姐平儿等忙与探春束裙整袂,\zhu{袂:衣袖。
}口内喝着王善保家的说:“妈妈吃两口酒就疯疯颠颠起来。
前儿把太太也冲撞了。
快出去,不要提起了。
”又劝探春休得生气。
探春冷笑道:“我但凡有气性,早一头碰死了!不然岂许奴才来我身上翻贼赃了。
明儿一早,我先回过老太太、太太,然后过去给大娘陪礼,该怎么,我就领。
”\par
那王善保家的讨了个没意思,在窗外只说:“罢了,罢了,这也是头一遭挨打。
我明儿回了太太,仍回老娘家去罢。
这个老命还要他做什么!”探春喝命丫鬟道:“你们听他说的这话!还等我和他对嘴去不成?”待书等听说,便出去说道:“你果然回老娘家去,倒是我们的造化了。
只怕舍不得去。
”凤姐笑道:“好丫头,真是有其主必有其仆。
”探春冷笑道:“我们作贼的人,嘴里都有三言两语的。
这还算笨的,背地里就只不会调唆主子。
”平儿忙也陪笑解劝,一面又拉了待书进来。
周瑞家的等人劝了一番。
凤姐直待伏侍探春睡下,方带着人往对过暖香坞来。
\par
彼时李纨犹病在床上,他与惜春是紧邻,又与探春相近,故顺路先到这两处。
因李纨才吃了药睡着,不好惊动,只到丫鬟们房中一一的搜了一遍,也没有什么东西,遂到惜春房中来。
因惜春年少,尚未识事,吓的不知当有什么事,故凤姐也少不得安慰他。
谁知竟在入画箱中寻出一大包金银锞子来,\zhu{锞子(锞音“课”):旧时做货币用的小金锭或银锭。
}约共三四十个,\geng{奇。
为察奸情,反得贼赃。
}又有一副玉带板子并一包男人的靴袜等物。
\zhu{玉带板子:男子腰带上的玉质带头。
}入画也黄了脸。
因问是那里来的,入画只得跪下哭诉真情,说:“这是珍大爷赏我哥哥的。
\geng{妙极是极。
盖入画本系宁府之人也。
}因我们老子娘都在南方,如今只跟着叔叔过日子。
我叔叔婶子只要吃酒赌钱,我哥哥怕交给他们又花了,所以每常得了,\zhu{每常:平时,平常。
}
悄悄的烦了老妈妈带进来叫我收着的。
”\par
惜春胆小,见了这个也害怕,说:“我竟不知道。
这还了得!二嫂子,你要打他,好歹带他出去打罢,我听不惯的。
”凤姐笑道:“这话若果真呢,也倒可恕,只是不该私自传送进来。
这个可以传递,什么不可以传递。
这倒是传递人的不是了。
若这话不真,倘是偷来的,你可就别想活了。
”入画跪着哭道:“我不敢扯谎。
奶奶只管明日问我们奶奶和大爷去,若说不是赏的,就拿我和我哥哥一同打死无怨。
”凤姐道:“这个自然要问的,只是真赏的也有不是。
谁许你私自传送东西的!你且说是谁作接应,我便饶你。
下次万万不可。
”惜春道:“嫂子别饶他这次方可。
这里人多,若不拿一个人作法,\zhu{作法:就是树立某种标准,给别人立规矩,通过责骂、惩罚等手段处理某人立威,杀鸡儆猴,以儆其馀。
}那些大的听见了,又不知怎样呢。
嫂子若饶他,我也不依。
”\geng{这是自己也不依的。
各得自然之理,各有自然之妙。
}凤姐道:“素日我看他还好。
谁没一个错,只这一次。
二次犯下,二罪俱罚。
但不知传递是谁。
”惜春道:“若说传递,再无别个,必是后门上的张妈。
他常肯和这些丫头们鬼鬼祟祟的,这些丫头们也都肯照顾他。
”凤姐听说,便命人记下,将东西且交给周瑞家的暂拿着,等明日对明再议。
于是别了惜春,方往迎春房内来。
\par
迎春已经睡着了,丫鬟们也才要睡,众人叩门半日才开。
凤姐吩咐:“不必惊动小姐。
”遂往丫鬟们房里来。
因司棋是王善保的外孙女儿,\geng{玄妙奇诡,出人意外。
}凤姐倒要看看王家的可藏私不藏,遂留神看他搜检。
先从别人箱子搜起,皆无别物。
及到了司棋箱子中搜了一回,王善保家的说:“也没有什么东西。
”才要盖箱时,周瑞家的道:“且住,这是什么?”说着,便伸手掣出一双男子的锦带袜并一双缎鞋来。
\zhu{掣:音“撤”:拽,拉。
}
\geng{险极!}又有一个小包袱,打开看时,里面有一个同心如意并一个字帖儿。
一总递与凤姐。
凤姐因当家理事,每每看开帖并帐目,也颇识得几个字了。
便看那帖子是大红双喜笺帖,\geng{纸就好。
余为司棋心动。
}上面写道:\par
\hop
上月你来家后,父母已觉察你我之意。
但姑娘未出阁,尚不能完你我之心愿。
若园内可以相见,你可托张妈给一信息。
若得在园内一见,倒比来家得说话。
千万,千万。
再所赐香袋二个,今已查收外,特寄香珠一串,略表我心。
千万收好。
表弟潘又安拜具。
\zhu{拜具:犹敬具。
书信结尾的敬辞。
用于署名之后。
如:某某拜具。
}
\geng{名字便妙。
\zhu{潘安:即潘安仁,晋代文人,著名美男子。}
}\par
\hop
凤姐看罢,不怒而反乐。
\geng{恶毒之至。
}别人并不识字。
王家的素日并不知道他姑表姊弟有这一节风流故事,见了这鞋袜,心内已是有些毛病,又见有一红帖,凤姐又看着笑,他便说道:“必是他们胡写的帐目,不成个字,所以奶奶见笑。
”凤姐笑道:“正是这个帐竟算不过来。
你是司棋的老娘,他的表弟也该姓王,怎么又姓潘呢?”王善保家的见问的奇怪,只得勉强告道:“司棋的姑妈给了潘家,所以他姑表兄弟姓潘。
上次逃走了的潘又安就是他表弟。
”凤姐笑道:“这就是了。
”因道:“我念给你听听。
”说着从头念了一遍,大家都唬了一跳。
这王家的一心只要拿人的错儿,不想反拿住了他外孙女儿,又气又臊。
周瑞家的四人又都问着他:“你老可听见了?明明白白,再没的话说了。
如今据你老人家,该怎么样?”这王家的只恨没地缝儿钻进去。
凤姐只瞅着他嘻嘻的笑,\geng{恶毒之至。
}向周瑞家的笑道:“这倒也好。
不用你们作老娘的操一点儿心,他鸦雀不闻的给你们弄了一个好女婿来,大家倒省心。
”\geng{刻毒之至!}\geng{按凤姐虽系刻毒,然亦不应在下人前为不\sout{寻}[尊]。
}\geng{此等人前不得不如是也。
}
周瑞家的也笑着凑趣儿。
王家的气无处泄,便自己回手打着自己的脸,骂道:“老不死的娼妇,怎么造下孽了!说嘴打嘴,\zhu{说嘴打嘴:才夸口就出丑。
}现世现报在人眼里。
”众人见这般,俱笑个不住,又半劝半调的。
\zhu{调:挑逗,嘲笑。
}凤姐见司棋低头不语,也并无畏惧惭愧之意,倒觉可异。
料此时夜深,且不必盘问,只怕他夜间自愧去寻拙志,\zhu{寻拙志:寻短见,自杀。
}遂唤两个婆子监守起他来。
带了人,拿了赃证回来,且自安歇,等待明日料理。
谁知到夜里又连起来几次,下面淋血不止。
\par
至次日,便觉身体十分软弱,起来发晕,遂撑不住。
请太医来,诊脉毕,遂立药案云:“看得少奶奶系心气不足,虚火乘脾,\zhu{虚火乘脾:乘:乘虚侵袭。
五行:金木水火土。
五行相生:木生火,火生土,土生金,金生水,水生木。
五行相克:金克木,木克土,土克水,水克火,火克金。
五行与五脏相对应,木对肝,火对心,土对脾,金对肺,水对肾。
五行(人体五脏)相克太过,失却正常协调叫相乘。
如木气(肝火)偏亢,而金(肺)不能对木加以正常克制时,太过的木(虚火),便去乘土(伤脾胃),就会出现“胃虚土弱”的病症。
}皆由忧劳所伤,以致嗜卧好眠,胃虚土弱,不思饮食。
今聊用升阳养荣之剂。
”\zhu{聊:姑且,暂且。
升阳养荣:升阳:是一种治疗脾失健运、消化力弱、不能上输精气的方法。
养荣:即养营,是一种治心气虚、血不能正常运行等病症的营养周身之法。
}写毕,遂开了几样药名,不过是人参,当归,黄芪等类之剂。
一时退去,有老嬷嬷们拿了方子回过王夫人,不免又添一番愁闷。
遂将司棋等事暂且未理。
\par
可巧这日尤氏来看凤姐,坐了一回,到园中去又看过李纨。
才要望候众姊妹们去,忽见惜春遣人来请,尤氏遂到了他房中来。
惜春便将昨晚之事细细告诉与尤氏,又命将入画的东西一概要来与尤氏过目。
尤氏道:“实是你哥哥赏他哥哥的,只不该私自传送,如今官盐竟成了私盐了。
”\zhu{官盐竟成了私盐:旧时,由国家运销或已经缴纳盐税为官方许可经营的食盐称为官盐,若逃避纳税私运私销则称为私盐,要受到官府的取缔。
这里比喻本系主人赏赐的“合法”的东西,因私自传送倒成为“不合法”的了。
}因骂入画,“糊涂脂油蒙了心的。
”惜春道:“你们管教不严,反骂丫头。
这些姊妹,独我的丫头这样没脸,我如何去见人。
昨儿我立逼着凤姐姐带了他去,他只不肯。
我想,他原是那边的人,凤姐姐不带他去,也原有理。
我今日正要送过去,嫂子来的恰好,快带了他去。
或打,或杀,或卖,我一概不管。
”入画听说,又跪下哭求,说:“再不敢了。
只求姑娘看从小儿的情常,\zhu{情常:情分。
}好歹生死在一处罢。
”尤氏和奶娘等人也都十分分解,\zhu{分解:分辩,解释。
}说他“不过一时糊涂了,下次再不敢的。
他从小儿伏侍你一场,到底留着他为是。
”\par
谁知惜春虽然年幼,却天生地一种百折不回的廉介孤独僻性,\zhu{廉介:清廉耿介。
}任人怎说,他只以为丢了他的体面,咬定牙断乎不肯。
更又说的好:“不但不要入画,如今我也大了,连我也不便往你们那边去了。
况且近日我每每风闻得有人背地里议论什么,多少不堪的闲话,我若再去,连我也编派上了。
”尤氏道:“谁议论什么?又有什么可议论的!姑娘是谁,我们是谁。
姑娘既听见人议论我们,就该问着他才是。
”惜春冷笑道:“你这话问着我倒好。
我一个姑娘家,只有躲是非的,我反去寻是非,成个什么人了!还有一句话,我不怕你恼:好歹自有公论,又何必去问人。
古人说得好,‘善恶生死,父子不能有所勖助’,\zhu{勖(音“序”)助:勉励帮助。
}何况你我二人之间。
我只知道保得住我就够了,不管你们。
从此以后,你们有事别累我。
”尤氏听了,又气又好笑,因向地下众人道:“怪道人人都说这四丫头年轻糊涂,我只不信。
你们听才一篇话,\zhu{才:刚刚。
}无原无故,又不知好歹,又没个轻重。
虽然是小孩子的话,却又能寒人的心。
”众嬷嬷笑道:“姑娘年轻,奶奶自然要吃些亏的。
”惜春冷笑道:“我虽年轻,这话却不年轻。
你们不看书不识几个字,所以都是些呆子,看着明白人,倒说我年轻糊涂。
”尤氏道:“你是状元、榜眼、探花,\zhu{状元、榜眼、探花:明清时代科举制度以廷试一甲(一等)第一名为状元,第二名为榜眼,第三名为探花。
}古今第一个才子。
我们是糊涂人,不如你明白,何如?”惜春道:“状元、榜眼难道就没有糊涂的不成?可知他们更有不能了悟的。
”尤氏笑道:“你倒好。
才是才子,这会子又作大和尚了,又讲起了悟来了。
”惜春道:“我不了悟,我也舍不得入画了。
”尤氏道:“可知你是个心冷口冷、心狠意狠的人。
”惜春道:“古人曾也说的,‘不作狠心人,难得自了汉’。
\zhu{不作狠心人,难得自了汉:意谓不下狠心断绝欲念便不能摒弃种种烦恼。
自了汉:俗称只管自身、不顾大局者为自了汉。
}我清清白白的一个人,为什么教你们带累坏了我!”\ping{宁国府的丑事,连惜春这样的小孩子都知道,可见人尽皆知。
}\par
尤氏心内原有病,怕说这些话。
听说有人议论,已是心中羞恼激射,只是在惜春分上不好发作,忍耐了大半。
今见惜春又说这句,因按捺不住,因问惜春道:“怎么就带累了你了?你的丫头的不是,无故说我,我倒忍了这半日,你倒越发得了意,只管说这些话。
你是千金万金的小姐,我们以后就不亲近,仔细带累了小姐的美名。
即刻就叫人将入画带了过去!”说着,便赌气起身去了。
惜春道:“若果然不来,倒也省了口舌是非,大家倒还清净。
”尤氏也不答话,一径往前边去了。
不知后事如何——\par
\qi{总评:诸院皆宴息,\zhu{宴息:休息。
}独探春秉烛以待,大有提防,的是干才,须另置一席款待。
\hang
凤姐喜事,\zhu{喜事:喜欢揽事。
}忽作打破虚空之语,惜春年幼,偏有老成练达之操,\zhu{操:品行,德行。
}世态何常,\zhu{常:永久的,固定的。
引申为常规,准则。
}知人其难!}
\dai{147}{探春怒扇王善保家的}
\dai{148}{凤姐当众念司棋潘又安情书}
\sun{p74-1}{因春囊重托王善保}{王夫人误以为园内发现的绣春囊是凤姐贾琏之物,遂来质问凤姐。
凤姐澄清不是自己的,并献计暗中访查。
邢夫人的心腹王善保家的趁机提议,等晚上园门关了,突然搜查。
王夫人同意,凤姐附和。
}
\sun{p74-2}{探春凛然不乱抗搜检}{来到探春院内,探春遂命众丫鬟秉烛开门而待,把自己的箱柜一齐打开,道“我是窝主,先来找我箱柜,要想找我丫头,却不能。
”王善保家的以为探春恼的是凤姐,拉起探春的衣襟,说:“连姑娘身上我都翻了,果然没有什么。
”话音未落,只听“啪”的一声,探春扇了她一个耳光。
}