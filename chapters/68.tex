\chapter{苦尤娘赚入大观园 \quad 酸凤姐大闹宁国府}
\qi{余读《左氏》见郑庄,\zhu{《左氏》:全称《春秋左氏传》,又名《左传》,是为《春秋》做注解的一部史书。
郑庄:指的是《左传》中郑伯克段于鄢(鄢音“烟”)的文章,内容梗概为,郑庄公的母亲武姜,因厌恶庄公,偏爱小儿子共叔段,于是帮助共叔段谋取王位。
共叔段暗中积蓄力量,阴谋发动叛乱,夺取君位。
郑庄公欲擒故纵,以退为进,等待时机,最后一举打败共叔段,共叔段逃离郑国。
郑庄公老谋深算,阴险狡猾。
主要表现在对自己的同胞兄弟“纵其欲而使之放,养其恶而使其成”(宋人吕祖谦语,见《东莱博议》),充分暴露共叔段的“不义”。
他尽量满足母亲和弟弟越轨请求和行为,不加劝阻教育,并驳回大臣们的建议。
但当共叔段“将袭郑”时,他先发制人,一举把他赶到了“共”,绝除后患。
}读《后汉》见魏武,谓古之大奸巨滑,惟此为最。
今读《石头记》,又见凤姐作威作福,用柔用刚,占步高,留步宽,杀得死,救得活。
天生此等人,斲丧元气不少!\zhu{斲:音“浊”,斧头,引申为砍、削。
}}\par
话说贾琏起身去后,偏值平安节度巡边在外,约一个月方回。
贾琏未得确信,只得住在下处等候。
及至回来相见,将事办妥,回程已是将两个月的限了。
\zhu{将:差不多。
}\par
谁知凤姐心下早已算定,只待贾琏前脚走了,回来便传各色匠役,收拾东厢房三间,照依自己正室一样装饰陈设。
至十四日便回明贾母王夫人,说十五日一早要到姑子庙进香去。
只带了平儿、丰儿、周瑞媳妇、旺儿媳妇四人,未曾上车,便将原故告诉了众人。
又吩咐众男人,素衣素盖,一径前来。
\par
兴儿引路,一直到了二姐门前扣门。
鲍二家的开了。
兴儿笑说:“快回二奶奶去,大奶奶来了。
”鲍二家的听了这句,顶梁骨走了真魂,忙飞进报与尤二姐。
尤二姐虽也一惊,但已来了,只得以礼相见,于是忙整衣迎了出来。
至门前,凤姐方下车进来。
尤二姐一看,只见头上皆是素白银器,身上月白缎袄,青缎披风,白绫素裙。
\ping{
王熙凤白衣素服有一个下马威的意思,根据国法家法,守孝期间不能婚嫁。
}
眉弯柳叶,高吊两梢,目横丹凤,神凝三角。
\zhu{丹凤、三角:第三回描写王熙凤出场时的相貌:“一双丹凤三角眼”。
丹凤眼:用丹顶凤凰细长的眼睛来比照说明某人的眼睛形状,形容眼角向上微翘。
三角眼:眼睛呈三角状。
}俏丽若三春之桃,清洁若九秋之菊。
周瑞旺儿二女人搀入院来。
尤二姐陪笑忙迎上来万福,\zhu{万福:这里指旧日女子与人相见时的一种礼节,也叫“福”。
行礼时上身略弯,两手抱拳在胸前右上方上下移动。
}张口便叫:“姐姐下降,不曾远接,望恕仓促之罪。
”说着便福了下来。
凤姐忙陪笑还礼不迭。
二人携手同入室中。
\par
凤姐上座,尤二姐命丫鬟拿褥子来便行礼,说:“奴家年轻,一从到了这里之事,\zhu{一从:自从。
}皆系家母和家姐商议主张。
今日有幸相会,若姐姐不弃奴家寒微,凡事求姐姐的指示教训。
奴亦倾心吐胆,只伏侍姐姐。
”说着,便行下礼去。
凤姐儿忙下座以礼相还,口内忙说:“皆因奴家妇人之见,一味劝夫慎重,不可在外眠花卧柳,恐惹父母担忧。
此皆是你我之痴心,怎奈二爷错会奴意。
眠花宿柳之事瞒奴或可,今娶姐姐二房之大事亦人家大礼,亦不曾对奴说。
奴亦曾劝二爷早行此礼,以备生育。
不想二爷反以奴为那等嫉妒之妇,私自行此大事,并不说知。
使奴有冤难诉,惟天地可表。
前于十日之先奴已风闻,恐二爷不乐,遂不敢先说。
今可巧远行在外,故奴家亲自拜见过,还求姐姐下体奴心,起动大驾,挪至家中。
你我姊妹同居同处,彼此合心谏劝二爷,慎重世务,保养身体,方是大礼。
若姐姐在外,奴在内,虽愚贱不堪相伴,奴心又何安。
再者,使外人闻知,亦甚不雅观。
二爷之名也要紧,倒是谈论奴家,奴亦不怨。
所以今生今世奴之名节全在姐姐身上。
那起下人小人之言,未免见我素日持家太严,背后加减些言语,自是常情。
姐姐乃何等样人物,岂可信真。
若我实有不好之处,上头三层公婆,中有无数姊妹妯娌,况贾府世代名家,岂容我到今日。
今日二爷私娶姐姐在外,若别人则怒,我则以为幸。
正是天地神佛不忍我被小人们诽谤,故生此事。
我今来求姐姐进去和我一样同居同处,同分同例,同侍公婆,同谏丈夫。
喜则同喜,悲则同悲,情似亲妹,和比骨肉。
不但那起小人见了,自悔从前错认了我,就是二爷来家一见,他作丈夫之人,心中也未免暗悔。
所以姐姐竟是我的大恩人,使我从前之名一洗无馀了。
若姐姐不随奴去,奴亦情愿在此相陪。
奴愿作妹子,每日伏侍姐姐梳头洗面。
只求姐姐在二爷跟前替我好言方便方便,容我一席之地安身,奴死也愿意。
”说着,便呜呜咽咽哭将起来。
尤二姐见了这般,也不免滴下泪来。
\par
二人对见了礼,\zhu{见礼:见面行礼。
}分序座下。
平儿忙也上来要见礼。
尤二姐见他打扮不凡,举止品貌不俗,料定是平儿,连忙亲身挽住,只叫:“妹子快休如此,你我是一样的人。
”凤姐忙也起身笑说:“折死他了!妹子只管受礼,他原是咱们的丫头。
以后快别如此。
”说着,又命周家的从包袱里取出四匹上色尺头,四对金珠簪环为拜礼。
尤二姐忙拜受了。
二人吃茶,对诉已往之事。
凤姐口内全是自怨自错,“怨不得别人,如今只求姐姐疼我”等语。
尤二姐见了这般,便认他作是个极好的人,小人不遂心诽谤主子亦是常理,故倾心吐胆,叙了一回,竟把凤姐认为知己。
又见周瑞等媳妇在旁边称扬凤姐素日许多善政,只是吃亏心太痴了,惹人怨,又说“已经预备了房屋,奶奶进去一看便知。
”尤氏心中早已要进去同住方好,今又见如此,岂有不允之理,便说:“原该跟了姐姐去,只是这里怎样?”凤姐儿道:“这有何难,姐姐的箱笼细软只管着小厮搬了进去。
这些粗笨货要他无用,还叫人看着。
姐姐说谁妥当就叫谁在这里。
”尤二姐忙说:“今日既遇见姐姐,这一进去,凡事只凭姐姐料理。
我也来的日子浅,也不曾当过家,世事不明白,如何敢作主。
这几件箱笼拿进去罢。
我也没有什么东西,那也不过是二爷的。
”凤姐听了,便命周瑞家的记清,好生看管着抬到东厢房去。
于是催着尤二姐穿戴了,二人携手上车,又同坐一处,又悄悄的告诉他:“我们家的规矩大。
这事老太太一概不知,倘或知二爷孝中娶你,管把他打死了。
如今且别见老太太、太太,我们有一个花园子极大,姊妹住着,容易没人去的。
你这一去且在园里住两天,等我设个法子回明白了,那时再见方妥。
”尤二姐道:“任凭姐姐裁处。
”那些跟车的小厮们皆是预先说明的,如今不去大门,只奔后门而来。
\par
下了车,赶散众人。
凤姐便带尤氏进了大观园的后门,来到李纨处相见了。
彼时大观园中十停人已有九停人知道了,今忽见凤姐带了进来,引动多人来看问。
尤二姐一一见过。
众人见他标致和悦,无不称扬。
凤姐一一的吩咐了众人:“都不许在外走了风声,若老太太、太太知道,我先叫你们死。
”园中婆子丫鬟都素惧凤姐的,又系贾琏国孝家孝中所行之事,知道关系非常,都不管这事。
凤姐悄悄的求李纨收养几日,“等回明了,我们自然过去的。
”李纨见凤姐那边已收拾房屋,况在服中,不好倡扬,自是正理,只得收下权住。
凤姐又变法将他的丫头一概退出,又将自己的一个丫头送他使唤。
暗暗吩咐园中媳妇们:“好生照看着他。
若有走失逃亡,一概和你们算帐。
”自己又去暗中行事。
合家之人都暗暗纳罕的说:“看他如何这等贤惠起来了。
” 那尤二姐得了这个所在,又见园中姊妹各各相好,倒也安心乐业的自为得其所矣。
\par
谁知三日之后,丫头善姐便有些不服使唤起来。
尤二姐因说:“没了头油了,你去回声大奶奶拿些来。
”善姐便道:“二奶奶,你怎么不知好歹没眼色。
我们奶奶天天承应了老太太,又要承应这边太太那边太太。
这些妯娌姊妹,上下几百男女,天天起来,都等他的话。
一日少说,大事也有一二十件,小事还有三五十件。
外头的从娘娘算起,以及王公侯伯家多少人情客礼,家里又有这些亲友的调度。
银子上千钱上万,一日都从他一个手一个心一个口里调度,那里为这点子小事去烦琐他。
我劝你能着些儿罢。
\zhu{能:通“耐”,禁得起,受得住,忍受。
}咱们又不是明媒正娶来的,这是他亘古少有一个贤良人才这样待你,若差些儿的人,听见了这话,吵嚷起来,把你丢在外,死不死,生不生,你又敢怎样呢!”一席话,说的尤氏垂了头,自为有这一说,少不得将就些罢了。
那善姐渐渐连饭也怕端来与他吃,或早一顿,或晚一顿,所拿来之物,皆是剩的。
尤二姐说过两次,他反先乱叫起来。
尤二姐又怕人笑他不安分,少不得忍着。
隔上五日八日见凤姐一面,那凤姐却是和容悦色,满嘴里姐姐不离口。
又说:“倘有下人不到之处,你降不住他们,只管告诉我,我打他们。
”又骂丫头媳妇说:“我深知你们,软的欺,硬的怕,背开我的眼,还怕谁。
倘或二奶奶告诉我一个不字,我要你们的命。
”尤氏见他这般的好心,思想“既有他,何必我又多事。
下人不知好歹,也是常情。
我若告了,他们受了委屈,反叫人说我不贤良。
”因此反替他们遮掩。
\ping{尤二姐从风尘从良,反而对获得贤良的名声更加渴望。
}\par
凤姐一面使旺儿在外打听细事,这尤二姐之事皆已深知。
原来已有了婆家的,女婿现在才十九岁,成日在外嫖赌,不理生业,\zhu{生业:产业、职业。
}家私花尽,父亲撵他出来,现在赌钱厂存身。
\zhu{赌钱厂:开赌场的地方。
}父亲得了尤婆十两银子退了亲的,这女婿尚不知道。
原来这小伙子名叫张华。
凤姐都一一尽知原委,便封了二十两银子与旺儿,悄悄命他将张华勾来养活,着他写一张状子,只管往有司衙门中告去,就告琏二爷“国孝家孝之中,背旨瞒亲,仗财依势,强逼退亲,停妻再娶”等语。
这张华也深知利害,先不敢造次。
旺儿回了凤姐,凤姐气的骂:“癞狗扶不上墙的种子。
你细细的说给他,便告我们家谋反也没事的。
不过是借他一闹,大家没脸。
若告大了,我这里自然能够平息的。
”旺儿领命,只得细说与张华。
凤姐又吩咐旺儿:“他若告了你,你就和他对词去。
”如此如此,这般这般,“我自有道理。
”旺儿听了有他做主,便又命张华状子上添上自己,说:“你只告我来往过付,\zhu{过付:双方交易时,由中间人经手往来钱货。
}一应调唆二爷做的。
”张华便得了主意,和旺儿商议定了,写了一纸状子,次日便往都察院喊了冤。
\par
察院坐堂看状,见是告贾琏的事,上面有家人旺儿一人,只得遣人去贾府传旺儿来对词。
青衣不敢擅入,\zhu{青衣:这里指穿黑衣的衙役,即“皂隶”。
}只命人带信。
那旺儿正等着此事,不用人带信,早在这条街上等候。
见了青衣,反迎上去笑道:“起动众位兄弟,必是兄弟的事犯了。
说不得,快来套上。
”众青衣不敢,只说:“你老去罢,别闹了。
”于是来至堂前跪了。
察院命将状子与他看。
旺儿故意看了一遍,碰头说道:“这事小的尽知,小的主人实有此事。
但这张华素与小的有仇,故意攀扯小的在内。
其中还有别人,求老爷再问。
”张华碰头说:“虽还有人,小的不敢告他,所以只告他下人。
”旺儿故意急的说:“糊涂东西,还不快说出来!这是朝廷公堂之上,凭是主子,也要说出来。
”张华便说出贾蓉来。
察院听了无法,只得去传贾蓉。
凤姐又差了庆儿暗中打听,告了起来,便忙将王信唤来,告诉他此事,命他托察院只虚张声势警唬而已,又拿了三百银子与他去打点。
是夜王信到了察院私第,安了根子。
\zhu{安根子:暗中行使贿赂,打通关节。
}那察院深知原委,收了赃银。
次日回堂,只说张华无赖,因拖欠了贾府银两,枉捏虚词,诬赖良人。
都察院又素与王子腾相好,王信也只到家说了一声,况是贾府之人,巴不得了事,便也不提此事,且都收下,只传贾蓉对词。
\par
且说贾蓉等正忙着贾珍之事,忽有人来报信,说有人告你们如此如此,这般这般,快作道理。
贾蓉慌了,忙来回贾珍。
贾珍说:“我防了这一着,只亏他大胆子。
”即刻封了二百银子着人去打点察院,又命家人去对词。
正商议之间,人报:“西府二奶奶来了。
”贾珍听了这个,倒吃了一惊,忙要同贾蓉藏躲。
不想凤姐进来了,说:“好大哥哥,带着兄弟们干的好事!”贾蓉忙请安,凤姐拉了他就进来。
贾珍还笑说:“好生伺候你婶娘,吩咐他们杀牲口备饭。
”说了,忙命备马,躲往别处去了。
\par
这里凤姐儿带着贾蓉走来上房,尤氏正迎了出来,见凤姐气色不善,忙笑说:“什么事这等忙?”凤姐照脸一口唾沫啐道:“你尤家的丫头没人要了,偷着只往贾家送!难道贾家的人都是好的,普天下死绝了男人了!你就愿意给,也要三媒六证,大家说明,成个体统才是。
你痰迷了心,脂油蒙了窍,国孝家孝两重在身,就把个人送来了。
这会子被人家告我们,我又是个没脚蟹,\zhu{没脚蟹:比喻行动不得,手足无措。
}连官场中都知道我利害吃醋,如今指名提我,要休我。
我来了你家,干错了什么不是,你这等害我?或是老太太、太太有了话在你心里,使你们做这圈套,要挤我出去。
如今咱们两个一同去见官,分证明白。
回来咱们公同请了合族中人,\zhu{公同:一起,共同。
}大家觌面说个明白。
\zhu{
觌:音“敌”,见;相见。
觌面:当面;会面。
}
给我休书,我就走路。
”一面说,一面大哭,拉着尤氏,只要去见官。
急的贾蓉跪在地下碰头,只求“姑娘婶子息怒。
”\zhu{姑娘:在本书中指姑姑,用在这里不合适,王熙凤是贾蓉的婶子,而不是姑姑。
庚辰本为“姑娘婶子”,己卯本、蒙本、戚本、列藏本与甲辰本则为“姑娘婶婶”。可能初稿此处原为“姑娘”两字,后来曹雪芹或脂砚发现漏洞就在稿本上点去或圈去这两字,旁添“婶子”或“婶婶”两字,可能由于点或圈的墨迹较淡,为后来的抄写者所忽视,以致把这四个字都抄上了。
至于初稿中为什么会出现让人困惑的“姑娘”二字,刘世德认为原因在于第六十八回是“红楼二尤”的一部分,而讲二尤故事的这几回原先属于曹雪芹的旧稿《风月宝鉴》,在那部旧稿中凤姐和贾蓉之间的亲属关系很可能为姑侄。
但我们看到贾蓉称凤姐为“姑娘”不止出现在第六十八回,还出现在第五十三回,它与二尤故事似乎沾不上边。这一奇怪的称呼与贾珍和凤姐之间特殊的关系有关。
贾珍与凤姐当面称对方为“大哥哥”、“大妹妹”,而且邢夫人、王夫人对贾珍都称凤姐为“你大妹妹”,两人的关系几同兄妹。
书中对此也有交代,第十三回贾珍恳请王夫人让凤姐协理宁国府,他相信凤姐能够胜任,因为“从小儿大妹妹顽笑着就有杀伐决断”,可见贾珍对凤姐从小就熟悉。
第五十四回凤姐也说她和贾珍“我们还是论哥哥妹妹,从小儿一处淘气了这么大”。既然两人是“论哥哥妹妹”,贾蓉称凤姐为“姑娘”也就无妨了。
}凤姐儿一面又骂贾蓉:“天雷劈脑子五鬼分尸的没良心的种子!不知天有多高,地有多厚,成日家调三窝四,\zhu{调三窝四:搬弄口舌、挑拨是非。
也作“调三斡四”、“挑三豁四”、“挑三窝四”。
}干出这些没脸面没王法败家破业的营生。
你死了的娘阴灵也不容你,祖宗也不容,还敢来劝我!”哭骂着扬手就打。
贾蓉忙磕头有声说:“婶子别动气,仔细手,让我自己打。
婶子别生气。
”说着,自己举手左右开弓自己打了一顿嘴巴子,又自己问着自己说:“以后可再顾三不顾四的混管闲事了?以后还单听叔叔的话不听婶子的话了?”众人又是劝,又要笑,又不敢笑。
\par
凤姐儿滚到尤氏怀里,嚎天动地,大放悲声,只说:“给你兄弟娶亲我不恼。
为什么使他违旨背亲,将混帐名儿给我背着?咱们只去见官,省得捕快皂隶来拿。
\zhu{捕快:衙门里担任缉拿人犯的差役,也称为“捕人”、“捕役”、“步快”。
皂:黑色。
隶:差役。
皂隶:这里指穿黑衣的衙役,即“青衣”。
}再者咱们只过去见了老太太、太太和众族人,大家公议了,我既不贤良,又不容丈夫娶亲买妾,只给我一纸休书,我即刻就走。
你妹妹我也亲身接来家,生怕老太太、太太生气,也不敢回,现在三茶六饭金奴银婢的住在园里。
我这里赶着收拾房子,一样和我的道理,\zhu{道理:处理或打算。
这里应该是安排的意思。
己卯本作“和我一样的道理”。
梦稿、戚序、甲辰本均作“和我的一样”。
}只等老太太知道了。
原说接过来大家安分守己的,我也不提旧事了。
谁知又有了人家的。
不知你们干的什么事,我一概又不知道。
如今告我,我昨日急了,纵然我出去见官,也丢的是你贾家的脸,少不得偷把太太的五百两银子去打点。
\ping{花费了三百,却说花了五百。
委托人按照采购和卖家协商的价格完成交易,那么存在采购和卖家串通作弊,标高售价。
比正常价格高的那部分,流入了卖家和采购的口袋。
}如今把我的人还锁在那里。
”说了又哭,哭了又骂,后来放声大哭起祖宗爹妈来,又要寻死撞头。
把个尤氏揉搓成一个面团,衣服上全是眼泪鼻涕,并无别语,只骂贾蓉:“孽障种子!和你老子作的好事!我就说不好的。
”凤姐儿听说,哭着两手搬着尤氏的脸紧对相问道:“你发昏了?你的嘴里难道有茄子塞着?不然他们给你嚼子衔上了?\zhu{嚼子:为了便于驾驭牲口和驯服动物,或防止动物伤人,横放在牲口或动物嘴里的小铁链或其它形状的铁制品,两端连在笼头或缰绳上,多用于马、骡子、牛等大个的牲口,偶尔也能用于犬科动物。
马嚼环\foot{
\footPic{马嚼环}{jiaozi.jpg}{0.5}
}:勒在马口里的小铁链,也称马嚼子,借以控制马匹的活动。
骑手一拉缰绳,马嚼子就被拉进马嘴巴里,骑手就这样来控制马匹的行进速度或者让马停步。
}为什么你不告诉我去?你若告诉了我,这会子平安不了?怎得经官动府,闹到这步田地,你这会子还怨他们。
自古说:‘妻贤夫祸少,表壮不如里壮。
’\zhu{表壮不如里壮:俗语。
“表”指丈夫,“里”指妻子。
意思是丈夫有才能还不如妻子能治家。
}你但凡是个好的,他们怎得闹出这些事来!你又没才干,又没口齿,锯了嘴子的葫芦,就只会一味瞎小心图贤良的名儿。
总是他们也不怕你,也不听你。
”说着啐了几口。
尤氏也哭道:“何曾不是这样。
你不信问问跟的人,我何曾不劝的,也得他们听。
叫我怎么样呢,怨不得妹妹生气,我只好听着罢了。
”\par
众姬妾、丫鬟、媳妇已是乌压压跪了一地,陪笑求说:“二奶奶最圣明的。
虽是我们奶奶的不是,奶奶也作践的够了。
当着奴才们,奶奶们素日何等的好来,如今还求奶奶给留脸。
”说着,捧上茶来。
凤姐也摔了,一面止了哭挽头发,又喝骂贾蓉:“出去请大哥哥来。
我对面问他,亲大爷的孝才五七,
\zhu{五七:即第五个“七”。人死后,每七天为一周期,请僧道祭祀超度,称为“七”。通常经七个“七”才掩灵停祭。}
侄儿娶亲,这个礼我竟不知道。
我问问,也好学着日后教导子侄的。
”贾蓉只跪着磕头,说:“这事原不与父母相干,都是儿子一时吃了屎,调唆叔叔作的。
我父亲也并不知道。
如今我父亲正要商量接太爷出殡,婶子若闹起来,儿子也是个死。
只求婶子责罚儿子,儿子谨领。
这官司还求婶子料理,儿子竟不能干这大事。
婶子是何等样人,岂不知俗语说的‘胳膊只折在袖子里’。
\zhu{胳膊只折在袖子里:俗语。
意思是家丑不可外扬,不好的事不要张扬出去。
}儿子糊涂死了,既作了不肖的事,\zhu{肖:像。
不肖:子不似父,不能继承父业;不贤,无才能;品性不良。
}就同那猫儿狗儿一般。
婶子既教训,就不和儿子一般见识的,少不得还要婶子费心费力将外头的事压住了才好。
原是婶子有这个不肖的儿子,既惹了祸,少不得委屈,还要疼儿子。
”说着,又磕头不绝。
\par
凤姐见他母子这般,也再难往前施展了,只得又转过了一副形容言谈来,与尤氏反陪礼说:“我是年轻不知事的人,一听见有人告诉了,把我吓昏了,不知方才怎样得罪了嫂子。
可是蓉儿说的‘胳膊折了往袖子里藏’,少不得嫂子要体谅我。
还要嫂子转替哥哥说了,先把这官司按下去才好。
”尤氏贾蓉一齐都说:“婶子放心,横竖一点儿连累不着叔叔。
婶子方才说用过了五百两银子,少不得我娘儿们打点五百两银子与婶子送过去,好补上的,不然岂有反教婶子又添上亏空之名,越发我们该死了。
但还有一件,老太太、太太们跟前婶子还要周全方便,别提这些话方好。
”\par
凤姐儿又冷笑道:“你们饶压着我的头干了事,\zhu{饶:不仅。
}这会子反哄着我替你们周全。
我虽然是个呆子,也呆不到如此。
嫂子的兄弟是我的丈夫,嫂子既怕他绝后,我岂不更比嫂子更怕绝后。
嫂子的令妹就是我的妹子一样。
我一听见这话,连夜喜欢的连觉也睡不成,赶着传人收拾了屋子,就要接进来同住。
倒是奴才小人的见识,他们倒说:‘奶奶太好性了。
若是我们的主意,先回了老太太、太太,看是怎样,再收拾房子去接也不迟。
’我听了这话,教我要打要骂的,才不言语。
谁知偏不称我的意,偏打我的嘴,半空里又跑出一个张华来告了一状。
我听见了,吓的两夜没合眼儿,又不敢声张,只得求人去打听这张华是什么人,这样大胆。
打听了两日,谁知是个无赖的花子。
\zhu{花子:乞丐。
也称为“叫花子”。
}我年轻不知事,反笑了,说:‘他告什么?’倒是小子们说:‘原是二奶奶许了他的。
他如今正是急了,冻死饿死也是个死,现在有这个理他抓着,纵然死了,死的倒比冻死饿死还值些。
怎么怨的他告呢。
这事原是爷做的太急了。
国孝一层罪,家孝一层罪,背着父母私娶一层罪,停妻再娶一层罪。
俗语说:拼着一身剐,敢把皇帝拉下马。
他穷疯了的人,什么事作不出来,况且他又拿着这满理,不告等请不成。
’嫂子说,我便是个韩信张良,听了这话,也把智谋吓回去了。
你兄弟又不在家,又没个商议,少不得拿钱去垫补,谁知越使钱越被人拿住了刀靶,\zhu{拿住刀靶:俗语,抓住把柄的意思。
}越发来讹。
我是‘耗子尾上长疮——多少脓血儿’。
\zhu{耗子尾上长疮——多少脓血儿:歇后语,老鼠尾巴非常细小,就是生了疮,也不会太大。
比喻才能有限,没多大的能耐。
也作“耗子尾巴上长疮”。
另一种说法,喻指钱财不够。
}所以又急又气,少不得来找嫂子。
”\par
尤氏、贾蓉不等说完,都说:“不必操心,自然要料理的。
”贾蓉又道:“那张华不过是穷急,故舍了命才告。
咱们如今想了一个法儿,竟许他些银子,只叫他应了妄告不实之罪,咱们替他打点完了官司,他出来时再给他些个银子就完了。
”凤姐儿笑道:“好孩子,怨不得你顾一不顾二的作这些事出来。
原来你竟糊涂。
若你说得这话,他暂且依了,且打出官司来又得了银子,眼前自然了事。
这些人既是无赖之徒,银子到手一旦光了,他又寻事故讹诈。
倘又叨登起来这事,咱们虽不怕,也终担心。
搁不住他说既没毛病为什么反给他银子,终久是不了之局。
”贾蓉原是个明白人,听如此一说,便笑道:“我还有个主意,‘来是是非人,去是是非者’,\zhu{来是是非人,去是是非者:由谁惹起的是非,还得由谁来了结,与“解铃还须系铃人”义近。
}这事还得我了才好。
如今我竟去问张华个主意,或是他定要人,或是他愿意了事得钱再娶。
他若说一定要人,少不得我去劝我二姨,叫他出来仍嫁他去;若说要钱,我们这里少不得给他。
”凤姐儿忙道:“虽如此说,我断舍不得你姨娘出去,我也断不肯使他去。
好侄儿,你若疼我,只能可多给他钱为是。
”\zhu{能可:宁可。
}贾蓉深知凤姐口虽如此,心却是巴不得只要本人出来,他却做贤良人。
如今怎说怎依。
\ping{贾蓉也不是省油的灯,自己看出了凤姐的不便明说的弦外之音,但是自己却依旧按照凤姐的表面意思来办。
死板教条极端化的“贯彻落实”是执行者消极反对命令者的最好办法。
毛泽东《反对本本主义》:“盲目地表面上完全无异议地执行上级的指示,这不是真正在执行上级的指示,这是反对上级指示或者对上级指示怠工的最妙方法。
”}\par
凤姐儿欢喜了,又说:“外头好处了,家里终久怎么样?你也同我过去回明才是。
”尤氏又慌了,拉凤姐讨主意如何撒谎才好。
凤姐冷笑道:“既没这本事,谁叫你干这事了。
这会子又这个腔儿,我又看不上。
待要不出个主意,\zhu{
这句话的意思是,等到需要出个主意的时候,又没个主意。
}我又是个心慈面软的人,凭人撮弄我,我还是一片痴心。
说不得让我应起来。
如今你们只别露面,我只领了你妹妹去与老太太、太太们磕头,只说原系你妹妹,我看上了很好。
正因我不大生长,\zhu{生长:生育。
}
原说买两个人放在屋里的,今既见你妹妹很好,而又是亲上做亲的,我愿意娶来做二房。
皆因家中父母姊妹新近一概死了,日子又艰难,不能度日,若等百日之后,无奈无家无业,实难等得。
我的主意接了进来,已经厢房收拾了出来暂且住着,等满了服再圆房。
\zhu{
满了服:服丧期满。
圆房:旧时嫁女有女子先到男家,虽有夫妻名分而不同房者,待适当时候方行同宿,称为“圆房”。
}仗着我不怕臊的脸,死活赖去,有了不是,也寻不着你们了。
你们母子想想,可使得?”尤氏贾蓉一齐笑说:“到底是婶子宽洪大量,足智多谋。
等事妥了,少不得我们娘儿们过去拜谢。
”尤氏忙命丫鬟们伏侍凤姐梳妆洗脸,又摆酒饭,亲自递酒拣菜。
\par
凤姐也不多坐,执意就走了。
进园中将此事告诉与尤二姐,又说我怎么操心打听,又怎么设法子,须得如此如此方救下众人无罪,少不得我去拆开这鱼头,\zhu{拆开这鱼头:也作“择鱼头”,比喻处理和排解复杂难办的事。
拆,这里读作“宅”,分解、清理的意思。
一说,把筵席上的鱼头拆开了好让大家吃,引申为与人方便,宁可自找麻烦。
}大家才好。
不知端详,且听下回分解。
\par
\qi{总评:人谓“闹宁国府”一节极凶猛,“赚尤二姐”一节极和蔼,吾谓“闹宁国府”情有可恕,“赚尤二姐”法不容诛,“闹宁国府”声声是泪,“赚尤二姐”字字皆锋。
}
\dai{135}{凤姐突访,平儿向尤二姐行礼}
\dai{136}{酸凤姐大闹宁国府}
\sun{p68-1}{酸凤姐大闹宁国府}{凤姐来到宁国府,一面说,一面大哭,拉着尤氏要去见官。
急的贾蓉跪在地下碰头,求凤姐息怒,凤姐又骂贾蓉。
}