\chapter{蜂腰桥设言传蜜意\quad 潇湘馆春困发幽情}
\qi{一个是时才得传消息,一个是旧喜化作新歌。
真真假假事堪疑,哭向花林月底。
\zhu{
第一句指“蜂腰桥设言传蜜意”,小红和贾芸通过坠儿互相交换手帕,暗通情愫。
第二句指“潇湘馆春困发幽情”,宝玉借《西厢记》词曲语句向黛玉婉转言情。宝玉和黛玉青梅竹马,情意由来有自,故曰 “旧喜”,“新歌”则指《西厢记》词曲的借用。
第三四句是针对本回结尾黛玉误会宝玉亲近宝钗而疏远自己,“独立墙角边花阴之下,悲悲戚戚呜咽起来”。
}
}\par
话说宝玉养过了三十三天之后,不但身体强壮,亦且连脸上疮痕平复,仍回大观园内去。
这也不在话下。
\par
且说近日宝玉病的时节,贾芸带着家下小厮坐更看守,\zhu{坐更:夜间警卫。
}昼夜在这里,那红玉同众丫鬟也在这里守着宝玉,彼此相见多日,都渐渐的混熟了。
那红玉见贾芸手里拿的手帕子,倒像是自己从前掉的,待要问他,又不好问的。
不料那和尚、道士来过,用不着一切男人,贾芸仍种树去了。
这件事待要放下,心内又放不下,待要问去,又怕人猜疑,正是犹豫不决、神魂不定之际,忽听窗外问道:“姐姐在屋里没有?”\jia{岔开正文,却是为正文作引。
}\geng{你看他偏不写正文,偏有许多闲文,却是补遗。
}红玉闻听,在窗眼内望外一看,原来是本院的小丫头名叫佳蕙的,因答说:“在家里,你进来罢。
”佳蕙听了跑进来,就坐在床上,笑道:“我好造化!才刚在院子里洗东西,宝玉叫往林姑娘那里送茶叶,\jia{交代井井有法。
}\geng{前文有言。
}花大姐姐交给我送去。
可巧老太太那里给林姑娘送钱来,\geng{是补写否?}正分给他们的丫头们呢。
\jia{潇湘常事出自别院婢口中,反觉新鲜。
}见我去了,林姑娘就抓了两把给我,也不知多少。
你替我收着。
”便把手帕子打开,把钱倒了出来,红玉替他一五一十的数了收起。
\geng{此等细事是旧族大家闺中常情,今特为暴发钱奴写来作鉴。
一笑。
壬午夏,雨窗。
}\par
佳蕙道:“你这一程子心里到底觉怎么样?\zhu{这一程子:最近一些天,这些日子。
}依我说,你竟家去住两日,请一个大夫来瞧瞧,吃两剂药就好了。
”红玉道:“那里的话,好好的,家去作什么!”佳蕙道:“我想起来了,林姑娘生的弱,时常他吃药,\geng{是补写否?}你就和他要些来吃,也是一样。
”\jia{闲言中叙出黛玉之弱。
草蛇灰线。
}红玉道:“胡说!\geng{如闻。
}药也是混吃的。
”佳蕙道:“你这也不是个长法儿,又懒吃懒喝的,终久怎么样?”\geng{从旁人眼中口中出,妙极!}红玉道:“怕什么,还不如早些儿死了倒干净!”\jia{此句令人气噎,总在无可奈何上来。
}佳蕙道:“好好的,怎么说这些话?”红玉道:“你那里知道我心里的事!”\par
佳蕙点头想了一会,道:“可也怨不得,这个地方难站。
就像昨儿老太太因宝玉病了这些日子,\geng{是补文否?}说跟着伏侍的这些人都辛苦了,如今身上好了,各处还完了愿,\geng{是补写否?}叫把跟着的人都按着等儿赏他们。
\geng{是补写否?}我算年纪小,上不去,不得我也不怨;像你怎么也不算在里头?\geng{道着心病。
}我心里就不服。
袭人那怕他得十个分儿,也不恼他,原该的。
说良心话,谁还敢比他呢?\geng{确论公论,方见袭卿身份。
}别说他素日殷勤小心,便是不殷勤小心,也拼不得。
可气晴雯、绮霰他们这几个,都算在上等里去,仗着老子娘的脸面,众人倒捧着他去。
\zhu{
从这里看,晴雯不仅有“老子娘”,而且她的“老子娘”在荣国府里还相当有“脸面”。林小红的父亲林之孝虽是贾府管家但“脸面”比不过晴雯的“老子娘”。
但是第七十七回写道晴雯是赖大家用银子买的,不记得家乡父母。只有个姑舅哥哥。依第七十七回之叙述,晴雯是个孤儿,或被拐卖。
这一矛盾体现了本书尚未完全成书付梓,其中既包含尚未完成的诗词(例如第二十二回灯谜处有脂批:“此后破失,俟再补”、“此回未成而芹逝矣”;第七十五回脂批:“缺中秋诗,俟雪芹”),也包含修改不彻底导致的矛盾(例如秦可卿悬梁自缢的结局,在第十三回被删去,但是在第五回判词中还保留)。
此处关于晴雯家世的交代,即为修改不彻底导致矛盾的情况。
}
你说可气不可气?”红玉道:“也不犯着气他们。
俗语说的,‘千里搭长棚,没有个不散的筵席’,\jia{此时写出此等言语,令人堕泪。
}
谁守谁一辈子呢?不过三年五载,各人干各人的去了。
那时谁还管谁呢?”这两句话不觉感动了佳蕙的心肠,\jia{不但佳蕙,批书者亦泪下矣。
}
由不得眼睛红了,又不好意思好端端的哭,只得勉强笑道:“你这话说的却是。
昨儿宝玉还说,\geng{还是补文。
}明儿怎么样收拾房子,怎么样做衣裳,倒像有几百年的熬煎。
”\jia{却是小女儿口中无味之谈,实是写宝玉不如一鬟婢。
}\par
红玉听了冷笑了两声,方要说话,\jia{文字又一顿。
}\jia{红玉一腔委屈怨愤,系身在怡红不能遂志,看官勿错认为芸儿害相思也。
[己卯冬。
]}\jia{“狱神庙”红玉、茜雪一大回文字惜迷失无稿。
}\geng{“狱神庙”回有茜雪、红玉一大回文字,惜迷失无稿。
叹叹!丁亥夏。
畸笏叟。
}只见一个未留头的小丫头子走进来,\zhu{留头:又叫“留满头”。
旧时女子幼年剃发,随着年事增长,先留顶心头发,再留全发,叫做“留头”。
}手里拿着些花样子并两张纸,说道:“这是两个样子,叫你描出来呢。
”说着向红玉掷下,回身就跑了。
红玉向外问道:“倒是谁的?也等不的说完就跑,谁蒸下馒头等着你,怕冷了不成!”那小丫头在窗外只说得一声:“是绮大姐姐的。
”\jia{又是不合式[之]言,\zhu{合式:合适。
}
擢心语。
\zhu{擢:音“卓”,拔,抽。
}}抬起脚来咕咚咕咚又跑了。
\jia{活龙活现之文。
}红玉便赌气把那样子掷在一边,\geng{何如?}向抽屉内找笔,找了半天都是秃了的,因说道:“前儿一枝新笔,\geng{是补文否?}放在那里了?怎么一时想不起来。
”\geng{既在矮檐下,怎敢不低头?}一面说,一面出神,\jia{总是画境。
}想了一会方笑道:“是了,前儿晚上莺儿拿了去了。
”\geng{还是补文。
}便向佳蕙道:“你替我取了来。
”佳蕙道:“花大姐姐还等着我替他抬箱子呢,你自取去罢。
”红玉道:“他等着你,你还坐着闲打牙儿?
\zhu{打牙:说闲话。}
\geng{袭人身份。
}我不叫你取去,他也不等着你了。
坏透了的小蹄子!”说着,自己便出房来,出了怡红院,一径往宝钗院内来。
\geng{曲折再四,方逼出正文来。
}\par
刚至沁芳亭畔,只见宝玉的奶娘李嬷嬷从那边走来。
\jia{奇文,真令人不得机关。
}红玉立住问道:“李奶奶,你老人家那去了?怎打这里来?”李嬷嬷站住,将手一拍道:“你说说,好好的又看上了\jia{囫囵不解语。
}
那个种树的什么云哥儿雨哥儿的,\jia{奇文神文。
}这会子逼着我叫了他来。
明儿叫上房里听见,可又是不好。
”\jia{更不解。
}红玉笑道:“你老人家当真的就依了他去叫了?”\jia{是遂心语。
}李嬷嬷道:“可怎么样呢?”\jia{妙!的是老妪口气。
}红玉笑道:“那一个要是知道好歹,\jia{更不解。
}
就回不进来才是。
”\jia{是私心语,神妙!}李嬷嬷道:“他又不痴,为什么不进来?”红玉道:“既是来了,你老人家该同他一齐来,回来叫他一个人乱碰,可是不好呢。
”\jia{总是私心语,要直问又不敢,只用这等语慢慢的套出。
有神理。
}李嬷嬷道:“我有那样工夫和他走?不过告诉了他,回来打发个小丫头子或是老婆子,带进他来就完了。
”说着,拄着拐一径去了。
红玉听说,便站着出神,且不去取笔。
\jia{总是不言神情,另出花样。
}\par
一时,只见一个小丫头子跑来,见红玉站在那里,便问道:“林姐姐,你在这里作什么呢?”红玉抬头见是小丫头子坠儿。
\jia{坠儿者,赘也。
人生天地间已是赘疣,\zhu{疣:音“油”,生在皮肤上的肉赘,通称瘊子。
}况又生许多冤情孽债。
叹叹!}红玉道:“那去?”坠儿道:“叫我带进芸二爷来。
”\geng{等的是这句话。
}说着一径跑了。
这里红玉刚走至蜂腰桥门前,只见那边坠儿引着贾芸来了。
\jia{妙!不说红玉不走,亦不说走,只说“刚走到”三字,可知红玉有私心矣。
若说出必定不走必定走,则文字死板,且亦棱角过露,非写女儿之笔也。
}那贾芸一面走,一面拿眼把红玉一溜;那红玉只装作和坠儿说话,也把眼去一溜贾芸。
四目恰相对时,红玉不觉脸红了,\jia{看官至此,须掩卷细想上三十回中篇篇句句点“红”字处,可与此处\sout{想}[相比]如何?}一扭身往蘅芜苑去了。
不在话下。
\par
这里贾芸随着坠儿,逶迤来至怡红院中。
坠儿先进去回明了,然后方领贾芸进去。
贾芸看时,只见院内略略的有几点山石,种着芭蕉,那边有两只仙鹤在松树下剔翎。
\zhu{
翎[líng]:鸟翅和鸟尾上长而硬的毛;泛指羽毛。
剔翎:鸟类用嘴啄刮自己的羽毛。
}一溜回廊上吊着各色笼子,各色仙禽异鸟。
上面小小五间抱厦,
\zhu{抱厦:原建筑之前或之后接建出来的小房子。}
一色雕镂新鲜花样隔扇,\zhu{隔扇:在房屋内部作隔开用的一扇扇木板墙或纸壁,上部一般做成窗棂,糊纸或装玻璃。
也作“槅扇”。
}上面悬着一个匾额,四个大字题道是“怡红快绿”。
贾芸想道:“怪道叫‘怡红院’,可知原来匾上是恁样四个字。
”\zhu{恁:音“嫩”,如此,这样,那。
}\jia{伤哉,转眼便红稀绿瘦矣。
叹叹!}正想着,只听里面隔着纱窗子笑道:\jia{此文若张僧繇点睛之龙,破壁飞矣,
\zhu{张僧繇[yáo]:南朝梁画家,擅作人物故事画及宗教画。
画龙点睛:典出自唐·张彦远《历代名画记》卷七:“梁武崇饰佛寺,多命僧繇画之……金陵安乐寺四白龙不点眼睛,
每云:‘点睛即飞去。’人以为妄诞,固请点之,须臾,雷电破壁,两龙乘云腾去上天,二龙未点睛者见在。”
“画龙点睛”用在这里表示描写气象生动。
}
焉得不拍案叫绝!}
“快进来罢。
我怎么就忘了你两三个月!”贾芸听得是宝玉的声音,连忙进入房内。
抬头一看,只见金碧辉煌,\jia{器皿叠叠。
}\geng{不能细览之文。
}文章熌灼,\zhu{熌:同“闪”。
}\jia{陈设垒垒。
\zhu{垒垒:重叠堆积的样子。
}}\geng{不得细玩之文。
\zhu{“不能细览”、“不得细玩”反映贾芸进入宝玉房内后对豪华装饰目不暇接的神态。}
}却看不见宝玉在那里。
\jia{武夷九曲之文。
}一回头,只见左边立着一架大穿衣镜,从镜后转出两个一般大的十五六岁的丫头来说:“请二爷里头屋里坐。
”贾芸连正眼也不敢看,连忙答应了。
又进一道碧纱橱,\zhu{
碧纱橱:装在房内起隔开作用的一扇一扇的木板墙,也称“隔扇”、“槅扇”。中间两扇平日可以开关,或加挂帘子帷帐,又叫“纱橱”、“纱厨”。
槅心部分常糊以绿纱,故称碧纱橱。
}只见一张小小填漆床上,\zhu{
填漆:漆器制作工艺的一种。即在漆器上雕刻花纹,在刻纹处填以彩漆。
填漆有两种工艺:一是填彩和漆面相平;二是雕填后花纹凹陷,不与漆面平,显出刀刻味。
}悬着大红销金撒花帐子。
\zhu{销:熔化金属。}
宝玉穿着家常衣服,靸着鞋,
\zhu{靸:音“洒”,穿鞋时把鞋后帮踩在脚后跟下,拖着走。}
倚在床上拿着本书看,\jia{这是等芸哥看,故作款式。
若果真看书,在隔纱窗子说话时已经放下了。
玉兄若见此批,必云:老货,他处处不放松我,可恨可恨!回思将余比作钗、颦等,乃一知己,余何幸也!一笑。
}见他进来,将书掷下,早堆着笑立起身来。
\geng{小叔身段。
}贾芸忙上前请了安。
宝玉让坐,便在下面一张椅子上坐了。
宝玉笑道:“只从那日见了你,我叫你往书房里来,谁知接接连连许多事情,就把你忘了。
”贾芸笑道:“总是我没福,偏偏又遇着叔叔身上欠安。
叔叔如今可大安了?”宝玉道:“大好了。
我倒听见说你辛苦了好几天。
”贾芸道:“辛苦也是该当的。
叔叔大安了,也是我们一家子的造化。
”\jia{不伦不理,迎合字样,口气逼肖,可笑可叹!}\geng{谁一家子?可发一大笑。
}\par
说着,只见有个丫鬟端了茶来与他。
那贾芸口里和宝玉说着话,眼睛却溜瞅那丫鬟:\jia{前写不敢正眼,今又如此写,是因茶来,有心人故留此神,于接茶时站起,方不突然。
}\geng{此句是认人,非前溜红玉之文。
}细挑身材,容长脸面,穿着银红袄子,青缎背心,白绫细折裙。
——不是别人,却是袭人。
\jia{《水浒》文法用的恰,当是芸哥眼中也。
}那贾芸自从宝玉病了,他在里头混了两天,他却把那有名人口认记了一半。
\jia{一路总是贾芸是个有心人,一丝不乱。
}他也知道袭人在宝玉房中比别个不同,\geng{如何?可知余前批不谬。
}今见他端了茶来,宝玉又在旁边坐着,便忙站起来笑道:“姐姐怎么替我倒起茶来。
我来到叔叔这里,又不是客,让我自己倒罢了。
”\jia{总写贾芸乖觉,一丝不乱。
}宝玉道:“你只管坐着罢。
丫头们跟前也是这样。
”贾芸笑道:“虽如此说,叔叔房里姐姐们,我怎么敢放肆呢?”\jia{红玉何以使得?}一面说,一面坐下吃茶。
\par
那宝玉便和他说些没要紧的散话。
\zhu{散话:闲话。
}\jia{妙极是极!况宝玉又有何正紧可说的!
\zhu{紧:紧要,重要。}
}\geng{此批被作者\sout{偏}[骗]过了。
\zhu{
“宝玉又有何正紧可说的”和“此批被作者骗过了”是两条矛盾的批语,应该是持有不同观点的两个人一前一后所写。
}
}又说道谁家的戏子好,谁家的花园好,又告诉他谁家的丫头标致,谁家的酒席丰盛,又是谁家有奇货,又是谁家有异物。
\jia{几个“谁家”,自北静王公侯驸马诸大家包括尽矣,写尽纨绔口角。
}\geng{脂砚斋再笔:对芸兄原无可说之话。
}\ping{阶级地位不同,成长环境不同,聊不到一起。
}那贾芸口里只得顺着他说,说了一回,见宝玉有些懒懒的了,便起身告辞。
宝玉也不甚留,只说:“你明儿闲了,只管来。
”仍命小丫头子坠儿送他出去。
\par
出了怡红院,贾芸见四顾无人,便把脚慢慢的停着些走,口里一长一短和坠儿说话,先问他“几岁了?名字叫什么?你父母在那一行上?在宝叔房内几年了?\jia{渐渐入港。
\zhu{入港:说话投机。
}}一个月多少钱?共总宝叔房内有几个女孩子?”那坠儿见问,便一桩桩的都告诉他了。
贾芸又道:“刚才那个与你说话的,他可是叫小红?”坠儿笑道:“他倒叫小红。
你问他作什么?”贾芸道:“方才他问你什么手帕子,我倒拣了一块。
”坠儿听了笑道:“他问了我好几遍,可有看见他的帕子。
我有那么大工夫管这些事!今儿他又问我,他说我替他找着了,他还谢我呢。
\geng{“传”字正文,此处方露。
\zhu{“传”:指标题“蜂腰桥设言传蜜意”。}
}才在蘅芜苑门口说的,二爷也听见了,不是我撒谎。
好二爷,你既拣着了,给我罢。
我看他拿什么谢我。
”\par
原来上月贾芸进来种树之时,便拣了一块罗帕,便知是所在园内的人失落的,但不知是那一个人的,故不敢造次。
\zhu{造次:轻率、鲁莽。
}今儿听见红玉问坠儿,便知是红玉的,心内不胜喜幸。
又见坠儿追索,心中早得了主意,便向袖内将自己的一块取了出来,向坠儿笑道:“我给是给你,你若得了他的谢礼,可不许瞒着我。
”坠儿满口里答应了,接了手帕子,送出贾芸,回来找红玉,不在话下。
\jia{至此一顿,狡猾之甚!原非书中正文之人,写来间色耳。
\zhu{间色:杂色,多色相配而成的颜色。用颜色的调和比喻情节的穿插,指艺术创作中注重各部分之间风格色调的参差变换,相间互补的一种笔法。如脂评所说“皆错综其事,不作一直笔也。”
用在这里的意思是指一笔岔开,写文章主脉之外的分脉,使文章更加丰富。
}}\ping{传蜜意即指此处手帕传情。
}\par
如今且说宝玉打发了贾芸去后,意思懒懒的歪在床上,似有朦胧之态。
袭人便走上来,坐在床沿上推他,说道:“怎么又要睡觉?闷的很,你出去逛逛不是?”宝玉见说,便拉他的手笑道:“我要去,只是舍不得你。
”袭人笑道:“快起来罢!”\jia{不答得妙!}\geng{不答上文,妙极!}
一面说,一面拉了宝玉起来。
宝玉道:“可往那里去呢?怪腻腻烦烦的。
”\geng{玉兄最得意之文,起笔却如此写。
}袭人道:“你出去了就好了。
只管这么葳蕤,\zhu{葳蕤[wēiruí]: 
本指草木茂盛枝叶下垂貌,此指无精打采、萎靡不振。
}越发心里烦腻。
”\par
宝玉无精打采的,只得依他。
晃出了房门,在回廊上调弄了一回雀儿;出至院外,顺着沁芳溪看了一回金鱼。
只见那边山坡上两只小鹿箭也似的跑来,宝玉不解何意,\jia{余亦不解。
}正自纳闷,只见贾兰在后面拿着一张小弓追了下来。
\jia{前文。
}\geng{此等文可是人能意料的?}一见宝玉在前面,便站住了,笑道:“二叔叔在家里呢,我只当出门去了。
”宝玉道:“你又淘气了。
好好的射他作什么?”贾兰笑道:“这会子不念书,闲着作什么?所以演习演习骑射。
”\jia{奇文奇语,默思之方意会。
为玉兄之毫无一正事,只知安富尊荣而写。
}\geng{答得何其堂皇正大,何其坦然之至!}
宝玉道:“把牙栽了,那时才不演呢。
”\ping{贾兰父亲早逝,可能促使他上进,暗伏贾兰之后“爵禄高登”。
}\par
说着,顺着脚一径来至一个院门前,\geng{像无意。
}只见凤尾森森,龙吟细细。
\zhu{凤尾森森:喻竹林茂盛。
龙吟:常用以形容箫笛之类管乐器的声音,这里喻风吹竹林发出的声响。
}\jia{与后文“落叶萧萧,寒烟漠漠”一对,可伤可叹!}\geng{原无意。
}举目望门上一看,只见匾上写着“潇湘馆”三字。
\jia{无一丝心机,反似初至者,故接有忘形忘情话来。
}\geng{三字如此出,足见真出无意。
}宝玉信步走入,只见湘帘垂地,悄无人声。
走至窗前,觉得一缕幽香从碧纱窗中暗暗透出。
\jia{写得出,写得出。
}宝玉便将脸贴在纱窗上,往里看时,耳内忽听得\jia{未曾看见先听见,有神理。
}细细的长叹了一声道:“‘每日家情思睡昏昏’。
”\zhu{每日家情思睡昏昏:《西厢记》第二本第一折莺莺的唱词。
描写崔莺莺思念张生的烦闷心绪。
家:一作“价”,语尾助词,无义。
}\jia{用情忘情神化之文。
}\geng{先用“凤尾森森,龙吟细细”八字,“一缕幽香自纱窗中暗暗透出”,“细细的长叹一声”等句,方引出“每日家情思睡昏昏”仙音妙音来,非纯化功夫之笔不能,可见行文之难。
}宝玉听了,不觉心内痒将起来,再看时,只见黛玉在床上伸懒腰。
\jia{有神理,真真画出。
}宝玉在窗外笑道:“为甚么‘每日家情思睡昏昏’?”一面说,一面掀帘进来了。
\geng{二玉这回文字,作者亦在无意上写来,所谓“信手拈来无不是”也。
}\par
林黛玉自觉忘情,不觉红了脸,拿袖子遮了脸,翻身向里装睡着了。
宝玉才走上来要搬他的身子,只见黛玉的奶娘并两个婆子却跟了进来 
\jia{一丝不漏,且避若干嚼蜡之文。
}说:“妹妹睡觉呢,等醒了再请来。
”刚说着,黛玉便翻身向外坐起来,笑道:“谁睡觉呢?”\jia{妙极!可知黛玉是怕宝玉去也。
}那两三个婆子见黛玉起来,便笑道:“我们只当姑娘睡着了。
”说着,便叫紫鹃说:“姑娘醒了,进来伺候。
”一面说,一面都去了。
\par
黛玉坐在床上,一面抬手整理鬓发,一面笑向宝玉道:“人家睡觉,你进来作什么?”宝玉见他星眼微饧,\zhu{
饧:音“行”,眼睛半睁半闭或呆滞无神。
眼饧:眼皮滞涩、朦胧欲睡。
}香腮带赤,不觉神魂早荡,一歪身坐在椅子上,笑道:“你才说什么?”黛玉道:“我没说什么。
”宝玉笑道:“给你个榧子呢,\zhu{榧(音“匪”)子:拇指和中指紧捏,猛然相捻发出声响,俗称“榧子”。
向对方“打榧子”含有轻佻、玩笑的意思。
}我都听见了。
”\par
二人正说话,只见紫鹃进来。
宝玉笑道:“紫鹃,把你们的好茶倒碗我吃。
”紫鹃道:“那里是好的呢?要好的,只是等袭人来。
”黛玉道:“别理他,你先给我舀水去罢。
”紫鹃笑道:“他是客,自然先倒了茶来再舀水去。
”说着倒茶去了。
宝玉笑道:“好丫头,‘若共你多情小姐同鸳帐,怎舍得叠被铺床?’”\zhu{“若共你”二句:《西厢记》第一本第二折张生的唱词。
原剧“多情小姐”指莺莺,“叠被铺床”者指莺莺的丫鬟红娘。
这里宝玉自比张生,把黛玉比作莺莺,把紫鹃比作红娘。
}\jia{真正无意忘情。
}\geng{真正无意忘情冲口而出之语。
}\geng{方才见芸哥所拿之书一定是《西厢》,不然如何忘情至此?}\ping{张生不舍得红娘“叠被铺床”,暗含的意思是娶莺莺为妻,还要收红娘这个丫鬟为妾,从此之后红娘就不用干“叠被铺床”这种丫鬟干的活了。
这里宝玉自比张生,把黛玉比作莺莺,把紫鹃比作红娘,也暗含要娶黛玉和紫鹃的意思。
}
林黛玉登时撂下脸来,\jia{我也要恼。
}说道:“二哥哥,你说什么?”宝玉笑道:“我何尝说什么。
”黛玉便哭道:“如今新兴的,外头听了村话来,
\zhu{村话:粗野庸俗的话(多指骂人的话)。}
也说给我听;看了混帐书,也来拿我取笑儿。
我成了替爷们解闷的。
”一面哭着,一面下床来,往外就走。
宝玉不知要怎样,心下慌了,忙赶上来,“好妹妹,我一时该死,你别告诉去。
我再要敢,嘴上就长个疔,\zhu{疔:音“丁”,一种毒疮,形小根深,坚硬如钉。
}烂了舌头。
”\par
正说着,只见袭人走来说道:“快回去穿衣服,老爷叫你呢。
”\geng{若无如此文字收拾二玉,写颦无非至再哭恸哭,
\zhu{恸[tòng]哭:极哀痛地哭。}
玉只以赔尽小心软求慢恳,二人一笑而止。
且书内若此亦多多矣,未免有犯雷同之病。
故用险句结住,使二玉心中不得不将现事抛却,各怀一惊心意,再作下文。
壬午孟夏,雨窗。
畸笏。
}宝玉听了,不觉的打了个焦雷的一般,\jia{不止玉兄一惊,即阿颦亦不免一吓,作者只顾写来收拾二玉之文,忘却颦儿也。
想作者亦似宝玉道《西厢》之句,忘情而出也。
}也顾不得别的,急忙回来穿衣服。
出园来,只见茗烟在二门前等着,宝玉便问道:“是作什么?”茗烟道:“爷快出来罢,横竖是见去的,到那里就知道了。
”一面说,一面催着宝玉。
\par
转过大厅,宝玉心里还自狐疑,只听墙角边一阵呵呵大笑,回头看时,见是薛蟠拍着手跳了出来,笑道:\jia{如此戏弄,非呆兄无人。
欲释二玉,非此戏弄不能立解,勿得泛泛看过。
不知作者胸中有多少丘壑。
}\geng{非呆兄行不出此等戏弄,但作者有多少丘壑在胸中,写来酷肖。
}“要不说姨夫叫你,你那里出来的这么快。
”茗烟也笑着跪下了。
宝玉怔了半天,方解过来是薛蟠哄他出来。
薛蟠连忙打恭作揖陪不是,\geng{酷肖。
}又求“不要难为了小子,都是我逼他去的”。
宝玉也无法了,只好笑,因说道:“你哄我也罢了,怎么说我父亲呢?我告诉姨娘去,评评这个理,可使得么?”薛蟠忙道:“好兄弟,我原为求你快些出来,就忘了忌讳这句话。
改日你也哄我,说我的父亲就完了。
”\jia{写粗豪无心人毕肖。
}\geng{真真乱话。
}\ping{薛蟠父亲已经去世了。
}
宝玉道:“嗳,嗳,越发该死了!”又向茗烟道:“反叛肏的,还跪着作什么!”茗烟连忙叩头起来。
薛蟠道:“要不是我也不敢惊动,只因明儿五月初三日是我的生日,谁知古董行的程日兴,他不知那里寻了来的这么粗、这么长粉脆的鲜藕,\geng{如见如闻。
}这么大的大西瓜,这么长的一尾新鲜的鲟鱼,这么大的一个暹罗国进贡的灵柏香熏的暹猪。
\zhu{暹罗:音“先罗”,泰国的旧称。
}\ping{“这么”道尽不学无术纨绔子弟语言修辞的匮乏。
}你说,他这四样礼可难得不难得?那鱼、猪不过贵而难得,这藕和瓜亏他怎么种出来的。
我连忙孝敬了母亲,赶着给你们老太太、姨父、姨母送了些去。
如今留了些,我要自己吃,恐怕折福,\jia{呆兄亦有此语,批书人至此诵《往生咒》至恒河沙数也。
\zhu{《往生咒》:全称《拔一切业障根本得生净土陀罗尼经》,佛教净土宗信徒经常持诵的一种咒语。
往生:佛教谓去娑婆世界,往弥陀如来之极乐净土,谓之往;化生于彼土七宝莲华中,谓之生。
故佛教有往生咒,谓念咒可使死者超生投胎为人身。
恒河沙数:比喻数量多得像恒河里的沙子一样无法计数。
}}左思右想,除我之外,惟有你还配吃,\jia{此语令人哭不得笑不得,亦真心语也。
}所以特请你来。
可巧唱曲儿的一个小子又才来了,我同你乐一日何如?”\par
一面说,一面来至他书房里。
只见詹光、程日兴、胡斯来、
\zhu{
胡斯来:就是“呼斯来”,斯是“就”的意思。“呼斯来”也就是“呼之即来”,暗合其帮闲(附庸权贵或富豪,为他们凑趣儿消闲、装点门面)的身份。
}
单聘仁等并唱曲儿的都在这里,见他进来,请安的,问好的,都彼此见过了。
吃了茶,薛蟠即命人摆酒来。
说犹未了,众小厮七手八脚摆了半天,\geng{又一个写法。
}才停当归坐。
宝玉果见瓜藕新异,因笑道:“我的寿礼还未送来,倒先扰了。
”薛蟠道:“可是呢,明儿你送我什么?”\geng{逼真酷肖。
}宝玉道:“我可有什么可送的?若论银钱吃穿等类的东西, 
\jia{谁说的出?经过者方说得出。
叹叹!}究竟还不是我的,惟有或写一张字,画一张画,才算是我的。
”\par
薛蟠笑道:“你提画儿,我才想起来了。
昨儿我看人家一张春宫,\zhu{春宫:淫画。
}
\geng{阿呆兄所见之画也!}画的着实好。
上面还有许多的字,我也没细看,只看落的款,是‘庚黄’\jia{奇文,奇文!}画的。
真真好的了不得!”宝玉听说,心下猜疑道:“古今字画也都见过些,那里有个‘庚黄’?”想了半天,不觉笑将起来,命人取过笔来,在手心里写了两个字,又问薛蟠道:“你看真了是‘庚黄’?”薛蟠道:“怎么看不真!”\jia{闲事顺笔,骂死不学之纨绔。
叹叹!}\geng{闲事顺笔,将骂死不学之纨绔。
壬午雨窗。
畸笏。
}宝玉将手一撒,与他看道:“别是这两个字罢?其实与‘庚黄’相去不远。
”众人都看时,原来是“唐寅”两个字,都笑道:“想必是这两字,大爷一时眼花了也未可知。
”薛蟠自觉没意思,\geng{实心人。
}笑道:“谁知他‘糖银’‘果银’的。
”\par
正说着,小厮来回:“冯大爷来了。
”宝玉便知是神武将军冯唐之子冯紫英来了。
薛蟠等一齐都叫:“快请。
”说犹未了,只见冯紫英一路说笑\geng{如见如闻。
}已进来。
\jia{一派英气如在纸上,特为金闺润色也。
}众人忙起席让坐。
冯紫英笑道:“好呀!也不出门了,在家里高乐罢。
” 
\geng{如见其人于纸上。
}宝玉、薛蟠都笑道:“一向少会,老世伯身上康健?”紫英答道:“家父倒也托庇康健。
\zhu{托庇:托人福庇,旧时的客套话。
}近来家母偶着了些风寒,不好了两天。
”\geng{紫英豪侠小文三段,是为金闺间色之文,壬午雨窗。
}\geng{写倪二、紫英、湘莲、玉菡侠文,皆各得传真写照之笔。
丁亥夏。
畸笏叟。
}\geng{惜“卫若兰射圃”文字无稿。
叹叹!丁亥夏。
畸笏叟。
}薛蟠见他面上有些青伤,便笑道:“这脸上又和谁挥拳的?挂了幌子了。
”\zhu{幌子:外露的标志或痕迹。
}冯紫英笑道:“从那一遭把仇都尉的儿子打伤了,我就记了再不怄气,如何又挥拳?这个脸上,是前日打围,\zhu{打围:围猎禽兽叫“打围”,亦即打猎。
}
在铁网山教兔鹘捎一翅膀。
”\zhu{兔鹘(鹘音“鼓”):即鹘,一种猎鹰,善扑兔雁等。
捎:拂,掠。
}\geng{如何着想?}新奇字样。
宝玉道:“几时的话?”紫英道:“三月二十八日去的,前儿也就回来了。
”宝玉道:“怪道前儿初三四儿,我在沈世兄家赴席不见你呢。
我要问,不知怎么就忘了。
单你去了,还是老世伯也去了?”紫英道:“可不是家父去,我没法儿,去罢了。
难道我闲疯了,咱们几个人吃酒听唱不乐,寻那个苦恼去?这一次,大不幸之中又大幸。
”\jia{似又伏一大事样,英侠人累累如是,令人猜摹。
}\ping{设置悬念,引人入胜。
}\par
薛蟠众人见他吃完了茶,都说道:“且入席,有话慢慢的说。
”\geng{馀文再述。
}冯紫英听说,便立起身来说道:“论礼,我该陪饮几杯才是,只是今儿有一件大大要紧事,回去还要见家父面回,实不敢领。
”薛蟠、宝玉众人那里肯依,死拉着不放。
冯紫英笑道:“这又奇了。
\geng{如闻如见。
}你我这些年,那一回有这个道理的?果然不能遵命。
若必定叫我领,拿大杯来,\geng{写豪爽人如此。
}我领两杯就是了。
”众人听说,只得罢了,薛蟠执壶,宝玉把盏,斟了两大海。
\zhu{大海:这里指大酒杯。
}那冯紫英站着,一气而尽。
\jia{令人快活煞。
}\geng{爽快人如此,令人羡煞。
}宝玉道:“你到底把这个‘不幸之幸’说完了再走。
”冯紫英笑道:“今儿说的也不尽兴。
我为这个,还要特治一东,请你们去细谈一谈;二则还有所恳之处。
”说着执手就走。
\zhu{执手:拱手。
}薛蟠道:“越发说的人热剌剌的丢不下。
\zhu{剌剌:音“辣辣”,语助词,表示加重语气。
热剌剌:形容情绪激动或焦躁。
}多早晚才请我们,告诉了也免的人犹豫。
”\geng{实心人如此,丝毫形迹俱无,
\zhu{形迹:指礼貌、规矩。}
令人痛快煞。
}\ping{心直口快无遮掩。
}冯紫英道:“多者十日,少则八天。
”一面说,一面出门上马去了。
众人回来,依席又饮了一回方散。
\jia{收拾得好。
}\ping{贾宝玉和贾芸无话可说,和薛蟠众人谈笑甚欢,不同阶级的人玩不到一起去。
}\par
宝玉回至园中,袭人正记挂着他去见贾政,\jia{“生员切己之事”,时刻难忘。
\zhu{
生员:科举时代指在国学或州、县学读书的学生;
明清时指经过本省各级考试录取进入府、州、县学的学生,通称秀才。
金圣叹《贯华堂第六才子书西厢记》第二本第二折·请宴【上小楼】:
“秀才们闻道请,似得了将军令,先是五脏神愿随鞭镫。”
批:“又嘲戏生员切己事情。”
脂砚斋的很多评点模式和术语,都是从金圣叹那里借用来的。
“生员”应该是指批书人自己,批书人看到书中宝玉严厉的父亲,想到自己在现实中也有类似的经历。
}
}不知是祸是福,\geng{下文伏线。
}只见宝玉醉醺醺的回来,问其原故,宝玉一一向他说了。
袭人道:“人家牵肠挂肚的等着,你且高乐去,也到底打发人来给个信儿。
”宝玉道:“我何尝不要送信儿,只因冯世兄来了,就混忘了。
”\par
正说着,只见宝钗走进来笑道:“偏了我们新鲜东西了。
”\zhu{偏了:谦词,占先、僭越之意。
这里是表示自己已经吃过了的客气话。
}宝玉笑道:“姐姐家的东西,自然先偏了我们了。
”宝钗摇头笑道:“昨儿哥哥倒特特的请我吃,\zhu{特特:特地。
}我不吃他,叫他留着送人请人罢。
我知道我的命小福薄,不配吃那个。
”\jia{暗对呆兄言宝玉配吃语。
}说着,丫鬟倒了茶来,吃茶说闲话儿,不在话下。
\par
却说那林黛玉听见贾政叫了宝玉去了,一日不回来,心中也替他忧虑。
\jia{本是切己事。
}至晚饭后,闻听宝玉来了,心里要找他问问是怎么样了。
\jia{呆兄此席,的是合和筵也。
一笑。
}\geng{这席东道是和事酒不是?}一步步行来,见宝钗进宝玉的院内去了,\jia{《石头记》最好看处是此等章法。
}
自己也便随后走了来。
刚到了沁芳桥,只见各色水禽都在池中浴水,也认不出名色来,但见一个个文彩炫耀,好看异常,因而站住看了一回。
\geng{避难法。
}再往怡红院来,只见院门关着,黛玉便以手扣门。
\par
谁知晴雯和碧痕正拌了嘴,没好气,忽见宝钗来了,那晴雯正把气移在宝钗身上,\geng{晴雯迁怒是常事耳,写钗、颦二卿身上,与踢袭人之文,
\zhu{踢袭人之文:第三十回,宝玉敲门,因开门慢而踢袭人。}
令人于何处设想着笔?丁亥夏。
畸笏叟。
}正在院内抱怨说:“有事没事跑了来坐着,\jia{犯宝钗如此写法。
}叫我们三更半夜不得睡觉!”\jia{指明人,则暗写。
}\ping{宝钗不仅来得次数多,而且来得时间很晚。
}忽听又有人叫门,晴雯越发动了气,也并不问是谁,\jia{犯黛玉如此写明。
}便说道:“都睡下了,明儿再来罢!”\jia{不知人,则明写。
}林黛玉素知丫头们的情性,他们彼此顽耍惯了,恐怕院内的丫头没听真是他的声音,只当是别的丫头们来了,所以不开门,因而又高声说道:“是我,还不开么?”晴雯偏生还没听出来,\jia{想黛玉高声亦不过你我平常说话一样耳,况晴雯素昔浮躁多气之人,如何辨得出?此刻须得批书人唱“大江东去”的喉咙,嚷着“是我林黛玉叫门”方可。
又想若开了门,如何有后面很多好字样好文章,看官者意为是否?}便使性子说道:“凭你是谁,二爷吩咐的,一概不许放人进来呢!”林黛玉听了,不觉气怔在门外,待要高声问他,斗起气来,自己又回思一番:“虽说是舅母家如同自己家一样,到底是客边。
\zhu{客边:以客人的身份寄居在别人家里。
}
\jia{寄食者着眼,况颦儿何等人乎?}如今父母双亡,无依无靠,现在他家依栖。
如今认真淘气,\zhu{淘气:这里是怄气的意思。
}也觉没趣。
”一面想,一面又滚下泪珠来。
正是回去不是,站着不是。
正没主意,只听里面一阵笑语之声,细听了一听,竟是宝玉、宝钗二人。
林黛玉心中益发动了气,左思右想,忽然想起早起的事来:“必定是宝玉恼我告他的原故。
但只我何尝告你去了,你也不打听打听,就恼我到这步田地。
你今儿不叫我进来,难道明儿就不见面了!”越想越伤感,也不顾苍苔露冷,花径风寒,独立墙角边花阴之下,悲悲戚戚呜咽起来。
\jia{可怜杀!可疼杀!余亦泪下。
}\par
原来这林黛玉秉绝代姿容,具希世俊美,不期这一哭,那附近柳枝花朵上的宿鸟栖鸦一闻此声,俱忒楞楞飞起远避,\zhu{忒楞楞[tè lèng lèng]:象声词,形容鸟飞的声音。
}不忍再听。
\jia{沉鱼落雁,闭月羞花,原来是哭出来的。
一笑。
}真是:\par
\hop
花魂默默无情绪,鸟梦痴痴何处惊。
\par
\hop
因有一首诗道:\par
颦儿才貌世应希,独抱幽芳出绣闺;\par
呜咽一声犹未了,落花满地鸟惊飞。
\par
\hop
那林黛玉正自啼哭,忽听“吱喽”一声,院门开处,不知是那一个出来。
且看下回。
\jia{每阅此本,掩卷者十有八九,不忍下阅看完,想作者此时泪下如豆矣。
}\par
\jia{此回乃颦儿正文,故借小红许多曲折琐碎之笔作引。
\hang
怡红院见贾芸,宝玉心内似有如无,贾芸眼中应接不暇。
\hang
“凤尾森森,龙吟细细”八字,“一缕幽香从碧纱窗中暗暗透出”,又“细细的长叹一声”等句方引出“每日家情思睡昏昏”仙音妙音,俱纯化工夫之笔。
\hang
二玉这回文字,作者亦在无意上写来,所谓“信手拈来无不是”也。
\hang
收拾二玉文字,写颦无非哭玉、再哭、恸哭,
\zhu{恸[tòng]哭:极哀痛地哭。}
玉只以陪事小心软求慢恳,二人一笑而止。
且书内若此亦多多矣,未免有犯雷同之病。
故险语结住,使二玉心中不得不将现事抛却,各怀以惊心意,再作下文。
\hang
前回倪二、紫英、湘莲、玉菡四样侠文皆得传真写照之笔,惜“卫若兰射圃”文字迷失无稿,叹叹!\hang
晴雯迁怒系常事耳,写于钗、颦二卿身上与踢袭人、打平儿之文,令人于何处设想着笔。
\hang
黛玉望怡红之泣,是“每日家情思睡昏昏”上来。
}\par
\qi{总评:喜相逢,三生注定;遗手帕,月老红丝。
幸得人语说连理,
\zhu{
这几句写小红和贾芸的情缘,二人一见钟情,互换手帕。
贾芸和坠儿关于交换手帕的对话中,坠儿实际上充当了贾芸和小红的“月老”,为他们牵了“红丝”。
}
又忽见他枝并蒂。难猜未解细追思,罔多疑,\zhu{
罔:音“网”,表示否定,相当于“不”“不要”。这里可能是“枉”的错讹,“枉”的意思是徒然、白白地。
}空向花枝哭月底。
\zhu{
后几句是针对黛玉误会宝玉而伤心,走到花阴下哭泣的故事,即“空向花枝哭月底”所本。
黛玉“只听里面一阵笑语之声,细听了一听,竟是宝玉、宝钗二人”,此即“又忽见他枝并蒂”所本。
}
}
\dai{051}{小红和佳蕙闲谈}
\dai{052}{潇湘馆春困发幽情}
\sun{p26-1}{蜂腰桥设言传蜜意,怡红院宝玉见贾芸}{图右侧:小红在去蘅芜苑路过蜂腰桥门前时遇见坠儿引着贾芸走过来,贾芸一面走,一面拿眼把小红一溜,小红只装着和坠儿说话,也把眼去一溜贾芸,四目恰好相对。
小红不觉把脸一红。
图左侧:贾芸来到怡红院,只见院内几点山石,几株芭蕉,两只仙鹤在树下剔翎。
宝玉隔着纱窗子笑道:“快进来罢。
我怎么就忘了你两三个月!”}
\sun{p26-2}{潇湘馆春困发幽情,赚出宝玉薛蟠请客}{图上侧:宝玉顺着沁芳溪漫步,见贾兰持弓追逐两头小鹿。
图右下:来到潇湘馆,只听黛玉在房里长叹,便在窗外笑道:“为什么‘每日家情思睡昏昏’的?”于是,进屋和黛玉调笑了一阵子。
图左侧:正说着,忽听老爷叫他,宝玉急忙换了衣服出来,只听墙角边一阵呵呵大笑,回头见薛蟠跳出来。
原来是薛蟠哄他出来喝酒行乐。
吃喝间,小厮来报,冯紫英来了,众人忙起身让座。
}