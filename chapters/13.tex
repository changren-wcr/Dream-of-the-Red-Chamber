\chapter{秦可卿死封龙禁尉\quad 王熙凤协理宁国府}
\zhu{龙禁尉:作者虚拟的皇帝侍卫官名。
}
\par
\jia{贾珍尚奢,岂有不请父命之理?因敬[老修炼]要紧,不问家事,故得恣意放为。
\hang
若明指一州名,似落《西游》〼〼〼〼〼〼〼地,\zhu{〼 代表缺字。
甲戌本此页被对角撕去,缺字较多,因与庚辰本相关批语内容类似,可参看,缺字不补。
}不待言可知,是光天〼〼〼〼〼〼〼〼矣。
不云国名更妙,〼〼〼〼〼〼〼〼〼〼义之乡也。
直与……\hang
今秦可卿托〼〼〼〼〼〼〼〼〼〼〼〼〼理宁府亦〼〼〼〼〼〼〼〼〼〼〼〼〼
凤〼〼〼〼〼〼〼〼〼〼
〼〼〼〼〼〼在封龙禁尉,写乃褒中之贬,隐去天香楼一节,是不忍下笔也。
}\par
\geng{此回可卿[托]梦阿凤,盖作者大有深意存焉。
可惜生不逢时,奈何奈何!然必写出自可卿之意也,则又有他意寓焉。
\hang
荣、宁世家,未有不尊家训者。
虽贾珍尚奢,岂明逆父哉?故写敬老不管,然后恣意,方见笔笔周到\foot{以上二条庚辰本批语及题诗,原在第二册目录后加页上,参照甲戌本回前评移此。
}。
}\par
\qi{生死穷通何处真?英明难遏是精神。
微密久藏偏自露,幻中梦里语惊人。
}\par
诗云:\par
一步行来错,回头已百年。
古今风月鉴,多少泣黄泉\foot{甲戌本自此回至第十六回,回前均有“诗云(曰)”字样而无诗。
此诗据庚辰本补。
}!\par
\hop
话说凤姐自贾琏送黛玉往扬州去后,心中实在无趣,每到晚间,不过和平儿说笑一回,就胡乱\jia{“胡乱”二字奇。
\ping{不知何为和贾琏正经睡?}
}睡了。
\ping{此时凤姐和贾琏的感情还好着呢。
}\par
这日夜间,正和平儿灯下拥炉倦绣,早命浓薰绣被,二人睡下,屈指算行程该到何处,\jia{所谓“计程今日到梁州”是也。
\zhu{
计程今日到梁州:白居易《同李十一醉忆元九》:
忽忆故人天际去,计程今日到梁州。
}
}不知不觉已交三鼓。
平儿已睡熟了。
凤姐方觉星眼微朦,
\zhu{星眼:明媚亮丽的眼睛。}
恍惚只见秦氏从外走了进来,含笑说道:“婶婶好睡!我今儿回去,你也不送我一程。
因娘儿们素日相好,我舍不得婶婶,故来别你一别。
还有一件心愿未了,非告诉婶子,别人未必中用。
”\jia{一语贬尽贾家一族空顶冠束带者。
\zhu{顶冠束带:指男子。}
}\par
凤姐听了,恍惚问道:“有何心愿?你只管托我就是了。
”秦氏道:“婶婶,你是个脂粉队内的英雄,\geng{称得起。
}连那些束带顶冠的男子也不能过你,你如何连两句俗语也不晓得?常言‘月满则亏,水满则溢’;又道是‘登高必跌重’。
如今我们家赫赫扬扬,已将百载,一日倘或\jia{“倘或”二字酷肖妇女口气。
}乐极悲生,若应了那句‘树倒猢狲散’的俗语,\jia{“树倒猢狲散”之语,今犹在耳,屈指三十五年矣。
哀哉伤哉,宁不痛杀!}
岂不虚称了一世的诗书旧族了!”凤姐听了此话,心胸大快,十分敬畏,忙问道:“这话虑的极是,但有何法可以永保无虞?”\jia{非阿凤不明,盖古今名利场中患失之同意也。
\zhu{
患失:怕失去得到的。
同意:心意相同。
}
}秦氏冷笑道:“婶婶好痴也!否极泰来,\zhu{
否:音“痞”。
否、泰:《周易》中的两个卦名。
“否”,表示滞塞、坏运气、凶险;“泰”,表示亨通、好运气、吉利。
否极泰来:意思是情况坏到极点,就会往好的方面转化。
}荣辱自古周而复始,岂是人力能可保常的。
但如今能于荣时筹画下将来衰时的世业,
\zhu{世业:可传世的功业。}
亦可谓常保永全了。
即如今日诸事都妥,只有两件事未妥,若把此事如此一行,则日后可保永全了。
”\par
凤姐便问何事。
秦氏道:“目今祖茔虽四时祭祀,\zhu{茔:音“迎”,墓地。
}只是无一定的钱粮;第二,家塾虽立,无一定的供给。
依我想来,如今盛时固不缺祭祀、供给,但将来败落之时,此二项有何出处?莫若依我定见,趁今日富贵,将祖茔附近多置田庄、房舍、地亩,以备祭祀供给之费皆出自此处,将家塾亦设于此。
合同族中长幼,大家定了则例,日后按房掌管这一年的地亩、钱粮、祭祀、供给之事。
如此周流,又无争竞,亦不有典卖诸弊。
便是有了罪,凡物可入官,这祭祀产业连官也不入的。
便败落下来,子孙回家读书务农,也有个退步,\qi{幻情文字中忽入此等警句,提醒多少热心人。
}祭祀又可永继。
若目今以为荣华不绝,不思日后,终非长策。
眼见不日又有一件非常喜事,真是烈火烹油、鲜花着锦之盛。
要知道,也不过是瞬息的繁华,一时的欢乐,\meng{“瞬息繁华,一时欢乐”二语,可共天下有志事业功名者同来一哭。
但天生人非无所为,遇机会,成事业,留名于后世者,亦必有奇传奇遇,方能成不世之功。
此亦皆苍天暗中扶助,虽有波澜,而无甚害,反觉其铮铮有声。
其不成也,亦由天命。
其奸人倾险之计,亦非天命不能行。
其繁华欢乐,亦自天命。
人于其间,知天命而存好生之心,尽己力以周旋其间,不计其功之成与否,所谓心安而理尽,又何患乎?一时瞬息,随缘遇缘,乌乎不可!}万不可忘了那‘盛筵必散\foot{原作“盛筵不散”,除戚本“不”改为“必”外,馀本均同。
一般认为,“不”是“必”的形讹,本书及其他古籍均有误例。
按此语疑为类似“盛筵不散,终须一散”的俗语的半句,与下文第七十二回写司棋与表弟相约“不娶不嫁”用法相近
(“不娶不嫁”显为“非卿不娶非君不嫁”的省略,并非打算单身)。
因别无佐证,暂依戚本改。
}’的俗语。
此时若不早为后虑,临期只恐后悔无益了。
”  
\jia{语语见道,字字伤心,读此一段,几不知此身为何物矣。
松斋。
}凤姐忙问:“有何喜事?”秦氏道:“天机不可泄漏。
\jia{伏得妙!}\ping{伏贾元春才选凤藻宫。
}只是我与婶子好了一场,临别赠你两句话,须要记着。
”因念道:\par
\hop
三春去后诸芳尽,各自须寻各自门。
\jia{此句令批书人哭死。
}\jia{不必看完,见此二句,即欲堕泪。
梅溪。
}\ping{三春即迎春、探春和惜春。
}\par
\hop
凤姐还欲问时,只听二门上传事云板连叩四下,\zhu{云板连叩四下:报凶丧大事的讯号。
旧俗吉事常用三数,凶事常用四数,有“神三鬼四”之说。
封建时代的官署或大官僚的私邸,二门旁常设有一种金属的响器叫“点”,击之报时或集众叫“传点”。
“点”多铸成云头形,故又称“云板”。
}正是丧音\foot{“正是丧音”,己、庚、戚、蒙等本无此语,疑系批语混入正文。
}。
\ping{秦可卿为贾府长远谋划,但是贾府依旧“事败抄没”后“子孙流散”,“一败涂地”。
可见凤姐经过指点之后并没有完全照做。
总是有人希望通过指点让人早悟,然而除非真的陷入悲剧,没有任何人能在悲剧发生前醒来。
贾府当下“烈火烹油、鲜花着锦”,怎么会想到败落的未来?}将凤姐惊醒。
人回:“东府蓉大奶奶没了。
”凤姐闻听,吓了一身冷汗,出了一回神,只得忙忙的穿衣服,往王夫人处来。
\par
彼时合家皆知,无不纳罕,都有些疑心。
\jia{九个字写尽天香楼事,是不写之写。
}\geng{可从此批。
}那长一辈的想他素日孝顺,平一辈的想他平日和睦亲密,\geng{松斋云:好笔力。
此方是文字佳处。
}下一辈的想他素日慈爱,以及家中仆从老小想他素日怜贫惜贱、慈老爱幼\geng{八字乃为上人\sout{之}[者]当铭于五衷。
\zhu{五衷:五脏。铭感五衷比喻非常感激,也作「铭感五内」。}
}之恩,莫不悲嚎痛哭者。
\geng{老健。
}\ping{人人皆爱,处处周到,难怪秦可卿思虑过多,用心过度。
}\par
闲言少叙,却说宝玉因近日林黛玉回去,剩得自己孤凄,也不和人顽耍,\jia{与凤姐反对。
}\jia{淡淡写来,方是二人自幼气味相投,可知后文皆非突然文字。
}每到晚间,便索然睡了。
如今从梦中听见说秦氏死了,连忙翻身爬起来,只觉心中似戳了一刀的,不忍“哇”的一声,喷出一口血来。
\jia{宝玉早已看定可继家务事者可卿也,今闻死了,大失所望。
急火攻心,焉得不有此血?为玉一叹!}袭人等慌慌忙忙上来搊扶,\zhu{搊(搊音“抽”)扶:意近搀扶。
}问是怎么样,又要回贾母来请大夫。
宝玉笑道:“不用忙,不相干,\geng{又淡淡抹去。
}这是急火攻心,\jia{如何自己说出来了?}血不归经。
”\zhu{急火攻心,血不归经:中医认为人的情绪受到突如其来的刺激,可以引起情志之火内发,而使心火肝火亢盛,干扰正常营血,逼血妄行,就会出现吐血、出鼻血等症状。
但这只是一时的现象,与虚劳损伤引起的吐血不同。
所以宝玉说“不用忙,不相干”。
}说着便爬起来,要衣服换了,来见贾母,即时要过去。
\geng{如此总是淡描轻写,全无痕迹,方见得有生以来,天分中自然所赋之性如此,非因色所\sout{感}[惑]也。
}袭人见他如此,心中虽放不下,又不敢拦,只是由他罢了。
贾母见他要去,因说:“才咽气的人,那里不干净;二则夜里风大,等明早再去不迟。
”宝玉那里肯依。
贾母命人备车,多派跟从人役,拥护前来。
\par
一直到了宁国府前,只见府门洞开,两边灯笼照如白昼,乱烘烘人来人往,里面哭声摇山振岳。
\jia{写大族之丧,如此起绪。
}宝玉下了车,忙忙奔至停灵之室,痛哭一番。
然后见过尤氏。
谁知尤氏正犯了胃疼旧疾,睡在床上。
\jia{妙!非此何以出阿凤!}\geng{紧处愈紧,密处愈密。
}\geng{所谓层峦叠翠之法也。
野史中从无此法。
即观者到此,亦为写秦氏未必全到,岂料更又写一尤氏哉!}然后又出来见贾珍。
彼时贾代儒带领贾敕、贾效、贾敦、贾赦、贾政、贾琮、贾㻞、贾珩、贾珖、贾琛、贾琼、贾璘、贾蔷、贾菖、贾菱、贾芸、贾芹、贾蓁、贾萍、贾藻、贾蘅、贾芬、贾芳、贾兰、贾菌、贾芝等\geng{将贾族约略一总,观者方不惑。
}都来了。
贾珍哭的泪人一般, 
\jia{可笑,如丧考妣,
\zhu{妣:音“笔“。考妣:称已死的父母。见《礼记·曲礼下》:「生曰父、曰母、曰妻;死曰考、曰妣、曰嫔。」}
此作者刺心笔也。
}正和贾代儒等说道:“合家大小,远亲近友,谁不知我这媳妇比儿子还强十倍。
如今伸腿去了,可见这长房内绝灭无人了。
”说着又哭起来。
众人忙劝道:“人已辞世,哭也无益,且商议如何料理要紧。
”\geng{淡淡一句,勾出贾珍多少文字来。
}贾珍拍手道:“如何料理,不过尽我所有罢了!”\qi{“尽我所有”,为媳妇是非礼之谈,父母又将何以待之?故前此有恶奴酒后狂言,及今复见此语,含而不露,吾不能为贾珍隐讳。
}\ping{呼应第七回焦大醉骂:“那里承望到如今生下这些畜牲来!每日家偷狗戏鸡,爬灰的爬灰……”秦可卿因和公公贾珍有不伦之情而自缢身死,贾珍有愧,试图以厚葬的奢华弥补良心的不安。
贾珍对于儿媳妇的死,“哭的泪人一般”,极力称赞儿媳妇“比儿子还强十倍”,痛惜“长房内绝灭无人”,
显然超过了公公对于儿媳妇的礼数,宛如死去的不是儿媳妇,而是父母。
所以脂评会对贾珍的所作所为有“可笑,如丧考妣”和“父母又将何以待之”的评论。
}\par
正说着,只见秦业、秦钟并尤氏的几个眷属\jia{伏后文。
}尤氏姊妹也都来了。
\zhu{伏后文尤二姐和尤三姐在贾府的故事。}
贾珍便命贾琼、贾琛、贾璘、贾蔷四个人去陪客,一面吩咐去请钦天监阴阳司来择日,\zhu{钦天监阴阳司:钦天监:明、清时代的官署名,主管观天文、定历数、卜吉凶、辨禁忌等事。
阴阳司:作者虚拟的官署名。
}择准停灵七七四十九日,三日后开丧送讣闻。
这四十九日,单请一百单八众禅僧在大厅上拜大悲忏,\zhu{拜大悲忏:拜忏:请僧众念经拜佛,代人消灾或超度亡魂的一种宗教活动。
拜大悲忏:是在拜忏时念“大悲咒”。
大悲咒是唐代伽梵达磨译《千手千眼观世音菩萨广大圆满无碍大悲心陀罗尼经》中的咒语。
}超度前亡后化诸魂,
\zhu{化:死。如:「物化」、「羽化」。}
以免亡者之罪;另设一坛于天香楼上,\jia{删。
却是未删之笔。
}是九十九位全真道士,\zhu{全真道士:本指道士中信奉全真教派的人,后来也作为各派道士的通称。
全真教:系道教的北宗。
}打四十九日解冤洗业醮。
\zhu{业:同“孽”,罪过、邪恶的意思。
打醮(醮音“叫”):旧时请僧道设坛念经,祈福消灾,超度亡魂的一种宗教仪式。
}\ping{从“以免亡者之罪”、“另设一坛于天香楼上”和“打四十九日解冤洗业醮”可知,秦可卿并非病死而是非正常死亡,即“淫丧天香楼”,她的死亡隐含了罪恶与冤仇。
}然后停灵于会芳园中,灵前另有五十众高僧、五十众高道,对坛按七作好事。
\zhu{按七作好事:旧时迷信,认为人死后还会转生。
从刚死之日算起,每七天为一期,期满后即再降生;若一期届满未得生缘,须再等一期;最多到第七期,必定降生。
由于从已死到再生之间祸福未定,所以死者的亲属每隔七天要设奠一次,请僧道替死者诵经修福,直到七七为止。
}那贾敬闻得长孙媳妇死了,因自为早晚就要飞升,\geng{可笑可叹。
古今之儒,中途多惑老佛。
王隐梅云:“若能再加东坡十年寿,亦能跳出这圈子来。
”斯言信矣。
\zhu{
东坡:苏轼。苏轼一生受儒、道、释影响,尤其“乌台诗案”以后,
佛老思想对他影响更深,这从他的作品和生活交往中均可看出。
圈子:指老佛思想。
}
}\meng{“就要飞升”的“要”,用得的当。
\zhu{的当:确实;恰当。}
凡“要”者,则身心急切;急切之者,百事无成。
正为后文作引线。
}如何肯又回家染了红尘,将前功尽弃呢,因此并不在意,只凭贾珍料理。
\par
贾珍见父亲不管,亦发恣意奢华。
看板时,几副杉木板皆不中用。
可巧薛蟠来吊问,因见贾珍寻好板,便说道:“我们木店里有一副,叫做什么樯木,\zhu{樯:音“强”,船上的桅杆。
这里指用来做船桅杆之木,譬如楠木。
}\jia{樯者,舟具也。
所谓“人生若泛舟”而已,宁不可叹!}出在潢海铁网山上,
\zhu{潢:音“黄”,积水池。}
\jia{所谓迷津易堕,尘网难逃也。
}作了棺材,万年不坏。
这还是当年先父带来,原系义忠亲王老千岁要的,因他坏了事,\zhu{坏了事:这里指因获罪而被革去官爵。
}\meng{“坏了事”等字毒极,写尽势利场中故套。
}就不曾拿去。
现今还封在店里,也没人出价敢买。
你若要,就抬来罢了。
”贾珍听了,喜之不禁,即命人抬来。
大家看时,只见帮底皆厚八寸,纹若槟榔,味若檀麝,以手扣之,玎珰如金玉。
大家都奇异称赏。
贾珍笑道:“价值几何?”薛蟠笑道:“拿一千两银子来,只怕也没处买去。
什么价不价,赏他们几两工银就是了。
”\jia{的是阿呆兄口气。
}贾珍听说,忙谢不尽,即命解锯糊漆。
\zhu{解:剖开。}
贾政因劝道:“此物恐非常人可享者,\jia{政老有深意存焉。
}殓以上等杉木也就是了。
”\jia{夹写贾政。
}\jia{写个个皆到,全无安逸之笔,深得《金瓶》壸奥!
\zhu{
壸:音“捆”,宫中的道路。
奥:室隅,原意是屋内的深处,后用以比喻事理的精微之处。
《红楼梦》在题材、情节结构、人物描写、生活细节、语言等等多方面吸取《金瓶梅》中的精华部分,
并扬弃其糟粕。脂批“深得《金瓶》壸奥”一句,也可以说是对整部《红楼梦》的概评。
}
}
\ping{僭越逾秩,恐埋下隐患,日后爆发。
}
此时贾珍恨不能代秦氏之死,这话如何肯听。
\meng{“代秦氏死”等句,总是填实前文。
}
\zhu{程高本删去“恨不能代秦氏之死”等文字,掩盖贾珍的罪恶行径。}
\par
因忽又听得秦氏之丫鬟名唤瑞珠者,见秦氏死了,他也触柱而亡。
\jia{补天香楼未删之文。
}此事可罕,合族中人也都称赞。
贾珍遂以孙女之礼殡殓,一并停灵于会芳园之登仙阁。
小丫鬟名宝珠者,因见秦氏身无所出,乃甘心愿为义女,誓任摔丧驾灵之任。
\zhu{摔丧驾灵:旧日出殡,将起动棺材时,先由主丧孝子在灵前摔碎瓦盆一只,叫做“摔丧”,也称“摔盆”。
主丧孝子亲自抬扶灵柩或牵引灵车叫做“驾灵”。
后来,主丧孝子只在灵柩前领路,也称“驾灵”。
}贾珍喜之不禁,即时传下:从此皆呼宝珠为小姐。
那宝珠按未嫁女之丧,在灵前哀哀欲绝。
\jia{非恩惠爱人,那能如是?惜哉可卿,惜哉可卿!}于是,合族人丁并家下诸人,都各遵旧制行事,自不敢紊乱。
\jia{两句写尽大家。
}\chen{转叠法,叙前文未及。
\zhu{
转叠法:转叠法是转用中国画中画泉水的表现手法,如画叠泉,则回环曲折。
脂评所谓转叠法是指情节平叙中,又转叙“前文未及”之事,犹如泉水之转叠。
这实是文学创作中追叙、补叙的表现手法,亦如脂评另外提到的补遗法,即补写以前未叙之事。
}
}\par
贾珍因想着贾蓉不过是个黉门监,\zhu{
黉:音“洪”,古代学校名。
黉门监:即监生。
本指在明清时代最高学府国子监读书的学生,后来也可以捐钱买得,不一定要在国子监里读书。
}\geng{又起波澜,却不突然。
}灵幡经榜上写时不好看,
\zhu{
幡:音“翻”,挑起来直着挂的长条形旗子。
}
便是执事也不多,\zhu{执事:这里指仪仗,有时也指差事或当差的人。
}因此心下甚不自在。
\jia{善起波澜。
}可巧这日正是首七第四日,早有大明宫掌宫内相戴权,\zhu{内相:本为翰林的别称。
这里是对太监的尊称。
翰林:官名。唐玄宗初置翰林待诏,为文学侍从之官。
至德宗以后,翰林学士职掌为撰拟机要文书。
明清则以翰林院为“储才”之地,在科举考试中选拔一部分人入院为翰林官。
}\jia{妙!大权也。
}先备了祭礼遣人抬来,次后坐了大轿,打伞鸣锣,亲来上祭。
贾珍忙接着,让至逗蜂轩\jia{轩名可思。
\zhu{
“逗蜂轩”这样轻佻的名字,本不应出现在正统的国公府中。这个名字有更深的寓意。
古人形容女子行为不检点,会用“招蜂引蝶”形容。
“逗蜂”可能是暗示逗蜂采蝶,表示这里是狂蜂浪蝶发生艳事的地点,即指贾珍和秦可卿的私情。
另一种说法,蜂带刺,逗蜂必被蜂蜇。这里的“蜂”指的是戴权等掌权太监。
贾府和掌权太监的交往,类比为逗蜂,暗示太监将在已经遗失的贾府倒台情节处起到推波助澜的作用。
贾珍和戴权卖官鬻爵的违法勾当可能在遗失的后半部书中造成了严重的后果。
}
}献茶。
贾珍心中打算定了主意,因而趁便就说要与贾蓉蠲个前程的话。
\zhu{蠲:音“捐”,免除。
在这里可能通“捐”,即捐官、买官。
}戴权会意,因笑道:“想是为丧礼上风光些?”\jia{得内相机括之快如此。
\zhu{机括:弩上控制箭矢发射的机件。
泛指机械发动、开启的部分。
这里比喻心机、计谋。
得:获得,在这里意思是用文字描摹人物跃然纸上,如闻如见。
}}贾珍忙笑道:“老内相所见不差。
”戴权道:“事倒凑巧,正有个美缺。
如今三百员龙禁尉短了两员,昨儿襄阳侯的兄弟老三来求我,现拿了一千五百两银子,送到我家里。
你知道,咱们都是老相与,不拘怎么样,看着他爷爷的分上,胡乱应了。
\jia{忙中写闲。
}还剩了一个缺,谁知永兴节度使冯胖子来求,要与他孩子蠲,我就没工夫应他。
既是咱们的孩子\jia{奇谈,画尽阉官口吻。
}要蠲,快写个履历来。
”贾珍听说,忙吩咐:“快命书房里人恭敬写了大爷的履历来。
”小厮不敢怠慢,去了一刻,便拿了一张红纸来与贾珍。
贾珍看了,忙送与戴权。
戴权看时,上面写道:\par
\hop
江南江宁府江宁县监生贾蓉,年二十岁。
曾祖,原任京营节度使世袭一等神威将军贾代化;祖,乙卯科进士贾敬;父,世袭三品爵威烈将军贾珍。
\par
\hop
戴权看了,回手便递与一个贴身的小厮收了,说道:“回来送与户部堂官老赵,\zhu{堂官:明、清时代称各衙署的长官叫堂官。
}说我拜上他,起一张五品龙禁尉的票,再给个执照,就把那履历填上,明儿我来兑银子送去。
”小厮答应了,戴权也就告辞了。
贾珍十分款留不住,只得送出府门。
临上轿,贾珍因问:“银子还是我到部兑,还是一并送入老内相府中?”戴权道:“若到部里,你又吃亏了。
不如平准一千二百银子,\zhu{平准:古代官府平抑物价的措施。
《史记·平准书》:“大农之诸官,尽笼天下之货物,贵即卖之,贱则买之。
如此,富商大贾无所牟大利,则反本,而万物不得腾踊,故抑天下物,名曰平准。
”这里的平准引申为政府对于官职的买卖,即卖官鬻爵。
}送到我家里就完了。
”贾珍感谢不尽,只说:“待服满后,\zhu{服满:指服丧期满。
据《清朝通典·礼·凶二》载,父母对嫡长子之妻服丧,为期一年。
}亲带小犬到府叩谢。
”于是作别。
\par
接着,又听喝道之声,
\zhu{喝道:旧时官吏出行,前导的仪卫大声吆喝,叫行人让路。}
原来是忠靖侯史鼎的夫人来了。
\jia{史小姐湘云消息也。
}\qi{伏史湘云一笔。
\foot{己、庚本作“伏史湘云”并混入正文。
}}\chen{伏下文。
}
王夫人、邢夫人、凤姐等刚迎至上房,又见锦乡侯、川宁侯、寿山伯三家祭礼摆在灵前。
少时,三家下轿,贾政等忙接上大厅。
如此亲朋你来我去,也不能胜数。
只这四十九日,\geng{就简去繁。
}宁国府街上一条白漫漫人来人往,\jia{是有服亲朋并家下人丁之盛。
\zhu{有服:有丧服在身,指在丧期。
}}花簇簇宦去官来。
\jia{是来往祭吊之盛。
}\par
贾珍命贾蓉次日换了吉服,领凭回来。
灵前供用执事等物,俱按五品职例。
灵牌疏上皆写“天朝诰授贾门秦氏恭人之灵位”。
\zhu{
诰:音“告”,古代一种训诫勉励的文告。明、清之后,帝王授官、封赠的命令亦称诰。
恭人:封建时代,妇女根据丈夫或子孙的官职品级受封赠。
明、清时四品官的妻子叫“恭人”,五品官的妻子叫“宜人”。
贾蓉是五品龙禁尉,秦氏本应称“宜人”,此处明写“恭人”,或隐含讽喻。
}会芳园的临街大门洞开,现在两边起了鼓乐厅,两班青衣按时奏乐,\zhu{青衣:即皂服,黑色衣着,旧时地位低下的人所穿,后作为贱役人等的代称,如称婢女、吹鼓手和衙役等,这里指穿青衣的乐工。
}一对对执事摆的刀斩斧齐。
更有四面朱红销金大字牌对竖在门外,\zhu{销:熔化金属。
}上面大书:\par
\hop
防护内廷紫禁道 御前侍卫龙禁尉。
\par
\hop
对面高起着宣坛,\zhu{宣坛:僧道讲经作法时所设置的台子。
}僧道对坛榜文,榜上大书:\par
\hop
世袭宁国公冢孙妇、\zhu{冢孙:嫡长孙。
冢:音“肿”,大,引申为嫡长之意。
}防护内廷御前侍卫龙禁尉贾门秦氏恭人之丧。
\geng{贾珍是乱费,可卿却实如此。
}四大部州至中之地,\zhu{四大部州:印度古代传说,称人类所居的世界为四大部州。
佛教袭用其说,但其名称佛家典籍中说法不一。
}奉天承运太平之国,\geng{奇文。
若明指一州名,似若《西游》之套,故曰至中之地,不待言可知是光天化日仁风德雨之下矣。
不云国名更妙,可知是尧街舜巷衣冠礼义之乡矣。
直与第一回呼应相接。
\zhu{第一回:“然朝代年纪,地舆邦国,却反失落无考。”}
}总理虚无寂静教门僧录司正堂万虚、总理元始三一教门道录司正堂叶生等,\zhu{僧录司、道录司:明、清时代掌管全国僧道事务的最高官衙。
万虚、叶生皆为人名。
}敬谨修斋,朝天叩佛。
\par
\hop
以及——\par
\hop
恭请诸伽蓝、\zhu{伽蓝:伽音“加”,梵语僧伽摩兰的简称。
意思是僧众居住的园林、寺院。
这里指卫护园林、寺院的伽蓝神。
}揭谛、\zhu{揭谛:佛教传说中的护法猛神。
}功曹等神,\zhu{功曹:也称“四值功曹”。
道教传说他们是值年、月、日、时的神,掌管传递人间呈文给玉皇大帝。
}圣恩普锡,\zhu{锡:音“次”,赐。
}神威远镇,四十九日消灾洗业平安水陆道场。
\zhu{水陆道场:又叫水陆斋,简称水陆,是一种用诵经拜佛、施舍斋食来“超度”所谓水陆二界鬼众的佛教迷信活动。
创始于梁武帝萧衍。
}\par
\hop
诸如等语,馀者亦不消烦记。
\par
只是贾珍虽然此时心意满足,\meng{可笑。
}但里头尤氏又犯了旧疾,不能料理事务,\ping{因何犯旧疾?恐怕是贾珍丑事打击所致,不愿料理贾珍情人秦可卿的葬礼。
从后文尤氏料理凤姐生日和贾敬丧礼可见,尤氏本身也有能力。
}惟恐各诰命来往,\zhu{诰命:本指皇帝赐爵授官的诏令,在此义同“命妇”,代指受皇帝封赠的贵妇人。
}亏了礼数,怕人笑话,因此心中不自在。
当下正忧虑时,因宝玉\jia{余正思如何高搁起玉兄了。
}在侧问道:“事事都算安贴了,大哥哥还愁什么?”贾珍见问,忙将里面无人的话说了出来。
宝玉听说笑道:“这有何难,我荐一个人\jia{荐凤姐须得宝玉,俱龙华会上人也。
\zhu{龙华会即浴佛节,为每年的农历四月初八,是中国佛教徒纪念教主释迦牟尼佛诞辰的一个重要节日,又称佛诞节、灌佛会、龙华会、华严会等。
中国历史记载佛诞为周昭王二十四年(公元前1027年),释迦牟尼从摩耶夫人的肋下降生时,一手指天,一手指地,说“天上天下,惟我独尊。
”于是大地为之震动,九龙吐水为之沐浴。
因此各国各民族的佛教徒通常都以浴佛等方式纪念佛的诞辰。
龙华会上人可能暗示了宝玉凤姐在小说中地位之尊贵特殊,也可能暗示了宝玉的佛缘和出家的结局。
另一种说法,脂批这里用“龙华会”指宝玉凤姐都是有宿根、爱热闹者,所以主持秦可卿出殡之事,必得由宝玉荐凤姐才合适。
}}与你权理这一个月的事,管必妥当。
”贾珍忙问:“是谁?”宝玉见座间还有许多亲友,不便明言,走至贾珍耳边说了两句。
贾珍听了喜不自禁,连忙起身笑道:“果然安贴,如今就去。
”说着拉了宝玉,辞了众人,便往上房里来。
\par
可巧这日非正经日期,\zhu{正经日期:丧礼诵经期间吊祭死者的正日子。
经:指诵经。
}亲友来的少,里面不过几位近亲堂客,邢夫人、王夫人、凤姐并合族中的内眷陪坐。
有人报说:“大爷进来了。
”吓的众婆娘唿的一声,往后藏之不迭,\jia{数日行止可知。
作者自是笔笔不空,批者亦字字留神之至矣。
}独凤姐款款站了起来。
\geng{又写凤姐。
}贾珍此时也有些病症在身,二则过于悲痛了,因拄了拐踱了进来。
邢夫人等因说道:“你身上不好,又连日事多,该歇歇才是,又进来做什么?”贾珍一面扶拐,\geng{一丝不乱。
}扎挣着要蹲身跪下请安道乏。
\zhu{扎挣:勉强支持。
}邢夫人等忙叫宝玉搀住,命人挪椅子来与他坐。
贾珍断不肯坐,因勉强陪笑道:“侄儿进来有一件事要恳求二位婶婶并大妹妹。
”邢夫人等忙问:“什么事?”贾珍忙笑道:“婶婶自然知道,如今孙子媳妇没了,侄儿媳妇偏又病倒,我看里头着实不成个体统。
怎么屈尊大妹妹一个月,\geng{不见突然。
}在这里料理料理,我就放心了。
”\geng{阿凤此刻心痒矣。
}邢夫人笑道:“原来为这个。
你大妹妹现在你二婶子家,只和你二婶子说就是了。
”\ping{邢夫人自己的儿媳妇,却在王夫人那里管家,邢夫人对自己的处境并不满意。
第六十五回兴儿的叙述:“如今连他(凤姐)正经婆婆大太太(邢夫人)都嫌了他,说他‘雀儿拣着旺处飞,黑母鸡一窝儿,自家的事不管,倒替人家去瞎张罗’。
”}
王夫人忙道:“他一个小孩子\geng{三字愈令人可爱可怜。
}家,何曾经过这样事,倘或料理不清,反叫人笑话,倒是再烦别人好。
”贾珍笑道:“婶子的意思侄儿猜着了,是怕大妹妹劳苦了。
若说料理不开,我包管必料理的开,便是错一点儿,别人看着还是不错的。
从小儿大妹妹顽笑着就有杀伐决断,\geng{阿凤身份。
}如今出了阁,又在那府里办事,越发历练老成了。
我想了这几日,除了大妹妹再无人了。
婶婶不看侄儿、侄儿媳妇的分上,只看死了的分上罢!”说着滚下泪来。
\geng{有笔力。
}\par
王夫人心中怕的是凤姐未经过丧事,怕他料理不清,惹人笑话。
今见贾珍苦苦的说到这步田地,心中已活了几分,却又眼看着凤姐出神。
那凤姐素日最喜揽事办,好卖弄才干,虽然当家妥当,也因未办过婚丧大事,恐人还不服,巴不得遇见这事。
今日见贾珍如此一来,他心中早已欢喜。
先见王夫人不允,后见贾珍说的情真,王夫人有活动之意,便向王夫人道:“大哥哥说的这么恳切,太太就依了罢。
”王夫人悄悄的道:“你可能么?”凤姐道:“有什么不能的。
外面的大事大哥哥\geng{王夫人是悄言,凤姐是响应,故称“大哥哥”。
}已经料理清了,\geng{已得三昧矣。
\zhu{三昧:佛教用语。
本意是心神专一,杂念止息,是佛家修持的重要方法之一。
后借指事物的奥秘和精义。
}}
不过是里头照管照管,便是我有不知道的,问问太太就是了。
”\jia{胸中成见已有之语。
}王夫人见说的有理,便不则声。
\zhu{则:做。
则声:开口发言、出声。
}贾珍见凤姐允了,又陪笑道:“也管不得许多了,横竖要求大妹妹辛苦辛苦。
我这里先与妹妹行礼,等事完了,我再到那府里去谢。
”说着,就作揖下去,凤姐儿还礼不迭。
\par
贾珍便忙向袖中取了宁国府对牌出来,\zhu{对牌:用木、竹制成的支领财物的凭证,上有标记,从中劈作两半。
支领财物时,以两半标记相合为凭。
}命宝玉送与凤姐,又说:“妹妹爱怎么样就怎么样,要什么只管拿这个取去,也不必问我。
只别存心替我省钱,只要好看为上;二则也要与那府里一样待人才好,不要存心怕人抱怨。
只这两件外,我再没不放心的了。
”凤姐不敢就接牌,\qi{凡有本领者断不越礼。
接牌小事而必待命于王夫人也,诚家道之规范,亦天下之规范也。
看是书者不可草草从事。
}只看着王夫人。
王夫人道:“你哥哥既这么说,你就照看照看罢了。
只是别自作主意,有了事,打发人问你哥哥、嫂子要紧。
”宝玉早向贾珍手里接过对牌来,强递与凤姐了。
贾珍又问:“妹妹还是住在这里,还是天天来呢?若是天天来,越发辛苦了。
不如我这里赶着收拾出一个院落来,妹妹住过这几日倒安稳。
”凤姐笑道:“不用。
\jia{二字句,有神。
}那边也离不得我,倒是天天来的好。
”贾珍听说,只得罢了。
然后又说了一回闲话,方才出去。
\par
一时女眷散后,王夫人因问凤姐:“你今儿怎么样?”凤姐儿道:“太太只管请回去,我须得先理出一个头绪来,才回去得呢。
”王夫人听说,便先同邢夫人等回去,不在话下。
\par
这里凤姐儿来至三间一所抱厦内坐了,
\zhu{抱厦:原建筑之前或之后接建出来的小房子。}
因想:头一件是人口混杂,遗失东西;第二件,事无专执,临期推委;第三件,需用过费,滥支冒领;第四件,任无大小,苦乐不均;第五件,家人豪纵,有脸者不服钤束,\zhu{
钤:音“前”,锁。
钤束:约束、管制的意思。
}无脸者不能上进。
\jia{旧族后辈受此五病者颇多,余家更甚。
三十年前事见书于三十年后,令余悲恸血泪盈面。
}\geng{读五件事未完,余不禁失声大哭,三十年前作书人在何处耶?}此五件实是宁国府中风俗。
不知凤姐如何处治,且听下回分解。
\jia{此回只十页,因删去天香楼一节,少去四五页也。
}正是:\qi{五件事若能如法整理得当,岂独家庭,国家天下治之不难。
}\par
金紫万千谁治国,裙钗一二可齐家。
\par
\jia{“秦可卿淫丧天香楼”,作者用史笔也。
老朽因有魂托凤姐贾家后事二件,嫡是安富尊荣坐享人能想得到处。
\zhu{嫡是:的确是。}
其事虽未漏,其言其意则令人悲切感服,姑赦之,因命芹溪删去。
}\par
\geng{通回将可卿如何死故隐去,是大发慈悲心也,叹叹!壬午春\foot{甲、庚本这两条批语,均批于回末空白处,但其性质并非总评,而属于侧批或眉批一类。
}。
}\par
\qi{总评:借可卿之死,又写出情之变态,上下大小,男女老少,无非情感而生情。
且又藉凤姐之梦,更化就幻空中一片贴切之情,所谓寂然不动,感而遂通。
所感之象,所动之萌,深浅诚伪,随种必报,所谓幻者此也,情者亦此也。
何非幻,何非情?情即是幻,幻即是情,明眼者自见。
}
\dai{025}{薛蟠带来上好棺木}
\dai{026}{贾珍求凤姐料理丧事}
\sun{p13-1}{秦可卿托梦王熙凤,贾宝玉闻丧讯吐血}{图右下:一日深夜,凤姐睡梦中恍惚见秦氏从外进来,含笑道别,说有一件心事要托付给她。
秦氏言道:“我家赫赫扬扬,已将百载,若不早为后虑,一旦乐极生悲,也只恐后悔不及。
”凤姐问:“如何可保无虞?”秦氏嘱道,将祖茔附近多置田庄、房舍、地亩,家塾亦设于此,合族轮流掌管地亩钱粮祭礼供给之事。
将来即使有罪败落,也有个退身处。
待凤姐还要问时,只听云板连叩四声,随即惊醒。
有人回道“东府蓉大奶奶没了。
”图右上:宝玉那边,梦中听见秦氏死了,连忙翻身爬起,不觉“哇”的一声,直喷出一口血来,图中部:又闹着贾母派车送他到宁府去。
}
\sun{p13-2}{秦可卿死封龙禁尉,贾珍托付王熙凤}{贾珍极尽奢华料理儿媳丧事。
宁国府街上白漫漫人来人往,花簇簇宦去官来。
图右侧:首七第四天,大明宫掌宫太监戴权打道鸣锣亲来上祭。
贾珍为办事体面,趁机用一千二百两银子给贾蓉捐了前程。
图左侧:尤氏犯了旧疾,不能料理事务。
贾珍听了宝玉的建议,径来恳请邢、王两位夫人让王熙凤协理家事。
那凤姐素日最喜揽事.卖弄才干,便向王夫人道:“大哥哥说得如此恳切,太太就依了吧。
”}