\chapter{变生不测凤姐泼醋 \quad 喜出望外平儿理妆}
\qi{云雨谁家院,飘来花自奇。
\zhu{
这首诗的前两句似乎是指贾琏偷情,“云雨”即男女幽会发生性关系的隐语,出自楚怀王梦见高唐神女,醒来后却只见云飘雨飞的故事。
“飘来花自奇”似是指鲍二家的,所谓家花不如野花香的意思。
}
莺莺燕燕斗芳菲,枝枝因风滴玉露,正春时。
\zhu{
“斗”当是指凤姐、平儿、鲍二家的厮打等情节。“滴玉露”似是指平儿受委屈哭泣,后面宝玉为平儿的命运感伤,也滴下泪来。
宝玉的“意淫”不仅让平儿获得意外之喜,自己也感到“亦今生意中不想之乐也”,这可能是“正春时”之寓意。
}
}\par
话说众人看演《荆钗记》,\zhu{《荆钗记》:南戏剧本,描写王十朋和钱玉莲悲欢离合的故事。
}宝玉和姐妹一处坐着。
林黛玉因看到《男祭》这一出上,便和宝钗说道:“这王十朋也不通的很,不管在那里祭一祭罢了,必定跑到江边子上来作什么!俗语说‘睹物思人’,天下的水总归一源,不拘那里的水舀一碗看着哭去,也就尽情了。
”\ping{宝玉在上一回非要去井边祭祀投井而死的金钏,黛玉知道宝玉的心事,所以如此说。
《荆钗记》是讲穷书生王十朋中了状元后,宰相逼他娶自己的女儿,可是王十朋不愿意。于是宰相暗中把他的家书改成了休书,他的妻子钱玉莲绝望之下就跳江自杀了。
消息传来,王十朋就跑到江边大哭,这个片段叫《祭江》。
黛玉的意思是:你用得着那么偷偷摸摸、大费周节地跑那么远去祭拜吗?如果这个人是跳水死的,天下的水都是一样,你舀碗水,对着哭就好了。黛玉其实有赞美他的意思,可是也好像在提醒他,何必拘泥这些形式。
}宝钗不答。
宝玉回头要热酒敬凤姐儿。
\par
原来贾母说今日不比往日,定要叫凤姐痛乐一日。
本来自己懒待坐席,只在里间屋里榻上歪着和薛姨妈看戏,随心爱吃的拣几样放在小几上,随意吃着说话儿;将自己两桌席面赏那没有席面的大小丫头并那应差听差的妇人等,命他们在窗外廊檐下也只管坐着随意吃喝,不必拘礼。
王夫人和邢夫人在地下高桌上坐着,外面几席是他姊妹们坐。
贾母不时吩咐尤氏等:“让凤丫头坐在上面,你们好生替我待东,\zhu{待东:代东,代东道主招待客人。
}难为他一年到头辛苦。
”尤氏答应了,又笑回说道:“他坐不惯首席,坐在上头横不是竖不是的,酒也不肯吃。
”贾母听了,笑道:“你不会,等我亲自让他去。
”凤姐儿忙也进来笑说:“老祖宗别信他们的话,我吃了好几钟了。
”贾母笑着,命尤氏:“快拉他出去,按在椅子上,你们都轮流敬他。
他再不吃,我当真的就亲自去了。
”尤氏听说,忙笑着又拉他出来坐下,命人拿了台盏斟了酒,\zhu{台盏:有台式盏托的酒盅。
一说台盏即为大酒杯。
}笑道:“一年到头难为你孝顺老太太、太太和我。
我今儿没什么疼你的,亲自斟杯酒,乖乖儿的在我手里喝一口。
”凤姐儿笑道:“你要安心孝敬我,跪下我就喝。
”尤氏笑道:“说的你不知是谁!我告诉你说,好容易今儿这一遭,过了后儿,知道还得像今儿这样不得了?趁着尽力灌丧两钟罢。
”\zhu{灌丧:骂人喝酒之辞。
}\geng{闲闲一戏语,伏下后文,令人可伤,所谓“盛筵难再”。
}凤姐儿见推不过,只得喝了两钟。
接着众姊妹也来,凤姐也只得每人的喝一口。
赖大妈妈见贾母尚这等高兴,也少不得来凑趣儿,领着些嬷嬷们也来敬酒。
凤姐儿也难推脱,只得喝了两口。
鸳鸯等也来敬,凤姐儿真不能了,忙央告道:“好姐姐们,饶了我罢,我明儿再喝罢。
”鸳鸯笑道:“真个的,我们是没脸的了?就是我们在太太跟前,太太还赏个脸儿呢。
往常倒有些体面,今儿当着这些人,倒拿起主子的款儿来了。
我原不该来。
不喝,我们就走。
”说着真个回去了。
凤姐儿忙赶上拉住,笑道:“好姐姐,我喝就是了。
”说着拿过酒来,满满的斟了一杯喝干。
鸳鸯方笑了散去,然后又入席。
\par
凤姐儿自觉酒沉了,\zhu{酒沉了:饮酒过量的意思。
}心里突突的似往上撞,要往家去歇歇,只见那耍百戏的上来,便和尤氏说:“预备赏钱,我要洗洗脸去。
”尤氏点头。
凤姐儿瞅人不防,便出了席,往房门后檐下走来。
平儿留心,也忙跟了来,凤姐儿便扶着他。
才至穿廊下,只见他房里的一个小丫头正在那里站着,见他两个来了,回身就跑。
凤姐儿便疑心忙叫。
那丫头先只装听不见,无奈后面连平儿也叫,只得回来。
凤姐儿越发起了疑心,忙和平儿进了穿堂,叫那小丫头子也进来,把槅扇关了,\zhu{槅扇:即“隔扇”,在房屋内部作隔开用的一扇扇木板墙或纸壁,上部一般做成窗棂,糊纸或装玻璃。
}凤姐儿坐在小院子的台阶上,命那丫头子跪了,喝命平儿:“叫两个二门上的小厮来,拿绳子鞭子,把那眼睛里没主子的小蹄子打烂了!”那小丫头子已经唬的魂飞魄散,哭着只管碰头求饶。
\zhu{碰头:磕头。
}凤姐儿问道:“我又不是鬼,你见了我,不说规规矩矩站住,怎么倒往前跑?”小丫头子哭道:“我原没看见奶奶来。
我又记挂着房里无人,所以跑了。
”凤姐儿道:“房里既没人,谁叫你来的?你便没看见我,我和平儿在后头扯着脖子叫了你十来声,越叫越跑。
离的又不远,你聋了不成?你还和我强嘴!”
\zhu{强嘴:现在一般写作“犟嘴”。}
说着便扬手一掌打在脸上,打的那小丫头一栽;这边脸上又一下,登时小丫头子两腮紫胀起来。
\ping{凤姐在第二十九回打误撞自己的、剪蜡花的小道士也是如此,有个九省统制(武职)的舅舅王子腾,凤姐也不逊色。
}平儿忙劝:“奶奶仔细手疼。
”\ping{平儿会劝,若是劝说这小丫头不懂事别这么打,估计还得被甩几个巴掌。
}凤姐便说:“你再打着问他跑什么。
他再不说,把嘴撕烂了他的!”那小丫头子先还强嘴,后来听见凤姐儿要烧了红烙铁来烙嘴,方哭道:“二爷在家里,打发我来这里瞧着奶奶的,若见奶奶散了,先叫我送信儿去的。
不承望奶奶这会子就来了。
”凤姐儿见话中有文章,便又问道:“叫你瞧着我作什么?难道怕我家去不成?必有别的原故,快告诉我,我从此以后疼你。
你若不细说,立刻拿刀子来割你的肉。
”说着,回头向头上拔下一根簪子来,向那丫头嘴上乱戳,唬的那丫头一行躲,一行哭求道:“我告诉奶奶,可别说我说的。
”平儿一旁劝,一面催他,叫他快说。
丫头便说道:“二爷也是才来房里的,睡了一会醒了,打发人来瞧瞧奶奶,说才坐席,还得好一会才来呢。
二爷就开了箱子,拿了两块银子,还有两根簪子,两匹缎子,叫我悄悄的送与鲍二的老婆去,叫他进来。
他收了东西就往咱们屋里来了。
二爷叫我来瞧着奶奶,底下的事我就不知道了。
”\par
凤姐听了,已气的浑身发软,忙立起来一径来家。
刚至院门,只见又有一个小丫头在门前探头儿,一见了凤姐,也缩头就跑。
\geng{如见其形。
}凤姐儿提着名字喝住。
那丫头本来伶俐,见躲不过了,越性跑了出来,笑道:“我正要告诉奶奶去呢,可巧奶奶来了。
”凤姐儿道:“告诉我什么?”那小丫头便说二爷在家这般如此如此,将方才的话也说了一遍。
凤姐啐道:“你早作什么了?这会子我看见你了,你来推干净儿!”说着也扬手一下打的那丫头一个趔趄,便摄手摄脚的走至窗前,往里听时,只听里头说笑。
那妇人笑道:“多早晚你那阎王老婆死了就好了。
”贾琏道:“他死了,再娶一个也是这样,又怎么样呢?”那妇人道:“他死了,你倒是把平儿扶了正,只怕还好些。
”贾琏道:“如今连平儿他也不叫我沾一沾了。
平儿也是一肚子委曲不敢说。
我命里怎么就该犯了‘夜叉星’。
”\par
凤姐听了,气的浑身乱战,\zhu{战:通“颤”,发抖。
}又听他俩都赞平儿,便疑平儿素日背地里自然也有愤怨语了,那酒越发涌了上来,也并不忖度,回身把平儿先打了两下,\geng{奇极!先打平儿可是世人想得着的?}\ping{平儿作为贾琏的妾,两人慑于凤姐的淫威,只有名分而没有实际。
凤姐知道平儿因自己受到了委屈,所以对于她愤怨自己早有心理预期,因为心虚,所以才很自然地恶意揣测平儿。
}一脚踢开门进去,也不容分说,抓着鲍二家的撕打一顿。
又怕贾琏走出去,便堵着门站着骂道:“好淫妇!你偷主子汉子,还要治死主子老婆!平儿过来!你们淫妇忘八一条藤儿,\zhu{忘八:可能是“王八”,骂人的话,指妻子有外遇的男人。
}多嫌着我,外面儿你哄我!”说着又把平儿打几下,打的平儿有冤无处诉,只气得干哭,骂道:“你们做这些没脸的事,好好的又拉上我做什么!”说着也把鲍二家的撕打起来。
贾琏也因吃多了酒,进来高兴,未曾作的机密,一见凤姐来了,已没了主意,又见平儿也闹起来,把酒也气上来了。
凤姐儿打鲍二家的,他已又气又愧,只不好说的,今见平儿也打,便上来踢骂道:“好娼妇!你也动手打人!”平儿气怯,忙住了手,哭道:“你们背地里说话,为什么拉我呢?”凤姐见平儿怕贾琏,越发气了,又赶上来打着平儿,偏叫打鲍二家的。
平儿急了,便跑出来找刀子要寻死。
外面众婆子丫头忙拦住解劝。
这里凤姐见平儿寻死去,便一头撞在贾琏怀里,叫道:“你们一条藤儿害我,被我听见了,倒都唬起我来。
你也勒死我!”贾琏气的墙上拔出剑来,说道:“不用寻死,我也急了,一齐杀了,我偿了命,大家干净。
”\ping{第二十一回,贾琏因凤姐吃醋不让自己碰别的女人,已经说过了:“多早晚都死在我手里”的话,那时候贾琏杀心已起,这时候贾琏算是动真格的了。
}正闹的不开交,只见尤氏等一群人来了,说:“这是怎么说,才好好的,就闹起来。
”贾琏见了人,越发“倚酒三分醉”,逞起威风来,\geng{天下小人大都如是。
}故意要杀凤姐儿。
凤姐儿见人来了,便不似先前那般泼了,\geng{天下奸雄妒妇恶妇大都如是,只是恨无阿凤之才耳。
}丢下众人,便哭着往贾母那边跑。
\par
此时戏已散出,凤姐跑到贾母跟前,爬在贾母怀里,只说:“老祖宗救我!琏二爷要杀我呢!”\geng{瞧他称呼。
}贾母、邢夫人、王夫人等忙问怎么了。
凤姐儿哭道:“我才家去换衣裳,不防琏二爷在家和人说话,我只当是有客来了,唬得我不敢进去。
在窗户外头听了一听,原来是和鲍二家的媳妇商议,说我利害,要拿毒药给我吃了治死我,把平儿扶了正。
我原气了,又不敢和他吵,原打了平儿两下,问他为什么要害我。
他臊了,就要杀我。
”贾母等听了,都信以为真,说:“这还了得!快拿了那下流种子来!”一语未完,只见贾琏拿着剑赶来,后面许多人跟着。
贾琏明仗着贾母素昔疼他们,\zhu{素昔:素来。
}连母亲婶母也无碍,故逞强闹了来。
邢夫人王夫人见了,气的忙拦住骂道:“这下流种子!你越发反了,老太太在这里呢!”贾琏乜斜着眼,\zhu{乜(音“咩”)斜 :眯着眼睛,斜眼看人。
}道:“都是老太太惯的他,他才这样,连我也骂起来了!”邢夫人气的夺下剑来,只管喝他“快出去!”那贾琏撒娇撒痴,涎言涎语的还只乱说。
\zhu{涎言涎语:厚着脸皮胡言乱语,撒赖。}
贾母气的说道:“我知道你也不把我们放在眼里,叫人把他老子叫来!”贾琏听见这话,方趔趄着脚儿出去了,赌气也不往家去,便往外书房来。
\par
这里邢夫人王夫人也说凤姐儿。
贾母笑道:“什么要紧的事!小孩子们年轻,馋嘴猫儿似的,那里保得住不这么着。
从小儿世人都打这么过的。
\ping{贾母也是经历过这种事的人。
}都是我的不是,他多吃了两口酒,又吃起醋来。
”说的众人都笑了。
贾母又道:“你放心,等明儿我叫他来替你赔不是。
你今儿别要过去臊着他。
”因又骂:“平儿那蹄子,素日我倒看他好,怎么暗地里这么坏。
”尤氏等笑道:“平儿没有不是,是凤丫头拿着人家出气。
两口子不好对打,都拿着平儿煞性子。
平儿委曲的什么似的呢,老太太还骂人家。
”贾母道:“原来这样,我说那孩子倒不像那狐媚魇道的。
\zhu{狐媚魇道:用邪魔外道来迷惑陷害人。
俗传狐狸精能幻化迷人,故称用阴柔手段迷惑人为狐媚。
魇:音“演”,俗传使人在睡梦中惊恐的鬼怪。
}既这么着,可怜见的,白受他们的气。
”因叫琥珀来:“你出去告诉平儿,就说我的话:我知道他受了委曲,明儿我叫凤姐儿替他赔不是。
今儿是他主子的好日子,不许他胡闹。
”\par
原来平儿早被李纨拉入大观园去了。
\geng{可知吃蟹一回非闲文也。
\zhu{第三十九回,李纨拉平儿坐下吃酒吃螃蟹,感叹平儿虽好体面模样儿,但只落得屋里使唤,并摸到了平儿随身带的钥匙。
}}平儿哭得哽咽难\sout{抬}[抑]。
宝钗劝道:“你是个明白人,\geng{必用宝钗评出方是身份。
}
素日凤丫头何等待你,今儿不过他多吃一口酒。
他可不拿你出气,难道倒拿别人出气不成?别人又笑话他吃醉了。
你只管这会子委曲,素日你的好处,岂不都是假的了?”正说着,只见琥珀走来,说了贾母的话。
平儿自觉面上有了光辉,方才渐渐的好了,也不往前头来。
宝钗等歇息了一回,方来看贾母凤姐。
\par
宝玉便让平儿到怡红院中来。
袭人忙接着,笑道:“我先原要让你的,只因大奶奶和姑娘们都让你,我就不好让的了。
”平儿也陪笑说:“多谢。
”因又说道:“好好儿的从那里说起,无缘无故白受了一场气。
”袭人笑道:“二奶奶素日待你好,这不过是一时气急了。
”平儿道:“二奶奶倒没说的,只是那淫妇治的我,他又偏拿我凑趣,况还有我们那糊涂爷倒打我。
”说着便又委曲,禁不住落泪。
宝玉忙劝道:“好姐姐,别伤心,我替他两个赔不是罢。
”平儿笑道:“与你什么相干?”宝玉笑道:“我们弟兄姊妹都一样。
他们得罪了人,我替他赔个不是也是应该的。
”又道:“可惜这新衣裳也沾了,这里有你花妹妹的衣裳,何不换了下来,拿些烧酒喷了熨一熨。
把头也另梳一梳,洗洗脸。
”一面说,一面便吩咐了小丫头子们舀洗脸水,烧熨斗来。
平儿素习只闻人说宝玉专能和女孩儿们接交;宝玉素日因平儿是贾琏的爱妾,又是凤姐儿的心腹,故不肯和他厮近,因不能尽心,也常为恨事。
平儿今见他这般,心中也暗暗的敁敠:\zhu{敁敠:音“颠多”,也写作“掂掇”,估量、盘算、斟酌的意思。
}果然话不虚传,色色想的周到。
\zhu{色色:样样。
}又见袭人特特的开了箱子,\zhu{特特:特地。
}拿出两件不大穿的衣裳来与他换,便赶忙的脱下自己的衣服,忙去洗了脸。
宝玉一旁笑劝道:“姐姐还该擦上些脂粉,不然倒像是和凤姐姐赌气了似的。
况且又是他的好日子,而且老太太又打发了人来安慰你。
”平儿听了有理,便去找粉,只不见粉。
宝玉忙走至妆台前,将一个宣窑磁盒揭开,\zhu{宣窑:明代宣德年间的官窑。
其所产瓷器,细巧精致,光彩夺目,小件尤胜,以鲜红色最为名贵。
磁:通「瓷」。以瓷土烧制成的器物。
}里面盛着一排十根玉簪花棒,
\zhu{玉簪花棒:指含苞未放的玉簪花的花冠。玉簪花:多年生草本植物,花洁白如玉,花冠如筒状,未开时状如簪头。}
拈了一根递与平儿。
又笑向他道:“这不是铅粉,这是紫茉莉花种,
\zhu{种:种子。}
研碎了兑上香料制的。
”平儿倒在掌上看时,果见轻白红香,四样俱美,摊在面上也容易匀净,且能润泽肌肤,不似别的粉青重涩滞。
然后看见胭脂也不是成张的,\zhu{成张胭脂:即绵胭脂,是把丝绵薄片、棉花薄片浸到红花液或紫矿液内,充分吸收红色水液,同时用重物将其压实,然后再晾干。
如此制出的绵胭脂可以长期保存,更方便运输、买卖。
使用时,剪下适量绵片,浸在温水中,以其染就的红液化妆或给绘画上色。
}却是一个小小的白玉盒子,里面盛着一盒,如玫瑰膏子一样。
宝玉笑道:“那市卖的胭脂都不干净,颜色也薄。
这是上好的胭脂拧出汁子来,淘澄净了渣滓,
\zhu{淘澄[dèng]:把某种东西放在水中淘洗过滤,以除去杂质。}
配了花露蒸叠成的。
\zhu{蒸叠:多次蒸馏精制提炼。}
只用细簪子挑一点儿抹在手心里,用一点水化开抹在唇上;手心里就够打颊腮了。
”平儿依言妆饰,果见鲜艳异常,且又甜香满颊。
宝玉又将盆内的一枝并蒂秋蕙用竹剪刀撷了下来,\zhu{竹剪刀:竹制的供剪花等用的剪刀。
}与他簪在鬓上。
忽见李纨打发丫头来唤他,方忙忙的去了。
\geng{忽使平儿在绛芸轩中梳妆,非世人想不到,宝玉亦想不到者也。
作者费尽心机了。
写宝玉最善闺阁中事,诸如脂粉等类,不写成别致文章,则宝玉不成宝玉矣。
然要写又不便特为此费一番笔墨,故思及借人发端。
然借人又无人,若袭人辈则逐日皆如此,又何必拣一日细写?似觉无味。
若宝钗等又系姊妹,更不便来细搜袭人之妆奁,况也是自幼知道的了。
因左想右想须得一个又甚亲、又甚疏、又可唐突、又不可唐突、又和袭人等极亲、又和袭人等不大常处、又得袭人辈之美、又不得袭人辈之修饰一人来,方可发端。
故思及平儿一人方如此,故放手细写绛芸闺中之什物也。
\zhu{什物:泛指家庭日用的各种器物。
}}\par
宝玉因自来从未在平儿前尽过心,——且平儿又是个极聪明极清俊的上等女孩儿,比不得那起俗蠢拙物——深为恨怨。
今日是金钏儿的生日,故一日不乐。
\geng{原来为此!宝玉之私祭,玉钏之潜哀俱针对矣。
然于此刻补明,又一法也。
真千变万化之文,万法具备,毫无脱漏,真好书也。
}不想落后闹出这件事来,竟得在平儿前稍尽片心,亦今生意中不想之乐也。
因歪在床上,心内怡然自得。
忽又思及贾琏惟知以淫乐悦己,并不知作养脂粉。
又思平儿并无父母、兄弟姊妹,独自一人,供应贾琏夫妇二人。
贾琏之俗,凤姐之威,他竟能周全妥贴,今儿还遭涂毒,
\zhu{涂毒:毒害,引申为戕害、欺负。}
想来此人薄命,比黛玉犹甚。
想到此间,便又伤感起来,不觉洒然泪下。
因见袭人等不在房内,尽力落了几点痛泪。
复起身,又见方才的衣裳上喷的酒已半干,便拿熨斗熨了叠好;见他的手帕子忘去,上面犹有泪渍,又拿至脸盆中洗了晾上。
又喜又悲,闷了一回,也往稻香村来,说一回闲话,掌灯后方散。
\par
平儿就在李纨处歇了一夜,凤姐儿只跟着贾母。
贾琏晚间归房,冷清清的,又不好去叫,只得胡乱睡了一夜。
次日醒了,想昨日之事,大没意思,后悔不来。
\zhu{后悔不来:表示事情已经发生,无法追悔或补救。}
邢夫人记挂着昨日贾琏醉了,忙一早过来,叫了贾琏过贾母这边来。
贾琏只得忍愧前来,在贾母面前跪下。
贾母问他:“怎么了?”贾琏忙陪笑说:“昨儿原是吃了酒,惊了老太太的驾了,今儿来领罪。
”贾母啐道:“下流东西,灌了黄汤,\zhu{黄汤:指酒(骂人喝酒时说)。
}不说安分守己的挺尸去,倒打起老婆来了!凤丫头成日家说嘴,霸王似的一个人,昨儿唬得可怜。
要不是我,你要伤了他的命,这会子怎么样?”贾琏一肚子的委屈,不敢分辩,只认不是。
贾母又道:“那凤丫头和平儿还不是个美人胎子?你还不足!成日家偷鸡摸狗,脏的臭的,都拉了你屋里去。
为这起淫妇打老婆,又打屋里的人,你还亏是大家子的公子出身,活打了嘴了。
若你眼睛里有我,你起来,我饶了你,乖乖的替你媳妇赔个不是,拉了他家去,我就喜欢了。
要不然,你只管出去,我也不敢受你的跪。
”贾琏听如此说,又见凤姐儿站在那边,也不盛妆,哭的眼睛肿着,也不施脂粉,黄黄脸儿,\geng{大妙大奇之文,此一句便伏下病根了,草草看去便可惜了作者行文苦心。
}比往常更觉可怜可爱。
\ping{贾琏要提剑杀的是昨天撒泼大闹给自己难堪的凤姐,面对柔弱委屈的凤姐反而能够接受,贾琏是吃着碗里的看着锅里的,要温柔的凤姐,也要别的女人。
}想着:“不如赔了不是,彼此也好了,又讨老太太的喜欢了。
”想毕,便笑道:“老太太的话,我不敢不依,只是越发纵了他了。
”贾母笑道:“胡说!我知道他最有礼的,再不会冲撞人。
他日后得罪了你,我自然也作主,叫你降伏就是了。
”\par
贾琏听说,爬起来,便与凤姐儿作了一个揖,笑道:“原来是我的不是,二奶奶饶过我罢。
”满屋里的人都笑了。
贾母笑道:“凤丫头,不许恼了,再恼我就恼了。
”说着,又命人去叫了平儿来,命凤姐儿和贾琏两个安慰平儿。
贾琏见了平儿,越发顾不得了,\geng{所谓“妻不如妾,妾不如偷\foot{所谓“妻不如妾,妾不如偷”:此语当是批语混入正文。
}”。
}听贾母一说,便赶上来说道:“姑娘昨日受了屈了,都是我的不是。
奶奶得罪了你,也是因我而起。
我赔了不是不算外,还替你奶奶赔个不是。
”说着,也作了一个揖,引的贾母笑了,凤姐儿也笑了。
贾母又命凤姐儿来安慰他。
平儿忙走上来给凤姐儿磕头,说:“奶奶的千秋,\zhu{千秋:祝颂长寿之词,代指生日。
}我惹了奶奶生气,是我该死。
”凤姐儿正自愧悔昨日酒吃多了,不念素日之情,浮躁起来,为听了旁人的话,无故给平儿没脸。
今反见他如此,又是惭愧,又是心酸,忙一把拉起来,落下泪来。
平儿道:“我伏侍了奶奶这么几年,也没弹我一指甲。
就是昨儿打我,我也不怨奶奶,都是那淫妇治的,怨不得奶奶生气。
”说着,也滴下泪来了。
\geng{妇人女子之情毕肖,但世之大英雄羽翼偶摧,尚按剑生悲,况阿凤与平儿哉?所谓此书真是哭成的。
}贾母便命人:“将他三人送回房去。
有一个再提此事,即刻来回我,我不管是谁,拿拐棍子给他一顿。
”\ping{贾琏事后只需要赔个不是就能万事大吉,不予追究。
}\par
三个人从新给贾母、邢王二位夫人磕了头。
\zhu{从新:重新。
}老嬷嬷答应了,送他三人回去。
至房中,凤姐儿见无人,方说道:“我怎么像个阎王,又像夜叉?那淫妇咒我死,你也帮着咒。
我千日不好,也有一日好。
可怜我熬的连个淫妇也不如了,我还有什么脸来过这日子?”说着又哭了。
\geng{辖治丈夫此是首计,懦夫来看此句。
}贾琏道:“你还不足?你细想想,昨儿谁的不是多?\geng{妙!不敢自说没不是,只论多少,懦夫来看。
}今儿当着人还是我跪了一跪,又赔不是,你也争足了光了。
这会子还叨叨,难道还叫我替你跪下才罢?\zhu{替:为,给。
}太要足了强也不是好事。
”说的凤姐儿无言可对,平儿嗤的一声又笑了。
贾琏也笑道:“又好了!真真我也没法了。
”\par
正说着,只见一个媳妇来回说:“鲍二媳妇吊死了。
”贾琏凤姐儿都吃了一惊。
凤姐忙收了怯色,反喝道:“死了罢了,有什么大惊小怪的!”\geng{写阿凤如此。
}一时,只见林之孝家的进来悄回凤姐道:“鲍二媳妇吊死了,\geng{倒也有气性,只是又是情累一个,\zhu{情累:为情所累。
累:连累,牵累。
}可怜!}他娘家的亲戚要告呢。
”凤姐儿笑道:\geng{偏于此处写阿凤笑。
坏哉阿凤!}“这倒好了,我正想要打官司呢!”林之孝家的道:“我才和众人劝了他们,又威吓了一阵,又许了他几个钱,也就依了。
”凤姐儿道:“我没一个钱!有钱也不给,只管叫他告去。
也不许劝他,也不用震吓他,只管让他告去。
告不成倒问他个‘以尸讹诈’!”\geng{写阿凤如此。
}林之孝家的正在为难,见贾琏和他使眼色儿,心下明白,便出来等着。
贾琏道:“我出去瞧瞧,看是怎么样。
”凤姐儿道:“不许给他钱。
”贾琏一径出来,和林之孝来商议,着人去作好作歹,\zhu{作好作歹:做好人,又做恶人,比喻用各种方式反复劝说。
}许了二百两发送才罢。
\zhu{发送:殡葬死者。
}贾琏生恐有变,又命人去和王子腾说,将番役仵作人等叫了几名来,\zhu{番役:
明清官衙中专司缉捕的差役,也称“番子”。《红楼梦》中泛指官衙中负责稽查缉捕的差役。
仵(音“五”)作:封建官衙中专司验尸的差役。
}帮着办丧事。
那些人见了如此,纵要复辨亦不敢辨,只得忍气吞声罢了。
贾琏又命林之孝将那二百银子入在流年帐上,分别添补开销过去。
\geng{大弊小弊,无一不到。
}
\ping{
鲍二家的因与贾琏通奸事发自杀,贾琏为了息事宁人补偿鲍二却用“官中的钱”,叫林之孝这个会计做假账,即虚构开销或夸大花费,从而套出现金。
贾琏遇事立即想到这个办法,可见此非第一次,没有贾琏指令时,林之孝私下也完全可用此法贪污。
王熙凤曾总结管理中的弊病:“需用过费,滥支冒领”即为其一。
探春曾不点名地指责林之孝敛财:凡事到了账房,就得“又剥一层皮”。
而这对夫妇掌控的又何止是账房。林之孝家的擅自委任秦显家的掌管大观园厨房,后者上任后立马“悄悄的备了一篓炭,五百斤木柴,一担粳米”送去林家,同时她“又打点送帐房的礼”。
书中没有披露林之孝利用职权究竟聚敛了多少财富,但他邀请贾母吃年酒的细节可供推测林之孝有宽敞气派的府第与充盈的实力。贾母说那些管家“都是财主”,其言不虚。
林之孝的富有毋庸置疑。可是尽管富有,林之孝夫妇仍是“世代的旧仆”,女儿小红也是须当差服役的奴隶。
而赖大儿子被开恩脱籍,对他们也是莫大刺激。当荣府经济颓势越来越明显时,林之孝终于按捺不住建议:“把这些出过力的老家人用不着的,开恩放几家出去。”
他们不再享受府内“官中”待遇,荣府“一年也省些口粮月钱”,而脱籍者本来就“各有营运”,有人身自由后可更好地发展。
}
又梯己给鲍二些银两,\zhu{梯己:这里是私下的意思。
亦作“体己”。
}安慰他说:“另日再挑个好媳妇给你。
”鲍二又有体面,又有银子,有何不依,便仍然奉承贾琏,\geng{为天下夫妻一哭。
}
不在话下。
\par
里面凤姐心中虽不安,面上只管佯不理论,因房中无人,便拉平儿笑道:“我昨儿灌丧了酒了,你别愤怨,打了那里,让我瞧瞧。
”平儿道:“也没打重。
”只听得说,奶奶姑娘都进来了。
要知端的,下回分解。
\par
\qi{总评:富贵少年多好色,那如宝玉会风流。
阎王夜叉谁曾说,死到临头身不由。
\zhu{身不由:身不由己。
由:顺从,听任。
“阎王夜叉”是鲍二家的与贾琏偷情时的对话,而这回结尾时鲍二家的上吊自杀了,所以说“死到临头”。
}\ping{一场闹剧,碎了的瓶子再粘起来,总不是原来的瓶子了,一场吵闹下来,凤姐面临的局面也更差了,但是错在她吗?}}
\dai{087}{凤姐捉奸互殴}
\dai{088}{喜出望外平儿理妆}
\sun{p44-1}{丫鬟望风凤姐捉奸,喜出望外平儿理妆}{图右侧:生日宴上凤姐不胜酒力, 由平儿扶其回家歇息。
才至穿廊下,见一丫鬟神色可疑,遂问原由,得知贾琏正与鲍二媳妇在屋里鬼混。
图左上:宝玉让平儿来到怡红院,拿出上好胭脂,劝其理妆,温存慰藉。
平儿素闻宝玉专爱与女儿交往,经此方知果非虚传。
}