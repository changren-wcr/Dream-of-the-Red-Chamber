\section*{诸芳流散}

1、大观园后角门(秋)\par
疏林斜晖。茗烟下马,气喘吁吁地敲门。\par
字幕(叠)\par
笫二十一集:诸芳流散\par
守园婆子闻声开门。茗烟:“嬷嬷好!烦嬷嬷走一趟,有要紧事立等着回宝二爷。”\par

2、潇湘馆院内\par
宝玉进门。宝玉:“林妹妹,林妹妹——“架上鹦鹉:“紫鹃!快打帘子,宝玉来了!”
紫鹃急步从房内走出:“哦,宝二爷?”
宝玉边进门边笑嘻嘻地:“林妹妹呢?”
紫鹃:“林姑娘送宝姑娘去了。二爷怎么没去?”
宝玉一愣:“宝姑娘要去哪儿?”紫鹃:“宝姑娘要搬回姨太太那边儿住,林姑娘听说,掉了半天儿眼泪,
没吃饭就到蘅芜苑送她去了。”宝玉惊疑地看着紫鹃,愣了一会儿,“嗐”了一声,转身离去。\par

3、蘅芜苑\par
宝玉急步走进院门,院内阒无人声。宝玉:“宝姐姐!宝姐姐!林妹妹!”
宝玉走进宝钗居室。一个老婆子自内室走出,手里拿着扫帚。宝玉急切地询问:“老嬷嬷,宝姐姐呢?”
老婆子:“宝姑娘出去了。这里交给我们看着,还没有搬清楚。
我们帮着送了些东西去,这也就完了。”
宝玉茫然四顾,房内已搬得空空落落。
老婆子:“宝二爷出去吧,让我们扫扫灰尘。
也好,从此你老人家省跑这一处的腿了。”
宝工怔怔地看着老婆子,仿佛什么也没听见。老婆子皱了皱眉,转身走入内室。
宝玉默默走出房门。院子里的青藤异蔓在微风中轻轻摆动着,透出一片清冷的苍翠。\par

4、园中曲径\par
袭人穿花度柳急步走来。\par

5、蘅芜苑院外翠樾埭\par
秋风萧瑟,黄叶纷飞。埭下流水溶溶逝去。宝玉独立残阳,怅然若失。\par
(闪回)少女们无拘无束的嬉笑声,翠樾埭上来往丫鬟络绎不绝。\par
一阵风过,卷起满地黄叶,夹着许多不知名的花花草草落入水中,随水飘转流去。袭人走来,在宝玉背后站住。
袭人:“二爷!”宝玉的目光追随着水中远去的落叶。袭人:“二爷!”
“嗯?”宝玉收回迷惘的目光,转过头来。“后角门上的嬷嬷来,说茗烟在外头等着回二爷话。”
“哦?”宝玉眼睛一亮,急忙转身掣步离去。\par

6、大观园后角门\par
袭人从微开着的门缝里向外窥看。门外,茗烟起劲地讲述着什么,宝玉时时顿足叹气。
宝玉嘱咐了茗烟几句,茗烟点头,上马而去。宝玉匆匆走入园门,袭人随后赶来。
袭人:“二爷!二爷!”宝玉停步回头。袭人吃惊地看着宝玉含着泪水的双眼,“怎么……”宝玉:“司棋……”

7、司棋家\par
司棋伏案痛哭。司棋母双手叉着腰站在司棋背后,满脸怒气,“没有廉耻的东西!你还有脸回来!”

8、司棋家(夜晚)\par
司棋斜靠在床上抽泣。一盏孤灯摇晃着她的身影。
司棋母:“仗着你姥娘在府里的脸面,带挈着你成日里主子一样的吃喝穿戴!
指望你伺候好主子,你姥娘脸上、我脸上都有光!谁知道你干出这种事来!”

9、司棋家(白天)\par
凄风苦雨。司棋凭窗呆坐。司棋母:“我要是你呀,我就一头碰死!”司棋的脸上毫无表情。
司棋母一摔门帘,走入内室。户外秋雨凄迷。风雨声中,有人轻轻地叩击窗棂。司棋:“谁?”
“我。”仿佛是潘又安的声音。“你还来干什么?”司棋眼圈儿一红,“你也算是个男人!出了事儿抬腿先跑了!你……你走吧!”
外面没人搭话。司棋起身倾听。司棋“唿”地一下站起来,朝着窗外:“……你别走!”窗外只有沙沙雨声。
司棋急忙开门。雨地里,潘又安失魂落魄地站着,浑身上下已经湿透了。
司棋无力地靠在门上,哽咽着“……你走吧……”潘又安失声痛哭,“表姐!”接着“扑通”一下双膝跪倒在泥水之中。
“哎呀!你……”司棋冲进雨里,把潘又安一把拉起来。“司棋!”司棋母站在门口一声怒喝。
潘又安一惊,刚站起身来,又“扑通”一声跪下:“舅妈……”
“谁是你舅妈!”司棋母气急败坏地骂道:“你这个猪狗不食的东西!你把司棋害得人不是人,鬼不是鬼的……
我,我打死你这个不要脸的东西!”骂着转身抄起一根门闩,冲出门来。
“妈!”司棋死命抱住门闩,哭道:“我是为他出来的,我也恨他没良心。
如今他来了,妈要打他,不如先打死我!”
司棋母气得浑身乱抖:“不害臊的东西,你还要怎么样?”
司祺:“我活着是他的人,死了是他的鬼。他一辈子不来,
我也一辈子不嫁人!我恨他为什么这样胆小,一身作事一身当,为什么要逃!……
妈说要把我配给别人,我原拼着一死的。
今儿他来了,求妈问问他,若是他不改心,我在妈跟前磕了头,
妈只当是我死了,他到哪里,我也跟到哪里,就是讨饭吃,我也心甘情愿!”
“什么?你说什么?”司棋母颤抖着:“你……你死了这份儿心吧!”
司棋:“妈!”
司棋母:“想跟他走?除非先把我勒死!”
司棋眼睛睁得圆圆的,死死盯住母亲由于恼怒而变了形的脸。
潘又安失神地跪着。雨水和着泪水从司棋的脸上流下来。司棋喃喃地:“妈不可怜可怜女儿?”
司棋母:“你就是死了我也不可怜你!”
司棋:“妈不后悔?”
司棋母:“呸!”
司棋站起身来,双手慢慢松开紧抱着的门闩,向后退了两步,转过头去,恨恨地看着潘又安:“没用的男人!”
潘又安突然睁大的双眼,绝望的目光。司棋母惊恐的脸。撕裂人心的喊声:
“表姐——”
“司棋——”
司棋倒在窗下,雨水冲刷着溅在墙上的鲜血。
无边的秋雨,把天地连成灰蒙蒙的一片。

10、紫菱洲(晚)\par
一支蜡烛流着大滴的烛泪。
迎春坐在几旁垂首啜泣。
宝玉伫立窗前,哽咽着:“……又一个清清白白的女孩……”
外间,几个丫鬟、婆子正在忙忙碌碌地收拾箱笼、打叠衣物。
王善保家的坐在通迎春内室的门侧,斜倚着槅扇,老泪纵横。
里间,迎春慢慢抬起头来:“……司棋临出去的时候,还央告过我,好歹打听她要是受罪,替她说个情,可谁知道……”说着,又掩面抽泣起来。
门外传报:“三姑娘来了!”
外间探春带着侍书进门。
王善保家的急忙站起来,揉着眼睛避在一边。
探春斜了王善保家的一眼,径直走进内室。
迎春、宝玉迎上。
探春笑嘻嘻地:“才听说二姐姐要搬回大太太那边儿住,我去找二哥哥来看看,谁知道二哥哥先来了。”
宝玉含泪不语。
探春:“二姐姐,不是说孙家秋后才来娶亲么?怎么这会子就搬出园子?”
迎春含泪不语。探春诧异地看看迎春,又看着宝玉,渐渐收了笑容。

11、紫菱洲院外(晨)\par
一把大铜锁挂在院门上。阒无人声的紫菱洲,轩窗寂寞,屏帐翛然。水中残荷在晨风中摆动。
香菱匆匆走来,发现独自伫立在水边的宝玉。
香菱悄悄走到宝玉背后,莞尔一笑:“又发什么呆呢?”
宝玉回头,又惊又喜:“哎呀,是香菱姐姐!这么多日子你也不进园来逛逛,这会子怎么有空来了?”
香菱:“我且先问你:袭人姐姐可好?怎么忽然把个晴雯姐姐没了,到底是什么病,听说司棋……”
宝玉眼圈一红。
香菱一眼瞥见,连忙咽下没说完的话。
宝玉:“哦……香菱姐姐,走,到怡红院去坐一会儿。”香菱:“这会子不行,我得赶紧找琏二奶奶,听说她一早
进园子来了。”
宝玉:“找她什么事?”香菱:“正经事——你薛大哥哥要娶嫂子了!”
宝玉:“哦?这事儿吵吵了有半年了,今儿说张家的好,明儿又要李家的,后儿又议论王家的……。
这些人家的女儿也不知道遭什么罪了,好端端的叫人家议论。”
香菱微微一笑:“这回定了。”宝玉:“定了谁家的?”
香菱:“和我们家一样,是在户部排名的行商。前儿说起来,你们两府也都知道。京城里上至王侯,下至买卖人,都称她家是‘桂花夏家。’”
宝玉不解:“怎么叫个‘桂花夏家’?”
香菱:“她家姓夏,种着几十顷地的桂花,这京城里外的桂花局都是她家的,连宫里一应陈设盆景也是她家贡奉,所以得了这个浑号。”
宝玉:“别的不管,这姑娘可好?”
香菱:“都说长得又俊,又通文墨。我真巴不得这会子就娶过来,咱们又添一个作诗的人了。”
宝玉冷笑一声:“说是这么说,可是……”
香菱:“怎么?”宝玉:“我倒是替你耽心呢!”
香菱脸一红,正色道:“这是什么话!素日咱们都是厮抬厮敬的,你说这些是什么意思?……怪不得人人都说你是个亲近不得的人!”
香菱转身离去。宝玉呆呆地站着。

12、潇湘馆\par
书案上铺着一张雪浪笺,上面是刚写上去的两行字:

茜纱窗下,我本无缘;黄土垄中,卿何薄命。

黛玉左手托腮,右手握笔,愣愣地看着雪浪笺出神。紫鹃急步进门:“姑娘,姑娘!快过去看看吧,宝二爷病了!”
“什么?”
“宝玉病了,折腾了一夜,这会子正说胡话呢!”
“啪”地一下,笔落在雪浪笺上,湮上了一个大大的墨点。

13、怡红院\par
宝玉昏睡,张太医坐在床前把脉,王夫人含泪站在一旁。
张太医出院门,拱手辞去。凤姐、鸳鸯搀着贾母进门。煎着草药的砂罐“扑扑”冒着热气。
张太医在灯下处方。张太医拱手辞去。一包草药抖入药罐。
袭人替宝玉擦去唇边的药汁。宝玉斜靠在榻上。宝玉坐在圈椅上。
宝玉立在窗前,凝视着枯萎的“女儿棠”。
(闪回)“女儿棠”丝垂翠缕、葩吐丹砂。(闪回)“女儿棠”枯死半边。
(幻觉)院子里的芭蕉枯死了。(幻觉)院子里所有的花草都枯死了。
袭人端着药碗走到宝玉背后,“二爷,该吃药了。”
宝玉叹了一口气,“早晚都是要去的……”袭人:“二爷!”
宝玉接过药碗,“我生病,林妹妹她们都来看过我,宝姐姐怎么不来?”
袭人:“想是忙得没有空闲,先是薛大爷娶亲,不几天又是家里闹气。”
宝玉:“闹什么气?”
袭人:“听说新娶的薛大奶奶厉害得不得了,竟然把薛大爷给挟制住了。后来又要占姨太太的上风,亏着宝姑娘随机应变,总是用道理暗暗弹压着,不然,真不知道要闹成个什么样了。“
宝玉担忧地:“哎呀,香菱……”

14、夏金桂房中\par
香菱用托盘托着茶盅进门。夏金桂端坐,上下打量着香菱。
在一旁侍立的婢女宝蟾接过茶盅,递给金桂。香菱欲退出。
金桂:“香菱,别走,坐下我问你话。”香菱笑嘻嘻地:“奶奶要问什么?”
金桂:“你坐下。”香菱斜欠着坐在金桂对面。
金桂:“你家乡是哪儿?”香菱摇摇头:“不记得了。”金桂:“父母在哪儿?”
香菱摇摇头:“不记得了。”
金桂不悦。金桂:“你这个名字是谁给起的?”
香菱:“宝姑娘起的。”金桂冷笑道:“人人都说姑娘通,只这一个名字起得就不通。”
香菱:“嗳哟,奶奶不知道,我们姑娘的学问连我们姨老爷还时常夸呢!”
金桂将脖颈一扭,嘴唇一撇,鼻孔里“哼”了两声,拍着掌冷笑着:“菱角花谁闻见香来着?若说菱角香了,正经那些香花放在哪里?可是不通之极!”
香菱:“不独菱角花,就连荷叶、莲蓬,都是有一股清香的。但那原不是花香可比,若是静日静夜或清早时候,那一股香比花儿都好闻呢!就连菱角、鸡头、苇叶、芦根,得了风露,那一股清香,就令人心神爽快的。”
金桂:“依你说,那兰花桂花倒香得不好了?”
香菱:“兰花桂花的香,又非别花之香可比……”宝蟾没等香菱说完,便忙指着她的脸儿说:“要死!要死!你怎么直叫起奶奶的名字来了?”
香菱一惊,连忙陪笑:“一时说顺了嘴,奶奶别计较。”金桂:“这有什么?你也太小心了。只是我想这个‘香’字到底不妥,换一个字,不知你服不服?”
香菱忙着陪笑:“奶奶说哪里话,此刻连我一身一体俱属奶奶,换一个名字反倒问我服不服,叫我怎么当得起!……奶奶说哪一个字好,就用哪一个。”
金桂微微一笑:“你虽说得是,只怕宝姑娘多心。”
香菱:“奶奶不知逍,当日买我来的时候,原是太太使唤的,所以姑娘起得名字,后来我服侍了爷,就和姑娘无涉了。
如今又有了奶奶,益发不与姑娘相干。况且姑娘又是极明白的人,怎么会恼这些呢?”
金桂:“既这样说,‘香’字不如‘秋’字妥当。菱角菱花皆盛于秋,岂不比‘香’字有来历些?”
香菱:“就依奶奶这样吧。”

15、后院宝钗房中\par
宝钗淡淡一笑:“那……我也叫你秋菱吧。”
香菱笑嘻嘻地点点头。 

16、薛家前院内(黄昏)\par
薛蟠送宝玉出门。宝玉拱手:“薛大哥留步。”
薛蟠微醮:“你这个嫂子怎么样?”宝玉:“……”香菱走过。
宝玉:“香菱姐姐!”香菱:“宝二爷,我如今叫秋菱了。”
薛蟠瞪眼:“名字也是乱叫的吗?谁改的?”夏金桂站在门口台阶上:“我。”
薛蟠一愣。金桂逼视着薛蟠。
“其实……这个这个名字……”薛蟠看着宝玉:“……秋菱……秋菱……改得还不错!啊?”
金桂微微-笑。宝玉惆怅地看着香菱。 

17、金桂房中(晚)\par
薛蟠闭目养神,不时打着酒嗝。宝蟾自内室走出。薛蟠的眼睛张开一条缝,盯着宝蟾。宝蟾轻佻顾盼的眼神儿。 薛蟠的目光追随着宝蟾。宝蟾白腻的脖子。薛蟠:“宝蟾!”宝蟾:“嗯?”
薛蟠:“茶!”夏金桂一面用蜡剪儿拨着烛花,一面斜睨着薛蟠,嘴角微微撇了一撇。
宝蟾双手捧着茶碗递给薛蟠。薛蟠的目光捕捉着宝蟾假意躲闪的眼神儿。金桂冷眼旁观。
薛蟠接茶碗时,故意捏了一下宝蟾的手,宝蟾乔装躲闪,连忙缩手。
两下失误,“豁啷”一声,茶碗落地,泼了一身一地的茶。
“哎呀!”宝蟾急忙掩饰:“姑爷不好生接!”
“你自己不好生拿,怎么反倒怪我?”薛蟠佯作嗔态。金桂冷笑:“两个人的腔调都够使了,别打谅谁是傻子。”
薛蟠低头微笑不语。宝蟾红着脸收拾茶碗出去。薛蟠伸了伸懒腰:“困了。”
金桂:“困了别处睡去,省得你馋痨饿眼的!”薛蟠看着金桂但笑不语。
金桂:“要做什么和我说,别偷偷摸摸地不中用!”薛蟠“扑通”跪在金桂面前,拉着金桂的手,嘻皮笑脸地:
“好姐姐,你要把宝蟾赏了我,你要怎么样就怎么样!你要人脑子也给你弄来!”
金桂把手一摔:“这话好不通。你爱谁,说明了,就收在房里,省得别人看着不雅!我可要什么呢?”
薛蟠又惊又喜,直直地跪着:“哎哟我的亲姐姐!这让我怎么谢你呢?”金桂冷笑。

18、薛家前院内(午后)\par
金桂高声叫着:“小舍儿,小舍儿!”丫头小舍儿自房内跑出:“奶奶叫我?”
金桂:“你跟着我去园子里逛逛去。”宝蟾急步走来:“我伺侯着去吧?”
金桂:“不用你了,你给我预备着酽茶,我去一个时辰就回来!”

19、金桂房内\par
薛蟠侧耳听着金桂在院内说话,面露喜色。

20、薛家前院内\par
香菱进门:“奶奶出去?”
金桂:“我比不得你,在园子里入过什么诗社,左不过自己去逛逛罢了。”
香菱一愣。金桂扶着小舍儿出门。
香菱怏怏走回自己房内。

21、金桂房内\par
薛蟠笑嘻嘻地看着宝蟾。宝蟾乜斜着眼看着薛蟠。
薛蟠:“把茶给我端过来。”宝蟾送过茶碗:“这回可好生接着!”
薛蟠一只手抓住宝蟾的腕子,另一只手接过茶,放在桌子上。
宝蟾:“姑爷放尊重些!看有人来了。”
薛蟠:“我这可是过了明路的。” 

22、藕香榭\par
金桂停步:“小舍儿,你回去一趟,告诉秋菱,到我屋里把手帕拿来。”
小舍儿:“哎。”“回来!”金桂略一沉吟:“不必说是我让拿的。”
小舍儿:“知道了。”金桂看着小舍儿远去的背影冷笑。水中残荷在西风中颤抖。

23 薛家前院门外\par
“菱姑娘!”小舍儿边跑边喊。
香菱刚要进门,听见喊声忙停步回头。
“菱姑娘,奶奶的手帕子忘在屋里了,你给拿出来好吗?”
香菱笑笑:“好,我去取。”

24、金桂房内\par
喘息声。薛蟠一只手搂着宝蟾,一只手去解宝蟾的中衣。宝蟒钗鬓蓬松,半推半就地:“门没掩上……”薛蟠:“不妨事。”
香菱推门而入。“哎呀!”宝蟾急忙挣开。“啊?”香菱羞得耳面飞红,忙转身回避。
薛蟠抬头:“嗯?”宝蟾一把推开薛蟠:“人家不愿意,姑爷硬逼着要……,没见过这样不顾脸面的爷!”
宝蟾恨恨地一跺脚,一径跑了。薛蟠一把没拉住:“哎哎,回来!”
香菱急忙跑出门外。薛蟠怒冲冲地指着香菱:“呸!小娼妇!你这会子干什么来撞尸游魂!”
香菱三步两步跑出院门。薛蟠:“宝蟾!宝蟾!”宝蟾无影无踪。
薛蟠:“香菱!秋菱!死娼妇臭婊子!你跑?回头咱们有帐算!”
金桂进门,笑嘻嘻地:“骂谁呢?这么大的火气?”
薛蟠:“哦……”

25、香菱房中(晚)\par
香菱捧着一本书出神。灯花爆了一下。香菱叹了口气,合上书,封面上是《断肠集》三个大字。
薛蟠醉醺醺地进门:“秋菱!打洗脚水!”香菱一惊:“就来!”
薛蟠“啪”地一拍桌子:“快点儿!”

26、金桂房中\par
宝蟾忙着给金桂铺床。
金桂:“放着吧,待会儿让秋菱过来收拾。”宝蟾疑问地看着金桂。
“今晚你去秋菱房里睡,让秋菱过来。”
宝蟾不解地:“怎么?”金桂微微一笑:“我成全你,让你和爷在秋菱房里成亲。”
宝蟾慌乱地:“哎呀,奶奶别说这种玩笑话,我们可当不起。”
金桂笑嘻嘻地:“别装你娘的蒜了,你只要日后有点儿良心,别老惦着合伙谋害我就行了。”
宝蟾“哧”地一笑,连忙用手捂住脸。

27、香菱房中\par
香菱端盆进门,放在薛蟠面前。薛蟠抬起脚。香菱战战兢兢地一面脱靴,一面偷觑着薛蟠的脸色。
薛蟠先脱好的一只脚踩进盆内:“哎哟!你想烫死我!”香菱急忙把手插进水里试了试,嗫嚅着:
“爷再试试,水不烫……”
“放屁!”薛蟠“啪”地一脚把水盆踢翻,盆里的水溅了香菱一头一脸。
香菱:“哎呀!”“黑了心的娼妇!”薛蟠一掌打过去:“我叫你害我!”香菱哭着跑出门外。薛蟠赤着一只脚追出来。

28、后院\par
薛姨妈急忙自房内走出。前院传来薛蟠的喝骂声和香菱的哭叫声。薛姨妈焦虑地朝前院张望。

29、前院\par
金桂笑嘻嘻地拦住薛蟠:“算了算了。”香菱双手捂着脸抽泣。薛蟠冲着香菱啐了一口,转身走回房内。
金桂:“秋菱,今晚你和宝蟾换换。”香菱含泪看着金桂。
金桂冷冷地:“让宝蟾伺侯你那个没见过世面的主子。”香菱吃惊地:“啊?”
金桂转身回房:“这会子就过来陪我先睡吧。”香菱犹豫地:“我……“
金桂脸一沉:“怎么?你还怕主子脏了你的屋子不成?”香菱急忙摇头:“哦,不……”
“再不就是怕夜里劳动你服侍我?”香菱:“不不……”
金桂倏地转身骂道:“你那见一个爱一个的主子呢?把我的人霸占了去,又不叫你来,到底是打的什么主意?”
薛蟠急步自房内走出,喝骂香菱:不识抬举的东西,快卷铺盖过去!”
香菱捂着脸跑回房内。

30、后院\par
宝钗走到薛姨妈身后:“妈……“薛姨妈:“唉……”

31、前院·金桂房内\par

香菱抱着铺盖不知所措地站着。金桂:“铺在地下。”香菱的嘴巴张了张,泪水夺眶而出。

32、香菱房内\par

烛影摇红。几上放着书,“断肠集”三个字忽明忽暗。宝蟾“咯咯”的笑声。
一只多毛的大手伸过来掰断蜡烛。烛泪滚滚。烛火被捻灭在《断肠集》上。

33、金桂房内\par
金桂躺在床上,两眼圆睁。
(心声)“且叫你们乐几天,等我慢慢摆布了秋菱再说!那时可别怨我!”
金桂厉声:“秋菱!”“哎!”香菱一骨碌从地上爬起来。
“倒碗茶来!”
“哎……”

34、夜空\par
云破月出,万籁俱寂。

35、金柱房内\par
月光透过窗棂,照着无寐的香菱。香菱眼角溢出一大滴泪水。
“秋菱!”金桂突然喊了一声。香菱急忙擦擦眼泪,从地上起来。
金桂:“给我捶捶腿!”香菱:“哎……”

36、夜空\par
月落乌啼。

37、金桂房内\par
香菱在窗前伫立。
金桂:“秋菱!”香菱一惊:“哦……”


38、贾赦院·迎春房内(晨)\par
窗外鼓乐盈耳。王善保家的督率着一群丫头、婆子们进进出出地忙碌着。
盛妆的迎春端端正正地坐在奁台前。绣桔、莲花立在两侧,仔细地往迎春的发髻上插戴着各种饰物。
凤姐、平儿含笑端详着迎春。门外传报:“三姑娘、四姑娘来了!”
探春、惜春联袂进门。王善保家的急忙避在一边。探春笑嘻嘻地朝菱花镜里瞥了一眼。
镜子里的迎春神色木然。门外传报:“薛姑娘、林姑娘来了!”
凤姐:“宝玉呢?怎么这会子还不来?”探春:“一大清早就不知道去哪儿了。”

39、水月庵高墙外\par
骑在马上的宝玉紧紧地拉着另一匹马的缰绳。茗烟欠着脚蹬在马背上,两手抠着墙脊,探头朝墙内张望着。
木鱼声、唪经声隐隐传来。宝玉仰着脑袋,急切地:“看见了吗?”
茗烟摇摇头。宝玉轻轻叹了口气。
茗烟把握十足地:“快了,上两回都是这会子出来的!”茗烟突然紧张地:”来了!”

40、水月庵院内\par
飒飒凉风拂动着满地落叶。尼姑装束的芳官拖了一把大扫帚从内院走出。茗烟从高墙后面探着头,压低了声音:“芳官!芳官!”芳官仿佛浑然不觉,略有些吃力地握着扫帚,缓缓地打扫着空旷的院落。 

41、高墙外\par
茗烟回过头来叹了口气,“还是不答理,上两回就这么着。”
宝玉:“你说我来看她了。”

42、院内\par
茗烟:“芳官!宝二爷来了,来看你,……就在墙外边儿呢!”
芳官微微一震,依旧缓缓地扫着落叶。

43、高墙外\par
宝玉对着高墙,动情地:“芳官,你在这儿……过得惯吗?有人难为你吗?”

44、院内\par
扫帚在一块已经扫过的地方机械地来回划动着……

45、高墙外\par
宝玉:“……你千万将就捱些日子,等太太气消了,我去求老太太,再接你回去……”

46、院内\par
芳官咬住嘴唇,泪水涌满了眼眶。

47、高墙外\par
宝玉哽咽着:“……我知道你冤屈,是我……是我带累的你……”

48、院内\par
芳官“呜”地一声,抱住扫帚哭了出来。茗烟含泪看着芳官。老尼净虚带着智通从内院走出,
抬头看见高墙外的茗烟,吃了一惊:“阿弥陀佛!是谁?”茗烟急忙缩回脑袋。

49、高墙外\par
茗烟“砰”地一屁股跨在马上:“坏了!那个老秃歪剌看见我了!”
宝玉两腿一夹坐骑:“快走!”

50、荣宁街口\par
宝玉、茗烟一前一后打马而来。薛蟠带肴四、五个家人立马街口,迎面高声招呼着:“宝兄弟!”
宝玉、茗烟忙带住马。薛蟠:“等了你这半天!”
宝玉疑问地看着薛蟠。薛蟠紧张而又兴奋地:“听说了么?朝廷要出兵了!”宝玉:“出兵?”
薛蟠:“南安王爷奉旨征讨西海沿子,说话就要出京!听说今儿在北校场操演,走,看热闹去!”
宝玉摇摇头笑笑:“今儿二姐姐出阁,我得去送送她。”薛蟠失望地看着宝玉:“嗐!早知道……,等了你这么半天!
那好,我找冯大哥他们去!”说着朝家人们摆摆手,打了个唿哨,纵马而去。
宝玉回过头嘱咐茗烟:“你再去打听打听四儿怎么着了……我去那院儿送二姐姐。”
茗烟答应一声,拨转马头驰去。

51、贾赦院·迎春房内\par
房门洞开,几只“囍”字彩灯在微风中晃动,给空荡荡的房内更增添了几分寥落、清冷。
宝玉呆呆地站在奁台前。

52、潇湘馆·黛玉房内(晚)\par
黛玉一面用蜡剪儿拨着烛花儿,一面朝着坐在对面的宝玉:“……二姐姐一直等着你,孙家的轿催了几回才走了。”
宝玉怔怔地坐着。黛玉:“绣桔、莲花、还有两个小丫头,一共陪过去四个……”
宝玉长叹一口气:“从今后这世上又少了五个清洁的人了……”

53、薛家后院(晨)\par
小舍儿神色慌张地奔入院内:“太太!太太!可不得了啦!”
薛姨妈急步自房内走出:“又怎么了?又怎么了?”
小舍儿:“出事儿了!太太快着请过去看看吧!”

54、前院\par
金桂房中传出嘈杂的人声。薛姨妈带着同喜、同贵,跟在小舍儿后面走进院子。小舍儿:“太太来了!太太来了!”
薛姨妈进金桂房门。

55、金桂房内\par
金桂平躺在床上呻吟,几个丫头、婆子惶惶不安地站着。
薛蟠指着丫头、婆子们:“都不许动!”说着抢上一步,把手中的一个纸人儿递给薛姨妈:“妈看看这个!”
薛姨妈惊异地看着。五根大针钉在纸人的心窝并四肢骨节处。
小舍儿:“上面还写着奶奶的生辰八字呢!怪不得这两天直叫着心口疼!”
薛姨妈的手哆嗦着:“这是……?”
小舍儿:“我今儿早上拾掇床,从奶奶的枕头里面抖出来的。”
薛蟠铁青着脸:“先把这几个捆起来拷问!”丫头、婆子们惊慌不安,面面厮觑。
薛姨妈:“慢着!”金桂有气无力地:“何必冤枉众人,一定是宝蟾弄的镇魇法儿。”
薛蟠:“宝蟾这些日子并没有多少空儿在你房里,怎么会是她?”
金桂:“除了她还有谁?总不是我自己吧?”
薛蟠指着丫头、婆子们:“还有这些人呢!”
“这些人?”金桂冷笑一声:“可她们哪一个敢进我的房呢?”
薛蟠:“谁敢进你的房?”薛姨妈冷眼旁观。
“秋菱!”薛蟠突然解悟:“对!秋菱眼下天天跟着你,她自然知道!先拷问她!”
金桂:“拷问谁?谁肯认?依我说,干脆装着不知道,大家丢开手算了。”
薛蟠:“那不行!”金桂:“行不行的吧,横竖治死我再娶好的!”说着哭起来。
薛蟠:“你!……嗐!”金桂痛哭:“要是凭良心上说,……左不过你们三个多嫌我一个!”
薛姨妈暗暗吃惊,看看金桂,又看看薛蟠。丫头、婆子们偷觑着薛蟠。
薛蟠咬着牙,脖子上的青筋暴了起来。薛蟠“啪”地一脚踢开门,大吼一声:“秋菱!”

56、香菱房内\par
香菱靠门站着,打着哆嗦,脑门上沁出冷汗。薛蟠暴怒的声音:“秋菱!”
“哎……”香菱颤声答应着,急忙开门走出。

57、金桂房内\par
薛蟠伸手抄过一根门闩冲出门去。
众人不约而同地:“哎呀!”

58、院内\par
薛蟠:“你干的好事!“恶狠狠地一棒打来。香菱急忙躲闪:“我没……”
薛蟠不容分说,劈头劈面打起来。“哎哟!”香菱被一棒打倒。
薛蟠上去一脚踩住,骂着:“下作小娼妇!……”香萎哭着分辩:“爷听我说,不是……”
“呸!我让你害人!”薛蟠举起门闩。“啊——”香菱看着门闩绝望地惨叫。
“住手!”薛姨妈怒声呵斥薛蟠。薛蟠一愣。
薛姨妈:“同喜!同贵!把菱丫头扶起来!”同喜、同贵:“哎!”
香菱呜呜地哭着。薛姨妈:“问明白了没有?你就打人!这丫头服侍了你这几年,哪一点儿不周到、不尽心?
这种没良心的事她做过一次吗?”薛蟠余怒未消地举着门闩。
薛姨妈:“就是动粗卤,也得问个青红皂白!……还不把门闩放下!”

59、金桂房内\par
金桂探起身子望着门外。

60、院内\par
薛蟠举着门闩的手慢慢垂下来。

61、金桂房内\par
金桂捶着床,嚎啕大哭。
小舍儿并两个婆子急忙走进来。
金桂:“这半个多月把我的宝蟾霸占了去,让秋菱跟着我睡,我说要拷问宝蟾,你百般护着,这会子又赌气打秋菱!”

62、院内\par
薛姨妈越听越气恼,狠狠地瞪着薛蟠。
金桂的嚎啕声:“打她干什么?索性合伙治死我,再拣富贵的、标致的娶来就是了,做出这些把戏来给谁看!”
薛蟠又急又恼,恨恨地用门闩捣着地。薛姨妈听着,气得浑身发抖,转身喝骂薛蟠:“不争气的孽障!
骚狗也比你体面些,谁知道你把陪房丫头也摸上手了!叫老婆说嘴,霸占了丫头,有什么脸出去见人!……动不动就打人!
我知道你是个得新弃旧的东西,白辜负了我当日的心!……香菱既然不好,你也不许打,我即刻叫人牙子来卖了她,
你就心净了!”香菱嘤嘤抽泣。
薛姨妈:“香菱!快收拾了东西跟我来!同喜!”同喜:“哎!”
“打发个人快去叫个人牙子来,多少卖几两银子,拔去眼中钉、肉中刺,大家过太平日子!”
薛蟠低下头去。金桂隔着窗子往外哭道:“你老人家只管卖人,不必说着一个扯着一个的!
我们是那吃醋拈酸容不下人的人吗?怎么叫‘拔去眼中钉、肉中刺’?是谁的钉?谁的刺?
但凡多嫌着她,也不肯把我的丫头也收在房里了!”
薛姨妈气得身战气噎:“这是谁家的规矩?婆婆在这里说话,媳妇隔着窗子拌嘴。亏你是旧家人家的女儿!
满嘴里大呼小喊,说的是些什么!”
薛蝠急得直跺脚:“罢哟,罢哟!看人听见笑话!”宝钗急步走进院门。
金桂发起泼来:“我不怕人笑话!你的小老婆治我害我,我倒怕人笑话了!
再不然,留下她,就卖了我!谁还不知道你薛家有钱,行动拿钱垫人,又有好亲戚挟制着别人!你不趁早还等什么?”
薛蟠急忙走进房去。

63、金桂房内\par
金桂滚在地下。薛蟠央求着:“别说了好不好?我的亲奶奶!”
金桂哭喊着:“我非说不可!嫌我不好,谁叫你们当初瞎了眼,三求四告地跑到我们家去!这会子人也来了,
金的银的也赔了,略有个眼睛鼻子的也霸占去了,该挤兑我了!”

64、院内\par
薛姨妈哆嗦着说不出话来。
宝钗上来扶着:“妈,别理她,快回去吧。”

65、金桂房内\par
金桂一面哭喊,一面滚揉拍打。
薛蟠“嗐”了一声,一屁股坐在门槛上,双手紧紧地抱着头。

66、后院内\par
薛姨妈进门,流着泪:“把人牙子找来!卖香菱!卖了心净!”
宝钗陪笑着:“妈!咱们家从来只知道买人,并不知道卖人。妈可是气得糊涂了,要是叫人听见不笑话吗?
哥哥嫂子嫌她不好,留下给我,我正愁没人使唤呢。”
薛姨妈:“留着她还是惹气,不如打发了倒干净。”香菱哭着追进院门,“扑通”跪在薛姨妈面前。
宝钗:“她跟着我也一样,横竖不叫她到前院儿去,跟卖了一样。”
香菱:“求求太太,千万别卖我!我情愿跟着姑娘,给姑娘当个粗使丫头。”
宝钗:“妈……”薛姨妈擦擦眼睛:“唉……”

67、宝钗房内(晚)\par
香菱躺在床上,双目紧闭,呼吸急促。宝钗坐在床前的凳子上。莺儿忙着拧了一个手巾把儿,把香菱前额上烤得半干的手
巾换了下来。
香菱的嘴唇翕翕地动着。宝钗、莺儿急忙把身体俯过来。香菱:“……爹……娘……”一大滴泪水挂在香菱的眼角上。

68、城门\par
清晨,一辆马车出城。李贵、茗烟骑着马跟在车后。马蹄声、车轮声重复着单调的节奏。车里坐着宝玉和两个老嬷嬷。

69、城外\par
宝玉出神地从车窗里往外看。广袤的田野上已经没有了一点儿绿色。

70、天齐庙\par
李贵扶着宝玉下车。宝玉抬眼望去。一座极其宏壮的庙宇,由于年深岁久,又极其荒凉,
斑驳的匾额上是同样斑驳的三个大字:大齐庙。
庙外挂着一块大招牌,画着各色丸散膏丹,“一贴病除”四个醒目的黑字赫入眼帘。

71、正殿\par
宝玉步入正殿。阴森森的殿堂里塑着一个个狰狞凶恶的神像,仿佛朝着宝玉不断地逼过来。宝玉惊惧地后退。
宝玉转身逃出殿门。李贵迎上来:“怎么了二爷?”宝玉惊魂未定地回头看看。李贵顺着宝玉的目光朝殿堂里望去。
宝玉:“你们进去吧,……我去别处走走。”李贵:“茗烟儿!”茗烟:“哎!”
李贵:“招呼着二爷!”茗烟:“二爷,这边儿走。”

72、道院静室门外\par
王道士匆匆走来。茗烟笑嘻嘻地:“王一贴!”王道士呵呵一笑:“这个小杂种!”
茗烟:“怎么这阵子老没到府里来?是不是给人家贴错了膏药,躲到哪儿去了?”
王道士:“早晚我得给你贴上一张,拔拔你那一包子坏水儿!”
王道士一把扒拉开茗烟:“宝玉哥儿呢?”茗烟嘻嘻笑着:“逛累了,炕上歪着呢。”

73、道院静室门内\par
宝玉懒洋洋地歪在炕上。李贵:“哥儿别睡着了。”王道士推门而入。
李贵:“来得好,来得好!王师父,你极会说古记的,说一个给我们小爷听听。”
王一贴笑呵呵地:“哥儿别睡,仔细肚里面筋作怪!”
满屋里人“哄”地一下笑起来。宝玉笑着起身整衣。
王道士冲着门外:“来呵!给哥儿泡壶好酽茶来!”茗烟从门外伸进头来:“我们爷不吃你的茶,
连在你这屋里坐着都嫌有膏药气味儿。”
王道士笑着骂茗烟:“这个小狗头!我的膏药从来不往这屋里搁。
知道哥儿要来,头三、五天就拿香熏了又熏的。”
宝玉:“王师父,早就听人说你的膏药好,到底能治什么病?”
王一贴摇头晃脑地:“哥儿要问我的膏药,说来话长,其中细理,一言难尽。
共药一百二十味,君臣相际,宾客得宜,温凉兼用,贵贱殊方。
内则调元补气,开胃口,养营卫,宁神安志,去寒去暑,化食化痰,外则和血脉,舒筋络,出死肌,生新肉,去风散毒。
其效如神,贴过的便知。”
宝玉:“我不信,一张膏药就治这些病。”
茗烟伸头:“要不怎么敢称‘王一贴’呢!”众笑。
王道士佯嗔着一扬巴掌,茗烟忙把头缩回去。
宝玉:“王师父,我问你,有一种病,……不知可贴得好?”
王一贴:“百病千灾,无不立效。若不见效,哥儿只管揪着胡子打我这老脸,拆我这小庙!……
只是,须要说出病源来。”宝玉笑嘻嘻地:“你猜。”
王道士装模作样地寻思了一会儿,笑笑:“难猜,怕膏药有些不灵了。”
宝玉看看李贵:“你们都出去吧,屋里人多气味儿大。”李贯并两位嬷嬷等人退出房门。
茗烟拿着一支梦甜香进来。宝玉指指炕沿:“坐这儿。”王道士看着宝玉,若有所悟,笑嘻嘻地走近了几步,悄悄说道:
“我猜着了,是不是哥儿如今有了房中之事,想要点儿滋助的药?”
茗烟连忙喝断:“该死!打嘴!”宝玉不解地问:“他说什么?”茗烟:“信他胡说!”
王道士:“哥儿明说了吧。宝玉略一沉吟。
(闪回)横眉立目的夏金桂。(闪回)嘤嘤抽泣的香菱。
宝玉:“我问你,可有贴女人妒嫉的膏药?”
王道士:“哟!这连听也没听说过。”
宝玉:“那还吹什么‘百病千灾,无不立效’!”
王道士眼珠一转,笑嘻嘻地:“别着急,贴妒病的膏药虽然没有,可有一种汤药,只是……”
宝玉眼睛一亮:“怎么?”
王道士:“药性慢些儿,不能收立竿见影之效。”
宝玉:“哦?什么汤药?怎么吃?”
王道上:“此方叫作‘疗妒汤’。用极好的秋梨一个,二钱冰糖,一钱陈皮,水三碗,梨熟为度,每天清早吃一付,就这么……吃来吃去就好了。”
宝玉:“真能见效?”王道士:“一剂不效吃十剂,今儿不效明儿再吃,今年不效吃到明年。
横竖这三味药都是润肺开胃不伤人,甜丝丝的,又止咳嗽又好吃。
吃过一百岁,人横竖是要死的,死了还妒什么?那时不就见效了?”
宝玉、茗烟大笑不止。
茗烟:“好个油嘴的牛头!”
王道士呵呵笑着:“不过是闲着解午盹儿罢了,有什么关系?说笑了你们就值钱。……实告诉你们说,连这膏药也是假的。
——我要有真药,我还吃了作神仙呢!有真的,跑到这儿来混?”宝玉、茗烟大笑。

74、荣国府前角门\par
林之孝带着几个男女仆人守在门口。远处传来马蹄声、车轮声。林之孝张望了一下:“哟,回来了!”
马车渐近,林之孝等人迎上。车停在角门口,两位老嬷嬷下车。林之孝朝车内:“宝二爷!宝……”
李贵下车。林之孝:“怎么?宝二爷呢?”
李贵:“不是先回来了吗?”
林之孝:“什么?”,
李贵:“一出庙门,二爷就骑马先走了。怎么?还没回来?”林之孝:“谁跟着呢?”
李贵:“茗烟。”林之孝:“嗐!快找去吧!这会子里头正急呢!”

75、荣府后街·薛家后院通街门\par
宝玉、茗烟下马。宝玉把马缰交给茗烟,上前叩门。一婆子开门:“是宝二爷。”
宝玉:“我来看一眼香菱的病好些了没有。”婆子冷漠地看着宝玉。宝玉进门。

76、薛家后院\par
院子里一片清冷的寂静。宝玉仿佛有一种不祥的预感,快步朝宝钗的房门走去。

77、宝钗房内\par
宝玉进门,惊异地看着房内。 薛姨妈、宝钗正无声地流泪,同喜、同贵、莺儿在旁边侍立。不时擦着眼睛。
宝玉:“怎么……?”薛姨妈哽咽着:“香菱……”
宝玉:“怎么了?”宝钗站起来,默默地看了宝玉一眼,朝内室走去。
宝玉快步跟进内室。香菱静静地平躺着,脸上蒙着一方惨白的罗帕。宝玉的喉咙上下动着,仿佛在极力吞咽着什么。
时间凝滞了……(闪回)香菱带着一些呆气的笑脸。宝玉的喉咙上下动着。

78、薛家后院通街门外\par
李贵急切地问茗烟:“二爷呢?”
茗烟朝门内努了努嘴。李贵:“哎呀!前头都找翻了天了!谁知道你把他带这儿来了!回去仔细你那皮吧!”

79、宝钗房内\par
宝玉从地上拣起《断肠集》,轻轻地放在香菱身旁……