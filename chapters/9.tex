\chapter{恋风流情友入家塾\quad 起嫌疑顽童闹学堂}
\qi{君子爱人以道,不能减牵恋之情;小人图谋以霸,何可逃侮慢之辱?幻境幻情,又造出一番晓妆新样。
}\par
话说秦业父子专候贾家的人来送上学择日之信。
原来宝玉急于要和秦钟相遇,\qi{妙!不知是怎样相遇。
}却顾不得别的,遂择了后日一定上学。
“后日一早,请秦相公先到我这里,会齐了,一同前去。
”打发人送了信。
\par
至是日一早,宝玉起来时,袭人早已把书笔文物包好,
\zhu{文物:这里应该是文具的意思。}
收拾得停停妥妥,坐在床沿上发闷。
\qi{神理可思,忽又写小儿学堂中一篇文字,亦别书中之未有。
}\meng{此等神理,方是此书的正文。
}见宝玉醒来,只得伏侍他梳洗。
宝玉见他闷闷的,因笑问道:“好姐姐,\qi{开口断不可少此三字。
}你怎么又不自在了?难道怪我上学去丢的你们冷清了不成?”袭人笑道:“这是那里话。
读书是极好的事,不然就潦倒一辈子,终久怎么样呢。
但只一件,只是念书的时节想着书,\meng{袭人方才的闷闷,此时的正论,请教诸公,设身处地,亦必是如此方是,真是曲尽情理,一字也不可少者。
}不念的时节想着家些。
别和他们一处玩闹,\meng{长亭之嘱,不过如是。
}碰见老爷不是玩的。
虽说是奋志要强,\zhu{要强:争强好胜,不肯认输。
}那功课宁可少些,一则贪多嚼不烂,二则身子也要保重。
这就是我的意思,你可要体谅。
”\qi{书正语细嘱一番。
盖袭卿心中,明知宝玉他并非真心奋志之人,袭人自别有说不出来之话。
}袭人说一句,宝玉答应一句。
袭人又道:“大毛衣服我也包好了,\zhu{大毛衣服:大毛:相对“小毛”而言,通常指白狐皮,也泛指其它狐、貂、猞猁等贵重皮毛中长毛可御严寒的。
}交出给小子们去了。
学里冷,好歹想着添换,比不得家里有人照顾。
脚炉手炉的炭也交出去了,你可逼着他们添。
那一起懒贼,你不说,他们乐得不动,白冻坏了你。
”宝玉道:“你放心,出外头我自己都会调停的。
\meng{无人体贴,自己扶持。
}
你们也别闷死在这屋里,长和林妹妹一处去顽笑才好。
”\ping{何不谈及宝钗?此乃宝玉心意流露。
}说着,俱已穿戴齐备,袭人催他去见贾母、贾政、王夫人等。
宝玉且又嘱咐了晴雯麝月等几句,\meng{这才是宝玉的本来面目。
}方出来见贾母。
贾母也未免有几句嘱咐的话。
然后去见王夫人,又出来书房中见贾政。
\par
偏生这日贾政回家早些,\qi{若俗笔则又云不在家矣。
试想若再不见,则成何文字哉?所谓不敢作安逸苟且塞责文字。
}正在书房中与相公清客们闲谈。
忽见宝玉进来请安,回说上学里去,贾政冷笑道:“你如果再提‘上学’两个字,连我也羞死了。
\qi{这一句才补出已往许多文字。
是严父之声。
}依我的话,你竟顽你的去是正理。
仔细站脏了我这地,靠脏了我的门!”\qi{画出宝玉的俯首挨壁之形象来。
}\ping{传统严父的刻板认知使得父亲对孩子总是以训斥的粗暴方法沟通,不论是传达爱还是恨。
}众清客相公们都早起身笑道:“老世翁何必又如此。
今日世兄一去,三二年就可显身成名的了,断不似往年仍作小儿之态了。
天也将饭时,世兄竟快请罢。
”说着便有两个年老的携了宝玉出去。
\par
贾政因问:“跟宝玉的是谁?”只听外面答应了两声,早进来三四个大汉,打千儿请安。
\zhu{打千儿:旧时满族男子向人请安,左膝前屈,右腿后弯,上身微俯,右手下垂,行半跪礼。}
贾政看时,认得是宝玉的奶母之子,名唤李贵。
因向他道:“你们成日家跟他上学,\zhu{家:一作“价”,语尾助词,无义。
成日家:一天到晚,终日里。
}他到底念了些什么书!倒念了些流言混话在肚子里,学了些精致的淘气。
等我闲一闲,先揭了你的皮,再和那不长进的算账!”\meng{此等话似觉无味无理,然而作父母的,到无可如何处,每多用此种法术,所谓百计经营、心力俱瘁者。
}吓的李贵忙双膝跪下,摘了帽子,碰头有声,连连答应“是”,又回说:“哥儿已经念到第三本《诗经》,什么‘呦呦鹿鸣,荷叶浮萍’,\zhu{呦呦鹿鸣,荷叶浮萍:《诗经·小雅·鹿鸣》:“哟哟鹿鸣,食野之苹。
”“荷叶浮萍”是李贵学舌闹出的笑话。
}小的不敢撒谎。
”说的满座哄然大笑起来。
贾政也撑不住笑了。
\ping{一本正经的严父也撑不住自己的人设了。
}因说道:“那怕再念三十本《诗经》,也都是掩耳偷铃,哄人而已。
你去请学里太爷的安,就说我说了:什么《诗经》、古文,\zhu{古文:通常指先秦两汉以及唐宋八大家的散文。
}一概不用虚应故事,\zhu{虚应故事:照旧例行事,敷衍应付。
故事:犹言成例、老例。
}只是先把《四书》一气讲明背熟,是最要紧的。
”李贵忙答应“是”,见贾政无话,方退出去。
\par
此时宝玉独站在院外屏声静候,待他们出来,便忙忙的走了。
李贵等一面弹衣服,一面说道:“哥儿可听见了不曾?先要揭我们的皮呢!人家的奴才跟主子赚些好体面,我们这等奴才白陪挨打受骂的。
从此后也可怜见些才好。
”\meng{可以谓能达主人之意,不辱君命。
}宝玉笑道:“好哥哥,你别委曲,我明儿请你。
”\ping{小主人和仆从的关系何其有趣。
李贵和宝玉一方面互为平等相待的玩伴,另一方面两个人之间又存在着支配关系。
随着宝玉年龄的增长,平等的玩伴会逐渐消失,或奴颜卑膝或钻营狡猾的奴才会越来越多。
第八回宝玉探望宝钗路上遇到的詹光、单聘仁、吴新登、戴良、钱华等人,宝玉未来会遇到更多。
这使我想起了鲁迅笔下的闰土。
}李贵道:“小祖宗,谁敢望你请?只求听一句半句话就有了。
”说着,又至贾母这边,秦钟已早来候着了,贾母正和他说话儿呢。
\qi{此处便写贾母爱秦钟一如其孙,至后文方不突然。
}
于是二人见过,辞了贾母。
宝玉忽想起未辞黛玉,\qi{妙极!何顿挫之至!余已忘却,至此心神一畅,一丝不漏。
}因又忙至黛玉房中来作辞。
彼时黛玉才在窗下对镜理妆,听宝玉说上学去,因笑道:“好!这一去,可定是要‘蟾宫折桂’去了。
\zhu{蟾宫折桂:晋代文士郤诜(音“戏身”)因长于答对策问当上了官,他认为自己“举贤良对策,为天下第一,犹桂林之一枝,昆山之片玉。
”见《晋书·郤诜传》。
后遂以“折桂”比喻科举及第。
又因传说蟾宫(月宫)中有桂树,所以又把折桂与蟾宫联系起来。
}\meng{此写黛玉,差强人意。
《西厢》双文,能不抱愧!\zhu{双文:即《西厢记》里的崔莺莺,因莺莺的名字是用两个“莺”字叠成。
脂批说的“双文抱愧”,是指在张生要进京应试时,崔莺莺舍不得分离,别离宴上她对张生唱道:
“年少呵轻远别,情薄呵易弃掷。全不想腿儿相挨,脸儿相偎,手儿相携。你与俺崔相国做女婿,妻荣夫贵,
但得一个并头莲,煞强如状元及第。”(出自西厢记第四本:张君瑞梦莺莺杂剧)
崔莺莺希望夫妻厮守而轻视功名,并由此而抱怨张生;
对比黛玉送别宝玉的祈愿祝贺语,脂批认为崔莺莺应该“抱愧”。
“差强人意”在这里令人费解,这里的“人意”可能指的是“人情”。
前文袭人送别,嘱咐宝玉重情而不要重名:“虽说是奋志要强,那功课宁可少些”。
这里黛玉送别,祝“蟾宫折桂”是重名而不重情。
所有有“差强人意(人情)”的评语。
}}
我不能送你了。
”宝玉道:“好妹妹,等我下学再吃晚饭。
和胭脂膏子也等我来再制。
” 唠叨了半日,方撤身去了。
\qi{如此总一句,更妙!}黛玉忙又叫住问道:“你怎么不去辞辞你宝姐姐来?”\qi{必有是语,方是黛玉,此又系黛玉平生之病。
}宝玉笑而不答。
\meng{黛玉之问,宝玉之笑,两心一照,何等神工鬼斧文章。
}一径同秦钟上学去了。
\par
原来这贾家义学离此也不甚远,不过一里之遥,原系始祖所立,恐族中子弟有贫穷不能请师者,即入此中肄业。
\zhu{肄:音“亿”。肄业:练习,学习。}
凡族中有官爵之人,皆供给银两,按俸之多寡帮助,为学中之费。
特共举年高有德之人为塾掌,\zhu{塾掌:塾:私塾。
旧时民间办的学校。
私塾的主管者,叫“塾掌”。
}
专为训课子弟。
\meng{创立者之用心,可谓至矣。
}如今宝秦二人来了,一一的都互相拜见过,读起书来。
自此以后,他二人同来同往,同起同坐,愈加亲密。
又兼贾母爱惜,也时常的留下秦钟,住上三天五日,与自己的重孙一般疼爱。
因见秦钟不甚宽裕,更又助他些衣履等物。
不上一月之工,秦钟在荣府便熟了。
\qi{交待得清。
}宝玉终是不安分之人,\qi{写宝玉总作如此笔。
}竟一味的随心所欲,因此又发了癖性,又特向秦钟悄说道:“咱们俩个人一样的年纪,况又是同窗,以后不必论叔侄,只论弟兄朋友就是了。
”\meng{悄说之时何时?舍尊就卑何心?随心所欲何癖?相亲爱密何情?}
先是秦钟不肯,当不得宝玉不依,只叫他“兄弟”,或叫他的表字“鲸卿”,秦钟也只得混着乱叫起来。
\par
原来这学中虽都是本族人丁与些亲戚家的子弟,俗语说的好,“一龙生九种,九种各别。
”\zhu{一龙生九种,九种各别:俗传龙生九子不成龙,各有所好。
九子说法不一。
这里以喻贾府族大人多,好坏不一,各种各样的人都有。
}未免人多了,就有龙蛇混杂,下流人物在内。
\qi{伏一笔。
}自宝、秦二人来了,都生的花朵儿一般的模样,又见秦钟腼腆温柔,未语面先红,怯怯羞羞,有女儿之风;宝玉又是天生成惯能做小服低,赔身下气,性情体贴,话语绵缠,\qi{凡四语十六字,上用“天生成”三字,真正写尽古今情种人也。
}因此二人更加亲厚,也怨不得那起同窗人起了疑,背地里你言我语,诟谇谣诼,\zhu{诟谇(音“够碎”):辱骂斥责。
谣诼(诼音“浊”):造谣诽谤。
《离骚》:“众女嫉余之蛾眉兮,谣诼谓余以善淫。
”}布满书房内外。
\qi{伏下文“阿呆争风”一回。
\zhu{第三十四回,
宝玉因挨打后,宝钗探视,心中暗想:“……我的哥哥(薛蟠)素日恣心纵欲,毫无防范的那种心性。
当日为一个秦钟,还闹的天翻地覆。”
}}\par
原来薛蟠自来王夫人处住后,便知有一家学,学中广有青年子弟,不免偶动了龙阳之兴,\zhu{龙阳之兴:即喜好男色。
战国时有个叫龙阳君的人,以男色事魏王而得宠。
见《战国策·魏策》。
后世因以“龙阳”代指“男色”。
}因此也假来上学读书,不过是三日打鱼,两日晒网,白送些束脩礼物与贾代儒,\zhu{束脩:脩,音“休”,干肉。
十条干肉扎成一束,叫束脩。
古代见面礼之最薄者。
《论语·述而》:“自行束脩以上,吾未尝无诲焉。
”原指孔子的学生向他送的见面礼,后世遂用作学费的代称。
}却不曾有一些儿进益,只图结交些契弟。
\zhu{契弟:拜把兄弟。
这里含有男色的意思。
}
谁想这学内就有好几个小学生,图了薛蟠的银钱吃穿,被他哄上手的,也不消多记。
\qi{先虚写几个淫浪蠢物,以陪下文,方不孤不板。
}\chen{伏下金荣。
}更有两个多情的小学生,\qi{此处用“多情”二字方妙。
}亦不知是那一房的亲眷,亦未考真名姓,\qi{一并隐其姓名,所谓“具菩提之心,秉刀斧之笔”。
}只因生得妩媚风流,满学中都送了他两个外号,一号“香怜”,一号“玉爱”。
谁都有窃慕之意,将不利于孺子之心,\zhu{将不利于孺子之心:一语出《尚书》。
周武王死时,其子成王年幼,成王之叔周公旦摄政,管叔、蔡叔、霍叔等诸叔散布流言说:“公将不利于孺子。
”意思是说,周公要篡夺成王的王位。
这里是说有人想在这两个孩子身上打主意。
孺子:小孩子;小后生。
}\qi{诙谐得妙,又似李笠翁书中之趣语。
\zhu{
李笠翁:名李渔。李渔作品中常有生活中的闲情逸趣及诙谐文辞。
}
}只是都惧薛蟠的威势,不敢来沾惹。
如今宝、秦二人一来了,见了他两个,也不免缱绻羡爱,亦因知系薛蟠相知,故未敢轻举妄动。
香、玉二人心中,也一般的留情与宝、秦。
因此四人心中虽有情意,只未发迹。
每日一入学中,四处各坐,却八目勾留,或设言托意,或咏桑寓柳,\zhu{咏桑寓柳:借咏赞桑树来赞美柳树,喻表面称赞某一事物,实际寓托着对另一事物的真实感情。
}遥以心照,却外面自为避人眼目。
\qi{小儿之态活现,掩耳盗铃者亦然,世人亦复不少。
}不意偏又有几个滑贼看出形景来,都背后挤眉弄眼,或咳嗽扬声,\meng{才子辈偏无不解之事。
}\qi{又画出历来学中一群顽皮来。
}这也非此一日。
\par
可巧这日代儒有事,早已回家去了,又留下一句七言对联,命学生对了,明日再来上书;将学中之事,又命贾瑞\qi{又出一贾瑞。
}暂且管理。
妙在薛蟠如今不大来学中应卯了,\zhu{应卯:古代军营、官府点名都在卯时(上午五时至七时),故称点名为“点卯”。
应卯:到班应名;也常引伸为按例到场,应付差事。
}因此秦钟趁此和香怜挤眉弄眼,递暗号儿,二人假装出小恭,
\zhu{出小恭:小便。}
走至后院说体己话。
\zhu{体己话:私下里的知心话。
“体已”亦作“梯己”,意即私人的、贴心的。
私蓄亦可称作“梯己”。
}秦钟先问他:“家里的大人可管你交朋友不管?”\qi{妙问,真真活跳出两个小儿来。
}一语未了,只听背后咳嗽了一声。
\qi{太急了些,该再听他二人如何结局,正所谓小儿之态也,酷肖之极。
}二人唬的忙回头看时,原来是窗友名金荣\qi{妙名,盖云有金自荣,廉耻何益哉?}者。
香怜本有些性急,羞怒相激,问他道:“你咳嗽什么?难道不许我两个说话不成?”金荣笑道:“许你们说话,难道不许我咳嗽不成?我只问你们:有话不明说,许你们这样鬼鬼祟祟的干什么故事?我可也拿住了,还赖什么!先得让我抽个头儿,\zhu{抽头:原指设局聚赌抽取头钱,即向赢钱的赌徒抽取一部分的利益给提供赌博场所的人。
这里是占便宜、从别人身上得到某些好处的意思。
}咱们一声儿不言语,不然大家就奋起来。
”\zhu{奋起来:声张开来的意思。
}秦、香二人急得飞红的脸,便问道:“你拿住什么了?”金荣笑道:“我现拿住了是真的。
”说着,又拍着手笑嚷道:“贴的好烧饼!\zhu{用在热的炉子内膛贴面饼经过烤制成为烧饼,比喻在热的炕上人挨人躺着,这里暗指男男同性性行为。
第六十五回三个小厮的对话可以证明。
“隆儿寿儿关了门,回头见喜儿直挺挺的仰卧炕上,二人便推他说:‘好兄弟,起来好生睡,只顾你一个人,我们就苦了。
’那喜儿便说道:‘咱们今儿可要公公道道的贴一炉子烧饼,要有一个充正经的人,我痛把你妈一肏。
’”}你们都不买一个吃去?”秦钟、香怜二人又气又急,忙进来向贾瑞前告金荣,说金荣无故欺负他两个。
\par
原来这贾瑞最是个图便宜没行止的人,每在学中以公报私,勒索子弟们请他;\meng{学中亦自有此辈,可为痛哭。
}后又附助着薛蟠,图些银钱酒肉,一任薛蟠横行霸道,他不但不去管约,反助纣为虐讨好儿。
\zhu{助纣为虐:亦作“助桀为虐”。
比喻帮助恶人作坏事。
桀、纣:夏、商两朝的末代君主,史称暴君。
}偏那薛蟠本是浮萍心性,今日爱东,明日爱西,近来又有了新朋友,把香、玉二人丢开一边。
就连金荣亦是当日的好朋友,自有了香、玉二人,便弃了金荣。
近日连香、玉亦已见弃。
故贾瑞也无了提携帮衬之人,不说薛蟠得新弃旧,只怨香、玉二人不在薛蟠前提携帮补他,\qi{无耻小人,真有此心。
}因此贾瑞金荣等一干人,也正在醋妒他两个。
今儿见秦、香二人来告金荣,贾瑞心中便不自在起来,不好呵叱秦钟,却拿着香怜作法,\zhu{“扎筏子”就是把整体脱下来的牛羊皮充上气绑扎好,作为渡水工具。
“气”是皮筏子的重要内容。
当人们把皮筏子充气后绑扎这一意象图式映射到人与人之间的关系时,容器隐喻的认知机制就开始发挥作用: 对于批评者而言,我批评、责骂或惩罚你,你就是出气的对象;对于被批评者而言,我接受批评、责骂或惩罚,就是受你的气。
无论出气还是受气,被批评者都是批评者撒气的容器。
“扎筏子”指“找借口出气、撒气”;“作筏子”则是动词“扎”词义泛化的结果,“扎筏子”和“作筏子”同义,词义都包含容器隐喻的认知机制;和“扎筏子”、“作筏子”词义中包含容器隐喻认知机制不同,“作法”一词的语义透明度很高,从字面义就可以推测其整体意义。
“作法”就是树立某种标准,给别人立规矩,通过责骂、惩罚等手段处理某人立威,杀鸡儆猴,以儆其馀。
“法”后加名词后缀“子”,“作法子”的整体意义不变。
客观事理上,撒气和立威本身在使用过程中容易产生语义纠葛。
因为很多时候,撒气和立威有些含混,拿某人撒气往往是立威的手段,而立威又常常是撒气的结果。
尽管“扎筏子”与“作法子”两组词使用过程中极容易出现语义纠葛现象,但二者表义的侧重点还是有区别,前者侧重出气,后者侧重立威,宜看作两组不同的词。
}
反说他多事,着实抢白了几句。
香怜反讨了没趣,连秦钟也讪讪的各归坐位去了。
金荣越发得了意,摇头咂嘴的,口内还说许多闲话,玉爱偏又听了不忿,\zhu{不忿:不高兴,不服气。
}两个人隔座咕咕唧唧的角起口来。
金荣只一口咬定说:“方才明明的撞见他两个在后院子里亲嘴摸屁股,两个商议定了,一对一肏,撅草棍儿抽长短,\meng{“怎么长短”四字,何等韵雅,何等浑含!俚语得文人提来,便觉有金玉为声之象\zhu{按:蒙、戚本正文金荣的话删去了脏字,作“方才明明的撞见他两个在后院里商议着什么长短。
”以金荣的性格说话不当这么含蓄。“韵雅”、“浑含”是对删去脏字后的文本的称赞。
}。
}谁长谁先干。
”金荣只顾得意乱说,却不防还有别人。
谁知早又触怒了一个。
你道这个是谁?\par
原来这一个名唤贾蔷,\qi{新而艳,得空便入。
}亦系宁府中之正派玄孙,父母早亡,从小儿跟贾珍过活,如今长了十六岁,比贾蓉生的还风流俊俏。
他兄弟二人最相亲厚,常相共处。
宁府人多口杂,那些不得志的奴仆们,专能造言诽谤主人,因此不知又有了什么小人诟谇谣诼之辞。
贾珍想亦风闻得些口声不大好,自己也要避些嫌疑,如今竟分与房舍,命贾蔷搬出宁府,自去立门户过活去了。
\meng{此等嫌疑不敢认真搜查,悄为分计,皆以含而不露为文,真是灵活至极之笔。
}这贾蔷外相既美,\qi{亦不免招谤,难怪小人之口。
}内性又聪明,虽然应名来上学,亦不过虚掩眼目而已。
仍是斗鸡走狗,赏花玩柳。
总恃上有贾珍溺爱,\qi{贬贾珍最重。
}下有贾蓉匡助,\zhu{匡助:帮助,辅助。
}\qi{贬贾蓉次之。
}因此族中人谁敢来触逆于他。
他既和贾蓉最好,今见有人欺负秦钟,如何肯依?如今自己要挺身出来报不平,心中却忖度一番,\qi{这一忖度,方是聪明人之心机,写得最好看,最细致。
}想道:“金荣贾瑞一干人,都是薛大叔的相知,向日我又与薛大叔相好,倘或我一出头,他们告诉了老薛,\qi{先曰“薛大叔”,此曰“老薛”,写尽骄侈纨绔。
}我们岂不伤和气?待要不管,如此谣言,说的大家没趣。
如今何不用计制服,又止息了口声,又不伤了脸面。
”想毕,也装出小恭,走至外面,悄悄的把跟宝玉的书童名唤茗烟\qi{又出一茗烟。
}者唤到身边,如此这般调拨他几句。
\qi{如此便好,不必细述。
}\par
这茗烟乃是宝玉第一个得用的,且又年轻不谙世事,如今听贾蔷说金荣如此欺负秦钟,连他爷宝玉都干连在内,不给他个利害,下次越发狂纵难制了。
这茗烟无故就要欺压人的,如今得了这个信,又有贾蔷助着,便一头进来找金荣,也不叫金相公了,只说:“姓金的,你是什么东西!”贾蔷遂跺一跺靴子,故意整整衣服,看看日影儿说:“是时候了。
”遂先向贾瑞说有事要早一步。
贾瑞不敢强他,只得随他去了。
\ping{贾蔷作为幕后主使在架桥拨火后采取隔岸观火的策略。
}这里茗烟先一把揪住金荣,\meng{豪奴辈,虽系主人亲故亦随便欺慢,即有一二不服气者,而豪家多是偏护家人。
理之所无,而事之尽有,不知是何心思,实非凡常可能测略。
}问道:“我们肏屁股不肏屁股,管你\jiji\baba 相干?横竖没肏你爹去罢了!你是好小子,出来动一动你茗大爷!”吓的满屋中子弟都怔怔的痴望。
贾瑞忙吆喝:“茗烟不得撒野!”金荣气黄了脸,说:“反了!奴才小子都敢如此,我和你主子说。
”便夺手要去抓打宝玉秦钟。
\qi{好看之极!}尚未去时,从脑后“飕”的一声,早见一方砚瓦飞来,\qi{好看好笑之极!}并不知系何人打来的,幸未打着,却又打了旁人的座上,这座上乃是贾兰、贾菌。
\par
贾菌亦系荣府近派的重孙,\qi{先写一宁派,又写一荣派,互相错综得妙。
}
其母亦少寡,独守着贾菌,这贾菌与贾兰最好,所以二人同桌而坐。
谁知贾菌年纪虽小,志气最大,极是淘气不怕人的。
\qi{要知没志气小儿,必不会淘气。
}他在座上冷眼看见金荣的朋友暗助金荣,飞砚来打茗烟,偏没打着茗烟,便落在他座上,正打在面前,将一个磁砚水壶打了个粉碎,溅了一书黑水。
\qi{这等忙,有此闲处用笔。
}贾菌如何依得,便骂:“好囚攮的们,
\zhu{
攮:骂人糊涂愚笨。例如“狗攮的”。 
囚攮的:骂人的话。意指囚犯的子女。
}
这不都动了手了么!”\qi{好听煞。
}骂着,也抓起砚砖来要飞。
\qi{先瓦砚,次砖砚,转换得妙极。
}贾兰是个省事的,忙按住砚,极口劝道:“好兄弟,不与咱们相干。
”\qi{是贾兰口气。
}贾菌如何忍得住,便两手抱起书匣子来,照那边抡了去。
\qi{先“飞”后“抡”,用字得神,好看之极!}
终是身小力薄,却抡不到那里,刚到宝玉秦钟桌案上就落了下来,只听“哗啷啷”一声,砸在桌上,书本纸片等至于笔砚之物撒了一桌,又把宝玉的一碗茶也砸得碗碎茶流。
\qi{好看之极!不打着别个,偏打着二人,亦想不到文章也。
此书此等笔法,与后文踢着袭人、误打平儿,是一样章法。
\zhu{
踢着袭人:第三十回,宝玉敲门,因开门慢而踢袭人。
误打平儿:第四十四回,贾琏偷情,凤姐捉奸在床却回身打平儿。
}
}贾菌便跳出来,要揪打那一个飞砚的。
金荣此时随手抓了一根毛竹大板在手,地狭人多,那里经得舞动长板。
茗烟早吃了一下,乱嚷:“你们还不来动手!”宝玉还有三个小厮:一名锄药,一名扫红,一名墨雨。
这三个岂有不淘气的,一齐乱嚷:“小妇养的!动了兵器了!”\qi{好听之极,好看之极!}墨雨遂掇起一根门闩,\zhu{掇:音“多”,拾取。
}扫红锄药手中都是马鞭子,蜂拥而上。
贾瑞急拦一回这个,劝一回那个,谁听他的话,肆行大闹。
众顽童也有趁势帮着打太平拳助乐的,\zhu{打太平拳:别人打架,在旁趁机捅几下冷拳,因不易为人发觉,所以叫“打太平拳”。
}也有胆小藏在一边的,也有直立在桌上拍着手儿乱笑、喝着声儿叫打的,登时间鼎沸起来。
\meng{燕青打擂台,也不过如此。
\zhu{
燕青打擂台:水浒故事。任原自称擎天柱在泰安州东岳庙摆擂台,两年未遇敌手,却被善于相扑的燕青打败。
}
}\par
外边李贵等几个大仆人听见里边作反起来,忙都进来一齐喝住。
问是何原故。
众声不一,这一个如此说,那一个又如彼说。
\qi{妙!如闻其声。
}李贵且喝骂了茗烟四个一顿,撵了出去。
\qi{处治得好。
}秦钟的头早撞在金荣的板上,打去一层油皮,宝玉正拿褂襟子替他揉呢,见喝住了众人,便命:“李贵,收书!拉马来,我回去回太爷去!我们被人欺负了,不敢说别的,守礼来告诉瑞大爷,瑞大爷反倒派我们不是,听人家骂我们,还调唆他们打我们。
茗烟见人欺负我,他岂有不为我的;他们反打伙儿打了茗烟,\zhu{打伙儿:结伴,合伙。
}连秦钟的头也打破了,还在这里念什么书!不如散了罢。
”李贵劝道:“哥儿不要性急。
太爷既有事回家去了,这会子为这点子事去聒噪他老人家,\zhu{聒:音“郭”,喧扰,声音嘈杂。
}倒显的咱们没理。
依我的主意,那里的事那里了结好,何必去惊动他老人家。
这都是瑞大爷的不是,太爷不在这里,你老人家就是这学里的头脑了,众人看你行事。
\meng{劝的心思,有个太爷得知,未必然之。
故巧为辗转以结其局,而不失其体。
}众人有了不是,该打的打,该罚的罚,如何等闹到这步田地不管?”贾瑞道:“我吆喝着都不听。
”\qi{如闻。
}李贵笑道:“不怕你老人家恼我,素日你老人家到底有些不正经,所以这些兄弟才不听。
就闹到太爷跟前去,连你老人家也脱不过的。
还不快作主意撕罗开了罢。
”\zhu{撕罗:调停,解决。
}宝玉道:“撕罗什么?我必是回去的!”秦钟哭道:“有金荣,我是不在这里念书的。
”宝玉道:“这是为什么?难道有人家来得的,咱们倒来不得?我必回明白众人,撵了金荣去。
”又问李贵:“金荣是那一房的亲戚?”李贵想了一想:“也不用问了。
若说起那一房的亲戚,更伤了弟兄们的和气了。
”\par
茗烟在窗外道:“他是东胡同里璜大奶奶的侄儿,
\zhu{璜:音“黄”。玉器,形状像半块璧。}
那是什么硬正仗腰子的,\zhu{硬正仗腰子的:犹言硬的后台。
仗腰子的:也作“仗腰眼子的”,指可作依仗的靠山。
}也来唬我们。
璜大奶奶是他姑娘。
\zhu{姑娘:姑姑,父亲的姐妹。}
你那姑妈只会打旋磨儿,\zhu{打旋(旋音“学”)磨:围着人打转转,向人献殷勤的意思。
}给我们琏二奶奶跪着借当头。
\zhu{借当头:旧时用实物作抵押去当铺借钱叫当或典当,用作抵押典质的东西叫当头。
借别人的东西去当铺典当,叫作“借当头”。
}\meng{可怜!开口告人,终身是玷。
\zhu{玷:音“店”。白玉上面的污点;弄脏,使有污点。}
}\ping{仗势豪奴敢骂主人家的穷亲戚。
}我眼里就看不起他那样的主子奶奶!”李贵忙断喝不止,说:“偏你这小狗肏的知道,有这些蛆嚼!”\zhu{蛆嚼:即嚼蛆。
骂人胡说八道的意思。
}宝玉冷笑道:“我只当是谁的亲戚,原来是璜嫂子的侄儿,我就去问问他来!”说着便要走,叫茗烟进来包书。
茗烟包着书,又得意道:“爷也不用自己去见,等我去到他家,就说老太太有说的话问他呢,雇上一辆车拉进去,当着老太太问他,岂不省事?”\qi{又以贾母欺压,更妙!}李贵忙喝道:“你要死!仔细回去我好不好先捶了你,然后再回老爷太太,就说宝玉全是你调唆的。
我这里好容易劝哄的好了一半了,你又来生个新法子。
你闹了学堂,不说变法儿压息了才是,倒要往大里奋!”\zhu{奋:振作、鼓气。前文金荣有“不然大家就奋起来”一语,可与此处互证。}茗烟方不敢作声儿了。
\par
此时贾瑞也怕闹大了,自己也不干净,只得委曲着来央告秦钟,又央告宝玉。
先是他二人不肯。
后来宝玉说:“不回去也罢了,只叫金荣赔不是便罢。
”金荣先是不肯,后来禁不得贾瑞也来逼他去赔不是,李贵等只得好劝金荣说:“原来是你起的端,你不这样,怎得了局?”金荣强不得,只得与秦钟作了揖。
宝玉还不依,偏定要磕头。
\par
贾瑞只要暂息此事,又悄悄的劝金荣说:“俗语说得好:‘杀人不过头点地。
’你既惹出事来,少不得下点气儿,磕个头就完事了。
”金荣无奈,只得进前来与秦钟磕头。
\ping{家学私塾里的学生虽然仍是孩子,但财大气粗者如薛蟠,可以靠贿赂随心所欲,家族核心的宝玉平时亲和,但是稍有冒犯时家境地位的差别立马会被他捡起作为武器。
}且听下回分解。
\par
\zhu{脂评各抄本中,现存有第九回的有八种。
而该回的结尾部分各本甚不相同,出现了五种异文。
为了与下回衔接,此处暂依戚本。
具体情况如下:\hang
己、庚、杨、蒙四本:\hang
此时,贾瑞也生恐闹大了,自己也不干净,只得委曲着来央告秦钟,又央告宝玉。
先是他二人不肯。
后来宝玉说:“不回去也罢了,只叫金荣赔不是便罢。
”金荣先是不肯,后来禁不得贾瑞也来逼他去赔不是,李贵等只得好劝金荣,说:“原是你起的端,你不这样,怎得了局?”金荣强不过,只得与秦钟作了揖。
宝玉还不依,偏定要磕头。
(此段其他各本也大致相同)\hang
贾瑞只要暂息此事,又悄悄的劝金荣说:“俗语说得好:‘杀人不过头点地。
’你既惹出事来,少不得下点气儿,磕个头就完事了。
”金荣无奈,只得进前来与宝玉磕头。
且听下回分解。
\hang
戚本基本相同,只有最后金荣是“与秦钟磕头”。
\hang
列本:\hang
贾瑞只要暂息此事,又悄悄的劝金荣磕头。
金荣无奈何。
俗语云:在他门下过,怎敢不低头。
\hang
甲辰本:\hang
贾瑞只要暂息此事,又悄悄的劝金荣说:“俗语云,忍得一时忿,终身无恼闷。
”\hang
舒本:\hang
贾瑞只要暂息此事,又悄悄的劝金荣说:“俗语说的‘光棍不吃眼前亏’。
咱们如今少不得委曲着陪个不是,然后再寻主意报仇。
不然,弄出事来,道是你起端,也不得干净。
”金荣听了有理,方忍气含愧的来与秦钟磕了一个头,方罢了。
贾瑞遂立意要去调拨薛蟠来报仇,与金荣计议已定,一时散学,各自回家。
不知他怎么去调拨薛蟠,且看下回分解。
\hang
因为下一回开头是这样:\hang
话说金荣因人多势众,又兼贾瑞勒令,赔了不是,给秦钟磕了头,宝玉方才不吵闹了。
……\hang
所以前面五种文字中,前后文衔接的比较好的自属戚本。
目前的新校本大都采用戚本的文字。
\hang
舒本的文字比较特别,它跟第十回开头衔接不起来,因为后面再没有提到贾瑞是怎样去调拨薛蟠来报仇的。
但是,到了第三十四回,为了误会宝玉挨打是薛蟠挑拨的,宝钗曾联想到,她哥“当日为一个秦钟还闹的天翻地覆”。
书中并没有其他地方有薛蟠和秦钟同时登场的,所以“闹的天翻地覆”应该就是指的这里提到的这件事。
\hang
关于怎么会出现这么多的异文,刘世德《红楼梦版本探微》第二章有很详细的分析。
刘先生的结论是所有的异文均出自曹雪芹之手,舒本的文字是初稿,而其他几种是后来的改稿。
郑庆山《脂本汇校石头记》有类似见解。
\hang
刘先生强调舒本某些异文的重要性是对的。
说这里舒本的文字出自曹雪芹之手我完全赞同。
但其他各本的文字是否作者改稿却值得怀疑。
虽说曹雪芹曾“于悼红轩中,披阅十载,增删五次”,但是否刚好把这个地方修改了五次,刚好五次修改的稿子都流传了出来,而且刚好都被过录而流传至今?天下就算有这么巧的事,为什么现在各本的其他地方没有这样戏剧性的多次修改的痕迹留存?\hang
如果没有这么巧,那么会是怎样的情况呢?\hang
我们知道,舒本是个拼凑本。
它的第九回文字较胜,所据底本当为曹雪芹的原稿,而且有可能是类似于甲戌本底本的定本。
而另外的本子在“贾瑞只要暂息此事又悄悄的劝金荣说俗语说”之后残缺了——抄本在回首或回末残缺的现象书中有好几处——后来的整理者以及过录者就根据上文文意,找了一句意思相关的俗语来补缺,所以才会出现所用俗语各各不同的情况。
列本和甲辰本补文简单,所引俗语的意思都是比较泄气的,也并不符合贾瑞、金荣的性格,显非曹雪芹原作。
己、庚、杨、蒙四本的文字意思比较完整而一致,但把磕头的对象错为宝玉(这个错误直到戚本或其母本才被人改正),仍然不是曹雪芹亲笔,很可能是脂砚或畸笏所为。
\hang
周汝昌《石头记会真》第九回回末按语说:\hang
此回之结式,“在苏本”独存其真,可贵之至。
《梦觉本》犹保持原式,却将俗语抽换,大背原意。
雪芹岂肯宣扬此等人生哲学乎?《舒序本》之妄纂收场,一片胡云,可发大噱。
由此以观,诸本之高下纯杂,一面秦镜,俨然可鉴。
\hang
周老先生经常有一些很独特的见解。
按他的意见,贾瑞成了曹雪芹“人生哲学”的代言人。
而各本之高下、何者存其真,本是仁智之见,周先生未加分析,用严厉措辞指责舒序本“妄纂收场,一片胡云,可发大噱”,就有点让人难以理解了。
}
\par
\qi{总评:此篇写贾氏学中,非亲即族,且学乃大众之规范,人伦之根本。
首先悖乱,以至于此极,其贾家之气数,即此可知。
挟用袭人之风流,群小之恶逆,一扬一抑,作者自必有所取。
}
\dai{017}{袭人叮嘱送别宝玉上学}
\dai{018}{起嫌疑顽童闹学堂}
\sun{p9-1}{训劣子李贵承申饬,辞黛玉宝玉携秦钟上学}{图右侧:宝玉来到贾政书房中请安时,贾政正与清客们说话。
贾政训斥道:“你要再提‘上学’两个字,连我也羞死了。
”又留下跟班李贵问话。
训道:“他到底念了些什么书?倒念了混话在肚子里, 学了些精致的淘气,等我闲一闲,先揭了你的皮,再和他算账!”吓得李贵双膝跪下,连连答应“是”。
那宝玉在房外屏气偷听。
图左侧:宝玉之后到黛玉房中告辞,携了秦钟上学去了。
}
\sun{p9-2}{起嫌疑顽童闹学堂}{图右侧:秦钟和香怜挤眉弄眼,递暗号儿,二人假装出小恭,走至后院说体己话。
秦钟先问他:“家里的大人可管你交朋友不管?”一语未了,只听背后咳嗽了一声,原来是金荣。
图左侧:由于金荣惹了秦钟,贾蓉族弟贾蔷挑唆宝玉书童茗烟出头报复,于是众学童在教室里大打出手。
}