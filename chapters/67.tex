\chapter[馈土物颦卿念故里 \quad 讯家童凤姐蓄阴谋]{馈土物颦卿念故里 \quad 讯家童凤姐蓄阴谋\foot{按:此回庚辰本缺。
其他各本存在两种类型文字,且出入较大。
列、戚、甲辰本此回情节安排完整合理,但较为罗嗦拖沓,玩其文字,当非出于曹雪芹手笔,或系脂砚等人据曹雪芹残稿补写而成。
杨本及程甲乙本一系文字比较简练,显系经过后人整理,且存在删减过度及某些情节欠合理的问题。
由于两种版本文字多寡悬殊,无法互校,故本书正文依据早出的列藏本,另将程甲本文字附录于书后附录。
}}
\zhu{讯:审问。}
\par
话说尤三姐自戕之后,
\zhu{戕[qiāng]:杀害;摧残。}
尤老娘以及尤二姐、贾珍、尤氏并贾蓉、贾琏等闻之,俱各不胜悲恸伤感,自不必说,忙着人治买棺木盛殓,送往城外埋葬。
却说柳湘莲见尤三姐身亡,迷性不悟,尚有痴情眷恋,被道人数句偈言打破迷关,\zhu{偈(音“记”):梵文音译“偈陀”或“伽陀”之略,意译为颂。
一般为四句之韵文。
}竟自削发出家,跟随疯道人飘然而去,不知何往。
后事暂且不表。
\par
且说薛姨妈闻知湘莲已说定了尤三姐为妻,心甚喜悦,正自高高兴兴要打算替他买房屋、治器用、办妆奁,择吉日迎娶过门等事,以报他救命之恩。
忽有家中小厮见薛姨妈,告知尤三姐自戕与柳湘莲出家的信息,心甚叹息。
正自猜疑是为什么原故,时值宝钗从园子里过来,薛姨妈便对宝钗说道:“我的儿,你听见了没有?你珍大嫂子的妹妹尤三姐,他不是已经许定了给你哥哥的义弟柳湘莲的?这也很好。
不知为什么尤三姐自刎了,柳湘莲也出了家了。
真正奇怪的事,叫人意想不到!”宝钗听了,并不在意,便说道:“俗语说的好,‘天有不测风云,人有旦夕祸福’。
这也是他们前生命定,活该不是夫妻。
妈所为的是因有救哥哥的一段好处,故谆谆感叹。
\zhu{谆谆:反覆多言的样子。
}如果他两人齐齐全全的,妈自然该替他料理,如今死的死了,出家的出了家了,依我说,也只好由他罢了。
妈也不必为他们伤感,损了自己的身子。
\ping{这里对比写薛姨妈的感伤和宝钗的冷,也是展示了宝钗性格中像雪一样冷的那一面。
另外这也体现了宝钗是很现实理性的女孩子,对于超出自己能力范围的无法挽回的事情,不再纠结考虑,更多的关注于眼下自己尚且可以有所作为的事情。
}倒是自从哥哥打江南回来了一二十日,贩了来的货物,想来也该发完了,那同伴去的伙计们辛辛苦苦的,来回几个月,妈同哥哥商议商议,也该请一请,酬谢酬谢才是。
不然,倒叫他们看着无礼似的。
”\par
母女正说之间,见薛蟠自外而入,眼中尚有泪痕未干。
一进门。
便向他母亲拍手说道:“妈,可知道柳大哥、尤三姐的事么?”薛姨妈说:“我在园子里听见大家议论,正在这里才和你妹子说这件公案呢。
”薛蟠道:“这事可奇不奇?”薛姨妈说:“可是柳相公那样一个年轻聪明的人,怎么就一时糊涂跟着道士去了呢?我想他前世必是有夙缘的有根基的人,\zhu{夙:平素,过去。
}所以才容易听得进这些度化他的话去。
想你们相好了一场,他又无父母兄弟,只身一人在此,你也该各处找一找才是。
靠那跛足道士疯疯癫癫的,能往那里远去!左不过在这房前左右的庙里寺里躲藏着罢咧。
”\zhu{左不过:反正,只不过,无非。
}薛蟠说:“何尝不是呢。
我一听见这个信儿,就连忙带了小厮们在各处寻找去,连个影儿也没有。
又去问人,人人都说不曾看见。
我因如此,急的没法,唯有望着西北上大哭了一场回来了。”
\ping{为何要“望着西北上”?可能是因为柳湘莲救薛蟠的地方平安州位于西北方位。}
说着,眼圈儿又红上来了。
薛姨妈说:“你既然找寻了没有,把你作朋友的心也尽了。
焉知他这一出家,不是得了好处去呢?你也不必太过虑了。
一则张罗张罗买卖,二则把你自己娶媳妇应办的事情,倒是早些料理料理。
咱们家里没人手儿,竟是‘笨雀儿先飞’,省得临期丢三忘四的不齐全,令人笑话。
再者,你妹妹才说,你也回家半个多月了,想货物也该发完了,同你作买卖去的伙计们,也该设桌酒席请请他们,酬酬劳乏才是。
他们固然是咱家约请的吃工食劳金的人,\zhu{吃工食:靠干活吃饭。
劳金:酬金,工钱。
}到底也算是外客,又陪着你走了一二千里的路程,受了四五个月的辛苦,而且在路上又替你担了多少的惊怕沉重。
”薛蟠闻听,说:“妈说的很是,妹妹想得周到。
我也这样想来着,只因这些日子为各处发货,闹得头晕。
又为柳大哥的亲事又忙了这几日,反倒落了一个空,白张罗了一会子,倒把正经事都误了。
要不然,就定了明儿后儿下帖子请请罢。
”薛姨妈道:“由你办去罢。
”\par
话犹未了,外面小厮回说:“张管总的伙计着人送了两个箱子来,\zhu{管总:掌管一切事务。
}
说这是爷各自买的,\zhu{各自:各方自己;个人自己,这里是第二个意思。
}不在货账里面。
本要早送来,因货物箱子压着,未得拿;昨日货物发完了,所以今儿才送来了。
”一面说,一面又见两个小厮搬进了两个夹板夹的大棕箱来。
薛蟠一见,说:“嗳哟,可是我怎么就糊涂到这一步田地了!特特的给妈和妹妹带来的东西都忘了,没拿了家里来,还是伙计送了来了。
”宝钗说:“亏你才说还是特特的带来的,还是这样放了一二十日才送来,若不是特特的带来,必定是要放到年底下才送进来呢。
你也诸事太不留心了。
”薛蟠笑道:“想是我在路上叫贼人把魂吓掉了,还没归壳呢。
”\par
说着,大家笑了一阵,便向回话的小厮说:“东西收下了,叫他们回去罢。
”薛姨妈同宝钗忙问:“是什么好东西,这样捆着夹着的?”便命人挑了绳子,去了夹板,开了锁看时,却是些绸缎、绫锦、洋货等家常应用之物。
独有宝钗他的那个箱子里,除了笔、墨、砚、各色笺纸、香袋、香珠、扇子、扇坠、花粉、胭脂、头油等物外,还有虎丘带来的自行人、\zhu{虎丘:山名,在江苏省苏州市,有虎丘塔、剑池等名胜古迹。
据东汉赵晔《吴越春秋》载:吴王阖闾葬此,葬三日有白虎踞其上,故名虎丘。
自行人:用发条驱动的机械玩具人。
}酒令儿、\zhu{酒令:古代宴会中,佐饮助兴的游戏。
推一人为令官,其余的人听其号令,轮流说诗词或做其他游戏,违令或输的人饮酒。
这里可能是指行酒令时所玩的游戏道具。
}水银灌的打筋斗的小小子,
\zhu{
该玩具的制作原理与不倒翁相近,灌水银,是因为水银有比重大、易流动的特点,这样,将“小小子”推倒,其体内水银自然顺势流动,是以带动玩偶打起筋头,前后左右,翻转不止,是非常有趣的玩具。
}
沙子灯,\zhu{沙子灯:一种玩具,里面多不能点蜡烛,外壳以薄柳木制成圆形,扁形,状如一层笼屉,正而以厚纸为之,上绘戏曲故事中人物的形象,头部和四肢另外用纸剪成,能来回摆动。
背面糊高丽纸,灯内贮细沙,有弦有斗,井有一个铃铛,略一触动,则沙流铃响,人物便头摇臂摆活动起来。
}一出一出的泥人儿的戏,用青纱罩的匣子装着,又有在虎丘上作的薛蟠的像,泥捏成的与薛蟠毫无相差,以及许多碎小玩意儿的东西。
宝钗一见,满心欢喜,便叫自己使的丫环来吩咐:“你将我的这个箱子与我拿了园子里去,我好就近从那边送送人。
”说着,便起身来,告辞母亲,往园子里来了。
这里薛姨妈将自己这个箱子里的东西取出,一分一分的打点清楚,着同喜丫头送往贾母并王夫人等处去不讲。
\par
且说宝钗随着箱子到了自己房中,将东西逐件逐件的过了目,除将自己留用之外,遂一分一分配合妥当:也有送笔、墨、纸、砚的,也有送香袋、扇子、香坠的,也有送脂粉、头油的,有单送玩意儿的;酌量其人分办。
只有黛玉的比别人不同,比众人加厚一倍。
一一打点完毕,使莺儿同一个老婆子跟着,送往各处。
\par
其李纨、宝玉等以及诸人,不过收了东西,赏赐来使,皆说些见面再谢等语而已。
惟有林黛玉他见江南家乡之物,反自触物伤情,因想起他的父母来了。
便对着这些东西,挥泪自叹,暗想:“我乃江南之人,父母双亡,又无兄弟,只身一人,可怜寄居外祖母家中,而且又多疾病,除外祖母以及舅母、姐妹看问外,那里还有一个姓林的亲人来看望看望,给我带些土物来。
使我送送人,粧粧脸面也好。
\zhu{粧:同“妆”。
}可见人若无至亲骨肉手足,是最寂寞、极冷清、极寒苦,没趣味的!”想到这里,不觉就大伤起心来了。
紫鹃他乃伏侍黛玉多年,朝夕不离左右的,深知黛玉的心腹:他为见了江南故土之物,\zhu{为:因,表示原因。
}因感动了心怀,追思亲人的原故。
但不敢说破,只在一旁劝说道:“姑娘的身子多病,早晚尚服丸药,这两日看着不过比那些日子略饮食好些,精神壮一点儿,还算不得十分大好。
今儿宝姑娘送来这些东西,可见宝姑娘素日看姑娘甚重,姑娘看着该欢喜才是,为什么反倒伤感。
这不是宝姑娘送东西为的是叫姑娘欢喜,这反倒是招姑娘烦恼了不成?若令宝姑娘知道了,怎么脸上下得来呢?再姑娘也要细想一想,老太太、太太们为姑娘的病症千方百计请好大夫诊脉配药调治,所为的是姑娘的病急好。
\zhu{急:迅速的。
}这如今才好些,又这样哭哭啼啼的,岂不是自己糟蹋自己的身子,不肯叫老太太看着欢喜?难道说姑娘这个病,不是因素日从忧虑过度上伤多了气血得的么?姑娘的千金贵体别自己看轻了。
”紫鹃正在这里劝解黛玉,只听见小丫头子在院内说:“宝二爷来了。
”紫鹃忙说:“快请。
”\par
话犹未毕,只见宝玉已进房来了。
黛玉让坐毕,宝玉见黛玉泪痕满面,便问:“妹妹,又是谁得罪了你了?你两眼都哭得红了,是为什么?”黛玉不回答。
旁边紫鹃将嘴向床里一扭,宝玉会意,便往床里一看,见堆着许多东西,就知是宝钗送来的,便笑着取笑说道:“好东西,想是妹妹要开杂货铺么?摆着这些东西作什么?”黛玉只是不理。
紫鹃说:“二爷还提东西呢。
因宝姑娘送了些东西来,我们姑娘一看,就伤心哭起来了。
我正在这里好劝歹劝,总劝不住呢。
而且又是才吃了饭,若只管哭,大发了,再吐了,犯了旧病,可不叫老太太骂死了我们么?倒是二爷来的很好,替我们劝一劝。
”宝玉他本是聪明人,而且一心总留意在黛玉身上最重,所以深知黛玉之为人心细心窄,而又多心要强,不落人后,因见了人家哥哥自江南带了东西来送人,又系故乡之物,勾想起别的痛肠来,是以伤感是实。
\zhu{实:实际,事实。
}这是宝玉他心里揣摩黛玉的心病,却不肯明明说出,恐黛玉越发动情,乃笑道:“你们姑娘的原故不为别的,为的是宝姑娘送来的东西少,所以生气伤心。
妹妹,你放心!等我明年往江南去,与你多多的带两船来,省得你淌眼抹泪的。
”黛玉听了这话,不由“嗤”的一声笑了,忙说道:“我凭他怎么没见过世面,也到不了这一步田地上,因送的东西少,就生气伤心。
我也不是两三岁的小孩子,你也忒把人看得平常小气了。
我有我的原故,你那里知道。
”说着说着,眼泪又流下来了。
宝玉忙移至床上,挨黛玉坐下,将那些东西一件一件的拿起来,摆弄着细瞧,故意问:“这是什么,叫什么名字?那是怎么做的,这样齐整?这是什么,要他做什么使用?妹妹,你瞧,这一件可以摆在书阁儿上作陈设,\zhu{书阁:收藏书籍的地方。
}那件放在条案上当古董儿倒好呢!”\zhu{条案:长条形的高桌子,也称为“条桌”。
}一味的将这些没要紧的话来支吾搭讪了一会,黛玉见宝玉那些呆样子,问东问西的,招人可笑,稍将烦恼丢开,略有些喜笑之意。
\ping{耐心见真情。
}宝玉见他有些喜色,便说道:“宝姐姐送东西来给咱们,我想着,咱们也该到他那里道个谢去才是,不知妹妹可去不去?”黛玉原不愿意为送些东西来就特特的道谢去,不过一时见了,说一声就完了。
今被宝玉说得有理难以推托,无奈只得同宝玉去了。
这且不提。
\par
且说薛蟠听了母亲之言,急忙下请帖,置办酒筵。
张罗了一日,果于次日,三四位伙计,俱各到齐。
未免说了些店内发货、帐目之事毕,列席让坐,薛蟠与各位奉酒酬劳。
里面薛姨妈又着人出来致谢道乏,毕,内有一位问道:“今日席上怎么少柳大哥不出来?\zhu{上一句话其实是两句话的拼凑,第一句是“今日席上怎么少柳大哥?”,第二句是“柳大哥怎么不出来?”。
日常口语中这样的拼接属于很正常的现象,并非错误。
}想是东家忘了,没请么?”薛蟠闻听,把眉一皱,叹了一口气,说道:“休提,休提,想来众位不知深情。
若说起此人,真真可叹!于一二日前,忽被一个疯道士度化的出了家,跟着他去了。
你们众位听一听,可奇不奇?”众人说道:“我们在店内也听见外面人吵嚷,说有一个道士三言两语把一个俗家子弟度了去了,又闻说一阵风刮了去了,又说驾着一片云彩去了,纷纷议论不一。
我们也因发货事忙,那里有工夫当正经事,也没去细问细打听,到如今还是似信不信的。
今听此言,那道士度化的原来就是柳大哥么?早知是他,我们大家也该劝解劝解。
凭他怎么,也不容他去。
嗳,又少了一个有趣儿的好朋友了!实实在在的可惜可叹。
也怨不得东家你心里不爽快。
想他那样一个伶俐人,未必是真跟了道士去罢。
柳大哥他会些武艺,又有力量,或者看破了道士有些什么妖术邪法的破绽出来,故意假跟了他去,在背地里摆布他也未可知。
”薛蟠说:“谁知道,果能如此,倒好罢咧,世上也少一个妖言惑众的人了。
”\ping{在大家心中,道士形象实在不够好。
}众人道:“难道你知道了的时候,也没寻找他去不成?”薛蟠说:“城里城外,那里没有找到!不怕你们笑话,我还哭了一场呢。
”言毕,只是长吁短叹,无精打彩的,不像往日高兴顽笑,让酒畅饮。
席上虽设了些鸡鹅鱼鸭,山珍海味,美品佳肴,怎奈东家皱眉叹气,众伙计看此光景,不便久坐,不过随便喝了几钟酒,吃了些饭食,就都散了。
这也不提。
\par
且说宝玉拉了黛玉至宝钗处来道谢。
彼此见面,未免说几句客言套语。
黛玉便对宝钗说道:“大哥哥辛辛苦苦的能带了多少东西来,搁得住送我们这些处,\ping{上句中的“处”可能是多余的。
}你还剩什么呢?”宝玉说:“可是这话呢。
”宝钗笑道:“东西不是什么好的,不过是远路带来的土物儿,大家看着略觉新鲜似的。
我剩不剩什么要紧,我如今果爱什么,今年虽然不剩,明年我哥哥去时,再叫他给我带些个来,有什么难呢?”宝玉听说,忙笑道:“明年再带了什么来,我们还要姐姐送我们呢。
可别忘了我们!”黛玉说:“你要,你只管说你要,不必拉扯上‘我们’不‘我们’的字眼,姐姐瞧宝哥哥不是给姐姐来道谢,竟是又要定下明年的东西来了。
”宝玉笑说:“我要出来,难道没有你一分儿不成?你不知道帮着说,反倒说起这散话来了。
”\zhu{散话:闲话。
}大家听了,笑了一阵。
宝钗问:“你二人如何来得这样巧,是谁会谁去的?”宝玉说:“休提,我因姐姐送我东西,想来林妹妹也必有,我想要来道谢,想林妹妹也必来道谢,故此我就到他房里会了他一同要到这里来。
谁知到了他家,他正在屋里伤心落泪,也不知是为什么这样爱哭。
”宝玉刚说到“落泪”两字,见黛玉瞪了他一眼,恐他往下还说。
宝玉会意,随即便换过口来说道:“林妹妹这几日因身上不爽快,恐怕又病扳嘴,\zhu{
扳:[bān]扭转;[pān]同“攀”。
扳嘴:可能意为撅嘴、撇嘴。
}故此着急落泪。
我劝解了一会子,才来了。
一则道谢;二则省的叫他一个人在房里坐着只是发闷。
”宝钗说:“妹妹怕病闷,固然是正理,也不过是在那饮食起居、穿脱衣服冷热上加些小心就是了,为什么伤起心来呢?妹妹,你难道不知伤心难免不伤气血精神,把要紧的伤了,反倒要受病的罢咧。
妹妹你细想想。
”黛玉说:“姐姐说的很是。
我何尝自己不知道呢,只因我这几年,姐姐是看见的,那一年不病一两场?病的我怕怕的了。
见了药,吃了见效不见效,一闻见,先就头疼发恶心,怎么不叫我怕病呢?”宝钗说:“虽然如此说,却也不该伤心,倒是觉着身上不爽快,反自己勉强扎挣着出来,\zhu{扎挣:勉强支持。
}各处走走逛逛,把心松散松散,比在屋里闷坐着还强呢。
伤心是自己添病的大毛病。
我那两日不时觉着发懒,浑身乏倦,只是要歪着,心里也是为时气不好,\zhu{为:因,表示原因。
}怕病,因此偏扭着他,寻些事情作作,一般里也混过去了。
妹妹别恼我说,越怕越有鬼。
”宝玉听说,忙问道:“宝姐姐,鬼在那里呢?我怎么看不见一个儿?”惹得众人哄声大笑。
宝钗道:“呆小爷,这是比喻的话,那里真有鬼呢!认真的果有鬼,你又该骇哭了。
”黛玉因此笑道:“姐姐说的很是。
很该说他,谁叫他嘴快!”宝玉说:“有人说我的不是,你就乐了。
你这会子心里也不懊恼了,咱们也该走罢。
”于是彼此又说笑了一回,二人辞了宝钗出来。
宝玉仍把黛玉送至潇湘馆门首,\zhu{门首:门前、门口。
}自己回家。
这且不提。
\par
且说赵姨娘因见宝钗送环哥之物,忙忙接下,心中甚喜,满嘴夸奖:“人人都说宝姑娘会行事,很大方,今日看来,果然不错。
他哥哥能带了多少东西来,他挨家送到,并不遗漏一处,也不露出谁薄谁厚,连我们搭拉嘴子,\zhu{搭拉:下垂的样子。
也作“耷拉”。
搭拉嘴子:形容人生气、失意闭嘴无言的样子。
程甲本作“没时运的”。
}他都想到,实在的可敬。
若是林姑娘——也罢么,也没人给他送东西带什么来;即或有人带了来,他也只是拣着那有势力、有体面的人头儿跟前才送去,那里还临的到我们娘儿们身上呢!可见人会行事,真真的露着各别另样的好。
”赵姨娘因环哥儿得了东西,深为得意,不住的托在掌上摆弄瞧看一会。
想宝钗乃系王夫人之表侄女,\zhu{表侄女:表兄弟的女儿,宝钗是王夫人姊妹薛姨妈的女儿,这里应该是外甥女。
}特要在王夫人跟前卖好儿。
自己叠叠歇歇的拿着那东西,\zhu{叠叠歇歇:程甲本作“蝎蝎螫螫”。
蝎蝎螫螫:用人们对蝎螫的惊恐神情,形容过分的担心、惶恐、大惊小怪。
}走至王夫人房中,站在一旁说道:“这是他宝姑娘才给环哥他兄弟送来的。
他年轻轻的人想的周到,我还给了送东西的小丫头二百钱。
听见说姨太太也给太太送来了,不知是什么东西?你们瞧瞧这一个门里头就是两分儿,能有多少呢?怪不的老太太同太太都夸他疼他,果然招人爱。
”说着,将抱的东西递过去与王夫人瞧,谁知王夫人头也没抬,手也没伸,只口内说了一声“好,给环哥儿玩罢咧”,并无正眼看一看。
赵姨娘因招了一鼻子灰,满肚气恼,无精打彩的回至自己房中,将东西丢在一边,说了许多的劳儿三、巴儿四,
\zhu{
劳儿三、巴儿四:“三、四”是虚词,类似于“颠三倒四”、“说三道四”中的“三、四”。
在传抄中,抄手把底本中的一个字,既可能看错成形状相近的另一个字,也可能为了简写而故意写成读音相同的另一个结构简单的字。
“劳”可能是“唠”的同音简写,而“巴”可能是“叨”首先被看错为形状相似的“叭”然后被同音简写为“巴”。
“劳儿三、巴儿四”删除虚词“三、四、儿”之后,再进行上述修改,得到“唠叨”,即底本上可能是“唠三叨四”,意思是啰嗦不停的样子。
第六十三回:晴雯笑说:“这位奶奶那里吃了一杯来了,唠三叨四的,又排场了我们一顿去了。”
}
不着要的一套闲话;\zhu{
不着要:抓不住重点。
}也无人问他,他却自己咕嘟着嘴,一边子坐着。
可见赵姨娘为人小器糊涂,饶得了东西,\zhu{饶:即使,尽管,表示让步关系。
}反说许多令人不入耳生厌的闲话,也怨不得探春生气,看不起他。
闲话休提。
\par
且说宝钗送东西的丫头回来,说:“也有道谢的,也有赏赐的,独有给巧姐儿的那一分儿,仍旧拿回来了。
”宝钗一见,不知何意,便问:“为什么这一分儿没送去呢,还是送了去没收呢?”莺儿说:“我方才给环哥儿送东西的时候,见琏二奶奶往老太太房里去了。
我想,琏二奶奶不在家,知道交给谁呢,所以没有送去。
”宝钗说:“你也太糊涂了。
二奶奶不在家,难道平儿、丰儿也不在家不成?你只管交给他们收下,等二奶奶回来,自有他们告诉就是了,必定要你当面交给才算么?”莺儿听了,复又拿着东西出了园子,往凤姐处去。
在路上走着,便对拿东西的老婆子说:“早知道一就事儿送了去不完了,省得又跑这一趟。
”老婆子说:“闲着也是白闲着,借此出来逛逛也好罢咧。
只是姑娘你今日来回各处走了好些路儿,想是不惯,乏了,咱们送了这个,可就完了,一打总儿再歇着。
”两人说着话,到了凤姐处,送了东西,回来见宝钗。
\par
宝钗问道:“你见了琏二奶奶没有?”莺儿说:“我没有见。
”宝钗说:“想是二奶奶还没回来么?”丫头说:“回是回来了。
因丰儿对我说:‘二奶奶自老太太屋里回房来,不似往日欢天喜地的,一脸的怒气,叫了平儿去,唧唧咕咕的说话,也不叫人听见。
连我都撵出来了,你不必去见,等我替你回一声儿就是了。
’因此便着丰儿他拿进去,\zhu{着:命令、差使。
}回了出来说:‘二奶奶说,给你们姑娘道生受。
’\zhu{生受:这里是道谢语,难为、有劳的意思。
}赏了我们一吊钱,我就回来了。
”宝钗听了,自己纳了一会子闷,也想不出凤姐是为什么有气。
这也不表。
\par
且说袭人见宝玉回来,便问:“你怎么不逛就回来了?你原说约着林姑娘,你们两个同到宝姑娘处道谢去,可去了没有?”宝玉说:“你别问,我原说是要会林姑娘同去的,谁知到了他家,他在房里守着东西很很的不自在呢。
我也知道林姑娘的那些原原故故的,又不好直问他,又不好说他,只装不知道儿,搭讪着说别的宽解了他一会子,才好了。
然后方拉了他同到了宝姐姐那里道了谢,说了一会子闲话,方散了。
我又送他到家,我才回来了。
”袭人说:“你看送林姑娘的东西,比送你的是多是少,还是一样呢?”宝玉说:“比送我的多着一两倍呢。
”袭人说:“这才是明白人,会行事。
宝姑娘他想别的姊妹等都有亲的热的跟着,有人送东西,唯有林姑娘离家二三千里地远,又无有一个亲人在这里,那有人送东西。
况且他们两个不但是亲戚,还是干姐妹,难道你不知道林姑娘去年曾认过薛姨太太作干妈的?论理多给他些也是该的。
”\par
宝玉笑说:“你就是会评事的一个公道老儿。
”说着话儿,便叫小丫头取了拐枕来,
\zhu{拐枕:像枕头那样的用品,供坐在炕上或床上支靠身体。}
要在床上歪着。
袭人说:“你不出去了?我有一句话告诉你。
”宝玉便问:“什么话?”袭人说:“素日琏二奶奶待我很好,你是知道的。
他自从病了一大场之后,如今又好了。
我早就想着要到那里看看去,只因为琏二爷在家不方便,始终总没有去,闻说琏二爷不在家,你今日又不往那里去,而且初秋天气,不冷不热,一则看二奶奶,尽个礼,省得日后见了受他的数落;二则借此也逛一逛。
你同他们看着家,我去去就来。
”晴雯说:“这却是该的,难得这个巧空儿。
”宝玉说:“我才为他议论宝姑娘,\zhu{为他议论宝姑娘:(因)为他(和)宝姑娘议论,即宝玉和宝钗聊天的话题是袭人。
}夸他是个公道人,这一件事行的,又是一个周到人了。
”袭人笑道:“好小爷,你也不用夸我,你只在家同他们好生玩;好歹别睡觉,看睡出病来,又是我担沉重。
”宝玉说:“我知道了,你只管去罢。
”言毕,袭人遂到自己房里,换了两件新鲜衣服,拿着把儿镜照着,\zhu{把儿镜:即“靶镜”。
靶:柄。
靶镜:带柄的镜子。
}抿了抿头,匀了匀脸上脂粉,步出下房。
复又嘱咐了晴雯、麝月几句话,便出了怡红院。
\par
来至沁芳桥上立住,往四下里观看那园中景致。
时值秋令,秋蝉鸣于树,草虫鸣于野;见这石榴花也开败了,荷叶也将残上来了,倒是芙蓉近着河边,都发了红铺铺的咕嘟子,衬着碧绿的叶儿,倒令人可爱。
一壁里瞧着,一壁里下了桥。
走了不远,迎见李纨房里使唤的丫头素云,跟着个老婆子,手里捧着一个洋漆盒儿走来。
袭人便问:“往那里去?送的是什么东西?”素云说:“这是我们奶奶给三姑娘送去的菱角、鸡头。
”\zhu{鸡头:指鸡头米,芡实之俗称。
芡是一种水生植物,其果仁可食。
}袭人说:“这个东西,还是咱们园子里河内采的,还是外头买来的呢?”素云说:“这是我们房里使唤的刘妈妈,他告假瞧亲戚去带来的,孝敬奶奶。
因三姑娘在我们那里坐着看见了,我们奶奶叫人剥了让他吃。
他说:‘才喝了热茶了,不吃,一会子再吃罢。
’故此给三姑娘送了家去。
”言毕,各自分路走了。
\par
袭人远远的看见那边葡萄架底下,有一个人拿着掸子在那里动手动脚的,因迎着日光,看不真切。
至离得不远,那祝老婆子见了袭人,便笑嘻嘻的迎上来,说道:“姑娘今日怎么得工夫出来闲逛,往那里去?”袭人说:“我那里还得工夫来逛,我往琏二奶奶家瞧瞧去。
你在这里做什么呢?”那祝婆子说:“我在这里赶马蜂呢。
今年三伏里的雨水少,不知怎么,这些果木树上长虫子,把果子吃得巴拉眼睛的,\zhu{巴拉:“疤瘌”的同音简写,意思是疤,疮口或伤口愈合后所留下的痕迹,这里指类似伤疤的痕迹。
“巴拉眼睛”程甲本作“疤瘌流星”,形容粗糙多疤痕的样子。
}掉了好些下来,可惜了儿的白扔了!就是这葡萄,刚成了珠儿,怪好看的,那马蜂、蜜蜂儿满满的围着来蚛,\zhu{蚛:音“众”,虫咬。
}都咬破了。
这还罢了,喜鹊、雀儿,他也来吃这个葡萄。
还有这样一个毛病儿,无论雀儿虫儿,一嘟噜上只咬破三五个,那破的水淌到好的上头,连这一嘟噜都是要烂的。
这些雀儿、马蜂可恶着呢,故此我在这里赶。
姑娘你瞧,咱们说话的空儿没赶,就蚛了许多上来了。
”袭人道:“你就是不住手的赶,也赶不了许多;你刚赶了这里,那里又来了。
倒是告诉买办说,叫他多多的作些冷布口袋来,\zhu{冷布:稀疏透气的纱布。
}
一嘟噜一嘟噜的套上,免得翎禽草虫糟蹋,\zhu{翎[líng]:鸟翅上、尾上的长羽毛。
}而且又透风,捂不坏。
”婆子笑道:“倒是姑娘说的是。
我今年才管上,那里就知道这些巧法儿呢。
”\par
袭人说:“如今这园子里这些果品有好些种,到是那样先熟的快些?”老祝婆子说:“如今才入七月的门,果子都是才红上来,要是好吃,想来还得月尽头儿才熟透了呢。
姑娘不信,我摘一个给姑娘尝尝。
”袭人正色说道:“这那里使得?不但没熟吃不得,就是熟了,一则没有供鲜,\zhu{供鲜:将新下来的瓜果给神佛、祖先上供。
按照规矩,上供之前不能先吃。
}二则主子们尚然没吃,咱们如何先吃得呢?你是这府里的陈人,难道连这个规矩也不晓得么?”老婆子忙笑道:“姑娘说得有理。
我因为姑娘问我,我白这样说。
”\zhu{“我白这样说”这句话程甲本中作“我才敢这么说”。
}心内暗说道:“够了!我方才幸亏是在这里赶马蜂,若是顺着手儿摘一个尝尝,叫他看见,还了得了!”袭人说:“我方才告诉你要口袋的话,你就回一回二奶奶,叫管事的作去罢。
”言毕,遂一直的出了园子的门,就到凤姐这里来了。
\par
正是凤姐与平儿议论贾琏之事。
因见袭人他是轻易不来之人,又不知是有什么事情,便连忙止住话语,勉强带笑说道:“贵人从那阵风儿刮了我们这个贱地来了?”袭人笑说:“我就知道奶奶见了我,是必定要先麻烦我一顿的,我有什么说的呢!但是奶奶欠安,本心惦着要过来请请安,头一件,琏二爷在家不便,二则奶奶在病中,又怕嫌烦,故未敢来。
想奶奶素日疼爱我的那个分儿上,自必是体谅我,再不肯恼我的。
”凤姐笑道:“宝兄弟屋里虽然人多,也就靠着你一个儿照看,也实在的离不开。
我常听见平儿告诉我,说你背地里还惦着我,常问,我听见就喜欢得的什么似的。
今日见了你,我还要给你道谢呢,我还舍得麻烦你吗?我的姑娘!”袭人说:“我的奶奶,若是这样说,这就是真疼我了。
”凤姐拉了袭人的手,让他坐下。
袭人那里肯坐,让之再三,方在挨炕沿脚踏上坐了。
\par
平儿忙自己端了茶来。
袭人说:“你叫小人儿们端罢,劳动姑娘我倒不安。
”一面站起,接过茶来吃着,一面回头看见床沿上放着一个活计簸罗儿,\zhu{
活计:手工制品。
簸罗:即“簸箩”,一种以竹或藤条编制的器具,用来盛物。
}内装着一个大红洋锦的小兜肚,袭人说:“奶奶一天七事八事的,忙的不了,还有工夫作活计么?”凤姐说:“我本来就不会作什么,如今病了才好,又兼着家务事闹个不清,那里还有工夫做这些呢?要紧要紧的我都丢开了。
这是我往老太太屋里请安去,正遇见薛姨太太送老太太这个锦,老太太说:‘这个花红柳绿的,倒对给小孩子们做小衣小裳儿的,\zhu{对:适合,例如“对症下药”。
}穿着倒好顽呢!’因此我就问老祖宗讨了来了。
还惹的老祖宗说了好些顽话,说我是老太太的命中小人,见了什么要什么,见了什么拿什么。
惹得众人都笑了。
你是知道我是脸皮儿厚、不怕说的人,老祖宗只管说,我只管装听不见,拿着就走。
所以才交给平儿,先给巧姐儿做件小兜肚穿着顽,剩下的等消闲有工夫再作别的。
”\zhu{消闲:谓闲暇无事,也指消磨空闲时间。
}\par
袭人听毕,笑道:“也就是奶奶,才能够怄的老祖宗喜欢罢咧。
”伸手拿起来一看,便夸道:“果然好看!各样颜色都有。
好材料也须得这样巧手的人做才对。
况又是巧姐儿他穿的,抱了出去,谁不多看一看。
”又问道:“巧姐儿那里去了?我怎么这半日没见他?”平儿说:“方才宝姑娘那里送了些顽的东西来,他一见了很希罕,就摆弄着顽了好一会子,他奶妈儿才抱了出去,想是乏了,睡觉去了。
”袭人说:“巧姐儿比先前自然越发会顽了。
”平儿说:“小脸蛋子吃得银盆似的,见了人就赶着笑,再不得罪人,真真是我奶奶的解闷的宝贝疙瘩儿。
”凤姐便问:“宝兄弟在家作什么呢?”袭人笑道:“我只求他同晴雯他们看家,我才告了假来了。
可是呢!只顾说话,我也来了好大半天了,要回去了。
别叫宝玉在家里抱怨,说我屁股沉,到那里就坐住了。
”说着,便立起身来告辞,回怡红院来了。
这也不提。
\par
且说凤姐见平儿送出袭人回来,复又把平儿叫入房中,追问前事,越说越气,说道:“二爷在外边偷娶老婆,你说你是听见二门上的小厮们说的。
到底是那一个说的呢?”平儿说:“是旺儿他说的。
”凤姐便命人把旺儿叫来,问道:“你二爷在外边买房子娶小老婆,你知道么?”旺儿说:“小的终日在二门上听差,如何知道二爷的事,这是听见兴儿告诉的。
”凤姐说:“兴儿是几时告诉你的?”旺儿说:“还是二爷没起身的头里告诉我的。
”凤姐又问:“兴儿在那里呢?”旺儿说:“兴儿在新二奶奶那里呢。
”凤姐闻听,满腔怒气,啐了一口,骂道:“下作猴儿崽子!什么是‘新奶奶’、‘旧奶奶’,你就私自封了奶奶了?满嘴里胡说,这就该打嘴巴。
”又问:“兴儿他是跟二爷的人,他怎么没有跟了二爷去呢?”旺儿说:“特留下他在家里照看尤二姐,故此未曾跟了去。
”凤姐听说,忙得一叠连声命旺儿:“快把兴儿叫了来!”\par
旺儿忙忙的跑了出去,见了兴儿只说:“二奶奶叫你呢。
”兴儿正在外边同小人儿们顽笑,听见叫他,妙在也不问旺儿“二奶奶叫我做什么”,便跟了旺儿,急急忙忙的来至二门前。
回明进去,见了凤姐,请了安,旁边侍立。
凤姐一见,便先瞪了两眼,问道:“你们主子奴才在外面干的好事!你们打量我是呆瓜,不知道?你是紧跟二爷的人,自必深知根由。
你须细细的对我实说,稍有一些儿隐瞒撒谎,我将你的腿打折了!”兴儿忙跪下磕头,说:“奶奶问的是什么事,是我同爷干的?”凤姐骂道:“好小杂种!你还敢来支吾我?我问你,二爷在外边,怎么就说成了尤二姐?怎么买房子、治家伙?怎么娶了过来?一五一十的说个明白,饶你的狗命!”\par
兴儿听说,仔细想了一想:“此事二府皆知,就是瞒着老爷、太太、老太太同二奶奶不知道,终久也是要知道的。
我如今何苦来瞒着,不如告诉了他,省得挨眼前打,受委屈。
”再兴儿一则年幼,不知事的轻重;二则素日又知道凤姐是个烈口子,连二爷还惧怕他五分;三则此事原是二爷同珍大爷、蓉哥他叔侄弟兄商量着办的,与自己无干。
故此把主意想定,壮着胆子,跪下说道:“奶奶别生气,等奴才回禀奶奶听:只因那府里的大老爷的丧事上穿孝,不知二爷怎么看见过尤二姐几次,大约就看中了,动了要说的心。
故此先同蓉哥商议,求蓉哥替二爷从中调停办理,作了媒人说合,事成之后,还许下谢候的礼。
\zhu{谢候:答报致谢并加问候的意思。
}蓉哥满应,\zhu{满应:满应满许,完全答应。
}将此话转告诉了珍大爷;珍大爷告诉了珍大奶奶和尤老娘。
尤老娘很愿意,但说是:‘二姐从小儿已许过张家为媳,如何又许二爷呢?恐张家知道,生出事来不妥当。
’珍大爷笑道:‘这算什么大事,交给我!便说那张姓的小子,本是个穷苦破落户,那里见得多给他几两银子,叫他写张退亲的休书,就完了。
’后来,果然找了姓张的来,如此说明,写了休书,给了银子去了。
二爷闻知,才放心大胆的说定了。
又恐怕奶奶知道,拦挡不依,所以在外边咱们后身儿买了几间房子,治了东西,就娶过来了。
珍大爷还给了两口人使唤。
二爷时常推说给老爷办事,又说给珍大爷张罗事,都是些支吾的谎话,竟是在外头住着。
从前原是娘儿三个住着,还要商量给尤三姐说人家,又许下厚聘嫁他;如今尤三姐也死了,只剩下尤老娘跟着尤二姐住着作伴儿呢。
这是一往从前的实话,并不敢隐瞒一句。
”说毕,复又磕头。
\par
凤姐听了这一篇言词,只气得痴呆了半天,面如金纸,\zhu{面如金纸:脸色像金纸一样毫无血色。
形容极为愤怒或恐惧。
金纸是做冥币用的纸。
}两只吊稍子眼越发直竖起来了,\zhu{吊稍子眼:即第三回王熙凤出场时的妆容“柳叶吊梢眉”,形容眉梢斜飞入鬓的样子。
}浑身乱战。
\zhu{战:通“颤”,发抖。}
半晌,连话也说不上来,只是发怔。
猛一低头,见兴儿在地下跪着,便说道:“这也没你的大不是,但只是二爷在外边行这样的事,你也该早些告诉我才是。
这却很该打,因你肯实说,不撒谎,且饶恕你这一次。
”兴儿说:“未能早回奶奶,这是奴才该死!”便叩头有声。
凤姐说:“你去罢。
”兴儿才立起身要走,凤姐又说:“叫你时,须要快来,不可远去。
”兴儿连连答应了几个“是”,就出去了。
到外面伸了伸舌头,说:“够了我的了,差一差儿没有捱一顿好打。
”\zhu{捱:同“挨”。
}暗自后悔不该告诉旺儿,又愁二爷回来怎么见,各自害怕。
这也不提。
\par
且说凤姐见兴儿出去,回头向平儿说:“方才兴儿说的话,你都听见了没有?”平儿说:“我都听见了。
”凤姐说:“天下那有这样没脸的男人!吃着碗里,看着锅里,见一个,爱一个,真成了喂不饱的狗,实在的是个弃旧迎新的坏货。
只是可惜这五六品的顶戴给他!他别想着俗语说的‘家花那有野花香’的话,他要信了这个话,可就大错了。
多早晚在外面闹一个很没脸、亲戚朋友见不得的事出来,他才罢手呢!”平儿一旁劝道:“奶奶生气,却是该的。
但奶奶身子才好了,也不可过于气恼。
看二爷自从鲍二的女人那一件事之后,倒很收了心,好了呢,如今为什么又干起这样事来?这都是珍大爷他的不是。
”凤姐说:“珍大爷固然有不是,也总因咱们那位下作不堪的爷他眼馋,人家才引诱他罢咧。
俗语说的‘牛不吃水,也强按头么?’”平儿说:“珍大爷干这样事,珍大奶奶也该拦着不依才是。
”凤姐说:“可是这话咧!珍大奶奶也不想一想,把一个妹子要许几家子弟才好,先许了姓张的,今又嫁了姓贾的;天下的男人都死绝了,都嫁了贾家来!难道贾家的衣饭这样好不成?这不是说幸而那一个没脸的尤三姐知道好歹,早早儿的死了,若是不死,将来不是嫁宝玉,就是嫁环哥儿呢。
总也不给那妹子留一些儿体面,叫妹子日后怎么抬头竖脸的见人呢?妹子好歹也罢咧!那妹子本来也不是他亲的,而且听见说原是个混帐烂桃。
难道珍大奶奶现做着命妇,家中有这样一个打嘴现世的妹子,\zhu{打嘴现世:说嘴打嘴,现世现报。
说嘴打嘴:才夸口就出丑,这里泛指丢脸。
}也不知道羞臊,躲避着些,反到大面儿上扬名打鼓的,在这门里丢丑,也不怕人笑话么?\ping{珍大奶奶作为填房,实在没地位,无论秦可卿的事儿还是尤氏姐妹的事儿,都只能捏着鼻子忍着。
}再者,珍大爷也是作官的人,别的律例不知道也罢了,连个服中娶亲、停妻再娶使不得的规矩,\zhu{服:丧服。
服中指的是贾敬丧期。
停妻再娶:抛弃未离异的妻子而再娶。
第六十五回:“那贾琏越看越爱,越瞧越喜,不知怎生奉承这二姐,乃命鲍二等人不许提三说二的,直以奶奶称之,自己也称奶奶,竟将凤姐一笔勾倒。”贾琏把尤二姐当作正妻对待,而非一般的妾,所以是“停妻再娶”。
}他也不知道不成?你替他细想一想,他干的这件事,是疼兄弟,还是害兄弟呢?”平儿说:“珍大爷只顾眼前,叫兄弟喜欢,也不管日后的轻重干系了。
”凤姐儿冷笑道:“这是什么‘叫兄弟喜欢’,这是给他毒药吃呢!若论亲叔伯弟兄中,他年纪又最大,又居长,不知教导兄弟学好,反引诱兄弟学不长进,担罪名儿,日后闹出事来,他在一边缸沿儿上站着看热闹,真真我要骂也骂不出口来。
再者,他那边府里的丑事坏名儿,已经叫人听不上了,必定也叫兄弟学他一样,才好显不出他的丑来。
这是什么作哥哥的道理?倒不如撒泡尿浸死了,替大老爷死了倒罢咧,活着作什么呢!你瞧东府里大老爷那样厚德,吃斋念佛行善,怎么反得了这样一个儿子孙子?大概是好风水都叫他老人家一个人全拔尽了。
”平儿说:“想来不错。
若不然,怎么这样差着格儿呢?”凤姐说:“这件事幸而老太太、老爷、太太不知道,倘或吹到这几位耳朵里去,不但咱们那没出息的二爷捱打受骂,就是珍大爷和珍大奶奶也保不住要吃不了要兜着走呢!”连说带詈,\zhu{詈:音“立”,骂。
}直闹了半天,连午饭也推头疼,没过去吃。
\par
平儿看此光景越说越气,劝道:“奶奶也煞一煞气,事从缓来,等二爷回来,慢慢的再商量就是了。
”凤姐听了此言,便从鼻孔内哼了两声,冷笑道:“好罢咧,等爷回来,可就迟了!”平儿便跪在地下,再三苦劝,安慰了一会子,凤姐才略消了些气恼。
喝了口茶,喘息了良久,便要了拐枕,歪在床上,闭着眼睛打主意。
平儿见凤姐儿躺着,方退出去。
偏有不懂眼的几起子回事的人来,都被丰儿撵出去了。
又有贾母处着玛瑙来问:“二奶奶为什么不吃饭?老太太不放心,着我来瞧来了。
”凤姐知是贾母处打发人来,遂勉强起来,说:“我白有些头疼,\zhu{白:单单,只是。
}并没别的病,请老太太放心。
我已经躺了一躺儿,好了。
”言毕,打发来人去后,却自己一个人将前事从头至尾细细的盘算多时,得了一个“一计害三贤”的狠主意出来。
\zhu{一计害三贤:《三国演义》第一百一十九回,邓艾钟会灭蜀,姜维诈降钟会,促使钟会邓艾相斗,密谋复兴蜀国,在动乱中三人都死了。
}自己暗想:须得如此如此方妥。
主意已定,也不告诉平儿,反外面作出嘻笑自若、无事的光景,并不露出恼恨妒嫉之意。
\par
于是叫丫头传了来旺来吩咐,令他明日传唤匠役人等,收拾东厢房,裱糊铺设等语。
平儿与众人皆不知为何缘故。
要知端的,且看下回分解。
\dai{133}{馈土物颦卿念故里}
\dai{134}{讯家童凤姐蓄阴谋}