\chapter{辱亲女愚妾争闲气 \quad 欺幼主刁奴蓄险心}
\qi{此回接上文,恰似黄钟大吕后,转出羽调商声,\zhu{
元·周德清《中原音韵》及明代朱权《太和正音谱·词林须知》对音乐的调性作了规定和分析,
如黄钟宫——高贵澄净……商调——凄怆怨恭。
恰似黄钟大吕后,转出羽调商声:以音乐调式的切换,比喻文章风格的切换。
}别有清凉滋味。
}\par
且说元宵已过,只因当今以孝治天下,目下宫中有一位太妃欠安,\zhu{目下:目前,近来。
}
故各嫔妃皆为之减膳谢妆,不独不能省亲,亦且将宴乐俱免。
故荣府今岁元宵亦无灯谜之集。
\par
刚将年事忙过,凤姐儿便小月了,\zhu{小月:即小产、流产。
}在家一月,不能理事,天天两三个太医用药。
凤姐儿自恃强壮,虽不出门,然筹画计算,想起什么事来,便命平儿去回王夫人,任人谏劝,他只不听。
王夫人便觉失了膀臂,一人能有许多的精神?凡有了大事,自己主张;将家中琐碎之事,一应都暂令李纨协理。
\ping{作为王夫人大儿媳妇的李纨,很可能是王熙凤这个管家的前任,只不过贾珠早夭,李纨不得不退居守节。
}李纨是个尚德不尚才的,未免逞纵了下人。
王夫人便命探春合同李纨裁处,只说过了一月,凤姐将息好了,\zhu{将息:养息,休息。
}仍交与他。
谁知凤姐禀赋气血不足,兼年幼不知保养,平生争强斗智,心力更亏,故虽系小月,竟着实亏虚下来,一月之后,复添了下红之症。
\zhu{下红之症:阴道非正常出血。
}他虽不肯说出来,众人看他面目黄瘦,便知失于调养。
王夫人只令他好生服药调养,不令他操心。
他自己也怕成了大症,遗笑于人,便想偷空调养,恨不得一时复旧如常。
谁知一直服药调养到八九月间,才渐渐的起复过来,下红也渐渐止了。
此是后话。
\ping{上一回凤姐讲了一个“放爆竹散了”的不吉利的笑话,果然本回形势急转直下,从过年的盛况转入丧事和病情。
有一种说法,红楼梦全书如同四季分为春夏秋冬四个章节,每个章节有二十七回。第五十四回是全盛的夏篇章的结束,第五十五回是萧瑟的秋篇章的开始。红楼梦八十回后遗失的部分即为冬篇章。
}\par
如今且说目今王夫人见他如此,\zhu{目今:现在,当前。
}探春与李纨暂难谢事,\zhu{谢:辞去。
}园中人多,又恐失于照管,因又特请了宝钗来,托他各处小心,“老婆子们不中用,得空儿吃酒斗牌,白日里睡觉,夜里斗牌,我都知道的。
凤丫头在外头,他们还有个惧怕,如今他们又该取便了。
好孩子,你还是个妥当人,你兄弟妹妹们又小,我又没工夫,你替我辛苦两天,照看照看。
凡有想不到的事,你来告诉我,别等老太太问出来,我没话回。
那些人不好,你只管说。
他们不听,你来回我。
别弄出大事来才好。
”宝钗听说只得答应了。
\par
时届孟春,\zhu{孟仲叔季:兄弟姊妹的长幼顺序,“孟”为最长,“季”为最幼。
“孟”也作“伯”。
孟春:春季的第一个月,即农历正月。
仲春:春季的第二个月,即农历二月。
季春:春季的第三个月,即农历三月。
}黛玉又犯了嗽疾。
湘云亦因时气所感,亦卧病于蘅芜苑,一天医药不断。
\ping{本回太妃薨了,凤姐病了,黛玉和湘云也病了,这时气其实是作者给贾府的衰气吧。
}探春同李纨相住间隔,二人近日同事,不比往年,来往回话人等亦不便,故二人议定:每日早晨皆到园门口南边的三间小花厅上去会齐办事,吃过早饭于午错方回房。
\zhu{午错:正午已过。}
这三间厅原系预备省亲之时众执事太监起坐之处,故省亲之后也用不着了,每日只有婆子们上夜。
如今天已和暖,不用十分修饰,只不过略略的铺陈了,便可他二人起坐。
这厅上也有一匾,题着“辅仁谕德”四字,\zhu{辅仁谕德:辅:补益。
谕:晓谕。
此匾意谓对己要常补仁爱之不足,对人应宣谕良好的德性,这是旧时官僚士大夫的自谦自勉之辞。
}家下俗呼皆只叫议事厅儿。
如今他二人每日卯正至此,午正方散。
\zhu{卯正:早上六点。午正:中午十二点。}
凡一应执事媳妇等来往回话者,络绎不绝。
\par
众人先听见李纨独办,各各心中暗喜,以为李纨素日原是个厚道多恩无罚的,自然比凤姐儿好搪塞。
便添了一个探春,也都想着不过是个未出闺阁的年轻小姐,且素日也最平和恬淡,因此都不在意,比凤姐儿前更懈怠了许多。
只三四日后,几件事过手,渐觉探春精细处不让凤姐,只不过是言语安静,性情和顺而已。
\geng{这是小姐身份耳,阿凤未出阁想亦如此。
}可巧连日有王公侯伯世袭官员十几处,皆系荣宁非亲即友或世交之家,或有升迁,或有黜降,或有婚丧红白等事,王夫人贺吊迎送,应酬不暇,前边更无人。
他二人便一日皆在厅上起坐。
宝钗便一日在上房监察,至王夫人回方散。
每于夜间针线暇时,临寝之先,坐了小轿带领园中上夜人等各处巡察一次。
他三人如此一理,更觉比凤姐儿当差时倒更谨慎了些。
因而里外下人都暗中抱怨说:“刚刚的倒了一个‘巡海夜叉’,又添了三个‘镇山太岁’,\zhu{巡海夜叉、镇山太岁:指担当巡逻和守卫职责的恶鬼凶神。
夜叉:一名“药叉”,吃人的恶鬼。
太岁:传说中的凶神;也用来比喻凶恶残暴的人。
}越性连夜里偷着吃酒顽的工夫都没了。
”\par
这日王夫人正是往锦乡侯府去赴席,李纨与探春早已梳洗,伺候出门去后,回至厅上坐了。
刚吃茶时,只见吴新登的媳妇进来回说:“赵姨娘的兄弟赵国基昨日死了。
昨日回过太太,太太说知道了,叫回姑娘奶奶来。
”说毕,便垂手旁侍,再不言语。
彼时来回话者不少,都打听他二人办事如何:若办得妥当,大家则安个畏惧之心;若少有嫌隙不当之处,不但不畏伏,出二门还要编出许多笑话来取笑。
吴新登的媳妇心中已有主意,若是凤姐前,他便早已献勤说出许多主意,又查出许多旧例来任凤姐儿拣择施行。
\geng{可知虽有才干,亦必有羽翼方可。
}如今他藐视李纨老实,探春是青年的姑娘,所以只说出这一句话来,试他二人有何主见。
探春便问李纨。
李纨想了一想,便道:“前儿袭人的妈死了,听见说赏银四十两。
这也赏他四十两罢了。
”吴新登家的听了,忙答应了是,接了对牌就走。
\ping{探春如果刚开始管家就多赏自己舅舅丧葬费,那么会无法在贾府树立公正的管理者形象,下人会更加轻侮怠慢。}
探春道:“你且回来。
”吴新登家的只得回来。
探春道:“你且别支银子。
我且问你:那几年老太太屋里的几位老姨奶奶,也有家里的也有外头的这两个分别。
\zhu{家里的:父母是家里的奴婢。
外头的:父母是自由身。
}家里的若死了人是赏多少,外头的死了人是赏多少,你且说两个我们听听。
”一问,吴新登家的便都忘了,忙陪笑回说:“这也不是什么大事,赏多少谁还敢争不成?”探春笑道:“这话胡闹。
依我说,赏一百倒好。
若不按例,别说你们笑话,明儿也难见你二奶奶。
”吴新登家的笑道:“既这么说,我查旧帐去,此时却记不得。
”探春笑道:“你办事办老了的,还记不得,倒来难我们。
你素日回你二奶奶也现查去?若有这道理,凤姐姐还不算利害,也就是算宽厚了!还不快找了来我瞧。
再迟一日,不说你们粗心,反像我们没主意了。
”吴新登家的满面通红,忙转身出来。
众媳妇们都伸舌头,这里又回别的事。
\par
一时,吴家的取了旧账来。
探春看时,两个家里的赏过皆二十两,两个外头的皆赏过四十两。
外还有两个外头的,一个赏过一百两,一个赏过六十两。
这两笔底下皆有原故:一个是隔省迁父母之柩,外赏六十两;一个是现买葬地,外赏二十两。
探春便递与李纨看了。
探春便说:“给他二十两银子。
把这帐留下,我们细看看。
”吴新登家的去了。
\par
忽见赵姨娘进来,李纨探春忙让坐。
赵姨娘开口便说道:“这屋里的人都踩下我的头去还罢了。
姑娘你也想一想,该替我出气才是。
”一面说,一面眼泪鼻涕哭起来。
探春忙道:“姨娘这话说谁,我竟不解。
谁踩姨娘的头?说出来我替姨娘出气。
”赵姨娘道:“姑娘现踩我,我告诉谁!”探春听说,忙站起来,说道:“我并不敢。
”李纨也站起来劝。
\par
赵姨娘道:“你们请坐下,听我说。
我这屋里熬油似的熬了这么大年纪,又有你和你兄弟,这会子连袭人都不如了,我还有什么脸?连你也没脸面,别说我了!”探春笑道:“原来为这个。
我说我并不敢犯法违理。
”一面便坐了,拿帐翻与赵姨娘看,又念与他听,又说道:“这是祖宗手里旧规矩,人人都依着,偏我改了不成?也不但袭人,将来环儿收了外头的,自然也是同袭人一样。
这原不是什么争大争小的事,讲不到有脸没脸的话上。
他是太太的奴才,我是按着旧规矩办。
说办的好,领祖宗的恩典、太太的恩典;若说办的不均,那是他糊涂不知福,也只好凭他抱怨去。
太太连房子赏了人,我有什么有脸之处;一文不赏,我也没什么没脸之处。
依我说,太太不在家,姨娘安静些养神罢了,何苦只要操心。
太太满心疼我,因姨娘每每生事,几次寒心。
我但凡是个男人,可以出得去,我必早走了,立一番事业,那时自有我一番道理。
偏我是女孩儿家,一句多话也没有我乱说的。
太太满心里都知道。
如今因看重我,才叫我照管家务,还没有做一件好事,姨娘倒先来作践我。
倘或太太知道了,怕我为难不叫我管,那才正经没脸,连姨娘也真没脸!”一面说,一面不禁滚下泪来。
\par
赵姨娘没了别话答对,便说道:“太太疼你,你越发拉扯拉扯我们。
你只顾讨太太的疼,就把我们忘了。
”探春道:“我怎么忘了?叫我怎么拉扯?这也问你们各人,那一个主子不疼出力得用的人?那一个好人用人拉扯的?”李纨在旁只管劝说:“姨娘别生气。
也怨不得姑娘,他满心里要拉扯,口里怎么说的出来。
”探春忙道:“这大嫂子也糊涂了。
我拉扯谁?谁家姑娘们拉扯奴才了?他们的好歹,你们该知道,与我什么相干。
”\ping{李纨帮倒忙,触动了探春心中关于母亲是奴才身份的自卑,不得不和母亲划清界限。
}赵姨娘气的问道:“谁叫你拉扯别人去了?你不当家我也不来问你。
你如今现说一是一,说二是二。
如今你舅舅死了,你多给了二三十两银子,难道太太就不依你?分明太太是好太太,都是你们尖酸刻薄,可惜太太有恩无处使。
姑娘放心,这也使不着你的银子。
明儿等出了阁,我还想你额外照看赵家呢。
如今没有长羽毛,就忘了根本,只拣高枝儿飞去了!”\par
探春没听完,已气的脸白气噎,抽抽咽咽的一面哭,一面问道:“谁是我舅舅?我舅舅年下才升了九省检点,那里又跑出一个舅舅来?我倒素习按理尊敬,越发敬出这些亲戚来了。
既这么说,每日环儿出去,为什么赵国基又站起来,又跟他上学?为什么不拿出舅舅的款来?何苦来,谁不知道我是姨娘养的,必要过两三个月寻出由头来,彻底来翻腾一阵,生怕人不知道,故意的表白表白。
也不知谁给谁没脸?幸亏我还明白,但凡糊涂不知理的,早急了。
”李纨急的只管劝,赵姨娘只管还唠叨。
\par
忽听有人说:“二奶奶打发平姑娘说话来了。
”赵姨娘听说,方把口止住。
只见平儿进来,赵姨娘忙陪笑让坐,又忙问:“你奶奶好些?我正要瞧去,就只没得空儿。
”李纨见平儿进来,因问他来做什么。
平儿笑道:“奶奶说,赵姨奶奶的兄弟没了,恐怕奶奶和姑娘不知有旧例,若照常例,只得二十两。
如今请姑娘裁夺着,再添些也使得。
”探春早已拭去泪痕,忙说道:“又好好的添什么,谁又是二十四个月养下来的?\zhu{养:这里指生孩子。
}
不然也是那出兵放马背着主子逃出命来过的人不成?你主子真个倒巧,叫我开了例,他做好人,拿着太太不心疼的钱,乐的做人情。
你告诉他,我不敢添减,混出主意。
他添他施恩,等他好了出来,爱怎么添了去。
”平儿一来时已明白了对半,今听这一番话,越发会意,见探春有怒色,便不敢以往日喜乐之时相待,只一边垂手默侍。
\par
时值宝钗也从上房中来,探春等忙起身让坐。
未及开言,又有一个媳妇进来回事。
因探春才哭了,便有三四个小丫鬟捧了沐盆、巾帕、靶镜等物来。
此时探春因盘膝坐在矮板榻上,\zhu{因:副词。
于是,就。
板榻:木板所制狭长而较矮的坐卧之具。
}那捧盆的丫鬟走至跟前,便双膝跪下,高捧沐盆;那两个小丫鬟,也都在旁屈膝捧着巾帕并靶镜脂粉之饰。
\zhu{靶:柄。
靶镜:带柄的镜子。
}
平儿见待书不在这里,便忙上来与探春挽袖卸镯,又接过一条大手巾来,将探春面前衣襟掩了。
探春方伸手向面盆中盥沐。
那媳妇便回道:“回奶奶姑娘,家学里支环爷和兰哥儿的一年公费。
”平儿先道:“你忙什么!你睁着眼看见姑娘洗脸,你不出去伺候着,先说话来。
二奶奶跟前你也这么没眼色来着?姑娘虽然恩宽,我去回了二奶奶,只说你们眼里都没姑娘,你们都吃了亏,可别怨我。
”唬的那个媳妇忙陪笑道:“我粗心了。
”一面说,一面忙退出去。
\ping{平儿看出来探春被藐视,自己作为有权势的凤姐的心腹,主动低眉顺眼伺候探春,并呵斥那些胆大妄为的奴才,给探春把排场撑起来了,实际上巩固了探春管家的地位。
}\par
探春一面匀脸,一面向平儿冷笑道:“你迟了一步,还有可笑的:连吴姐姐这么个办老了事的,也不查清楚了,就来混我们。
幸亏我们问他,他竟有脸说忘了。
我说他回你主子事也忘了再找去?我料着你那主子未必有耐性儿等他去找。
”平儿忙笑道:“他有这一次,管包腿上的筋早折了两根。
姑娘别信他们。
那是他们瞅着大奶奶是个菩萨,姑娘又是个腼腆小姐,固然是托懒来混。
”说着,又向门外说道:“你们只管撒野,等奶奶大安了,咱们再说。
”门外的众媳妇都笑道:“姑娘,你是个最明白的人,俗语说,‘一人作罪一人当’,我们并不敢欺蔽小姐。
如今小姐是娇客,\zhu{娇客:旧俗女婿或女儿都可称娇客,这里指探春。
}若认真惹恼了,死无葬身之地。
”平儿冷笑道:“你们明白就好了。
”又陪笑向探春道:“姑娘知道二奶奶本来事多,那里照看的这些,保不住不忽略。
俗语说‘旁观者清’,这几年姑娘冷眼看着,或有该添该减的去处二奶奶没行到,姑娘竟一添减,头一件于太太的事有益,第二件也不枉姑娘待我们奶奶的情义了。
”话未说完,宝钗李纨皆笑道:“好丫头,真怨不得凤丫头偏疼他!本来无可添减的事,如今听你一说,倒要找出两件来斟酌斟酌,不辜负你这话。
”探春笑道:“我一肚子气,没人煞性子,正要拿他奶奶出气去,偏他碰了来,说了这些话,叫我也没了主意了。
”一面说,一面叫进方才那媳妇来问:“环爷和兰哥儿家学里这一年的银子,是做那一项用的?”那媳妇便回说:“一年学里吃点心或者买纸笔,每位有八两银子的使用。
”探春道:“凡爷们的使用,都是各屋领了月钱的。
环哥的是姨娘领二两,宝玉的是老太太屋里袭人领二两,兰哥儿的是大奶奶屋里领。
怎么学里每人又多这八两?原来上学去的是为这八两银子!从今儿起,把这一项蠲了。
平儿,回去告诉你奶奶,说我的话,把这一条务必免了。
”平儿笑道:“早就该免。
旧年奶奶原说要免的,因年下忙,就忘了。
”那个媳妇只得答应着去了。
就有大观园中媳妇捧了饭盒来。
\par
待书素云早已抬过一张小饭桌来,平儿也忙着上菜。
探春笑道:“你说完了话干你的去罢,在这里忙什么。
”平儿笑道:“我原没事的。
二奶奶打发了我来,一则说话,二则恐这里人不方便,原是叫我帮着妹妹们伏侍奶奶姑娘的。
”探春因问:“宝姑娘的饭怎么不端来一处吃?”丫鬟们听说,忙出至檐外命媳妇去说:“宝姑娘如今在厅上一处吃,叫他们把饭送了这里来。
”探春听说,便高声说道:“你别混支使人!那都是办大事的管家娘子们,你们支使他要饭要茶的,连个高低都不知道!平儿这里站着,你叫叫去。
”\ping{对于戏弄自己的管家娘子们,探春心里含怨气,但是管家的时候也不得不依赖于她们,所以也不能彻底闹掰,所以和丫鬟一唱一和,既要出气,也不能大伤和气。
丫鬟首先唱黑脸给了管家娘子一巴掌,使得管家娘子受辱若惊。
丫鬟命她们干低级奴仆的活送饭来,挑战了她们管家的地位,提醒她们自己不过是主子的仆人,不要妄图挑衅。
探春紧接着唱红脸给挨打的管家娘子揉了揉,使得管家娘子受宠若惊。
探春说她们是“办大事”的人,不该去要饭要茶,是承认了她们在下人中的特殊地位,有拉拢缓和关系之意。
宠辱若惊,得之若惊,失之若惊,荣辱皆系主之所赐,可以予之,亦可夺之,失去主人的庇护,小丫鬟也可以随时可以夺去其管家之位贬其为干杂活的奴才。
探春最后说小丫鬟“高低都不知道”,也是对管家娘子说的。
小丫鬟不知道奴才辈里面的高低等级,错误地把高级奴才当作低级奴才使用,是不对的;管家娘子不知道主仆之间的高低等级,错误地把身为奴仆的自己也看做主人,甚至欺凌主人,是更不对的。
}\par
平儿忙答应了一声出来。
那些媳妇们都忙悄悄的拉住笑道:“那里用姑娘去叫,我们已有人叫去了。
”一面说,一面用手帕掸石矶上说:“姑娘站了半天乏了,这太阳影里且歇歇。
”平儿便坐下。
又有茶房里的两个婆子拿了个坐褥铺下,说:“石头冷,这是极干净的,姑娘将就坐一坐儿罢。
”平儿忙陪笑道:“多谢。
”一个又捧了一碗精致新茶出来,也悄悄笑说:“这不是我们的常用茶,原是伺候姑娘们的,姑娘且润一润罢。
”平儿忙欠身接了,因指众媳妇悄悄说道:“你们太闹的不像了。
\zhu{不像:指言行超越常轨,不成话。}
他是个姑娘家,不肯发威动怒,这是他尊重,你们就藐视欺负他。
果然招他动了大气,不过说他一个粗糙就完了,你们就现吃不了的亏。
他撒个娇,太太也得让他一二分,二奶奶也不敢怎样。
你们就这么大胆子小看他,可是鸡蛋往石头上碰。
”众人都忙道:“我们何尝敢大胆了,都是赵姨奶奶闹的。
”平儿也悄悄的说:“罢了,好奶奶们。
‘墙倒众人推’,那赵姨奶奶原有些倒三不着两,\zhu{倒三不着两:即“到三不着两”,也作“着三不着两”、“道三不着两”,谓说话或行事轻重失宜,考虑不周,注意这里,顾不到那里。
}有了事都就赖他。
你们素日那眼里没人,心术利害,我这几年难道还不知道?二奶奶若是略差一点儿的,早被你们这些奶奶治倒了。
饶这么着,\zhu{饶:即使,尽管,表示让步关系。
}得一点空儿,还要难他一难,好几次没落了你们的口声。
\zhu{口声:即口实、话柄。
}众人都道他利害,你们都怕他,惟我知道他心里也就不算不怕你们呢。
前儿我们还议论到这里,再不能依头顺尾,必有两场气生。
那三姑娘虽是个姑娘,你们都横看了他。
\zhu{横看:错看。
}二奶奶这些大姑子小姑子里头,也就只单畏他五分。
你们这会子倒不把他放在眼里了。
”\ping{平儿作为有权势的凤姐的心腹,在探春面前低眉顺眼,跑前跑后,算是给那些桀骜不驯的管家娘子一个榜样,平儿出去给她们好好上了一课,树立探春管家的威信。
}\par
正说着,只见秋纹走来。
众媳妇忙赶着问好,又说:“姑娘也且歇一歇,里头摆饭呢。
等撤下饭桌子,再回话去。
”秋纹笑道:“我比不得你们,我那里等得。
”说着便直要上厅去。
平儿忙叫:“快回来。
”秋纹回头见了平儿,笑道:“你又在这里充什么外围的防护?”一面回身便坐在平儿褥上。
\ping{秋纹是宝玉的丫鬟,更有威风。}
平儿悄问:“回什么?”秋纹道:“问一问宝玉的月银我们的月钱多早晚才领。
”平儿道:“这什么大事。
你快回去告诉袭人,说我的话,凭有什么事今儿都别回。
若回一件,管驳一件;回一百件,管驳一百件。
”秋纹听了,忙问:“这是为什么了?”平儿与众媳妇等都忙告诉他原故,又说:“正要找几件利害事与有体面的人开例作法子,\zhu{作法:就是树立某种标准,给别人立规矩,通过责骂、惩罚等手段处理某人立威,杀鸡儆猴,以儆其馀。
}镇压与众人作榜样呢。
何苦你们先来碰在这钉子上。
你这一去说了,他们若拿你们也作一二件榜样,又碍着老太太、太太;若不拿着你们作一二件,人家又说偏一个向一个,仗着老太太、太太威势的就怕,也不敢动,只拿着软的作鼻子头。
\zhu{鼻子头:开头第一个,这里是开例以儆众人的意思。
}你听听罢,二奶奶的事,他还要驳两件,才压的众人口声呢。
”秋纹听了,伸舌笑道:“幸而平姐姐在这里,没的臊一鼻子灰。
我赶早知会他们去。
”说着,便起身走了。
\par
接着宝钗的饭至,平儿忙进来伏侍。
那时赵姨娘已去,三人在板床上吃饭。
\zhu{板床:指木板坐榻。}
宝钗面南,探春面西,李纨面东。
\zhu{《鸿门宴》:项王即日因留沛公与饮。
项王、项伯东向坐;亚父南向坐,——亚父者,范增也;沛公北向坐;张良西向侍。
南面称孤:南面:面朝南;孤:皇帝、王侯的谦称。
朝南坐着,自称孤家。
指统治一方,称帝称王。
北面称臣:古代君主面南而北,臣子拜见君主则面北,指臣服于人。
可见座位从尊到卑依次是:坐西朝东,坐北朝南,坐南朝北,坐东朝西。
}众媳妇皆在廊下静候,里头只有他们紧跟常侍的丫鬟伺候,别人一概不敢擅入。
这些媳妇们都悄悄的议论说:“大家省事罢,别安着没良心的主意。
连吴大娘才都讨了没意思,咱们又是什么有脸的。
”\ping{这势力上可能真的就是不是东风压倒西风就是西风压倒东风,凤姐之威,亦是不得已,不然,无以发号施令,反被下人戏耍。
}他们一边悄议,等饭完回事。
只觉里面鸦雀无声,并不闻碗箸之声。
一时只见一个丫鬟将帘栊高揭,又有两个将桌抬出。
茶房内早有三个丫头捧着三沐盆水,见饭桌已出,三人便进去了。
一回又捧出沐盆并漱盂来,方有待书、素云、莺儿三个,每人用茶盘捧了三盖碗茶进去。
一时等他三人出来,待书命小丫头子:“好生伺候着,我们吃饭来换你们,别又偷坐着去。
”众媳妇们方慢慢的一个一个的安分回事,不敢如先前轻慢疏忽了。
\par
探春气方渐平,因向平儿道:“我有一件大事,早要和你奶奶商议,如今可巧想起来。
你吃了饭快来。
宝姑娘也在这里,咱们四个人商议了,再细细的问你奶奶可行可止。
”平儿答应回去。
\par
凤姐因问为何去了这一日,平儿便笑着将方才的原故细细说与他听了。
凤姐儿笑道:“好,好,好,好个三姑娘!我说他不错。
只可惜他命薄,没托生在太太肚里。
”平儿笑道:“奶奶也说糊涂话了。
他便不是太太养的,难道谁敢小看他,不与别的一样看了?”凤姐儿叹道:“你那里知道,虽然庶出一样,女儿却比不得男人,将来攀亲时,如今有一种轻狂人,先要打听姑娘是正出是庶出,多有为庶出不要的。
\ping{探春在本回前文说了,“我但凡是个男人,可以出得去,我必早走了,立一番事业,那时自有我一番道理。
偏我是女孩儿家”,可以看出嫡庶对于女孩影响更大,庶子还能出去闯一番,自己给人生变个样子,庶女就是出身影响更多了。
}殊不知别说庶出,便是我们的丫头,比人家的小姐还强呢。
将来不知那个没造化的挑庶正误了事呢,也不知那个有造化的不挑庶正的得了去。
”说着,又向平儿笑道:“你知道,我这几年生了多少省俭的法子,一家子大约也没个不背地里恨我的。
我如今也是骑上老虎了。
虽然看破些,无奈一时也难宽放;二则家里出去的多,进来的少。
凡百大小事仍是照着老祖宗手里的规矩,却一年进的产业又不及先时。
多省俭了,外人又笑话,老太太、太太也受委屈,家下人也抱怨刻薄;若不趁早儿料理省俭之计,再几年就都赔尽了。
”\par
平儿道:“可不是这话!将来还有三四位姑娘,还有两三个小爷,一位老太太,这几件大事未完呢。
”凤姐儿笑道:“我也虑到这里,倒也够了:宝玉和林妹妹他两个一娶一嫁,可以使不着官中的钱,老太太自有梯己拿出来。
二姑娘是大老爷那边的,也不算。
剩下三四个,满破着每人花上一万银子。
环哥娶亲有限,花上三千两银子,不拘那里省一抿子也就够了。
\zhu{一抿子:一点点、一小宗。
抿子:原指刮刷头发的小刷子,所蘸十分有限。
引申作量词。
}老太太事出来,一应都是全了的,不过零星杂项,便费也满破三五千两。
如今再俭省些,陆续也就够了。
只怕如今平空又生出一两件事来,可就了不得了。
——咱们且别虑后事,你且吃了饭,快听他商议什么。
这正碰了我的机会,我正愁没个膀臂。
虽有个宝玉,他又不是这里头的货,纵收伏了他也不中用。
大奶奶是个佛爷,也不中用。
二姑娘更不中用,亦且不是这屋里的人。
四姑娘小呢。
兰小子更小。
环儿更是个燎毛的小冻猫子,只等有热灶火坑让他钻去罢。
\zhu{
用“小冻猫子钻热灶”来形容贾环病病歪歪,畏寒怕冷,身体难以舒展的样子。第十八回元妃省亲时贾环并未参加,是因为“从年内染病未痊”。
也可能用“钻热灶”形容旧社会中趋炎附势的行为。
}
真真一个娘肚子里跑出这样天悬地隔的两个人来,我想到这里就不伏。
\zhu{伏:通“服”,信服。
}再者林丫头和宝姑娘他两个倒好,偏又都是亲戚,又不好管咱家务事。
况且一个是美人灯儿,风吹吹就坏了;一个是拿定了主意,‘不干己事不张口,一问摇头三不知’,也难十分去问他。
倒只剩了三姑娘一个,心里嘴里都也来的,又是咱家的正人,太太又疼他,虽然面上淡淡的,皆因是赵姨娘那老东西闹的,心里却是和宝玉一样呢。
比不得环儿,实在令人难疼,要依我的性早撵出去了。
如今他既有这主意,正该和他协同,大家做个膀臂,\ji{阿凤有才处全在择人,收纳膀臂羽翼,并非一味倚才自恃者可知。
这方是大才。
}我也不孤不独了。
按正理,天理良心上论,咱们有他这个人帮着,咱们也省些心,于太太的事也有些益。
若按私心藏奸上论,我也太行毒了,也该抽头退步。
回头看看了,再要穷追苦克,\zhu{苦克:刻薄,苛刻。
}人恨极了,暗地里笑里藏刀,咱们两个才四个眼睛、两个心,一时不防,倒弄坏了。
趁着紧溜之中,\zhu{紧溜:紧要关头。
}他出头一料理,众人就把往日咱们的恨暂可解了。
还有一件,我虽知你极明白,恐怕你心里挽不过来,\zhu{挽:扭转。
}如今嘱咐你:他虽是姑娘家,心里却事事明白,不过是言语谨慎;他又比我知书识字,更厉害一层了。
如今俗语‘擒贼必先擒王’,他如今要作法开端,一定是先拿我开端。
倘或他要驳我的事,你可别分辩,你只越恭敬,越说驳的是才好。
千万别想着怕我没脸,和他一犟,就不好了。
”\par
平儿不等说完,便笑道:“你太把人看糊涂了。
我才已经行在先,这会子又反嘱咐我。
”凤姐儿笑道:“我是恐怕你心里眼里只有了我,一概没有别人之故,不得不嘱咐。
既已行在先,更比我明白了。
你又急了,满口里‘你’‘我’起来。
”平儿道:“偏说‘你’!你不依,这不是嘴巴子,再打一顿。
难道这脸上还没尝过的不成!”\zhu{第四十四回,贾琏和鲍二家的通奸被捉,凤姐怀疑并迁怒于平儿,打了平儿。
}凤姐儿笑道:“你这小蹄子,要掂多少过子才罢。
\zhu{掂多少过子:以一事作话柄,反复提起。
掂,掂量。
}看我病的这样,还来怄我。
过来坐下,横竖没人来,咱们一处吃饭是正经。
”\par
说着,丰儿等三四个小丫头子进来放小炕桌。
凤姐只吃燕窝粥,两碟子精致小菜,每日分例菜已暂减去。
丰儿便将平儿的四样分例菜端至桌上,与平儿盛了饭来。
平儿屈一膝于炕沿之上,半身犹立于炕下,陪凤姐儿吃了饭,\ji{凤姐之才又在能买邀人心。
}伏侍漱盥。
漱毕,嘱咐了丰儿些话,方往探春处来。
只见院中寂静,人已散出。
要知端的——\par
\qi{总评:噫!事亦难矣哉!探春以姑娘之尊,以贾母之爱,以王夫人之付托,以凤姐之未谢事,暂代数月,而奸奴蜂起,内外欺侮,锱铢小事,突动风波,不亦难乎!以凤姐之聪明,以凤姐之才力,以凤姐之权术,以凤姐之贵宠,以凤姐之日夜焦劳,百般弥缝,\zhu{弥缝:缝合,补救,设法遮掩以免暴露,勉强维持。
}犹不免骑虎难下,为移祸东吴之计,\zhu{移祸东吴:嫁祸他人,这里的意思是,凤姐让探春、李纨、宝钗承担得罪人的管家之责。
移祸东吴出处:吴军偷袭荆州,关羽败走麦城被杀,孙权将关羽首级送给曹魏,曹魏识破东吴嫁祸于人的奸计,以诸侯之礼将其安葬于洛阳。
}不亦难乎!况聪明才力不及凤姐,权术贵宠不及凤姐,焦劳弥缝不及凤姐,又无贾母之爱,姑娘之尊,太太之付托,而欲左支右吾,\zhu{
左支右吾:原谓左右抵拒,引申谓多方面穷于应付。
}撑前达后,不更难乎!士方有志作一番事业,每读至此,不禁为之投书以起,三复流连而欲泣也!\zhu{流连:亦作“流涟”,哭泣流泪貌。
}}
\dai{109}{辱亲女愚妾争闲气}
\dai{110}{平儿教训轻视探春的众媳妇}