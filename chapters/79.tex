\chapter{薛文龙悔娶河东狮 \quad 贾迎春误嫁中山狼}
\zhu{河东狮:比喻妒悍的妇人。
宋代洪迈《容斋三笔》卷三载:陈慥自称龙邱先生,“好宾客,喜畜声妓。
然其妻柳氏绝凶妒,故东坡有诗云:‘龙邱居士亦可怜,谈空说有夜不眠,忽闻河东狮子吼,拄杖落手心茫然。
’”河东为柳姓郡望(柳姓望族世居于河东地方),暗指陈妻柳氏。
狮子吼:佛家以喻威严。
《传灯录》:“释迦佛生时,一手指天,一手指地,作狮子吼,云,天上地下,惟我独尊。
”陈慥好谈佛,故以此相讥。
}
\par
\qi{静含天地自宽,动荡吉凶难定,\zhu{前两句的意思是,动不如静。
“动荡”的未来已经“吉凶难定”,这当然是指家族不可逆转的衰败和众女儿无法挽救的悲剧人生。
}一啄一饮系生成,\zhu{一啄一饮:《庄子·养生主》中的典故:“泽雉十步一啄,百步一饮,不蔪畜乎樊中(蔪:通“祈”,祈求)。”鸟类适情于林籁,随心饮食,逍遥自在。
后泛指人的饮食。
这里应该是指“一饮一啄,莫非前定!”,出自唐·高郢《沙洲独鸟赋》,比喻人的生活际遇,都是命中注定的。
}何必梦中说醒。
\zhu{后两句说只有听天由命罢,不必自诩高明,梦中说醒。
}}\par
话说宝玉祭完了晴雯,只听花影中有人声,倒唬了一跳。
及走出来细看,不是别人,却是林黛玉,满面含笑,口内说道:“好新奇的祭文!可与曹娥碑并传的了。
”\zhu{曹娥碑:曹娥:东汉时孝女,会稽上虞人,父溺于江而死,曹娥寻父尸不得,沿江号哭,最后亦投江而死。
上虞县长官度尚悲悯其义,为之立碑,并命其弟子作诔辞刻于碑上。
见《世说新语·捷悟》刘峻注及《后汉书·曹娥传》李贤注所引东晋虞预《会稽典录》。
相传东汉文学家蔡邕曾称赞碑文为“绝妙好辞”,从此曹娥碑几乎成了祭文的典范。
其实碑文为后人杜撰,同赞语并不相符。
}\ping{用父亲给予的最珍贵的生命作为祭品证明孝心。
父亲溺亡,孩子更应该引以为戒,多加小心,避免重蹈覆辙。
作为孩子,难道不应该更加珍惜死去的父亲给予的生命吗?这条生命可以看作父亲生命的另一种延续方式了。
}宝玉听了,不觉红了脸,笑答道:“我想着世上这些祭文都蹈于熟滥了,所以改个新样,原不过是我一时的顽意,谁知又被你听见了。
有什么大使不得的,何不改削改削。
”\par
黛玉道:“原稿在那里?倒要细细一读。
长篇大论,不知说的是什么,只听见中间两句,什么‘红绡帐里,\zhu{绡:音“消”,轻软的丝织品。
}公子多情,黄土垄中,女儿薄命。
’这一联意思却好,只是‘红绡帐里’未免熟滥些。
放着现成真事,为什么不用?”宝玉忙问:“什么现成的真事?”黛玉笑道:“咱们如今都系霞影纱糊的窗槅,何不说‘茜纱窗下,公子多情’呢?”\ping{宝玉的原文是“帐”,活动空间是在床上;黛玉改为了“窗”,活动空间是屋中。
这就避免了流于淫佚。
}宝玉听了,不禁跌足笑道:\zhu{跌足:跺脚。
}“好极,是极!到底是你想的出,说的出。
可知天下古今现成的好景妙事尽多,只是愚人蠢子说不出想不出罢了。
但只一件:虽然这一改新妙之极,但你居此则可,在我实不敢当。
”说着,又接连说了一二百句“不敢”。
\par
黛玉笑道:“何妨。
我的窗即可为你之窗,何必分晰得如此生疏。
古人异姓陌路,尚然同肥马,衣轻裘,敝之而无憾,
\zhu{
敝:音“币”,坏,破旧。
这里是使动用法,使……破旧。
}
何况咱们。
”宝玉笑道:“论交之道,不在肥马轻裘,即黄金白璧,亦不当锱铢较量。
\zhu{论交……较量:这里意谓论交友的道理,即使比“肥马轻裘”更贵重的“黄金白璧”,也应毫不计较。
《论语·公冶长》:“子路曰:‘愿车马衣裘与朋友共,敝之而无憾。
’”又《论语·雍也》:“赤之适齐也,乘肥马,衣轻裘。
”锱铢:古代很小的重量单位。
锱,四分之一两;铢,二十四分之一两。
}倒是这唐突闺阁,万万使不得的。
如今我越性将‘公子’‘女儿’改去,竟算是你诔他的倒妙。
况且素日你又待他甚厚,故今宁可弃此一篇大文,万不可弃此‘茜纱’新句。
竟莫若改作‘茜纱窗下,小姐多情,黄土垄中,丫鬟薄命。
’如此一改,虽于我无涉,我也是惬怀的。
”黛玉笑道:“他又不是我的丫头,何用作此语。
况且小姐丫鬟亦不典雅,等我的紫鹃死了,我再如此说,还不算迟。
”\geng{明是为与阿颦作谶,却先偏说紫鹃,总用此狡猾之法。
}宝玉听了,忙笑道:“这是何苦又咒他。
”\geng{又画出宝玉来,究竟不知是咒谁,使人一笑一叹。
}黛玉笑道:“是你要咒的,并不是我说的。
”宝玉道:“我又有了,这一改可妥当了。
莫若说:‘茜纱窗下,我本无缘;\ping{“我”暗指宝玉。
}\geng{双关句,意妥极。
}黄土垄中,卿何薄命。
’”\ping{“卿”暗指黛玉。
}\geng{如此我亦谓妥极。
但试问当面用“尔”“我”字样究竟不知是为谁之谶,一笑一叹。
}\geng{一篇诔文总因此二句而有,又当知虽诔晴雯而又实诔黛玉也。
奇幻至此!若云必因晴雯诔,则呆之至矣。
}\par
黛玉听了,忡然变色,\zhu{忡:音“冲”,忧虑的样子。
}\geng{慧心人可为一哭。
观此句便知诔文实不为晴雯而作也。
}心中虽有无限的狐疑乱拟,\geng{用此事更妙,盖又欲瞒观者。
}外面却不肯露出,反连忙含笑点头称妙,说:“果然改的好。
再不必乱改了,快去干正经事罢。
才刚太太打发人叫你明儿一早快过大舅母那边去。
你二姐姐已有人家求准了,想是明儿那家人来拜允,所以叫你们过去呢。
”宝玉拍手道:“何必如此忙?我身上也不大好,明儿还未必能去呢。
”黛玉道:“又来了,我劝你把脾气改改罢。
一年大二年小,……”一面说话,一面咳嗽起来。
\geng{总为后文伏笔。
阿颦之\sout{问}[文]可见不是一笔两笔所写。
}宝玉忙道:“这里风冷,咱们只顾呆站在这里,快回去罢。
”黛玉道:“我也家去歇息了,明儿再见罢。
”说着,便自取路去了。
\par
宝玉只得闷闷的转步,又忽想起来黛玉无人随伴,忙命小丫头子跟了送回去。
自己到了怡红院中,果有王夫人打发老嬷嬷来,吩咐他明日一早过贾赦那边去,与方才黛玉之言相对。
\par
原来贾赦已将迎春许与孙家了。
这孙家乃是大同府人氏,\geng{设云“大概相同”也,若必云真大同府则呆。
}祖上系军官出身,乃当日宁荣府中之门生,算来亦系世交。
如今孙家只有一人在京,现袭指挥之职,\zhu{指挥:官名,历代职掌品级不同,清代京师兵马司指挥约在六七品之间。
}此人名唤孙绍祖,生得相貌魁梧,体格健壮,弓马娴熟,应酬权变,\geng{画出一个俗物来。
}年纪未满三十,且又家资饶富,\geng{此句断不可少。
}现在兵部候缺题升。
\zhu{题升:犹提升。
}因未有室,贾赦见是世交子侄,且人品家当都相称合,遂青目择为东床娇婿。
亦曾回明贾母。
\par
贾母心中却不十分称意,想来拦阻亦恐不听,\ping{贾赦和贾母关系不好,前有贾赦强娶鸳鸯,后有贾赦中秋讲笑话暗指贾母“偏心”,所以贾母不方便阻拦。
}\ping{步步紧逼,无回旋余地。
}儿女之事自有天意前因,况且他是亲父主张,何必出头多事,为此只说“知道了”三字,馀不多及。
贾政又深恶孙家,虽是世交,当年不过是彼祖希慕荣宁之势,有不能了结之事才拜在门下的,并非诗礼名族之裔,因此倒劝谏过两次,无奈贾赦不听,也只得罢了。
\par
宝玉却从未会过这孙绍祖一面的,次日只得过去聊以塞责。
只听见说娶亲的日子甚急,不过今年就要过门的,又见邢夫人等回了贾母将迎春接出大观园去等事,越发扫去了兴头,每日痴痴呆呆的,不知作何消遣。
又听得说陪四个丫头过去,更又跌足自叹道:“从今后这世上又少了五个清洁人了。
”因此天天到紫菱洲一带地方徘徊瞻顾,见其轩窗寂寞,\zhu{轩窗:窗子。
}屏帐翛然,\zhu{
    翛[xiāo]然:萧然,空寂的样子。
屏帐翛然:人去屋空的景况。
}不过有几个该班上夜的老妪。
\geng{先为“对\sout{竟}[景]悼颦儿”作引。
}再看那岸上的蓼花苇叶,池内的翠荇香菱,也都觉摇摇落落,似有追忆故人之态,迥非素常逞妍斗色之可比。
既领略得如此寥落凄惨之景,\ping{在凄惨景色中出现香菱,乃作者有意为之,明为迎春,暗为香菱。
}是以情不自禁,乃信口吟成一歌曰:\geng{此回题上半截是“悔娶河东狮”,今却偏逢“中山狼”,倒装上下情孽,细腻写来,可见迎春是书中正传,阿呆夫妻是副,宾主次序严肃之至。
其婚娶俗礼一概不及,只用宝玉一人过去,正是书中之大旨。
}\par
\hop
池塘一夜秋风冷,吹散芰荷红玉影。
\zhu{芰(音“记”)荷:这里指出水的荷花。
红玉影:代指红色的荷花。
这两句以一夜秋风吹落了红玉般的荷花花瓣,喻宝玉同迎春的即将分离。
}\par
蓼花菱叶不胜愁,重露繁霜压纤梗。
\par
不闻永昼敲棋声,燕泥点点污棋枰。
\par
古人惜别怜朋友,况我今当手足情!\par
\hop
宝玉方才吟罢,忽闻背后有人笑道:“你又发什么呆呢?”宝玉回头忙看是谁,原来是香菱。
\ping{宝玉诗中有“蓼花菱叶不胜愁”,虽是为迎春之离去而伤感,但是其中的“菱”字也可能暗示了香菱的命运。
香菱的生命如同这枯萎的菱角,因夏金桂折磨而凋零。
诗作后面紧接着就是香菱进园子找人,可能也是一种佐证。
}宝玉便转身笑问道:“我的姐姐,你这会子跑到这里来做什么?许多日子也不进来逛逛。
”香菱拍手笑嘻嘻的说道:“我何曾不来。
如今你哥哥回来了,那里比先时自由自在的了。
才刚我们奶奶使人找你凤姐姐的,竟没找着,说往园子里来了。
我听见了这信,我就讨了这件差进来找他。
遇见他的丫头,说在稻香村呢。
如今我往稻香村去,谁知又遇见了你。
我且问你,袭人姐姐这几日可好?怎么忽然把个晴雯姐姐也没了,到底是什么病?二姑娘搬出去的好快,你瞧瞧这地方好空落落的。
”\par
宝玉应之不迭,又让他同到怡红院去吃茶。
\geng{断不可少。
}香菱道:“此刻竟不能,等找着琏二奶奶,说完了正经事再来。
”宝玉道:“什么正经事这么忙?”香菱道:“为你哥哥娶嫂子的事,所以要紧。
”\geng{出题却闲闲引出。
}宝玉道:“正是。
说的到底是那一家的?只听见吵嚷了这半年,今儿又说张家的好,明儿又要李家的,后儿又议论王家的。
这些人家的女儿他也不知道造了什么罪了,叫人家好端端议论。
”香菱道:“这如今定了,可以不用搬扯别家了。
”宝玉忙问:“定了谁家的?”香菱道:“因你哥哥上次出门贸易时,在顺路到了个亲戚家去。
这门亲原是老亲,且又和我们是同在户部挂名行商,\zhu{行商:音“形商”,往来各地贩售货品的商人。
也称为“行贩”、“行贾”。
}也是数一数二的大门户。
前日说起来,你们两府都也知道的。
合长安城中,上至王侯,下至买卖人,都称他家是‘桂花夏家’。
”\geng{夏日何得有桂?\zhu{桂花秋季开花。
桂指夏金桂。
}又桂花时节焉得又有雪?\zhu{雪指薛家。
}三事原系风马牛,\zhu{风马牛:语出《左传·僖公四年》:“齐侯以诸侯之师侵蔡,蔡溃,遂伐楚。
楚子使与师言:‘君处北海,寡人处南海,唯是风马牛不相及也。
不虞君之涉吾地也。
’”比喻事物之间毫不相干。
}今若强凑合,故终不相符。
来此败运之事,大都如此,当局者自不解耳。
}宝玉笑问道:\geng{听得“桂花”字号原觉新雅,故不觉一笑,余亦欲笑问。
}“如何又称为‘桂花夏家’?”香菱道:“他家本姓夏,非常的富贵。
其馀田地不用说,单有几十顷地独种桂花,凡这长安城里城外桂花局俱是他家的,\zhu{局:商店的称号。
如:“书局”。
}连宫里一应陈设盆景亦是他家贡奉,因此才有这个浑号。
如今太爷也没了,只有老奶奶带着一个亲生的姑娘过活,也并没有哥儿兄弟,可惜他竟一门尽绝了后。
”\par
宝玉忙道:“咱们也别管他绝后不绝后,只是这姑娘可好?你们大爷怎么就中意了?”\geng{补出阿呆素日难得中意来。
}香菱笑道:“一则是天缘,二则是‘情人眼里出西施’。
当年又是通家来往,\zhu{通家:世交;姻亲。
}从小儿都一处厮混过。
叙起亲是姑舅兄妹,又没嫌疑。
虽离开了这几年,前儿一到他家,夏奶奶又是没儿子的,一见了你哥哥出落的这样,又是哭,又是笑,竟比见了儿子的还胜。
又令他兄妹相见,谁知这姑娘出落得花朵似的了,在家里也读书写字,所以你哥哥当时就一心看准了。
连当铺里老朝奉伙计们一群人蹧扰了人家三四日,\zhu{朝奉:原为宋代官名,后用作对富豪或店铺中高级雇员的称呼。
}他们还留多住几日,好容易苦辞才放回家。
你哥哥一进门,就咕咕唧唧求我们奶奶去求亲。
我们奶奶原也是见过这姑娘的,且又门当户对,也就依了。
和这里姨太太、凤姑娘商议了,打发人去一说就成了。
只是娶的日子太急,所以我们忙乱的很。
\geng{阿呆求妇一段文字却从香菱口中补明,省却许多闲文累笔。
}我也巴不得早些过来,又添一个作诗的人了。
”\geng{妙极!香菱口声,断不可少。
看他下“作\sout{死}[诗]”语,便知其心中略无忌讳疑虑等意,直是浑然天真之人,余为一哭。
}宝玉冷笑道:\geng{忽曰“冷笑”,二字便有文章。
}“虽如此说,但只我倒替你耽心虑后呢。
”\geng{又为香菱之谶,偏是此等事体等到。
}香菱听了,不觉红了脸,正色道:“这是什么话!素日咱们都是厮抬厮敬的,\zhu{厮:互相。
抬:抬举。
}今日忽然提起这些事来,是什么意思!怪不得人人都说你是个亲近不得的人。
”一面说,一面转身走了。
\ping{香菱就算对薛蟠没有感情,自己的领导又给自己找了个领导,总会担心吧?虽说薛蟠妻妾的职责,不需要自己一个人完全承担了,但是香菱全是高兴还是太天真了。
}\par
宝玉见他这样,便怅然如有所失,呆呆的站了半天,思前想后,不觉滴下泪来,只得没精打彩,还入怡红院来。
一夜不曾安稳,睡梦之中犹唤晴雯,或魇魔惊怖,\zhu{
魇[yǎn]魔:同“魇昧”,用法术使人受祸或使之神智迷糊;指用药物之类使人迷糊。
}种种不宁。
次日便懒进饮食,身体作热。
此皆近日抄检大观园、逐司棋、别迎春、悲晴雯等羞辱惊恐悲凄之所致,兼以风寒外感,故酿成一疾,卧床不起。
贾母听得如此,天天亲来看视。
王夫人心中自悔不合因晴雯过于逼责了他。
心中虽如此,脸上却不露出。
只吩咐众奶娘等好生伏侍看守,一日两次带进医生来诊脉下药。
一月之后,方才渐渐的痊愈。
贾母命好生保养,过百日方许动荤腥油面等物,方可出门行走。
这一百日内,连院门前皆不许到,只在房中顽笑。
四五十日后,就把他拘约的火星乱迸,那里忍耐得住。
虽百般设法,无奈贾母王夫人执意不从,也只得罢了。
因此和那些丫鬟们无所不至,恣意耍笑作戏。
又听得薛蟠摆酒唱戏,热闹非常,已娶亲入门,闻得这夏家小姐十分俊俏,也略通文翰,\zhu{文翰:文章笔墨之事。
}宝玉恨不得就过去一见才好。
\par
再过些时,又闻得迎春出了阁。
宝玉思及当时姊妹们一处,耳鬓厮磨,从今一别,纵得相逢,也必不似先前那等亲密了。
眼前又不能去一望,真令人凄惶迫切之至。
少不得潜心忍耐,暂同这些丫鬟们厮闹释闷,幸免贾政责备逼迫读书之难。
这百日内,只不曾拆毁了怡红院,和这些丫头们无法无天,凡世上所无之事,都顽耍出来。
如今且不消细说。
\ping{迎春和薛蟠的婚礼,本可以大书特书,但是这里用宝玉抱病一笔带过,去繁俗,用省笔,强干弱枝。
}\par
且说香菱自那日抢白了宝玉之后,心中自为宝玉有意唐突他,“怨不得我们宝姑娘不敢亲近,可见我不如宝姑娘远矣;怨不得林姑娘时常和他角口气的痛哭,自然唐突他也是有的了。
从此倒要远避他才好。
”因此,以后连大观园也不轻易进来。
日日忙乱着,薛蟠娶过亲,自为得了护身符,自己身上分去责任,到底比这样安宁些;二则又闻得是个有才有貌的佳人,自然是典雅和平的。
因此,他心中盼过门的日子比薛蟠还急十倍。
好容易盼得一日娶过了门,他便十分殷勤小心伏侍。
\par
原来这夏家小姐今年方十七岁,生得亦颇有姿色,亦颇识得几个字。
若论心中的丘壑经纬,颇步熙凤之后尘。
只吃亏了一件,从小时父亲去世的早,又无同胞弟兄,寡母独守此女,娇养溺爱,不啻珍宝,\zhu{啻:音“赤”,但、只、仅。
常用于疑问词或否定词之后。
如“不啻”、“何啻”。
}凡女儿一举一动,彼母皆百依百随,因此未免娇养太过,竟酿成个盗跖的性气。
\zhu{盗跖:人名,传说中的大盗。
见《史记·伯夷列传》张守节《正义》及《庄子·盗跖》。
}
爱自己尊若菩萨,窥他人秽如粪土;外具花柳之姿,内秉风雷之性。
在家中时常就和丫鬟们使性弄气,轻骂重打的。
今日出了阁,自为要作当家的奶奶,比不得作女儿时腼腆温柔,须要拿出这威风来,才钤压得住人;\zhu{钤:音“前”,锁。
}况且见薛蟠气质刚硬,举止骄奢,若不趁热灶一气炮制熟烂,将来必不能自竖旗帜矣;又见有香菱这等一个才貌俱全的爱妾在室,越发添了“宋太祖灭南唐”之意,“卧榻之侧岂容他人酣睡”之心。
\zhu{宋太祖灭南唐之意:这里是妒嫉、不能容人之意。
《宋史纪事本末·平江南》:南唐后主李煜遣徐铉向宋太祖乞求缓师,“帝按剑怒曰:‘不须多言,江南主亦有何罪,但天下一家,卧榻之侧,岂容他人酣睡耶!’”}因他家多桂花,他小名就唤做金桂。
他在家时不许人口中带出金桂二字来,凡有不留心误道一字者,他便定要苦打重罚才罢。
他因想桂花二字是禁止不住的,须另换一名,因想桂花曾有广寒嫦娥之说,\zhu{嫦娥奔月:传说中后羿向西王母求得不死之药,他的妻子嫦娥偷吃了以后,而飞上月亮,常住于广寒宫中的故事。
吴刚伐桂:传说中吴刚学仙有过,遭天帝惩罚到月宫砍伐桂树,其树随砍随合,所以必须不断砍伐。
}便将桂花改为嫦娥花,又寓自己身分如此。
\par
薛蟠本是个怜新弃旧的人,且是有酒胆无饭力的,\zhu{有酒胆无饭力:比喻人表面上很强硬,骨子里却很懦弱,不敢诉诸行动。
}如今得了这样一个妻子,正在新鲜兴头上,凡事未免尽让他些。
那夏金桂见了这般形景,便也试着一步紧似一步。
一月之中,二人气概还都相平;至两月之后,便觉薛蟠的气概渐次低矮了下去。
一日薛蟠酒后,不知要行何事,先与金桂商议,金桂执意不从。
薛蟠忍不住便发了几句话,赌气自行了,\zhu{自行了:呼应前文的“不知要行何事”。
}这金桂便气的哭如醉人一般,茶汤不进,装起病来。
请医疗治,医生又说“气血相逆,当进宽胸顺气之剂。
”\par
薛姨妈恨的骂了薛蟠一顿,说:“如今娶了亲,眼前抱儿子了,还是这样胡闹。
人家凤凰蛋似的,好容易养了一个女儿,比花朵儿还轻巧,原看的你是个人物,才给你作老婆。
你不说收了心安分守己,一心一计和和气气的过日子,还是这样胡闹,噇嗓了黄汤,\zhu{噇嗓:音“床桑”,犹言“狂吃滥饮”。
黄汤:指酒(骂人喝酒时说)。
}折磨人家。
这会子花钱吃药白遭心。
”
\zhu{遭心:犹操心。}
一席话说的薛蟠后悔不迭,反来安慰金桂。
金桂见婆婆如此说丈夫,越发得了意,便装出些张致来,\zhu{张致:故作姿态。
}总不理薛蟠。
薛蟠没了主意,惟自怨而已,好容易十天半月之后,才渐渐的哄转过金桂的心来,自此便加一倍小心,不免气概又矮了半截下来。
\par
那金桂见丈夫旗纛渐倒,\zhu{纛:音“道”,古代军中大旗。
}婆婆良善,也就渐渐的持戈试马起来。
先时不过挟制薛蟠,后来倚娇作媚,将及薛姨妈,又将至薛宝钗。
宝钗久察其不轨之心,每随机应变,暗以言语弹压其志。
金桂知其不可犯,每欲寻隙,又无隙可乘,只得曲意附就。
一日金桂无事,因和香菱闲谈,问香菱家乡父母。
香菱皆答忘记,金桂便不悦,说有意欺瞒了他。
因问他“香菱”二字是谁起的名字,香菱便答:“姑娘起的。
”金桂冷笑道:“人人都说姑娘通,只这一个名字就不通。
”香菱忙笑道:“嗳哟,奶奶不知道,我们姑娘的学问连我们姨老爷时常还夸呢。
”欲明后事,且见下回。
\par
\qi{总评:作诔后,黛玉飘然而至,增一番感慨,及说至迎春事,遂飘然而去。
作词后,香菱飘然而至,增一番感慨,及说至薛蟠事,遂飘然而去。
一点一逗,为下文引线。
且二段俱以“正经事”三字作眼,
\zhu{正经事:本回黛玉说让宝玉干“正经事”去见邢夫人,香菱说要找凤姐说“正经事”。}
而正经里更有大不正经者在,文家固无一呆字死句。
\hang
从起名上设色,\zhu{起名:指的是给宝钗给甄英莲取名“香菱”,夏金桂在后文改名为“秋菱”。
设色:用颜料渲染,形成各种美丽的色彩。
}别有可玩。
}
\dai{160}{迎春出嫁,宝玉对景伤感}
\dai{157}{夏金桂故意不理哄自己的的薛蟠}
\dai{158}{孙绍祖好色,贾迎春被打骂}
\sun{p79-1}{因病禁足嬉笑怡红}{宝玉因抄检大观园、逐司棋、别迎春、悲晴雯,兼以风寒外感,故酿成一疾,卧床不起。
宝玉因不能外出,因此和那些丫鬟们无所不至,恣意耍笑作戏。
}