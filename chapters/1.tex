\chapter{甄士隐梦幻识通灵\quad 贾雨村风尘怀闺秀}
\geng{此开卷第一回也。
作者自云:因曾历过一番梦幻之后,故将真事隐去,而借通灵之说,撰此《石头记》一书也,故曰“甄士隐”云云。
但书中所记何事何人?自又云:“今风尘碌碌,一事无成,忽念及当日所有之女子,一一细考较去,觉其行止见识皆出于我之上。
何我堂堂须眉,\zhu{须眉:代指男子。
}诚不若此裙钗哉?\zhu{裙钗:代指女子。
}\meng{何非梦幻,何不通灵?作者托言,原当有自。
受气清浊,本无男女之别。
}实愧则有馀,悔又无益之大无可如何之日也!当此,则自欲将已往所赖天恩祖德,锦衣纨绔之时,\zhu{锦衣纨绔:富贵者的穿着,引申为富家子弟的代称。
 锦:色彩华美的丝织物 。
 纨[wán]:细绢。
}饫甘餍肥之日,\zhu{饫甘餍肥:犹言饱食香甜肥美的食品。
 饫[yù]、餍[yàn]:吃饱吃腻的意思。
}背父兄教育之恩,负师友规谈之德,以至今日一技无成、半生潦倒之罪,\meng{明告看者。
}\ping{富贵荣华时,其亦不过舞勺之年,因祖荫生活优渥,因天性纯粹而厌恶逐荣利而读书,稚子何罪之有。
父兄尚无力令家族再续荣华,兴家之重皆由其一人承担,其实荒唐。
然繁华梦散后,却因所尝富贵而心惭,实乃纯善之人。
}编述一集,以告天下人:我之罪固不免,然闺阁中本自历历有人,万不可因我之不肖,\zhu{肖:像。
不肖:子不似父,不能继承父业;不贤,无才能;品性不良。
}自护己短,一并使其泯灭也。
\meng{因为传他,并可传我。
}虽今日之茅椽蓬牖,\zhu{茅椽蓬牖:代指草房陋室,贫者所居。
茅、蓬都是野草。
椽[chuán]:房椽子(中国古代建筑结构图见页脚\foot{\footPic{中国古代建筑结构图}{house.jpg}{0.8}});牖[yǒu],窗户。
}瓦灶绳床,\zhu{瓦灶绳床:瓦灶为土还烧成的简陋的灶,俗称行灶。
绳床亦名胡床、交床,为一种简易的坐具。
}其晨夕风露,阶柳庭花,亦未有妨我之襟怀笔墨。
虽我未学,\zhu{“未学”一词可能是“末学”的错别字。
“末学”一词是我国古代常用的自谦之词,指没有多少学问,其实就是客套话而已,学问越大的人往往越谦虚,越喜欢说自己是“末学”。
如苏轼在《与封守朱朝请》信中写道:“前日蒙示所藏诸书,使末学稍窥家传之秘,幸甚,幸甚!”苏轼学问可谓够大了,但在信中他却依然自称“末学”。
而“未学”也是古代经常用到的词语,多指没有上学或是没有学习的意思。
曹雪芹如果没上学读过书,怎么能写成《红楼梦》这样登峰造极之巨著,可见“未学”用在此处不够妥帖。
}下笔无文,又何妨用假语村言敷演出一段故事来?\zhu{敷演:叙述生发。
}亦可使闺阁昭传,复可悦世之目,破人愁闷,不亦宜乎?”\ping{繁华事散,昔盛今衰,抚今追昔,无限感慨萦于心头。
于是曹雪芹著《红楼梦》,类似的,张岱在《陶庵梦忆》自序写道:因思昔人生长王、谢,颇事豪华,今日罹此果报。
以笠报颅,以篑报踵,仇簪履也;以衲报裘,以苎报絺,仇轻暖也;以藿报肉,以粝报粻,仇甘旨也;以荐报牀,以石报枕,仇温柔也;以绳报枢,以瓮报牖,仇爽垲也;以烟报目,以粪报鼻,仇香艳也;以途报足,以囊报肩,仇舆从也。
种种罪案,从种种果报中见之。
鸡鸣枕上,夜气方回,因想余生平,繁华靡丽,过眼皆空,五十年来,总成一梦。
今当黍熟黄粱,车旅蚁穴,当作如何消受?遥思往事,忆即书之,持向佛前,一一忏悔。
不次岁月,异年谱也;不分门类,别志林也。
偶拈一则,如游旧径,如见故人,城郭人民,翻用自喜,真所谓痴人前不得说梦矣。
}
故曰“贾雨村”云云。
\zhu{贾雨村:谐音“假语存”。}
\hang
此回中凡用“梦”用“幻”等字,是提醒阅者眼目,亦是此书立意本旨\foot{以上文字见于庚、戚、蒙、列、辰、舒、杨诸本,其中甲辰本为回前批,馀本均为正文。
此段与甲戌本凡例第五条略同,玩其文意应非正文,现作为回前批处理。
}。
}\par
列位看官,你道此书从何而来?说起根由虽近荒唐,\jia{自占地步。
}\jia{自首荒唐,妙!}细谙则深有趣味。
待在下将此来历注明,方使阅者了然不惑。
\par
原来,女娲氏炼石补天之时,\jia{补天济世,勿认真,用常言。
}于大荒山\jia{荒唐也。
}无稽崖\jia{无稽也。
}\zhu{稽:考证,考核。
}炼成高经十二丈、\jia{总应十二钗。
}方经二十四丈\jia{照应副十二钗。
}顽石三万六千五百零一块。
娲皇氏只用了三万六千五百块,\jia{合周天之数。
 }\meng{数足,偏遗我。
“不堪入选”句中透出心眼。
}只单单的剩了一块未用,\jia{剩了这一块便生出这许多故事。
使当日虽不以此补天,就该去补地之坑陷,使地平坦,而不得有此一部鬼话。
}便弃在此山青埂峰下。
\jia{妙!自谓落堕情根,故无补天之用。
}谁知此石自经煅炼之后,灵性已通,\jia{煅炼后性方通,甚哉!人生不能[不]学也。
}因见众石俱得补天,独自己无材不堪入选,遂自怨自叹,日夜悲号惭愧。
\par
一日,正当嗟悼之际,俄见一僧一道远远而来,生得骨格不凡,丰神迥别,\qi{这是真像,非幻像也。
}说说笑笑来至峰下,坐于石边,高谈快论。
先是说些云山雾海、神仙玄幻之事,后便说到红尘中荣华富贵。
\ping{神仙逍遥不过心,红尘富贵却起意。
灵怪修形不修心,自然有此劫。
}此石听了,不觉打动凡心,也想要到人间去享一享这荣华富贵,但自恨粗蠢,不得已,便口吐人言,\jia{竟有人问:“口生于何处?”其无心肝,可笑可恨之极!}向那僧道说道:“大师,弟子蠢物,\jia{岂敢岂敢。
}不能见礼了。
\zhu{见礼:见面行礼。
}
适闻二位谈那人世间荣耀繁华,心切慕之。
弟子质虽粗蠢,\jia{岂敢岂敢。
}性却稍通,况见二师仙形道体,定非凡品,必有补天济世之材,利物济人之德。
如蒙发一点慈心,携带弟子得入红尘,在那富贵场中、温柔乡里受享几年,自当永佩洪恩,万劫不忘也。
”二仙师听毕,齐憨笑道:“善哉,善哉!那红尘中有却有些乐事,但不能永远依恃,况又有‘美中不足,好事多魔’八个字紧相连属,瞬息间则又乐极悲生、人非物换,\ping{第十三回,秦可卿托梦凤姐:“常言‘月满则亏,水满则溢’;又道是‘登高必跌重’。
”}究竟是到头一梦、万境归空。
\jia{四句乃一部之总纲。
}\ping{一切有为法,如梦幻泡影,如露亦如电,当作如是观。
}倒不如不去的好。
”\par
这石凡心已炽,那里听得进这话去,乃复苦求再四。
二仙知不可强制,乃叹道:“此亦静极思动,无中生有之数也。
既如此,我们便携你去受享受享,只是到不得意时,切莫后悔。
”\ping{不经不悟,自知质蠢,不能超脱。
此石有轮回,下世乃悟,芸芸众生仅活一世,如何不执着。
}石道:“自然,自然。
”那僧又道:“若说你性灵,却又如此质蠢,并更无奇贵之处,如此也只好踮脚而已。
\zhu{踮脚:犹言“垫脚”。
}\jia{煅炼过尚与人踮脚,不学者又当如何?}也罢,我如今大施佛法助你[一]助,待劫终之日,复还本质,以了此案。
\jia{妙!佛法亦须偿还,况世人之债乎?近之赖债者来看此句。
所谓游戏笔墨也。
}你道好否?”石头听了,感谢不尽。
那僧便念咒书符,大展幻\jia{明点“幻”字。
好!}术,将一块大石登时变成\foot{“说说笑笑……将一块大石登时变成”四百二十馀字为甲戌本独有,各本皆缺,补以“来至石下,席地而坐长谈,见”数语连接下文。
}一块鲜明莹洁的美玉,且又缩成扇坠大小的可佩可拿。
\jia{奇诡险怪之文,有如髯苏《石钟》《赤壁》用幻处。
}那僧托于掌上,笑道:“形体倒也是个宝物了!\jia{自愧之语。
}\meng{世上人原自据看得见处为凭。
}还只没有实在的好处,\jia{妙极!今之金玉其外败絮其中者,见此大不欢喜。
}须得再镌上数字,使人一见便知是奇物方妙。
\jia{世上原宜假,不宜真也。
}\jia{谚云:“一日卖了三千假,三日卖不出一个真。
”信哉!}然后好携你到那昌明隆盛之邦,\jia{伏长安大都。
}诗礼簪缨之族,\zhu{诗礼簪缨之族:指书香门第,官宦家族。
诗礼:读诗书,讲礼仪。
簪缨:贵者的冠饰,这里代指作官。
簪:一种横插髻上或连接冠与髻的长针。
缨:帽带。
}\jia{伏荣国府。
}花柳繁华地,\jia{伏大观园。
}温柔富贵乡\jia{伏紫芸轩。
}去安身乐业。
”\jia{何不再添一句云“择个绝世情痴作主人”? }\jia{昔子房后谒黄石公,惟见一石。
子房当时恨不随此石去。
\zhu{黄石公:典故出自《史记·留侯世家》。(张)良尝闲从容步游下邳圯上,有一老父,衣褐,至良所,直堕其履圯下,顾谓良曰:“孺子,下取履!”良鄂然,欲殴之。为其老,强忍,下取履。父曰:“履我!”良业为取履,因长跪履之。父以足受,笑而去。良殊大惊,随目之。父去里所,复还,曰:“孺子可教矣。后五日平明,与我会此。”良因怪之,跪曰:“诺。”五日平明,良往。父已先在,怒曰:“与老人期,后,何也?”去,曰:“后五日早会。”五日鸡鸣,良往。父又先在,复怒曰:“后,何也?”去,曰:“后五日复早来。”五日,良夜未半往。有顷,父亦来,喜曰:“当如是。”出一编书,曰:“读此则为王者师矣。后十年兴。十三年孺子见我济北,谷城山下黄石即我矣。”遂去,无他言,不复见。旦日视其书,乃太公兵法也。…………子房始所见下邳圯上老父与太公书者,后十三年从高帝过济北,果见谷城山下黄石,取而葆祠之。留侯死,并葬黄石(冢)。每上冢伏腊,祠黄石。}
余亦恨不能随此石而去也。
聊供阅者一笑。
}石头听了,喜不能禁,乃问:“不知赐了弟子那几件奇处,\jia{可知若果有奇贵之处,自己亦不知者。
若自以奇贵而居,究竟是无真奇贵之人。
}又不知携了弟子到何地方?望乞明示,使弟子不惑。
”那僧笑道:“你且莫问,日后自然明白的。
”说着,便袖了这石,同那道人飘然而去,竟不知投奔何方何舍。
\par
后来,又不知过了几世几劫,\zhu{劫:佛家用语。
梵文音译“劫波”之略,意为“远大时节"。
佛教认为,世界有周期性的生灭过程,它经历若干万年后,就要毁灭一次,重新开始,此一周期称为一“劫”。
每“劫”中还包括“成”、“住”、“坏”、“空”四个阶段。
到“坏劫”时,有水、火、风三灾出现.世界便归于毁灭.故后人又将“劫”引伸作灾难解,如后文“劫终之日”、“生关死劫”等。
世:古人以三十年为一世。佛教说法,宇宙有生成的时候,也有破坏的时候,二者交替循环,永不终止。生成时,世界出现,叫做“世”;破坏时,世界毁灭,叫做“劫”。
}因有个空空道人访道求仙,忽从这大荒山无稽崖青埂峰下经过,忽见一大石上字迹分明,编述历历。
空空道人乃从头一看,原来就是无材补天、幻形入世,\jia{八字便是作者一生惭恨。
}蒙茫茫大士、渺渺真人携入红尘,历尽离合悲欢、炎凉世态的一段故事。
\ping{人间一趟,玉又化石,石乃真实。
}后面又有一首偈云:\zhu{偈(音“记”):梵文音译“偈陀”或“伽陀”之略,意译为颂。
一般为四句之韵文。
}\par
\hop

    无材可去补苍天,\jia{书之本旨。
}枉入红尘若许年。
\jia{惭愧之言,呜咽如闻。
}\par
此系身前身后事,倩谁记去作奇传?\zhu{倩(音“庆”)谁:请谁。
}\par
\hop

诗后便是此石堕落之乡,投胎之处,亲自经历的一段陈迹故事。
其中家庭闺阁琐事,以及闲情诗词倒还全备,或\jia{“或”字谦得好。
}可适趣解闷,然朝代年纪,地舆邦国,\jia{若用此套者,胸中必无好文字,手中断无新笔墨。
}却反失落无考。
\jia{据余说,却大有考证。
} \meng{妙在“无考”。
}\par
空空道人遂向石头说道:“石兄,你这一段故事,据你自己说有些趣味,故编写在此,意欲问世传奇。
据我看来:第一件,无朝代年纪可考,\jia{先驳得妙。
}第二件,并无大贤大忠理朝廷、治风俗的善政,\jia{
将世人欲驳之腐言预先代人驳尽。
妙!}其中只不过几个异样的女子,或情或痴,或小才微善,亦无班姑、蔡女之德能。
\zhu{班姑、蔡女之德能:班姑:即班昭,东汉史学家班固之妹,博学,曾参与续《汉书》。
和帝时担任过宫廷教师,号称“大家(家:音“姑”)”, 故称“班姑”。
编有《女诫》七篇,历来奉为妇德的典范。
见《后汉书·曹世叔妻传》。
蔡女:指蔡文姬,名琰,东汉文学家蔡邕之女,博学多才,精通音律,是历史上有名的“才女”。
见《后汉书·董祀妻传》。
}我纵抄去,恐世人不爱看呢。
”石头笑答道:“我师何太痴耶!若云无朝代可考,今我师竟假借汉唐等年纪添缀,又有何难?\jia{所以答得好。
}但我想,历来野史,\zhu{野史:一般是指与官修正史相对而言的私家编撰的史类著作。
“野史”之名始见于《新唐书·艺文志》,后渐与小说家言的“禅官”连用,称“裨官野史”。
这里即指小说。
}皆蹈一辙,\ping{人人皆认为自己一生独一无二,可若社会没有大的变动,更可能的是人们只是重复某种模式的生活,只有具体细节的不同。
}莫如我这不借此套者,反倒新奇别致,不过只取其事体情理罢了,又何必拘拘于朝代年纪哉!再者,市井俗人喜看理治之书者甚少,\zhu{理治之书:泛指古代“理朝廷治风俗”的书籍。
}爱看适趣闲文者特多。
历来野史,或讪谤君相,或贬人妻女,\jia{先批其大端。
}
奸淫凶恶,不可胜数。
更有一种风月笔墨,\zhu{风月笔墨:原指描写风花雪月、儿女私情的文字。
这里专指着意渲染色情的作品。
}其淫秽污臭,涂毒笔墨,\zhu{涂毒:甲戌本、杨本均为“涂毒”,而戚本、蒙本作“屠毒”,其他各本无此句。
俞平伯、蔡义江校作“荼毒”。
}坏人子弟,又不可胜数。
至若佳人才子等书,则又千部共出一套,\ping{第五十四回,贾母批斥才子佳人小说:“这些书都是一个套子,左不过是些佳人才子,最没趣儿。
把人家女儿说的那样坏,还说是佳人,编的连影儿也没有了。
开口都是书香门第,父亲不是尚书就是宰相,生一个小姐必是爱如珍宝。
这小姐必是通文知礼,无所不晓,竟是个绝代佳人。
只一见了一个清俊的男人,不管是亲是友,便想起终身大事来,父母也忘了,书礼也忘了,鬼不成鬼,贼不成贼,那一点儿是佳人?便是满腹文章,做出这些事来,也算不得是佳人了。
比如男人满腹文章去作贼,难道那王法就说他是才子,就不入贼情一案不成?可知那编书的是自己塞了自己的嘴。
再者,既说是世宦书香大家小姐都知礼读书,连夫人都知书识礼,便是告老还家,自然这样大家人口不少,奶母丫鬟伏侍小姐的人也不少,怎么这些书上,凡有这样的事,就只小姐和紧跟的一个丫鬟?你们白想想,那些人都是管什么的,可是前言不答后语?”}且其中终不能不涉于淫滥,以致满纸潘安子建、西子文君,\zhu{潘安子建、西子文君:这里代指才子佳人。
潘安:即潘安仁,晋代文人,著名美男子。
子建:曹植的字,三国时文学家,以才高著称。
西子:即西施.春秋时越国美女。
文君:汉代卓王孙的女儿.新寡后“私奔”文学家司马相如,结为夫妇。
}不过作者要写出自己的那两首情诗艳赋来,故假拟出男女二人名姓,又必旁出一小人其间拨乱,\meng{放笔以情趣世人,并评倒多少传奇。
文气淋漓,字句切实。
}\ping{古今交汇,对套路吐槽的心是一致的。
}亦如剧中之小丑然。
且鬟婢开口即者也之乎,非文即理。
故逐一看去,悉皆自相矛盾,大不近情理之话。
竟不如我半世亲睹亲闻的这几个女子,虽不敢说强似前代书中所有之人,但事迹原委,亦可以消愁破闷,也有几首歪诗熟话,可以喷饭供酒。
至若离合悲欢,兴衰际遇,则又追踪蹑迹,不敢稍加穿凿,\zhu{穿凿:非常牵强地解释,把没有某种意思的说成有某种意思,穿凿附会。
}徒为供人之目而反失其真传者。
\jia{事则实事,然亦叙得有间架、有曲折、有顺逆、有映带、有隐有见、有正有闰,\zhu{闰:偏,对“正”而言。}
以致草蛇灰线、\zhu{草蛇灰线:古代小说评点中较为常见的一个技法术语,它较早渊源于堪舆理论。作为一种小说技法,它或以一种结构线索的形式而存在、或以伏笔照应的形式而存在、或以象征隐喻手法的形式而存在,具有较为复杂的意涵。}
空谷传声、\zhu{空谷传声:人在山谷里发出声音,立刻听到回声。后来借用作评点派文学批评的术语,意谓传其声而隐其形,不着痕迹。}
一击两鸣、\zhu{一击两鸣:指作者描绘某一人或某一事物而实际上连另外的人或事也描绘出来了。}
明修栈道、暗渡陈仓、\zhu{明修栈道、暗渡陈仓:用明显的行为迷惑对方,使人不备的策略。后世小说戏曲评点家用来指小说中明写某一事或某一人,而实际上在暗示另一事或另一人。}
云龙雾雨、\zhu{云龙雾雨:“云龙作雨”或“云龙雾雨”,总的含义是指《红楼梦》在情节安排上出人意表,神妙变化的艺术手法,但每条批评又有各自具体的含义。}
两山对峙、\zhu{两山对峙:是一种对照、对比的艺术手法。}
烘云托月、\zhu{烘云托月:烘云托月本是中国画中一种艺术表现手法,即画月亮并不涂白色或勾勒,而是用较淡或极淡的水墨去烘染月亮周围的云朵,即烘染云朵而衬托出月亮来,简言之即烘托法。这实际是文学创作中的衬托法或反衬法。}
背面傅粉、\zhu{背面傅粉:背面傅粉是中国画的一种绘画技法,在绢或纸的背面打上粉底,然后再作画,背面傅粉是为起到衬托的作用,在文学中,相当于反衬。}
千皴万染诸奇。\zhu{皴:音“村”,一种国画画法。国画山水树石中,表现凹凸阴阳之感及线条、纹理、形态等的笔法。千皴万染这里泛指《红楼梦》中对人物、环境的由简到繁,由粗到细,由浅到深的一步步、一层层的刻画描写,亦即对人物、环境用多次笔墨、多种手法来皴染。}
书中之秘法,亦不复少。
余亦于逐回中搜剔刳剖,\zhu{刳:音“哭“,剖开,挖空。}明白注释,以待高明,再批示误谬。
}今之人,贫者日为衣食所累,富者又怀不足之心,纵一时稍闲,又有贪淫恋色、好货寻愁之事,那里有工夫去看那理治之书?所以,我这一段故事,也不愿世人称奇道妙,也不定要世人喜悦检读,\jia{转得更好。
 }\jia{开卷一篇立意,真打破历来小说窠臼。
阅其笔则是《庄子》《离骚》之亚。
}\jia{斯亦太过。
}只愿他们当那醉馀饱卧之时,或避世去愁之际,把此一玩,岂不省了些寿命筋力?就比那谋虚逐妄去,也省了口舌是非之害、腿脚奔忙之苦。
再者,亦令世人换新眼目,不比那些胡牵乱扯,忽离忽遇,满纸才人淑女、子建文君、红娘小玉等通共熟套之旧稿。
\zhu{红娘小玉:红娘:唐代元稹《会真记》(至元代王实甫衍为杂剧《西厢记》)中崔莺莺的丫鬟。
小玉:唐代蒋防《霍小玉传》中的女主人公。
后文将多次提及《西厢记》及其中人物情节,所以在这里简单介绍大致剧情。
在山西普救寺借宿的书生张珙(字君瑞),偶遇扶柩回乡在寺中西厢借住的原崔相国的女儿崔莺莺,由于互相吟诗而产生爱慕。
叛将孙飞虎带手下慕名围寺,要强抢崔莺莺,三日之内若不交出莺莺,“伽蓝尽皆焚烧,僧俗寸斩,不留一个”。
莺莺的母亲老夫人郑氏宣称谁能救他女儿就将女儿许配他,张生向他一位故旧“白马将军”蒲州杜太守写了一封求救信,由一位僧人(惠明)突出包围送出,杜太守发兵解围。
过后老夫人因门第不当悔婚,只是赠金并让莺莺拜张生为义兄以谢搭救。
张生在悲恸之下患病,莺莺也大为伤痛,后来在莺莺的丫鬟红娘的帮助下,两人暗通书信,并最终成功幽会。
最后私情被老夫人发现,欲责罚二人,但由于红娘据理力争,无可奈何之下,老夫人命令张生上京赶考,如能蟾宫折桂便真的把莺莺许配与他,于是张生进京赴试,考中并回来迎娶莺莺,有情人终成眷属。
}我师意为何如?”\jia{余代空空道人答曰:“不独破愁醒盹,且有大益。
”}\par
空空道人听如此说,思忖半晌,
\zhu{忖[cǔn]:揣度[duó];思量。}
将这《石头记》\jia{本名。
}再检阅一遍,\jia{这空空道人也太小心了,想亦世之一腐儒耳。
}因见上面虽有些指奸责佞、贬恶诛邪之语,\jia{亦断不可少。
}亦非伤时骂世之旨,\jia{要紧句。
}\ping{文字狱?}
及至君仁臣良、父慈子孝,凡伦常所关之处,\zhu{伦常:即封建伦理道德。
伦:人伦,封建社会指人与人之间关系及行为的准则。
《孟子·滕文公上》:“使契为司徒,教以人伦,父子有亲,君臣有义,夫妇有别,长幼有序,朋友有信。
”封建社会以上述君臣、父子、夫妇、兄弟、朋友为五伦,认为是不可改变的常道,亦称五常。
}皆是称功颂德,眷眷无穷,实非别书之可比。
虽其中大旨谈情,亦不过实录其事,又非假拟妄称,\jia{要紧句。
}一味淫邀艳约、私订偷盟之可比。
因毫不干涉时世,\jia{要紧句。
}方从头至尾抄录回来,问世传奇。
\zhu{传奇:流传奇文。}
因空见色,\zhu{空:“空”与下文的“色”、“情”,均佛教用语。
佛教认为“空”乃天地万物的本体.一切终属空虚。
“色”乃万物本体(空)的瞬息生灭的假象;“情”乃对此等假象(色)所产生的种种感情,如爱、憎等等。
这里是借用,巳注入了作家的人生体验。
}由色生情,传情入色,自色悟空,遂易名为情僧,改《石头记》为《情僧录》。
至吴玉峰题曰《红楼梦》。
东鲁孔梅溪则题曰《风月宝鉴》。
\zhu{风月:指男女之情。
宝鉴:宝镜。
}\jia{雪芹旧有《风月宝鉴》之书,乃其弟棠村序也。
今棠村已逝,余睹新怀旧,故仍因之。
}后因曹雪芹于悼红轩中,披阅十载,增删五次,\jia{若云雪芹披阅增删,然则开卷至此这一篇楔子又系谁撰?
\zhu{楔子:音“蝎子”,戏曲、小说的引子。
一般在篇首,用以点明、补充正文。}
足见作者之笔,狡猾之甚。
后文如此处者不少。
这正是作者用画家烟云模糊处,观者万不可被作者瞒蔽了去,方是巨眼。
\zhu{巨眼:比喻善于鉴别的眼力。
}\zhu{这条评语的意思是,虽然正文中记载的曹雪芹只是做了“增删”工作,但是其实曹雪芹是作者,此处是作者曹雪芹用狡猾之笔刻意迷惑读者的。}}纂成目录,分出章回,则题曰《金陵十二钗》。
\zhu{金陵十二钗:金陵,古邑名,楚威王七年(公元前333 年)置,在今南京市。
后即为南京市的别称。
钗:本为妇女的头饰。
旧称女子为“裙钗”或“金钗”。
十二钗,语本《古乐府》:“头上金钗十二行”,原言髻高插钗之多。
唐代白居易《酬牛思黯》诗则用“金钗十二行”借指女子排列之众。
至宋人沈立《海棠百韵》:“金钗人十二,珠履客三千。
“明指十二个女子。
此书又“题曰《金陵十二钗》”,通常认为是由第五回“册子”上所写的十二个女子得名。
}并题一绝云:\par
\hop

   满纸荒唐言,一把辛酸泪!\par
都云作者痴,谁解其中味?\jia{此是第一首标题诗。
} \hop


\jia{能解者方有辛酸之泪,哭成此书。
壬午除夕。
}\jia{书未成,芹为泪尽而逝。
余尝哭芹,泪亦待尽。
每意觅青埂峰再问石兄,奈不遇癞头和尚何!\zhu{癞:音“赖“。癞头:头上生癣或疥疮而致毛发脱落。}怅怅!}\jia{今而后,惟愿造化主再出一芹一脂,是书何幸,余二人亦大快遂心于九泉矣。
甲\sout{午}[申]八\sout{日}[月]泪笔\foot{著名批语。
以前著作引录此批多将“壬午除夕”断归下句,作“书未成,芹为泪尽而逝”的时间定语,并因此成为曹雪芹卒年“壬午说”的主要证据,但“壬午说”并不能解决现存文献资料中的矛盾。
香港梅节先生认为此处应系两批连抄,“壬午除夕”是前一条批语的作批时间,而不是芹逝时间。
其说理由充分,今从之。
}。
}\par
至脂砚斋甲戌抄阅再评,仍用《石头记》。
\par
出则\zhu{“出则”,甲辰本作“出处”,馀本均同甲戌本。
吴恩裕认为“则”字系“处”字简写的草书形讹,近是。
}既明,且看石上是何故事。
按那石上书云:\jia{以石上所记之文。
}\par
当日地陷东南,\zhu{地陷东南:古代神话:共工与颛顼(颛顼:音“专须”)争帝,怒而触不周山,折天柱,绝地维,天倾西北,地不满东南。
见《淮南子·天文训》。
}这东南一隅有处曰姑苏,\zhu{姑苏:苏州的别称,因其西南有姑苏山而得名。
这里是指旧苏州府辖境。
}\jia{是金陵。
}有城曰阊门者,\zhu{阊门:“阊”音“昌”,苏州城的西北门,又名破楚门。
这里代指苏州城。
}
最是红尘中一二等富贵风流之地。
\jia{妙极!是石头口气,惜米颠不遇此石。
\zhu{米颠:即米芾,北宋著名书画家。}
}
这阊门外有个十里\jia{开口先云势利,是伏甄、封二姓之事。
}街,街内有个仁清\jia{又言人情,总为士隐火后伏笔。
}巷,巷内有个古庙,因地方窄狭,\jia{世路宽平者甚少。
}\jia{亦凿。
}
人皆呼作葫芦\jia{糊涂也,故假语从此具焉。
}庙。
\meng{画的虽不依样,却是葫芦。
}庙旁住着一家乡宦,\jia{不出荣国大族,先写乡宦小家,从小至大,是此书章法。
}姓甄,\jia{真。
}\jia{后之甄宝玉亦借此音,后不注。
}
名费,\jia{废。
}字士隐。
\jia{托言将真事隐去也。
}嫡妻封\jia{风。
因风俗来。
}
氏,情性贤淑,深明礼义。
\jia{八字正是写日后之香菱,见其根源不凡。
}家中虽不甚富贵,然本地便也推他为望族了。
\jia{本地推为望族,宁、荣则天下推为望族,叙事有层落。
}因这甄士隐禀性恬淡,不以功名为念,\jia{自是羲皇上人,\zhu{羲皇上人:指太古之人。羲:指伏羲;皇:指娲皇,即女娲。羲、皇是中国神话中之人祖。古人想象伏羲氏以前的人,无忧无虑,生活闲适。脂批这里比为甄士隐。}便可作是书之朝代年纪矣。
总写香菱根基,原与正十二钗无异。
 }\meng{伏笔。
}\ping{不念功名自有儿孙劫,连环劫待着。
}
每日只以观花修竹,酌酒吟诗为乐,倒是神仙一流人品。
只是一件不足:如今年已半百,膝下无儿,\jia{所谓“美中不足”也。
}只有一女,乳名英莲,\jia{设云“应怜”也。
}年方三岁。
\par
 一日,炎夏永昼。
\jia{热日无多。
}士隐于书房闲坐,至手倦抛书,伏几少憩,不觉朦胧睡去。
梦至一处,不辨是何地方。
忽见那厢来了一僧一道,\jia{是方从青埂峰袖石而来也,接得无痕。
}\zhu{那厢:那边。}且行且谈。
\par
 只听道人问道:“你携了这蠢物,意欲何往?”那僧笑道:“你放心,如今现有一段风流公案正该了结,这一干风流冤家,\zhu{风流冤家:“冤家”,原为佛教用语。
《五灯会元》:“佛教慈悲,冤亲平等。
”后既作“仇人”、“对头”解,也用作对所爱之人的昵称.即爱极的反语。
“风流冤家”指极相爱恋之男女。
}尚未投胎入世。
趁此机会,就将此蠢物夹带于中,使他去经历经历。
”那道人道:“原来近日风流冤孽又将造劫历世去不成?\meng{苦恼是“造劫历世”,又不能不“造劫历世”,悲夫!}
\zhu{造劫历世:佛教用语。造:往、到。“造劫历世”,犹言“经劫历世”,指来到人世上经历一番苦难。}但不知落于何方何处?”\par
 那僧笑道:“此事说来好笑,竟是千古未闻的罕事。
只因西方灵河岸上三生石畔,\zhu{西方灵河岸上三生石:西方灵河岸上:作者假想的神仙境界。
西方:原指佛教的发源地天竺(古代印度)。
灵河:原指恒河,今印度人犹称之为“圣水”。
三生:指前生、今生和来生,这是佛教宣扬转世投胎的迷信说法。
三生石:传说唐代李源与和尚圆观交情很好.一次,圆观对他说:十二年后的中秋月夜,在杭州天竺寺外,和你相见,说完就死了。
后来李源如期去杭州,遇见一牧童口唱山歌:“三生石上旧精魂,赏月吟风不要论;惭愧情人远相访,此身虽异性常存。
”这个牧童就是圆观的后身。
见唐代袁郊《甘泽谣·圆观》。
后常用“三生石”比喻因缘前定。
}\jia{妙!所谓“三生石上旧精魂”也。
 }\jia{全用幻。
情之至,莫如此。
今采来压卷,其后可知。
}有绛\jia{点“红”字。
}珠\jia{细思“绛珠”二字岂非血泪乎。
}草一株,\zhu{绛:音“匠”,大红色。
}时有赤瑕\jia{点“红”字“玉”字二。
} \jia{按“瑕”字本注:“玉小赤也,又玉有病也。
”以此命名恰极。
}宫神瑛\jia{单点“玉”字二。
}
侍者,日以甘露灌溉,这绛珠草便得久延岁月。
后来既受天地精华,复得雨露滋养,遂得脱却草胎木质,得换人形,仅修成个女体,终日游于离恨天外,\zhu{离恨天:俗传“三十三天,离恨天最高;四百四病,相思病最苦。
”}饥则食密青果为膳,\zhu{蜜青:谐音“秘情”。
}渴则饮灌愁海水为汤。
\zhu{灌愁海:喻愁深。
}\jia{饮食之名奇甚,出身履历更奇甚,写黛玉来历自与别个不同。
}只因尚未酬报灌溉之德,故其五衷便郁结着一段缠绵不尽之意。
\zhu{五衷:五脏,即心、肝、脾、肺、肾。
亦泛言内心深处。
}\jia{妙极!恩怨不清,西方尚如此,况世之人乎?趣甚警甚! }\jia{以顽石草木为偶,实历尽风月波澜,尝遍情缘滋味,至无可如何,始结此木石因果,以泄胸中悒郁。
古人之“一花一石如有意,不语不笑能留人”,此之谓耶?} \meng{点题处,清雅。
}恰近日神瑛侍者凡心偶炽,\jia{总悔轻举妄动之意。
}乘此昌明太平朝世,意欲下凡造历幻\jia{点“幻”字。
}缘,已在警幻\jia{又出一警幻,皆大关键处。
}仙子案前挂了号。
\ping{下凡亦要排队,警幻业务繁忙。
}警幻亦曾问及,灌溉之情未偿,趁此倒可了结的。
那绛珠仙子道:‘他是甘露之惠,我并无此水可还。
他既下世为人,我也去下世为人,但把我一生所有的眼泪还他,也偿还得过他了。
’\jia{观者至此请掩卷思想,历来小说中可曾有此句?千古未闻之奇文。
}\jia{知眼泪还债,大都作者一人耳。
余亦知此意,但不能说得出。
}\meng{恩情山海债,惟有泪堪还。
}因此一事,就勾出多少风流冤家来,\jia{余不及一人者,盖全部之主惟二玉二人也。
\zhu{这条评语的意思是,在下世的一干风流孽鬼中,只强调了宝玉黛玉,说明他们是本书的主人公。}
}陪他们去了结此案。
”\par
 那道人道:“果是罕闻,实未闻有还泪之说。
\meng{作想得奇!}想来这一段故事,比历来风月事故更加琐碎细腻了。
”那僧道:“历来几个风流人物,不过传其大概以及诗词篇章而已,至家庭闺阁中一饮一食,总未述记。
再者,大半风月故事,不过偷香窃玉、暗约私奔而已,并不曾将儿女之真情发泄一二。
\meng{所以别致。
}想这一干人入世,其情痴色鬼,贤愚不肖者,\zhu{肖:像。
不肖:子不似父,不能继承父业;不贤,无才能;品性不良。
}悉与前人传述不同矣。
”\par
 那道人道:“趁此何不你我也去下世度脱\meng{“度脱”,请问是幻不是幻?}几个,\zhu{度脱:佛家用语。
超度解脱。
}岂不是一场功德?”那僧道:“正合吾意,你且同我到警幻仙子宫中,将这蠢物交割清楚,待这一干风流孽鬼下世已完,你我再去。
\meng{幻中幻,何不可幻?情中情,谁又无情?不觉僧道亦入幻中矣。
}如今虽已有一半落尘,然犹未全集。
”\jia{若从头逐个写去,成何文字?《石头记》得力处在此。
丁亥春。
}道人道:“既如此,便随你去来。
”\par
 却说甄士隐俱听得明白,但不知所云“蠢物”系何东西。
遂不禁上前施礼,笑问道:“二仙师请了。
”那僧道也忙答礼相问。
士隐因说道:“适闻仙师所谈因果,实人世罕闻者。
但弟子愚浊,不能洞悉明白,若蒙大开痴顽,备细一闻,弟子则洗耳谛听,稍能警省,\zhu{警省:佛家用语。
警觉省悟。
}亦可免沉沦之苦。
”\zhu{沉沦:佛家用语。
指在生死轮回中永远不得解脱。
}
二仙笑道:“此乃玄机不可预泄者。
\zhu{玄机:道家用语。
谓玄奥微妙的道理。
这里义同天机。
}到那时只不要忘了我二人,便可跳出火坑矣。
”\zhu{火坑:佛家用语,指苦难的人世。
}\ping{时候未到便悟不得,寥寥无几有慧根之人早早开悟也帮其他人不得,令人想起《黑客帝国》。
还有比这更寂寥的事吗?生在红尘便是个痴迷,细细想来何必早醒。
便是有法子让世人都醒了,那怕也不是红尘了。
}士隐听了,不便再问,因笑道:“玄机不可预泄,但适云‘蠢物’,不知为何,或可一见否?”那僧道:“若问此物,倒有一面之缘。
”说着,取出递与士隐。
士隐接了看时,原来是块鲜明美玉,上面字迹分明,镌着“通灵宝玉”四字,\jia{凡三四次始出明玉形,隐屈之至。
}
后面还有几行小字。
正欲细看时,那僧便说已到幻境,\jia{又点“幻”字,云书已入幻境矣。
}\meng{幻中言幻,何等法门。
}便强从手中夺了去,与道人竟过一大石牌坊,那牌坊上大书四字,乃是“太虚幻境”。
\zhu{太虚幻境:作者虚拟的仙境。
太虚:空幻虚无的意思。
}\jia{四字可思。
}
两边又有一副对联,道是:\qi{无极太极之轮转,色空之相生,四季之随行,皆不过如此。
\zhu{无极:中国古代哲学中称派生宇宙万物的本源。
《老子》:“知其白,守其黑,为天下式;为天下式,常德不忒,复归于无极。
”太极:中国古代道家指原始混沌之气。
《易·系辞》:“易有太极,是生两仪,两仪生四象,四象生八卦。”
无极太极轮转即本源推为万物,万物归于本源。
色,佛教谓凡诸事物如五根(眼耳鼻舌身)五境(色声香味触)等足以引起变碍者,皆称色。
空,佛教指超乎色相现实的境界为空。
色空之相生,指佛教的 空即是色,色即是空。}
}
\hop

 假作真时真亦假,无为有处有还无。
\jia{叠用真假有无字,妙!}
 \hop
 

士隐意欲也跟了过去,方举步时,忽听一声霹雳,有若山崩地陷。
士隐大叫一声,定睛一看,\meng{真是大警觉大转身。
}只见烈日炎炎,芭蕉冉冉,\jia{醒得无痕,不落旧套。
}梦中之事便忘了对半。
\jia{妙极!若记得,便是俗笔了。
}\par
 又见奶姆正抱了英莲走来。
士隐见女儿越发生得粉妆玉琢,乖觉可喜,\zhu{乖觉:机警,聪敏。
}便伸手接来,抱在怀中,斗他顽耍一回,又带至街前,看那过会的热闹。
\zhu{过会:旧时遇节庆,随地聚演百戏杂耍、笙乐鼓吹之类,观者如潮。
}方欲进来时,只见从那边来了一僧一道,\jia{所谓“万境都如梦境看”也。
}那僧则癞头跣足,\zhu{跣音“显”。跣足:光着脚,没穿鞋袜。}那道则跛足蓬头,\jia{此则是幻像。
}疯疯癫癫,挥霍谈笑而至。
\zhu{挥霍:亦作“挥擂”。
《韵会》:“摇手曰挥,反手曰擂。
”本谓动作轻捷,这里是挥洒自如的意思。
}及至到了他门前,看见士隐抱着英莲,那僧便哭起来,\jia{奇怪!所谓情僧也。
}又向士隐道:“施主,你把这有命无运,\zhu{有命无运:旧时“算命”,用人出生的年、月、日、时所属的干支和金、木、水、火、土五行的生克来推断人的吉凶祸福;称一生的境遇好坏为“命”,一段时间的遭际为“运”。
有命无运,这里意谓平生“行运”乖逆.遭际堪悲。
}累及爹娘\jia{八个字屈死多少英雄?屈死多少忠臣孝子?屈死多少仁人志士?屈死多少词客骚人?今又被作者将此一把眼泪洒与闺阁之中,见得裙钗尚遭逢此数,况天下之男子乎?看他所写开卷之第一个女子便用此二语以定终身,则知托言寓意之旨,谁谓独寄兴于一“情”字耶!}\jia{武侯之三分,
\zhu{武侯即诸葛亮,被封为武乡侯。
他助刘备建立蜀汉政权,与曹操、孙权三分天下。诸
葛亮多次伐曹魏,力图统一天下,最后病死五丈原军中,壮志未酬。}
武穆之二帝,
\zhu{武穆即岳飞,南宋抗金名将,以“莫须有”(也许有)的罪名被杀害,孝宗时谥武穆。
二帝指被金兵掳去的宋徽宗、宋钦宗父子。}
二贤之恨,
\zhu{二贤指伯夷、叔齐二兄弟,他们反对周武王出兵讨伐商王朝。
武王灭商后,兄弟二人又逃避到首阳山,不食周粟而死。
伯夷、叔齐未能阻止武王进军,最后商王朝灭亡。这即是脂批说的“二贤之恨”。}
及今不尽,况今之草芥乎?
\zhu{这条批语列举了诸葛亮、岳飞、伯夷叔齐的例子,证明即使英雄志士,尚且“有命无运”,壮志未酬,抱恨终生,更何况现在的平民百姓呢?}
}\jia{家国君父事有大小之殊,其理其运其数则略无差异。
知运知数者则必谅而后叹也。
}之物,抱在怀内作甚?”士隐听了,知是疯话,也不去睬他。
那僧还说:“舍我罢,舍我罢!”士隐不耐烦,便抱着女儿撤身进去,\meng{如果舍出,则不成幻境矣。
行文至此,又不得不有此一语。
}那僧乃指着他大笑,口内念了四句言词,道是:\par
\hop

 惯养娇生笑你痴,\jia{为天下父母痴心一哭。
}\par
 菱花空对雪澌澌。
\jia{生不遇时。
遇又非偶。
}\zhu{澌澌(音“司司”):形容雪声。“菱花”句:隐喻英莲被呆霸王薛蟠强占作妾的不幸遭遇。
菱花:指后来英莲改名香菱。
雪:谐音“薛”,指薛蟠。
菱在夏日开花而竟遇冰雪,喻英莲“生不逢时,遇又非偶”,定然遭到摧残。
}\par
 好防佳节元宵后,\jia{前后一样,不直云前而云后,是讳知者。
\zhu{雍正五年十二月,始因曹頫“勒索驿站”、“行为不端,织造款项亏空甚多”、“将家中财物暗移他处,企图隐蔽”,被撤职抄家。曹家被抄在元宵节前,不直接说“前”,而用“后”瞒过读者。}
}\zhu{“元宵后”实际点出第五十三回“荣国府元宵开夜宴”将成为贾府由盛而衰的转折点。}\par
 便是烟消火灭时。
\jia{伏后文。
}
\zhu{
本回后文,三月十五(元宵后),因葫芦庙炸供失火导致甄家焚毁。
甄家“烟消火灭”,其实是喻指贾府最终“落了片白茫茫大地真干净”的结局,
可能贾家最后也是在一场大火中化为灰烬。}
\par
\hop

士隐听得明白,心下犹豫,意欲问他们来历。
只听道人说道:“你我不必同行,就此分手,各干营生去罢。
三劫后,\jia{佛以世谓“劫”,凡三十年为一世。
三劫者,想以九十春光寓言也。
}我在北邙山等你,\zhu{北邙(邙:音“芒”)山:也作“北芒山”,即邙山。
在今河南省洛阳市北。
东汉及北魏的王侯公卿多葬于此。
后常被用来泛指墓地。
}会齐了同往太虚幻境销号。
”那僧道:“妙,妙,妙!”说毕,二人一去,再不见个踪影了。
士隐心中此时自忖:这两个人必有来历,该试一问,如今悔却晚也。
\par
 这士隐正痴想,忽见隔壁\jia{“隔壁”二字极细极险,记清。
}葫芦庙内寄居的一个穷儒,姓贾名化,\jia{假话。
妙!}表字时飞,\jia{实非。
妙!}别号雨村\jia{雨村者,村言粗语也。
言以村粗之言演出一段假话也。
}者走了出来。
这贾雨村原系胡州\jia{胡诌也。
\zhu{诌:音“周”,信口胡说,编瞎话。
}}人氏,也是诗书仕宦之族,因他生于末世,\jia{又写一末世男子。
}父母祖宗根基一尽,人口衰丧,只剩得他一身一口,在家乡无益。
\meng{形容落破诗书子弟,逼真。
}因进京求取功名,再整基业。
自前岁来此,又淹蹇住了,\zhu{淹蹇(音“烟简”):即偃蹇。
原指境遇困顿、不得意,这里是耽搁、阻滞的意思。
}暂寄庙中安身,每日卖字作文为生,\meng{“庙中安身”、“卖字为生”,想是过午不食的了。
}故士隐常与他交接。
\jia{又夹写士隐实是翰林文苑,\zhu{翰林:谓文翰荟萃之所,犹词坛文苑。
}非守钱虏也,直灌入“慕雅女雅集苦吟诗”一回。
}当下雨村见了士隐,忙施礼陪笑道:“老先生倚门伫望,敢是街市上有甚新闻否?”士隐笑道:“非也,适因小女啼哭,引他出来作耍,正是无聊之甚,兄来得正妙,请入小斋一谈,彼此皆可消此永昼。
”说着,便令人送女儿进去,自携了雨村来至书房中。
小童献茶。
方谈得三五句话,忽家人飞报:“严\jia{“炎”也。
炎既来,火将至矣。
}老爷来拜。
”士隐忙的起身谢罪道:“恕诳驾之罪,\zhu{诳(音“狂”)或作“诓”,欺骗的意思。
驾:对客人的尊称。
诳驾:邀来客人后.因故不能陪待,向客人道歉之词,犹言“失陪”。
}略坐,即来陪。
”雨村忙起身亦让道:“老先生请便。
晚生乃常造之客,稍候何妨。
”\meng{世态人情,如闻其声。
}说着,士隐已出前厅去了。
\par
 这里雨村且翻弄书籍解闷。
忽听得窗外有女子嗽声,雨村遂起身往窗外一看,原来是一个丫鬟,在那里撷花,\zhu{撷(音“协”):采摘、捋取。
唐代王维《相思》:“愿君多采撷,此物(红豆)最相思。
”}生得仪容不俗,眉目清朗,\jia{八字足矣。
}虽无十分姿色,却亦有动人之处。
\jia{更好。
这便是真正情理之文。
可笑近之小说中满纸“羞花闭月”等字。
这是雨村目中,又不与后之人相似。
}
雨村不觉看得呆了。
\jia{今古穷酸色心最重。
}那甄家丫鬟撷了花,方欲走时,猛抬头见窗内有人,敝巾旧服,虽是贫窘,然生得腰圆背厚,面阔口方,更兼剑眉星眼,直鼻权腮。
\zhu{权腮:俗称颧(音“权”)骨腮,指人颧骨(颧骨:眼睛下边两腮上面突出的骨头)长得很高,相法认为是一种贵相。
沈括《梦溪笔谈·人事》:“公满面权骨.不十年必总枢柄。
”}\jia{是莽、操遗容。
}\jia{最可笑世之小说中,凡写奸人则用“鼠耳鹰腮”等语。
}这丫鬟忙转身回避,心下乃想:“这人生得这样雄壮,却又这样褴褛,想他定是我家主人常说的什么贾雨村了,每有意帮助周济,只是没甚机会。
我家并无这样贫穷亲友,想定是此人无疑了。
怪道又说他必非久困之人。
”如此想,不免又回头两次。
\jia{这方是女儿心中意中正文。
又最恨近之小说中满纸红拂紫烟。
\zhu{红拂:唐代杜光庭《虬髯客传》(虬髯:音“求然”,蜷曲的须髯。
)的女主人公,姓张,初为隋朝大臣杨素的侍女,后私奔李靖。
她在杨家时手执红拂(掸灰尘的用具),见李靖时又自称“红拂妓”。
紫烟:袁紫烟,隋末唐初的传奇女子,出自《隋唐演义》。
}
}\meng{如此忖度,岂得为无情?}雨村见他回了头,便自为这女子心中有意于他,\jia{今古穷酸皆会替女妇心中取中自己。
}便狂喜不禁,自为此女子必是个巨眼英豪,\zhu{巨眼英豪:有远见,能识鉴人才的人。
}风尘中之知己也。
\zhu{风尘:这里指扰攘的尘世,又有旅居在外,备尝艰辛之意。
}\meng{在此处已把种点出。
}一时小童进来,雨村打听得前面留饭,不可久待,遂从夹道中自便出门去了。
士隐待客既散,知雨村自便,也不去再邀。
\par
 一日,早又中秋佳节。
士隐家宴已毕,乃又另具一席于书房,却自己步月至庙中来邀雨村。
\jia{写士隐爱才好客。
}原来雨村自那日见了甄家之婢曾回头顾他两次,自为是个知己,便时刻放在心上。
\meng{也是不得不留心。
不独因好色,多半感知音。
}今又正值中秋,不免对月有怀,因而口占五言一律云:\zhu{口占五言一律:口占:随口吟成,与下文“口号”义同。
五言一律:每句五个字的律诗一首。
}\jia{这是第一首诗。
后文香奁闺情皆不落空。
余谓雪芹撰此书,中亦有传诗之意。
}\par
\hop

 未卜三生愿,频添一段愁。
\zhu{未卜:不能预知。
频:屡屡;时时。
全句意为:同娇杏结姻缘的愿望不知能否实现。
}\par
 闷来时敛额,行去几回头。
\zhu{敛额:皱眉头。
全句意为:把这段愁绪时刻挂在心上。
}\par
 自顾风前影,谁堪月下俦?\zhu{自顾风前影:由“顾影自怜”化出。
堪:能,配得上。
俦(音“愁”):伴侣。
月下俦:成婚配的意思。
传说唐代韦固在宋城遇一老人在月下检天下婚姻之书,囊中并有赤绳.一系男女之足,则必成夫妇(见李复言《续玄怪录》)。
后因称管婚姻之神为“月下老人”或“月老”,也用来代称媒人。
全句意为:风前自顾身影,有谁能赏识自己,成为我的终身伴侣呢。
}\par
 蟾光如有意,先上玉人楼。
\zhu{蟾光:指月光,亦寓“蟾宫折桂”(即科举及第)之意。
玉人楼:美人居住的地方,玉人,指娇杏。
全句意为:月光如真有情意,希望先照玉人的妆楼。
暗含若得科举及第,定先到玉人楼上求婚之意。
}\par
\hop
雨村吟罢,因又思及平生抱负,苦未逢时,乃又搔首对天长叹,复高吟一联云:\par
\hop

 玉在匮中求善价,钗于奁内待时飞。
\zhu{上句意谓美玉藏在匣子里希望卖得好价钱。
匮(音“愧”):柜子。
《论语·
子罕》:“子贡曰:‘有美玉于斯(这里),韫(韫:音“运”,藏)匵(匵:音“读”,即“椟”,木匣;木柜。
)而藏诸(藏起来呢)?求善贾(价)而沽诸(出卖呢)?’子曰:‘沽之哉!沽之哉?我待贾(等待好的价钱)者也。
’” 下句意谓玉钗放在镜盒中,等待时机而飞腾。
传说汉武帝元鼎元年,有神女留一玉钗,昭帝时,有人偷开匣子,不见玉钗,只见一只白燕从中飞出,升天而去(见郭宪《洞冥记》)。
全句意为:贾雨村自比玉、钗,企图得到赏识,以求飞黄腾达。
}\jia{前用二玉合传,今用二宝合传,自是书中正眼。
\zhu{二玉合传:指前文写到甄士隐在梦中听一僧一道说到的“神瑛侍者”、“绛珠仙草”一段话,暗指宝玉和黛玉。二宝合传暗指宝玉和宝钗。}
}\jia{表过黛玉,则紧接上宝钗。
}\meng{偏有些脂气。
}\par

恰值士隐走来听见,笑道:“雨村兄真抱负不浅也!”雨村忙笑道:“岂敢!不过偶吟前人之句,何敢狂诞至此。
”因问:“老先生何兴至此?”士隐笑道:“今夜中秋,俗谓‘团圆之节’,想尊兄旅寄僧房,不无寂寞之感,故特具小酌,邀兄到敝斋一饮,不知可纳芹意否?”\zhu{芹意:古时有人认为芹菜的味道很美,就向乡豪称赞,乡豪尝后,却觉得很难吃。
见《列子·杨朱篇》。
后比喻自己欣赏的事物推荐给别人,却无法获得认同。
亦用为人对所献东西或意见的自谦之词。
后人常用“献芹”、“芹意”等作为送礼或请客的谦词。
}雨村听了,并不推辞,\meng{不推辞,语便不入故套。
}便笑道:“既蒙谬爱,何敢拂此盛情。
”\jia{写雨村豁达,气象不俗。
}说着,便同了士隐复过这边书院中来。
\par
 须臾茶毕,早已设下杯盘,那美酒佳肴自不必说。
二人归坐,先是款斟漫饮,次渐谈至兴浓,不觉飞觥限斝起来。
\zhu{飞觥限斝:觥筹交错、饮宴尽欢的情景。
觥(音“工”)、斝(音“甲”):两种古代酒器,前者为角形.后者圆口平底。
飞觥:挥杯。
限斝:行酒令时限定饮酒数量。
酒令:古代宴会中,佐饮助兴的游戏。
推一人为令官,其余的人听其号令,轮流说诗词或做其他游戏,违令或输的人饮酒。
}当时街坊上家家箫管,户户弦歌,当头一轮明月,飞彩凝辉,二人愈添豪兴,酒到杯干。
雨村此时已有七八分酒意,狂兴不禁,乃对月寓怀,口号一绝云:\zhu{口号:犹言“口占”,不借笔墨,随口吟成。
}\par
\hop
 时逢三五便团圆,\jia{是将发之机。
}
\zhu{三五:十五,指阴历十五日。}
满把晴光护玉栏。
\jia{奸雄心事,不觉露出。
}\zhu{
满把:满握。
满把晴光:极言月光皎洁充盈。
护玉栏:玉石栏杆沉浸在皎洁的月光里。
}\par
 天上一轮才捧出,人间万姓仰头看。
\zhu{据说赵匡胤未登极时,曾拿《咏月》诗给徐铉看,徐铉读到“未离海底千山黑,才到中天万国明”这两句时,认为帝王之兆已显。
见宋代陈师道《后山诗话》。
贾诗后两句所抒胸腌类此,故甄士隐说他“飞腾之兆已见”。
}\jia{这首诗非本旨,不过欲出雨村,不得不有者。
用中秋诗起,用中秋诗收,又用起诗社于秋日。
\zhu{中秋诗起:这里贾雨村的诗。中秋诗收:第七十六回中秋之夜,史湘云与林黛玉在一起联诗。起诗社:第三十七回众姐妹结海棠社。}
所叹者三春也,\zhu{三春:迎春、探春、惜春。}却用三秋作关键。
}\ping{感觉有点僭越呢。
}\par
\hop
士隐听了,大叫:“妙哉!吾每谓兄必非久居人下者,今所吟之句,飞腾之兆已见,不日可接履于云霓之上矣。
\zhu{接履于云霓之上:犹言平步青云。
接履:一步紧接一步。
云霓:喻高位。
}可贺,可贺!”\meng{伏笔,作巨眼语。
妙!}乃亲斟一斗为贺。
\jia{这个“斗”字莫作升斗之斗看。
}\jia{可笑。
}雨村因干过,叹道:“非晚生酒后狂言,若论时尚之学,\zhu{时尚之学:时人所崇尚的学问。
这里指明清科举考试用的“八股文”和“试帖诗”等。
}\jia{四字新而含蓄最广,若必指明,则又落套矣。
}晚生也或可去充数沽名,只是目今行囊、路费一概无措,\zhu{目今:现在,当前。
}神京路远,非赖卖字撰文可能到者。
”士隐不待说完,便道:“兄何不早言。
愚每有此心,但每遇兄时,兄并未谈及,愚故未敢唐突。
今既及此,愚虽不才,‘义利’二字却还识得。
\zhu{“义利”二字:《论语·里仁》:“君子喻于义,小人喻于利。
”义:道义。
利:功利,这里指钱财。
}\meng{“义利”二字,时人故自不识。
}\ping{甄为义,贾为利。
}且喜明岁正当大比,兄宜作速入都,春闱一战,\zhu{大比、春闱:明清科举制,考试分为三级。
第一级是院试.考府县的童生,考取的为“生员”(秀才);第二级是乡试,考一省的生员,考取的为“举人”;第三级是会试,考全国的举人,考取的为“贡士”(再经殿试赐进士出身)。
乡试、会试均三年一科,也称“大比”。
乡试在秋天,称为“秋闱”;会试在春天,称为“春闱”。
闱(音“围”):指考场。
这里的“大比”是指会试。
}方不负兄之所学也。
其盘费馀事,弟自代为处置,亦不枉兄之谬识矣!”当下即命小童进去,速封五十两白银,并两套冬衣。
\jia{写士隐如此豪爽,又全无一些粘皮带骨之气相,愧杀近之读书假道学矣。
}又云:“十九日乃黄道之期,兄可即买舟西上,待雄飞高举,明冬再晤,岂非大快之事耶!”雨村收了银、衣,不过略谢一语,并不介意,仍是吃酒谈笑。
\jia{写雨村真是个英雄。
}\meng{托大处,\zhu{托大:骄傲自大。}即遇此等人,又不得太琐细。
}\ping{什么真英雄?}那天已交三鼓,二人方散。
\zhu{北宋时开始将每个时辰分为“初”、“正”两部分,分十二时辰为二十四,称“小时”。
子时从当天夜里十一点到第二天凌晨一点,子初从夜里十一点开始到零点,子正从零点开始到凌晨一点。
以此类推,子丑寅卯辰巳午未申酉戌亥代表一天中的十二个时辰。
清代正式规定一昼夜为九十六刻,每个时辰八刻,又区分为上四刻和下四刻:初初刻、初一刻、初二刻、初三刻、正初刻、正一刻、正二刻、正三刻;正初刻又叫初四刻,下一时辰之初初刻又叫正四刻。
汉代皇宫中值班人员分五个班次,按时更换,叫“五更”,由此便把一夜分为五更,每更为一个时辰。
戌时为一更,亥时为二更,子时为三更,丑时为四更,寅时为五更。
由于古代报更使用击鼓方式,故又以鼓指代更。
如杜甫《阁夜》:“五更鼓角声悲壮,三峡星河影动摇。
”又如,白居易的《长恨歌》:“迟迟钟鼓初长夜,耿耿星河欲曙天。
”其中的“鼓角”、“钟鼓”都是古时用来打更的器具。}
\par
 士隐送雨村去后,回房一觉,直至红日三竿方醒。
\jia{是宿酒。
}因思昨夜之事,意欲再写两封荐书,与雨村带至神京,使雨村投谒个仕宦之家,为寄足之地。
\jia{又周到如此。
}因使人过去请时,那家人去了回来说:“和尚说,贾爷今日五鼓已进京去了,也曾留下话与和尚转达老爷,说:‘读书人不在黄道黑道,\zhu{黄道黑道:为我国古代天文学的专名,黄道指日,黑道指月。
《汉书·天文志》:“日有中道”,“中道者黄道,一曰光道。
”又云:“月有九行者,黑道二。
”后星占迷信者将每日的干支阴阳分为“黄道”和“黑道”,黄道主吉,黑道主凶。
}总以事理为要,不及面辞了。
’”\jia{写雨村真令人爽快。
}士隐听了,也只得罢了。
\par
 真是闲处光阴易过,倏忽又是元宵佳节矣。
\zhu{倏:音“述”,急速。
}士隐命家人霍启\jia{妙!祸起也。
此因事而命名。
}抱了英莲去看社火花灯,\zhu{社火花灯:这里指元宵节灯火。
社:社日。
旧时祭祀土神之日,分春秋两祭,立春后第五个戊日为春社,立秋后第五个戊日为秋社。
戊日是按六十甲子的排列顺序从老黄历上推,每六十天为一轮,其中凡是逢戊子、戊寅、戊辰、戊午、戊申、戊戌这六天就叫戊日,称“六戊”,也叫“明戊”。
社火:社日扮演的各种杂戏。
花灯:旧时正月十五元宵节有放花灯的习俗。
}半夜中,霍启因要小解,便将英莲放在一家门槛上坐着。
待他小解完了来抱时,那有英莲的踪影?急得霍启直寻了半夜,至天明不见,那霍启也就不敢回来见主人,便逃往他乡去了。
那士隐夫妇,见女儿一夜不归,便知有些不妥,再使几个人去寻找,回来皆云连音响皆无。
夫妻二人,半世只生此女,一旦失落,岂不思想,因此昼夜啼哭,几乎不曾寻死。
\jia{喝醒天下父母之痴心。
}\meng{天下作子弟的,看了想去。
}看看一月,\zhu{看看:估量时间之词。有渐渐、眼看着、转瞬间等意思。}士隐先就得了一病,当时封氏孺人也因思女构疾,\zhu{孺人:《礼记·曲札下》:“天子之妃曰后,诸侯曰夫人,大夫曰孺人,士曰妇人,庶人曰妻。
”孺人在明清为七品官之母或妻的封号。
旧时也通用为妇人的尊称。
}日日请医疗病。
\par
 不想这日三月十五,葫芦庙中炸供,\zhu{炸供:油炸供神用的食品。
}那些和尚不加小心,致使油锅火逸,便烧着窗纸。
此方人家多用竹篱木壁者,\jia{土俗人风。
}\meng{交竹滑溜婉转。
\zhu{交竹:可能是“交代”的错讹。}
}大抵也因劫数,于是接二连三,牵五挂四,将一条街烧得如火焰山一般。
\jia{写出南直召祸之实病。
\zhu{
南直:南直隶简称。直接隶属于南京的地区称“南直隶”。
南直召祸:指曹家被抄。
雍正元年,苏州织造李煦被抄家冶罪;
雍正四年,傅鼐被革职抵罪发往黑龙江、平郡王被革爵圈禁;
雍正五年,李煦被流放、曹家被抄、杭州织造孙文成被革职。
五年间,曹家及其几门亲戚“接二连三,牵五挂四”地相继败落,
脂评借葫芦庙失火事点曹家招祸之“实病”。
}
}彼时虽有军民来救,那火已成了势,如何救得下去?直烧了一夜,方渐渐熄去,也不知烧了几家。
只可怜甄家在隔壁,早已烧成一片瓦砾场了。
只有他夫妇并几个家人的性命不曾伤了。
急得士隐惟跌足长叹而已。
\zhu{跌足:跺脚。
}只得与妻子商议,且到田庄上去安身。
偏值近年水旱不收,鼠盗蜂起,无非抢粮夺食,鼠窃狗偷,民不安生,因此官兵剿捕,难以安身。
士隐只得将田庄都折变了,便携了妻子与两个丫鬟投他岳丈家去。
\par
他岳丈名唤封肃,\qi{风俗。
}本贯大如州人氏,\jia{托言大概如此之风俗也。
}虽是务农,家中都还殷实。
今见女婿这等狼狈而来,心中便有些不乐。
\jia{所以大概之人情如是,风俗如是也。
}\jia{大都不过如此。
}幸而\meng{若非“幸而”,则有不留之意。
}士隐还有折变田地的银子未曾用完,拿出来托他随分就价薄置些须房地,为后日衣食之计。
那封肃便半哄半赚,些须与他些薄田朽屋。
士隐乃读书之人,不惯生理稼穑等事,\zhu{稼穑:音“价色”,种植与收割,泛指农业劳动。}勉强支持了一二年,越觉穷了下去。
封肃每见面时,便说些现成话,且人前人后又怨他们不善过活,只一味好吃懒作\jia{此等人何多之极!}等语。
士隐知投人不着,心中未免悔恨,再兼上年惊唬,\zhu{唬:同“吓”。
}急忿怨痛,已有积伤,暮年之人,贫病交攻,竟渐渐露出那下世的光景来。
\zhu{下世:此指死亡。
全句是指快要死亡、不久于世的意思。
}\meng{几几乎。
\zhu{几几乎:犹几乎。}
世人则不能止于几几乎,可悲!观至此不……(下缺)}\par
 可巧这日,拄了拐挣挫在街前散散心时,忽见那边来了一个跛足道人,疯狂落脱,\zhu{落脱:即“落拓”、“落托”。
这里是行为狂放的意思。
}麻屣鹑衣,\zhu{麻屣鹑衣:麻屣(屣:音“洗”):麻鞋。
鹑(音“纯”):鹌鹑,鸟名,其尾短秃,如补绽百结,故称破烂衣服为鹑衣。
《荀子·大略》:“子夏贫,衣若县(悬)鹑。
”}口内念着几句言词,道是:\par
\hop
 世人都晓神仙好,惟有功名忘不了!\par
 古今将相在何方?荒冢一堆草没了。
\par
 世人都晓神仙好,只有金银忘不了!\par
 终朝只恨聚无多,及到多时眼闭了。
\par
 世人都晓神仙好,只有姣妻忘不了!\par
 君生日日说恩情,君死又随人去了。
\par
 世人都晓神仙好,只有儿孙忘不了!\par
 痴心父母古来多,孝顺儿孙谁见了? \par
 \hop
士隐听了,便迎上来道:“你满口说些什么?只听见些‘好’‘了’‘好’‘了’。
”那道人笑道:“你若果听见‘好’‘了’二字,还算你明白。
可知世上万般,好便是了,了便是好。
若不了,便不好,若要好,须是了。
我这歌儿,便名《好了歌》。
”士隐本是有宿慧的,\zhu{宿慧:佛家用语。
指超越常人的智慧,认为这种智慧是宿世(即前世)带来的。
}一闻此言,心中早已彻悟,\zhu{彻悟:即佛教所说的大彻大悟,看破红尘。
}因笑道:“且住!待我将你这《好了歌》解注出来何如?”道人笑道:“你解,你解。
”士隐乃说道:\qi{要写情要写幻境,偏先写出一篇奇人奇境来。
}
\hop

 陋室空堂,当年笏满床,\zhu{笏满床:形容家中做大官的人很多。
笏(音“户”):一名“手板”。
封建时代臣僚上朝时手中所拿的狭长板子,用象牙或木、竹片制成,可作临时记事之用。
《旧唐书·崔神庆传》:“开元中,神庆子琳等皆至大官。
……每岁时家宴,组堁辉映,以一榻置笏,重叠于其上。
”后误传为唐代汾阳郡王郭子仪家的事,并将它编成《满床笏》(一名《打金枝》)一剧。
}\jia{宁、荣未有之先。
}\par
 衰草枯杨,曾为歌舞场。
\jia{宁、荣既败之后。
}\par
 蛛丝儿结满雕梁,\jia{潇湘馆、紫芸轩等处。
}\par
 绿纱今又糊在蓬窗上。
\jia{雨村等一干新荣暴发之家。
}\jia{先说场面,忽新忽败,忽丽忽朽,已见得反覆不了。
}\par
 说什么脂正浓,粉正香,如何两鬓又成霜?\jia{宝钗、湘云一干人。
}\par
 昨日黄土陇头送白骨,\zhu{黄土陇头:指坟墓。
陇(音“拢”):通“垄”,田中高地;坟墓。
}\jia{黛玉、晴雯一干人。
}\par
 今宵红灯帐底卧鸳鸯。
\jia{一段妻妾迎新送死,倏恩倏爱,倏痛倏悲,缠绵不了。
\zhu{倏:音“述”,急速。
}}\par
 金满箱,银满箱,\jia{熙凤一干人。
}\par
 展眼乞丐人皆谤。
\jia{甄玉、贾玉一干人。
}\par
 正叹他人命不长,那知自己归来丧!\jia{一段石火光阴,悲喜不了。
风露草霜,富贵嗜欲,贪婪不了。
}\par
 训有方,保不定日后\jia{言父母死后之日。
}作强梁。
\zhu{强梁:横暴;蛮不讲理。
《庄子·山木》:“从其强梁。
”唐代陆德明《释文》:“强梁,多力也。
”这里指强盗。
}\jia{柳湘莲一干人。
}\par
 择膏粱,谁承望流落在烟花巷!\zhu{择膏粱:意谓挑选富贵人家子弟作婿。
膏:脂肪;油。
粱:精米。
膏粱:本指精美的饭菜,这里用作“膏粱子弟”的省称。
烟花巷:旧时妓院聚集的地方。
烟花:歌女;娼妓。
}\jia{一段儿女死后无凭,生前空为筹划计算,痴心不了。
}\par
因嫌纱帽小,致使锁枷扛,\jia{贾赦、雨村一干人。
}\par
 昨怜破袄寒,今嫌紫蟒长。
\zhu{紫蟒:紫色的蟒袍。
紫:古代按官阶等级穿着不同颜色的公服;唐制,亲王及三品服用紫色。
}\jia{贾兰、贾菌一干人。
}\jia{一段功名升黜无时,强夺苦争,喜惧不了。
}\par
 乱烘烘你方唱罢我登场,\jia{总收。
}\jia{总收古今亿兆痴人,共历幻场,此幻事扰扰纷纷,无日可了。
}\par
 反认他乡是故乡。
\zhu{反认他乡是故乡:这里把现实人生比作暂时寄居的他乡,而把超脱尘世的虚幻世界当作人生本源的故乡;因而说那些为功名利禄、姣妻美妾、儿女后事奔忙而忘掉人生本源的人是错将他乡当作故乡。
}\jia{太虚幻境青埂峰一并结住。
}\par
 甚荒唐,到头来都是为他人作嫁衣裳!\zhu{为他人作嫁衣裳:喻白白替他人奔忙,死后一切皆空。
唐代秦韬玉《贫女》诗:“苦恨年年压金线,为他人作嫁衣裳。
”}\jia{语虽旧句,用于此妥极是极。
苟能如此,便能了得。
}\jia{此等歌谣原不宜太雅,恐其不能通俗,故只此便妙极。
其说得痛切处,又非一味俗语可到。
}\qi{谁不解得世事如此,有龙象力者方能放得下。
}\par
\hop
那疯跛道人听了,拍掌笑道:“解得切,解得切!”士隐便笑一声“走罢!”\jia{如闻如见。
}\jia{“走罢”二字真悬崖撒手,若个能行?}\meng{一转念间登彼岸。
}将道人肩上褡裢抢了过来背着,\zhu{搭链:一种中间开口而两端装钱物的长方口袋,小的可以挂在腰带上,大的可以搭在肩膀上。
}竟不回家,同了疯道人飘飘而去。
\par
 当下烘动街坊,众人当作一件新闻传说。
封氏闻得此信,哭个死去活来,\ping{出世须趁早,不然少不得牵累旁人。
}只得与父亲商议,遣人各处访寻,那讨音信?无奈何,少不得依靠着他父母度日。
幸而身边还有两个旧日的丫鬟伏侍,主仆三人,日夜作些个针线发卖,帮着父亲用度。
那封肃虽然日日抱怨,也无可奈何了。
\par
 这日,那甄家的大丫鬟在门前买线,忽听得街上喝道之声,众人都说新太爷到任。
丫鬟于是隐在门内看时,只见军牢快手,\zhu{军牢快手:封建官吏手下执行缉捕、防卫和行刑的隶卒。
官僚出巡,常由他们前呼后拥,以示威势。
}一对一对的过去,俄而大轿内抬着一个乌帽猩袍的官府过去。
\zhu{猩:深红、血红。}
\jia{雨村别来无恙否?可贺可贺。
}\jia{所谓“乱哄哄,你方唱罢我登场”是也。
}丫鬟倒发了个怔,自思这官好面善,倒像在那里见过的。
于是进入房中,也就丢过,不在心上。
\jia{是无儿女之情,故有夫人之分。
}\meng{起初到底有心乎?无心乎?}至晚间,正该歇息之时,忽听一片声打的门响,许多人乱嚷,说:“本府太爷差人来传人问话。
”\meng{不忘情的先写出头一位来了。
}封肃听了,唬得目瞪口呆,不知有何祸事。
 \par
\qi{总评:出口神奇,幻中不幻。
文势跳跃,情里生情。
借幻说法,而幻中更自多情,因情捉笔,而情里偏成痴幻。
试问君家识得否,色空空色两无干。
}
\ping{红楼梦的解读应该从文本角度出发,而不是陷入自作聪明,牵强附会的考证之中。
红楼梦是一部对于青春的颂诗与挽歌:红楼梦不仅描写了十几岁的少男少女在大观园中这个青春王国里的懵懂情愫,是一首对于青春的颂歌;也感概了青春逝去,花落人亡,大观园的美好神话终将散场的不舍无奈,是一首对于青春的挽歌。
}
\dai{001}{甄士隐抱英莲遇僧道}
\dai{002}{贾雨村吟诗,甄士隐赠银}
\sun{p1-1}{一僧一道高谈快论,空空道人检阅石头记}{当初女娲炼石补天,遗弃一顽石于青埂峰下。
此石自经锻炼,灵性已通。
图右侧:松下,一僧一道席地而谈。
只见那僧手托一块鲜明莹洁的石头,述说那顽石“无才补天” 的怨愧。
图左侧:不知又过了几世几劫,空空道人从青埂峰下经过,读了顽石上所述,其间事迹原委,倒也可以警世,遂将顽石所述抄写回来。
这便是《红楼梦》又名《石头记》故事的缘起。
}
\sun{p1-2}{甄士隐神游太虚幻境,遇一僧一道,看通灵宝玉}{姑苏城内葫芦庙旁住着一乡宦,名甄士稳。
一日中午,士隐伏几少憩, 朦胧中走至一处,忽见来了一僧一道,且行且谈,只听得“这石正该下世”云云,士隐上前笑问:“适闻仙师所谈因果……若蒙大开痴顽, 能否备细一闻?“那二仙笑道:“此乃玄机,不可预泄。
”说着取出一块鲜明美玉,镌着“通灵宝玉”四字,士隐正欲细看时,那僧便说“已到幻境” ,强从手中夺了去。
士隐方举步时,忽听一声霹雳,大叫一声醒来。
}
\sun{p1-3}{甄士隐抱英莲遇僧道}{甄士隐醒来,定睛看时,只见奶母抱了女儿英莲走来。
士隐将英莲抱在怀中,来到街前玩耍。
方欲进门,来了一僧一道。
只听那僧哭道:“施主,你把这有命无运,累及爹娘之物抱在怀里做甚!”士隐知是疯话,有些不耐烦, 欲抱女儿回去,又听那道人说:“三劫后,同往太虚幻境销号”,才要上前问时,那一僧一道忽地不见了。
远处, 寄居葫芦庙中的穷儒贾雨村,正要前来搭讪。
}
\sun{p1-4}{贾雨村见娇杏,甄士隐中秋宴请贾雨村}{图右侧:士隐遂邀雨村至书房叙谈,适逢家里来了客人,只好留雨村一人闲坐。
无聊中,雨村忽听窗外有女子嗽声,抬头一看,原是一个仪容不俗,眉目清秀的丫鬟,顿生爱慕之心。
图左侧:中秋佳节,士隐来邀雨村小酌,觥筹交错,士隐得知雨村才情虽好,无奈盘缠无措,正愁进京路远,当即命小童速封五十两白银奉上,资助他进京赶考。
}
\sun{p1-5}{元宵观灯,霍启小解,英莲被拐}{又逢元宵佳节,士隐令家人霍启抱女儿英莲去看社火花灯。
半夜时,霍启因要小解,便将英莲放在一家门槛上坐着,待小解完了来抱时,哪有英莲踪影。
急得霍启寻了半夜,到天明也未寻见,自忖无颜回见主人,便只身逃往他乡去了。
}
\sun{p1-6}{葫芦庙失火甄家遭殃,甄士隐随跛足道人出家,娇杏见贾雨村上任}{图右上:葫芦庙失火,将一条街烧得“火焰山”一般,甄家早已烧成了一片瓦砾场。
士隐走投无路,只好携了妻子与两个丫鬟投奔岳丈。
不想岳丈封肃见士隐穷了,人前人后说出许多难听怨言,传到士隐耳中,不免急忿怨痛。
图左上:一日,街前来了一个跛足道人,士隐只听他口中唱道“好了”“好了”,一闻此言,士隐早已彻悟,便随着疯道人飘然而去。
图左下:一日,甄家的大丫鬟在门前买线,俄见大轿内抬着一个乌帽猩袍的官员过来。
那丫鬟自思道:这官儿好面善,倒像在哪里见过。
}