\chapter{魇魔法叔嫂逢五鬼\quad 通灵玉蒙蔽遇双真}
\zhu{魇(魇音“演”)魔法:一种迷信活动,认为施行一种所谓“法术”可以驱使鬼神折磨人,致人于死。
五鬼:旧时星命家所称的恶煞之一,取象于鬼宿第五星。
真:真人,即仙人。
双真:这里指癞头和尚与跛足道人。
}
\par
\qi{有缘的,推不开;知心的,死不改。
纵然是通灵神玉也遭尘败。
梦里徘徊,醒后疑猜,时时兜底上心来。
怕人窥破笑盈腮,独自无言偷打咳。
\zhu{咳:音“孩”,小儿笑,泛指笑貌。
}这的是,前生造定今生债。
}\par
话说红玉情思缠绵,忽朦胧睡去,见贾芸要拉他,却回身一跑,被门槛子绊了一跤,唬醒过来,方知是梦。
因此翻来覆去,一夜无眠。
至次日天明,方才起来,就有几个丫头来会他去打扫屋子地,提洗脸水。
这红玉也不梳洗,向镜中胡乱挽了一挽头发,洗了洗手,腰内束了一条汗巾子,便来扫地。
\par
谁知宝玉昨儿见了红玉,也就留了心。
若要直点名唤他来使用,一则怕袭人等寒心;\jia{是宝玉心中想,不是袭人拈酸。
}二则又不知红玉是何等行为,若好还罢了,\jia{不知“好”字是如何讲?答曰:在“何等行为”四字上看便知,玉儿每情不情,况有情者乎?}若不好起来,那时倒不好退送的。
因此心中闷闷的,一早起来也不梳洗,只坐着出神。
一时下了窗子,隔着纱屉子,\zhu{纱屉子:旧时的窗户分两层,里面一层是用纱绷上的,透明、通气,称“纱屉子”。
外面一层是用纸糊或木板装的,白天可以卸下来或支起,晚间再安上或放下。
}向外看的真切,只见好几个丫头在那里扫地,都擦胭抹粉,簪花插柳的,\jia{八字写尽蠢鬟,是为衬红玉,亦如用豪贵人家浓妆艳饰插金戴银的衬宝钗、黛玉也。
}独不见昨儿那一个。
宝玉便靸了鞋,\zhu{靸:音“洒”,穿鞋时把鞋后帮踩在脚后跟下,拖着走。
}晃出了房门,只装着看花儿,这里瞧瞧,那里望望,\geng{文字有层次。
}一抬头,只见西南角上游廊底下栏杆外,似有一个人在那里倚着,却恨面前有一株海棠花遮着,看不真切。
\jia{余所谓此书之妙皆从诗词句中翻出者,皆系此等笔墨也。
试问观者,此非“隔花人远天涯近”乎?可知上几回非余妄拟也。
}只得又转了一步,仔细一看,可不是昨儿的那个丫头在那里出神?待要迎上去,又不好去的。
正想着,忽见碧痕来催他洗脸,只得进去了。
不在话下。
\par
却说红玉正自出神,忽见袭人招手叫他,\jia{此处方写出袭人来,是衬贴法。
\zhu{衬贴法:先写怡红院好几个丫头在那里扫地,红玉也是做这种扫地、提水等粗活丫头中的一个,
然后写“袭人招手叫他”,这即显出袭人在众丫头中地位的重要。
衬贴法是先写周围一般的人和事,然后再写中心的人,则显出这个人的重要,是通过一般人物衬托中心人物。
}
}只得走来。
袭人道:“你到林姑娘那里去,把他们的喷壶借来使使,我们的还没有收拾了来呢。
”红玉答应了,便往潇湘馆去。
正走上翠烟桥,抬头一望,只见山坡上高处都拦着帏幕,方想起今儿有匠人在里头种树。
因转身一望,只见那边远远的一簇人在那里掘土,贾芸正坐在山子石上。
红玉待要过去,又不敢过去,只得闷闷的向潇湘馆取了喷壶回来,无精打彩,自向房内倒着去。
众人只说他一时身上不快,都不理论。
\zhu{理论:理会。}
\jia{文字到此一顿,狡猾之甚。
}\par
展眼过了一日,\jia{必云“展眼过了一日”者,是反衬红玉“捱一刻似一夏”也,知乎?}原来次日就是王子腾夫人的寿诞,那里原打发人来请贾母王夫人的,王夫人见贾母不去,自己也便不去了。
\jia{所谓一笔两用也!}倒是薛姨妈同凤姐儿并贾家三\foot{原作“四”,庚、戚宁、蒙、列、舒本均同,当系早期原稿本之误。
据有正、甲辰本改为“三”,杨本则改为“几”。
}个姊妹、宝钗、宝玉一齐都去了,至晚方回。
\par
且说王夫人见贾环下了学,便命他来抄个《金刚咒》\jia{用《金刚咒》引五鬼法。
}唪诵。
\zhu{
唪[fěng]诵:大声念经。
《金刚咒》:《金刚经》后面附的咒语。
佛家说唪诵这几句咒语可以消灾祈福。
}那贾环在王夫人炕上坐了,命人点上灯,拿腔作势的抄写。
\jia{小人乍得意者齐来一玩。
}一时叫彩云倒茶来,一时又叫玉钏儿来剪剪灯花,
\zhu{灯花:灯烛的灯芯燃烧时爆出火花,或余烬所结成的花状物,习俗上认为是吉祥的征兆。}
一时又说金钏儿挡了灯影。
众丫头们素日厌恶他,都不答理。
只有彩霞还和他合的来,\jia{暗中又伏一风月之隙。
}倒了一钟茶来递与他。
见王夫人和人说话儿,便悄悄的向贾环说道:“你安些分罢,何苦讨这个厌呢。
”贾环道:“我也知道了,你别哄我。
如今你和宝玉好,把我不答理,我也看出来了。
”彩霞咬着嘴唇,向贾环头上戳了一指头,说道:“没良心的!才是狗咬吕洞宾,不识好人心。
”\jia{风月之情,皆系彼此业障所牵。
\zhu{
业障:也作「孽障」。
佛教上指由于过去的恶行所造成的障碍;
骂人的话,旧时长辈严厉指责子弟不肖的话。
}
虽云“惺惺惜惺惺”,但亦从业障而来。
蠢妇配才郎,世间固不少,然俏女慕村夫者尤多,所谓业障牵魔,
\zhu{
魔:佛教指修道的障害、破坏者;入迷,使入迷;指迷恋某物的人,亦指使之迷恋之物;引申为纠缠不清、难以对付的人。
}
不在才貌之论。
}\geng{此等世俗之言,亦因人而用,妥极当极!壬午孟夏,雨窗。
畸笏。
}\par
两人正说着,只见凤姐来了,拜见过王夫人。
王夫人便一长一短的问他,今儿是那几位堂客在那里,戏文如何,酒席好歹等话。
说了不多几句,宝玉也来了,进门见了王夫人,不过规规矩矩说了几句话,\jia{是大家子弟模样。
}便命人除去抹额,脱了袍服,拉了靴子,便一头滚在王夫人怀内。
\jia{余几几失声哭出。
}王夫人便用手满身满脸摩挲抚弄他, 
\jia{普天下幼年丧母者齐来一哭。
}宝玉也搬着王夫人的脖子说长道短的。
\jia{慈母娇儿写尽矣。
}王夫人道:“我的儿,你又吃多了酒,脸上滚热。
你还只是揉搓,一会闹上酒来。
还不在那里静静的倒一会子呢。
”说着,便叫人拿个枕头来。
宝玉听了便下来,在王夫人身后倒下,又叫彩霞来替他拍着。
宝玉便和彩霞说笑,只见彩霞淡淡的不大答理,两眼睛只向贾环处看。
宝玉便拉他的手笑道:“好姐姐,你也理我一理儿呢。
”彩霞夺了手道:“再闹,我就嚷了。
”\par
二人正说,原来贾环听的见,素日原恨宝玉,如今又见他和彩霞厮闹,心中越发按不下这口毒气。
虽不敢明言,却每每暗中算计,\jia{已伏金钏回矣。
\zhu{第三十三回。
}}只是不得下手。
今儿相离甚近,便要用蜡灯里的滚油烫他一下。
因而故意装作失手,把那一盏油汪汪的蜡灯向宝玉脸上只一推。
只听宝玉“嗳哟”了一声,满屋人都唬一跳。
连忙把地下的戳灯挪过来,
\zhu{戳灯:又名高灯。
是一种竖在地上的灯笼,有长柄,可插在底座上,也可扛着行走。}
又将里外屋的拿了三四盏看时,只见宝玉满脸满头都是蜡油。
王夫人又急又气,一面命人来给宝玉擦洗,一面又骂贾环。
凤姐三步两步跑上炕去,给宝玉收拾着,\jia{阿凤活现纸上。
}一面笑道:“老三还是这样慌脚鸡似的,我说你上不得高台盘。
赵姨娘时常也该教导教导他才是。
”\geng{为下文紧一步。
}一句话提醒了王夫人,王夫人便不骂贾环,便叫过赵姨娘来骂道:“养出这样不知道理、下流黑心种子来,也不管管!几番几次我都不理论,\jia{补出素日来。
}你们倒得了意了,这不益发上来了!”\par
那赵姨娘素日虽然也常怀嫉妒之心,不忿凤姐宝玉两个,也不敢露出来;如今贾环又生了事,受这场恶气,不但吞声承受,而且还要替宝玉来收拾。
只见宝玉左边脸上烫了一溜燎泡,幸而眼睛没动。
王夫人看了,又是心疼,又怕明日贾母问怎么回答,急的又把赵姨娘数落一顿。
\jia{总是为楔紧“五鬼”一回文字。
}然后又安慰了宝玉一回,又命取败毒消肿药来敷上。
宝玉道:“有些疼,还不妨事。
明儿老太太问,就说是我自己烫的罢了。
”凤姐笑\jia{两笑,坏极。
}\geng{为五鬼法作引,非泛文也。
雨窗。
}道:“便说自己烫的,\jia{玉兄自是悌弟之心性,一叹。
}也要骂人为什么不小心看着,叫你烫了!横竖有一场气生,到明儿凭你怎么说去罢。
”\jia{坏极!总是调唆口吻,赵氏宁不觉乎?}王夫人命人好生送了宝玉回房,袭人等见了,都慌的了不得。
\par
林黛玉见宝玉出了一天门,就觉得闷闷的,没个可说话的人。
至晚正打发人来问了两三遍回来没有,这遍方才说回来,偏生又烫了脸。
林黛玉便赶着来瞧,只见宝玉正拿镜子照呢,左边脸上满满的敷着一脸药。
黛玉只当烫的十分利害,忙上来问怎么烫了,要瞧瞧。
宝玉见他来了,忙把脸遮着,摇手不肯叫他看。
知道他的癖性喜洁,见不得这些东西。
\jia{写宝玉文字,此等方是正紧笔墨。
\zhu{紧:紧要,重要。
}}林黛玉自己也知道有这件癖性,\jia{写林黛玉文字,此等方是正经笔墨。
故二人文字虽多,如此等暗伏淡写处亦不少,观者实实看不出者。
}知道宝玉的心内怕他嫌脏,\jia{二人纯用体贴功夫。
}\jia{将二人一并,真真写他二人之心玲珑七窍。
}因笑道:“我瞧瞧烫了那里了,有什么遮着藏着的。
”一面说,一面就凑上来,强搬着脖子瞧了一瞧,问疼的怎么样。
宝玉道:“也不很疼,养一两日就好了。
”黛玉坐了一回,闷闷的回房去了。
一宿无话。
次日,宝玉见了贾母,虽然自己承认是自己烫的,不与别人相干,免不得贾母又把跟从的人骂一顿。
\jia{此原非正文,故草草写去。
}\par
过了一日,就有宝玉寄名的干娘马道婆进荣国府来请安。
\zhu{寄名的干娘:这里是指把子弟寄其名下为义子的道姑。
“寄名”是为了得到神的保佑,免除灾难。
}见了宝玉,唬了一跳,问起原故,说是烫的,便点头叹惜一回,又向宝玉脸上用指头画了几画,又口内嘟嘟囔囔的持诵了一回,就说道:“管保你好了,这不过是一时飞灾。
”又向贾母道:“祖宗老菩萨,那里知道那经典佛法上说的利害。
\jia{一段无伦无理信口开河的混话,却句句都是耳闻目睹者,并非杜撰而有。
作者与余实实经过。
}大凡那王公卿相人家的子弟,只一生下来,暗中就有许多促狭鬼跟着他,\zhu{促狭:刁钻机灵,爱捉弄人。
}得空便拧他一下,掐一下,或吃饭时打下他的饭碗来,或走着推他一跤,所以往往的那大家子的子孙多有长不大的。
”贾母听见如此说,便赶着问:“这可有什么佛法解释没有呢?”\zhu{佛法解释:这里是用佛法消灾去病之意。
解释:消散。
}
马道婆道:“这个容易,只是替他多多作些因果善事也就罢了。
再那经上还说,西方有位大光明普照菩萨,专管照耀阴暗邪祟,若有那善男子善女人虔心供奉者,可以永佑儿孙康宁安静,再无惊恐邪祟撞客之灾。
”\zhu{撞客:旧时迷信认为突然神智昏迷、胡言乱语,是鬼、神附体,俗称“撞客”。
}
贾母道:“倒不知怎么供奉这位菩萨呢?”马道婆道:“也不值什么,除香烛供养之外,一天多使几斤香油,添在大海灯里。
这海灯,就是菩萨的现身法像,\zhu{现身法像:佛教说,佛为化度众生而变幻出无数的“化身”来。
“现身法像”就是佛所变幻出的形象。
这里是说“海灯”就是佛的“化身”。
佛前海灯:即长明灯,供于寺庙佛像前,灯内大量贮油,中燃一焰,长年不灭。
}昼夜是不敢熄的。
”贾母道:“一天一夜也得多少油?明白告诉我,我好作这件功德。
”\zhu{功德:佛家称去恶行善为“功德”。
施舍财物、诵经,以祈福消灾叫“作功德”。
}\ping{中国版赎罪券。
}马道婆听说,便笑道:“这也不拘,随施主们心愿舍罢了。
像我们庙里,就有好几处的王妃诰命供奉:南安郡王太妃有许多,愿心大,\jia{贼婆先用大铺排试之。
}一天是四十八斤油,一斤灯草,那海灯也只比缸略小些;锦田侯的诰命次一等,一天不过二十四斤;再还有几家,也有五斤的、三斤的、一斤的,都不拘数。
那小家子舍不起这些,就是四两半斤,也少不得替他点。
”贾母听了,点头思忖。
\zhu{忖[cǔn]:揣度[duó];思量。}
\jia{“点头思忖”是量事之大小,非吝啬也。
日费香油四十八斤,每月油二百五十馀斤,合钱三百馀串。
为一小儿,如何服众?太君细心若是。
}马道婆又道:“还有一件,若是为父母尊亲长上点,多舍些不妨;像老祖宗如今为宝玉,若舍多了倒不好,\jia{贼道婆!是自“太君思忖”上来,后用如此数语收之,使太君必心悦诚服愿行。
贼婆,贼婆,费我作者许多心机摹写也。
}还怕他禁不起,倒折了福,也不当家。
\zhu{不当家:亦作“不当家花拉的”、“不当家花花的”。
“不当家”亦作“不当价”。
意即不应该、当不起、罪过。
“价”为助词,“花花的”是词尾,无义。
}要舍,大则七斤,小则五斤,也就是了。
”贾母说:“既这样,你便一日五斤合准了,每月来打趸关了去。
”\zhu{打趸关了去:凑总数领走。
趸:音“盹”,整数。
关:领取。
}\ping{给宝玉这个小孩子用的油量,不能超过尊长,否则就有逾秩僭越之嫌。
}马道婆念了一声“阿弥陀佛,慈悲大菩萨”。
贾母又命人来吩咐道:“以后大凡宝玉出门的日子,拿几串钱交给他小子们带着,遇见僧道穷苦之人好施舍的。
”说毕,那马道婆又坐了一回,便又往各院各房问安,闲逛了一回。
\par
一时来至赵姨娘房内,\jia{有“各院各房”,接此方不觉突然。
}二人见过,赵姨娘叫小丫头倒了茶来与他吃。
马道婆因见炕上堆着些零碎绸缎弯角,赵姨娘正粘鞋呢。
马道婆道:“可是我正没有鞋面子。
\jia{见者有分是也。
}赵奶奶你有零碎缎子,不拘什么颜色,弄一双给我。
”赵姨娘听说,叹口气道:“你瞧瞧,那里头还有那一块是成样的?成样的东西,也到不了我手里来!有的没的都在那里,你不嫌,就挑两块子去。
”那马道婆见说,果真挑了两块袖起来。
\ping{蝇头微利亦取。
}\par
赵姨娘问道:“可是前儿我送了五百钱去,在药王跟前上供,\zhu{药王:菩萨名。
传说以神农或扁鹊为药王;一说唐代孙思邈为药王。
迷信传说认为祈求药王可以愈病。
}你可收了没有?”马道婆道:“早已替你上了供了。
”赵姨娘叹口气道:“阿弥陀佛!我手里但凡从容些,也时常的上个供,只是心有馀力量不足。
”马道婆道:“你只放心,将来熬的环哥儿大了,得个一官半职,那时你要做多大的功德不能?”赵姨娘听了,鼻子里笑了一声,道:“罢,罢,再别说起。
如今就是个样儿,我们娘儿们跟的上那一个?也不是有了宝玉,竟是得了个活龙。
他还是小孩子家,长的得人意儿,大人偏疼他些也还罢了;\jia{赵妪数语,可知玉兄之身份,况在背后之言。
}我只不服这个主儿。
”\jia{活现赵妪。
}一面说,一面又伸出俩指头来。
\jia{活现阿凤。
}马道婆会意,便问道:“可是琏二奶奶么?”赵姨娘唬的忙摇手儿,走到门前,掀帘子向外看看无人,\jia{是心胆俱怕破。
}方进来向马道婆悄悄的说道:“了不得,了不得!提起这个主儿,这一分家私要不教他搬送了娘家去,我就不是个人。
”\geng{这是妒心正题目。
}\par
马道婆见他如此说,便探他口气说道\foot{原只作“马道婆道”,为兼顾庚本批语,据诸本补。
}:\geng{有隙即入,所谓贼婆,是极!}“我还用你说,难道都看不出来?也亏你们心里都不理论,只凭他去。
倒也妙。
”赵姨娘道:“我的娘,不凭他去,难道谁还敢把他怎么样?”马道婆听说,鼻子里一笑,\geng{二笑。
}半晌说道:“不是我说句造孽的话,你们没本事也难怪。
明不敢怎么样,暗里也就算计了,\jia{贼婆操必胜之券,赵妪已堕术中,故敢直出明言。
可畏可怕!}还等到这时候!”赵姨娘听这话有道理,心里暗暗的欢喜,便问道:“怎么暗里算计?我倒有这心,只是没这样的能干人。
你若教给我这法子,我大大的谢你。
”马道婆听说,这话打拢了一处,\zhu{打拢了一处:指两人想到一处去了。
}他便又故意说道:“阿弥陀佛!你快休问我,我那里知道这些事。
罪过,罪过。
”\jia{远一步却是近一步。
贼婆,贼婆!}赵姨娘道:“又来了。
你是最肯济困扶危的人,难道就眼睁睁的看着人家来摆布死了我们娘儿两个不成?还是怕我不谢你?”马道婆听如此说,便笑道:“若说我不忍叫你娘儿们受了委屈还犹可,若说‘谢’的这个字,可是你错打算盘了。
就便是我希图你的谢,靠你又有什么东西能打动了我?”\jia{探谢礼大小是如此说法,可怕可畏!}赵姨娘听这话口气松了些,便说道:“你这么个明白人,怎么也糊涂起来了。
你若果然法子灵验,把他两个绝了,明日这家私不怕不是我环儿的。
那时你要什么不得?”马道婆听说,低了头,半晌说道:“那时候事情妥当了,又无凭据,你还理我呢!”赵姨娘道:“这有何难。
如今我虽手里没什么,也零零碎碎攒了几两梯己,还有几件衣服、簪子,你先拿了去。
下剩的,我写个欠银子的文契给你,你要什么保人也有,到那时我照数给你。
”马道婆道:“果然这样?”赵姨娘道:“这如何撒得谎!”说着便叫过一个心腹婆子来,在耳根底下嘁嘁喳喳说了几句话。
\jia{所谓“狐群狗党,大家难免”,看官着眼。
}那婆子出去了,一时回来,果然写了个五百两的欠契来。
赵姨娘便印了手模,\jia{痴妇,痴妇!}走到橱柜里将梯己拿了出来,与马道婆看看,道:“这个你先拿了去,做香烛供奉使费,可好不好?”马道婆看看白花花的一堆银子,又有欠契,并不顾青红皂白,\jia{有道婆作干娘者来看此句。
“并不顾”三字怕杀人。
千万件恶事皆从三字生出来。
可怕可畏可警,可长存戒之。
}满口里应着,伸手先去接了银子掖起来,然后收了欠契。
又向裤腰里掏了半晌,掏出十几个纸铰的青脸红发的鬼来,并两个纸人,\jia{如此现成,更可怕。
}\geng{如此现成,想贼婆所害之人岂止宝玉、阿凤二人哉?大家太君夫人诫之慎之。
}递与赵姨娘,又悄悄的道:“把他两个的年庚八字写在这两个纸人身上,\zhu{年庚八字:年庚:人诞生的年月日时。
以天干地支记年月日时,如甲子年、乙丑月、丙寅日、丁卯时,共八个字,故称为“八字”。
迷信说法,“八字”有好坏,它注定人的一生命运。
用巫术诅咒人时,要写明被咒者的年庚八字。
}一并五个鬼都掖在他们各人的床上就完了。
我只在家里作法,自有效验。
千万小心,不要害怕!”\jia{宝玉乃贼婆之寄名干儿,一样下此毒手,况阿凤乎?三姑六婆之害如此,即贾母之神明,在所不免。
其他只知吃斋念佛之夫人太君,岂能防范得来?此系老太君一大病。
作者一片婆心,不避嫌疑,特为写出,使看官再四着眼,吾家儿孙慎之戒之!}正才说完,只见王夫人的丫鬟进来找道:“奶奶可在这里,太太等你呢。
”二人方散了,不在话下。
\par
却说黛玉因见宝玉近日烫了脸,总不出门,倒时常在一处说说话儿。
这日饭后看了二三篇书,自觉无味,便同紫鹃、雪雁做了一回针线,更觉得烦闷。
便倚着房门出了一回神,\jia{所谓“闲倚绣房吹柳絮”是也。
\zhu{李商隐《访人不遇留别馆》:
卿卿不惜锁窗春,去作长楸走马身。
闲倚绣帘吹柳絮,日高深院断无人。
}
}信步出来,看阶下新迸出的稚笋,\jia{妙妙!“笋根稚子无人见”,今得颦儿一见,何幸如之。
\zhu{
杜甫《漫兴》九首之七:
糁径杨花铺白毡,点溪荷叶叠青钱。
笋根雉子无人见,沙上凫雏傍母眠。
}
}不觉出了院门。
一望园中,四顾无人,\jia{恐冷落园亭花柳,故有是十数字也。
}惟见花光柳影,鸟语溪声。
\jia{纯用画家笔写。
}林黛玉信步便往怡红院来,只见几个丫头舀水,都在回廊上围着看画眉洗澡呢。
\jia{闺中女儿乐事。
}听见房内有笑声,林黛玉便入房中看时,原来是李宫裁、凤姐、宝钗都在这里呢,一见他进来,都笑道:“这不又来了一个。
”林黛玉笑道:“今日齐全,倒像谁下帖子请来的。
”凤姐道:“前儿我打发人送了两瓶茶叶去,\geng{有照应。
}你往那去了?”黛玉笑道:“哦,可是我倒忘了,\jia{该云“我正看《会真记》呢”。
一笑。
}多谢多谢。
”凤姐又道:“你尝了可还好不好?”没有说完,宝玉便道:“论理可倒罢了,只是我说不大甚好,可也不知别人尝着怎么样。
”宝钗道:“味倒轻,只是颜色不大很好。
”\geng{二宝答言是补出诸艳俱领过之文。
乙酉冬,雪窗。
畸笏老人。
}凤姐道:“那是暹罗进贡来的。
\zhu{暹(暹音“先”)罗:古国名,在今泰国一带。
}我尝着也没什么趣儿,还不如我每日吃的呢。
”黛玉道:“我吃着好。
\jia{卿爱因味轻也。
卿如何担的起味厚之物耶?}”宝玉道:“你果然吃着好,把我这个也拿了去罢。
”凤姐道:“你真爱吃,我那里还有呢。
”林黛玉道:“果真的,我就打发人取去了。
”凤姐道:“不用取去,我叫人送来就是了。
我明日还有一件事求你,一同打发人送来。
”黛玉听了笑道:“你们听听,这是吃了他一点子茶叶,就来使唤我来了。
”凤姐笑道:“倒求你,你倒说这些闲话。
你既吃了我们家的茶,怎么还不给我们家作媳妇?”\zhu{吃了我们家的茶:女子受聘,俗谓“吃茶”。
《七修类稿》:种茶下子,不可移植,移植则不复生,故以喻女子受聘。
}\jia{二玉事,在贾府上下诸人,即看书人、批书人皆信定一段好夫妻,书中常常每每道及,岂其不然,叹叹!}众人听了都一齐笑起来。
\geng{二玉之配偶,在贾府上下诸人,即观者、批者、作者皆为无疑,故常常有此等点题语。
我也要笑。
}\par
黛玉便红了脸,一声儿也不言语,回过头去了。
李宫裁笑向宝钗道:“真真我们二婶子的诙谐是好的。
”\geng{好赞!该他赞。
}林黛玉含羞笑道:“什么诙谐,不过是贫嘴贱舌讨人厌恶罢了。
”\jia{此句还要候查。
\ping{此条脂评的意思大概是,提醒作者这句话可以在修改润色的时候,想想是否可以改得更好一些。
}}说着便啐了一口。
凤姐笑道:“你别作梦!给我们家作了媳妇,你想想——”便指宝玉道:“你瞧,人物儿、门第配不上,\jia{大大一泄,好接后文。
}
还是根基配不上?模样儿配不上,是家私配不上?那一点玷辱了谁呢?”林黛玉便起身要走。
宝钗便叫道:“颦儿急了,还不回来坐着。
走了倒没意思。
”说着便站起来拉住。
\par
只见赵姨娘和周姨娘两个人进来瞧宝玉。
李宫裁、宝钗、宝玉等都让他两个坐。
独凤姐只和黛玉说笑,正眼也不看他们。
宝钗方欲说话时,只见王夫人房内的丫头来说:“舅太太来了,请姑娘奶奶们出去呢。
”李宫裁听了,忙叫着凤姐等要走。
周、赵两个也忙辞了宝玉出去。
宝玉道:“我也不能出去,你们好歹别叫舅母进来。
”又道:“林妹妹,你先站一站,我和你说一句话。
”凤姐听了,回头向黛玉笑道:“有人叫你说话呢。
”说着,便把林黛玉往里一推,和李纨一同去了。
\par
这里宝玉拉着林黛玉的袖子,只是嘻嘻的笑,\geng{此刻好看之至!}心里有话,只是口里说不出来。
\jia{是已受镇,“说不出来”。
勿得错会了意。
}
此时林黛玉只是禁不住把脸红涨起来了,挣着要走。
宝玉忽然“嗳哟”了一声,说:“好头疼!”\jia{自黛玉看书起分三段写来,真无容针之空。
如夏日乌云四起,疾闪长雷不绝,不知雨落何时,忽然霹雳一声,倾盆大注,何快如之,何乐如之,其令人宁不叫绝!}林黛玉道:“该,阿弥陀佛!”\geng{黛玉念佛,是吃茶之语在心故也。
然摹写神妙,一丝不漏如此。
己卯冬夜。
}只见宝玉大叫一声:“我要死!”将身一纵,离地跳有三四尺高,嘴里乱嚷乱叫,说起胡话来了。
林黛玉并丫头们都唬慌了,忙去报知贾母、王夫人等。
此时王子腾的夫人也在这里,都一齐来看时,宝玉越发拿刀弄杖,寻死觅活的。
贾母、王夫人见了,唬的抖衣乱颤,且“儿”一声“肉”一声恸哭起来。
\zhu{恸[tòng]哭:极哀痛地哭。}
于是惊动众人,连贾赦、邢夫人、贾珍、贾政、贾琏、贾蓉、贾芸、贾萍、薛姨妈、薛蟠并家中一干家人,上上下下里里外外众媳妇丫头等,都来园内看视。
登时乱麻一般。
\jia{写玉兄惊动若许人忙乱,正写太君一人之钟爱耳。
看官勿被作者瞒过。
}正都没个主见,只见凤姐儿手持一把明晃晃钢刀砍进园来,见鸡杀鸡,见狗杀狗,见人就要杀人。
\jia{此处焉用鸡犬?然辉煌富丽,非处家之常也,鸡犬闲闲,始为儿孙千年之业,故于此处必用“鸡犬”二字,方是一族腾腾大舍。
}众人越发慌了。
周瑞媳妇忙带着几个有力量的胆壮的婆娘上去抱住,夺下刀来,抬回房去。
平儿、丰儿等哭的泪天泪地。
贾政等心中也有些烦难,顾了这里,丢不下那里。
\par
别人慌张自不必讲,独有薛蟠更比诸人忙到十分去:\jia{写呆兄忙是愈觉忙中之愈忙,且避正文之絮烦。
好笔仗,写得出。
}\geng{写呆兄是躲烦碎文字法。
好想头,好笔力。
《石头记》最得力处在此。
}又恐薛姨妈被人挤倒,又恐薛宝钗被人瞧见,又恐香菱被人臊皮——知道贾珍等是在女人身上做工夫的,\jia{从阿呆兄意中,又写贾珍一笔,妙!}因此忙的不堪。
忽一眼瞥见了林黛玉风流婉转,已酥倒在那里。
\jia{忙到容针不能。
此似唐突颦儿,却是写情字万不能禁止者,又可知颦儿之丰神若仙子也。
}\jia{忙中写闲,真大手眼,大章法。
}\par
当下众人七言八语,有的说请端公送祟的,\zhu{端公送祟:请巫师焚烧纸钱等物“送走鬼祟”的迷信仪式。
端公:即巫师。
}有的说请巫婆跳神的,\zhu{巫婆跳神:巫婆:旧时以装神弄鬼替人治病或祈祷的女人。
巫婆烧香上供,请“神仙附体”,手舞足蹈,代神降旨,叫跳神。
}
有的又荐什么玉皇阁的张真人,种种喧腾不一。
也曾百般的医治祈祷,问卜求神,总无效验。
堪堪的日落。
\zhu{堪堪:即“看看”,估量时间之辞,义近转眼。
}王子腾的夫人告辞去后,次日王子腾自己亲来瞧问。
\jia{写外戚,亦避正文之繁。
}接着小史侯家、邢夫人兄弟辈并各亲眷都来瞧看,也有送符水的,
\zhu{
符水:溶有符箓灰烬的水。道士用来治病。
符箓:亦作“符录”,音“符录”,道士巫师所画的一种图形或线条,相传可以役鬼神,辟病邪。
}
也有荐僧道的,也都不见效。
他叔嫂二人越发糊涂,不省人事,睡在床上,浑身火炭一般,口内无般不说。
到夜时,那些婆娘、媳妇、丫头们都不敢上前。
因此把他二人都抬到王夫人的上房内,\jia{收拾得干净有着落。
}\geng{收拾得得体正大。
}夜间派了贾芸等带着小子们挨次轮班看守。
贾母、王夫人、邢夫人、薛姨妈等寸地不离,只围着干哭。
\par
此时贾赦、贾政又恐哭坏了贾母,日夜熬油费火,闹的人口不安,也都没有主意。
贾赦还是各处去寻僧觅道。
贾政见都不灵效,着实懊恼,\jia{四字写尽政老矣。
}因阻贾赦道:“儿女之数,皆由天命,非人力可强者。
他二人之病出于不意,百般医治不效,想天意该当如此,也只好由他们去罢。
”\jia{念书人自应如是语。
}贾赦也不理此话,仍是百般忙乱,那里见些效验。
看看三日光阴,那凤姐和宝玉躺在床上,一发连气都将没了。
合家人口无不惊慌,都说没了指望,忙着将他二人的后世衣履都治备下了。
贾母、王夫人、贾琏、平儿、袭人这几个人,更比诸人哭的忘餐废寝,觅死寻活。
赵姨娘、贾环等心中欢喜称愿。
\jia{补明赵妪进怡红为作法也。
}\par
到了第四日早晨,贾母等正围着他两个哭时,只见宝玉睁开眼说道:\jia{“语不惊人死不休”,此之谓也。
}“从今以后,我可不在你家了!快些收拾,打发我走罢。
”贾母听了这话,就如同摘去心肝一般。
赵姨娘在旁劝道:“老太太也不必过于悲痛了。
\geng{断不可少此句。
}哥儿已是不中用了,不如把哥儿的衣裳穿好,让他早些回去罢,也免些苦。
只管舍不得他,这口气不断,他在那世里也受罪不安生。
”\geng{大遂心人必有是语。
}
这些话还没说完,被贾母照脸啐了一口唾沫,骂道:“烂了舌根的混帐老婆,谁叫你来多嘴多舌的!你怎么知道他在那世里受罪不安生?怎么见得不中用了?你愿他死了,有什么好处?你别做梦!他死了,我只和你们要命。
素日都是你们调唆着逼他写字念书,\jia{奇语,所谓溺爱者不明,然天生必有是一段文字的。
}把胆子唬破了,见了他老子还不像个避猫鼠儿?都不是你们这起淫妇调唆的!\zhu{都不是:难道不是。
}这会子逼死了他,你们遂了心了。
我饶那一个!”一面骂,一面哭。
贾政在旁听见这些话,心中越发难过,便喝退赵姨娘,自己上来委婉解劝。
一时又有人来回说:“两口棺材都作齐备了,\jia{偏写一头不了又一头之文,真步步紧之文。
}请老爷出去看。
”贾母听了,如火上浇油一般,便骂道:“是谁做了棺材?”一叠连声只叫把做棺材的拉来打死。
\par
正闹的天翻地覆,没个开交,只闻得隐隐的木鱼声响,\jia{不费丝毫勉强,轻轻收住数百言文字,《石头记》得力处全在此处。
以幻作真,以真作幻,看书人亦要如是看法为幸。
}念了一句:“南无解冤孽菩萨。
”\zhu{南无:佛教用语。
读“那摸”,虔诚皈依、祈求度我之意。
这里是普救众生的意思。
}又听说道:“有那人口不安,家宅颠倾,或逢凶险,或中邪祟不利者,我们善能医治。
”贾母、王夫人等听见这些话,那里还耐得住,便命人去快请来。
贾政虽不自在,奈贾母之言如何违拗;又想如此深宅,何得听的如此真切, 
\jia{作者是幻笔,合屋俱是幻耳,焉能无闻?}心中亦是希罕,\jia{政老亦落幻中。
}
便命人请了进来。
众人举目看时,原来是一个癞头和尚与一个跛足道人。
\jia{僧因凤姐,道因宝玉,一丝不乱。
\zhu{
癞僧、跛道出场的目的都是解救或超脱书中的人物———即度人。下面分析这八次度人活动的特点。
第一回癞僧、跛道度甄英莲(女);第一回跛道度甄士隐(男);第三回癞僧度林黛玉(女);
第七回癞僧度薛宝钗(女);第八回癞僧度薛宝钗(女);第十二回跛道度贾瑞(男);
第二十五回癞僧、跛道度王熙凤(女)、贾宝玉(男);第六十六回跛道度柳湘莲(男)。
除第一次、第七次僧、道共同出场外,凡是癞僧出场所度者均为女性,跛道出场所度者均为男性。
所以评语说“僧因凤姐,道因宝玉”。
}
}只见那和尚是怎生模样:\par
\hop
鼻如悬胆两眉长,目似明星蓄宝光,\par
破衲芒鞋无住迹,腌臜更有满头疮。
\zhu{衲:音“那”,本意是缝补,因为和尚的衣服多用碎布拼成,故引申为僧衣。
腌臜[āza]:口语。意为肮脏、不干净。
}\par
\hop
看那道人又是怎生模样,但见:\par
\hop
一足高来一足低,浑身带水又拖泥。
\par
相逢若问家何处,却在蓬莱弱水西。
\zhu{蓬莱:传说渤海中的仙山。
弱水,汉代东方朔《十洲记》:“凤麟洲在西海之中央,四面有弱水绕之,鸿毛不浮,不可越也。
”}\par
\hop
贾政问道:“你道友二人在那庙焚修?”
\zhu{道友:一起修道的朋友。焚修:焚香修道。}
那僧笑道:“长官不须多言。
\jia{避俗套法。
}因闻得尊府人口不利,故特来医治。
”贾政道:“倒有两个人中邪,不知二位有何符水?”那道笑道:“你家现放着希世奇珍,如何倒还问我们有符水?”贾政听这话有意思,心中便动了,因说道:“小儿落草时虽带了一块宝玉下来,上面说能除邪祟,\geng{点题。
}谁知竟不灵验。
”那僧笑道:“长官,你那里知道那物的妙用。
只因他如今被声色货利所迷,\jia{石且能迷,可知其害不小。
观者着眼,方可读《石头记》。
}
故此不灵验了。
\jia{读书者观之。
}你今且取他出来,待我们持诵持诵,只怕就好了。
”\geng{“只怕”二字,是不知此石肯听持诵否?}\par
贾政听说,便向宝玉项上取下那玉来递与他二人。
那和尚接了过来,擎在掌上,长叹一声道:“青埂峰一别,展眼已过十三载矣!\geng{正点题,大荒山手捧时语。
}\ping{宝玉尚小。
}人世光阴,如此迅速,尘缘满日,若似弹指!\jia{见此一句,令人可叹可惊,不忍往后再看矣!}可羡你当时的那段好处:\par
\hop
天不拘兮地不羁,心头无喜亦无悲;\jia{所谓越不聪明越快活。
}\par
却因煅炼通灵后,便向人间觅是非。
\par
\hop
可叹你今朝这番经历:\par
\hop
粉渍脂痕污宝光,绮栊昼夜困鸳鸯。
\zhu{渍:音“字”,沾染。
脂:胭脂、香粉之类。
栊:音“龙”,房屋的窗户,在此代指房屋。
绮栊:华丽的房屋。
鸳鸯:借指男女。
上句谓通灵玉被脂粉玷污失去精彩的光泽。
下句是说宝玉在富贵的环境里,整天和姊妹丫鬟们在一起厮混。
}\par
沉酣一梦终须醒,\jia{无百年的筵席。
}冤孽偿清好散场!”\zhu{这两句意谓人生如梦境般虚幻,终有醒悟的时候;生活如还债般苦恼,还清了孽债,大家便散伙收场。
沉酣:浓睡貌。
}\jia{三次煅炼,焉得不成佛作祖?
\zhu{三:可能是指很多或多数。}
}\par
\hop
念毕,又摩弄一回,说了些疯话,递与贾政道:“此物已灵,不可亵渎,悬于卧室上槛。
\zhu{槛:本意是指门下的横木,即门槛,这里应该是指门上的横木。
}将他二人安在一室之内,除亲身妻母外,不可使外人冲犯。
\geng{是要紧语,是不可不写之套语。
}三十三天之后,包管身安病退,复旧如初。
”说着,回头便走了。
\geng{通灵玉除邪,全部百回只此一见,何得再言?僧道踪迹虚实,幻笔幻想,写幻人于幻文也。
壬午孟夏,雨窗。
}贾政赶着,还说让他二人坐了吃茶,要送谢礼,他二人早已出去了。
贾母等还只管使人去赶,那里有个踪影?少不得依言将他二人就安在王夫人卧室之内,将玉悬在门上。
王夫人亲自守着,不许别个人进来。
\par
至晚间,他二人竟渐渐的醒来,\jia{能领持诵,故如此灵效。
}说腹中饥饿。
贾母、王夫人等如得了珍宝一般,\jia{昊天罔极之恩如何报得?哭杀幼而丧亲者。
}旋熬了米汤来与他二人吃了,\zhu{旋:随即。
}精神渐长,邪祟少退,一家子才把心放下来。
\jia{通灵玉听癞和尚二偈即刻灵应,抵却前回若干《庄子》及语录机锋偈子。
正所谓物各有所主也。
}
\jia{
叹不得见玉兄“悬崖撒手”文字为恨。
[丁亥夏,畸笏叟。]
}李宫裁并贾府三艳、薛宝钗、林黛玉、平儿、袭人等在外间听信。
闻得吃了米汤,省了人事,别人未开口,林黛玉先就念了声“阿弥陀佛”。
\jia{针对得病时那一声。
}宝钗便回头看了他半日,“嗤”的一笑。
众人都不会意,惜春问道:“宝姐姐,好好的笑什么?”宝钗笑道:“我笑如来佛比人还忙:\zhu{如来佛:佛教对佛有十种称号,每种称号表示某种德行。
“如来”是佛的十种称号之一,通常就佛的“法身”而言,解释为无往而不在。
}\geng{这一句作正意看,馀皆雅谑,但此一谑抵颦儿半部之谑。
\ping{钗粉黛黑。
}}又要讲经说法,又要普渡众生;这如今宝玉与二姐姐病了,又是烧香还愿、\zhu{还愿:求神保佑的人实践对神许下的报酬,如祭祀、慈善、捐献。
}赐福消灾;今儿才好些,又要管林姑娘的姻缘了。
你说忙的可笑不可笑。
”黛玉不觉红了脸,啐了一口道:“你们这起人不是好人,不知怎么死!再不跟着好人学,只跟那些贫嘴恶舌的人学。
”一面说,一面摔帘子出去了。
\par
\jia{总批:先写红玉数行引接正文,是不作开门见山文字。
\hang
灯油引大光明普照菩萨,大光明普照菩萨引五鬼魇魔法是一线贯成。
\hang
通灵玉除邪,全部只此一见,却又不灵,遇癞和尚、跛道人一点方灵应矣。
写利欲之害如此。
\hang
此回本意是为禁三姑六婆进门之害,难以防范。
}\par
\geng{此回书因才干乖觉太露,\zhu{乖觉:机警,聪敏。
}引出事来,作者婆心为世之乖觉人为鉴。
}\par
\qi{总评:欲深魔重复何疑,苦海冤河解者谁?结不休时冤日盛,井天甚小性难移。
\zhu{
诗对“魇魔法”故事发感叹。冤家宜解不宜结,但大多数人偏偏不懂这个道理,所以“结不休时冤日盛”,引发了家族内部的 “自杀自灭”。
“井天”即坐井观天,比喻见识短浅。“井天甚小性难疑”则感叹人性的缺陷是造成“冤日甚”的原因。
}
}
\dai{049}{贾环故意烫伤宝玉}
\dai{050}{马道婆赵姨娘密谋}
\sun{p25-1}{戏彩霞贾环烫宝玉,马道婆索油保平安}{图右侧:一日,王夫人命贾环抄写《金刚咒》,不多时,宝玉回来了, 滚在王夫人怀里撒娇,躺下后又拉着贾环喜欢的丫鬟彩霞的手亲热。
素日原恨宝玉的贾环,心上越发按不下这口气,故作失手,将油灯向宝玉脸上一推,只听宝玉“哎哟”一声,但见满脸是油,起了一溜燎泡。
王夫人遂将赵姨娘叫过来,骂了一顿。
图左侧:过了一日,马道婆到府里来,见了宝玉,问其缘由,一面叹息,一面向宝玉脸上指画了指画,口内嘟嘟嚷嚷一番,说道:“包管好了。
这不过是一时飞灾。
”贾母听从马道婆建议,给庙里舍香油供奉菩萨,保佑宝玉平安。
}
\sun{p25-2}{魇魔法叔嫂逢五鬼,通灵玉蒙蔽遇双真}{马道婆收了赵姨娘的银子,暗地里作起法来。
图右侧:一日,宝玉拉着林黛玉的袖子,只是嘻嘻的笑,此时林黛玉只是禁不住把脸红涨起来了,挣着要走。
宝玉忽然大叫一声,一跳三四尺高,口内乱嚷;图中部:又见凤姐手持刀砍进园来,众人慌了。
次日,叔嫂二人不醒人事,眼看二人气息渐无,图左侧:忽听空中隐隐有木鱼声,原来是一僧一道。
请进后,那道人笑道:“你家现有稀世之宝,待我拿来持诵持诵,就好了。
”便从贾政手中接过宝玉项上那块玉,摩弄了一回,说了些疯话,递与贾政道:“此物已灵,三十三日之后,包管好了。
”}