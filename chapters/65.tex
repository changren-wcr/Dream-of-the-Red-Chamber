\chapter{贾二舍偷娶尤二姨 \quad 尤三姐思嫁柳二郎}
\zhu{二舍:舍:即舍人,原是官名。
宋元以来俗称贵族官僚子弟为舍人。
二舍:犹言二公子,二少爷。
}
\par
\qi{笔笔叙二姐温柔和顺,高凤姐十倍,言语行事,胜凤姐五分,堪为贾琏二房,所以深着凤姐不念宗祠血食,\zhu{着:觉得,如“着急”。
}为贾宅第一罪人。
《纲目》书法!\zhu{《纲目》书法:《纲目》即指《资治通鉴纲目》,亦省称“《通鉴纲目》”。
宋·朱熹据司马光《资治通鉴》而作,门人赵师渊帮助编纂而成。
其凡例,大书以题要者称纲,分注以备言者称目。
朱熹是理学家,《资治通鉴纲目》中含有伦理纲常之教。
清·吕熊引韩子云:“朱子作《纲目》,操褒贬之大权,所以立纲常也。
兹稗官者流,亦可谓得其微旨。
”这条脂批认为此回是通过褒二姐而贬凤姐,故云“《纲目》书法”。
}\hang
文有双管齐下法,此文是也。
事在宁府,却把凤姐之奸酸刻薄、平儿之任侠直鲠、\zhu{直鲠:同“鲠直”、“耿直”、“梗直”,正直的意思。
}李纨之号菩萨、探春之号玫瑰、林姑娘之怕倒、薛姑娘之怕化,一时齐现,是何等妙文!}\par
话说贾琏、贾珍、贾蓉等三人商议,事事妥贴,至初二日,先将尤老和三姐送入新房。
尤老一看,虽不似贾蓉口内之言,也十分齐备,母女二人已称了心。
鲍二夫妇见了如一盆火,赶着尤老一口一声唤老娘,又或是老太太;赶着三姐唤三姨,或是姨娘。
至次日五更天,一乘素轿,将二姐抬来。
各色香烛纸马,\zhu{纸马:旧俗用于祭祀时供焚化的纸糊的人、车、马等造型,也指供焚化的印有神像的纸片。
}并铺盖以及酒饭,早已备得十分妥当。
一时,贾琏素服坐了小轿而来,拜过天地,焚了纸马。
那尤老见二姐身上头上焕然一新,不似在家模样,十分得意。
搀入洞房。
是夜贾琏同他颠鸾倒凤,百般恩爱,不消细说。
\par
那贾琏越看越爱,越瞧越喜,不知怎生奉承这二姐,乃命鲍二等人不许提三说二的,直以奶奶称之,自己也称奶奶,竟将凤姐一笔勾倒。
有时回家中,只说在东府有事羁绊,凤姐辈因知他和贾珍相得,自然是或有事商议,也不疑心。
再家下人虽多,都不管这些事。
便有那游手好闲专打听小事的人,也都去奉承贾琏,乘机讨些便宜,谁肯去露风。
于是贾琏深感贾珍不尽。
贾琏一月出五两银子做天天的供给。
若不来时,他母女三人一处吃饭;若贾琏来了,他夫妻二人一处吃,他母女便回房自吃。
贾琏又将自己积年所有的梯己,\zhu{梯己:意即私人的、贴心的。
私蓄亦可称作“梯己”。
}一并搬了与二姐收着,又将凤姐素日之为人行事,枕边衾内尽情告诉了他,只等一死,便接他进去。
二姐听了,自是愿意。
当下十来个人,倒也过起日子来,十分丰足。
\par
眼见已是两个月光景。
这日贾珍在铁槛寺作完佛事,晚间回家时,因与他姨妹久别,竟要去探望探望。
先命小厮去打听贾琏在与不在,小厮回来说不在。
贾珍欢喜,将左右一概先遣回去,只留两个心腹小童牵马。
一时,到了新房,已是掌灯时分,悄悄入去。
两个小厮将马拴在圈内,自往下房去听候。
\par
贾珍进来,屋内才点灯,先看过了尤氏母女,然后二姐出见,贾珍仍唤二姨。
\ping{贾珍叫二姨而不叫弟妹,说明贾珍内心认为贾琏虽然偷娶了尤二姐,但是并不能独占尤二姐,自己作为尤二姐的姐夫还是可以和贾琏共享尤二姐。
}大家吃茶,说了一回闲话。
贾珍因笑说:“我作的这保山如何?
\zhu{保山:指媒人。}
若错过了,打着灯笼还没处寻,过日你姐姐还备了礼来瞧你们呢。
”说话之间,尤二姐已命人预备下酒馔,关起门来,都是一家人,原无避讳。
那鲍二来请安,贾珍便说:“你还是个有良心的小子,所以叫你来伏侍。
日后自有大用你之处,不可在外头吃酒生事。
我自然赏你。
倘或这里短了什么,你琏二爷事多,那里人杂,你只管去回我。
我们弟兄不比别人。
”
\ping{贾珍有一点在摆出主人的感觉,把别人的佣人叫来骂一顿,好像这是他包养的女人。}
鲍二答应道:“是,小的知道。
若小的不尽心,除非不要这脑袋了。
”贾珍点头说:“要你知道。
”当下四人一处吃酒。
尤二姐知局,\zhu{知局:知趣;识相。
}便邀他母亲说:“我怪怕的,妈同我到那边走走来。
”尤老也会意,便真个同他出来,只剩小丫头们。
贾珍便和三姐挨肩擦脸,百般轻薄起来。
\ping{
程本改为“况且尤老娘在旁,贾珍也不好意思太露轻薄”,将尤老娘的形象,从纵容女儿淫乱的帮凶,改为了淫乱的阻碍。
}
小丫头子们看不过,也都躲了出去,凭他两个自在取乐,不知作些什么勾当。
\par
跟的两个小厮都在厨下和鲍二饮酒,鲍二女人上灶。
忽见两个丫头也走了来嘲笑,要吃酒。
鲍二因说:“姐儿们不在上头伏侍,也偷来了。
一时叫起来没人,又是事。
”他女人骂道:“糊涂浑呛了的忘八!\zhu{浑呛了的:骂人的话,指被愚浊意识迷住了心窍的。
忘八:即“王八”,乌龟或鳖的俗称,骂人的话,指妻子有外遇的男人。
}你撞丧那黄汤罢。
\zhu{撞丧:音“创桑”,犹言“狂吃滥饮”。
黄汤:指酒(骂人喝酒时说)。
}撞丧醉了,夹着你那膫子挺你的尸去。
\zhu{膫子:音“辽子”,俗秽语,指男子生殖器。
}叫不叫,与你屄相干!一应有我承当,风雨横竖洒不着你头上来。
”这鲍二原因妻子发迹的,近日越发亏他。
\zhu{亏:损耗,损害。
引申为辜负、对不起。
}自己除赚钱吃酒之外,一概不管,贾琏等也不肯责备他,故他视妻如母,百依百随,且吃够了便去睡觉。
这里鲍二家的陪着这些丫鬟小厮吃酒,讨他们的好,准备在贾珍前上好。
\par
四人正吃的高兴,忽听扣门之声,鲍二家的忙出来开门,看见是贾琏下马,问有事无事。
鲍二女人便悄悄告他说:“大爷在这里西院里呢。
”贾琏听了,便回至卧房。
只见尤二姐和他母亲都在房中,见他来了,二人面上便有些讪讪的。
贾琏反推不知,只命:“快拿酒来,咱们吃两杯好睡觉。
我今日很乏了。
”尤二姐忙上来陪笑接衣奉茶,问长问短。
贾琏喜的心痒难受。
一时鲍二家的端上酒来,二人对饮。
他丈母不吃,自回房中睡去了。
两个小丫头分了一个过来伏侍。
\par
贾琏的心腹小童隆儿拴马去,见已有了一匹马,细瞧一瞧,知是贾珍的,心下会意,也来厨下。
只见喜儿寿儿两个正在那里坐着吃酒,见他来了,也都会意,故笑道:“你这会子来的巧。
我们因赶不上爷的马,恐怕犯夜,\zhu{犯夜:干犯夜行禁例。
古法城中宵禁,不准夜行。
}往这里来借宿一宵的。
”隆儿便笑道:“有的是炕,只管睡。
我是二爷使我送月银的,交给了奶奶,我也不回去了。
”喜儿便说:“我们吃多了,你来吃一钟。
”隆儿才坐下,端起杯来,忽听马棚内闹将起来。
原来二马同槽,不能相容,互相蹶踢起来。
\ping{影射贾珍和贾琏两个人同在尤氏姐妹这里厮混,两人一起争夺女人。
}隆儿等慌的忙放下酒杯,出来喝马,好容易喝住,另拴好了,方进来。
鲍二家的笑说:“你三人就在这里罢,茶也现成了,我可去了。
”说着,带门出去。
这里喜儿喝了几杯,已是楞子眼了。
\zhu{楞子眼:北平方言。
喝醉酒,眼睛发直、看不清楚的样子。
}隆儿寿儿关了门,回头见喜儿直挺挺的仰卧炕上,二人便推他说:“好兄弟,起来好生睡,只顾你一个人,我们就苦了。
”那喜儿便说道:“咱们今儿可要公公道道的贴一炉子烧饼,\zhu{贴一炉子烧饼:用在热的炉子内膛贴面饼经过烤制成为烧饼,比喻在热的炕上人挨人躺着,这里暗指男男同性性行为。
}要有一个充正经的人,我痛把你妈一肏。
”隆儿寿儿见他醉了,也不必多说,只得吹了灯,将就睡下。
\par
尤二姐听见马闹,心下便不自安,只管用言语混乱贾琏。
那贾琏吃了几杯,春兴发作,便命收了酒果,掩门宽衣。
尤二姐只穿着大红小袄,散挽乌云,满脸春色,比白日更增了颜色。
贾琏搂他笑道:“人人都说我们那夜叉婆齐整,如今我看来,给你拾鞋也不要。
”尤二姐道:“我虽标致,却无品行。
看来到底是不标致的好。
”贾琏忙问道:“这话如何说?我却不解。
”尤二姐滴泪说道:“你们拿我作愚人待,什么事我不知。
我如今和你作了两个月夫妻,日子虽浅,我也知你不是愚人。
我生是你的人,死是你的鬼,如今既作了夫妻,我终身靠你,岂敢瞒藏一字。
我算是有靠,将来我妹子却如何结果?据我看来,这个形景恐非长策,要作长久之计方可。
”贾琏听了,笑道:“你且放心,我不是拈酸吃醋之辈。
前事我已尽知,\zhu{由第六十四回“二姐又是水性的人,在先已合姐夫不妥”可知,这里的前事指的是尤二姐和姐夫贾珍之间的事情。
}你也不必惊慌。
你因妹夫倒是作兄的,\zhu{从后文可知,此时尤二姐的妹妹尤三姐正在和贾珍调笑,这里的妹夫指的是贾珍,是贾琏的堂哥。
}自然不好意思,不如我去破了这例。
”说着走了,便至西院中来,只见窗内灯烛辉煌,二人正吃酒取乐。
\par
贾琏便推门进去,笑说:“大爷在这里,兄弟来请安。
”贾珍羞的无话,只得起身让坐。
贾琏忙笑道:“何必又作如此景象,咱们弟兄从前是如何样来!大哥为我操心,我今日粉身碎骨,感激不尽。
大哥若多心,我意何安。
从此以后,还求大哥如昔方好;不然,兄弟能可绝后,\zhu{能可:宁可。
}
再不敢到此处来了。
”说着,便要跪下。
慌的贾珍连忙搀起,只说:“兄弟怎么说,我无不领命。
”贾琏忙命人:“看酒来,我和大哥吃两杯。
”又拉尤三姐说:“你过来,陪小叔子一杯。
”\ping{这里贾琏自称尤三姐的小叔子,实际上默认了尤三姐是贾珍的女人,这可能导致了尤三姐在下面的激烈爆发,因为结合后文来看,尤三姐想嫁给柳湘莲。
}贾珍笑着说:“老二,到底是你,哥哥必要吃干这钟。
”说着,一扬脖。
\par
尤三姐站在炕上,指贾琏笑道:“你不用和我花马吊嘴的。
\zhu{花马吊嘴:花言巧语;耍贫嘴,哄骗人。
}‘清水下杂面,你吃我看见’;\zhu{清水下杂面,你吃我看见:歇后语,意思是说你安的什么心我看得清清楚楚,杂面是一种以绿豆为主制成的面条,下在清水里煮时,面是面,水是水,分得很清楚,故歇后语的后句说“你吃我看见”。
另一种说法,杂面是绿豆渣子一类豆面做成的粗粮,很涩,没有油水难以下咽,是旧时北方穷苦人的食粮,因有“清水下杂面——我看你怎么吃”的歇后语,意谓这样难咽的东西若能吃得下去,我就服了你。
}‘提着影戏人子上场,\zhu{影戏人子:也叫“影戏人儿”,影戏中用皮或纸剪的人物。
}好歹别戳破这层纸儿’。
你别油蒙了心,打量我们不知道你府上的事。
这会子花了几个臭钱,你们哥儿俩拿着我们姐儿两个权当粉头来取乐儿,\zhu{粉头:娼妓。
有时用来辱骂青年女子。
}你们就打错了算盘了。
我也知道你那老婆太难缠,如今把我姐姐拐了来做二房,‘偷的锣儿敲不得’。
我也要会会那凤奶奶去,看他是几个脑袋几只手。
若大家好取和便罢;倘若有一点叫人过不去,我有本事先把你两个的牛黄狗宝掏了出来,\zhu{牛黄狗宝:两种中药,均为结石,前者生在病牛的胆内,后者长于癞狗的腹中。
这里用来骂人,喻黑心肠,坏心思。
}再和那泼妇拼了这命,也不算是尤三姑奶奶!喝酒怕什么,咱们就喝!”说着,自己绰起壶来斟了一杯,
\zhu{绰:同“抄”,抓;拿。}
自己先喝了半杯,搂过贾琏的脖子来就灌,说:“我和你哥哥已经吃过了,咱们来亲香亲香。
”唬的贾琏酒都醒了。
贾珍也不承望尤三姐这等无耻老辣。
弟兄两个本是风月场中耍惯的,不想今日反被这闺女一席话说住。
尤三姐一叠声又叫:“将姐姐请来,要乐咱们四个一处同乐。
俗语说‘便宜不过当家’,\zhu{便宜不过当家:意谓便利和好处不能让外人得去,好处留给自家人。
当家:同姓或本家。
}他们是弟兄,咱们是姊妹,又不是外人,只管上来。
”尤二姐反不好意思起来。
贾珍得便就要一溜,尤三姐那里肯放。
贾珍此时方后悔,不承望他是这种为人,与贾琏反不好轻薄起来。
\par
这尤三姐松松挽着头发,大红袄子半掩半开,露着葱绿抹胸,\zhu{抹胸:挂束在胸腹间的贴身小衣,只盖住胸、肚,无袖,无后背。
徐珂《清稗类钞·服饰类》:“抹胸,胸间小衣也。
一名抹腹,又名抹肚。
以尺方之布为之,紧束前胸,以防风之内侵者。
俗谓之兜肚,男女皆有之。
”}一痕雪脯。
底下绿裤红鞋,一对金莲或翘或并,没半刻斯文。
两个坠子却似打秋千一般,灯光之下,越显得柳眉笼翠雾,\zhu{柳眉:像柳叶一样的眉毛,旧时用来形容女子眉毛好看。
翠:青绿色,这里可能是青黑色的意思。
古代女子用青黛(青黑色颜料)画眉。
“青”在古汉语里有黑色的意思,如“朝如青丝暮成雪”。
}檀口点丹砂。
\zhu{檀:浅红色的。
檀口:形容妇女红艳的嘴唇。
丹砂:即朱砂,主要成分为硫化汞,红色或棕红色。
}本是一双秋水眼,再吃了酒,又添了饧涩淫浪,\zhu{
饧:音“行”,眼睛半睁半闭或呆滞无神。
饧涩:形容眼睛半睁半闭,眼光黏滞。
}不独将他二姊压倒,据珍琏评去,所见过的上下贵贱若干女子,皆未有此绰约风流者。
二人已酥麻如醉,不禁去招他一招,他那淫态风情,反将二人禁住。
那尤三姐放出手眼来略试了一试,他弟兄两个竟全然无一点别识别见,连口中一句响亮话都没了,不过是酒色二字而已。
自己高谈阔论,任意挥霍洒落一阵,拿他弟兄二人嘲笑取乐,竟真是他嫖了男人,并非男人淫了他。
一时他的酒足兴尽,也不容他弟兄多坐,撵了出去,自己关门睡去了。
\par
自此后,或略有丫鬟婆娘不到之处,便将贾琏、贾珍、贾蓉三个泼声厉言痛骂,说他爷儿三个诓骗了他寡妇孤女。
贾珍回去之后,以后亦不敢轻易再来。
有时尤三姐自己高了兴悄命小厮来请,方敢去一会,到了这里,也只好随他的便。
谁知这尤三姐天生脾气不堪,仗着自己风流标致,偏要打扮的出色,另式作出许多万人不及的淫情浪态来,\zhu{另式:与众不同的款式。
}哄的男子们垂涎落魄,欲近不能,欲远不舍,迷离颠倒,他以为乐。
他母姊二人也十分相劝,他反说:“姐姐糊涂。
咱们金玉一般的人,白叫这两个现世宝沾污了去,
\zhu{现世宝:指不成器、没有用的人;出丑的人、丢脸的人。}
也算无能。
而且他家有一个极利害的女人,如今瞒着他不知,咱们方安。
倘或一日他知道了,岂有干休之理,势必有一场大闹,不知谁生谁死。
趁如今我不拿他们取乐作践准折,\zhu{准折:抵销,弥补,抵偿。
}到那时白落个臭名,后悔不及。
”因此一说,他母女见不听劝,也只得罢了。
那尤三姐天天挑拣穿吃,打了银的,又要金的;有了珠子,又要宝石;吃的肥鹅,又宰肥鸭。
或不趁心,连桌一推;衣裳不如意,不论绫缎新整,\zhu{整:整齐。
}便用剪刀剪碎,撕一条,骂一句。
究竟贾珍等何曾随意了一日,反花了许多昧心钱。
\par
贾琏来了,只在二姐房内,心中也悔上来。
无奈二姐倒是个多情人,以为贾琏是终身之主了,凡事倒还知疼着痒。
若论起温柔和顺,凡事必商必议,不敢恃才自专,实较凤姐高十倍;若论标致,言谈行事,也胜五分。
虽然如今改过,但已经失了脚,有了一个“淫”字,凭他有甚好处也不算了。
偏这贾琏又说:“谁人无错,知过必改就好。
”故不提已往之淫,\zhu{由第六十四回“二姐又是水性的人,在先已合姐夫不妥”可知,这里的“淫”指的是尤二姐和姐夫贾珍之间的事情。
}只取现今之善,便如胶授漆,似水如鱼,一心一计,誓同生死,那里还有凤平二人在意了?二姐在枕边衾内,也常劝贾琏说:“你和珍大哥商议商议,拣个相熟的人,把三丫头聘了罢。
留着他不是常法子,终久要生出事来,怎么处?”贾琏道:“前日我曾回过大哥的,他只是舍不得。
我说‘是块肥羊肉,只是烫的慌;玫瑰花儿可爱,刺大扎手。
咱们未必降的住,正经拣个人聘了罢。
’他只意意思思,就丢开手了。
你叫我有何法。
”二姐道:“你放心。
咱们明日先劝三丫头,他肯了,让他自己闹去。
闹的无法,少不得聘他。
”贾琏听了说:“这话极是。
”\par
至次日,二姐另备了酒,贾琏也不出门,至午间特请他小妹过来,与他母亲上坐。
尤三姐便知其意,\ji{全用醍醐灌顶,全是大翻身大解悟法。
\zhu{这里的翻身解悟指的是尤三姐痛改前非,洗心革面。
}}
酒过三巡,不用姐姐开口,先便滴泪泣道:\ji{全用如是等语,一洗孽障。
}
“姐姐今日请我,自有一番大礼要说。
但妹子不是那愚人,也不用絮絮叨叨提那从前丑事,我已尽知,说也无益。
既如今姐姐也得了好处安身,妈也有了安身之处,我也要自寻归结去,方是正理。
但终身大事,一生至一死,非同儿戏。
我如今改过守分,只要我拣一个素日可心如意的人方跟他去。
若凭你们拣择,虽是富比石崇,
\zhu{石崇:以豪富奢靡著称,见《晋书·石崇传》。}
才过子建,
\zhu{子建:曹植字子建,三国时诗人,有“才高八斗”之誉,见《南史·谢灵运传》。}
貌比潘安的,\zhu{
潘安即潘岳,晋人,古代著名美男子,见《世说新语·容止》。
}我心里进不去,也白过了一世。
”贾琏笑道:“这也容易。
凭你说是谁就是谁,一应彩礼都有我们置办,母亲也不用操心。
”尤三姐泣道:“姐姐知道,不用我说。
”贾琏笑问二姐是谁,二姐一时也想不起来。
大家想来,贾琏便料定是此人无移了,便拍手笑道:“我知道了。
这人原不差,果然好眼力。
”二姐笑问是谁,贾琏笑道:“别人他如何进得去,一定是宝玉。
”二姐与尤老听了,亦以为然。
尤三姐便啐了一口,道:\ji{奇,不知何为。
}“我们有姊妹十个,也嫁你弟兄十个不成?\ji{有理之极!}难道除了你家,天下就没了好男子了不成!”\ji{一骂反有理。
}众人听了都诧异:“除去他,还有那一个?”\ji{余亦如此想。
}尤三姐笑道:“别只在眼前想,姐姐只在五年前想就是了。
”\ji{奇甚!}\par
正说着,忽见贾琏的心腹小厮兴儿走来请贾琏说:“老爷那边紧等着叫爷呢。
小的答应往舅老爷那边去了,小的连忙来请。
”贾琏又忙问:“昨日家里没人问?”兴儿道:“小的回奶奶说,爷在家庙里同珍大爷商议作百日的事,只怕不能来家。
”贾琏忙命拉马,隆儿跟随去了,留下兴儿答应人来事务。
\par
尤二姐拿了两碟菜,命拿大杯斟了酒,就命兴儿在炕沿下蹲着吃,一长一短向他说话儿。
问他家里奶奶多大年纪,怎个利害的样子,老太太多大年纪,太太多大年纪,姑娘几个,各样家常等语。
兴儿笑嘻嘻的在炕沿下一头吃,一头将荣府之事备细告诉他母女。
\zhu{备细:详尽。
}又说:“我是二门上该班的人。
我们共是两班,一班四个,共是八个。
这八个人有几个是奶奶的心腹,有几个是爷的心腹。
奶奶的心腹我们不敢惹,爷的心腹奶奶的就敢惹。
提起我们奶奶来,心里歹毒,口里尖快。
我们二爷也算是个好的,那里见得他。
\zhu{见:音“县”,显露、显出。
}
\ping{第二回:“谁知自娶了他令夫人之后,倒上下无一人不称颂他夫人的,琏爷倒退了一射之地。
”}
倒是跟前的平姑娘为人很好,虽然和奶奶一气,他倒背着奶奶常作些个好事。
小的们凡有了不是,奶奶是容不过的,只求求他去就完了。
如今合家大小除了老太太、太太两个人,没有不恨他的,只不过面子情儿怕他。
皆因他一时看的人都不及他,只一味哄着老太太、太太两个人喜欢。
他说一是一,说二是二,没人敢拦他。
又恨不得把银子钱省下来堆成山,好叫老太太、太太说他会过日子,殊不知苦了下人,他讨好儿。
估着有好事,他就不等别人去说,他先抓尖儿;或有了不好事或他自己错了,他便一缩头推到别人身上来,他还在旁边拨火儿。
如今连他正经婆婆大太太都嫌了他,\zhu{正经婆婆大太太:邢夫人。
}说他‘雀儿拣着旺处飞,黑母鸡一窝儿,自家的事不管,倒替人家去瞎张罗’。
若不是老太太在头里,早叫过他去了。
”\par
尤二姐笑道:“你背着他这等说他,将来你又不知怎么说我呢。
我又差他一层儿,越发有的说了。
”兴儿忙跪下说道:“奶奶要这样说,小的不怕雷打!但凡小的们有造化起来,先娶奶奶时若得了奶奶这样的人,小的们也少挨些打骂,也少提心吊胆的。
如今跟爷的这几个人,谁不背前背后称扬奶奶圣德怜下。
我们商量着叫二爷要出来,情愿来答应奶奶呢。
”尤二姐笑道:“猴儿肏的,还不起来呢。
说句顽话,就唬的那样起来。
你们作什么来,我还要找了你奶奶去呢。
”\par
兴儿连忙摇手说:“奶奶千万不要去。
我告诉奶奶,一辈子别见他才好。
嘴甜心苦,两面三刀;上头一脸笑,脚下使绊子;明是一盆火,暗是一把刀:都占全了。
只怕三姨的这张嘴还说他不过。
奶奶这样斯文良善人,那里是他的对手!”尤氏笑道:“我只以礼待他,他敢怎么样!”兴儿道:“不是小的吃了酒放肆胡说,奶奶便有礼让,他看见奶奶比他标致,又比他得人心,他怎肯干休善罢?人家是醋罐子,他是醋缸醋瓮。
凡丫头们二爷多看一眼,他有本事当着爷打个烂羊头。
\zhu{烂羊头:指人的头被打得血肉模糊。
}虽然平姑娘在屋里,大约一年二年之间两个有一次到一处,他还要口里掂十个过子呢,\zhu{掂十个过子:以一事作话柄,反复提起。
掂,掂量。
}
气的平姑娘性子发了,哭闹一阵,说:‘又不是我自己寻来的,你又浪着劝我,\zhu{浪着:撒娇,撒泼,耍赖(含诅咒意)。
}我原不依,你反说我反了,这会子又这样。
’他一般的也罢了,倒央告平姑娘。
”尤二姐笑道:“可是扯谎?这样一个夜叉,怎么反怕屋里的人呢?”兴儿道:“这就是俗语说的‘天下逃不过一个理字去’了。
这平儿是他自幼的丫头,陪了过来一共四个,嫁人的嫁人,死的死了,只剩了这个心腹。
他原为收了屋里,一则显他贤良名儿,二则又叫拴爷的心,好不外头走邪的。
又还有一段因果:我们家的规矩,凡爷们大了,未娶亲之先都先放两个人伏侍的。
二爷原有两个,谁知他来了没半年,都寻出不是来,都打发出去了。
别人虽不好说,自己脸上过不去,所以强逼着平姑娘作了房里人。
那平姑娘又是个正经人,从不把这一件事放在心上,也不会挑妻窝夫的,\zhu{挑妻窝夫:在夫妻家庭生活中挑拨是非。
}倒一味忠心赤胆伏侍他,才容下了。
”\par
尤二姐笑道:“原来如此。
但我听见你们家还有一位寡妇奶奶和几位姑娘。
他这样利害,这些人如何依得?”兴儿拍手笑道:“原来奶奶不知道。
我们家这位寡妇奶奶,他的浑名叫作‘大菩萨’,第一个善德人。
我们家的规矩又大,寡妇奶奶们不管事,只宜清净守节。
妙在姑娘又多,只把姑娘们交给他,看书写字,学针线,学道理,这是他的责任。
除此问事不知,说事不管。
只因这一向他病了,事多,这大奶奶暂管几日。
究竟也无可管,不过是按例而行,不像他多事逞才。
我们大姑娘不用说,但凡不好也没这段大福了。
二姑娘的浑名是‘二木头’,戳一针也不知嗳哟一声。
三姑娘的浑名是‘玫瑰花’。
”尤氏姊妹忙笑问何意。
兴儿笑道:“玫瑰花又红又香,无人不爱的,只是刺戳手。
也是一位神道,\zhu{神道:这里比喻有能耐、了不起的人物。
}可惜不是太太养的,‘老鸹窝里出凤凰’。
四姑娘小,他正经是珍大爷亲妹子,因自幼无母,老太太命太太抱过来养这么大,也是一位不管事的。
奶奶不知道,我们家的姑娘不算,另外有两个姑娘,真是天上少有,地下无双。
一个是咱们姑太太的女儿,姓林,小名儿叫什么黛玉,面庞身段和三姨不差什么,一肚子文章,只是一身多病,这样的天,还穿夹的,\zhu{夹的:即夹衣,有面有里,中间不衬垫絮类的衣服。
}出来风儿一吹就倒了。
我们这起没王法的嘴都悄悄的叫他‘多病西施’。
还有一位姨太太的女儿,姓薛,叫什么宝钗,竟是雪堆出来的。
\ping{
表面上通过“雪堆出来的”形容宝钗体态丰满、肤色白皙。二十八回宝玉眼中的宝钗有“雪白一段酥臂”。
宝钗姓薛,寓雪之意,联系到冷。宝钗的病缘于从娘胎里带来一股热毒,所以她的药叫做“冷香丸”。
所住的蘅芜苑“奇草仙藤愈冷愈苍翠……及进了房屋,雪洞一般”。
宝钗不施脂粉钗环:在第七回薛姨妈说“宝丫头古怪呢,他从来不爱这些花儿粉儿的。”
第五十七回宝钗劝岫烟时说“你看我从头至脚可有这些富丽闲妆?……总要一色从实守分为主。”
第六十三回宝钗掷出牡丹签,上附诗句:“任是无情也动人。”无情即是指宝钗的“冷”———事不关己不开口,一问摇头三不知。
}
每常出门或上车,或一时院子里瞥见一眼,我们鬼使神差,见了他两个,不敢出气儿。
”尤二姐笑道:“你们大家规矩,虽然你们小孩子进的去,然遇见小姐们,原该远远藏开。
”兴儿摇手道:“不是,不是。
那正经大礼,自然远远的藏开,自不必说。
就藏开了,自己不敢出气,是生怕这气大了,吹倒了姓林的;气暖了,吹化了姓薛的。
”说的满屋里都笑起来了。
要知端的,下回分解。
\par
\qi{总评:房内兄弟聚麀,\zhu{聚麀:指父子共占一个女子的禽兽行为。
麀:音“优”,母鹿。
}棚内两马相闹;小厮与家母饮酒,
\zhu{这句指本回中鲍二家的陪贾珍的小厮吃酒,讨他们的好,准备在贾珍前上好。}
小姨与姐夫同床。
可见有是主必有是奴,有是兄必有是弟,有是姐必有是妹,有是人必有是马。
}
\dai{130}{尤三姐酒席斥骂贾琏贾珍}
\dai{129}{兴儿介绍贾府情况}