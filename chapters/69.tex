\chapter{弄小巧用借剑杀人 \quad 觉大限吞生金自逝}
\zhu{大限:寿数,亦指死期。
}
\par
\qi{写凤姐写不尽,却从上下左右写。
写秋桐极淫邪,正写凤姐极淫邪;写平儿极义气,正写凤姐极不义气;写使女欺压二姐,正写凤姐欺压二姐;写下人感戴二姐,正写下人不感戴凤姐。
史公用意,非念死书子之所知。
}\par
话说尤二姐听了,又感谢不尽,只得跟了他来。
尤氏那边怎好不过来的,少不得也过来跟着凤姐去回,方是大礼。
凤姐笑说:“你只别说话,等我去说。
”尤氏道:“这个自然。
但一有个不是,是往你身上推的。
”说着,大家先来至贾母房中。
\par
正值贾母和园中姊妹们说笑解闷,忽见凤姐带了一个标致小媳妇进来,忙觑着眼看,\zhu{觑:音“去”,眯着眼注视。
}说:“这是谁家的孩子!好可怜见的。
”凤姐上来笑道:“老祖宗倒细细的看看,好不好?”说着,忙拉二姐说:“这是太婆婆,快磕头。
”二姐忙行了大礼,展拜起来。
\zhu{展拜:叩首,行跪拜之礼。
}又指着众姊妹说:这是某人某人,你先认了,太太瞧过了再见礼。
\zhu{见礼:见面行礼。
}二姐听了,一一又从新故意的问过,\zhu{从新:重新。
}垂头站在旁边。
贾母上下瞧了一遍,因又笑问:“你姓什么?今年十几了?”凤姐忙又笑说:“老祖宗且别问,只说比我俊不俊。
”贾母又戴了眼镜,命鸳鸯琥珀:“把那孩子拉过来,我瞧瞧肉皮儿。
”众人都抿嘴儿笑着,只得推他上去。
贾母细瞧了一遍,又命琥珀:“拿出手来我瞧瞧。
”鸳鸯又揭起裙子来。
贾母瞧毕,摘下眼镜来,笑说道:“更是个齐全孩子,我看比你俊些。
”凤姐听说,笑着忙跪下,将尤氏那边所编之话,一五一十细细的说了一遍,“少不得老祖宗发慈心,先许他进来,住一年后再圆房。
”贾母听了道:“这有什么不是。
既你这样贤良,很好。
只是一年后方可圆得房。
”凤姐听了,叩头起来,又求贾母着两个女人一同带去见太太们,说是老祖宗的主意。
贾母依允,遂使二人带去见了邢夫人等。
王夫人正因他风声不雅,深为忧虑,见他今行此事,岂有不乐之理。
于是尤二姐自此见了天日,挪到厢房住居。
\zhu{厢房:指四合院中东西两边的房子。}
\par
凤姐一面使人暗暗调唆张华,只叫他要原妻,这里还有许多赔送外,还给他银子安家过活。
张华原无胆无心告贾家的,后来又见贾蓉打发人来对词,那人原说的:“张华先退了亲。
我们皆是亲戚。
接到家里住着是真,并无娶嫁之说。
皆因张华拖欠了我们的债务,追索不与,方诬赖小的主人那些个。
”察院都和贾王两处有瓜葛,况又受了贿,只说张华无赖,以穷讹诈,状子也不收,打了一顿赶出来。
庆儿在外替他打点,也没打重。
又调唆张华:“亲原是你家定的,你只要亲事,官必还断给你。
”于是又告。
王信那边又透了消息与察院,察院便批:“张华所欠贾宅之银,令其限内按数交还,其所定之亲,仍令其有力时娶回。
”又传了他父亲来当堂批准。
他父亲亦系庆儿说明,乐得人财两进,便去贾家领人。
\par
凤姐儿一面吓的来回贾母,说如此这般,都是珍大嫂子干事不明,并没和那家退准,惹人告了,如此官断。
贾母听了,忙唤了尤氏过来,说他作事不妥,“既是你妹子从小曾与人指腹为婚,又没退断,\zhu{退断:彻底退亲了断。
}使人混告了。
”尤氏听了,只得说:“他连银子都收了,怎么没准。
”凤姐在旁又说:“张华的口供上现说不曾见银子,也没见人去。
他老子说:‘原是亲家母说过一次,并没应准。
亲家母死了,你们就接进去作二房。
’如此没有对证,只好由他去混说。
幸而琏二爷不在家,没曾圆房,这还无妨。
只是人已来了,怎好送回去,岂不伤脸。
”贾母道:“又没圆房,没的强占人家有夫之人,名声也不好,不如送给他去。
那里寻不出好人来。
”尤二姐听了,又回贾母说:“我母亲实于某年月日给了他十两银子退准的。
他因穷急了告,又翻了口。
我姐姐原没错办。
”贾母听了,便说:“可见刁民难惹。
既这样,凤丫头去料理料理。
”凤姐听了无法,只得应着。
回来只命人去找贾蓉。
\par
贾蓉深知凤姐之意,若要使张华领回,成何体统,便回了贾珍,暗暗遣人去说张华:“你如今既有许多银子,何必定要原人。
若只管执定主意,岂不怕爷们一怒,寻出个由头,你死无葬身之地。
你有了银子,回家去什么好人寻不出来。
你若走时,还赏你些路费。
”张华听了,心中想了一想,这倒是好主意,和父亲商议已定,约共也得了有百金,父子次日起个五更,回原籍去了。
贾蓉打听得真了,来回了贾母凤姐,说:“张华父子妄告不实,惧罪逃走,官府亦知此情,也不追究,大事完毕。
”凤姐听了,心中一想:若必定着张华带回二姐去,\zhu{着:命令、差使。
}未免贾琏回来再花几个钱包占住,不怕张华不依。
还是二姐不去,自己相伴着还妥当,且再作道理。
只是张华此去不知何往,他倘或再将此事告诉了别人,或日后再寻出这由头来翻案,岂不是自己害了自己。
\ping{张华最后并没有被杀死,这为后文凤姐因此东窗事发做铺垫。
}原先不该如此将刀靶付与外人去的。
因此悔之不迭,复又想了一条主意出来,悄命旺儿遣人寻着了他,或说他作贼,和他打官司将他治死;或暗中使人算计,务将张华治死,方剪草除根,保住自己的名誉。
旺儿领命出来,回家细想:人已走了完事,何必如此大作,人命关天,非同儿戏,我且哄过他去,再作道理。
因此在外躲了几日,回来告诉凤姐,只说张华是有了几两银子在身上,逃去第三日在京口地界五更天已被截路人打闷棍打死了。
他老子唬死在店房,\zhu{唬:同“吓”。
}在那里验尸掩埋。
凤姐听了不信,说:“你要扯谎,我再使人打听出来敲你的牙!”自此方丢过不究。
\ping{这原本不该成为纰漏,但是当时凤姐显然因为情绪影响心智,水准大失,何况做了许多亏心事,总会有失手的时候。
}凤姐和尤二姐和美非常,更比亲姊亲妹还胜十倍。
\par
那贾琏一日事毕回来,先到了新房中,已竟悄悄的封锁,只有一个看房子的老头儿。
贾琏问他原故,老头子细说原委,贾琏只在镫中跌足。
\zhu{镫[dèng]:挂在马鞍两旁的踏脚。
跌足:跺脚。
}
少不得来见贾赦与邢夫人,将所完之事回明。
贾赦十分欢喜,说他中用,赏了他一百两银子,又将房中一个十七岁的丫鬟名唤秋桐者,赏他为妾。
贾琏叩头领去,喜之不尽。
见了贾母和家中人,回来见凤姐,未免脸上有些愧色。
谁知凤姐儿他反不似往日容颜,同尤二姐一同出迎,叙了寒温。
贾琏将秋桐之事说了,未免脸上有些得意之色,骄矜之容。
\zhu{矜:自夸、自负;骄傲自大。
}凤姐听了,忙命两个媳妇坐车在那边接了来。
心中一刺未除,又平空添了一刺,说不得且吞声忍气,将好颜面换出来遮掩。
一面又命摆酒接风,一面带了秋桐来见贾母与王夫人等。
贾琏心中也暗暗的纳罕。
\par
那日已是腊月十二日,贾珍起身,先拜了宗祠,然后过来辞拜贾母等人。
和族中人直送到洒泪亭方回,独贾琏贾蓉二人送出三日三夜方回。
一路上贾珍命他好生收心治家等语,二人口内答应,也说些大礼套话,不必烦叙。
\par
且说凤姐在家,外面待尤二姐自不必说得,只是心中又怀别意。
无人处只和尤二姐说:“妹妹的声名很不好听,连老太太、太太们都知道了,说妹妹在家做女孩儿就不干净,又和姐夫有些首尾,\zhu{首尾:头和尾,引申指事情的经过始末。
这里指男女关系。
}‘没人要的了你拣了来,还不休了再寻好的。
’我听见这话,气得倒仰,查是谁说的,又查不出来。
这日久天长,这些个奴才们跟前,怎么说嘴。
我反弄了个鱼头来拆。
”\zhu{拆鱼头:也作“择鱼头”,比喻处理和排解复杂难办的事。
拆,这里读作“宅”,分解、清理的意思。
一说,把筵席上的鱼头拆开了好让大家吃,引申为与人方便,宁可自找麻烦。
}说了两遍,自己又气病了,茶饭也不吃,除了平儿,众丫头媳妇无不言三语四,指桑说槐,暗相讥刺。
秋桐自为系贾赦之赐,无人僭他的,\zhu{僭:音“贱”,超越本分。
}连凤姐平儿皆不放在眼里,岂肯容他。
张口是“先奸后娶没汉子要的娼妇,也来要我的强。
”凤姐听了暗乐,尤二姐听了暗愧暗怒暗气。
凤姐既装病,便不和尤二姐吃饭了。
每日只命人端了菜饭到他房中去吃,那茶饭都系不堪之物。
平儿看不过,自拿了钱出来弄菜与他吃,或是有时只说和他园中去顽,在园中厨内另做了汤水与他吃,也无人敢回凤姐。
只有秋桐一时撞见了,便去说舌告诉凤姐说:“奶奶的名声,生是平儿弄坏了的。
这样好菜好饭浪着不吃,\zhu{浪着:撒娇,撒泼,耍赖(含诅咒意)。
}却往园里去偷吃。
”凤姐听了,骂平儿说:“人家养猫拿耗子,我的猫只倒咬鸡。
”平儿不敢多说,自此也要远着了。
又暗恨秋桐,难以出口。
\par
园中姊妹如李纨、迎春、惜春等人,皆为凤姐是好意,然宝、黛一干人暗为二姐担心。
虽都不便多事,惟见二姐可怜,常来了,倒还都悯恤他。
每日常无人处说起话来,尤二姐便淌眼抹泪,又不敢抱怨。
凤姐儿又并无露出一点坏形来。
贾琏来家时,见了凤姐贤良,也便不留心。
况素习以来因贾赦姬妾丫鬟最多,贾琏每怀不轨之心,只未敢下手。
如这秋桐辈等人,皆是恨老爷年迈昏愦,贪多嚼不烂,没的留下这些人作什么,因此除了几个知礼有耻的,馀者或有与二门上小幺儿们嘲戏的。
\zhu{小幺儿:身边使唤的小仆人。
幺(音“妖”):幼小。
}
甚至于与贾琏眉来眼去相偷期的,\zhu{期:预定的时间,引申为约会。
}只惧贾赦之威,未曾到手。
这秋桐便和贾琏有旧,从未来过一次。
今日天缘凑巧,竟赏了他,真是一对烈火干柴,如胶投漆,燕尔新婚,\zhu{燕尔新婚:新婚和美。
燕尔:即燕好、和美,指夫妻和谐。
燕:安乐。
}连日那里拆的开。
那贾琏在二姐身上之心也渐渐淡了,只有秋桐一人是命。
凤姐虽恨秋桐,且喜借他先可发脱二姐,\zhu{发脱:打发。
}自己且抽头,用“借剑杀人”之法,“坐山观虎斗”,等秋桐杀了尤二姐,自己再杀秋桐。
主意已定,没人处常又私劝秋桐说:“你年轻不知事。
他现是二房奶奶,你爷心坎儿上的人,我还让他三分,你去硬碰他,岂不是自寻其死?”那秋桐听了这话,越发恼了,天天大口乱骂说:“奶奶是软弱人,那等贤惠,我却做不来。
奶奶把素日的威风怎都没了。
奶奶宽洪大量,我却眼里揉不下沙子去。
让我和他这淫妇做一回,他才知道。
”凤姐儿在屋里,只装不敢出声儿。
气的尤二姐在房里哭泣,饭也不吃,又不敢告诉贾琏。
次日贾母见他眼红红的肿了,问他,又不敢说。
秋桐正是抓乖卖俏之时,\zhu{抓乖卖俏:运用聪明,卖弄机巧。
}他便悄悄的告诉贾母王夫人等说:“专会作死,好好的成天家号丧,背地里咒二奶奶和我早死了,他好和二爷一心一计的过。
”贾母听了便说:“人太生娇俏了,可知心就嫉妒。
凤丫头倒好意待他,他倒这样争锋吃醋的。
可是个贱骨头。
”因此渐次便不大喜欢。
众人见贾母不喜,不免又往下踏践起来,弄得这尤二姐要死不能,要生不得。
还是亏了平儿,时常背着凤姐,看他这般,与他排解排解。
\par
那尤二姐原是个花为肠肚雪作肌肤的人,如何经得这般磨折,不过受了一个月的暗气,便恹恹得了一病,\zhu{恹恹:音“烟烟”,困倦或忧郁的样子。
}四肢懒动,茶饭不进,渐次黄瘦下去。
夜来合上眼,只见他小妹子手捧鸳鸯宝剑前来说:“姐姐,你一生为人心痴意软,终吃了这亏。
休信那妒妇花言巧语,外作贤良,内藏奸狡,他发恨定要弄你一死方休。
若妹子在世,断不肯令你进来,即进来时,亦不容他这样。
此亦系理数应然,你我生前淫奔不才,使人家丧伦败行,故有此报。
你依我将此剑斩了那妒妇,一同归至警幻案下,听其发落。
不然,你则白白的丧命,且无人怜惜。
”尤二姐泣道:“妹妹,我一生品行既亏,今日之报既系当然,何必又生杀戮之冤。
随我去忍耐。
\zhu{随:听任、任凭。
}
若天见怜,使我好了,岂不两全。
”小妹笑道:“姐姐,你终是个痴人。
自古‘天网恢恢,疏而不漏’,\zhu{天网恢恢,疏而不漏:《老子》:“天网恢恢,疏而不失”。
明代薛蕙集解:“世之禁网虽密,然人多幸免者;惟天网恢恢广大,有若疏而不密,而为恶之人,无有能逃也。
”天网:天道如网;后亦喻指国法。
恢恢:宽广的样子。
}天道好还。
\zhu{天道好还:上天所行之道是喜欢报复的。
好:音“号”,喜爱。
还:报偿;报复。
《老子》:“以道佐人主者,不以兵强天下,其事好还。
师之所处,荆棘生焉;大军之后,必有凶年。
”}你虽悔过自新,然已将人父子兄弟致于麀聚之乱,\zhu{麀聚:即“聚麀”。
聚麀:指父子共占一个女子的禽兽行为。
麀:音“优”,母鹿。
}天怎容你安生。
”尤二姐泣道:“既不得安生,亦是理之当然,奴亦无怨。
”小妹听了,长叹而去。
尤二姐惊醒,却是一梦。
等贾琏来看时,因无人在侧,便泣说:“我这病便不能好了。
我来了半年,腹中也有身孕,但不能预知男女。
倘天见怜,生了下来还可,若不然,我这命就不保,何况于他。
”贾琏亦泣说:“你只放心,我请明人来医治。
”\zhu{明:英明,明智,高明。
}于是出去即刻请医生。
\par
谁知王太医亦谋干了军前效力,回来好讨荫封的。
小厮们走去,便请了个姓胡的太医,名叫君荣。
进来诊脉看了,说是经水不调,全要大补。
贾琏便说:“已是三月庚信不行,\zhu{庚信:亦称月信,即月经。
}又常作呕酸,恐是胎气。
”胡君荣听了,复又命老婆子们请出手来再看看。
尤二姐少不得又从帐内伸出手来。
胡君荣又诊了半日,说:“若论胎气,肝脉自应洪大。
\zhu{
因患者左手关部脉象常反映了脏腑内肝的生理病理情况,故称“肝脉”。
}然木盛则生火,经水不调亦皆因由肝木所致。
医生要大胆,须得请奶奶将金面略露露,医生观观气色,方敢下药。
”贾琏无法,只得命将帐子掀起一缝,尤二姐露出脸来。
胡君荣一见,魂魄如飞上九天,通身麻木,一无所知。
一时掩了帐子,贾琏就陪他出来,问是如何。
胡太医道:“不是胎气,只是瘀血凝结。
如今只以下瘀血通经脉要紧。
”于是写了一方,作辞而去。
\par
贾琏命人送了药礼,抓了药来,调服下去。
只半夜,尤二姐腹痛不止,谁知竟将一个已成形的男胎打了下来。
于是血行不止,二姐就昏迷过去。
贾琏闻知,大骂胡君荣。
一面再遣人去请医调治,一面命人去打告胡君荣。
胡君荣听了,早已卷包逃走。
这里太医便说:“本来气血生成亏弱,受胎以来,想是着了些气恼,郁结于中。
这位先生擅用虎狼之剂,如今大人元气十分伤其八九,一时难保就愈。
煎丸二药并行,还要一些闲言闲事不闻,庶可望好。
”说毕而去。
急的贾琏查是谁请了姓胡的来,一时查了出来,便打了半死。
凤姐比贾琏更急十倍,只说:“咱们命中无子,好容易有了一个,又遇见这样没本事的大夫。
”于是天地前烧香礼拜,\zhu{天地:天地神灵。
}自己通陈祷告说:\zhu{通陈:详细地陈述。
}“我或有病,只求尤氏妹子身体大愈,再得怀胎生一男子,我愿吃长斋念佛。
”贾琏众人见了,无不称赞。
贾琏与秋桐在一处时,凤姐又做汤做水的着人送与二姐。
又骂平儿不是个有福的,“也和我一样。
我因多病了,你却无病也不见怀胎。
如今二奶奶这样,都因咱们无福,或犯了什么,冲的他这样。
”因又叫人出去算命打卦。
偏算命的回来又说:“系属兔的阴人冲犯。
”大家算将起来,只有秋桐一人属兔,说他冲的。
\par
秋桐近见贾琏请医治药,打人骂狗,为尤二姐十分尽心,他心中早浸了一缸醋在内了。
今又听见如此说他冲了,凤姐儿又劝他说:“你暂且别处去躲几个月再来。
”秋桐便气的哭骂道:“理那起瞎肏的混咬舌根!我和他‘井水不犯河水’,怎么就冲了他!好个爱八哥儿,\zhu{爱八哥儿:可爱的东西。
亦可用作反语,讽刺被宠爱的人。
}在外头什么人不见,偏来了就有人冲了。
白眉赤脸,\zhu{白眉赤脸:也作“白眉赤眼”,平白无故、没来由的意思。
}那里来的孩子?他不过指着哄我们那个棉花耳朵的爷罢了。
\zhu{指:希望、依赖。
如“指望”。
棉花耳朵:比喻人耳软,没有主见,喜欢听信别人的闲话。
}纵有孩子,也不知姓张姓王。
奶奶希罕那杂种羔子,我不喜欢!老了谁不成?谁不会养!\zhu{养:这里指生孩子。
}一年半载养一个,倒还是一点搀杂没有的呢!”骂的众人又要笑,又不敢笑。
可巧邢夫人过来请安,秋桐便哭告邢夫人说:“二爷奶奶要撵我回去,我没了安身之处,太太好歹开恩。
”邢夫人听说,慌的数落凤姐儿一阵,又骂贾琏:“不知好歹的种子,凭他怎不好,是你父亲给的。
为个外头来的撵他,连老子都没了。
你要撵他,你不如还你父亲去倒好。
”说着,赌气去了。
秋桐更又得意,越性走到他窗户根底下大哭大骂起来。
尤二姐听了,不免更添烦恼。
\par
晚间,贾琏在秋桐房中歇了,凤姐已睡,平儿过来瞧他,又悄悄劝他:“好生养病,不要理那畜生。
”尤二姐拉他哭道:“姐姐,我从到了这里,多亏姐姐照应。
为我,姐姐也不知受了多少闲气。
我若逃的出命来,我必答报姐姐的恩德,只怕我逃不出命来,也只好等来生罢。
”平儿也不禁滴泪说道:“想来都是我坑了你。
我原是一片痴心,从没瞒他的话。
既听见你在外头,岂有不告诉他的。
谁知生出这些个事来。
”尤二姐忙道:“姐姐这话错了。
若姐姐便不告诉他,他岂有打听不出来的,不过是姐姐说的在先。
况且我也要一心进来,方成个体统,与姐姐何干。
”二人哭了一回,平儿又嘱咐了几句,夜已深了,方去安息。
\par
这里尤二姐心下自思:“病已成势,日无所养,反有所伤,料定必不能好。
况胎已打下,无可悬心,何必受这些零气,
\zhu{
零气:随时随地为一些小事所受的气。
}
不如一死,倒还干净。
常听见人说,生金子可以坠死,岂不比上吊自刎又干净。
”想毕,拃挣起来,\zhu{拃挣:也作“扎挣”,勉强支持。
}打开箱子,找出一块生金,也不知多重,恨命含泪便吞入口中,几次狠命直脖,方咽了下去。
于是赶忙将衣服首饰穿戴齐整,上炕躺下了。
当下人不知,鬼不觉。
到第二日早晨,丫鬟媳妇们见他不叫人,乐得且自己去梳洗。
凤姐便和秋桐都上去了。
\zhu{上去了:可能是指早上去向长辈问安,即“晨省”。
}平儿看不过,说丫头们:“你们就只配没人心的打着骂着使也罢了,一个病人,也不知可怜可怜。
他虽好性儿,你们也该拿出个样儿来,别太过逾了,墙倒众人推。
”丫鬟听了,急推房门进来看时,却穿戴的齐齐整整,死在炕上。
于是方吓慌了,喊叫起来。
平儿进来看了,不禁大哭。
众人虽素习惧怕凤姐,然想尤二姐实在温和怜下,比凤姐原强,如今死去,谁不伤心落泪,只不敢与凤姐看见。
\par
当下合宅皆知。
贾琏进来,搂尸大哭不止。
凤姐也假意哭:“狠心的妹妹!你怎么丢下我去了,辜负了我的心!”尤氏贾蓉等也来哭了一场,劝住贾琏。
贾琏便回了王夫人,讨了梨香院停放五日,挪到铁槛寺去,王夫人依允。
贾琏忙命人去开了梨香院的门,收拾出正房来停灵。
贾琏嫌后门出灵不像,\zhu{不像:不像样儿,不体面。
}便对着梨香院的正墙上通街现开了一个大门。
两边搭棚,安坛场做佛事。
用软榻铺了锦缎衾褥,将二姐抬上榻去,用衾单盖了。
八个小厮和几个媳妇围随,从内子墙一带抬往梨香院来。
\zhu{内子墙:是和“外围墙”或“外界墙”相对而言的墙,即一座府第中两路相邻院落之间夹道两侧的墙。
}那里已请下天文生预备,\zhu{天文生:本为明、清时代钦天监官员的职称之一,主要掌管对星辰、晴雨、风雷、云霓等天象气候的观测与推算。
这里是指旧时以择日、占卜、看风水、选阴阳宅等迷信活动为职业的人,也称“阴阳先生”、“风水先生”或“堪舆先生”。
}揭起衾单一看,只见这尤二姐面色如生,比活着还美貌。
贾琏又搂着大哭,只叫“奶奶,你死的不明,都是我坑了你!”贾蓉忙上来劝:“叔叔解着些儿,我这个姨娘自己没福。
”说着,又向南指大观园的界墙,贾琏会意,只悄悄跌脚说:\zhu{跌脚:跺脚。
}“我忽略了,终久对出来,我替你报仇。
”\ping{伏后文凤姐东窗事发与夫妻反目。
}天文生回说:“奶奶卒于今日正卯时,
\zhu{正卯:即卯正,上午六点。}
五日出不得,或是三日,或是七日方可。
明日寅时入殓大吉。
\zhu{寅时:指凌晨3—5点的时间。}
”贾琏道:“三日断乎使不得,竟是七日。
因家叔家兄皆在外,小丧不敢多停,等到外头,还放五七,
\zhu{五七:即第五个“七”。人死后,每七天为一周期,请僧道祭祀超度,称为“七”。通常经七个“七”才掩灵停祭。}
做大道场才掩灵。
\zhu{掩灵:暂掩灵柩,等待安葬。
}明年往南去下葬。
”天文生应诺,写了殃榜而去。
\zhu{殃榜:旧时由阴阳先生给死者写的文书,上有死者的年寿及“招魂”的话。
}宝玉已早过来陪哭一场。
众族中人也都来了。
\par
贾琏忙进去,找凤姐要银子治办棺椁丧礼。
凤姐见抬了出去,推有病,回:“老太太、太太说我病着,忌三房,\zhu{忌三房:旧俗,生病的人忌进新房、产房和灵房,称为“忌三房”。
}不许我去。
”因此也不出来穿孝,且往大观园中来。
绕过群山,至北界墙根下往外听,隐隐绰绰听了一言半语,\zhu{隐隐绰绰:即影影绰绰,模模糊糊,不真切。
}\zhu{本回前文:贾蓉忙上来劝:“叔叔解着些儿,我这个姨娘自己没福。
”说着,又向南指大观园的界墙,贾琏会意,只悄悄跌脚说:“我忽略了,终久对出来,我替你报仇。
”}回来又回贾母说如此这般。
贾母道:“信他胡说,谁家痨病死的孩子不烧了一撒,\ping{凤姐骗贾母。
}也认真的开丧破土起来。
既是二房一场,也是夫妻之分,停五七日抬出来,或一烧或乱葬地上埋了完事。
”凤姐笑道:“可是这话。
我又不敢劝他。
”正说着,丫鬟来请凤姐,说:“二爷等着奶奶拿银子呢。
”凤姐只得来了,便问他“什么银子?家里近来艰难,你还不知道?咱们的月例,一月赶不上一月,鸡儿吃了过年粮。
昨儿我把两个金项圈当了三百银子,你还做梦呢。
这里还有二三十两银子,你要就拿去。
”说着,命平儿拿了出来,递与贾琏,指着贾母有话,又去了。
恨的贾琏没话可说,只得开了尤氏箱柜,去拿自己的梯己。
及开了箱柜,一滴无存,只有些折簪烂花并几件半新不旧的绸绢衣裳,都是尤二姐素习所穿的,不禁又伤心哭了起来。
自己用个包袱一齐包了,也不命小丫鬟来拿,便自己提着来烧。
\par
平儿又是伤心,又是好笑,忙将二百两一包的碎银子偷了出来,到厢房拉住贾琏,悄递与他说:“你只别作声才好,你要哭,外头多少哭不得,又跑了这里来点眼。
”\zhu{点眼:画上眼睛。
也作“点睛”。
故意做出引人注目的动作。
}贾琏听说,便说:“你说的是。
”接了银子,又将一条裙子递与平儿,说:“这是他家常穿的,你好生替我收着,作个念心儿。
”\zhu{念心儿:留作纪念的东西。
也作“念信儿”、“念相儿”、“念想儿”。
}平儿只得掩了,自己收去。
贾琏拿了银子与衣服,走来命人先去买板。
好的又贵,中的又不要。
贾琏骑马自去要瞧,至晚间果抬了一副好板进来,价银五百两赊着,连夜赶造。
一面分派了人口穿孝守灵,晚来也不进去,\zhu{晚来:傍晚,入夜。
}只在这里伴宿。
正是——\par
\qi{总评:凤姐初念在张华领出二姐,转念又恐仍为外宅,转念即欲杀张华,为斩草除根计。
一时写来觉满腔都是荆棘,浑身都是爪牙,安得借鸳鸯剑手刃其首,以寒千古奸妇之胆。
\hang
看三姐梦中相叙一段,真有孝子悌弟、义士忠臣之慨,我不禁泪流一斗,湿地三尺。
}
\dai{137}{秋桐屋外骂尤二姐}
\dai{138}{觉大限吞生金自逝}
\sun{p69-1}{尤二姐见贾母,尤三姐夜托梦}{图右侧:凤姐带尤二姐拜见贾母。
贾母戴了眼镜,瞧尤二姐的手。
图左上:尤二姐夜里做梦,见尤三姐捧鸳鸯剑前来劝二姐剑斩妒妇,不受此窝囊气。
}