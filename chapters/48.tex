\chapter{滥情人情误思游艺 \quad 慕雅女雅集苦吟诗}
\zhu{游艺:《论语·述而》:“游于艺”。
指沉潜于六艺之教,后亦泛指从事技术或艺术的锻炼。
}\par
\geng{题曰“柳湘莲走他乡”,
\zhu{前一回的题名为“冷郎君惧祸走他乡”。}
必谓写湘莲如何走,今却不写,反细写阿呆兄之游艺\sout{了心却}[之心切];柳湘莲之分内走者而不细写其走,反写阿呆不应走而写其走,文牵歧路,\zhu{文牵歧路:大概是写文章剑走偏锋,不按常理出牌的意思。
}令人不识者如此。
\hang
至“情小妹”回中,
\zhu{“情小妹”回:第六十六回“情小妹耻情归地府,冷二郎一冷入空门”。}
方写湘莲文字,真神化之笔。
}\par
\qi{心地聪明性自灵,喜同雅品讲诗经,\zhu{诗经:“诗经”本指周初到春秋五百多年间的诗歌总集,共三百零五篇,汉朝以后被尊为儒家“五经”之首,不过这里是泛指黛玉和香菱谈讲的诗词方面的经典,即“《王摩诘全集》”、“老杜的七言律”、“李青莲的七言绝句”、“陶渊明、应、谢、阮、庾、鲍等人”这些“诗经”。
}姣柔倍觉可怜形。
 皓齿朱唇真袅袅,痴情专意更娉娉,\zhu{袅袅、娉娉[pīng]:女子体态轻盈柔美的样子。
杜牧《赠别》诗:“娉娉袅袅十三余,豆蔻梢头二月初。
”}宜人解语小星星。
\zhu{小星星:是指香菱的身份是薛蟠的小妾,不是正室。
《诗经·召南》有《小星》篇,首句“嘒彼小星,三五在东”被汉朝郑玄解释为是说周王的众妾,后人就以“小星”作为妾的代称。
}}\par
且说薛蟠听见如此说了,气方渐平。
三五日后,疼痛虽愈,伤痕未平,只装病在家,愧见亲友。
\par
展眼已到十月,因有各铺面伙计内有算年帐要回家的,少不得家内治酒饯行。
内有一个张德辉,年过六十,自幼在薛家当铺内揽总,\zhu{揽总:总揽,全面掌握。
}家内也有二三千金的过活,\zhu{过活:用以维生的产业、财物。
}今岁也要回家,明春方来。
因说起“今年纸札香料短少,\zhu{纸札:也作“纸扎”。
这里是“纸张”的意思。
札:古代无纸,字写在小木板上,叫“札”。
}明年必是贵的。
明年先打发大小儿上来当铺内照管,\zhu{大小儿:应该是店铺大小伙计的意思。
}赶端阳前我顺路贩些纸札香扇来卖。
除去关税花销,亦可以剩得几倍利息。
”薛蟠听了,心中忖度:“我如今捱了打,正难见人,想着要躲个一年半载,又没处去躲。
天天装病,也不是事。
况且我长了这么大,文又不文,武又不武,虽说做买卖,究竟戥子算盘从没拿过,\zhu{戥(戥音“等”)子:一种称量金银、药品等所用的小秤,计量单位从分厘到两,构造和原理跟杆秤相同,盛物体的部分是一个小盘子。
}地土风俗远近道路又不知道,不如也打点几个本钱,和张德辉逛一年来。
赚钱也罢,不赚钱也罢,且躲躲羞去。
二则逛逛山水也是好的。
”心内主意已定,至酒席散后,便和张德辉说知,命他等一二日一同前往。
\par
晚间薛蟠告诉了他母亲。
薛姨妈听了虽是欢喜,但又恐他在外生事,花了本钱倒是末事,因此不命他去,只说:“好歹你守着我,我还能放心些。
况且也不用做这买卖,也不等着这几百银子来用。
你在家里安分守己的,就强似这几百银子了。
”\ping{薛蟠不学无术,和薛姨妈的宠溺娇惯很有关系。
}薛蟠主意已定,那里肯依,只说:“天天又说我不知世事,这个也不知,那个也不学。
如今我发狠把那些没要紧的都断了,如今要成人立事,学习着做买卖,又不准我了,叫我怎么样呢?我又不是个丫头,把我关在家里,何日是个了日?况且那张德辉又是个年高有德的,咱们和他世交,我同他去,怎么得有舛错?\zhu{舛:音“穿”三声,违背,错谬。
}我就一时半刻有不好的去处,他自然说我劝我。
就是东西贵贱行情,他是知道的,自然色色问他,\zhu{色色:样样。
}何等顺利,倒不叫我去。
过两日我不告诉家里,私自打点了一走,明年发了财回家,那时才知道我呢。
”说毕,赌气睡觉去了。
\par
薛姨妈听他如此说,因和宝钗商议。
宝钗笑道:“哥哥果然要经历正事,正是好的了。
只是他在家时说着好听,到了外头旧病复犯,越发难拘束他了。
但也愁不得许多。
他若是真改了,是他一生的福。
若不改,妈也不能又有别的法子。
一半尽人力,一半听天命罢了。
这么大人了,若只管怕他不知世路,出不得门,干不得事,今年关在家里,明年还是这个样儿。
他既说的名正言顺,妈就打量着丢了八百一千银子,竟交与他试一试。
横竖有伙计们帮着,也未必好意思哄骗他的。
二则他出去了,左右没有助兴的人,又没了倚仗的人,到了外头,谁还怕谁,有了的吃,没了的饿着,举眼无靠,他见这样,只怕比在家里省了事也未可知。
”\geng{作书者曾吃此亏,批书者亦曾吃此亏,故特于此注明,使后人深思默戒。
脂砚斋。
}
薛姨妈听了,思忖半晌说道:“倒是你说的是。
花两个钱,叫他学些乖来也值了。
”商议已定,一宿无话。
\par
至次日,薛姨妈命人请了张德辉来,在书房中命薛蟠款待酒饭,自己在后廊下,隔着窗子,向里千言万语嘱托张德辉照管薛蟠。
张德辉满口应承,吃过饭告辞,又回说:“十四日是上好出行日期,大世兄即刻打点行李,雇下骡子,十四一早就长行了。
”薛蟠喜之不尽,将此话告诉了薛姨妈。
薛姨妈便和宝钗香菱并两个老年的嬷嬷连日打点行装,派下薛蟠之乳父老苍头一名,\zhu{乳父:乳母的丈夫。
苍头:指老年的奴仆。
}当年谙事旧仆二名,\zhu{谙:音“安”,熟悉。
}外有薛蟠随身常使小厮二人,主仆一共六人,雇了三辆大车,单拉行李使物,
\zhu{使物:日常使用的物品。}
又雇了四个长行骡子。
薛蟠自骑一匹家内养的铁青大走骡,\zhu{青:黑。
铁青:青黑色。
}外备一匹坐马。
诸事完毕,薛姨妈宝钗等连夜劝戒之言,自不必备说。
\zhu{备:周详,齐全。
}\par
至十三日,薛蟠先去辞了他舅舅,然后过来辞了贾宅诸人。
贾珍等未免又有饯行之说,也不必细述。
至十四日一早,薛姨妈宝钗等直同薛蟠出了仪门,母女两个四只泪眼看他去了,方回来。
\par
薛姨妈上京带来的家人不过四五房,并两三个老嬷嬷小丫头,今跟了薛蟠一去,外面只剩了一两个男子。
因此薛姨妈即日到书房,将一应陈设玩器并帘幔等物尽行搬了进来收贮,命那两个跟去的男子之妻一并也进来睡觉。
又命香菱将他屋里也收拾严紧,“将门锁了,晚间和我去睡。
”宝钗道:“妈既有这些人作伴,不如叫菱姐姐和我作伴去。
我们园里又空,夜长了,我每夜作活,越多一个人岂不越好。
”薛姨妈听了,笑道:“正是我忘了,原该叫他同你去才是。
我前日还同你哥哥说,文杏又小,道三不着两,\zhu{道三不着两:也作“着三不着两”、“到三不着两”,谓说话或行事轻重失宜,考虑不周,注意这里,顾不到那里。
}莺儿一个人不够伏侍的,还要买一个丫头来你使。
”宝钗道:“买的不知底里,倘或走了眼,花了钱小事,没的淘气。
倒是慢慢的打听着,有知道来历的,买个还罢了。
”\geng{闲言过耳无迹,然又伏下一事矣。
\ping{第八十回,薛蟠之妻夏金桂吃醋,设计陷害香菱,激怒薛蟠打香菱,大闹中薛姨妈气急,要卖了香菱,宝钗拦住不让卖,香菱从此跟着服侍薛宝钗,和薛蟠断绝了往来。
}}一面说,一面命香菱收拾了衾褥妆奁,命一个老嬷嬷并臻儿送至蘅芜苑去,然后宝钗和香菱才同回园中来。
\geng{细想香菱之为人也,根基不让迎、探,容貌不让凤、秦,端雅不让纨、钗,风流不让湘、黛,贤惠不让袭、平,所惜者青年罹祸,命运乖蹇,至为侧室,且虽曾读书,不能与林、湘辈并驰于海棠之社耳。
然此一人岂可不入园哉?故欲令入园,终无可入之隙,筹划再四,欲令入园必呆兄远行后方可。
然阿呆兄又如何方可远行?曰名,不可;利,不可;无事,不可;必得万人想不到,自己忽一发机之事方可。
\zhu{发机:拨动弩弓的射箭机关,引申为开始行动的时机。
}因此思及“情”之一字及呆素所误者,故借“情误”二字生出一事,使阿呆游艺之志已坚,则菱卿入园之隙方妥。
回思因欲香菱入园,是写阿呆情误,因欲阿呆情误,先写一赖尚荣,实委婉严密之甚也。
脂砚斋评。
}\par
香菱道:“我原要和奶奶说的,大爷去了,我和姑娘作伴儿去。
又恐怕奶奶多心,说我贪着园里来顽;谁知你竟说了。
”宝钗笑道:“我知道你心里羡慕这园子不是一日两日了,只是没个空儿。
就每日来一趟,慌慌张张的,也没趣儿。
所以趁着机会,越性住上一年,我也多个作伴的,你也遂了心。
”香菱笑道:“好姑娘,你趁着这个功夫,教给我作诗罢。
”\geng{写得何其有趣,今忽见菱卿此句,合卷从纸上另走出一娇小美人来,并不是湘、林、探、凤等一样口气声色。
真神骏之技,虽驱驰万里而不见有倦怠之色。
}
宝钗笑道:“我说你‘得陇望蜀’呢。
\zhu{得陇望蜀:《后汉书·岑彭传》:“人苦不知足,既平陇(陇西,古郡名,在今甘肃省),复望蜀(古郡名,今四川)。
”后以喻人之贪得无厌。
}我劝你今儿头一日进来,先出园东角门,从老太太起,各处各人你都瞧瞧,问候一声儿,也不必特意告诉他们说搬进园来。
若有提起因由,你只带口说我带了你进来作伴儿就完了。
\zhu{带口:犹顺口,带句话,带口信。
}回来进了园,再到各姑娘房里走走。
”\ping{此处乃是宝钗和黛玉的不同,宝钗更重俗务,在她眼里做诗算是业余消遣,没必要也犯不上专门学习,更何况宝钗做事的风格也很“端庄”,也不是香菱合适的老师。
}\par
香菱应着才要走时,只见平儿忙忙的走来。
\geng{“忙忙”二字奇,不知有何妙文。
}香菱忙问了好,平儿只得陪笑相问。
宝钗因向平儿笑道:“我今儿带了他来作伴儿,正要去回你奶奶一声儿。
”平儿笑道:“姑娘说的是那里话?我竟没话答言了。
”宝钗道:“这才是正理。
店房也有个主人,庙里也有个住持。
虽不是大事,到底告诉一声,便是园里坐更上夜的人知道添了他两个,\zhu{他两个:指香菱和服侍香菱的丫鬟文杏。
}也好关门候户的了。
\zhu{关门候户:等候关门,泛指照应门户。}
你回去告诉一声罢,我不打发人去了。
”平儿答应着,因又向香菱笑道:“你既来了,也不拜一拜街坊邻舍去?”\geng{是极,恰是戏言,实欲支出香菱去也。
}宝钗笑道:“我正叫他去呢。
”平儿道:“你且不必往我们家去,二爷病了在家里呢。
”香菱答应着去了,先从贾母处来,不在话下。
\par
且说平儿见香菱去了,便拉宝钗忙说道:“姑娘可听见我们的新闻了?”宝钗道:“我没听见新闻。
因连日打发我哥哥出门,所以你们这里的事,一概也不知道,连姊妹们这两日也没见。
”平儿笑道:“老爷把二爷打了个动不得,难道姑娘就没听见?”宝钗道:“早起恍惚听见了一句,也信不真。
我也正要瞧你奶奶去呢,不想你来了。
又是为了什么打他?”平儿咬牙骂道:“都是那贾雨村什么风村,半路途中那里来的饿不死的野杂种!认了不到十年,生了多少事出来!今年春天,老爷不知在那个地方看见了几把旧扇子,回家看家里所有收着的这些好扇子都不中用了,立刻叫人各处搜求。
谁知就有一个不知死的冤家,混号儿世人叫他作石呆子,穷的连饭也没的吃,偏他家就有二十把旧扇子,死也不肯拿出大门来。
二爷好容易烦了多少情,见了这个人,说之再三,把二爷请到他家里坐着,拿出这扇子略瞧了一瞧。
据二爷说,原是不能再有的,全是湘妃、棕竹、麋鹿、玉竹的,\zhu{湘妃、棕竹、麋鹿、玉竹:四种名贵的竹子,纹理美观,可以制做扇骨。
湘妃:指湘妃竹,又称斑竹。
产于湖南、广西,竹上有紫色斑点。
传说舜帝南巡,死于苍梧,其妃湘夫人追至,哭甚哀,以泪挥竹,故竹上斑点若泪痕。
见晋代张华《博物志》。
棕竹:元代刘美之《续竹谱》:“棕竹,蜀中多有之,皮叶皆似棕,亦谓之桃花竹。
”麋鹿:是一种表皮像麋鹿角纹的竹子。
玉竹:《群芳谱》:“玉竹,青黄相间。
”}皆是古人写画真迹,因来告诉了老爷。
老爷便叫买他的,要多少银子给他多少。
偏那石呆子说:‘我饿死冻死,一千两银子一把我也不卖!’老爷没法子,天天骂二爷没能为。
\zhu{能为:能耐。}
已经许了他五百两,先兑银子后拿扇子。
他只是不卖,只说:‘要扇子,先要我的命!’姑娘想想,这有什么法子?谁知雨村那没天理的听见了,便设了个法子,讹他拖欠了官银,拿他到衙门里去,说所欠官银,变卖家产赔补,把这扇子抄了来,作了官价送了来。
那石呆子如今不知是死是活。
老爷拿着扇子问着二爷说:‘人家怎么弄了来?’二爷只说了一句:‘为这点子小事,弄得人坑家败业,也不算什么能为!’老爷听了就生了气,说二爷拿话堵老爷,因此这是第一件大的。
这几日还有几件小的,我也记不清,所以都凑在一处,就打起来了。
也没拉倒用板子棍子,就站着,不知拿什么混打一顿,脸上打破了两处。
我们听见姨太太这里有一种丸药,上棒疮的,姑娘快寻一丸子给我。
”\zhu{第三十三回宝玉挨打之后,第三十四回宝钗亲来探视,送治淤血的丸药。
}宝钗听了,忙命莺儿去要了一丸来与平儿。
宝钗道:“既这样,替我问候罢,我就不去了。
”平儿答应着去了,不在话下。
\par
且说香菱见过众人之后,吃过晚饭,宝钗等都往贾母处去了,自己便往潇湘馆中来。
此时黛玉已好了大半,见香菱也进园来住,自是欢喜。
香菱因笑道:“我这一进来了,也得了空儿,好歹教给我作诗,就是我的造化了!”黛玉笑道:“既要作诗,你就拜我作师。
我虽不通,大略也还教得起你。
”香菱笑道:“果然这样,我就拜你作师。
你可不许腻烦的。
”\ping{黛玉和香菱有着她们都不知道的缘分呢,香菱的父亲甄士隐资助贾雨村,黛玉才有进士做老师,现在黛玉又是香菱的老师了。
}黛玉道:“什么难事,也值得去学!不过是起承转合,\zhu{起承转合:旧体诗文章法结构的术语。
起:开端。
承:承接上文进一步加以申述。
转:转折,从另一方面论述主题。
合:全文结语。
}当中承转是两副对子,平声对仄声,虚的对实的,实的对虚的,\zhu{平仄:
平声指国语的一声、二声。仄声指三声、四声和短而促的入声(今国语中入声已消失)。
格律诗每句每字的声调有规定的平仄格式,一般以平声对仄声。
虚实:律诗共八句,中间四句规定为两副对子(也称对仗),要按照词性的虚实相对,虚词对虚词,实词对实词。
这里林黛玉说:“虚的对实的,实的对虚的”,可能是作者或传抄中的笔误。
}若是果有了奇句,连平仄虚实不对都使得的。
”香菱笑道:“怪道我常弄一本旧诗偷空儿看一两首,又有对的极工的,又有不对的,又听见说‘一三五不论,二四六分明’。
\zhu{一三五不论,二四六分明:格律诗对平仄声的规定,每句的第一、三、五字要求的较宽,平仄皆可,可以不论(第五字一般也是不宜违律的);第二、四、六字则要求较严,平仄必须依律,故云。
但这只是初学诗的一种入门歌诀,其实并非完全这样,第一、三、五字能否调换平仄声,也是有许多具体条件限制的。
}看古人的诗上亦有顺的,亦有二四六上错了的,所以天天疑惑。
如今听你一说,原来这些格调规矩竟是末事,只要词句新奇为上。
”黛玉道:“正是这个道理。
词句究竟还是末事,第一立意要紧。
若意趣真了,连词句不用修饰,自是好的,这叫做‘不以词害意’。
”\zhu{不以词害意:这是说作诗要以“意”(内容)为先,文辞格律次之,不要因过分注重辞采形式而损害了内容。
}\par
香菱笑道:“我只爱陆放翁的诗‘重帘不卷留香久,
\zhu{重帘:一重一重的帘幕。}
古砚微凹聚墨多’,说的真有趣!”黛玉道:“断不可学这样的诗。
你们因不知诗,所以见了这浅近的就爱,一入了这个格局,再学不出来的。
你只听我说,你若真心要学,我这里有《王摩诘全集》,\zhu{王摩诘:唐代诗人王维,字摩诘。
唐肃宗时官尚书右丞,人称王右丞。
}你且把他的五言律读一百首,
\zhu{
律:是律诗的简称,每首八句,中间四句为“对仗”。
每句五字的叫五言律;每句七字的叫七言律。
超过八句的律诗,叫排律。
}
细心揣摩透熟了,然后再读一二百首老杜的七言律,\zhu{
老杜:指盛唐时期大诗人杜甫,为了区别于稍后的晚唐诗人杜牧,故世称杜甫为“老杜”,杜牧为“小杜”。
}次再李青莲的七言绝句读一二百首。
\zhu{李青莲:即唐代大诗人李白,幼时曾随父迁居四川绵州彰明县(今四川江油县)青莲乡,自号青莲居士。
七言:每句七个字。
绝句:每首四句的格律诗。
}肚子里先有了这三个人作了底子,然后再把陶渊明、应玚、谢、阮、庾、鲍等人的一看。
\zhu{应玚、谢、阮、庾、鲍:应玚:玚音“羊”,东汉末年诗人,“建安七子”之一。
谢:指南朝宋诗人谢灵运。
阮:指三国时魏诗人阮籍,“竹林七贤”之一。
庾:指北朝周诗人庾信。
鲍:指南朝宋诗人鲍照。
}你又是一个极聪敏伶俐的人,不用一年的工夫,不愁不是诗翁了!”香菱听了,笑道:“既这样,好姑娘,你就把这书给我拿出来,我带回去夜里念几首也是好的。
”黛玉听说,便命紫鹃将王右丞的五言律拿来,递与香菱,又道:“你只看有红圈的都是我选的,有一首念一首。
不明白的问你姑娘,或者遇见我,我讲与你就是了。
”香菱拿了诗,回至蘅芜苑中,诸事不顾,只向灯下一首一首的读起来。
宝钗连催他数次睡觉,他也不睡。
宝钗见他这般苦心,只得随他去了。
\par
一日,黛玉方梳洗完了,只见香菱笑吟吟的送了书来,又要换杜律。
\zhu{杜律:指杜甫的律诗。
}
黛玉笑道:“共记得多少首?”香菱笑道:“凡红圈选的我尽读了。
”黛玉道:“可领略了些滋味没有?”香菱笑道:“领略了些滋味,不知可是不是,说与你听听。
”黛玉笑道:“正要讲究讨论,方能长进。
你且说来我听。
”香菱笑道:“据我看来,诗的好处,有口里说不出来的意思,想去却是逼真的。
有似乎无理的,想去竟是有理有情的。
”黛玉笑道:“这话有了些意思,但不知你从何处见得?”香菱笑道:“我看他《塞上》一首,\zhu{《塞上》一首:指王维《使至塞上》一诗。
}那一联云:‘大漠孤烟直,长河落日圆。
’想来烟如何直?日自然是圆的:这‘直’字似无理,‘圆’字似太俗。
合上书一想,倒像是见了这景的。
若说再找两个字换这两个,竟再找不出两个字来。
再还有‘日落江湖白,潮来天地青’,\zhu{白:指日落时江湖上的茫茫白光。
青:指潮来时天地间的苍莽昏暗。
}这‘白’‘青’两个字也似无理。
想来,必得这两个字才形容得尽,念在嘴里倒像有几千斤重的一个橄榄。
还有‘渡头馀落日,墟里上孤烟’,\zhu{墟里:村落。
}这‘馀’字和‘上’字,难为他怎么想来!我们那年上京来,那日下晚便湾住船,\zhu{下晚:近黄昏的时候。
}岸上又没有人,只有几棵树,远远的几家人家作晚饭,那个烟竟是碧青,连云直上。
谁知我昨日晚上读了这两句,倒像我又到了那个地方去了。
”\par
正说着,宝玉和探春也来了,也都入坐听他讲诗。
宝玉笑道:“既是这样,也不用看诗。
会心处不在多,听你说了这两句,可知三昧你已得了。
”\zhu{三昧:佛教用语。
本意是心神专一,杂念止息,是佛家修持的重要方法之一。
后借指事物的奥秘和精义。
}黛玉笑道:“你说他这‘上孤烟’好,你还不知他这一句还是套了前人来的。
我给你这一句瞧瞧,更比这个淡而现成。
”说着便把陶渊明的“暧暧远人村,依依墟里烟”翻了出来,\zhu{暧暧:昏暗模糊的样子。
依依:隐约可见的样子。
}递与香菱。
香菱瞧了,点头叹赏,笑道:“原来‘上’字是从‘依依’两个字上化出来的。
”宝玉大笑道:“你已得了,不用再讲,越发倒学杂了。
你就作起来,必是好的。
”探春笑道:“明儿我补一个柬来,请你入社。
”香菱笑道:“姑娘何苦打趣我,我不过是心里羡慕,才学着顽罢了。
”探春黛玉都笑道:“谁不是顽?难道我们是认真作诗呢!若说我们认真成了诗,出了这园子,把人的牙还笑倒了呢。
”宝玉道:“这也算自暴自弃了。
前日我在外头和相公们商议画儿,他们听见咱们起诗社,求我把稿子给他们瞧瞧。
我就写了几首给他们看看,谁不真心叹服。
他们都抄了刻去了。
”探春黛玉忙问道:“这是真话么?”宝玉笑道:“说谎的是那架上的鹦哥。
”黛玉探春听说,都道:“你真真胡闹!且别说那不成诗,便是成诗,我们的笔墨也不该传到外头去。
”\zhu{笔墨:在这里代指诗文作品。
}宝玉道:“这怕什么!古来闺阁中的笔墨不要传出去,如今也没有人知道了。
”说着,只见惜春打发了入画来请宝玉,宝玉方去了。
香菱又逼着黛玉换出杜律来,又央黛玉探春二人:“出个题目,让我诌去,诌了来,替我改正。
”黛玉道:“昨夜的月最好,我正要诌一首,竟未诌成,你竟作一首来。
‘十四寒’的韵,\zhu{十四寒:诗韵中上平声第十四部以“寒”字开头的韵目,称为十四寒。
有“寒”、“韩”、“翰”、“安”、“难”等。
}由你爱用那几个字去。
”\par
香菱听了,喜的拿回诗来,又苦思一回作两句诗,又舍不得杜诗,又读两首。
如此茶饭无心,坐卧不定。
宝钗道:“何苦自寻烦恼。
都是颦儿引的你,我和他算账去。
你本来呆头呆脑的,再添上这个,越发弄成个呆子了。
”\geng{“呆头呆脑的”有趣之至!最恨野史有一百个女子皆曰“聪敏伶俐”,究竟看来,他行为也只平平。
今以“呆”字为香菱定评,何等妩媚之至也。
}
香菱笑道:“好姑娘,别混我。
”\geng{如闻如见。
}一面说,一面作了一首,先与宝钗看。
宝钗看了笑道:“这个不好,不是这个作法。
你别怕臊,只管拿了给他瞧去,看他是怎么说。
”香菱听了,便拿了诗找黛玉。
黛玉看时,只见写道是:\par
\hop
月挂中天夜色寒,清光皎皎影团团。
\par
诗人助兴常思玩,野客添愁不忍观。
\par
翡翠楼边悬玉镜,珍珠帘外挂冰盘。
\par
良宵何用烧银烛,晴彩辉煌映画栏。
\par
\hop
黛玉笑道:“意思却有,只是措词不雅。
皆因你看的诗少,被他缚住了。
把这首丢开,再作一首。
只管放开胆子去作。
”\par
香菱听了,默默的回来,越性连房也不入,只在池边树下,或坐在山石上出神,或蹲在地下抠土,来往的人都诧异。
李纨、宝钗、探春、宝玉等听得此信,都远远的站在山坡上瞧着他。
只见他皱一回眉,又自己含笑一回。
宝钗笑道:“这个人定要疯了!昨夜嘟嘟哝哝直闹到五更天才睡下,没一顿饭的工夫天就亮了。
我就听见他起来了,忙忙碌碌梳了头就找颦儿去。
一回来了,呆了一日,作了一首又不好,这会子自然另作呢。
”宝玉笑道:“这正是‘地灵人杰’,\zhu{地灵人杰:意为山川灵秀,人物杰出。
}老天生人再不虚赋情性的。
我们成日叹说可惜他这么个人竟俗了,谁知到底有今日。
可见天地至公。
”宝钗笑道:“你能够像他这苦心就好了,学什么有个不成的。
”宝玉不答。
\ping{宝钗这见缝插针的功夫,时刻不忘规劝宝玉。
}\par
只见香菱兴兴头头的又往黛玉那边去了。
探春笑道:“咱们跟了去,看他有些意思没有。
”说着,一齐都往潇湘馆来。
只见黛玉正拿着诗和他讲究。
众人因问黛玉作的如何。
黛玉道:“自然算难为他了,只是还不好。
这一首过于穿凿了,\zhu{穿凿:牵强附会,任意牵合意义,强求其通。
}还得另作。
”众人因要诗看时,只见作道:\par
\hop
非银非水映窗寒,试看晴空护玉盘。
\par
淡淡梅花香欲染,丝丝柳带露初干。
\par
只疑残粉涂金砌,恍若轻霜抹玉栏。
\par
梦醒西楼人迹绝,馀容犹可隔帘看。
\par
\hop
宝钗笑道:“不像吟月了,月字底下添一个‘色’字倒还使得,你看句句倒是月色。
这也罢了,原来诗从胡说来,再迟几天就好了。
”香菱自为这首妙绝,听如此说,自己扫了兴,不肯丢开手,便要思索起来。
因见他姊妹们说笑,便自己走至阶前竹下闲步,挖心搜胆,耳不旁听,目不别视。
一时探春隔窗笑说道:“菱姑娘,你闲闲罢。
”香菱怔怔答道:“‘闲’字是‘十五删’的,\zhu{“十五删”则是上平声第十五部以“删”字开头的韵目。
}你错了韵了。
”众人听了,不觉大笑起来。
宝钗道:“可真是诗魔了。
都是颦儿引的他!”黛玉笑道:“圣人说:‘诲人不倦。
’他又来问我,我岂有不说之理。
”李纨笑道:“咱们拉了他往四姑娘房里去,引他瞧瞧画儿,叫他醒一醒才好。
”\par
说着,真个出来拉了他过藕香榭,至暖香坞中。
惜春正乏倦,在床上歪着睡午觉,画缯立在壁间,\zhu{
缯:音“增”,古代对丝织品的统称。
画缯:绘画用的绢。
壁间:墙角。
}用纱罩着。
众人唤醒了惜春,揭纱看时,十停方有了三停。
\zhu{十停有三停:十分之三。
}香菱见画上有几个美人,因指着笑道:“这一个是我们姑娘,那一个是林姑娘。
”探春笑道:“凡会作诗的都画在上头,快学罢。
”说着,顽笑了一回。
\par
各自散后,香菱满心中还是想诗。
至晚间对灯出了一回神,至三更以后上床卧下,两眼鳏鳏,\zhu{鳏:一种大鱼,其性独行。
鱼目常睁不闭,故常用“鳏鳏”形容忧愁失眠的样子。
}直到五更方才朦胧睡去了。
一时天亮,宝钗醒了,听了一听,他安稳睡了,心下想:“他翻腾了一夜,不知可作成了?这会子乏了,且别叫他。
”正想着,只听香菱从梦中笑道:“可是有了,难道这一首还不好?”宝钗听了,又是可叹,又是可笑,连忙唤醒了他,问他:“得了什么?你这诚心都通了仙了。
学不成诗,还弄出病来呢。
”一面说,一面梳洗了,会同姊妹往贾母处来。
原来香菱苦志学诗,精血诚聚,日间做不出,忽于梦中得了八句。
梳洗已毕,便忙录出来,自己并不知好歹,便拿来又找黛玉。
刚到沁芳亭,只见李纨与众姊妹方从王夫人处回来,宝钗正告诉他们说他梦中作诗说梦话。
\geng{一部大书起是梦,宝玉情是梦,贾瑞淫又是梦,秦[氏]之家计长策又是梦,今作诗也是梦,一并“风月鉴”亦从梦中所有,\zhu{“风月鉴”:第十二回贾瑞所照的“风月宝鉴”。
}故[曰]“红楼梦”也。
余今批评亦在梦中,特为梦中之人特作此一大梦也。
脂砚斋。
}众人正笑,抬头见他来了,便都争着要诗看。
且听下回分解。
\par
\qi{总评:一扇之微,而害人如此其毒。
藏之者故是无味,构求者更觉可笑。
多少没天理处,全不自觉。
可见好爱之端,断不可生。
求古董于古坟,争盆景而荡产,势所必至,可不慎诸。
\zhu{诸:代词“之”和语气词“乎”的合音,意义等于“之乎”。
}}
\dai{095}{黛玉教香菱学作诗}
\dai{096}{暖香坞看画上美人,香菱指认宝钗黛玉}
\sun{p48-1}{香菱学诗成诗魔,偶得佳句争传看}{图右侧:宝钗让香菱搬来大观园同住。
香菱便来潇湘馆求黛玉教她作诗,一日,二人正在谈诗,宝玉、探春也来听她讲诗,宝玉道:“听你说了这两句,可知三昧你已得了”,探春听笑道:“明儿我补一个柬来,请你入社。
”图上侧:自此,香菱越发痴迷于作诗。
香菱梦中偶得佳句,便忙录出来,自己并不知好歹,便拿来又找黛玉。
刚到沁芳亭,只见众姊妹方从王夫人处回来,宝钗正告诉他们说香菱梦中作诗说梦话。
众人正笑,抬头见他来了,便都争着要诗看。
}
