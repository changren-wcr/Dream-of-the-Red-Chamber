\chapter{庆寿辰宁府排家宴\quad 见熙凤贾瑞起淫心}
\qi{幻景无端换境生,玉楼春暖述乖情。
\zhu{乖:背离,不正常。乖情:明指贾瑞对嫂子凤姐的觊觎垂涎,暗指秦可卿和公公贾珍的不伦之情。}
闹中寻静浑闲事,运得灵机属凤卿。
}\par
话说是日贾敬的寿辰,贾珍先将上等可吃的东西,稀奇些的果品,装了十六大捧盒,着贾蓉带领家下人等与贾敬送去,向贾蓉说道:“你留神看太爷喜欢不喜欢,你就行了礼来。
你说:‘我父亲遵太爷的话未敢来,在家里率领合家都朝上行了礼了。
’”贾蓉听罢,即率领家人去了。
\par
这里渐渐的就有人来了。
先是贾琏贾蔷到来,先看了各处的座位,并问:“有什么顽意儿没有?”家人答道:“我们爷原算计请太爷今日来家来,所以未敢预备顽意儿。
前日听见太爷又不来了,现叫奴才们找了一班小戏儿并一档子打十番的,\zhu{一档(档音“荡”)子打十番的:一班演奏十番的艺人。
一档子:一班、一拨儿。
十番:又称十番锣鼓,一种用乐器合奏的套曲。
李斗《扬州画舫录》:“是乐不用小锣、金锣、铙钹、号筒,只用笛、管、箫、弦、提琴、云锣、汤锣、木鱼、檀板、大鼓十种,故名十番鼓。
番者更番之谓。
……若夹用锣铙之属,则为粗细十番。
”
钹:音“伯”,铜制打击乐器,两个圆片为一副,中间凸起成半球状,正中有孔,可以穿绸布条供手持,两片相击发声。
铙:音“挠”,打击乐器。与钹形制相似,只是中间隆起部分较小,发音较低,余音较长。
}都在园子里戏台上预备着呢。
”\par
次后邢夫人、王夫人、凤姐儿、宝玉都来了,贾珍并尤氏接了进去。
尤氏的母亲已先在这里呢。
大家见过了,彼此让了坐。
贾珍尤氏二人亲自递了茶,因说道:“老太太原是老祖宗,我父亲又是侄儿,这样日子,原不敢请他老人家,但是这个时候,天气正凉爽,满园的菊花又盛开,请老祖宗过来散散闷,看着众儿孙热闹热闹,是这个意思。
谁知老祖宗又不肯赏脸。
”凤姐儿未等王夫人开口,先说道:“老太太昨日还说要来着呢,因为晚上看着宝兄弟他们吃桃儿,老人家又嘴馋,吃了有大半个,五更天的时候就一连起来了两次,\meng{此一问一答,即景生情,请教是真是假?非身经其事者,想不到,写不出。
}今日早晨略觉身子倦些。
因叫我回大爷,今日断不能来了,说有好吃的要几样,还要很烂的。
”\meng{是。
}
贾珍听了笑道:“我说老祖宗是爱热闹的,今日不来,必定有个原故,若是这么着就是了。
”\par
王夫人道:“前日听见你大妹妹说,蓉哥儿媳妇儿身上有些不大好,到底是怎么样?”尤氏道:“他这个病得的也奇。
上月中秋还跟着老太太、太太们顽了半夜,回家来好好的。
到了二十后,一日比一日觉懒,也懒待吃东西,这将近有半个多月了。
经期又有两个月没来。
”邢夫人接着说道:“别是喜罢?”\meng{此书总是一幅《云龙图》。
\zhu{云龙图:指《红楼梦》在情节安排上出人意表,神妙变化的艺术手法。}
}\par
正说着,外头人回道:“大老爷,二老爷并一家子的爷们都来了,在厅上呢。
”贾珍连忙出去了。
这里尤氏方说道:“从前大夫也有说是喜的。
昨日冯紫英荐了他从学过的一个先生,医道很好,瞧了说不是喜,竟是很大的一个症候。
昨日开了方子,吃了一剂药,今日头眩的略好些,别的仍不见怎么样大见效。
”凤姐儿道:“我说他不是十分支持不住,今日这样的日子,再也不肯不扎挣着上来。
”\zhu{扎挣:勉强支持。
}\ping{补出秦可卿平日多么谨慎,不惜委屈自己。
}尤氏道:“你是初三日在这里见他的,他强扎挣了半天,也是因你们娘儿两个好的上头,他才恋恋的舍不得去。
”凤姐儿听了,眼圈儿红了半天,半日方说道:“真是‘天有不测风云,人有旦夕祸福’。
\meng{揣摩的极平常言语来写无涯之幻景幻情,反作了悟之意,且又转至别处,真是月下梨花,几不能辨。
}这个年纪,倘或就因这个病上怎么样了,人还活着有甚么趣儿!”\meng{大英雄多在此等处悟得,每能超凡入圣。
}正说话间,贾蓉进来,给邢夫人、王夫人、凤姐儿前都请了安,方回尤氏道:“方才我去给太爷送吃食去,并回说我父亲在家中伺候老爷们,款待一家子的爷们,遵太爷的话未敢来。
太爷听了甚喜欢,说:‘这才是。
’叫告诉父亲母亲好生伺候太爷太太们,叫我好生伺候叔叔婶子们并哥哥们。
还说那《阴骘文》,叫急急的刻出来,印一万张散人。
我将此话都回了我父亲了。
我这会子得快出去打发太爷们并合家爷们吃饭。
”凤姐儿说:“蓉哥儿,你且站住。
你媳妇今日到底是怎么着?”贾蓉皱皱眉说道:“不好么!婶子回来瞧瞧去就知道了。
”\meng{伏线自然。
}于是贾蓉出去了。
\par
这里尤氏向邢夫人、王夫人道:“太太们在这里吃饭阿,还是在园子里吃去好?小戏儿现预备在园子里呢。
”王夫人向邢夫人道:“我们索性吃了饭再过去罢,也省好些事。
”邢夫人道:“很好。
”于是尤氏就吩咐媳妇婆子们:“快送饭来。
”门外一齐答应了一声,都各人端各人的去了。
不多一时,摆上了饭。
尤氏让邢夫人、王夫人并他母亲都上了坐,他与凤姐儿、宝玉侧席坐了。
邢夫人、王夫人道:“我们来原为给大老爷拜寿,这不竟是我们来过生日来了么?”凤姐儿说道:“大老爷原是好养静的,已经修炼成了,也算得是神仙了。
太太们这么一说,这就叫作‘心到神知’了。
”\meng{此等趣语,亦不肯无着落。
}一句话说的满屋里的人都笑起来了。
\par
于是,尤氏的母亲并邢夫人、王夫人、凤姐儿都吃毕饭,漱了口,净了手,才说要往园子里去,贾蓉进来向尤氏说道:“老爷们并众位叔叔、哥哥、兄弟们也都吃了饭了。
大老爷说家里有事,二老爷是不爱听戏又怕人闹的慌,都才去了。
别的一家子爷们都被琏二叔并蔷兄弟让过去听戏去了。
方才南安郡王、东平郡王、西宁郡王、北静郡王四家王爷,并镇国公牛府等六家,忠靖侯史府等八家,都差人持了名帖送寿礼来,俱回了我父亲,先收在帐房里了,礼单都上上档子了。
\zhu{上档子:档子:分门别类登记的簿册。
据《清稗类钞》,清初尚无此类册籍,有事记在木片上,年久积多,用皮条穿挂,叫“档子”或“牌子”;后来写在簿册上,也相沿叫“档子”。
上档子:就是记在簿册上。
}老爷的领谢的名帖都交给各来人了,各来人也都照旧例赏了,众来人都让吃了饭才去了。
母亲该请二位太太、老娘、婶子都过园子里坐着去罢。
”\meng{人送寿礼,是为园子;回人去的去了在的在,是为可以过园子里坐;园子里坐可以转入正文中之幻情;幻情里有乖情,而乖情初写,偏不乖。
真是慧心神手!}尤氏道:“也是才吃完了饭,就要过去了。
”\par
凤姐儿说:“我回太太,我先瞧瞧蓉哥儿媳妇,我再过去。
”王夫人道:“很是,我们都要去瞧瞧他,倒怕他嫌闹的慌,\meng{为下文留地步。
}
说我们问他好罢。
”尤氏道:“好妹妹,媳妇听你的话,你去开导开导他,我也放心。
你就快些过园子里来。
”宝玉也要跟了凤姐儿去瞧秦氏去,王夫人道:“你看看就过去罢,那是侄儿媳妇。
”于是尤氏请了邢夫人、王夫人并他母亲都过会芳园去了。
\par
凤姐儿、宝玉方和贾蓉到秦氏这边来。
进了房门,悄悄的走到里间房门口,秦氏见了,就要站起来,凤姐儿说:“快别起来,看起猛了头晕。
”\meng{知心每每如此。
}于是凤姐儿就紧走了两步,拉住秦氏的手,说道:“我的奶奶!怎么几日不见,就瘦的这么着了!”于是就坐在秦氏坐的褥子上。
宝玉也问了好,坐在对面椅子上。
贾蓉叫:“快倒茶来,婶子和二叔在上房还未喝茶呢。
”\par
秦氏拉着凤姐儿的手,强笑道:“这都是我没福。
这样人家,公公婆婆当自己的女孩儿似的待。
\meng{正写幻情,偏作锥心刺骨语。
\zhu{
公公婆婆把儿媳秦可卿当自己的女孩,表面上说明关系融洽和谐,
但是可能暗示公公和儿媳超越伦理的亲密关系。所以有“锥心刺骨”的评语。
}
呼渡河者三,是一意。
\zhu{
呼渡河者三:宗泽在任东京留守期间,曾二十多次上书高宗赵构,力主还都东京,
并制定了收复中原的方略,均未被采纳。他因壮志难酬,忧愤成疾,临终三呼“过河”而卒。
此条评语将秦氏将死之言,比作宗泽遗言,大有“出师未捷身先死”之叹。
}
}婶娘的侄儿虽说年轻,却也是他敬我,我敬他,从来没有红过脸儿。
就是一家子的长辈同辈之中,除了婶子倒不用说了,别人也从无不疼我的,也无不和我好的。
这如今得了这个病,把我那要强的心一分也没了。
\zhu{要强:争强好胜,不肯认输。
}公婆跟前未得孝顺一天,就是婶娘这样疼我,我就有十分孝顺的心,如今也不能够了。
我自想着,未必熬的过年去呢。
”\par
宝玉正眼瞅着那《海棠春睡图》并那秦太虚写的“嫩寒锁梦因春冷,芳气笼人是酒香”的对联,不觉想起在这里睡晌觉梦到“太虚幻境”的事来。
正自出神,听得秦氏说了这些话,如万箭攒心,那眼泪不知不觉就流下来了。
\ping{会不会是通灵宝玉做出预言,宝玉有感,以泪水作为秦可卿死亡的谶语。
}凤姐儿心中虽十分难过,但恐怕病人见了众人这个样儿反添心酸,倒不是来开导劝解的意思了。
见宝玉这个样子,因说道:“宝兄弟,你忒婆婆妈妈的了。
他病人不过是这么说,那里就到得这个田地了?况且能多大年纪的人,略病一病儿就这么想那么想的,这不是自己倒给自己添病了么?”贾蓉道:“他这病也不用别的,只是吃得些饮食就不怕了。
”\meng{各人是各人伎俩,一丝不乱,一毫不遗。
}凤姐儿道:“宝兄弟,太太叫你快过去呢。
你别在这里只管这么着,倒招的媳妇也心里不好。
太太那里又惦着你。
”因向贾蓉说道:“你先同你宝叔叔过去罢,\meng{为本。
\zhu{这条评语令人费解。}
}我还略坐一坐儿。
”贾蓉听说,即同宝玉过会芳园来了。
\par
这里凤姐儿又劝解了秦氏一番,又低低的说了许多衷肠话儿,尤氏打发人请了两三遍,凤姐儿才向秦氏说道:“你好生养着罢,我再来看你。
合该你这病要好,所以前日就有人荐了这个好大夫来,再也是不怕的了。
”秦氏笑道:“任凭神仙也罢,治得病治不得命。
婶子,我知道我这病不过是挨日子。
”凤姐儿说道:“你只管这么想着,病那里能好呢?总要想开了才是。
况且听得大夫说,‘若是不治,怕的是春天不好’,如今才九月半,还有四五个月的工夫,什么病治不好呢?咱们若是不能吃人参的人家,这也难说了,你公公婆婆听见治得好你,别说一日二钱人参,就是二斤也能够吃的起。
好生养着罢,我过园子里去了。
”秦氏又道:“婶子,恕我不能跟过去了。
闲了时候还求婶子常过来瞧瞧我,咱们娘儿们坐坐,多说几遭话儿。
”凤姐儿听了,不觉得又眼圈儿一红,遂说道:“我得了闲儿必常来看你。
”于是凤姐儿带领跟来的婆子丫头并宁府的媳妇婆子们,从里头绕进园子的便门来。
\meng{偏不独行,用此等反克文字。
\zhu{
生克乘侮:中医学基础理论之一。五行间相互关系的四种类型。中国古代医家以五行(木、火、土、金、水)代表五脏,
以其相生、相克、相乘、相侮关系来解释人体内脏间的一些生理现象和病理变化。
以相生(木生火、火生土、土生金、金生水、水生木)解释五脏之间的相互资生作用;
以相克(木克土、土克水、水克火、火克金、金克木)解释五脏之间的相互制约作用;
相乘,指相克太过,如木乘土,即木克土太过;
相侮,指反侮,亦称“反克”。
在五行相克序列中,当前项很弱而后项很强时,前项不但不能克制后项,反而会被后项所克制,叫做反克。
反克文字:这里应该是指文字反常。
}
}但只见:\par
\hop
黄花满地,白柳横坡。
小桥通若耶之溪,
\zhu{
若耶溪:在浙江省绍兴县南。
传说春秋时越国的美女西施曾在这里浣纱。
}
曲径接天台之路。
\zhu{
天台路:传说汉代刘晨、阮肇入天台山采药,遇到两个仙女留住半年。
若耶溪和天台路借典形容园中溪水、路径的幽美别致,不同一般。
}
\meng{点明题目。
}石中清流激湍,篱落飘香;树头红叶翩翻,疏林如画。
西风乍紧,初罢莺啼;暖日当暄,\zhu{暖日当暄:温和的日光晒得正暖。
当:正当。
暄:音“宣”,暖和。
}又添蛩语。
\zhu{蛩语:蟋蟀的鸣声。
蛩:音“穷”,蟋蟀。
}遥望东南,建几处依山之榭;\zhu{榭:音“谢”,高台上建筑的房屋。
}纵观西北,结三间临水之轩。
\zhu{轩:敞亮别致的小屋或小室。
}笙簧盈耳,\zhu{
笙:音“生”,乐器,用若干根装有簧的竹管和一根吹气管装在一个锅形的座子上制成。
簧:音“黄”,乐器里用铜或其他质料制成的发声薄片。
这里用“笙簧”代指笙簧之声以形容流水声的悠扬悦耳。
杜甫《题终明府水楼》诗:“绝壁过云开锦绣,疏松夹水奏笙簧。
”}别有幽情;罗绮穿林,\zhu{罗绮(绮音“起”):皆丝织品。
这里代指服饰华丽的人们。
}倍添韵致。
\par
\hop
凤姐儿正自看园中景致,一步步行来赞赏。
猛然从假山石后走过一个人来,向前对凤姐儿说道:“请嫂子安。
”凤姐儿猛然见了,将身子望后一退,说道:“这是瑞大爷不是?”贾瑞说道:“嫂子连我也不认得了?不是我是谁!”凤姐儿道:“不是不认得,猛然一见,不想到是大爷到这里来。
”\meng{作者何等心思,能在此等事想到如此出言。
渐入之妙,无过于此。
}
贾瑞道:“也是合该我与嫂子有缘。
我方才偷出了席,在这个清净地方略散一散,不想就遇见嫂子也从这里来。
这不是有缘么?”\meng{重点“有缘”二字,方是笔力。
}一面说着,一面拿眼睛不住的觑着凤姐儿。
\zhu{觑:音“去”,眯着眼注视。}
\par
凤姐儿是个聪明人,见他这个光景,如何不猜透八九分呢,因向贾瑞假意含笑道:“怨不得你哥哥时常提你,说你很好。
今日见了,听你说这几句话儿,就知道你是个聪明和气的人了。
这会子我要到太太们那里去,不得和你说话儿,等闲了咱们再说话儿罢。
”贾瑞道:“我要到嫂子家里去请安,又恐怕嫂子年轻,不肯轻易见人。
”凤姐儿假意笑道:“一家子骨肉,说什么年轻不年轻的话。
”贾瑞听了这话,再不想到今日得这个奇遇,那神情光景亦发不堪难看了。
凤姐儿说道:“你快入席去罢,仔细他们拿住罚你酒。
”贾瑞听了,身上已木了半边,慢慢的一面走着,一面回过头来看。
凤姐儿故意的把脚步放迟了些儿,见他去远了,心里暗忖道:“这才是知人知面不知心呢,那里有这样禽兽的人呢!\meng{大英雄气概。
作者以此命凤,其有为耶?}他如果如此,几时叫他死在我的手里,他才知道我的手段!”\ping{秦可卿和贾瑞的死亡在后文基本上是同步展开的,两人都是因为不伦之情而死:贾瑞是因为和嫂子的不伦之情;秦可卿是因为和贾珍的不伦之情。
这两件事很可能有联系。
王熙凤在探望秦可卿之后才遇到调戏自己的贾瑞,联系到秦可卿最后的结局是“淫丧天香楼”,秦可卿很可能和王熙凤谈及了自己和公公贾珍的羞耻之事,使得王熙凤对于好色无耻的男人的仇恨增加。
凤姐的这股怒气无法向贾府现任族长贾珍发泄,而罪不至死的贾瑞正好撞到枪口上,成为了发泄怒火的替罪羊,所以凤姐非要把他置之于死地方解恨。
}\par
于是凤姐儿方移步前来。
将转过了一重山坡,见两三个婆子慌慌张张的走来,见了凤姐儿,笑说道:“我们奶奶见二奶奶只是不来,急的了不得,叫奴才们又来请奶奶来了。
”\meng{别者必将遇贾瑞的事声张一番,以表清节。
此文偏若无事,一则可以见熙凤非凡,一则可以见熙凤包含广大。
}凤姐儿说道:“你们奶奶就是这么急脚鬼似的。
”凤姐儿慢慢的走着,问:“戏唱了几出了?”那婆子回道:“有八九出了。
”说话之间,已来到了天香楼的后门,见宝玉和一群丫头们在那里玩呢。
凤姐儿说道:“宝兄弟,别忒淘气了。
”\meng{照应前文。
\zhu{
贾宝玉和一群丫头在玩,照应了第二回:
贾宝玉:“女儿是水作的骨肉,男人是泥作的骨肉。
我见了女儿,我便清爽;见了男人,便觉浊臭逼人”
}
}有一个丫头说道:“太太们都在楼上坐着呢,请奶奶就从这边上去罢。
”\par
凤姐儿听了,款步提衣上了楼,见尤氏已在楼梯口等着呢。
尤氏笑说道:“你们娘儿两个忒好了,见了面总舍不得来了。
你明日搬来和他住着罢。
你坐下,我先敬你一钟。
”于是凤姐儿在邢、王二夫人前告了坐,又在尤氏的母亲前周旋了一遍,仍同尤氏坐在一桌上吃酒听戏。
尤氏叫拿戏单来,让凤姐儿点戏,凤姐儿说道:“亲家太太和太太们在这里,我如何敢点。
”邢夫人、王夫人说道:“我们和亲家太太都点了好几出了,你点两出好的我们听。
”凤姐儿立起身来答应了一声,方接过戏单,从头一看,点了一出《还魂》,\zhu{《还魂》:明代汤显祖著《牡丹亭》的第三十五出。
《牡丹亭》写柳梦梅和杜丽娘的爱情故事。
《还魂》一出写杜丽娘死而复生和柳梦梅结为夫妇。
}一出《弹词》,\zhu{《弹词》:清初洪升著《长生殿》的第三十八出。
《长生殿》写唐玄宗、杨贵妃故事。
《弹词》一出写唐玄宗的乐工李龟年,经“安史之乱”,流落江南,以弹琵琶卖唱为生。
唱的是唐玄宗和杨贵妃的悲欢离合及唐王朝的盛衰陈迹。
}递过戏单去说:“现在唱的这《双官诰》,\meng{点下文。
}\zhu{
诰:即诰命。
封建朝廷以皇帝的名义颁赐品爵的诏令。
《双官诰》:清代陈二白著《双官诰》传奇。
写冯琳如的婢妾碧莲守节教子,后来得了夫子双份官诰的故事。
地方戏中的《三娘教子》即由此而来。
}唱完了,再唱这两出,也就是时候了。
”王夫人道:“可不是呢,也该趁早叫你哥哥嫂子歇歇,他们又心里不静。
”尤氏说道:“太太们又不常过来,娘儿们多坐一会子去,才有趣儿,天还早呢。
”凤姐儿立起身来望楼下一看,说:“爷们都往那里去了?”旁边一个婆子道:“爷们才到凝曦轩,带了打十番的那里吃酒去了。
”凤姐儿说道:“在这里不便宜,背地里又不知干什么去了!”\meng{偏是爱吃酸醋。
}尤氏笑道:“那里都像你这么正经人呢。
”\par
于是说说笑笑,点的戏都唱完了,方才撤下酒席,摆上饭来。
吃毕,大家才出园子来,到上房坐下,吃了茶,方才叫预备车,向尤氏的母亲告了辞。
尤氏率同众姬妾并家下婆子媳妇们方送出来,贾珍率领众子侄都在车旁侍立,等候着呢,见了邢夫人、王夫人道:“二位婶子明日还过来逛逛。
”王夫人道:“罢了,我们今日整坐了一日,也乏了,明日歇歇罢。
”于是都上车去了。
贾瑞犹不时拿眼睛觑着凤姐儿。
\meng{无有不足不尽处。
}贾珍等进去后,李贵才拉过马来,宝玉骑上,随了王夫人去了。
这里贾珍同一家子的弟兄子侄吃过了晚饭,方大家散了。
\par
次日,仍是众族人等闹了一日,不必细说。
此后,凤姐儿不时亲自来看秦氏。
秦氏也有几日好些,也有几日仍是那样。
贾珍、尤氏、贾蓉好不焦心。
\meng{陪衬补足。
}\par
且说贾瑞到荣府来了几次,偏都遇见凤姐儿往宁府那边去了。
这年正是十一月三十日冬至。
到交节的那几日,
\zhu{交节:换节气的时候。冷暖节气的更替,对体弱多病者不利。}
贾母、王夫人、凤姐儿日日差人去看秦氏,回来的人都说:“这几日也没见添病,也不见甚好。
”王夫人向贾母说:“这个症候,遇着这样大节不添病,就有好大的指望了。
”贾母说:“可是呢,好个孩子,要是有些原故,可不叫人疼死。
”说着,一阵心酸,叫凤姐儿说道:“你们娘儿两个也好了一场,明日大初一,过了明日,你后日再去看一看他去。
你细细的瞧瞧他那光景,倘或好些儿,你回来告诉我,我也喜欢喜欢。
那孩子素日爱吃的,你也常叫人做些给他送过去。
”凤姐儿一一的答应了。
\par
到了初二日,吃了早饭,来到宁府,看见秦氏的光景,虽未甚添病,但是那脸上身上的肉全瘦干了。
于是和秦氏坐了半日,说了些闲话儿,又将这病无妨的话开导了一遍。
秦氏说道:“好不好,春天就知道了。
如今现过了冬至,又没怎么样,或者好的了也未可知。
婶子回老太太、太太放心罢。
\meng{文字一变。
人于将死时也应有一变。
}昨日老太太赏的那枣泥馅的山药糕,我倒吃了两块,倒像克化的动似的。
”\zhu{克化:消化。
}凤姐儿说道:“明日再给你送来。
我到你婆婆那里瞧瞧,就要赶着回去回老太太的话去。
”秦氏道:“婶子替我请老太太、太太安罢。
”\par
凤姐儿答应着就出来了,到了尤氏上房坐下。
尤氏道:“你冷眼瞧媳妇是怎么样?”凤姐儿低了半日头,\ping{如何见过这样的凤姐?}说道:“这实在没法儿了。
你也该将一应的后事用的东西给他料理料理,冲一冲也好。
”\zhu{冲:一种迷信习俗,有冲散噩运之意。
如为重病人预先准备丧事或提前举行婚礼等等,认为可以冲掉病灾。
}\meng{伏下文代办理丧事。
}尤氏道:“我也叫人暗暗的预备了。
就是那件东西不得好木头,暂且慢慢的办罢。
”于是凤姐儿吃了茶,说了一会子话儿,说道:“我要快回去回老太太的话去呢。
”尤氏道:“你可缓缓的说,别吓着老太太。
”凤姐儿道:“我知道。
”于是凤姐儿就回来了。
到了家中,见了贾母,说:“蓉哥儿媳妇请老太太安,给老太太磕头,说他好些了,求老祖宗放心罢。
他再略好些,还要给老祖宗磕头请安来呢。
”贾母道:“你看他是怎么样?”凤姐儿说:“暂且无妨,精神还好呢。
”\meng{“精神还好呢”五字,写得出神入化。
}贾母听了,沉吟了半日,因向凤姐儿说:“你换换衣服歇歇去罢。
”\par
凤姐儿答应着出来,见过了王夫人,到了家中,平儿将烘的家常的衣服给凤姐儿换了。
凤姐儿方坐下,问道:“家里没有什么事么?”平儿方端了茶来,递了过去,说道:“没有什么事。
就是那三百银子的利银,\ping{凤姐挪用公款放贷收取利息。
第三十九回,袭人向平儿询问为何拖欠月钱迟迟不发,平儿解释道:“这个月的月钱,我们奶奶早已支了,放给人使呢。
等别处的利钱收了来,凑齐了才放呢。
”}旺儿媳妇送进来,我收了。
\meng{陪。
}再有瑞大爷使人来打听\meng{正。
}
奶奶在家没有,他要来请安说话。
”凤姐儿听了,哼了一声,说道:“这畜生合该作死,看他来了怎么样!”平儿因问道:“这瑞大爷是因什么只管来?”凤姐儿遂将九月里宁府园子里遇见他的光景,他说的话,都告诉了平儿。
平儿说道:“癞蛤蟆想天鹅肉吃,没人伦的混帐东西,起这个念头,叫他不得好死!”凤姐儿道:“等他来了,我自有道理。
”不知贾瑞来时作何光景,且听下回分解。
\ping{这一回若说灵魂人物,必定是凤姐,书写到这里,凤姐是个粗线条的人物,虽然性格豪爽通透,家里处处拿得起放得下,但是只是个模糊的灿烂人物,不曾涉及情感描写。
这回开头凤姐是周全的善解人意的解语花,中间是为闺蜜命运伤感的姐妹,会触景伤情,性格中恶毒的一面也开始显现,看这回的感觉就好像一个一镜到底的片子,永远聚焦在凤姐身上,用她的行动串联起这一回的故事。
}\par
\qi{总评:将可卿之病将死,作幻情一劫;又将贾瑞之遇唐突,作幻情一变。
下回同归幻境,真风马牛不相及之谈。
同范并趋,毫无滞碍,灵活之至,飘飘欲仙。
默思作者其人之心,其人之形,其人之神,其人之文,必宋玉、子建一般心性,一流人物。
\zhu{
宋玉:战国楚辞赋家。
子建:三国魏诗人,字子建。
}
}
\dai{021}{凤姐探视可卿}
\dai{022}{见熙凤贾瑞起淫心}
\sun{p11-1}{王熙凤探视秦可卿,见熙凤贾瑞起淫心}{图右上:宁荣二府合家为在外修炼的老太爷贾敬祝寿欢宴,凤姐宝玉惦记秦氏,便来秦氏房中看望。
宝玉听秦氏说自己恐熬不过年去,不觉流下泪来。
凤姐和秦氏说了会儿安慰话,遂向会芳园来。
图下侧:正一步步行着,猛然从假山后走出一个人来,向凤姐请安。
凤姐吃了一惊,待回过神来才看出是贾瑞。
凤姐是聪明人,见其举止已经猜出八九分,故意说些亲热话,心里却说:“这种禽兽一般的人,死在我手里,方知我的手段。
”}