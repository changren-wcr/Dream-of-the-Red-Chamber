\chapter{薛小妹新编怀古诗 \quad 胡庸医乱用虎狼药}
\zhu{怀古诗:感怀古人古事之作。
这十首诗虽是用作谜语,但小说作者抑或另有寓意。
}
\par
\qi{文有一语写出大景者,如“园中不见一女子”句,俨然大家规模。
“疑是姑娘”一语,又俨然庸医口角,新医行径。
笔大如椽。
\zhu{笔大如椽:比喻大作家、大书法家的大手笔。
《晋书·王珣传》:珣梦人以大笔如掾与之,既觉,语人云:“此当有大手笔事。
”俄而,帝崩,哀册谥议,皆珣所草。
}}\par
众人闻得宝琴将素习所经过各省内的古迹为题,作了十首怀古绝句,内隐十物,皆说这自然新巧。
都争着看时,只见写道是:\par
\hop
赤壁怀古\quad 其一\par
赤壁沉埋水不流,徒留名姓载空舟。
\zhu{赤壁:东汉建安十三年,孙权、刘备联军大败曹军于此。
沉埋水不流:意思是火烧曹操战船后,馀骸沉埋江中,江水为之不流。
名姓载空舟:战舰上插帜,上书将帅姓氏,兵败后,空见船上旗号而已。
}\par
喧阗一炬悲风冷,无限英魂在内游。
\zhu{喧阗:音“宣田”,声音喧哗、噪杂。
一炬:一把火,指三江口周瑜纵火。
}\par
\ping{
本回后文写道“大家猜了一回,皆不是”,众姊妹猜不到的,并非是走马灯之类的东西,而是她们所决不可能猜到的“谜外之谜”,即众姊妹命运的谶语。不交代谜底,是因为猜对猜错,对小说来说都是毫无意义的。
《赤壁怀古》是总说,用阴森凄惨的战场描画,写这个封建大家庭在衰败过程中,死亡累累,恰如赤壁鏖战中曹家人马之“一败涂地”。
曹操与作者同姓,这是巧合,但小说中有作者的家世感慨在,这也是不言而喻的。
“无限英魂在内游”,既是下面各首内容的提示,也表示死亡者实际上还不限于写到的这九个人。
}
\par
\hop
交趾怀古\quad 其二\par
\zhu{交趾:古郡名,辖境相当今越南北部。
}\par
铜铸金镛振纪纲,声传海外播戎羌。
\zhu{金镛:镛音“庸”,铜铸的大钟。
纪纲:指国家的法纪和政令。
戎羌:这里代指我国各少数民族地区。
这两句的意思是用宫中钟声传播四方,形容国威远扬。
}\par
马援自是功劳大,铁笛无烦说子房。
\zhu{马援:东汉人,光武帝刘秀的大将,封伏波将军、新息侯,曾带兵西击羌族,南征交趾,北逐匈奴,晚年在进兵武陵五溪时染疫身亡。
“铁笛”句:晋代崔豹《古今注》:“《武溪深》,乃马援南征之所作也。援门生爰寄生善吹笛,援作歌以和之。”
子房:西汉人张良的字,他曾辅刘邦建立汉朝。
传说当汉军围项羽于垓下时张良命军士用笛吹奏楚歌,瓦解楚军军心,此实出好事者附会。
这句紧连上句,意思是若论劳苦功高,当数马援,有笛曲可征其事迹,用不着去说汉初的张良。
}\par
\ping{
《交趾怀古》是说贾元春的。用“金镛”是为了隐指宫闱。与元春“册子”中所说的“榴花开处照宫闱”用意相同。
“声传海外”句与她所作灯谜中说爆竹如雷,震得人恐妖魔惧一样,都喻进封贵妃时的显赫声势。
马援正受皇帝的恩遇而忽然病死于远征途中,这也可以说是“喜荣华正好,恨无常又到”、“望家乡,路远山高”
但由于元春之死详情莫知,诗末句的隐义,也就难以索解了。
}
\par
\hop
钟山怀古\quad 其三\par
\zhu{
钟山:即紫金山,又称北山,在今南京市中山门外。
南齐孔稚珪在《北山移文》中说:有个“周子”,为了欺世盗名,曾来钟山隐居。
但等皇帝诏书一到,他便“形驰魄散,志变神动”,趋炎附势,出山作了海盐令,因而受到山灵的嘲笑。
}
\par
名利何曾伴汝身,无端被诏出凡尘。
\zhu{
汝:指周子。
出凡尘:从隐居之地出来到凡尘中,即“入凡尘”。
}\par
牵连大抵难休绝,莫怨他人嘲笑频。
\par
\ping{
《钟山怀古》是说李纨的。她青春丧偶,“如槁木死灰一般,一概无见无闻”。
所以说她不曾为“名利”所系。她后来“被诏出凡尘”“戴珠冠,披凤袄”,这完全是因为她儿子贾兰“爵禄高登”的缘故,并非她自己不愿当“稻香老农”。所以说“牵连大抵难休绝”。
至于被他人嘲笑,在她的“册子”中也早有判词,所谓“枉与他人作笑谈”是也。
}
\par
\hop
淮阴怀古\quad 其四\par
\zhu{淮阴:古县名,秦代所置,即今江苏省清江市。
西汉名将韩信出生于此。
}\par
壮士须防恶犬欺,三齐位定盖棺时。
\zhu{上句指韩信青年时曾受辱于淮阴恶少,从其胯下爬过的事。
下句意谓当韩信受封为齐王时,已经决定了他最后被杀的命运。
三齐:项羽灭秦后,将齐地分封给胶东、齐、济北三王,故“齐”又称“三齐”。
当刘邦同项羽的斗争相持不下之时,韩信举足轻重,刘邦为笼络他,乘其破赵平齐后要求封王之机,立他为齐王,但这只是迫不得已的权宜之计。
齐人蒯通劝韩信不如割据一方,谁也不依靠,“三分天下,鼎足而居”。否则“勇略震主者身危”,将来必自取其祸。
韩信因受刘邦之封,不愿背汉。后来,他伏罪被处死前说:“吾悔不听蒯通之计。”
盖棺:即“盖棺论定”意谓人死后方能做出结论。
这里指韩信最后被杀之事。
}\par
寄言世俗休轻鄙,一饭之恩死也知。
\zhu{下句意谓韩信贫贱时,曾钓于淮阴城北淮水之上,一个漂洗丝绵的妇人出于怜悯,供他饭食。
后来韩信作了楚王,曾以千金相报。
}\par
\ping{
《淮阴怀古》是说王熙风的。“壮士须防恶犬欺”中的“恶犬”也许是指贾琏。眼前,他怕凤姐,将来凤姐反被他所欺,终至遭休弃。或者这一句是隐其被人告发,以至获罪遭厄也难说。
王熙风独操大权,主持荣国府,协理宁国府,以及弄权铁槛寺包揽外界诉讼、放债等事的“三齐位”,既确“定”于秦可卿“盖棺”之时,同时,这也正是决“定”她将来下场的时刻。她日后获罪受难,正是她自食恶果。
蒯通预言过韩信的下场,秦可卿也曾托梦凤姐要她为自己留后路。他们都是不见棺材不落泪的。
诗的后两句,则是说刘姥姥报她“一饭之恩”。当初刘姥姥来贾府伸手告贷,虽得了凤姐二十两银子,却受尽了“轻鄙”。
谁料到后来全凭刘姥姥,才把凤姐的女儿巧姐从火坑里给救了出来。
}
\par
\hop
广陵怀古\quad 其五\par
\zhu{广陵:古郡名,隋初先设扬州,后改作江都郡,治所在今江苏省扬州市。
隋炀帝杨广于大业元年强征河南、淮北各郡民上百馀万开通济渠,从洛阳直达江都。
渠宽四十步,渠旁筑“御道”,两岸种垂柳,世称隋堤。
又沿渠大造离宫,率后妃、百官等南游江都,穷极侈靡。
}\par
蝉噪鸦栖转眼过,隋堤风景近如何。
\par
只缘占得风流号,惹得纷纷口舌多。
\zhu{这两句是说,只因为隋炀帝喜欢游玩逸乐,得了个“风流”皇帝的称号,所以才招来后世纷纷讥贬。}
\par
\ping{
《广陵怀古》是说晴雯的。前两句是写欢乐宴游生活的短暂。怡红院“粉垣环护,绿柳周垂”,通往柳叶渚,还有一条柳堤,正好用“隋堤”作比。
宝玉、晴雯“相与共处者,仅五年八月有奇”,所以说“转眼过”。晴雯的“册子”中说她是“风流灵巧招人怨,寿夭多因诽谤生”。诗的后两句所说,亦即此意。
}
\par
\hop
桃叶渡怀古\quad 其六\par
\zhu{桃叶渡:故址在今江苏省南京市,秦淮河与青溪合流处。
《古今乐录》载:晋代王献之曾与其妾桃叶在此作别,作《桃叶歌》相赠,故后人称此渡为桃叶渡。
}\par
衰草闲花映浅池,桃枝桃叶总分离。
\par
六朝梁栋多如许,
\zhu{
六朝:指三国吴、东晋、宋、齐、梁、陈,皆先后在今之南京建都。
梁栋:既指屋宇,又指大臣。
如许:如此;像这样。
上句意思是:六朝的大臣们大多像王献之一样同亲人作别。
}
小照空悬壁上题。
\zhu{
小照:肖像画的一种,画面除人物外,还可以点缀简单的景物。
题:额头。
壁上题:墙壁的上部。
下句意谓小照徒然地挂在墙壁上。
}\par
\ping{
《桃叶渡怀古》是说贾迎春的。“衰草闲花映浅池”的景象,第七十九回中已经写到:宝玉“天天到紫菱洲一带地方,徘徊瞻顾”,“看那岸上的蓼花苇叶,池内的翠荇香菱,也都觉摇摇落落,似有追忆故人之态”。宝玉感伤之馀,口吟一诗,以“池塘一夜秋风冷,吹散芰荷红玉影”起头。
“桃枝桃叶”本是同根,恰好喻迎春与宝玉的姊弟关系。
诗的后两句是八十回之后的细节,无从揣测。所谓“六朝梁栋多如许”很像是“金陵诸钗的遭遇多半如此”的隐语。
至于后半部佚稿中是否会有宝玉空对迎春所遗之小照并伤悼题句一类的情节,就不得而知了。
}
\par
\hop
青冢怀古\quad 其七\par
\zhu{青冢:即王昭君墓。
在今内蒙古自治区呼和浩特市南大黑河岸上。
或谓:“塞草皆白,唯此冢草青,故名”(见《大同府志》);或谓:“墓无草木,远而望之,冥蒙作黛色,故曰青冢”(见清宋荦《筠廊偶记》)。
一说,蒙语“呼和”意谓“青”,“浩特”意谓“城”,昭君葬该地,故名“青冢”。
昭君名嫱,西汉元帝宫人,元帝为同南匈奴和亲,嫁昭君与呼韩邪单于。
}\par
黑水茫茫咽不流,冰弦拨尽曲中愁。
\zhu{
黑水:黑河,即今呼和浩特南之大黑河。
一说在北方的雪地里水看上去是黑的,所以黑水一般用来形容北方的水流。南方的水在绿色的草地里,看上去是白色的。
咽不流:河水哽咽不流,极写愁怨。
冰弦:一种优质的丝弦,其音激越清亮,色光洁,明透如水,故称冰弦。
一说为冰蚕丝制成的弦。
这里指王昭君琵琶上的弦。
}\par
汉家制度诚堪叹,樗栎应惭万古羞。
\zhu{
汉家制度:指汉元帝遣王昭君和亲事。《西京杂记》中说,汉元帝因后宫女子多,就叫画工画了像来,看图召见。
宫人都贿赂画工,独王嫱不肯,所以她的像画得最坏,不得见元帝。
后来,匈奴来求亲,元帝就按图像选昭君去。临行前,才发现她最美,悔之不及,就把毛延寿等许多画工都杀了。
这个故事并不符合史实(昭君是自愿和亲的),但流传很广。
樗:音“出”,臭椿。
栎:音“力”,柞[zuò]树。
古人认为这两种树不能成材,故常用来比喻无用之人。
这里指汉元帝。
}\par
\ping{
《青冢怀古》是说香菱的。她的“册子”上所画“下面有一池沼,其中水涸泥干”的图景,与本诗首句所写黑水咽而不流相合。
香菱永别故乡亲人,身世寂寞孤凄,这就是第二句冰弦寄愁所寓的意思。
“汉家制度”的“汉”,在这里是借作“汉子”,亦即“丈夫”解的。薛蟠为人横暴,却被悍妇夏金桂捏在手里,由她说了算。
这样的家庭关系,在封建时代,尤其显得“堪叹”。“呆霸王”是草包,是不成材的“樗栎”,他连好坏也分不清,屈从金桂,虐待香菱,真该永远蒙羞。
}
\par
\hop
马嵬怀古\quad 其八\par
\zhu{马嵬(嵬音“韦”):即马嵬驿,在今陕西省兴平县马嵬镇。
杨贵妃因受宠于唐玄宗,遂一门显贵,势压天下。
后范阳节度使安禄山以讨杨贵妃之兄宰相杨国忠为名起兵反唐,攻破潼关,直逼长安。
唐玄宗携杨妃逃往四川,行至马嵬,军士以罪在杨门,杀杨国忠并请诛杨贵妃。
玄宗被迫缢杀杨贵妃,埋于驿西道旁。
}\par
寂寞脂痕渍汗光,温柔一旦付东洋。
\zhu{渍:音“自”,液体粘在东西上。
上句以汗水浸渍胭脂残痕来形容杨贵妃被缢死时的面容。
付东洋:付之东流,成空。
}\par
只因遗得风流迹,此日衣衾尚有香。
\zhu{这两句意指杨贵妃的风流遗韵至今犹存。
}\par
\ping{
《马嵬怀古》是说秦可卿的。前两句写她“淫丧天香楼”,悬梁自尽。“渍汗光”三字,状缢者遗容,想像逼真。
书中曾说可卿“生得袅娜纤巧,行事又温柔和平”,所以用“温柔”二字。
后两句说的就是贾宝玉在她房中“神游太虚幻境”事。
}
\par
\hop
蒲东寺怀古\quad 其九\par
\zhu{蒲东寺:即唐代元稹《会真记》中张生与崔莺莺相会的普救寺,因寺在山西省蒲津之东,故又称蒲东寺。
}\par
小红骨贱最身轻,私掖偷携强撮成。
\zhu{小红:即崔莺莺的丫鬟红娘。
下句指红娘瞒着老夫人为张生和莺莺撮合。
从封建道学眼光看来,不安分的红娘是所谓骨头生得轻贱,即“骨贱”、“身轻”。
}\par
虽被夫人时吊起,已经勾引彼同行。
\zhu{这二句指《西厢记》中《拷红》一折。
意谓夫人虽拷打红娘问出私情,但为时已晚。
夫人:即崔莺莺的母亲郑氏。
吊起:当为牵合谜底而用,是泛说,剧中只言拷打。
}\par
\ping{
《蒲东寺怀古》是说金钏儿的。第三句中的“夫人”即王夫人。“小红”即为金钏。
王夫人“今忽见金钏儿行此无耻之事”,说的是第三十回中金钏和宝玉调笑,即“私掖偷携”(第二十三回也有),故有“身轻骨贱”之语。
书中写金钏儿与宝玉的关系是有隐笔的。这从第四十三回“不了情暂撮土为香”宝玉偷偷祭奠她的描写可以看出。
}
\par
\hop
梅花观怀古\quad 其十\par
\zhu{梅花观:《牡丹亭》中杜家为守护杜丽娘坟墓而建造的庙宇。
柳梦梅曾寄居观中,拾得丽娘生前自画像,引来丽娘游魂,并挖墓开棺,救活丽娘,结为夫妻。
}\par
不在梅边在柳边,个中谁拾画婵娟。
\zhu{首句是杜丽娘题自画像诗的最后一句,句中暗含柳梦梅的名字。
个中:此中。
婵娟:美好的样子,多形容女子。
拾画婵娟:柳梦梅在观中拾得杜丽娘生前自画像。
}\par
团圆莫忆春香到,一别西风又一年。
\zhu{春香:杜丽娘丫鬟的名字。
这两句的意思是不要去回想春香来到而得团圆的情景,别离以来,西风又起,又过去一年了。
剧中柳梦梅在外怀念丽娘,有“砧声又报一年秋”等语。
}\par
\ping{过去这种游历过各地的闺秀是人间顶配了吧。
}
\par
\ping{
《梅花观怀古》是说林黛玉的。杜丽娘受封建礼教压迫,爱情理想未实现,抑郁而死。与林黛玉很像。
小说中黛玉还常常有意无意地引用丽娘的唱词,可见两心是相通的。
“画婵娟”在这里是《秋窗风雨夕》处脂评所谓的“画儿中爱宠”的意思,亦即成了“镜中花”、“水中月”。
黛玉不能像丽娘那样死而复生,所以诗的第三句用否定语气说不能“团圆”。
黛玉死于何时,脂评虽无明文,但《葬花吟》中已作过“谶语”:“试看春残花渐落,便是红颜老死时。”
同时,春天又是宝黛曾经以为可以实现美好理想的时节,所谓“三月香巢已垒成”是也。但后来“人去梁空巢也倾”理想全破灭了。
所以,“团圆莫忆春香到”句,还可能包含这些双关意在。
脂评还说后来潇湘馆“落叶萧萧,寒烟漠漠”,如果这是宝玉流亡遭厄后回大观园“对景悼颦儿”时所见的景象,那就恰好与诗的末句写秋风时节相符合。
}
\par
\hop
众人看了,都称奇道妙。
宝钗先说道:“前八首都是史鉴上有据的;后二首却无考,我们也不大懂得,不如另作两首为是。
”\geng{如何?必得宝钗此驳,方是好文。
后文若真另作,亦必无趣;若不另作,又有何法省之。
看他下文如何。
}黛玉忙拦道:\geng{好极!非黛玉不可。
脂砚。
}“这宝姐姐也忒胶柱鼓瑟、\zhu{胶柱鼓瑟:语出《史记·廉颇蔺相如列传》,蔺相如劝赵王不要任用赵括为将说:“王以名使括,若胶柱而鼓瑟耳。
括徒能读其父书传,不知合变也。
”瑟:乐器名。
柱:瑟上架弦的柱,能移动,可调音。
用胶粘柱则音不能调,比喻拘泥固执不知灵活变通。
}矫揉造作了。
这两首虽于史鉴上无考,咱们虽不曾看这些外传,不知底里,
\ping{宝玉黛玉在第二十三回桃花树下共读《西厢记》。此处欲盖弥彰。}
难道咱们连两本戏也没有见过不成?那三岁孩子也知道,何况咱们?”探春便道:“这话正是了。
”\geng{余谓颦儿必有尖语来讽,不望竟有此饰词代为解释,此则真心以待宝钗也。
}李纨又道:“况且他原是走到这个地方的。
这两件事虽无考,古往今来,以讹传讹,好事者竟故意的弄出这古迹来以愚人。
比如那年上京的时节,
\zhu{上京:进入京城。}
单是关夫子的坟,倒见了三四处。
关夫子一生事业,\zhu{关夫子:即关羽,字云长,三国时蜀汉大将,后世封建统治阶级把他神化,到处修庙塑像,并尊称为“关公”、“关帝”,或把他同“文圣”孔夫子并列而为“武圣”,故亦称“关夫子”。
}皆是有据的,如何又有许多的坟?自然是后来人敬爱他生前为人,只怕从这敬爱上穿凿出来,\zhu{穿凿:牵强附会,任意牵合意义,强求其通。
}也是有的。
及至看《广舆记》上,\zhu{《广舆记》:地理书。
}不止关夫子的坟多,自古来有些名望的,坟就不少,无考的古迹更多。
\ping{争夺名人故里原来是传统。
}如今这两首虽无考,凡说书唱戏,甚至于求的签上皆有注批,老小男女,俗语口头,人人皆知皆说的。
况且又并不是看了《西厢》《牡丹》的词曲,怕看了邪书。
这竟无妨,只管留着。
”宝钗听说,方罢了。
\geng{此为三染无痕也,
\zhu{三染:这里指针对宝黛共读西厢的情节,反复提及,多角度渲染。}
妙极!天\sout{花}[衣]无缝之文。
}大家猜了一回,皆不是。
\par
冬日天短,不觉又是前头吃晚饭之时,一齐前来吃饭。
因有人回王夫人说:“袭人的哥哥花自芳进来说,他母亲病重了,想他女儿。
他来求恩典,接袭人家去走走。
”王夫人听了,便道:“人家母女一场,岂有不许他去的。
”一面就叫了凤姐儿来,告诉了凤姐儿,命酌量去办理。
\par
凤姐儿答应了,回至房中,便命周瑞家的去告诉袭人原故。
又吩咐周瑞家的:“再将跟着出门的媳妇传一个,你两个人,再带两个小丫头子,跟了袭人去。
外头派四个有年纪跟车的。
要一辆大车,你们带着坐;要一辆小车,给丫头们坐。
”周瑞家的答应了,才要去,凤姐儿又道:“那袭人是个省事的,你告诉说我的话:叫他穿几件颜色好衣裳,大大的包一包袱衣裳拿着,包袱也要好好的,手炉也要拿好的。
临走时,叫他先来我瞧瞧。
”周瑞家的答应去了。
\ping{袭人母亲病重,凤姐更在乎的是排场。}
\par
半日,果见袭人穿戴来了,两个丫头与周瑞家的拿着手炉与衣包。
凤姐儿看袭人头上戴着几枝金钗珠钏,倒华丽;又看身上穿着桃红百花刻丝银鼠袄子,\zhu{刻丝:在丝织品上用丝平织成的图案,与凸出的绣花不同。
}葱绿盘金彩绣绵裙,\zhu{盘金:用金线在织物上盘出花样。
}外面穿着青缎灰鼠褂。
凤姐儿笑道:“这三件衣裳都是太太的,赏了你倒是好的;但只这褂子太素了些,如今穿着也冷,你该穿一件大毛的。
”袭人笑道:“太太就只给了这灰鼠的,还有一件银鼠的。
说赶年下再给大毛的,还没有得呢。
”凤姐儿笑道:“我倒有一件大毛的,我嫌风毛儿出不好了,\zhu{风毛儿:皮毛衣服有的特意将领、袖、襟、摆等边缘部分的皮毛露在外面,以增添美观及显示皮毛的珍贵,因其露毛在外,故称“风毛儿”,也叫“出锋”。
}正要改去。
也罢,先给你穿去罢。
等年下太太给作的时节我再作罢,只当你还我一样。
”众人都笑道:“奶奶惯会说这话。
成年家大手大脚的,替太太不知背地里赔垫了多少东西,真真的赔的是说不出来,那里又和太太算去?偏这会子又说这小气话取笑儿。
”凤姐儿笑道:“太太那里想的到这些?究竟这又不是正经事,再不照管,也是大家的体面。
说不得我自己吃些亏,把众人打扮体统了,宁可我得个好名也罢了。
一个一个像‘烧糊了的卷子’似的,人先笑话我当家倒把人弄出个花子来。
”\zhu{花子:乞丐。
也称为“叫花子”。
}众人听了,都叹说:“谁似奶奶这样圣明!在上体贴太太,在下又疼顾下人。
”一面说,一面只见凤姐儿命平儿将昨日那件石青刻丝八团天马皮褂子拿出来,\zhu{石青:淡灰青色。
刻丝:在丝织品上用丝平织成的图案,与凸出的绣花不同。
八团:衣面上刻丝或绣成的八个彩团的图案。
天马皮:沙狐腹下之皮。
}与了袭人。
又看包袱,只得一个弹墨花绫水红绸里的夹包袱,\zhu{弹墨:以纸剪镂空图案覆于织品上,用墨色或其它颜色弹或喷成各种图案花样。
绫:古代丝织物名。
水红:比粉红略深而鲜艳。
夹[jiá]:里外两层的(衣被等)。
}里面只包着两件半旧棉袄与皮褂。
凤姐儿又命平儿把一个玉色绸里的哆啰呢的包袱拿出来,\zhu{哆啰呢:一种西洋传入的阔幅呢料。
}又命包上一件雪褂子。
\zhu{雪褂子:御雪外衣。
}\par
平儿走去拿了出来,一件是半旧大红猩猩毡的,一件是大红羽纱的。
\zhu{羽纱:毛织物,也称羽毛纱,疏细者称羽纱,厚密者称羽缎,制成衣服均可防雨雪。
}
袭人道:“一件就当不起了。
”平儿笑道:“你拿这猩猩毡的。
把这件顺手拿将出来,叫人给邢大姑娘送去。
昨儿那么大雪,人人都是有的,不是猩猩毡就是羽缎羽纱的,十来件大红衣裳,映着大雪好不齐整。
就只他穿着那件旧毡斗蓬,越发显的拱肩缩背,好不可怜见的。
如今把这件给他罢。
”凤姐儿笑道:“我的东西,他私自就要给人。
我一个还花不够,再添上你提着,更好了!”众人笑道:“这都是奶奶素日孝敬太太,疼爱下人。
若是奶奶素日是小气的,只以东西为事,不顾下人的,姑娘那里还敢这样了。
”凤姐儿笑道:“所以知道我的心的,也就是他还知三分罢了。
”说着,又嘱咐袭人道:“你妈若好了就罢;若不中用了,只管住下,打发人来回我,我再另打发人给你送铺盖去。
可别使人家的铺盖和梳头的家伙。
”又吩咐周瑞家的道:“你们自然也知道这里的规矩的,也不用我嘱咐了。
”周瑞家的答应:“都知道。
我们这去到那里,总叫他们的人回避。
若住下,必是另要一两间内房的。
”说着,跟了袭人出去,又吩咐预备灯笼,遂坐车往花自芳家来,不在话下。
\par
这里凤姐又将怡红院的嬷嬷唤了两个来,吩咐道:“袭人只怕不来家,你们素日知道那大丫头们,那两个知好歹,派出来在宝玉屋里上夜。
你们也好生照管着,别由着宝玉胡闹。
”两个嬷嬷去了,一时来回说:“派了晴雯和麝月在屋里,我们四个人原是轮流着带管上夜的。
”凤姐儿听了,点头道:“晚上催他早睡,早上催他早起。
”老嬷嬷们答应了,自回园去。
一时果有周瑞家的带了信回凤姐儿说:“袭人之母业已停床,\zhu{停床:人刚死停尸于床,尚未入殓,叫“停床”。
}
不能回来。
”凤姐儿回明了王夫人,一面着人往大观园去取他的铺盖妆奁。
\par
宝玉看着晴雯麝月二人打点妥当,送去之后,晴雯麝月皆卸罢残妆,脱换过裙袄。
晴雯只在熏笼上围坐。
\zhu{熏笼\foot{
\footPic{陈洪绶所绘《斜倚熏笼图》}{xunlong.jpg}{0.8}
}:罩在炭盆上的、供熏香烘物和取暖用的箱形罩笼,又名“火箱”。
}麝月笑道:“你今儿别装小姐了,我劝你也动一动儿。
”晴雯道:“等你们都去尽了,我再动不迟。
有你们一日,我且受用一日。
”\ping{袭人作为第一大丫鬟,在之前光芒太大,别的丫鬟都黯然失色。
袭人母亲去世,才腾出空间来写晴雯。
}麝月笑道:“好姐姐,我铺床,你把那穿衣镜的套子放下来,上头的划子划上,\zhu{划子:这里指穿衣镜框子上押镜帘的活动小签子。
}你的身量比我高些。
”说着,便去与宝玉铺床。
晴雯嗐了一声,笑道:“人家才坐暖和了,你就来闹。
”此时宝玉正坐着纳闷,想袭人之母不知是死是活,忽听见晴雯如此说,便自己起身出去,放下镜套,划上消息,
\zhu{消息:控制机件转动的开关、枢纽。又称「机关」、「机括」。}
进来笑道:“你们暖和罢,都完了。
”晴雯笑道:“终久暖和不成的,我又想起来汤婆子还没拿来呢。
”
\zhu{汤婆子:亦叫“暖脚壶”,注入热水,塞口,放入被中取暖。}
麝月道:“这难为你想着!他素日又不要汤婆子,咱们那熏笼上暖和,比不得那屋里炕冷,今儿可以不用。
”宝玉笑道:“这个话,你们两个都在那上头睡了,我这外边没个人,我怪怕的,一夜也睡不着。
”晴雯道:“我是在这里。
麝月往他外边睡去。
”说话之间,天已二更,麝月早已放下帘幔,移灯炷香,伏侍宝玉卧下,二人方睡。
\par
晴雯自在熏笼上,麝月便在暖阁外边。
至三更以后,宝玉睡梦之中,便叫袭人。
叫了两声,无人答应,自己醒了,方想起袭人不在家,自己也好笑起来。
晴雯已醒,因笑唤麝月道:“连我都醒了,他守在旁边还不知道,真是个挺死尸的。
”麝月翻身打个哈气笑道:“他叫袭人,与我什么相干!”因问“作什么?”宝玉说要吃茶,麝月忙起来,单穿红绸小棉袄儿。
宝玉道:“披上我的袄儿再去,仔细冷着。
”麝月听说,回手便把宝玉披着起夜的一件貂颏满襟暖袄披上,\zhu{
颏:音“科”,下巴。
满襟:亦称“大襟”。左右两襟大小不一,大襟将小襟盖着,胸前不开扣,纽扣在一侧的腋下。
}下去向盆内洗手,先倒了一钟温水,拿了大漱盂,宝玉漱了一口;然后才向茶槅上取了茶碗,\zhu{茶槅:搁茶碗的架子。
}
先用温水涮了一涮,向暖壶中倒了半碗茶,递与宝玉吃了;自己也漱了一漱,吃了半碗。
晴雯笑道:“好妹子,也赏我一口儿。
”麝月笑道:“越发上脸儿了!”晴雯道:“好妹妹,明儿晚上你别动,我伏侍你一夜,如何?”麝月听说,只得也伏侍他漱了口,倒了半碗茶与他吃过。
\ping{晴雯此时吃茶,何其尊贵;而第七十七回,晴雯临死前,吃茶又何其凄凉。}
麝月笑道:“你们两个别睡,说着话儿,我出去走走回来。
”晴雯笑道:“外头有个鬼等着你呢。
”宝玉道:“外头自然有大月亮的,我们说话,你只管去。
”一面说,一面便嗽了两声。
\par
麝月便开了后门,揭起毡帘一看,果然好月色。
晴雯等他出去,便欲唬他顽耍。
仗着素日比别人气壮,不畏寒冷,也不披衣,只穿着小袄,便蹑手蹑脚的下了薰笼,随后出来。
宝玉笑劝道:“看冻着,不是顽的。
”晴雯只摆手,随后出了房门。
只见月光如水,忽然一阵微风,只觉侵肌透骨,不禁毛骨森然。
心下自思道:“怪道人说热身子不可被风吹,这一冷果然利害。
”一面正要唬麝月,只听宝玉高声在内道:“晴雯出去了!”晴雯忙回身进来,笑道:“那里就唬死了他?偏你惯会这蝎蝎螫螫老婆汉像的!”\zhu{
蝎蝎螫螫:用人们对蝎螫的惊恐神情,形容过分的担心、惶恐、大惊小怪。
老婆汉像:虽是男子汉却婆婆妈妈的,意为碰上点小事就大惊小怪。
}宝玉笑道:“倒不为唬坏了他,头一则你冻着也不好;二则他不防,不免一喊,倘或唬醒了别人,不说咱们是顽意,倒反说袭人才去了一夜,你们就见神见鬼的。
你来把我的这边被掖一掖。
”晴雯听说,便上来掖了掖,伸手进去渥一渥时,
\zhu{渥:同“焐”,音“物”,用热的东西接触凉的使变暖。}
宝玉笑道:“好冷手!我说看冻着。
”一面又见晴雯两腮如胭脂一般,用手摸了一摸,也觉冰冷。
宝玉道:“快进被来渥渥罢。
”一语未了,只听咯噔的一声门响,麝月慌慌张张的笑了进来,说道:“吓了我一跳好的。
黑影子里,山子石后头,只见一个人蹲着。
我才要叫喊,原来是那个大锦鸡,\zhu{锦鸡:野鸡的一种,雄者羽毛美丽,头部有金黄色丝状羽冠,易于驯养,可供玩赏,羽毛可作装饰。
}见了人一飞,飞到亮处来,我才看真了。
若冒冒失失一嚷,倒闹起人来。
”一面说,一面洗手,又笑道:“晴雯出去我怎么不见?一定是要唬我去了。
”宝玉笑道:“这不是他,在这里渥呢!我若不叫的快,可是倒唬一跳。
”晴雯笑道:“也不用我唬去,这小蹄子已经自怪自惊的了。
”一面说,一面仍回自己被中去了。
麝月道:“你就这么‘跑解马’似的打扮得伶伶俐俐的出去了不成?”\zhu{跑解马:也叫跑马卖解,即在马上表演各种技艺,表演者皆著短装。
}宝玉笑道:“可不就这么出去了。
”麝月道:“你死不拣好日子!你出去站一站,把皮不冻破了你的。
”说着,又将火盆上的铜罩揭起,拿灰锹重将熟炭埋了一埋,拈了两块素香放上,\zhu{素香:家常用的普通香料。
}仍旧罩了,至屏后重剔了灯,方才睡下。
\zhu{剔灯:挑起灯芯,剔除馀烬,使灯更亮。
}\par
晴雯因方才一冷,如今又一暖,不觉打了两个喷嚏。
宝玉叹道:“如何?到底伤了风了。
”麝月笑道:“他早起就嚷不受用,一日也没吃饭。
他这会还不保养些,还要捉弄人。
明儿病了,叫他自作自受。
”宝玉问:“头上可热?”晴雯嗽了两声,说道:“不相干,那里这么娇嫩起来了。
”说着,只听外间房中十锦格上的自鸣钟当当两声,\zhu{
十锦:即“什锦”,由多种原料制成或多种花样拼成的。
格子:架子,放置器物的木器。木架上分不同形状的许多层小格,格内可放入各种器皿、用具。
}外间值宿的老嬷嬷嗽了两声,因说道:“姑娘们睡罢,明儿再说罢。
”宝玉方悄悄的笑道:“咱们别说话了,又惹他们说话。
”说着,方大家睡了。
\ping{隔墙有耳,怡红院无私人空间,宝玉和丫鬟的一举一动,似乎都在监视之中,这可能解释了后面抄检大观园的时候,王夫人为何对怡红院了如指掌,以及晴雯为何被举报构陷。
}\par
至次日起来,晴雯果觉有些鼻塞声重,懒怠动弹。
宝玉道:“快不要声张!太太知道,又叫你搬了家去养息。
家去虽好,到底冷些,不如在这里。
你就在里间屋里躺着,我叫人请了大夫,悄悄的从后门来瞧瞧就是了。
”晴雯道:“虽如此说,你到底要告诉大奶奶一声儿,不然一时大夫来了,人问起来,怎么说呢?”宝玉听了有理,便唤一个老嬷嬷吩咐道:“你回大奶奶去,就说晴雯白冷着了些,\zhu{白:单单,只是。
}不是什么大病。
袭人又不在家,他若家去养病,这里更没有人了。
传一个大夫,悄悄的从后门进来瞧瞧,别回太太罢了。
”老嬷嬷去了半日,来回说:“大奶奶知道了,说两剂药吃好了便罢,若不好时,还是出去为是。
如今时气不好,恐沾带了别人事小,姑娘们的身子要紧的。
”晴雯睡在暖阁里,只管咳嗽,听了这话,气的喊道:“我那里就害瘟病了,只怕过了人!我离了这里,看你们这一辈子都别头疼脑热的。
”说着,便真要起来。
宝玉忙按他,笑道:“别生气,这原是他的责任,唯恐太太知道了说他,不过白说一句。
你素习好生气,如今肝火自然盛了。
”\par
正说时,人回大夫来了。
宝玉便走过来,避在书架之后。
只见两三个后门口的老嬷嬷带了一个大夫进来。
这里的丫鬟都回避了,有三四个老嬷嬷放下暖阁上的大红绣幔,晴雯从幔中单伸出手去。
那大夫见这只手上有两根指甲,足有三寸长,
\zhu{寸:市寸,长度非法定计量单位,1市寸等于3⅓厘米。三寸:十厘米长。}
尚有金凤花染的通红的痕迹,\zhu{金凤花:即“凤仙花”,一年生草本植物,夏季开花,红色的花瓣可用来染指甲,因又名指甲花或指甲草。
}便忙回过头来。
\ping{长指甲是不事劳动的贵族的象征,因为长指甲不适合劳动。
晴雯作为一个丫鬟,却留着贵族的长指甲,这呼应了第五回晴雯判词里写的“心比天高,身为下贱”。
晴雯把自己当作贵族小姐,和贵族一样坐享其成而不劳动,这一点袭人是知道的。
第六十二回,袭人(对晴雯)笑道:“……你倒别和我拿三撇四的,我烦你做个什么,把你懒的横针不拈,竖线不动。
一般也不是我的私活烦你,横竖都是他的,你就都不肯做。
……”作为一个受到压迫的苦命丫鬟,晴雯身上体现出反抗命运,不满现实的倔强,而不是接受自己低人一等的身份,做一个驯服的奴才,这是晴雯身上值得肯定的闪光点。
但是晴雯反对贵族特权和等级制度的办法,是努力摆脱奴才的身份,成为一个拥有特权的贵族,而不是消灭特权,消灭等级制度,实现真正的平等。
受压迫者在没有科学的反抗压迫的理论指导下,在等级制度下的浸淫下,很容易默认等级制度是不可动摇的公理,从而使得反抗压迫的行动,难以真正铲除压迫的根源,使得为广大受压迫者争取平等的运动,堕落为为少数反抗者谋取私利的运动。
}有一个老嬷嬷忙拿了一块手帕掩了。
那大夫方诊了一回脉,起身到外间,向嬷嬷们说道:“小姐的症是外感内滞,\zhu{外感:指感受风、寒、暑、湿、燥、热而致病。
内滞:在消化系统内有饮食积滞。
}近日时气不好,竟算是个小伤寒。
幸亏是小姐素日饮食有限,风寒也不大,不过是血气原弱,偶然沾带了些,吃两剂药疏散疏散就好了。
”说着,便又随婆子们出去。
\par
彼时,李纨已遣人知会过后门上的人及各处丫鬟回避,那大夫只见了园中的景致,并不曾见一女子。
一时出了园门,就在守园门的小厮们的班房内坐了,开了药方。
老嬷嬷道:“你老爷且别去,我们小爷罗唆,恐怕还有话说。
”大夫忙道:“方才不是小姐,是位爷不成?那屋子竟是绣房一样,又是放下幔子来的,如何是位爷呢?”老嬷嬷悄悄笑道:“我的老爷,怪道小厮们才说今儿请了一位新大夫来了,真不知我们家的事。
那屋子是我们小哥儿的,那人是他屋里的丫头,倒是个大姐,那里的小姐?若是小姐的绣房,小姐病了,你那么容易就进去了?”说着,拿了药方进去。
\par
宝玉看时,上面有紫苏、桔梗、防风、荆芥等药,后面又有枳实、
\zhu{枳:音“纸”。}
麻黄。
宝玉道:“该死,该死,他拿着女孩儿们也像我们一样的治,如何使得!凭他有什么内滞,这枳实、麻黄如何禁得。
谁请了来的?快打发他去罢!再请一个熟的来。
”老婆子道:“用药好不好,我们不知道这理。
如今再叫小厮去请王太医去倒容易,只是这大夫又不是告诉总管房请来的,这轿马钱是要给他的。
”宝玉道:“给他多少?”婆子道:“少了不好看,也得一两银子,才是我们这门户的礼。
”宝玉道:“王太医来了给他多少?”婆子笑道:“王太医和张太医每常来了,也并没个给钱的,不过每年四节大趸送礼,\zhu{大趸:也作“打趸”,打总、凑总数的意思。
趸:音“盹”,整数、整批。
}那是一定的年例。
这人新来了一次,须得给他一两银子去。
”宝玉听说,便命麝月去取银子。
麝月道:“花大奶奶还不知搁在那里呢?”宝玉道:“我常见他在螺甸小柜子里取钱,\zhu{螺甸小柜子:用“螺甸”这种工艺装饰的小柜子。
螺甸,亦作“螺钿”或“螺填”,即用贝壳磨薄制成各种花样镶嵌装饰在漆器或雕镂器物上。
}我和你找去。
”说着,二人来至宝玉堆东西的房子,开了螺甸柜子,上一格子都是些笔墨、扇子、香饼、各色荷包、汗巾等物;下一格却是几串钱。
于是开了抽屉,才看见一个小簸箩内放着几块银子,
\zhu{簸箩[bǒluo]:盛物的竹筐。}
倒也有一把戥子。
\zhu{戥:音“等”,一种称量金银、药品等所用的小秤,计量单位从分厘到两,构造和原理跟杆秤相同,盛物体的部分是一个小盘子。
}麝月便拿了一块银子,提起戥子来问宝玉:“那是一两的星儿?”宝玉笑道:“你问我?有趣,你倒成了才来的了。
”麝月也笑了,又要去问人。
宝玉道:“拣那大的给他一块就是了。
又不作买卖,算这些做什么!”麝月听了,便放下戥子,拣了一块掂了一掂,笑道:“这一块只怕是一两了。
宁可多些好,别少了,叫那穷小子笑话,不说咱们不识戥子,倒说咱们有心小器似的。
”那婆子站在外头台矶上,笑道:“那是五两的锭子夹了半边,这一块至少还有二两呢!这会子又没夹剪,姑娘收了这块,再拣一块小些的罢。
”麝月早掩了柜子出来,笑道:“谁又找去!多了些你拿了去罢。
”宝玉道:“你只快叫茗烟再请王大夫去就是了。
”婆子接了银子,自去料理。
\ping{宝玉和麝月都是无心持家的富贵散人。
}\par
一时茗烟果请了王太医来,诊了脉后,说的病症与前相仿,只是方上果没有枳实、麻黄等药,倒有当归、陈皮、白芍等,药之分量较先也减了些。
宝玉喜道:“这才是女孩儿们的药,虽然疏散,也不可太过。
旧年我病了,却是伤寒,内里饮食停滞,他瞧了,还说我禁不起麻黄、石膏、枳实等狼虎药。
我和你们一比,我就如那野坟圈子里长的几十年的一棵老杨树,你们就如秋天芸儿进我的那才开的白海棠,连我禁不起的药,你们如何禁得起。
”麝月等笑道:“野坟里只有杨树不成?难道就没有松柏?我最嫌的是杨树,那么大笨树,叶子只一点子,没一丝风,他也是乱响。
你偏比他,也太下流了。
”宝玉笑道:“松柏不敢比。
连孔子都说:‘岁寒然后知松柏之后凋也。
’\zhu{岁寒然后知松柏之后凋也:语出《论语·子罕》。
以松柏之后凋喻处于浊世而能保持自身的节操。
}可知这两件东西高雅,不怕羞臊的才拿他混比呢。
”\par
说着,只见老婆子取了药来。
宝玉命把煎药的银吊子找了出来,\geng{“找”字神理,乃不常用之物也。
}就命在火盆上煎。
晴雯因说:“正经给他们茶房里煎去,弄得这屋里药气,如何使得。
”宝玉道:“药气比一切的花香果子香都雅。
神仙采药烧药,再者高人逸士采药治药,最妙的一件东西。
这屋里我正想各色都齐了,就只少药香,如今恰好全了。
”一面说,一面早命人煨上。
又嘱咐麝月打点东西,遣老嬷嬷去看袭人,劝他少哭。
一一妥当,方过前边来贾母王夫人处问安吃饭。
\ping{宝玉真乃妇女之友,中央空调,周旋于众丫鬟中目不暇接。
}\par
正值凤姐儿和贾母王夫人商议说:“天又短又冷,不如以后大嫂子带着姑娘们在园子里吃饭一样。
等天长暖和了,再来回的跑也不妨。
”王夫人笑道:“这也是好主意。
刮风下雪倒便宜。
吃些东西受了冷气也不好;空心走来,一肚子冷风,压上些东西也不好。
不如后园门里头的五间大房子,横竖有女人们上夜的,挑两个厨子女人在那里,单给他姊妹们弄饭。
新鲜菜蔬是有分例的,在总管房里支去,或要钱,或要东西;那些野鸡、獐、狍各样野味,分些给他们就是了。
”贾母道:“我也正想着呢,就怕又添一个厨房多事些。
”凤姐道:“并不多事。
一样的分例,这里添了,那里减了。
就便多费些事,小姑娘们冷风朔气的,\geng{“朔”字又妙!“朔”作“韶”,北音也。
用北音,奇想奇想。
\zhu{这里可能是说,“朔”在北方方言里念作“韶”。
}}别人还可,第一林妹妹如何禁得住?就连宝兄弟也禁不住,何况众位姑娘。
”贾母道:“正是这话了。
上次我要说这话,我见你们的大事太多了,如今又添出这些事来,……”要知端的——\par
\qi{总评:此回再从猜谜着色,便与前回重复,且又是一幅即景联诗图矣,成何趣味?就灯谜中生一番讥评,别有清思,迥非凡艳。
\hang
搁起灯谜,接入袭人了,却不就袭人一面写照,作者大有苦心。
盖袭人不盛饰,则非大家威仪,如盛饰,又岂有其母临危而盛饰者乎?在凤姐一面,于衣服车马仆从房屋铺盖等物一一检点,色色亲嘱,既得掌家人体统,而袭人之俊俏风神毕现。
\hang
文有数千言写一琐事者,如一吃茶,偏能于未吃以前、既吃以后,细细描写;如一拿银,偏能于开柜时生无数波折,\sout{平}[秤]银时又生无数波折。
心细如发。
}
\dai{101}{袭人探望病重母亲,凤姐精心安排}
\dai{102}{宝玉和麝月开柜子取银子}