\chapter{金寡妇贪利权受辱\quad 张太医论病细穷源}
\qi{新样幻情欲收拾,可卿从此世无缘。
和肝益气浑闲事,谁识今朝寻病源?}\par
话说金荣因人多势众,又兼贾瑞勒令,赔了不是,给秦钟磕了头,宝玉方才不吵闹了。
大家散了学,金荣回到家中,越想越气,说:“秦钟不过是贾蓉的小舅子,又不是贾家的子孙,附学读书,也不过和我一样。
他因仗着宝玉和他好,他就目中无人。
他既是这样,就该行些正经事,人也没的说。
他素日又和宝玉鬼鬼祟祟的,只当我们都是瞎子,看不见。
今日他又去勾搭人,偏偏的撞在我眼里。
\meng{偏是鬼鬼祟祟者,多以为人不见其行,不知其心。
}就是闹出事来,我还怕什么不成?”\par
他母亲胡氏听见他咕咕嘟嘟的说,因问道:“你又要增什么闲事?好容易\meng{“好容易”三字,写尽天下迎逢要便宜苦恼。
}\zhu{好容易:好不容易。
}我望你姑妈说了,你姑妈千方百计的才向他们西府里的琏二奶奶跟前说了,你才得了这个念书的地方。
若不是仗着人家,咱们家里还有力量请的起先生?况且人家学里,茶也是现成的,饭也是现成的。
你这二年在那里念书,家里也省好大的嚼用呢。
\zhu{嚼用:家常的开支、费用。}
省出来的,你又爱穿件鲜明衣服。
再者,不是因你在那里念书,你就认得什么薛大爷了?那薛大爷一年不给不给,这二年也帮了咱们有七八十两银子。
\ji{因何无故给许多银子?金母亦当细思之。
}
\meng{可怜!妇人爱子,每每如此。
自知所得者多,而不知所失者大,可胜叹者!}你如今要闹出了这个学房,再要找这么个地方,我告诉你说罢,比登天还难呢!\ji{如此弄银,若有金荣在,亦可得。
}你给我老老实实的顽一会子睡你的觉去,好多着呢。
”\ping{孩子的情绪永远不被当作什么要紧事,父母时常忽视其忧恼,然又忧虑子女与自己离心。
}于是金荣忍气吞声,不多一时他自去睡了。
次日仍旧上学去了。
不在话下。
\par
且说他姑娘,原聘给的是贾家玉字辈的嫡派,名唤贾璜。
但其族人那里皆能像宁荣二府的富势,原不用细说。
这贾璜夫妻守着些小的产业,又时常到宁荣二府里去请请安,又会奉承凤姐儿并尤氏,所以凤姐儿尤氏也时常资助资助他,\meng{原来根由如此,大与秦钟不同。
}方能如此度日。
今日正遇天气晴明,又值家中无事,遂带了一个婆子,坐上车,来家里走走,瞧瞧寡嫂并侄儿。
\par
闲话之间,金荣的母亲偏提起昨日贾家学房里的那事,从头至尾,一五一十都向他小姑子说了。
这璜大奶奶不听则已,听了,一时怒从心上起,说道:“这秦钟小崽子是贾门的亲戚,难道荣儿不是贾门的亲戚?\ji{这贾门的亲戚比那贾门的亲戚。
}人都别忒势利了,况且都作的是什么有脸的好事!就是宝玉,也犯不上向着他到这个样。
等我去到东府瞧瞧我们珍大奶奶,再向秦钟他姐姐说说,叫他评评这个理。
\ji{未必能如此说。
}
\meng{狗仗人势者,开口便有多少必胜之谈,事要三思,免劳后悔。
}这金荣的母亲听了这话,急的了不得,忙说道:“这都是我的嘴快,告诉了姑奶奶了,求姑奶奶别去,别管他们谁是谁非。
\ji{不论谁是谁非,有钱就可矣。
}
\meng{胡氏可谓善哉!}倘或闹起来,怎么在那里站得住。
若是站不住,家里不但不能请先生,反倒在他身上添出许多嚼用来呢。
”\ping{金荣的母亲还是存了要说法的心思,不然何必要说,但是看到对方反应过度又恐惧,怕事情闹大了不好收拾。
}璜大奶奶听了,说道:“那里管得许多,你等我说了,看是怎么样!”也不容他嫂子劝,一面叫老婆子瞧了车,就坐上往宁府里来。
\meng{何等气派,何等声势,有射石饮羽之力,
\zhu{
饮:隐没;羽:箭尾的羽毛。射石饮羽:箭射到石头里,隐没了箭尾的羽毛。
原形容发箭的力量极强。后也形容武艺高强。
}
动天摇地,如项羽喑咤。
\zhu{喑:音“印”,喑呜;咤:音“炸”,叱咤。
两者都是怒喝声的意思。
}}\par
到了宁府,进了车门,到了东边小角门前下了车,进去见了贾珍之妻尤氏。
也未敢气高,殷殷勤勤叙过寒温,说了些闲话,方问道:\meng{何故兴致索然?}“今日怎么没见蓉大奶奶?”\ji{何不叫“秦钟的姐姐”?}尤氏说道:“他这些日子不知怎么着,经期有两个多月没来。
叫大夫瞧了,又说并不是喜。
那两日,到了下半天就懒待动,话也懒待说,眼神也发眩。
我说他:‘你且不必拘礼,早晚不必照例上来,你就好生养养罢。
就是有亲戚一家儿来,有我呢。
就有长辈们怪你,等我替你告诉。
’\ping{金荣的姑姑,即贾璜的妻子,是玉字辈,显然是贾蓉妻子秦可卿这个草字辈的长辈,这里的“长辈们怪你”莫非是暗指璜大奶奶这个兴师问罪的举动?}连蓉哥我都嘱咐了,我说:‘你不许累掯他,\zhu{累(音“勒”)掯(音“肯”,四声):亦作“勒掯”,强制、逼勒的意思。
}不许招他生气,叫他静静的养养就好了。
\meng{只一丝不露。
}他要想什么吃,只管到我这里取来。
倘或我这里没有,只管望你琏二婶子那里要去。
倘或他有个好和歹,你再要娶这么一个媳妇,这么个模样儿,这么个性情的人儿,打着灯笼也没地方找去。
’\ji{还有这么个好小舅子。
}他这为人行事,那个亲戚,那个一家的长辈不喜欢他?所以我这两日好不烦心,焦的我了不得。
偏偏今日早晨他兄弟来瞧他,谁知那小孩子家不知好歹,看见他姐姐身上不大爽快,就有事也不当告诉他,别说是这么一点子小事,就是你受了一万分的委曲,也不该向他说才是。
谁知他们昨儿学房里打架,不知是那里附学来的一个人欺侮了他了。
\ji{眼前竟像不知者。
}\meng{文笔之妙,妙至于此。
本是璜大奶奶不忿来告,又偏从尤氏口中先出,确是秦钟之语,且是情理必然,形势逼近。
孙悟空七十二变,未有如此灵巧活跳。
}里头还有些不干不净的话,都告诉了他姐姐。
婶子,你是知道那媳妇的:虽则见了人有说有笑,会行事儿,他可心细,心又重,不拘听见个什么话儿,都要度量个三日五夜才罢。
这病就是打这个秉性上头思虑出来的。
\ping{秦可卿是秦业收养的弃婴,出身寒微,可能因高嫁入贾府而不得不事事留心处处在意。
}今儿听见有人欺负了他兄弟,又是恼,又是气。
恼的是那群混帐狐朋狗友的扯是搬非、调三惑四那些人;气的是他兄弟不学好,不上心念书,以致如此学里吵闹。
他听了这事,今日索性连早饭也没吃。
我听见了,我方到他那边安慰了他一会子,又劝解了他兄弟一会子。
我叫他兄弟到那府里去找宝玉去了,我才看着他吃了半盏燕窝汤,我才过来了。
婶子,你说我心焦不心焦?\meng{这会子金氏听了这话,心里当如何料理?实在令人悔杀从前高兴。
天下事不得不预为三思,先为防渐。
}况且如今又没个好大夫,我想到他这病上,我心里倒像针扎似的。
你们知道有什么好大夫没有?”\meng{作无意相问语,是逼近一分,非有此一句,则金氏犹不免当为分诉。
一逼之下,实无可赘之词。
}\par
金氏听了这半日话,把方才在他嫂子家的那一团要向秦氏理论的盛气,早吓的都丢在爪洼国去了。
\zhu{爪洼国:古代南洋国名,今属印度尼西亚。
明清时,常用以喻指极遥远的地方。
}\ji{又何必为金母着急。
\ping{金母尚且为金荣上学,不仅能省钱还能挣钱而窃喜,何必为她着急。}
}听见尤氏问他有知道好大夫的话,连忙答道:“我们这么听着,实在也没见人说有个好大夫。
如今听起大奶奶这个来,定不得还是喜呢。
嫂子倒别教人混治。
倘或认错了,这可是了不得的。
”尤氏道:“可不是呢。
”正是说话间,贾珍从外进来,见了金氏,便向尤氏问道:“这不是璜大奶奶么?”金氏向前给贾珍请了安。
贾珍向尤氏说道:“让这大妹妹吃了饭去。
”贾珍说着话,就过那屋里去了。
金氏此来,原要向秦氏说说秦钟欺负了他侄儿的事,听见秦氏有病,不但不能说,亦且不敢提了。
况且贾珍尤氏又待的很好,反转怒为喜,\ping{怒气冲天的前来讨要说法,被哄笑了回去,最初要替自己娘家侄儿金荣出头,也存了炫耀人脉的心思,此条人脉比侄儿更重要些。
}又说了一会子话儿,方家去了。
\meng{金氏何面目再见江东父老?然而如金氏者,世不乏其人。
}\par
金氏去后,贾珍方过来坐下,问尤氏道:“今日他来,有什么说的事情么?”尤氏答道:“倒没说什么。
一进来的时候,脸上倒像有些着了恼的气色似的,及说了半天话,又提起媳妇这病,他倒渐渐的气色平定了。
你又叫让他吃饭,他听见媳妇这么病,也不好意思只管坐着,又说了几句闲话儿就去了,倒没求什么事。
如今且说媳妇这病,你到那里寻一个好大夫来与他瞧瞧要紧,可别耽误了。
现今咱们家走的这群大夫,那里要得?\meng{医毒。
非止近世,从古有之。
}一个个都是听着人的口气儿,人怎么说,他也添几句文话儿说一遍。
可倒殷勤的很,三四个人一日轮流着倒有四五遍来看脉。
他们大家商量着立个方子,吃了也不见效,倒弄得一日换四五遍衣裳,坐起来见大夫,其实于病人无益。
”贾珍说道:“可是。
这孩子也糊涂,何必脱脱换换的,倘再着了凉,更添一层病,那还了得。
衣裳任凭是什么好的,可又值什么,孩子的身子要紧,就是一天穿一套新的,也不值什么。
我正进来要告诉你:方才冯紫英来看我,他见我有些抑郁之色,问我是怎么了。
我才告诉他说,媳妇忽然身子有好大的不爽快,因为不得个好太医,断不透是喜是病,又不知有妨碍无妨碍,所以我这两日心里着实着急。
冯紫英因说起他有一个幼时从学的先生,姓张名友士,学问最渊博的,更兼医理极深,且能断人的生死。
\ji{未必能如此。
}\meng{举荐人的通套,多是如此说。
}今年是上京给他儿子来捐官,\zhu{捐官:封建时代向政府纳钱粮买官,叫捐官。
捐资买官制,始于秦代。
《史记·秦始皇本纪》:“百姓内(纳)粟千石,拜爵一级。
”西汉景帝时晁错建议,明令纳粟可以拜爵赎罪,以扩大国库收入。
见晁错《论贵粟疏》。
迄于清代,捐官之制尤滥,用钱买官,已成当时重要的入仕途径。
}现在他家住着呢。
这么看来,竟是合该媳妇的病在他手里除灾亦未可知。
我即刻差人拿我的名帖请去了。
\meng{父母之心,昊天罔极。
}今日倘或天晚了不能来,明日想必一定来。
况且冯紫英又即刻回家亲自去求他,务必叫他来瞧瞧。
等这个张先生来瞧了再说罢。
”\par
尤氏听了,心中甚喜,因说道:“后日是太爷的寿日,到底怎么办?”贾珍说道:“我方才到了太爷那里去请安,兼请太爷来家来受一受一家子的礼。
太爷因说道:‘我是清净惯了的,我不愿意往你们那是非场中去闹去。
你们必定说是我的生日,要叫我去受众人些头,莫过你把我从前注的《阴骘文》给我令人好好的写出来刻了,\zhu{《阴骘(音“治”)文》:相传为文昌帝君所作,是一篇宣扬因果报应等迷信思想的“劝善”文字。
文昌帝君:又称梓潼帝君,姓张名亚子,道家说他死后成为掌握文昌府事和人间禄籍(科举等事)之神,所以元代封他为文昌帝君。
见《明史·礼志》。
}比叫我无故受众人的头还强百倍呢。
倘或后日这两日一家子要来,你就在家里好好的款待他们就是了。
也不必给我送什么东西来,连你后日也不必来,你要心中不安,你今日就给我磕了头去。
\meng{将写可卿之好事多虑。
至于天生之文中,转出好清静之一番议论,清新醒目,立见不凡。
}倘或后日你要来,又跟随多少人来闹我,我必和你不依。
’如此说了又说,后日我是再不敢去的了。
且叫来升来,吩咐他预备两日的筵席。
”尤氏因叫人叫了贾蓉来:“吩咐来升照旧例预备两日的筵席,要丰丰富富的。
你再亲自到西府里去请老太太、大太太、二太太和你琏二婶子来逛逛。
你父亲今日又听见一个好大夫,业已打发人请去了,想必明日必来。
你可将他这些日子的病症细细的告诉他。
”\par
贾蓉一一的答应着出去了。
正遇着方才去冯紫英家请那先生的小子回来了,因回道:“奴才方才到了冯大爷家,拿了老爷的名帖请那先生去。
那先生说道:‘方才这里大爷也向我说了。
但是今日拜了一天的客,才回到家,此时精神实在不能支持,就是去到府上也不能看脉。
’他说等调息一夜,明日务必到府。
\meng{医生多是推三阻四,拿腔作调。
}他又说,他‘医学浅薄,本不敢当此重荐,因	我们冯大爷和府上的大人既已如此说了,又不得不去,你先替我回明大人就是了。
大人的名帖实不敢当。
’仍叫奴才拿回来了。
哥儿替奴才回一声儿罢。
”贾蓉转身复进去,回了贾珍尤氏的话,方出来叫了来升来,吩咐他预备两日的筵席的话。
来升听毕,自去照例料理。
不在话下。
\par
且说次日午间,人回道:“请的那张先生来了。
”贾珍遂延入大厅坐下。
\zhu{延:引进,迎接。
}茶毕,方开言道:“昨承冯大爷示知老先生人品学问,又兼深通医学,小弟不胜钦仰之至。
”张先生道:“晚生粗鄙下士,本知见浅陋,昨因冯大爷示知,大人家第谦恭下士,又承呼唤,敢不奉命。
但毫无实学,倍增颜汗。
”\zhu{颜汗:或作“汗颜”。
因受到称许恭维而羞愧得脸上出汗,实即“惭愧”一词的形象化表述。
颜,颜面。
}贾珍道:“先生何必过谦。
就请先生进去看看儿妇,仰仗高明,以释下怀。
”于是,贾蓉同了进去。
到了贾蓉居室,见了秦氏,向贾蓉说道:“这就是尊夫人了?”贾蓉道:“正是。
请先生坐下,让我把贱内的病症说一说再看脉如何?”那先生道:“依小弟的意思,竟先看过脉再说的为是。
我是初造尊府的,本也不晓得什么,但是我们冯大爷务必叫小弟过来看看,小弟所以不得不来。
如今看了脉息,看小弟说的是不是,再将这些日子的病势讲一讲,大家斟酌一个方儿,可用不可用,那时大爷再定夺。
”贾蓉道:“先生实在高明,如今恨相见之晚。
就请先生看一看脉息,可治不可治,以便使家父母放心。
”于是家下媳妇们捧过大迎枕来,\zhu{迎枕:中医切脉时,垫在病人手背下的小枕,亦作“迎手”。
}一面给秦氏拉着袖口,露出脉来。
先生方伸手按在右手脉上,调息了至数,\zhu{调息了至数:中医诊脉,医生先要稳定自己的呼吸,叫作“调息”。
一呼一吸叫作一息。
正常人一息间脉搏跳动的次数叫作“至数”。
调息了至数,指医生诊病时先调整好自己的呼吸,然后诊察病人在医生一息间的脉动次数。
}宁神细诊了有半刻的工夫,方换过左手,亦复如是。
诊毕脉息,说道:“我们外边坐罢。
”\par
贾蓉于是同先生到外间房里床上坐下,一个婆子端了茶来。
贾蓉道:“先生请茶。
”于是陪先生吃了茶,遂问道:“先生看这脉息,还治得治不得?”先生道:“看得尊夫人这脉息:左寸沉数,左关沉伏,右寸细而无力,右关需而无神。
其左寸沉数者,乃心气虚而生火;左关沉伏者,乃肝家气滞血亏。
右寸细而无力者,乃肺经气分太虚;右关需而无神者,乃脾土被肝木克制。
心气虚而生火者,应现经期不调,夜间不寐。
肝家血亏气滞者,必然肋下疼胀,月信过期,心中发热。
肺经气分太虚者,头目不时眩晕,寅卯间必然自汗,如坐舟中。
脾土被肝木克制者,必然不思饮食,精神倦怠,四肢酸软。
据我看这脉息,应当有这些症候才对。
或以这个脉为喜脉,则小弟不敢从其教也。
”\zhu{切脉是中医“望、问、闻、切”四诊之一。
医者从病人诊脉部位的不同脉象,可以诊断出病症之所在、深浅和性质。
医者以食指、中指、无名指三指切按病人两手桡动脉的“寸、关、尺”三个诊脉部位,中指所按桡骨茎突处为“关”、其前近腕端为“寸”、其后近肘端为“尺”。
两手寸、关、尺,共六部脉,各自反映不同脏腑的病情。
如左手:寸为心;关为肝;尺为肾。
右手:寸为肺;关为脾胃;尺为命门。
脉象种类甚多。
在较常见的二十八脉中,这里提到的是:沉脉:脉来轻取不应,重按始得,主病在里。
数脉:至数比正常人多,一息六至以上,主热症;数而无力为虚热。
伏脉:甚于沉脉,重按着骨始得;见于邪气内闭,痰饮积聚等症。
细脉:脉细如丝;见于气血虚亏等症。
需脉:“需”当为“濡”,“濡脉”即软脉,轻按可触知,重按反不明显;见于病邪滞留,气血不能运化等症。
中医又以五脏与五行相配:即肝为木;心为火;脾为土;肺为金;肾为水。
五者之间又有相生、相克之说。
如木可克土,即:肝有病可以克制脾胃的功能,故有“脾土被肝木克制”之语。
月信:月经。
寅卯间:我国古时以十二地支代表十二时辰。
寅时为上午三至五时,卯时为五至七时。
寅卯间约指凌晨五时前后这段时间。
喜脉:妇女怀孕后的脉象。
}旁边一个贴身伏侍的婆子道:“何尝不是这样呢。
真正先生说的如神,倒不用我们告诉了。
如今我们家里现有好几位太医老爷瞧着呢,都不能的当真切的这么说。
有一位说是喜,有一位说是病,这位说不相干,那位说怕冬至,总没有个准话儿。
求老爷明白指示指示。
”\par
那先生笑\meng{说是了,不觉笑,描出神情跳跃,如见其人。
}道:“大奶奶这个症候,可是那众位耽搁了。
要在初次行经的日期就用药治起来,不但断无今日之患,而且此时已全愈了。
如今既是把病耽误到这个地位,也是应有此灾。
依我看来,这病尚有三分治得。
吃了我的药看,若是夜里睡的着觉,那时又添了二分拿手了。
据我看这脉息:大奶奶是个心性高强聪明不过的人,聪明忒过,则不如意事常有,不如意事常有,则思虑太过。
\ping{联系到秦可卿淫丧天香楼的结局,因何而思虑太过呢?一个假设是,秦可卿怀疑自己可能的身孕是公公贾珍搞得,担心生下乱伦的证据。
}此病是忧虑伤脾,肝木忒旺,经血所以不能按时而至。
大奶奶从前的行经的日子问一问,断不是常缩,必是常长的。
\meng{恐不合其方,又加一番议论,一为合方药,一为夭亡症,无一字一句不前后照应者。
}是不是?”这婆子答道:“可不是,从没有缩过,或是长两日三日,以至十日都长过。
”先生听了道:“妙啊!这就是病源了。
从前若能够以养心调经之药服之,何至于此。
这如今明显出一个水亏木旺的症候来。
待用药看看。
”于是写了方子,递与贾蓉,上写的是:\par
\hop
益气养荣补脾和肝汤\par
人参\smalltext{二钱}\quad 白术\smalltext{二钱土炒}\quad 云苓\smalltext{三钱}\quad 熟地\smalltext{四钱}\par
归身\smalltext{二钱酒洗}\quad 白芍\smalltext{二钱}\quad 川芎\smalltext{钱半}\quad 黄芪\smalltext{三钱}\par
香附米\smalltext{二钱制\zhu{制:加工,这里指用炮炒等法炼成中药。}}\quad 醋柴胡\smalltext{八分}\quad 怀山药\smalltext{二钱炒}\quad 真阿胶\smalltext{二钱蛤粉炒}\par
延胡索\smalltext{钱半酒炒}\quad 炙甘草\smalltext{八分}\par
引用建莲子七粒去心\quad 红枣二枚\par
\zhu{引:即“药引”,指处方中能引药力达到病变部位的药物,是中医方剂中“君、臣、佐、使”四个部分的“使”的俗称,也叫“引经报使”或“引经药”。}\par
\hop
贾蓉看了,说:“高明的很。
还要请教先生,这病与性命终久有妨无妨?”先生笑道:“大爷是最高明的人。
人病到这个地位,非一朝一夕的症候,吃了这药也要看医缘了。
依小弟看来,今年一冬是不相干的。
总是过了春分,就可望全愈了。
”贾蓉也是个聪明人,也不往下细问了。
于是贾蓉送了先生去了,方将这药方子并脉案都给贾珍看了,说的话也都回了贾珍并尤氏了。
尤氏向贾珍说道:“从来大夫不像他说的这么痛快,想必用的药也不错。
”贾珍道:“人家原不是混饭吃、久惯行医的人。
因为冯紫英我们好,他好容易求了他来了。
既有这个人,媳妇的病或者就能好了。
他那方子上有人参,就用前日买的那一斤好的罢。
”贾蓉听毕话,方出来叫人打药去煎给秦氏吃。
不知秦氏服了此药病势如何,下回分解。
\par
\qi{总评:欲速可卿之死,故先有恶奴之凶顽,\zhu{恶奴当是指醉骂宁府爬灰之事的焦大。}而后及以秦钟来告,层层克入,
\zhu{克:完成。《春秋·宣公八年》:“日中而克葬。”}
点露其用心过当,种种文章逼之。
虽贫女得居富室,诸凡遂心,终有不能不夭亡之道。
我不知作者于着笔时何等妙心绣口,能道此无碍法语,
\zhu{
无碍:佛家指心里自在通达,没有障隔为无碍。
法语:讲说佛法之言。
}
令人不禁眼花撩乱。
}
\dai{019}{可卿病重}
\dai{020}{张太医论病细穷源}
\sun{p10-1}{金寡妇贪利权受辱,张太医论病细穷源}{图左侧:金荣母亲将学堂之事告知小姑贾璜妻金氏,金氏听后愤愤不平,到宁府中来要评理,见尤氏说到秦氏有病,也就不好再说什么。
图右侧:秦氏本来病重,再加上因兄弟在家塾中挨打,又气又恼,请了数人医治无效。
情急之下,请来冯紫英引荐张友士前来看病。
}