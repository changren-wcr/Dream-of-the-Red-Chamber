\chapter{寿怡红群芳开夜宴 \quad 死金丹独艳理亲丧}
\qi{此书写世人之富贵子弟易流邪鄙,其作长上者,有不能稽查之处,如宝玉之夜宴,始见之文雅韵致,细思之,何事生端不基于此?更能写贾蓉之恶赖无耻,亦世家之必有者,读者当以“三人行必有我师”之说为念,方能领会作者之用意也。
戒之!}\par
话说宝玉回至房中洗手,因与袭人商议:“晚间吃酒,大家取乐,不可拘泥。
如今吃什么,好早说给他们备办去。
”袭人笑道:“你放心,我和晴雯、麝月、秋纹四个人,每人五钱银子,共是二两。
芳官、碧痕、小燕、四儿四个人,每人三钱银子,他们有假的不算,共是三两二钱银子,早已交给了柳嫂子,预备四十碟果子。
我和平儿说了,已经抬了一坛好绍兴酒藏在那边了。
我们八个人单替你过生日。
”宝玉听了,喜的忙说:“他们是那里的钱,不该叫他们出才是。
”晴雯道:“他们没钱,难道我们是有钱的!这原是各人的心。
那怕他偷的呢,只管领他们的情就是。
”宝玉听了,笑说:“你说的是。
”袭人笑道:“你一天不挨他两句硬话村你,\zhu{村:粗俗。
这里作动词,顶撞之意。
}你再过不去。
”晴雯笑道:“你如今也学坏了,专会架桥拨火儿。
”\zhu{架桥拨火儿:从旁怂恿挑拨促成别人吵嘴打架。
架桥:比喻勾起双方矛盾。
拨火:比喻拨人心火,使之动气。
}说着,大家都笑了。
宝玉说:“关院门罢。
”袭人笑道:“怪不得人说你是‘无事忙’,这会子关了门,人倒疑惑,越性再等一等。
”宝玉点头,因说:“我出去走走,四儿舀水去,小燕一个跟我来罢。
”说着,走至外边,因见无人,便问五儿之事。
小燕道:“我才告诉了柳嫂子,他倒喜欢的很。
只是五儿那夜受了委屈烦恼,回家去又气病了,那里来得。
只等好了罢。
”宝玉听了,不免后悔长叹,因又问:“这事袭人知道不知道?”小燕道:“我没告诉,不知芳官可说了不曾。
”宝玉道:“我却没告诉过他,也罢,等我告诉他就是了。
”说毕,复走进来,故意洗手。
\par
已是掌灯时分,听得院门前有一群人进来。
大家隔窗悄视,果见林之孝家的和几个管事的女人走来,前头一人提着大灯笼。
晴雯悄笑道:“他们查上夜的人来了。
这一出去,咱们好关门了。
”只见怡红院凡上夜的人都迎了出去,林之孝家的看了不少。
林之孝家的吩咐:“别耍钱吃酒,放倒头睡到大天亮。
我听见是不依的。
”众人都笑说:“那里有那样大胆子的人。
”林之孝家的又问:“宝二爷睡下了没有?”众人都回不知道。
袭人忙推宝玉。
宝玉靸了鞋,便迎出来,笑道:“我还没睡呢。
妈妈进来歇歇。
”又叫:“袭人倒茶来。
”林之孝家的忙进来,笑说:“还没睡?如今天长夜短了,该早些睡,明儿起的方早。
不然到了明日起迟了,人笑话说不是个读书上学的公子了,倒像那起挑脚汉了。
”\zhu{挑脚汉:挑夫,旧时指以给人挑货物、行李为业的人。
}
说毕,又笑。
宝玉忙笑道:“妈妈说的是。
我每日都睡的早,妈妈每日进来可都是我不知道的,已经睡了。
今儿因吃了面怕停住食,所以多顽一会子。
”林之孝家的又向袭人等笑说:“该潗些个普洱茶吃。
”\zhu{潗茶:同“沏茶”。
普洱茶:产于云南普洱一带的名茶,多压制成团状。
}袭人晴雯二人忙笑说:“潗了一盄子女儿茶,\zhu{盄:音“吊”,吊子,又称铫子,一种烧水或熬煮食物的器皿。
女儿茶:泰山附近采青桐芽当饮料,号女儿茶。
一说女儿茶即为普洱茶之一种。
}已经吃过两碗了。
大娘也尝一碗,都是现成的。
”说着,晴雯便倒了一碗来。
\par
林之孝家的又笑道:“这些时我听见二爷嘴里都换了字眼,赶着这几位大姑娘们竟叫起名字来。
虽然在这屋里,到底是老太太、太太的人,还该嘴里尊重些才是。
若一时半刻偶然叫一声使得,若只管叫起来,怕以后兄弟侄儿照样,便惹人笑话,说这家子的人眼里没有长辈。
”宝玉笑道:“妈妈说的是。
我原不过是一时半刻的。
”袭人晴雯都笑说:“这可别委屈了他。
直到如今,他可姐姐没离了口。
不过顽的时候叫一声半声名字,若当着人却是和先一样。
”林之孝家的笑道:“这才好呢,这才是读书知礼的。
越自己谦越尊重,别说是三五代的陈人,现从老太太、太太屋里拨过来的,便是老太太、太太屋里的猫儿狗儿,轻易也伤他不的。
这才是受过调教的公子行事。
”说毕,吃了茶,便说:“请安歇罢,我们走了。
”宝玉还说:“再歇歇。
”那林之孝家的已带了众人,又查别处去了。
\par
这里晴雯等忙命关了门,进来笑说:“这位奶奶那里吃了一杯来了,唠三叨四的,又排场了我们一顿去了。
”麝月笑道:“他也不是好意的,
少不得也要常提着些儿。
也隄防着怕走了大褶儿的意思。
”\zhu{隄防:即提防。
}说着,一面摆上酒果。
袭人道:“不用围桌,咱们把那张花梨圆炕桌子放在炕上坐,又宽绰,
\zhu{绰:宽;不狭窄。}
又便宜。
”说着,大家果然抬来。
麝月和四儿那边去搬果子,用两个大茶盘做四五次方搬运了来。
两个老婆子蹲在外面火盆上筛酒。
宝玉说:“天热,咱们都脱了大衣裳才好。
”众人笑道:“你要脱你脱,我们还要轮流安席呢。
”\zhu{安席:旧时宴席入座时主人对宾客的一套礼节,叫安席。
}宝玉笑道:“这一安就安到五更天了。
知道我最怕这些俗套子,在外人跟前不得已的,这会子还怄我就不好了。
”众人听了,都说:“依你。
”于是先不上坐,且忙着卸妆宽衣。
\ji{凡吃酒从未先如此者,此独怡红风俗。
故王夫人云“他行事总是与世人两样的”,知子莫过母也。
}\par
一时将正装卸去,头上只随便挽着䰖儿,\zhu{䰖:音“钻”三声。䰖儿:也写作“纂儿”。
妇女的发髻。
}身上皆是长裙短袄。
宝玉只穿着大红棉纱小袄子,下面绿绫弹墨袷裤,\zhu{弹墨:以纸剪镂空图案覆于织品上,用墨色或其它颜色弹或喷成各种图案花样,叫弹墨。
袷:同“夹”。
袷裤:即“夹裤”,有面有里,中间不衬垫絮类的裤子。
}散着裤脚,倚着一个各色玫瑰芍药花瓣装的玉色夹纱新枕头,\zhu{玉色:淡青如玉的颜色。
}和芳官两个先划拳。
当时芳官满口嚷热,\ji{余亦此时太热了,恨不得一冷。
既冷时思此热,果然一梦矣。
}只穿着一件玉色红青酡绒三色缎子斗的水田小夹袄,\zhu{
红青:略泛微红的黑色。
酡:音“驼”,酒后脸上出现的红晕。
斗:亦作“逗”,这里指两种以上的色彩或衣料拼接一起组成图案。
水田:即“水田衣”,是用多种颜色的零碎衣料剪成小方块,或用两块不同颜色的三角形拼成方形,缝到一起做成的衣服,形似块块水田,故名。
夹袄:有面有里,中间不衬垫絮类的袄。
}束着一条柳绿汗巾,底下是水红撒花夹裤,\zhu{
水红:比粉红略深而鲜艳。
夹裤:有面有里,中间不衬垫絮类的裤子。
}也散着裤腿。
头上眉额编着一圈小辫,总归至顶心,结一根鹅卵粗细的总辫,拖在脑后。
右耳眼内只塞着米粒大小的一个小玉塞子,左耳上单带着一个白果大小的硬红镶金大坠子,越显的面如满月犹白,眼如秋水还清。
引的众人笑说:“他两个倒像是双生的弟兄两个。
”袭人等一一的斟了酒来,说:“且等等再划拳,虽不安席,每人在手里吃我们一口罢了。
”于是袭人为先,端在唇上吃了一口,馀依次下去,一一吃过,大家方团圆坐定。
小燕四儿因炕沿坐不下,便端了两张椅子,近炕放下。
那四十个碟子,皆是一色白粉定窑的,不过只有小茶碟大,里面不过是山南海北,中原外国,或干或鲜,或水或陆,天下所有的酒馔果菜。
宝玉因说:“咱们也该行个令才好。
”袭人道:“斯文些的才好,别大呼小叫,惹人听见。
二则我们不识字,可不要那些文的。
”麝月笑道:“拿骰子咱们抢红罢。
”\zhu{抢红:掷骰为戏,以得红点多少定输赢,叫抢红。
}宝玉道:“没趣,不好。
咱们占花名儿好。
”晴雯笑道:“正是早已想弄这个顽意儿。
”袭人道:“这个顽意虽好,人少了没趣。
”小燕笑道:“依我说,咱们竟悄悄的把宝姑娘林姑娘请了来顽一回子,到二更天再睡不迟。
”\ping{宝林二位姑娘请来之后,本来是众丫鬟给宝玉过生日的宴会,估计众丫鬟又成了配角。
}袭人道:“又开门喝户的闹,倘或遇见巡夜的问呢?”宝玉道:“怕什么,咱们三姑娘也吃酒,再请他一声才好。
还有琴姑娘。
”众人都道:“琴姑娘罢了,他在大奶奶屋里,叨登的大发了。
”
\zhu{
叨登:亦作“叨蹬”。
噜嗦,找麻烦。
}
宝玉道:“怕什么,你们就快请去。
”小燕四儿都得不了一声,\zhu{得不了:义同“巴不得”,迫切盼望。
}二人忙命开了门,分头去请。
\par
晴雯、麝月、袭人三人又说:“他两个去请,只怕宝林两个不肯来,须得我们请去,死活拉他来。
”于是袭人晴雯忙又命老婆子打个灯笼,二人又去。
果然宝钗说夜深了,黛玉说身上不好,他二人再三央求说:“好歹给我们一点体面,略坐坐再来。
”探春听了却也欢喜。
因想:“不请李纨,倘或被他知道了倒不好。
”便命翠墨同了小燕也再三的请了李纨和宝琴二人,会齐,先后都到了怡红院中。
袭人又死活拉了香菱来。
炕上又并了一张桌子,方坐开了。
\par
宝玉忙说:“林妹妹怕冷,过这边靠板壁坐。
”\zhu{板壁:分隔房间的木板墙。
}又拿个靠背垫着些。
袭人等都端了椅子在炕沿下一陪。
黛玉却离桌远远的靠着靠背,因笑向宝钗、李纨、探春等道:“你们日日说人夜聚饮博,\zhu{饮博:饮酒博戏。
博戏:赌输赢、角胜负的游戏。
}今儿我们自己也如此,以后怎么说人。
”李纨笑道:“这有何妨。
一年之中不过生日节间如此,并无夜夜如此,这倒也不怕。
”说着,晴雯拿了一个竹雕的签筒来,里面装着象牙花名签子,摇了一摇,放在当中。
又取过骰子来,盛在盒内,摇了一摇,揭开一看,里面是五点,数至宝钗。
宝钗便笑道:“我先抓,不知抓出个什么来。
”说着,将筒摇了一摇,伸手掣出一根,大家一看,只见签上画着一支牡丹,题着“艳冠群芳”四字,下面又有镌的小字一句唐诗,道是:\par
\hop
任是无情也动人。
\par
\zhu{唐代罗隐《牡丹花》诗:
“似共东风别有因,绛罗高卷不胜春。若教解语应倾国,
任是无情也动人。芍药与君为近侍,芙蓉何处避芳尘。
可怜韩令功成后,辜负秾华过此身。”
韩令指韩弘,唐元和十四年曾为中书令。末联所咏之事,见《唐国史补》:“京城贵游尚牡丹三十馀年矣。
每春暮,车马若狂,以不耽玩为耻。……元和末,韩令始至长安,居第有之,遽命斫去。曰:‘吾岂效儿女子邪?’”
这里韩弘是借来比宝玉的。“功成”一词也常用以表达对宗教意识的“彻悟”。所以,皈依佛门、修炼得道等,都可以说“功德圆满”。
宝玉的“悬崖撒手”,正是一种斩断缠绵情意,不肯“效儿女子”之态的决绝行为;
而宝钗也就像被韩令所弃的牡丹一样,只能“辜负秾华”,寂寞地了却“此身”(太虚幻境中宝钗的曲子名《终身误》,也是这个意思)。
}
\par
\hop
又注着:“在席共贺一杯,此为群芳之冠,随意命人,不拘诗词雅谑,道一则以侑酒。
”\zhu{侑(侑音“又”)酒:劝酒。
}众人看了,都笑说:“巧的很,你也原配牡丹花。
”说着,大家共贺了一杯。
宝钗吃过,便笑说:“芳官唱一支我们听罢。
”芳官道:“既这样,大家吃门杯好听的。
”\zhu{门杯:对公杯而言,酒宴时用以敬酒、罚酒等公用的酒杯叫公杯,放在各人面前的酒杯叫门杯,也叫门前杯。
}于是大家吃酒。
芳官便唱:“寿筵开处风光好……”众人都道:“快打回去。
这会子很不用你来上寿,拣你极好的唱来。
”芳官只得细细的唱了一支《赏花时》:\zhu{《邯郸记》是汤显祖根据唐人沈既济《枕中记》传奇情节改编的戏曲,写吕洞宾下凡去度一人上天,代替何仙姑天门扫花之役,何仙姑好去参加蟠桃宴。吕洞宾到了邯郸(今属河北)客店,遇卢生,把神奇的磁枕给他睡,让他做了一场黄粱美梦后,把他带到仙界去执帚。
《赏花时》是曲调名。
剧本第三折“度世”中何仙姑所唱,她嘱吕洞宾速去速回,不要误期。
}\par
\hop
翠凤毛翎扎帚叉,闲踏天门扫落花。
\zhu{翎[líng]:鸟翅上、尾上的长羽毛。
这二句意为何仙姑扫花于天门,所执之帚叉乃用翠凤的翎毛所扎成。
}您看那风起玉尘沙。
\zhu{玉尘沙:天界并无尘土泥沙,有的也只是玉屑,所以叫玉尘沙。
}猛可的那一层云下,\zhu{猛可的:突然间。
}抵多少门外即天涯。
\zhu{“您看……天涯”三句:说风吹落天上碧桃花,不知要比云层之下人间好花开时辜负春光多少呢!何仙姑在天门外扫花,“门外即天涯”,还有怕错过了蟠桃宴的意思。
语出唐代刘禹锡《和令狐相公别牡丹》诗:“平章宅里一栏花,临到开时不在家;莫道两京非远别,春明门外即天涯。
”}您再休要剑斩黄龙一线儿差,\zhu{“斩黄龙”句:意思是嘱咐吕洞宾再也不要像斩黄龙那样冒冒失失地差点丢了性命。
吕洞宾因曾与黄龙禅师顶撞,被禅师打了一戒尺,一怒间,半夜祭起“降魔太阿神光宝剑”去斩黄龙,结果剑被收去,他亲身去取剑,也被押入魔岩,多亏他师父钟离权说情才得救。
见明代冯梦龙《醒世恒言·吕洞宾飞剑斩黄龙》。
}
再休向东老贫穷卖酒家。
\zhu{“再休向”句:意思是劝吕洞宾在路上不要贪酒误事。
东老:宋代湖州东林沈氏自称东老,家贫而好客,善酿酒,留饮时常使客醉。
见苏东坡《次韵回先生三首序》。
}您与俺眼向云霞。
\zhu{眼向云霞:只留意仙界的事,兼以云霞喻天上盛开的碧桃花。
}洞宾呵,您得了人可便早些儿回话;若迟呵,错教人留恨碧桃花。
\zhu{留恨碧桃花:因何仙姑此时已入仙班,吕洞宾入尘世度人来代她扫花。
这句的意思是若代替扫花的人来迟,耽误了何仙姑参加蟠桃宴的时间,那只好怨恨碧桃花了。
}\par
\ping{芳官唱“寿筵开处风光好”被众人“打回去”,却唱了一支看破红尘到天界“扫落花”的曲子,显然是对后文情节的隐喻。
意思是贾府终将破败,再无“寿筵风光”,芳官在第七十七回将出家,贾宝玉在佚稿中一度看破红尘,都与曲子内容相关。
全书以“花”象征大观园众女儿,这里却“扫落花”,当然也就是最后“花落水流红”、“一朝春尽红颜老,花落人亡两不知”了。
“留恨碧桃花”在唱词中意思是误了王母的蟠桃会,字面上当然也能和“花落人亡”的意象衔接。
卢生“黄粱一梦”的曲子,既有作者人生如梦的思想的流露,也可以看出他正确地预示封建大家族必将没落的深刻而周密的艺术构思。
}
\par
\hop
才罢。
宝玉却只管拿着那签,口内颠来倒去念“任是无情也动人”,听了这曲子,眼看着芳官不语。
湘云忙一手夺了,掷与宝钗。
宝钗又掷了一个十六点,数到探春。
探春笑道:“我还不知得个什么呢。
”伸手掣了一根出来,自己一瞧,便掷在地下,红了脸,笑道:“这东西不好,不该行这令。
这原是外头男人们行的令,许多混话在上头。
”众人不解,袭人等忙拾了起来,众人看上面是一枝杏花,那红字写着“瑶池仙品”四字,诗云:\par
\hop
日边红杏倚云栽。
\par
\zhu{
唐代高蟾《下第后上永崇高侍郎》诗:“天上碧桃和露种,日边红杏倚云栽。芙蓉生在秋江上,不向东风怨未开。”
花签上说“必得贵婿”暗示探春远嫁不归。以花在“江上”,点她离家时亲人“涕送江边望”;“不向东风怨未开”句,则与她所作的风筝谜诗中“莫向东风怨别离”的隐义完全一样。
}
\par
\hop
注云:“得此签者,必得贵婿,大家恭贺一杯,共同饮一杯。
”众人笑道:“我说是什么呢。
这签原是闺阁中取戏的,除了这两三根有这话的,并无杂话,这有何妨。
我们家已有了个王妃,难道你也是王妃不成。
大喜,大喜。
”说着,大家来敬。
探春那里肯饮,却被史湘云、香菱、李纨等三四个人强死强活灌了下去。
探春只命蠲了这个,再行别的,众人断不肯依。
湘云拿着他的手强掷了个十九点出来,便该李氏掣。
李氏摇了一摇,掣出一根来一看,笑道:“好极。
你们瞧瞧,这劳什子竟有些意思。
”众人瞧那签上,画着一枝老梅,是写着“霜晓寒姿”四字,那一面旧诗是:\par
\hop
竹篱茅舍自甘心。
\par
\zhu{宋代王淇《梅》诗:“不受尘埃半点侵,竹篱茅舍自甘心。只因误识林和靖,惹得诗人说到今。”
北宋诗人林逋,赐谥和靖先生,诗《山园小梅》中“疏影横斜水清浅,暗香浮动月黄昏”最著名。
李纨是封建时代寡欲守节妇女的典型,《梅》诗中前两句片来比她的操守。
三、四句说她后来得以荣耀,并非本意想占风情,而是受人“牵连”之故,恰如梅花本自处幽独,被林逋诗一赞,结果“枉与他人作笑谈”,倒弄得十分热闹。
作者对李纨的命运虽有同情叹愧,却并不赞扬标榜。
}
\par
\hop
注云:“自饮一杯,下家掷骰。
”李纨笑道:“真有趣,你们掷去罢。
我只自吃一杯,不问你们的废与兴。
”\ping{李纨不问旁人废与兴,体现了李纨的孤独寂寞,也反映了她冷漠无情。
}说着,便吃酒,将骰过与黛玉。
黛玉一掷,是个十八点,便该湘云掣。
湘云笑着,揎拳掳袖的伸手掣了一根出来。
\zhu{揎拳掳[lǔ]袖:即“揎拳捋袖”,伸出拳头,卷起衣袖。
形容粗野、准备动武的样子。
}大家看时,一面画着一枝海棠,题着“香梦沉酣”四字,那面诗道是:\par
\hop
只恐夜深花睡去。
\par
\zhu{
宋代苏轼《海棠》诗:“东风袅袅泛崇光,香雾空蒙月转廊。只恐夜深花睡去,故烧高烛照红妆。”
苏轼原诗是惜春光短促,好景难留,所以他连夜里都要点蜡烛赏花。湘云后来的遭遇正是如此:虽有洞房花烛照红妆新人之喜,可惜转眼就“云散高唐,水涸湘江”,春光别去了。
}
\par
\hop
黛玉笑道:“‘夜深’两个字,改‘石凉’两个字。
”众人便知他趣白日间湘云醉卧的事,\zhu{趣:取笑,打趣。
}都笑了。
湘云笑指那自行船与黛玉看,又说:“快坐上那船家去罢,别多话了。
”众人都笑了。
因看注云:“既云‘香梦沉酣’,掣此签者不便饮酒,只令上下二家各饮一杯。
”湘云拍手笑道:“阿弥陀佛,真真好签!”恰好黛玉是上家,宝玉是下家。
二人斟了两杯只得要饮。
宝玉先饮了半杯,瞅人不见,递与芳官,端起来便一扬脖。
黛玉只管和人说话,将酒全折在漱盂内了。
湘云便绰起骰子来一掷个九点,\zhu{绰:音“辍”,抓取。
}数去该麝月。
麝月便掣了一根出来。
大家看时,这面上一枝荼蘼花,\zhu{荼蘼:初夏开花,朵大色白,清香馥郁。
其枝条软垂,常倚架而生。
}
题着“韶华胜极”四字,那边写着一句旧诗,道是:\par
\hop
开到荼縻花事了。
\par
\zhu{
宋代王淇《春暮游小园》诗:“一从梅粉褪残妆,涂抹新红上海棠。开到荼蘼花事了,丝丝天棘出莓墙。”
荼縻春末开花,故苏轼有诗说:“荼縻不争春,寂寞开最晚。”
韶华胜极字面上是说好得很,实质上是指到了头。
“花事了”和“送春”使宝玉产生好景不长的预感。
}
\par
\hop
注云:“在席各饮三杯送春。
”麝月问怎么讲,宝玉愁眉忙将签藏了说:“咱们且喝酒。
”\ping{第十三回,三春去后诸芳尽,各自须寻各自门。
}说着,大家吃了三口,以充三杯之数。
麝月一掷个十九点,该香菱。
香菱便掣了一根并蒂花,题着“联春绕瑞”,那面写着一句诗,道是:\par
\hop
连理枝头花正开。
\par
\zhu{
宋代朱淑贞《落花》:“连理枝头花正开,妒花风雨便相催。愿教青帝长为主,莫遣纷纷落翠苔。”
连理枝:枝干连生在一起的草木,喻恩爱夫妻。
“妒花风雨便相催”指香菱被妒妇夏金桂摧残夭亡。
}
\par
\hop
注云:“共贺掣者三杯,大家陪饮一杯。
”香菱便又掷了个六点,该黛玉掣。
黛玉默默的想道:“不知还有什么好的被我掣着方好。
”一面伸手取了一根,只见上面画着一枝芙蓉,题着“风露清愁”四字,那面一句旧诗,道是:\par
\hop
莫怨东风当自嗟。
\par
\zhu{
宋代欧阳修《明妃曲·再和王介甫》诗:“汉宫有佳人,天子初未识;一朝随汉使,远嫁单于国。绝色天下无,一失难再得。虽能杀画工,于事竟何益!耳目所及尚如此,万里安能制夷狄!汉计诚已拙,女色难自夸。明妃去时泪,洒向枝上花;狂风日暮起,漂泊落谁家?红颜胜人多薄命,莫怨东风当自嗟。”
隐去原诗的前一句“红颜胜人多薄命”。
“明妃去时泪,洒向枝上花”与黛玉葬花吟场景相似。
“狂风日暮起”指贾府事败、宝玉出走那阵骤然而至的政治“狂风”。
“莫怨东风当自嗟”指黛玉的悲剧命运是因为悬心宝玉而不自惜多病之身,“怨”不得别人而该“自嗟”。第三回脂评引《论语》“求仁而得仁,又何怨!”
}
\par
\hop
注云:“自饮一杯,牡丹陪饮一杯。
”\ping{黛玉和宝钗双峰对峙,同等重要。
}众人笑说:“这个好极。
除了他,别人不配作芙蓉。
”\ping{第七十八回,宝玉作的《芙蓉女儿诔》,明为悼念晴雯,实则悼念黛玉。
}黛玉也自笑了。
于是饮了酒,便掷了个二十点,该着袭人。
袭人便伸手取了一支出来,却是一枝桃花,题着“武陵别景”四字,那一面旧诗写着道是:\par
\hop
桃红又是一年春。
\par
\zhu{
宋代谢枋得《庆全庵桃花》诗:“寻得桃源好避秦,桃红又是一年春。花飞莫遣随流水,怕有渔郎来问津。”
避秦:逃避秦二世时由苛政引起的社会动乱。《桃花源记》中武陵人入桃花源,见到“避秦时乱,率妻子邑人来此绝境”的山中人。
“寻得桃源好避秦”指贾府没落之时她只好去另找归宿。“桃红又是一年春”说袭人嫁给蒋玉函好比两度春风。
第三四句以“渔郎”追寻流水中的落花来到桃花源,比喻“蒋玉函”在酒席上知宝玉有佳丽名袭人者后,追寻袭人的下落并最终结为夫妻。
}
\par
\hop
注云:“杏花陪一盏,坐中同庚者陪一盏,\zhu{同庚:谓年龄相同。
}同辰者陪一盏,\zhu{同辰:谓生日相同。
}同姓者陪一盏。
”众人笑道:“这一回热闹有趣。
”大家算来,香菱、晴雯、宝钗三人皆与他同庚,黛玉与他同辰,\zhu{第六十二回:宝玉笑指袭人道:“他和林妹妹是一日,所以他记的。
”}只无同姓者。
芳官忙道:“我也姓花,我也陪他一钟。
”于是大家斟了酒,黛玉因向探春笑道:“命中该着招贵婿的,你是杏花,快喝了,我们好喝。
”探春笑道:“这是个什么,大嫂子顺手给他一下子。
”李纨笑道:“人家不得贵婿反挨打,我也不忍的。
”说的众人都笑了。
\par
袭人才要掷,只听有人叫门。
老婆子忙出去问时,原来是薛姨妈打发人来了接黛玉的。
众人因问几更了,人回:“二更以后了,钟打过十一下了。
”宝玉犹不信,要过表来瞧了一瞧,已是子初初刻十分了。
\zhu{子初:二十三点。初刻:第一刻即零分到十五分钟。十分:初刻中的具体时刻,即十分钟。}
黛玉便起身说:“我可撑不住了,回去还要吃药呢。
”众人说:“也都该散了。
”袭人宝玉等还要留着众人。
李纨宝钗等都说:“夜太深了不像,
\zhu{不像:指言行超越常轨,不成话。}
这已是破格了。
”
袭人道:“既如此,每位再吃一杯再走。
”说着,晴雯等已都斟满了酒,每人吃了,都命点灯。
袭人等直送过沁芳亭河那边方回来。
\par
关了门,大家复又行起令来。
袭人等又用大钟斟了几钟,用盘攒了各样果菜与地下的老嬷嬷们吃。
彼此有了三分酒,便猜拳赢唱小曲儿。
\zhu{猜拳赢唱小曲:应该是猜拳(石头、剪刀、布)输了的唱小曲给赢了的听。
}
那天已四更时分,老嬷嬷们一面明吃,一面暗偷,酒坛已罄,众人听了纳罕,方收拾盥漱睡觉。
芳官吃的两腮胭脂一般,眉梢眼角越添了许多丰韵,身子图不得,\zhu{图不得:忍耐不住;支持不住。根据上下文,这里是因熬夜饮酒而酒醉困倦,难以维持清醒的意识。
}便睡在袭人身上,“好姐姐,心跳的很。
”袭人笑道:“谁许你尽力灌起来。
”小燕四儿也图不得,早睡了。
晴雯还只管叫。
宝玉道:“不用叫了,咱们且胡乱歇一歇罢。
”自己便枕了那红香枕,身子一歪,便也睡着了。
袭人见芳官醉的很,恐闹他唾酒,\zhu{唾酒:吐酒。
}
只得轻轻起来,就将芳官扶在宝玉之侧,由他睡了。
自己却在对面榻上倒下。
\ping{袭人在和宝玉偷试云雨并在王夫人面前稳固了自己的地位之后,反而刻意要和宝玉避嫌,免得引起担心儿子被勾引坏了的王夫人的猜忌。
袭人害怕芳官吐酒让自己麻烦,把潜在的麻烦交给了宝玉,而且很可能让芳官因此背上“狐狸精”的骂名。
}\par
大家黑甜一觉,不知所之。
及至天明,袭人睁眼一看,只见天色晶明,忙说:“可迟了。
”向对面床上瞧了一瞧,只见芳官头枕着炕沿上,睡犹未醒,连忙起来叫他。
宝玉已翻身醒了,笑道:“可迟了!”因又推芳官起身。
那芳官坐起来,犹发怔揉眼睛。
袭人笑道:“不害羞,你吃醉了,怎么也不拣地方儿乱挺下了。
”\ping{贼喊捉贼,明明是袭人故意把芳官扶到宝玉身边睡下。
袭人要么是开玩笑,要么是嫉妒芳官得宠,所以故意陷害不懂世故分寸的芳官。
}芳官听了,瞧了一瞧,方知道和宝玉同榻,忙笑的下地来,说:“我怎么吃的不知道了。
”宝玉笑道:“我竟也不知道了。
若知道,给你脸上抹些黑墨。
”说着,丫头进来伺候梳洗。
宝玉笑道:“昨儿有扰,今儿晚上我还席。
”袭人笑道:“罢罢罢,今儿可别闹了,再闹就有人说话了。
”宝玉道:“怕什么,不过才两次罢了。
咱们也算是会吃酒了,那一坛子酒,怎么就吃光了。
正是有趣,偏又没了。
”袭人笑道:“原要这样才有趣。
必至兴尽了,反无后味了。
昨儿都好上来了,晴雯连臊也忘了,我记得他还唱了一个。
”四儿笑道:“姐姐忘了,连姐姐还唱了一个呢。
在席的谁没唱过!”众人听了,俱红了脸,用两手握着笑个不住。
\ping{袭人刻意要通过指责芳官和晴雯的出格,凸显自己是光明正大的道德楷模,其实并非如此,甚至袭人自己做过更加出格的事,这只不过是因心虚而掩饰。
}\par
忽见平儿笑嘻嘻的走来,说亲自来请昨日在席的人:“今儿我还东,\zhu{还东:设席做东,还请别人。
}\zhu{第六十二回,大家偶然得知平儿和宝玉同一天生日,所以也为她过了生日,所以这时候平儿要回请大家。
}
短一个也使不得。
”众人忙让坐吃茶。
晴雯笑道:“可惜昨夜没他。
”平儿忙问:“你们夜里做什么来?”袭人便说:“告诉不得你。
昨儿夜里热闹非常,连往日老太太、太太带着众人顽也不及昨儿这一顽。
一坛酒我们都鼓捣光了,一个个吃的把臊都丢了,三不知的又都唱起来。
四更多天才横三竖四的打了一个盹儿。
”平儿笑道:“好,白和我要了酒来,也不请我,还说着给我听,气我。
”晴雯道:“今儿他还席,必来请你的,等着罢。
”平儿笑问道:“他是谁,谁是他?”晴雯听了,赶着笑打,说道:“偏你这耳朵尖,听得真。
”\ping{这里的“他”就是宝玉,因为大家伙给宝玉过生日,宝玉需要回请还席。
晴雯用“他”来称呼宝玉,反映了晴雯对宝玉不经意间流露的亲近随意,体现了晴雯对宝玉的情思。
这种超越了丫鬟身份的近乎于情侣之间的亲昵语言,被平儿挑了出来,惹得晴雯娇羞笑打平儿。
}平儿笑道:“这会子有事不和你说,我干事去了。
一回再打发人来请,一个不到,我是打上门来的。
”宝玉等忙留,他已经去了。
\par
这里宝玉梳洗了正吃茶,忽然一眼看见砚台底下压着一张纸,因说道:“你们这随便混压东西也不好。
”袭人晴雯等忙问:“又怎么了,谁又有了不是了?”宝玉指道:“砚台下是什么?一定又是那位的样子忘记了收的。
”\zhu{样子:花样,即绣花用的底样。
}晴雯忙启砚拿了出来,却是一张字帖儿,递与宝玉看时,原来是一张粉笺子,上面写着“槛外人妙玉恭肃遥叩芳辰”。
\ji{帖文亦蹈俗套之外。
}宝玉看毕,直跳了起来,忙问:“这是谁接了来的?也不告诉。
”袭人晴雯等见了这般,不知当是那个要紧的人来的帖子,忙一齐问:“昨儿谁接下了一个帖子?”四儿忙飞跑进来,笑说:“昨儿妙玉并没亲来,只打发个妈妈送来。
我就搁在那里,谁知一顿酒就忘了。
”众人听了,道:“我当谁的,这样大惊小怪。
这也不值的。
”宝玉忙命:“快拿纸来。
”当时拿了纸,研了墨,看他下着“槛外人”三字,自己竟不知回帖上回个什么字样才相敌。
只管提笔出神,半天仍没主意。
因又想:“若问宝钗去,他必又批评怪诞,不如问黛玉去。
”\par
想罢,袖了帖儿,径来寻黛玉。
刚过了沁芳亭,忽见岫烟颤颤巍巍的迎面走来。
\lie{四个俗字写出一个活跳美人,转觉别书中若干“莲步香尘”、“纤腰玉体”字样无味之甚。
}宝玉忙问:“姐姐那里去?”岫烟笑道:“我找妙玉说话。
”宝玉听了诧异,说道:“他为人孤癖,不合时宜,万人不入他目。
原来他推重姐姐,竟知姐姐不是我们一流的俗人。
”岫烟笑道:“他也未必真心重我,但我和他做过十年的邻居,只一墙之隔。
他在蟠香寺修炼,我家原寒素,赁房居住,就赁的是他庙里的房子,住了十年,无事到他庙里去作伴。
我所认的字都是承他所授。
我和他又是贫贱之交,又有半师之分。
因我们投亲去了,闻得他因不合时宜,权势不容,竟投到这里来。
如今又天缘凑合,我们得遇,旧情竟未易。
承他青目,更胜当日。
”宝玉听了,恍如听了焦雷一般,喜的笑道:“怪道姐姐举止言谈,超然如野鹤闲云,原来有本而来。
\zhu{本:根源,来源。
}正因他的一件事我为难,要请教别人去。
如今遇见姐姐,真是天缘巧合,求姐姐指教。
”说着,便将拜帖取与岫烟看。
岫烟笑道:“他这脾气竟不能改,竟是生成这等放诞诡僻了。
从来没见拜帖上下别号的,这可是俗语说的‘僧不僧,俗不俗,女不女,男不男’,成个什么道理。
”宝玉听说,忙笑道:“姐姐不知道,他原不在这些人中算,他原是世人意外之人。
因取我是个些微有知识的,\zhu{有知识:这里指有不同流俗的见识。
}方给我这帖子。
我因不知回什么字样才好,竟没了主意,正要去问林妹妹,可巧遇见了姐姐。
”\par
岫烟听了宝玉这话,且只顾用眼上下细细打量了半日,方笑道:“怪道俗语说的‘闻名不如见面’,又怪不得妙玉竟下这帖子给你,又怪不得上年竟给你那些梅花。
既连他这样,\ping{妙玉孤高自赏,却对宝玉青眼相加,不仅给了梅花,还写了书信。
宝玉也并不觉得妙玉放诞诡僻,反而很敬重妙玉。
既然两人确实愿意相交,岫烟就助一臂之力。
}少不得我告诉你原故。
他常说:‘古人中自汉晋五代唐宋以来皆无好诗,只有两句好,说道:纵有千年铁门槛,终须一个土馒头。
’
\zhu{
铁门槛:据传陈、隋间智永禅师书法绝妙,求书者络绎不绝,门槛都磨穿了,只得用铁皮包起来。
“铁门槛”由此借喻家道繁盛,也用来比喻生死界限。
土馒头:比喻坟丘。
这两句诗谓即使家道兴盛千年,也终会衰落消亡。也指人无论能活多久,终须一死。
}
所以他自称‘槛外之人’。
又常赞文是庄子的好,故又或称为‘畸人’。
\zhu{畸(音“基”)人:行事乖僻,与世俗礼仪悖谬的人。
《庄子·大宗师》载:桑户死,其友子反、琴张颜色不变,临尸而歌,“乖异人伦,不偶于俗”。
子贡问于孔子,孔子曰:“畸人者,畸于人而侔于天。
”是说畸人虽与世俗不合,但率性而为,与自然之理相通。
侔:音“谋”,相等,等同。
}他若帖子上是自称‘畸人’的,你就还他个‘世人’。
畸人者,他自称是畸零之人;你谦自己乃世中扰扰之人,他便喜了。
如今他自称‘槛外之人’,是自谓蹈于铁槛之外了;故你如今只下‘槛内人’,便合了他的心了。
”宝玉听了,如醍醐灌顶,\zhu{醍醐灌顶:佛家用语。
比喻向人灌输智慧佛性。
这里引申为经人指点顿然领悟的意思。
醍醐:音“提胡”,从牛乳中提炼出来的最精华成分,佛家用以比喻佛性和智慧。
}嗳哟了一声,方笑道:“怪道我们家庙说是‘铁槛寺’呢,原来有这一说。
姐姐就请,让我去写回帖。
”岫烟听了,便自往栊翠庵来。
宝玉回房写了帖子,上面只写“槛内人宝玉熏沐谨拜”几字,亲自拿了到栊翠庵,只隔门缝儿投进去便回来了。
\par
因又见芳官梳了头,挽起纂来,带了些花翠,忙命他改妆,又命将周围的短发剃了去,露出碧青头皮来,当中分大顶,\zhu{顶:头的最上部。
分大顶:应该是把头顶的头发从中间分开。
}又说:“冬天作大貂鼠卧兔儿带,\zhu{大貂鼠卧兔儿:貂皮做的一种帽子,样式如卧兔。
}脚上穿虎头盘云五彩小战靴,或散着裤腿,只用净袜厚底镶鞋。
”\zhu{净袜:指白色袜子。
镶鞋:一种镶嵌之鞋,亦作“厢鞋”,以黑缎作面,厚底,鞋帮用同色料子镶嵌成各种花卉图案。
}又说:“芳官之名不好,竟改了男名才别致。
”因又改作“雄奴”。
芳官十分称心,又说:“既如此,你出门也带我出去。
有人问,只说我和茗烟一样的小厮就是了。
”宝玉笑道:“到底人看的出来。
”芳官笑道:“我说你是无才的。
\ji{用芳官一骂,有趣。
}咱家现有几家土番,你就说我是个小土番儿。
\zhu{土番:古时称边境少数民族为番,俗呼为土番。
}况且人人说我打联垂好看,\zhu{联垂:辫子。
}你想这话可妙?”宝玉听了,喜出意外,忙笑道:“这却很好。
我亦常见官员人等多有跟从外国献俘之种,图其不畏风霜,鞍马便捷。
既这等,再起个番名,叫作‘耶律雄奴’。
‘雄奴’二音,又与匈奴相通,都是犬戎名姓。
\zhu{犬戎:中国古代西戎种族的别名。
}
况且这两种人自尧舜时便为中华之患,晋唐诸朝,深受其害。
幸得咱们有福,生在当今之世,大舜之正裔,圣虞之功德仁孝,赫赫格天,\zhu{格:到。
}同天地日月亿兆不朽,所以凡历朝中跳梁猖獗之小丑,到了如今竟不用一干一戈,皆天使其拱手俛头缘远来降。
\zhu{俛:同“俯”,低头。
缘:沿、循。
}我们正该作践他们,为君父生色。
”芳官笑道:“既这样着,你该去操习弓马,学些武艺,挺身出去拿几个反叛来,岂不进忠效力了。
何必借我们,你鼓唇摇舌的,自己开心作戏,却说是称功颂德呢。
”宝玉笑道:“所以你不明白。
如今四海宾服,八方宁静,千载百载不用武备。
咱们虽一戏一笑,也该称颂,方不负坐享升平了。
”芳官听了有理,二人自为妥贴甚宜。
宝玉便叫他“耶律雄奴”。
\par
究竟贾府二宅皆有先人当年所获之囚赐为奴隶,只不过令其饲养马匹,皆不堪大用。
湘云素习憨戏异常,他也最喜武扮的,每每自己束銮带、穿折袖,\zhu{銮:系在君王车驾上的铃铛,也指君王的车驾,亦借指君王。
銮带;装饰有响铃的带子。
折袖:俗称“马蹄袖”,袍褂袖口上接出一个半圆形袖头的服式,本为骑射时覆手御寒,后用作礼服,平时挽起,行礼时掸下。
}近见宝玉将芳官扮成男子,他便将葵官也扮了个小子。
那葵官本是常刮剔短发,好便于面上粉墨油彩,手脚又伶便,打扮了又省一层手。
\ping{
芳官和葵官女改男妆的时候,要剃短发露出头皮来。此种将周围头发剃去,只留颅后发,然后编结为辫的发式,是满族男子独特的传统。
}
李纨探春见了也爱,便将宝琴的荳官也就命他打扮了一个小童,头上两个丫髻,
\zhu{丫髻:女孩梳在头顶两边的发髻。}
短袄红鞋,只差了涂脸,便俨是戏上的一个琴童。
湘云将“葵官”改了换作“大英”,因他姓韦,便叫他作韦大英,方合自己的意思,暗有“惟大英雄能本色”之语,何必涂朱抹粉,才是男子。
荳官身量年纪皆极小,又极鬼灵,故曰荳官。
园中人也有唤他作“阿荳”的,也有唤作“炒豆子”的。
宝琴反说琴童书童等名太熟了,竟是荳字别致,便换作“荳童”。
\par
因饭后平儿还席,说红香圃太热,便在榆荫堂中摆了几席新酒佳肴。
\lie{榆荫中者,馀荫也。
兹既感灵,今故怀亲,所谓不失忠孝之大纲也。
}可喜尤氏又带了配凤\foot{此侍妾的名字,诸本在本回及第七十四、七十五回均作“佩凤”,在第七十一回则作“配凤”,现统一为“配凤”。
}、偕鸾二妾过来游玩。
这二妾亦是青年姣憨女子,不常过来的,今既入了这园,再遇见湘云、香菱、芳、蕊一干女子,所谓“方以类聚,物以群分”二语不错,\zhu{方以类聚:语本《易经.系辞上》:“方以类聚,物以群分,吉凶生矣。
”性质相近的东西常聚集在一起。
指具有相同特性或嗜好的物或人聚集在一起。
此处的“方”有两解,一解为道,如君子依正道而行,小人依恶道而行,同道者才会相聚;一解为“人”字的讹误,应为“人以类聚”,意为同类的人会聚集在一起。
无论何者,都是同类相聚的意思。
}只见他们说笑不了,也不管尤氏在那里,只凭丫鬟们去伏侍,且同众人一一的游顽。
一时到了怡红院,忽听宝玉叫“耶律雄奴”,把配凤、偕鸾、香菱三个人笑在一处,问是什么话,大家也学着叫这名字,又叫错了音韵,或忘了字眼,甚至于叫出“野驴子”来,引的合园中人凡听见者无不笑倒。
宝玉又见人人取笑,恐作践了他,忙又说:“海西福朗思牙,\zhu{福朗思牙:即法兰西。
}闻有金星玻璃宝石,他本国番语以金星玻璃名为‘温都里纳’。
\zhu{温都里纳:法文音译,意为内涵金星的棕黄色宝石,或仿照这种宝石所制的玻璃或陶瓷。
}如今将你比作他,就改名唤叫‘温都里纳’可好?”芳官听了更喜,说:“就是这样罢。
”因此又唤了这名。
众人嫌拗口,仍翻汉名,就唤“玻璃”。
\par
闲言少述,且说当下众人都在榆荫堂中以酒为名,大家顽笑,命女先儿击鼓。
平儿采了一枝芍药,大家约二十来人传花为令,热闹了一回。
因人回说:“甄家有两个女人送东西来了。
”探春和李纨尤氏三人出去议事厅相见,这里众人且出来散一散。
配凤、偕鸾两个去打秋千顽耍,\ji{大家千金不令作此戏,故写不及探春等人也。
}宝玉便说:“你两个上去,让我送。
”慌的配凤说:“罢了,别替我们闹乱子,倒是叫‘野驴子’来送送使得。
”宝玉忙笑说:“好姐姐们别顽了,没的叫人跟着你们学着骂他。
”偕鸾又说:“笑软了,怎么打呢。
掉下来栽出你的黄子来。
”\zhu{黄:指幼儿。
}
配凤便赶着他打。
\par
正顽笑不绝,忽见东府中几个人慌慌张张跑来说:“老爷宾天了。
”\zhu{宾天:古时专称帝王之死,后世泛称尊者的死亡。
}
众人听了,唬了一大跳,忙都说:“好好的并无疾病,怎么就没了?”家下人说:“老爷天天修炼,定是功行圆满,升仙去了。
”尤氏一闻此言,又见贾珍父子并贾琏等皆不在家,一时竟没个着己的男子来,\zhu{着己:亲近,贴心。
}未免忙了。
只得忙卸了妆饰,命人先到玄真观将所有的道士都锁了起来,等大爷来家审问。
一面忙忙坐车带了赖升一干家人媳妇出城。
又请太医看视到底系何病。
\par
大夫们见人已死,何处诊脉来,素知贾敬导气之术总属虚诞,\zhu{导气之术:亦作导引之术,是“导气令和,引体令柔”的意思。
原为我国古代锻炼身体和医疗疾病的一种方法,近似今之气功疗法。
后被道教披上神秘外衣,利用作“修仙”、“长生”的方术之一。
}更至参星礼斗,\zhu{参、礼:这里是“拜”的意思。
星、斗,指北斗星(一说,指太白金星(即金星)和北斗)。
参星礼斗:参拜北斗星。
道教称经常礼拜北斗星,有助于成仙,为道家炼丹的仪式。
}守庚申,\zhu{守庚申:道教迷信,说人腹中有一种怪物叫“三尸”(也叫“三彭”、“三虫”)专门伺察人的隐私过恶,每到庚申日就到天帝面前告发,减人禄命。
若人在庚申日不眠,“三尸”便不能上天告状,就可以长生不死。
见宋代张君房辑《云笈七签》。
}服灵砂,妄作虚为,过于劳神费力,反因此伤了性命的。
如今虽死,肚中坚硬似铁,面皮嘴唇烧的紫绛皱裂。
便向媳妇回说:“系玄教中吞金服砂,烧胀而殁。
”众道士慌的回说:“原是老爷秘法新制的丹砂吃坏事,小道们也曾劝说‘功行未到且服不得’,不承望老爷于今夜守庚申时悄悄的服了下去,便升仙了。
这恐是虔心得道,已出苦海,脱去皮囊,自了去也。
”尤氏也不听,只命锁着,等贾珍来发放,且命人去飞马报信。
一面看视这里窄狭,不能停放,横竖也不能进城的,\zhu{不能进城:可能因为老太妃国丧期间,贾家都要入朝随祭,根据礼制所以贾敬棺木不能进城。
}忙装裹好了,用软轿抬至铁槛寺来停放。
掐指算来,至早也得半月的工夫,贾珍方能来到。
目今天气炎热,实不得相待,遂自行主持,命天文生择了日期入殓。
\zhu{天文生:本为明、清时代钦天监官员的职称之一,主要掌管对星辰、晴雨、风雷、云霓等天象气候的观测与推算。
这里是指旧时以择日、占卜、看风水、选阴阳宅等迷信活动为职业的人,也称“阴阳先生”、“风水先生”或“堪舆先生”。
}寿木已系早年备下寄在此庙的,甚是便宜。
三日后便开丧破孝。
\zhu{开丧破孝:丧事开始并开始带孝。}
一面且做起道场来等贾珍。
\par
荣府中凤姐儿出不来,李纨又照顾姊妹,宝玉不识事体,只得将外头之事暂托了几个家中二等管事人。
贾㻞、贾珖、贾珩、贾璎、贾菖、贾菱等各有执事。
尤氏不能回家,便将他继母接来在宁府看家。
他这继母只得将两个未出嫁的小女带来,一并起居才放心。
\ji{原为放心而来,终是放心而去,妙甚!}\par
且说贾珍闻了此信,即忙告假,并贾蓉是有职之人。
礼部见当今隆敦孝弟,\zhu{
当今:意同“今上”,即当朝皇帝。
隆:高,盛。
敦:厚,引申为注重,推崇,敦促。
隆敦:很重视。
孝:尽心奉养和服从父母。
弟:同“悌”,弟弟顺从兄长。
}不敢自专,具本请旨。
原来天子极是仁孝过天的,且更隆重功臣之裔,一见此本,便诏问贾敬何职。
礼部代奏:“系进士出身,祖职已荫其子贾珍。
贾敬因年迈多疾,常养静于都城之外玄真观。
\zhu{养静:在宁静环境中修养身心。
}今因疾殁于寺中,其子珍,其孙蓉,现因国丧随驾在此,故乞假归殓。
”天子听了,忙下额外恩旨曰:“贾敬虽白衣无功于国,\zhu{白衣:古时平民穿白衣,故后为老百姓的代称。
}念彼祖父之功,追赐五品之职。
令其子孙扶柩由北下之门进都,入彼私第殡殓。
任子孙尽丧礼毕扶柩回籍外,着光禄寺按上例赐祭。
朝中由王公以下准其祭吊。
钦此。
”此旨一下,不但贾府中人谢恩,连朝中所有大臣皆嵩呼称颂不绝。
\zhu{嵩呼:也叫“山呼”。
封建时代臣子颂祝皇帝,高呼万岁,叫“嵩呼”。
}\par
贾珍父子星夜驰回,半路中又见贾㻞贾珖二人领家丁飞骑而来,看见贾珍,一齐滚鞍下马请安。
贾珍忙问:“作什么?”贾㻞回说:“嫂子恐哥哥和侄儿来了,老太太路上无人,叫我们两个来护送老太太的。
”贾珍听了,赞称不绝,又问家中如何料理。
贾㻞等便将如何拿了道士,如何挪至家庙,怕家内无人接了亲家母和两个姨娘在上房住着。
贾蓉当下也下了马,听见两个姨娘来了,便和贾珍一笑。
\ping{父子聚麀?}贾珍忙说了几声“妥当”,加鞭便走,店也不投,连夜换马飞驰。
一日到了都门,先奔入铁槛寺。
那天已是四更天气,坐更的闻知,忙喝起众人来。
贾珍下了马,和贾蓉放声大哭,从大门外便跪爬进来,至棺前稽颡泣血,\zhu{
颡[sǎng]:额,脑门。
稽颡:古时礼节,屈膝下拜,以额触地。
泣血:眼泪流尽继之以血,极言悲痛。
稽颡泣血:以头叩地,哀痛号泣。
}直哭到天亮喉咙都哑了方住。
尤氏等都一齐见过。
贾珍父子忙按礼换了凶服,在棺前俯伏,无奈自要理事,竟不能目不视物,耳不闻声,少不得减些悲戚,好指挥众人。
因将恩旨备述与众亲友听了。
一面先打发贾蓉家中料理停灵之事。
\par
贾蓉得不得一声儿,\zhu{得不得:义同“巴不得”,迫切盼望。
}先骑马飞来至家,忙命前厅收桌椅,下槅扇,
\zhu{
槅扇:在房屋内部作隔开用的一扇扇木板墙或纸壁,上部一般做成窗棂,糊纸或装玻璃。
也作“隔扇”。
}
挂孝幔子,门前起鼓手棚牌楼等事。
\zhu{
牌楼:装饰性的建筑物,由两个或四个并列的柱子支撑,上有檐额,跨度较宽,多建在重要路口或名胜景点处。有时也临时用竹木搭建。
}
又忙着进来看外祖母两个姨娘。
原来尤老安人年高喜睡,\zhu{安人:妇人封赠的号,宋代朝奉郎以上封安人,明、清六品封安人。
也指夫人,对妇人的尊称。
}常歪着,他二姨娘三姨娘都和丫头们作活计,他来了都道烦恼。
贾蓉且嘻嘻的望他二姨娘笑说:“二姨娘,你又来了,我们父亲正想你呢。
”尤二姐便红了脸,骂道:“蓉小子,我过两日不骂你几句,你就过不得了。
越发连个体统都没了。
还亏你是大家公子哥儿,每日念书学礼的,越发连那小家子瓢坎的也跟不上。
”\zhu{瓢坎:可能是“刨坎”的错讹,《红楼梦》戚序本此处作“刨坎”,但各校本都不采纳。
“坎”有土块的意思,“刨坎”指翻松土块。
由于“刨坎”是种田常见的工序,所以引申为泛指用镐头等工具劳作。
“刨坎”是个动词,“刨坎的”义为种田的,乡下人。
“越发连那小家子刨坎的也跟不上”是说干脆连那小户人家种田的都不如。
“刨坎”是如何变成“瓢坎”的呢?这跟《红楼梦》抄本的形成过程有关。
《红楼梦》抄本流传到社会上后,大受欢迎。
一部抄本能卖上数十两金子(或银子),为了一次多抄几份,就采用一人念底本、多个抄手听写的方法,这就难免产生音误,传世抄本中有不少音误字就是这样造成的。
如庚辰本第七十六回:“影是只有一个魂字可对”“我竟要搁必了”“冷月葬死魂”,其中的“是”“必”“死”分别是“字”“笔”“诗”的音误。
“瓢”与“刨”读音近似,“瓢”正是抄手误听的结果。
}说着顺手拿起一个熨斗来,搂头就打,吓的贾蓉抱着头滚到怀里告饶。
尤三姐便上来撕嘴,又说:“等姐姐来家,咱们告诉他。
”\par
贾蓉忙笑着跪在炕上求饶,他两个又笑了。
贾蓉又和二姨抢砂仁吃,
\zhu{砂仁:亦称“缩砂仁”、“缩砂蜜”,为姜科植物阳春砂或缩砂的成熟果实,产于热带,经常咀嚼,口齿芳香。}
尤二姐嚼了一嘴渣子,吐了他一脸。
\ping{李煜的《一斛珠》中“秀床斜凭娇无那。烂嚼红茸,笑向檀郎唾”便是以吐渣子的方式调情。}
贾蓉用舌头都舔着吃了。
众丫头看不过,都笑说:“热孝在身上,老娘才睡了觉,他两个虽小,到底是姨娘家,你太眼里没有奶奶了。
回来告诉爷,你吃不了兜着走。
”贾蓉撇下他姨娘,便抱着丫头们亲嘴:“我的心肝,你说的是,咱们馋他两个。
”丫头们忙推他,恨的骂:“短命鬼儿,你一般有老婆丫头,只和我们闹。
知道的说是顽;\ji{妙极之“顽”,天下有是之顽亦有趣甚,此语余亦亲闻者,非编有也。
}不知道的人,再遇见那脏心烂肺的爱多管闲事嚼舌头的人,吵嚷的那府里谁不知道,谁不背地里嚼舌说咱们这边乱帐。
”贾蓉笑道:“各门另户,谁管谁的事。
都够使的了。
从古至今,连汉朝和唐朝,人还说脏唐臭汉,何况咱们这宗人家。
谁家没风流事,别讨我说出来。
连那边大老爷这么利害,琏叔还和那小姨娘不干净呢。
\zhu{小姨娘应该是指后来赏给贾琏的贾赦的妾秋桐。}
凤姑娘那样刚强,瑞叔还想他的帐。
那一件瞒了我!”\par
贾蓉只管信口开合胡言乱道之间,只见他老娘醒了,
\ping{
在程本中原本躺在二尤和丫鬟一旁睡觉的尤老娘被改成了在屋内睡觉,直到被三姐叫醒才从房间里出来。这一情节改变修改了她的形象,尤老娘并不是纵容女儿行淫的妈妈,单纯是一个年高喜睡的老人,由于形象的净化,她在故事中就转变成了贾家三人不能随意淫乱的阻碍,而非脂本中的帮凶。
}
请安问好,又说:“难为老祖宗劳心,又难为两位姨娘受委屈,我们爷儿们感戴不尽。
惟有等事完了,我们合家大小,登门去磕头。
”尤老人点头道:“我的儿,倒是你们会说话。
亲戚们原是该的。
”又问:“你父亲好?几时得了信赶到的?”贾蓉笑道:“才刚赶到的,先打发我瞧你老人家来了。
好歹求你老人家事完了再去。
”说着,又和他二姨挤眼,那尤二姐便悄悄咬牙含笑骂:“很会嚼舌头的猴儿崽子,留下我们给你爹作娘不成!”贾蓉又戏他老娘道:“放心罢,我父亲每日为两位姨娘操心,要寻两个又有根基又富贵又年青又俏皮的两位姨爹,好聘嫁这二位姨娘的。
这几年总没拣得,可巧前日路上才相准了一个。
”尤老只当真话,忙问是谁家的,二姊妹丢了活计,一头笑,一头赶着打。
说:“妈别信这雷打的。
”连丫头们都说:“天老爷有眼,仔细雷要紧!”又值人来回话:“事已完了,请哥儿出去看了,回爷的话去。
”那贾蓉方笑嘻嘻的去了。
不知如何,且听下回分解。
\par
\qi{总评:宝玉品高性雅,其终日花围翠绕,用力维持其间,淫荡之至,而能使旁人不觉,被人不厌。
贾蓉不分长幼微贱,纵意驰骋于中,恶习可恨。
二人之形景天渊,而终归于邪,其滥一也,所谓五十步之间耳。
持家有意于子弟者,揣此以照察之可也!}
\dai{125}{寿怡红群芳开夜宴}
\dai{126}{死金丹独艳理亲丧}
\sun{p63-1}{因斗草香菱裙湿,寿怡红群芳开夜宴}{图右侧:众人采了些花草来,坐在花草地上斗草。
香菱因别人开自己玩笑,打闹起来,自己的裙子都污湿了。
宝玉见了后唤袭人送来新裙让香菱换上。
图左侧:众姊妹无拘无束又行令喝酒,闹了一夜。
}