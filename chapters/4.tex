\chapter{薄命女偏逢薄命郎\quad 葫芦僧乱判葫芦案}
\zhu{葫芦:“糊涂”的谐音,“葫芦案”即“糊涂案”。
这里语意双关,既寓葫芦僧原住的葫芦庙,又含“糊涂”之意,隐喻后文“胡乱判断了此案”。
}
\par
\qi{阴阳交结变无伦,幻境生时即是真。
秋月春花谁不见,朝晴暮雨自何因。
心肝一点劳牵恋,可意偏长遇喜嗔。
我爱世缘随分定,至诚相感作痴人。
\hang
请君着眼护官符,把笔悲伤说世途。
作者泪痕同我泪,燕山仍旧窦公无。
\zhu{
燕山窦公指窦禹钧,五代周渔阳人,官至谏议大夫,荐举四方贤士,五子相继登科。
这里借指世家大族的祖辈。最后两句宇面上是问燕山窦公还在不在呢,实际上是感叹世家大族早已败落了。
}
}\par
题曰:\par
捐躯报君恩,未报躯犹在。
眼底物多情,君恩或可待\foot{甲戌本无此回前诗,列、杨本有,文字小异,据两本互校补录。
}。
\par
\hop
却说黛玉同姊妹们至王夫人处,见王夫人与兄嫂处的来使计议家务,又说姨母家遭人命官司等语。
\meng{又来一位,宝钗将出现矣。
}因见王夫人事情冗杂,姊妹们遂出来,至寡嫂李氏房中来了。
\meng{慢慢度入法。
}\par
原来这李氏即贾珠之妻。
\jia{起笔写薛家事,他偏写宫裁,是结黛玉,明李纨本末,又在人意料之外。
}珠虽夭亡,幸存一子,取名贾兰,今已五岁,已入学攻书。
这李氏亦系金陵名宦之女,父名李守中,\jia{妙!盖云人能以理自守,安得为情所陷哉!}曾为国子监祭酒,\zhu{国子监祭酒:是国子监的主管官,封建时代的最高学官。
国子监:封建王朝的最高学府,简称“国学”,始建于西晋,称国子学,隋唐改称国子监,后代因之,清末始废。
祭酒:古代举行盛大宴会时,必推举宾客中一位长者先举酒以祭.叫祭酒,后来衍为学官名。
}族中男女无有不诵诗读书者。
\jia{未出李纨,先伏下李纹、李绮。
}至李守中承继以来,便说“女儿无才便有\jia{“有”字改得好。
}德”,\meng{确论。
}故生了李氏时,便不十分令其读书,只不过将些《女四书》、《列女传》、《贤媛集》等三四种书,\zhu{《女四书》:明代王相仿朱熹编《四书》的办法把东汉班昭的《女诫》、唐代宋若莘、宋若昭的《女论语》、明代永乐皇后徐氏的《内训》和王相母刘氏的《女范捷录》编辑为一书,井加注,总名为《女四书》。
《列女传》:西汉刘向编。
这些书记载了古代妇女的言行,内容大多宣扬封建伦理观念,但有的可以补证史传的不足。
《贤媛集》:未详。
《世说新语》第十九篇为《贤媛》,或即所指。
}使他认得几个字,记得这前朝几个贤女便罢了;却只以纺绩井臼为要,
\zhu{
绩:音“积”,缉线,把麻搓成绳或线。
井:汲水。
臼:音“旧”,舂米的器具(舂:音“充“,把谷物的壳捣掉),用石头凿成,这里代指舂米。
纺绩井臼:指纺线、绩麻、汲水、舂米,泛指家务劳作。
}因取名为李纨,字宫裁。
\jia{一洗小说窠臼俱尽,且命名字,亦不见红香翠玉恶俗。
}
因此这李纨虽青春丧偶,且居处于膏粱锦绣之中,竟如槁木死灰一般,\zhu{槁:音“搞”,草木枯干,引申为羸瘦憔悴。
槁木死灰:这里喻情性欲望已归寂灭。
《庄子·齐物论》:”形固可使如槁木,而心固可使如死灰乎?”郭象注:“槁木死灰,取其寂寞无情耳。
”}
\jia{此时处此境,最能越理生事,彼竟不然,实罕见者。
}\meng{反有此等文章。
}一概无见无闻,唯知侍亲养子,外则陪侍小姑等针黹诵读而已。
\zhu{
黹:音“旨”,缝纫,刺绣。
针黹:旧时妇女针线活儿的统称。
也叫“女红(红:音“工”)。
}\jia{一段叙出李纨,不犯熙凤。
}\meng{此中不得不有如此人。
天地覆载,何物不有?而才子手中,亦何物不有?}今黛玉虽客寄于斯,日有这般姐妹相伴,除老父外,馀者也就无庸虑及了。
\jia{仍是从黛玉身上写来,以上了结住黛玉,复找前文。
}\par
\chai{liwan}{李纨养性}
如今且说贾雨村,因补授了应天府,一下马,就有一件人命官司详至案下,\zhu{详:旧时下属向上司呈报请示的一种公文。
这里是动词,作“上报”解。
}\meng{非雨村难以了结此案。
}乃是两家争买一婢,各不相让,以致殴伤人命。
彼时雨村即问原告。
那原告道:“被殴死者乃小人之主人。
因那日买了一个丫头,不想系拐子所拐来卖的。
这拐子先已得了我家银子,我家小爷原说第三日方是好日子,再接入门。
\jia{所谓“迟则有变”,往往世人因不经之谈误却大事。
}这拐子便又悄悄的卖与了薛家,被我们知道了,去找那卖主夺取丫头。
无奈薛家原系金陵一霸,倚财仗势,众豪奴将我主人竟打死了。
\meng{一派世境恶习活现。
}凶身主仆已皆逃走,无影无踪,只剩了几个局外之人。
小人告了一年的状,\ping{忠诚如此,不忘主人冤屈,反观英莲,却因家仆霍启一时疏忽而被拐,实可伤也。
}竟无人作主。
\meng{悲夫!千古世情,不过如此。
}望大老爷拘拿凶犯,剪恶除凶,以救孤寡,死者感戴天恩不尽!”\par

雨村听了大怒道\meng{偏能用反跌法。
}:“岂有这样放屁的事!打死人命就白白的走了,再拿不来的?”因发签差公人立刻将凶犯族中人拿来拷问,\zhu{签:封建官府差役外出办事的凭证,一般系木制,长条形,插在公案签筒中,用时取出,称“发签”。
}令他们实供藏在何处,一面再动海捕文书。
\zhu{海捕文书:封建时代官府通令各地捕拿逃犯的公文,即后来的通缉令。
}正要发签时,只见案边立着一个门子,\zhu{门子:旧时官衙中从事看门、传达、站班等杂务的差役。
}使眼色儿不令他发签之意。
雨村心中甚是疑怪,\jia{原可疑怪,余亦疑怪。
}\meng{请看文字递出递转,闲中皆是要笔。
}只得停了手。
即时退堂,至密室,使从皆退去,只留下门子一人伏侍。
这门子忙上来请安,笑问:“老爷一向加官进禄,八九年来就忘了我了?”\jia{语气傲慢,怪甚!}\meng{似闲语,是要人。
}雨村道:“却十分面善得紧,只是一时想不起来。
”那门子笑道:“老爷真是贵人多忘事,把出身之地竟忘了,\jia{刹心语。
自招其祸,亦因夸能恃才也。
}不记当年葫芦庙里之事了?”雨村听了,如雷震一惊,\jia{余亦一惊,但不知门子何知,尤为怪甚。
}方想起往事。
原来这门子本是葫芦庙内一个小沙弥,\zhu{沙弥:梵语音译之略,意为“息恶”“行慈”。
凡男子初出家受十戒者通称沙弥,俗多以称呼年幼的小和尚。
}因被火之后,无处安身,欲投别庙去修行,又耐不得清凉景况,因想这件生意到还轻省热闹,\jia{新鲜字眼。
}遂趁年纪蓄了发,充了门子。
\jia{一路奇奇怪怪,调侃世人,总在人意臆之外。
}雨村那里料得是他,便忙携手笑道:“原来是故人。
”\jia{妙称!全是假态。
}又让坐了好谈。
\jia{假极!}这门子不敢坐。
雨村笑道:“贫贱之交不可忘,\jia{全是奸险小人态度,活现活跳。
}你我故人也,二则此系私室,\meng{如此亲近,其先必有故事。
}既欲长谈,岂有不坐之理?”这门子听说,方告了座,\zhu{告座:同“告坐”,上级或长辈让下级或晚辈坐,下级或晚辈谦让或道谢后坐下。
}斜签着坐了。
\zhu{斜签着坐:侧身直腰坐在凳子边沿,表示谦恭。
}\par
雨村因问方才何故不令发签。
这门子道:“老爷既荣任到这一省,难道就没抄一张‘护官符’\jia{可对“聚宝盆”,一笑。
}\jia{三字从来未见,奇之至!}来不成?”雨村忙问:“何为‘护官符’?\jia{余亦欲问。
}我竟不知。
”门子道:“这还了得!连这不知,怎能作得长远!\jia{骂得爽快!}
\meng{真是警世之言。
使我看之,不知要哭要笑。
}如今凡作地方官者,皆有一个私单,上面写的是本省最有权有势、极富极贵的大乡绅名姓,各省皆然。
倘若不知,一时触犯了这样的人家,不但官爵,只怕连性命还保不成呢!\jia{可怜可叹,可恨可气,变作一把眼泪也。
}\meng{快论。
请问其言是乎否乎?}所以绰号叫作‘护官符’。
\jia{奇甚趣甚,如何想来?}方才所说的这薛家,老爷如何惹得他!他这一件官司并无难断之处,皆因都碍着情分脸面,所以如此。
”一面说,一面从顺袋中取出一张抄写的‘护官符’来,\zhu{顺袋:衣服小襟上的口袋,右手顺襟便可探得。
一说为系在腰间的竖向小袋。
}递与雨村,看时,上面皆是本地大族名宦之家的谚俗口碑。
\zhu{口碑:《五灯会元》:“劝君不用镌顽石,路上行人口似碑。
”比喻人们口头上所传诵的,如同刻在石碑上一样不可磨灭。
}\ping{贾雨村是科举出身,属于“官”,在各地调动中升迁,受中央管理;门子是地方政府雇佣的办事人员,属于“吏”,一般不调动,也不升迁,不受中央管理。
“吏”虽然不入流,但是由于是基层政权的执行人员,且长期在某地任职,是百姓直接并且长期接触的政府公务员,吏治的清明与否决定了政府的清明与否。
}其口碑排写得明白,下面皆注着始祖官爵并房次。
\zhu{房:家族的一支叫“一房”。}
\qi{此等人家岂必欺霸方始成名耶?总因子弟不肖,招接匪人,一朝生事则百计营求,父为子隐,群小迎合,虽暂时不罹祸网,而从此放胆,必破家灭族不已,哀哉!}\meng{可怜伊等始祖。
}石头亦曾照样抄写一张,\jia{忙中闲笔,用得好。
}今据石上所抄云:\par
\hop
贾不假,白玉为堂金作马。
\jia{宁国、荣国二公之后,共二十房分,除宁、荣亲派八房在都外,现原籍住者十二房。
}\zhu{此句极言贾府的尊贵豪富。
白玉为堂金作马:汉乐府《相逢行》:“黄金为君门,白玉为君堂。
”又《三辅黄图》:“金马门,宦者署。
武帝时,得大宛马,以铜铸像,立于署门,因以为名。
”
}\par
阿房宫,三百里,住不下金陵一个史。
\jia{保龄侯尚书令史公之后,房分共十八。
都中现住者十房,原籍现居八房。
}\zhu{此句形容史府的门第显赫。
《三辅黄图》:“秦惠文王造阿房(阿房:音“婀旁”)宫,未成,始皇广其宫,规恢三百余里,阁道通骊山。
”尚书令:秦代始置,权限历代有异,秦时掌章奏文书,东汉时为总理政事,魏晋时事实上即为宰相。
明清时废。
}\par
丰年好大雪,\jia{隐“薛”字。
}珍珠如土金如铁。
\jia{紫薇舍人薛公之后,现领内府帑银行商,\zhu{
帑:音“躺“,本指藏钱财货币的府库,后引申为国有、公有的钱财。
帤银:国库所藏之钱财。
行商:音“形商”,往来各地贩售货品的商人。
也称为“行贩”、“行贾”。
}共八房分。
}\zhu{此句极言薛家钱财之多。
雪:谐音“薛”。
紫薇舍人,即中书舍人.为撰拟诰敕之专官,以有文学资望者充任。
唐开元年间曾改中书省为紫薇省。
}\par
东海缺少白玉床,龙王来请金陵王。
\jia{都太尉统制县伯王公之后,共十二房。
都中二房,馀皆在籍\foot{“护官符”小注,己、庚、舒本无,戚、蒙、列、甲辰本为文后双行小字,杨本为文后单行小字。
}。
}\zhu{此句极言王家多奇珍异宝。
传说四海龙王极富有,尤以东海龙王为最。
太尉:古官名。
秦汉为全国军事首脑,“三公”之一;后渐变为加官(加官:于原有官职外,兼领的其他官职),无实权;至宋徽宗时,成为武官最高官阶;元以后废。
统制:北宋所置官名,可节制兵马,南宋后成为禁军将官职衔。
}\par
\hop
雨村犹未看完,\jia{妙极!若只有此四家,则死板不活,若再有两家,又觉累赘,故如此断法。
}忽闻传点人报:\zhu{传点:封建时代的官署或大官僚的私邸,二门旁常设有一种金属的响器叫“点”,击之报时或集众叫“传点”。
“点”多铸成云头形,故又称“云板”。
}“王老爷来拜。
”雨村听说,忙具衣冠出去迎接。
\jia{横云断岭法,是板定大章法。
\zhu{板定:一定,必定。}
}有顿饭工夫,方回来细问。
这门子道:“这四家皆连络有亲,一损皆损,一荣皆荣,扶持遮饰,皆有照应的。
\jia{早为下半部伏根。
}\meng{此四家不相为结亲,则无门当户对者,亦理势之必然。
既结亲之后,岂不照应,又人情之不可无。
}\ping{南京国民政府的“四大家族”(蒋宋孔陈),名称来自于红楼梦金陵的“四大家族”(贾史王薛)。
}今告打死人之薛,就系‘丰年大雪’之薛也。
不单靠这三家,他的世交亲友在都在外者,本亦不少。
老爷如今拿谁去?”雨村听如此说,便笑问门子道:“如你这样说来,却怎么了结此案?你大约也深知这凶犯躲的方向了?”\par
门子笑道:“不瞒老爷说,不但这凶犯躲的方向我知道,一并这拐卖之人\jia{斯何人也。
}我也知道,死鬼买主也深知道。
待我细说与老爷听:\meng{放胆一说,毫无避忌。
世态人情被门子参透了。
}这个被打之死鬼,乃是本地一个小乡宦之子,名唤冯渊,\jia{真真是冤孽相逢。
}自幼父母早亡,又无兄弟,只他一个人守着些薄产过日。
\meng{我为幼而失父母者一哭。
}长到十八九岁上,酷爱男风,\zhu{男风:即男色,也叫男宠。
}最厌女子。
\jia{最厌女子,仍为女子丧生,是何等大笔!不是写冯渊,正是写英莲。
}这也是前生冤孽,可巧\jia{善善恶恶,多从可巧而来,可畏可怕。
}遇见这拐子卖丫头,他便一眼看上了这丫头,立意买来作妾,立誓再不交结男子,\jia{谚云:“人若改常,非病即亡。
”信有之乎?}也再不娶第二个了,\jia{虚写一个情种。
}\meng{也是幻中情魔。
}所以三日后方过门。
\ping{夜长梦多,薄命郎偏逢薄命女。
}
谁晓这拐子又偷卖与了薛家,\meng{一定情即了结,请问是幻不是?点醒幻字,人皆不醒。
我今日看了、批了,仍也是不醒。
}他意欲卷了两家银子,再逃往他省。
谁知又不曾走脱,两家拿住,打了个臭死,都不肯收银,只要领人。
那薛家公子岂是让人的,便喝着手下人一打,将冯公子打了个稀烂,\meng{有情反是无情。
\zhu{有情反是无情:有情人冯渊因殴致死反而无缘和英莲的情缘。}
}抬回家去,三日死了。
这薛公子原是早已择定日子上京去的,头起身两日前,就偶然遇见了这丫头,意欲买了就进京的,谁知闹出这事来。
既打了冯公子,夺了丫头,他便没事人一般,\ping{少年不谙世事,不受世俗规矩束缚,一切从自心而行,天真乎?残忍乎?因天真故残忍。
}只管带了家眷走他的路。
他这里自有兄弟奴仆在此料理,并不为此些些小事值得他一逃走的。
\jia{妙极!人命视为些些小事,总是刻画阿呆耳。
}这且别说,老爷你当被卖之丫头是谁?”\jia{问得又怪。
}雨村道:“我如何得知?”门子冷笑道:“这人算来还是老爷的大恩人呢!\meng{当心一脚。
请看后文,并无蹴动。
\zhu{蹴:音“促“,踏踩、踢。}
}他就是葫芦庙旁住的甄老爷的小姐,名唤英莲的。
”\jia{至此一醒。
}
雨村罕然道:“原来就是他!闻得养至五岁被人拐去,却如今才来卖呢?”\meng{“闻得”只说一层,并无言及要娇杏自道\sout{子}[之]语。非作者忘怀,欲写世态,故作幻笔。
}\par
门子道:“这一种拐子单管偷拐五六岁的儿女,养在一个僻静之处,到十一二岁时,度其容貌,带至他乡转卖。
当日这英莲我们天天哄他顽耍,虽隔了七八年,如今十二三岁的光景,其模样虽然出脱得齐整好些,然大概相貌,自是不改,熟人易认。
况且他眉心中原有米粒大小的一点胭脂㾵,\zhu{
㾵(音“记”):天生的色癍。
胭脂㾵:红色胎记。
}从胎里带来的,\jia{宝钗之热,黛玉之怯,悉从胎中带来。
今英莲有㾵,其人可知矣。
}所以我却认得。
偏生这拐子又租了我的房舍居住,\qi{作者要说容貌势力,要说情,要说幻,又要说小人之居心,豪强之托大,了结前文旧案,铺设后文根基。
点明英莲,收叙宝钗等项诸事:只借先之沙弥、今日门子之口层层叙来,真是大悲菩萨,千手千眼一时转动,毫无遗露。
可见具大光明者,故无难事,诚然。
}那日拐子不在家,我也曾问他。
他是被拐子打怕了的,\jia{可怜!}
\meng{世家子女至此。
可想见其先世亦必有如薛公子者。
\zhu{这条评语的意思可能是,英莲的先世有像薛蟠这样强霸的人,所以报应到后代英莲身上。}
}万不敢说,只说拐子系他亲爹,因无钱偿债,故卖他。
我又哄之再四,他又哭了,\meng{写其心机,总为后文。
}只说:‘我原不记得小时之事!’这可无疑了。
那日冯公子相看了,兑了银子,拐子醉了,他自叹道:‘我今日罪孽可满了!’\meng{天下英雄,失足匪人,偶得机会可以跳出者,与英莲同声一哭!}后又听得冯公子三日后才娶过门,他又转有忧愁之态。
我又不忍其形景,等拐子出去,又命内人去解释他:
\zhu{
内人:对人称自己的妻子为“内人”。
解释:劝解疏通。
}
‘这冯公子必待好日期来接,可知必不以丫鬟相看。
况他是个绝风流人品,家里颇过得,素习又最厌恶堂客,\zhu{堂客:旧时称妇女内眷为堂客,称男子为官客。
}今竟破价买你,后事不言可知。
只耐得三两日,何必忧闷!’\meng{良人者所望而终身也。
}
他听如此说,方才略解忧闷,自为从此得所。
谁料天下竟有这等不如意事,\jia{可怜真可怜!}\jia{一篇《薄命赋》,特出英莲。
}\meng{天下同患难者同来一哭!}第二日,他偏又卖与了薛家。
若卖与第二个人还好,这薛公子的混名人称‘呆霸王’,最是天下第一个弄性尚气的人,而且使钱如土,\jia{世路难行钱作马。
}\meng{“使钱如土”,方能称霸王。
}遂打了个落花流水,生拖死拽,把个英莲拖去,如今也不知死活。
\jia{为英莲留后步。
}这冯公子空喜一场,一念未遂,反花了钱,送了命,岂不可叹!”\jia{又一首《薄命叹》。
英、冯二人一段小悲欢幻景从葫芦僧口中补出,省却闲文之法也。
所谓“美中不足,好事多魔”,
\zhu{
好事多魔:喜庆美好的事,往往要经过很多波折才能如愿。常用来指男女佳期不顺。
也作「好事多磨」、「好事多妨」。
}
先用冯渊作一开路之人。
}\par
雨村听了,亦叹道:“这也是他们的孽障遭遇,\zhu{孽障:佛教名词,又作“业障”。
意谓前世所作种种恶因,致为今生的障碍。
}亦非偶然。
不然这冯渊如何偏只看准了这英莲?这英莲受了拐子这几年折磨,才得了个头路,且又是个多情的,若能聚合了,倒是一件美事,偏又生出这段事来。
\meng{冯渊之事之人,是英莲之幻景中之痴情人。
}这薛家纵比冯家富贵,想其为人,自然姬妾众多,淫佚无度,
\zhu{淫佚:行为放荡。也作「淫逸」。}
未必及冯渊定情于一人者。
这正是梦幻情缘,\meng{点明白了,直入本题。
}恰遇见一对薄命儿女。
\jia{使雨村一评,方补足上半回之题目。
所谓此书有繁处愈繁,省中愈省;又有不怕繁中繁,只要繁中虚;不畏省中省,只要省中实。
此则省中实也。
}且不要议论他,只目今这官司,\zhu{目今:现在,当前。
}如何剖断才好?”\ping{贾雨村之前侍才放犷,如今却不耻下问,宦海沉浮历练人。
}门子笑道:“老爷当年何等明决,今日何翻成个没主意的人了!\meng{利欲薰心,必致如此。
}小的闻得老爷补升此任,亦系贾府、王府之力,此薛蟠即贾府之老亲,老爷何不顺水行舟,做个整人情,将此案了结,日后也好见贾、王二公的。
”雨村道:“你说的何尝不是。
\jia{可发一长叹。
这一句已见奸雄,全是假。
}但事关人命,蒙皇上隆恩,起复委用,\jia{奸雄。
}实是重生再造,正当殚心竭力图报之时,\jia{奸雄。
}岂可因私而废法?\jia{奸雄。
}\meng{良明不昧势难当。
}是我实不能忍为者。
”\jia{全是假。
}门子听了,冷笑道:“老爷说的何尝不是大道理,但只是如今世上是行不去的。
岂不闻古人有云‘大丈夫相时而动’,\zhu{相时而动:审察时势采取行动。
}\meng{误尽多少苍生!}又曰‘趋吉避凶者为君子’。
\jia{近时错会书意者多多如此。
}依老爷这一说,不但不能报效朝廷,亦且自身不保,\meng{说了来也是一团道理。
}
还要三思为妥。
”\par
雨村低了半日头,\jia{奸雄欺人。
}方说道:“依你怎么样?”门子道:“小人已想了个极好的主意在此:老爷明日坐堂,只管虚张声势,动文书发签拿人。
原凶自然是拿不来的,原告固是定要将薛家族中及奴仆人等拿几个来拷问。
小的在暗中调停,令他们报个暴病身亡,合族中及地方上共递一张保呈,老爷只说善能扶鸾请仙,\zhu{扶鸾(鸾:音“峦”):即扶乩(乩:音“机”),亦作“扶箕”。
多用木制的丁字架,以竖木为笔,下设沙盘.由两人扶持横木之两端,假作施术请神降临,问以休咎吉凶(休:美善。
咎:灾祸,罪过),木笔便在沙盘上画出文字作答,实际上是一种迷信占卜术。
}堂上设下乩坛,令军民人等只管来看。
老爷就说:‘乩仙批了,死者冯渊与薛蟠原因夙孽相逢,
\zhu{夙:旧有的、过去的。}
今狭路既遇,原应了结。
薛蟠今已得无名之症,\jia{“无名之症”却是病之名,而反曰“无”,妙极!}被冯魂追索已死。
其祸皆由拐子某人而起,拐之人原系某乡某姓人氏,按法处治,馀不略及’等语。
小人暗中嘱托拐子,令其实招。
众人见乩仙批语与拐子相符,馀者自然也都不虚了。
薛家有的是钱,老爷断一千也可,五百也可,与冯家作烧埋之费。
那冯家也无甚要紧的人,不过为的是钱,见有了这个银子,想也就无话了。
\ping{挟尸索银,非为死者讨回公道,却为生者赚取补偿。
痴人为情失了性命,到头来肥了无关之人。
}老爷细想此计如何?”雨村笑道:“不妥,不妥。
\jia{奸雄欺人。
}等我再斟酌斟酌,\meng{一张口就是了结,真腐臭。
以“再斟酌”收结,真是不凡之笔。
}\ping{贾雨村不能立即接受门子的方案,即使最后也是按照这个方案断案,因为他岂能允许自己智巧不如地位卑贱的门子。
门子逞炫己之能,显露人之短,埋下了“远远的充发”的祸根。
}或可压服口声。
”\zhu{口声:指众人的议论。
}二人计议,天色已晚,别无话说。
\par
至次日坐堂,勾取一应有名人犯,雨村详加审问,果见冯家人口稀疏,不过赖此欲多得些烧埋之费,\jia{因此三四语收住,极妙!此则重重写来,轻轻抹去也。
}薛家仗势倚情,偏不相让,故致颠倒未决。
雨村便徇情枉法,胡乱判断了此案。
\jia{实注一笔,更好。
不过是如此等事,又何用细写。
可谓此书不敢干涉廊庙者,即此等处也,莫谓写之不到。
盖作者立意写闺阁尚不暇,何能又及此等哉!}冯家得了许多烧埋银子,也就无甚话说了。
\jia{盖宝钗一家不得不细写者。
若另起头绪,则文字死板,故仍只借雨村一人穿插出阿呆兄人命一事,且又带叙出英莲一向之行踪,并以后之归结,是以故意戏用“葫芦僧乱判”等字样,撰成半回,略一解颐,略一叹世,盖非有意讥刺仕途,实亦出人之闲文耳。
\zhu{出人:使人物登场。}
}\jia{又注冯家一笔,更妥。
可见冯家正不为人命,实赖此获利耳。
故用“乱判”二字为题,虽曰不涉世事,或亦有微辞耳。
但其意实欲出宝钗,不得不做此穿插,故云此等皆非《石头记》之正文。
}雨村断了此案,急忙作书信二封,\ping{初做亏心事,急求心安,赶紧卖了贾王两家面子。
}与贾政并京营节度使王子腾,\jia{随笔带出王家。
}不过说“令甥之事已完,不必过虑”等语。
\ping{贾雨村依靠英莲的父亲甄士隐的资助进京赶考,步入仕途。
如今英莲落难,贾雨村非但不思拯救,反而徇私枉法,“乱判葫芦案”,趋炎附势,忘恩负义。
另外,冯家仆人告状道:“小人告了一年的状,竟无人作主”可见香菱被薛蟠买走已经一年了,可能已经成为薛蟠的妾,甚至都有了孩子。
这时即使贾雨村良心发现,救出香菱,送还她母亲封氏。
香菱作为再婚且可能有过生育的女子,在那个时代,基本不可能做正室,倘若继续做妾,又和做薛蟠的妾有什么区别呢?同样都是由他人决定毫无自主的婚姻安排,甚至可能更差一些,毕竟薛家是四大家族之一,物质条件更好一些。
况且封氏的丈夫甄士隐已经“同了疯道人飘飘而去”,封氏自己“少不得依靠着他父母度日”,封氏的父亲“封肃虽然日日抱怨,也无可奈何了”。
封肃对落难的亲闺女尚且如此凉薄,对于落难的外孙女什么态度也可想而知。
封氏家产败落,丈夫出家,在娘家投靠,并没有什么话语权。
而其父亲封肃又是趋炎附势的人,当贾雨村求娶丫鬟娇杏时,“封肃喜得屁滚尿流,巴不得去奉承”,当得知自己的外孙女能嫁给贾雨村都要奉承的薛家为妾时,肯定喜得昏死过去,怎么能够接受贾雨村“救出”英莲,他一定会想方设法把回家的英莲送回薛家。
综上,贾雨村不救英莲固然薄情寡恩,但是救与不救,对于女性只是男性附属物的时代下的英莲,都无法改变她的苦命。
}此事皆由葫芦庙内之沙弥新门子所出,雨村又恐他对人说出当日贫贱时的事来,因此心中大不乐业。
\jia{瞧他写雨村如此,可知雨村终不是大英雄。
}后来到底寻了个不是,远远的充发了才罢。
\zhu{充发:即充军发配。
把死刑减等的罪犯或其他重犯押解到边远地方去服役。
}\jia{至此了结葫芦庙文字。
}\jia{又伏下千里伏线。
}\jia{起用“葫芦”字样,收用“葫芦”字样,盖云一部书皆系葫芦提之意也,
\zhu{葫芦提:糊里糊涂。}
此亦系寓意处。
}\meng{口如悬河者,当于出言时小心。
}\par

当下言不着雨村。
且说那买了英莲、打死冯渊的薛公子,\jia{本是立意写此,却不肯特起头绪,故意设出“乱判”一段戏文,其中穿插,至此却淡淡写来。
}亦系金陵人氏,本是书香继世之家。
\meng{为书香人家一叹。
}只是如今这薛公子幼年丧父,寡母又怜他是个独根孤种,未免溺爱纵容些,\meng{受病处。
富而且孤,自多溺爱。
孟母三迁,故难再见。
}遂致老大无成,且家中有百万之富,现领着内帑钱粮,
\zhu{帑:音“躺“,本指藏钱财货币的府库,后引申为国有、公有的钱财。}
采办杂料。
这薛公子学名薛蟠,表字文龙,今年方十有五岁,性情奢侈,言语傲慢。
虽也上过学,不过略识几字,\jia{这句加于老兄,却是实写。
}终日惟有斗鸡走马,\zhu{斗鸡走马:形容贵族子弟不务正业.游荡享乐的寄生生活。
斗鸡:用鸡相斗博输赢的一种游戏。
走马:驰马游猎。
}游山玩水而已。
虽是皇商,\zhu{皇商:专为宫廷采办购置各种用品的人。
}一应经纪世事,全然不知,不过赖祖父旧日的情分,户部挂虚名,支领钱粮,其馀事体,自有伙计老家人等措办。
寡母王氏,乃现任京营节度使王子腾之妹,\zhu{节度使:官职名,始设于唐代景云二年。
开元时.将边境地区每数州划为一镇,镇置节度使,统揽镇内军政大权,世称藩镇,实际上都成了拥权自重,割据独立的军阀。
到宋代始成为有虚名而无实权的名誉职位,元以后废。
}与荣国府贾政的夫人王氏,是一母所生的姊妹,今年方四十上下年纪,只有薛蟠一子。
\meng{非母溺爱,非家道殷实,非节度、荣国之至亲,则不能到如此强霸。
富贵者其思之。
}还有一女,比薛蟠小两岁,乳名宝钗,\qi{初见。
}生得肌骨莹润,举止娴雅。
\jia{写宝钗只如此,更妙!}
当日有他父亲在日,酷爱此女,令其读书识字,较之乃兄竟高过十倍。
\jia{又只如此写来,更妙!}自父亲死后,见哥哥不能体贴母怀,他便不以书字为事,只留心针黹家计等事,
\zhu{黹:音“旨”,缝纫,刺绣。}
好为母亲分忧解劳。
近因今上崇诗尚礼,\zhu{今上:封建时代对当朝皇帝的称谓。
}
征采才能,降不世出之隆恩,\zhu{不世出之隆恩:特别大的恩典。
不世出:不常出现。
}除聘选妃嫔外,凡世宦名家之女,皆亲名达部,\zhu{甲戌本上这句话是“皆报名达部”,甲戌本的文字应是作者本意,这里应该是把“报”误抄为“亲”,因为此两字的繁体字在结构上非常相似:“報”与“親”,如果再碰上墨汁发散等巧合因素,手抄本就容易发生混淆。
}以备选择为公主、郡主入学陪侍,\zhu{郡主:旧称诸王之女曰郡主。
唐宋称太子之女为郡主,元明清均以称亲王之女。
}充为才人、赞善之职。
\zhu{才人:宫中女官名,品位低于皇帝妃嫔。
初设于魏晋时,南北朝到明代多沿置,清代宫中无此称。
赞善:本太子宫中官名,掌侍从、讲授。
始置于唐,设左右赞善大夫,相当于朝廷的谏议大夫。
元明清皆因之,只称赞善。
这里为宫中女官名。
}\jia{一段称功颂德,千古小说中所无。
}二则自薛蟠父亲死后,各省中所有的买卖承局、
\zhu{承局:宋代殿前司属下阶级较低的将校。后泛指官府的公差。}
总管、伙计人等,见薛蟠年轻不识世事,便趁时拐骗起来,\meng{我为创家立业者一哭。
}京都中几处生意,渐亦消耗。
\meng{有治人,无治法。
}薛蟠素闻得都中乃第一繁华之地,正思一游,便趁此机会,一为送妹待选,二为望亲,三因亲自入部销算旧帐目,再计新支,其实则为游览上国风景之意。
\zhu{上国:汉代诸侯称帝室为上国。
后人多用来指国都京城。
}因此早已打点下行装细软,以及馈送亲友各色土物人情等类,正择日已定,不想偏遇见了那拐子重卖英莲。
薛蟠见英莲生得不俗,\jia{阿呆兄亦知不俗,英莲人品可知矣。
}立意买了,又遇冯家来夺人,因恃强喝令手下豪奴将冯渊打死。
他便将家中事务嘱了族中人并几个老家人,他便同了母妹竟自起身长行去了。
\meng{破销不顾业已之事,业已如此,到是走的妙。
\zhu{破销:这两个字令人费解。}
}人命官司一事,他却视为儿戏,自为花上几个臭钱,没有不了的。
\jia{是极!人谓薛蟠为呆,余则谓是大彻悟。
}\par
在路不计其日。
\jia{更妙!必云程限,
\zhu{程限:可能是期限的意思。}
则又有落套,岂暇又记路程单哉?
\zhu{这条评语的意思是,小说中不提及路程有多远、路上时间有多长,从而避免了落入俗套窠臼。}
}那日已将入都时,却又闻得母舅王子腾升了九省统制,\zhu{九省统制:沿古虚拟的官名。
统制之称,始于宋代,为武官职衔。
}奉旨出都查边。
\meng{天下之母舅再无不教外甥以正途者。
必使其升任出京,亦是留下文地步。
}薛蟠心中暗喜道:“我正愁进京去有个嫡亲的母舅管辖着,不能任意挥霍挥霍,偏如今又升出去了,可知天从人愿。
”\jia{写尽五陵心意。
\zhu{
五陵:长陵、安陵、阳陵、茂陵、平陵五个汉代帝王的陵寝。
皆位于长安,为当时豪侠巨富聚集的地方。
}
}\meng{写不肖子弟如画。
}因和母亲商议道:“咱们京中虽有几处房舍,只是这十来年没人进京居住,那看守的人未免偷着租赁与人,须得先着几人去打扫收拾才好。
”他母亲道:“何必如此招摇!咱们这一进京,原是先拜望亲友,或是在你舅舅家,\jia{陪笔。
}或是你姨爹家。
\jia{正笔。
}他两家的房舍极是方便的,咱们先能着住下,\zhu{能着:犹言“耐着”、“忍着”,引申为“将就着”。
}再慢慢的着人去收拾,岂不消停些。
”\zhu{消停:安闲、妥当。
}薛蟠道:“如今舅舅正升了外省去,家里自然忙乱起身。
\meng{好游荡不要管束的子弟,每惯会说此等话。
}咱们这工夫反一窝一拖的奔了去,岂不没眼色些?”\zhu{没眼色:不知趣、不识相。
}他母亲道:“你舅舅家虽升了去,还有你姨爹家。
况这几年来,你舅舅、姨娘两处,每每带信捎书,接咱们来。
如今既来了,你舅舅虽忙着起身,你贾家的姨娘未必不苦留我们。
咱们且忙忙收拾房舍,岂不使人见怪?\jia{闲语中补出许多前文,此画家之云罩峰尖法也。
}你的意思我却知道,\jia{知子莫如父。
}守着舅舅、姨爹住着,未免拘紧了你,不如你各自住着,好任意施为的。
\jia{寡母孤儿一段,写得毕肖毕真。
}\meng{用为子不得放荡一逼,再收入本意。
}你既如此,你自己去挑所宅子去住。
我和你姨娘,姊妹们别了这几年,却要厮守几日,我带了你妹子去投你姨娘家去,\jia{薛母亦善训子。
}你道好不好?”薛蟠见母亲如此说,情知扭不过的,\meng{情理如真。
}只得吩咐人夫一路奔荣国府来。
\ping{宝钗进京,为了追寻表姐贾元春的发展路线,入宫待选。
薛姨妈带着全家去荣国府居住,可能是为了就近向姐妹王夫人讨教送女儿进宫的经验。
}\par
那时王夫人已知薛蟠官司一事,亏贾雨村就中维持了结,才放了心。
又见哥哥升了边缺,正愁又少了娘家亲戚来往,\jia{大家尚义,人情大都是也。
}略加寂寞。
过了几日,忽家人传报:“姨太太带了哥儿姐儿,合家进京,正在门外下车。
”\meng{开留住之根。
}喜的王夫人忙带了媳妇、女儿人等,接出大厅,将薛姨妈等接了进去。
姊妹们暮年相见,自不必说悲喜交集,泣笑叙阔一番。
\zhu{叙阔:叙说离别之情。}
忙又引了拜见贾母,将人情土物各种酬献了,合家俱厮见了,忙又治席接风。
\par
薛蟠已拜见过贾政,贾琏又引着拜见了贾赦、贾珍等。
贾政便使人上来对王夫人说:“姨太太已有了春秋,\zhu{春秋:常用作对别人年岁的敬称。
}外甥年轻不知世路,在外住着恐有人生事。
咱们东北角上梨香院\jia{好香色。
}一所十来间白空闲,赶着打扫了,请姨太太和哥儿姐儿住了甚好。
”\jia{用政老一段,不但王夫人得体,且薛母亦免靠亲之嫌。
}王夫人未及留,贾母也就遣人来说“请姨太太就在这里住下,大家亲密些”等语。
\jia{老太君口气得情。
}\jia{偏不写王夫人留,方不死板。
}薛姨妈正要同居一处,方可拘紧些儿子,若另住在外,又恐他纵性惹祸,\meng{父母为子弟处每每如此。
}遂忙道谢应允。
又私与王夫人说明:“一应日费供给一概免却,\jia{作者题清,犹恐看官误认今之靠亲投友者一例。
}方是处常之法。
”\meng{补足。
真是一丝不漏。
}王夫人知他家不难于此,遂亦从其愿。
从此后,薛家母子就在梨香院中住了。
\par
原来这梨香院即当日荣公暮年养静之所,小小巧巧,约有十馀间房舍,前厅后舍俱全。
另有一门通街,薛蟠家人就走此门出入。
西南有一角门,通一夹道,出了夹道便是王夫人正房的东院了。
每日或饭后,或晚间,薛姨妈便过来,或与贾母闲谈,或和王夫人相叙。
宝钗日与黛玉、迎春姊妹等一处,\jia{金玉初见,却如此写,虚虚实实,总不相犯。
\zhu{犯:重复。}
}或看书着棋,或作针黹,倒也十分乐业。
\jia{这一句衬出后文黛玉之不能乐业,细甚妙甚!}只是薛蟠起初之心,原不欲在贾宅中居住者,但恐姨父管约拘禁,料必不自在的,无奈母亲执意在此,且贾宅中又十分殷勤苦留,只得暂且住下,一面使人打扫出自家的房屋,再移居过去的。
\jia{交代结构,曲曲折折,笔墨尽矣。
}谁知自在此间住了不上一月的日期,贾宅族中凡有的子侄,俱已认熟了一半,凡是那些纨绔气习者,莫不喜与他来往,今日会酒,明日观花,甚至聚赌嫖娼,渐渐无所不至,引诱的薛蟠比当日更坏了十倍。
\jia{虽说为纨绔设鉴,其意原只罪贾宅,故用此等句法写来。
} 
\meng{膏粱子弟每习成的风化。
处处皆然,诚为可叹!}虽说贾政训子有方,治家有法,\jia{八字特洗出政老来,又是作者隐意。
}一则族大人多,照管不到这些,二则现任族长乃是贾珍,彼乃宁府长孙,又现袭职,凡族中\foot{甲戌本下缺半页。
胡适据庚辰本抄补94 字。
}事,自有他掌管,三则公私冗杂,且素性潇洒,不以俗务为要,每公暇之时,不过看书着棋而已,\qi{其用笔墨何等灵活,能足前摇后,即境生文,真到不期然而然,所谓水到渠成,不劳着力者也。
}馀事多不介意。
况且这梨香院相隔两层房舍,又有街门另开,任意可以出入,\meng{既为作姨父的开一条生路。
若无此段,则姨父非木偶即不仁,则不成为姨父矣。
}所以这些子弟们竟可以放意畅怀的闹,因此,遂将移居之念,渐渐打灭了。
\par
\qi{总评:看他写一宝钗之来,先以英莲事逼其进京,及以舅氏官出,惟姨可倚。
辗转相逼来,且加以世态人情,隐跃其间,如人饮醇酒,不期然而已醉矣。
}
\dai{007}{薛蟠纵奴打死冯渊,强抢香菱}
\dai{008}{薛家来到荣国府}
\sun{p4-1}{葫芦僧乱判葫芦案}{图左侧:门子出示“护官符”;图右侧:雨村徇情枉法,胡乱判案。
}
\sun{p4-2}{宝黛钗初会}{薛家进京,姊妹相见。
}