\chapter{程甲本第 六 十 七 回}
话说尤三姐自尽之后,尤老娘和二姐儿、贾珍、贾琏等俱不胜悲恸,自不必说,忙令人盛殓,送往城外埋葬。
柳湘莲见尤三姐身亡,痴情眷恋,却被道人数句冷言打破迷关,竟自截发出家,跟随疯道人飘然而去,不知何往。
暂且不表。
\par
且说薛姨妈闻知湘莲已说定了尤三姐为妻,心中甚喜,正是高高兴兴要打算替他买房子,治傢伙,择吉迎娶,以报他救命之恩。
忽有家中小厮吵嚷“三姐儿自尽了”,被小丫头们听见,告知薛姨妈。
薛姨妈不知为何,心甚叹息。
正在猜疑,宝钗从园里过来,薛姨妈便对宝钗说道:“我的儿,你听见了没有?你珍大嫂子的妹妹三姑娘,他不是已经许定给你哥哥的义弟柳湘莲了么,不知为什么自刎了。
那柳湘莲也不知往那里去了。
真正奇怪的事,叫人意想不到。
”宝钗听了,并不在意,便说道:“俗话说的好,‘天有不测风云,人有旦夕祸福’。
这也是他们前生命定。
前日妈妈为他救了哥哥,商量着替他料理,如今已经死的死了,走的走了,依我说,也只好由他罢了。
妈妈也不必为他们伤感了。
倒是自从哥哥打江南回来了一二十日,贩了来的货物,想来也该发完了,那同伴去的伙计们辛辛苦苦的,回来几个月了,妈妈和哥哥商议商议,也该请一请,酬谢酬谢才是。
别叫人家看着无理似的。
”\par
母女正说话间,见薛蟠自外而入,眼中尚有泪痕。
一进门来,便向他母亲拍手说道:“妈妈可知道柳二哥、尤三姐的事么?”薛姨妈说:“我才听见说,正在这里和你妹妹说这件公案呢。
”薛蟠道:“妈妈可听见说柳湘莲跟着一个道士出了家了么?”薛姨妈道:“这越发奇了。
怎么柳相公那样一个年轻的聪明人,一时糊涂,就跟着道士去了呢。
我想你们好了一场,他又无父母兄弟,只身一人在此,你该各处找找他才是。
靠那道士能往那里远去,左不过是在这方近左右的庙里寺里罢了。
”薛蟠说:“何尝不是呢。
我一听见这个信儿,就连忙带了小厮们在各处寻找,连一个影儿也没有。
又去问人,都说没看见。
”薛姨妈说:“你既找寻过没有,也算把你作朋友的心尽了。
焉知他这一出家不是得了好处去呢。
只是你如今也该张罗张罗买卖,二则把你自己娶媳妇应办的事情,倒早些料理料理。
咱们家没人,俗语说的‘夯雀儿先飞’,省得临时丢三落四的不齐全,令人笑话。
再者你妹妹才说,你也回家半个多月了,想货物也该发完了,同你去的伙计们,也该摆桌酒给他们道道乏才是。
人家陪着你走了二三千里的路程,受了四五个月的辛苦,而且在路上又替你担了多少的惊怕沉重。
”薛蟠听说,便道:“妈妈说的很是。
倒是妹妹想的周到。
我也这样想着,只因这些日子为各处发货闹的脑袋都大了。
又为柳二哥的事忙了这几日,反倒落了一个空,白张罗了一会子,倒把正经事都误了。
要不然定了明儿后儿下帖儿请罢。
”薛姨妈道:“由你办去罢。
”\par
话犹未了,外面小厮进来回说:“管总的张大爷差人送了两箱子东西来,说这是爷各自买的,不在货账里面。
本要早送来,因货物箱子压着,没得拿,昨儿货物发完了,所以今日才送来了。
”一面说,一面又见两个小厮搬进了两个夹板夹的大棕箱。
薛蟠一见,说:“嗳哟,可是我怎么就糊涂到这步田地了!特特的给妈和妹妹带来的东西,都忘了没拿了家里来,还是伙计送了来了。
”宝钗说:“亏你说,还是特特的带来的才放了一二十天,若不是特特的带来,大约要放到年底下才送来呢。
我看你也诸事太不留心了。
”薛蟠笑道:“想是在路上叫人把魂吓掉了,还没归窍呢。
”说着大家笑了一回,便向小丫头说:“出去告诉小厮们,东西收下,叫他们回去罢。
”薛姨妈同宝钗因问:“到底是什么东西,这样捆着绑着的?”薛蟠便命叫两个小厮进来,解了绳子,去了夹板,开了锁看时,这一箱都是绸缎、绫锦、洋货等家常应用之物。
薛蟠笑着道:“那一箱是给妹妹带的。
”亲自来开。
母女二人看时,却是些笔、墨、纸、砚、各色笺纸、香袋、香珠、扇子、扇坠、花粉、胭脂等物,外有虎丘带来的自行人、酒令儿、水银灌的打筋斗小小子、沙子灯、一出一出的泥人儿的戏,用青纱罩的匣子装着,又有在虎丘山上泥捏的薛蟠的小像,与薛蟠毫无相差。
宝钗见了,别的都不理论,倒是薛蟠的小像,拿着细细看了一看,又看看他哥哥,不禁笑起来了。
因叫莺儿带着几个老婆子将这些东西连箱子送到园里去,又和母亲哥哥说了一回闲话儿,才回园里去了。
这里薛姨妈将箱子里的东西取出,一分一分的打点清楚,叫同喜送给贾母并王夫人等处不提。
\par
且说宝钗到了自己房中,将那些玩意儿一件一件的过了目,除了自己留用之外,一分一分配合妥当,也有送笔、墨、纸、砚的,也有送香袋、扇子、香坠的,也有送脂粉、头油的,有单送顽意儿的。
只有黛玉的比别人不同,且又加厚一倍。
一一打点完毕,使莺儿同着一个老婆子,跟着送往各处。
\par
这边姊妹诸人都收了东西,赏赐来使,说见面再谢。
惟有林黛玉看见他家乡之物,反自触物伤情,想起父母双亡,又无兄弟,寄居亲戚家中,那里有人也给我带些土物?想到这里,不觉的又伤起心来了。
紫鹃深知黛玉心肠,但也不敢说破,只在一旁劝道:“姑娘的身子多病,早晚服药,这两日看着比那些日子略好些。
虽说精神长了一点儿,还算不得十分大好。
今儿宝姑娘送来的这些东西,可见宝姑娘素日看得姑娘很重,姑娘看着该喜欢才是,为什么反倒伤起心来。
这不是宝姑娘送东西来倒叫姑娘烦恼了不成?就是宝姑娘听见,反觉脸上不好看。
再者这里老太太们为姑娘的病体,千方百计请好大夫配药诊治,也为是姑娘的病好。
这如今才好些,又这样哭哭啼啼,岂不是自己遭塌了自己身子,叫老太太看着添了愁烦了么?况且姑娘这病,原是素日忧虑过度,伤了血气。
姑娘的千金贵体,也别自己看轻了。
”紫鹃正在这里劝解,只听见小丫头子在院内说:“宝二爷来了。
”紫鹃忙说:“请二爷进来罢。
”\par
只见宝玉进房来了,黛玉让坐毕,宝玉见黛玉泪痕满面,便问:“妹妹,又是谁气着你了?”黛玉勉强笑道:“谁生什么气。
”旁边紫鹃将嘴向床后桌上一努,宝玉会意,往那里一瞧,见堆着许多东西,就知道是宝钗送来的,便取笑说道:“那里这些东西,不是妹妹要开杂货铺啊?”黛玉也不答言。
紫鹃笑着道:“二爷还提东西呢。
因宝姑娘送了些东西来,姑娘一看就伤起心来了。
我正在这里劝解,恰好二爷来的很巧,替我们劝劝。
”宝玉明知黛玉是这个缘故,却也不敢提头儿,只得笑说道:“你们姑娘的缘故想来不为别的,必是宝姑娘送来的东西少,所以生气伤心。
妹妹,你放心,等我明年叫人往江南去,与你多多的带两船来,省得你淌眼抹泪的。
”\par
黛玉听了这些话,也知宝玉是为自己开心,也不好推,也不好任,因说道:“我任凭怎么没见世面,也到不了这步田地,因送的东西少,就生气伤心。
我又不是两三岁的小孩子,你也忒把人看得小气了。
我有我的缘故,你那里知道。
”说着,眼泪又流下来了。
宝玉忙走到床前,挨着黛玉坐下,将那些东西一件一件拿起来摆弄着细瞧,故意问这是什么,叫什么名字;那是什么做的,这样齐整;这是什么,要他做什么使用。
又说这一件可以摆在面前,又说那一件可以放在条桌上当古董儿倒好呢。
一味的将些没要紧的话来厮混。
黛玉见宝玉如此,自己心里倒过不去,便说:“你不用在这里混搅了。
咱们到宝姐姐那边去罢。
”宝玉巴不得黛玉出去散散闷,解了悲痛,便道:“宝姐姐送咱们东西,咱们原该谢谢去。
”黛玉道:“自家姊妹,这倒不必。
只是到他那边,薛大哥回来了,必然告诉他些南边的古迹儿,我去听听,只当回了家乡一趟的。
”说着,眼圈儿又红了。
宝玉便站着等他。
黛玉只得同他出来,往宝钗那里去了。
\par
且说薛蟠听了母亲之言,急下了请帖,办了酒席。
次日,请了四位伙计,俱已到齐,不免说些贩卖账目发货之事。
不一时,上席让坐,薛蟠挨次斟了酒。
薛姨妈又使人出来致意。
大家喝着酒说闲话儿。
内中一个道:“今日这席上短两个好朋友。
”众人齐问是谁,那人道:“还有谁,就是贾府上的琏二爷和大爷的盟弟柳二爷。
”大家果然都想起来,问着薛蟠道:“怎么不请琏二爷和柳二爷来?”薛蟠闻言,把眉一皱,叹口气道:“琏二爷又往平安州去了,头两天就起了身的。
那柳二爷竟别提起,真是天下头一件奇事。
什么是柳二爷,如今不知那里作柳道爷去了。
”众人都诧异道:“这是怎么说?”薛蟠便把湘莲前后事体说了一遍。
众人听了,越发骇异,因说道:“怪不的前日我们在店里仿仿佛佛也听见人吵嚷说,有一个道士三言两语把一个人度了去了,又说一阵风刮了去了。
只不知是谁。
我们正发货,那里有闲工夫打听这个事去,到如今还是似信不信的,谁知就是柳二爷呢。
早知是他,我们大家也该劝他劝才是。
任他怎么着,也不叫他去。
”内中一个道:“别是这么着罢?”众人问怎么样,那人道:“柳二爷那样个伶俐人,未必是真跟了道士去罢。
他原会些武艺,又有力量,或看破那道士的妖术邪法,特意跟他去,在背地摆布他,也未可知。
”薛蟠道:“果然如此倒也罢了。
世上这些妖言惑众的人,怎么没人治他一下子。
”众人道:“那时难道你知道了也没找寻他去?”薛蟠说:“城里城外,那里没有找到?不怕你们笑话,我找不着他,还哭了一场呢。
”言毕,只是长吁短叹无精打彩的,不像往日高兴。
众伙计见他这样光景,自然不便久坐,不过随便喝了几杯酒,吃了饭,大家散了。
\par
且说宝玉同着黛玉到宝钗处来。
宝玉见了宝钗,便说道:“大哥哥辛辛苦苦的带了东西来,姐姐留着使罢,又送我们。
”宝钗笑道:“原不是什么好东西,不过是远路带来的土物儿,大家看着新鲜些就是了。
”黛玉道:“这些东西我们小时候倒不理会,如今看见,真是新鲜物儿了。
”宝钗因笑道:“妹妹知道,这就是俗语说的‘物离乡贵’,其实可算什么呢。
”宝玉听了这话正对了黛玉方才的心事,连忙拿话岔道:“明年好歹大哥哥再去时,替我们多带些来。
”黛玉瞅了他一眼,便道:“你要你只管说,不必拉扯上人。
姐姐你瞧,宝哥哥不是给姐姐来道谢,竟又要定下明年的东西来了。
”说的宝钗宝玉都笑了。
三个人又闲话了一回,因提起黛玉的病来。
宝钗劝了一回,因说道:“妹妹若觉着身子不爽快,倒要自己勉强拃挣着出来各处走走逛逛,散散心,比在屋里闷坐着到底好些。
我那两日不是觉着发懒,浑身发热,只是要歪着,也因为时气不好,怕病,因此寻些事情自己混着。
这两日才觉着好些了。
”黛玉道:“姐姐说的何尝不是。
我也是这么想着呢。
”大家又坐了一会子方散。
宝玉仍把黛玉送至潇湘馆门首,才各自回去了。
\par
且说赵姨娘因见宝钗送了贾环些东西,心中甚是喜欢,想道:“怨不得别人都说那宝丫头好,会做人,很大方,如今看起来果然不错。
他哥哥能带了多少东西来,他挨门儿送到,并不遗漏一处,也不露出谁薄谁厚,连我们这样没时运的,他都想到了。
若是那林丫头,他把我们娘儿们正眼也不瞧,那里还肯送我们东西?”一面想,一面把那些东西翻来覆去的摆弄瞧看一回。
忽然想到宝钗系王夫人的亲戚,为何不到王夫人跟前卖个好儿呢。
自己便蝎蝎螫螫的拿着东西,走至王夫人房中,站在旁边,陪笑说道:“这是宝姑娘才刚给环哥儿的。
难为宝姑娘这么年轻的人,想的这么周到,真是大户人家的姑娘,又展样,又大方,怎么叫人不敬服呢。
怪不得老太太和太太成日家都夸他疼他。
我也不敢自专就收起来,特拿来给太太瞧瞧,太太也喜欢喜欢。
”王夫人听了,早知道来意了,又见他说的不伦不类,也不便不理他,说道:“你自管收了去给环哥顽罢。
”赵姨娘来时兴兴头头,谁知抹了一鼻子灰,满心生气,又不敢露出来,只得讪讪的出来了。
到了自己房中,将东西丢在一边,嘴里咕咕哝哝自言自语道:“这个又算了个什么儿呢。
”一面坐着,各自生了一回闷气。
\par
却说莺儿带着老婆子们送东西回来,回复了宝钗,将众人道谢的话并赏赐的银钱都回完了,那老婆子便出去了。
莺儿走近前来一步,挨着宝钗悄悄的说道:“刚才我到琏二奶奶那边,看见二奶奶一脸的怒气。
我送下东西出来时,悄悄的问小红,说刚才二奶奶从老太太屋里回来,不似往日欢天喜地的,叫了平儿去,唧唧咕咕的不知说了些什么。
看那个光景,倒像有什么大事的似的。
姑娘没听见那边老太太有什么事?”宝钗听了,也自己纳闷,想不出凤姐是为什么有气,便道:“各人家有各人的事,咱们那里管得。
你去倒茶去罢。
”莺儿于是出来,自去倒茶不提。
\par
且说宝玉送了黛玉回来,想着黛玉的孤苦,不免也替他伤感起来。
因要将这话告诉袭人,进来时却只有麝月秋纹在房中。
因问:“你袭人姐姐那里去了?”麝月道:“左不过在这几个院里,那里就丢了他。
一时不见,就这样找。
”宝玉笑着道:“不是怕丢了他。
因我方才到林姑娘那边,见林姑娘又正伤心呢。
问起来却是为宝姐姐送了他东西,他看见是他家乡的土物,不免对景伤情。
我要告诉你袭人姐姐,叫他闲时过去劝劝。
”正说着,晴雯进来了,因问宝玉道:“你回来了,你又要叫劝谁?”宝玉将方才的话说了一遍。
晴雯道:“袭人姐姐才出去,听见他说要到琏二奶奶那边去。
保不住还到林姑娘那里。
”宝玉听了,便不言语。
秋纹倒了茶来,宝玉漱了一口,递给小丫头子,心中着实不自在,就随便歪在床上。
\par
却说袭人因宝玉出门,自己作了回活计,忽想起凤姐身上不好,这几日也没有过去看看,况闻贾琏出门,正好大家说说话儿。
便告诉晴雯:“好生在屋里,别都出去了,叫宝玉回来抓不着人。
”晴雯道:“嗳哟,这屋里单你一个人记挂着他,我们都是白闲着混饭吃的。
”袭人笑着,也不答言,就走了。
\par
刚来到沁芳桥畔,那时正是夏末秋初,池中莲藕新残相间,红绿离披。
袭人走着,沿堤看顽了一回。
猛抬头看见那边葡萄架底下有人拿着掸子在那里掸什么呢,走到跟前,却是老祝妈。
那老婆子见了袭人,便笑嘻嘻的迎上来,说道:“姑娘怎么今日得工夫出来逛逛?”袭人道:“可不是。
我要到琏二奶奶家瞧瞧去。
你在这里做什么呢?”那婆子道:“我在这里赶蜜蜂儿。
今年三伏里雨水少,这果子树上都有虫子,把果子吃的疤瘌流星的掉了好些下来。
姑娘还不知道呢,这马蜂最可恶的,一嘟噜上只咬破三两个儿,那破的水滴到好的上头,连这一嘟噜都是要烂的。
姑娘你瞧,咱们说话的空儿没赶,就落上许多了。
”袭人道:“你就是不住手的赶,也赶不了许多。
你倒是告诉买办,叫他多多做些小冷布口袋儿,一嘟噜套上一个,又透风,又不遭塌。
”婆子笑道:“倒是姑娘说的是。
我今年才管上,那里知道这个巧法儿呢。
”因又笑着说道:“今年果子虽遭塌了些,味儿倒好,不信摘一个姑娘尝尝。
”袭人正色道:“这那里使得。
不但没熟吃不得,就是熟了,上头还没有供鲜,咱们倒先吃了。
你是府里使老了的,难道连这个规矩都不懂了。
”老祝忙笑道:“姑娘说得是。
我见姑娘很喜欢,我才敢这么说,可就把规矩错了,我可是老糊涂了。
”袭人道:“这也没有什么。
只是你们有年纪的老奶奶们,别先领着头儿这么着就好了。
”说着遂一径出了园门,来到凤姐这边。
\par
一到院里,只听凤姐说道:“天理良心,我在这屋里熬的越发成了贼了。
”袭人听见这话,知道有原故了,又不好回来,又不好进去,遂把脚步放重些,隔着窗子问道:“平姐姐在家里呢么?”平儿忙答应着迎出来。
袭人便问:“二奶奶也在家里呢么,身上可大安了?”说着,已走进来。
凤姐装着在床上歪着呢,见袭人进来,也笑着站起来,说:“好些了,叫你惦着。
怎么这几日不过我们这边坐坐?”袭人道:“奶奶身上欠安,本该天天过来请安才是。
但只怕奶奶身上不爽快,倒要静静儿的歇歇儿,我们来了,倒吵的奶奶烦。
”凤姐笑道:“烦是没的话。
倒是宝兄弟屋里虽然人多,也就靠着你一个照看他,也实在的离不开。
我常听见平儿告诉我,说你背地里还惦着我,常常问我。
这就是你尽心了。
”一面说着,叫平儿挪了张杌子放在床旁边,让袭人坐下。
丰儿端进茶来,袭人欠身道:“妹妹坐着罢。
”一面说闲话儿。
只见一个小丫头子在外间屋里悄悄的和平儿说:“旺儿来了。
在二门上伺候着呢。
”又听见平儿也悄悄的道:“知道了。
叫他先去,回来再来,别在门口儿站着。
”袭人知他们有事,又说了两句话,便起身要走。
凤姐道:“闲来坐坐,说说话儿,我倒开心。
”因命平儿:“送送你妹妹。
”平儿答应着送出来。
只见两三个小丫头子,都在那里屏声息气齐齐的伺候着。
袭人不知何事,便自去了。
\par
却说平儿送出袭人,进来回道:“旺儿才来了,因袭人在这里我叫他先到外头等等儿,这会子还是立刻叫他呢,还是等着?请奶奶的示下。
”凤姐道:“叫他来。
”平儿忙叫小丫头去传旺儿进来。
这里凤姐又问平儿:“你到底是怎么听见说的?”平儿道:“就是头里那小丫头子的话。
他说他在二门里头听见外头两个小厮说:‘这个新二奶奶比咱们旧二奶奶还俊呢,脾气儿也好。
’不知是旺儿是谁,吆喝了两个一顿,说:‘什么新奶奶旧奶奶的,还不快悄悄儿的呢,叫里头知道了,把你的舌头还割了呢。
’”平儿正说着,只见一个小丫头进来回说:“旺儿在外头伺候着呢。
”凤姐听了,冷笑了一声说:“叫他进来。
”那小丫头出来说:“奶奶叫呢。
”旺儿连忙答应着进来。
\par
旺儿请了安,在外间门口垂手侍立。
凤姐儿道:“你过来,我问你话。
”旺儿才走到里间门旁站着。
凤姐儿道:“你二爷在外头弄了人,你知道不知道?”旺儿又打着千儿回道:“奴才天天在二门上听差事,如何能知道二爷外头的事呢。
”凤姐冷笑道:“你自然不知道。
你要知道,你怎么拦人呢。
”旺儿见这话,知道刚才的话已经走了风了,料着瞒不过,便又跪回道:“奴才实在不知。
就是头里兴儿和喜儿两个人在那里混说,奴才吆喝了他们两句。
内中深情底里奴才不知道,不敢妄回。
求奶奶问兴儿,他是长跟二爷出门的。
”凤姐听了,下死劲啐了一口,骂道:“你们这一起没良心的混账忘八崽子!都是一条藤儿,打量我不知道呢。
先去给我把兴儿那个忘八崽子叫了来,你也不许走。
问明白了他,回来再问你。
好,好,好!这才是我使出来的好人呢!”那旺儿只得连声答应几个是,磕了个头爬起来出去,去叫兴儿。
\par
却说兴儿正在账房儿里和小厮们玩呢,听见说二奶奶叫,先唬了一跳,却也想不到是这件事发作了,连忙跟着旺儿进来。
旺儿先进去,回说:“兴儿来了。
”凤姐儿厉声道:“叫他!”那兴儿听见这个声音儿,早已没了主意了,只得乍着胆子进来。
凤姐儿一见,便说:“好小子啊!你和你爷办的好事啊!你只实说罢!”兴儿一闻此言,又看见凤姐儿气色及两边丫头们的光景,早唬软了,不觉跪下,只是磕头。
凤姐儿道:“论起这事来,我也听见说不与你相干。
但只你不早来回我知道,这就是你的不是了。
你要实说了,我还饶你,再有一字虚言,你先摸摸你腔子上几个脑袋瓜子!”兴儿战兢兢的朝上磕头道:“奶奶问的是什么事,奴才同爷办坏了?”凤姐听了,一腔火都发作起来,喝命:“打嘴巴!”旺儿过来才要打时,凤姐儿骂道:“什么糊涂忘八崽子!叫他自己打,用你打吗!一会子你再各人打你那嘴巴子还不迟呢。
”那兴儿真个自己左右开弓打了自己十几个嘴巴。
\par
凤姐儿喝声“站住”,问道:“你二爷外头娶了什么新奶奶旧奶奶的事,你大概不知道啊。
”兴儿见说出这件事来,越发着了慌,连忙把帽子抓下来在砖地上咕咚咕咚碰的头山响,口里说道:“只求奶奶超生,\zhu{超生:宽宥其生命。
常用于祈求他人怜悯救助。
}奴才再不敢撒一个字儿的谎。
”凤姐道:“快说!”兴儿直蹶蹶的跪起来回道,“这事头里奴才也不知道。
就是这一天,东府里大老爷送了殡,俞禄往珍大爷庙里去领银子。
二爷同着蓉哥儿到了东府里,道儿上爷儿两个说起珍大奶奶那边的二位姨奶奶来。
二爷夸他好,蓉哥儿哄着二爷,说把二姨奶奶说给二爷。
”凤姐听到这里,使劲啐道:“呸,没脸的忘八蛋!他是你那一门子的姨奶奶!”兴儿忙又磕头说:“奴才该死!”往上瞅着,不敢言语。
凤姐儿道:“完了吗?怎么不说了?”兴儿方才又回道:“奶奶恕奴才,奴才才敢回。
”凤姐啐道:“放你妈的屁,这还什么恕不恕了。
你好生给我往下说,好多着呢。
”兴儿又回道:“二爷听见这个话就喜欢了。
后来奴才也不知道怎么就弄真了。
”\par
凤姐微微冷笑道:“这个自然么,你可那里知道呢!你知道的只怕都烦了呢。
是了,说底下的罢!”兴儿回道:“后来就是蓉哥儿给二爷找了房子。
”凤姐忙问道:“如今房子在那里?”兴儿道:“就在府后头。
”凤姐儿道:“哦。
”回头瞅着平儿道:“咱们都是死人哪。
你听听!”平儿也不敢作声。
兴儿又回道:“珍大爷那边给了张家不知多少银子,那张家就不问了。
”凤姐道:“这里头怎么又扯拉上什么张家李家咧呢?”兴儿回道:“奶奶不知道,这二奶奶……”刚说到这里,又自己打了个嘴巴,把凤姐儿倒怄笑了。
两边的丫头也都抿嘴儿笑。
兴儿想了想,说道:“那珍大奶奶的妹子……”凤姐儿接着道:“怎么样?快说呀。
”兴儿道:“那珍大奶奶的妹子原来从小儿有人家的,姓张,叫什么张华,如今穷的待好讨饭。
珍大爷许了他银子,他就退了亲了。
”\par
凤姐儿听到这里,点了点头儿,回头便望丫头们说道:“你们都听见了?小忘八崽子,头里他还说他不知道呢!”兴儿又回道:“后来二爷才叫人裱糊了房子,娶过来了。
”凤姐道:“打那里娶过来的?”兴儿回道:“就在他老娘家抬过来的。
”凤姐道:“好罢咧。
”又问:“没人送亲么?”兴儿道:“就是蓉哥儿。
还有几个丫头老婆子们,没别人。
”凤姐道:“你大奶奶没来吗?”兴儿道:“过了两天,大奶奶才拿了些东西来瞧的。
”凤姐儿笑了一笑,回头向平儿道:“怪道那两天二爷称赞大奶奶不离嘴呢。
”掉过脸来又问兴儿,“谁伏侍呢?自然是你了。
”兴儿赶着碰头不言语。
凤姐又问,“前头那些日子说给那府里办事,想来办的就是这个了。
”兴儿回道:“也有办事的时候,也有往新房子里去的时候。
”凤姐又问道:“谁和他住着呢。
”兴儿道:“他母亲和他妹子。
昨儿他妹子各人抹了脖子了。
”凤姐道:“这又为什么?”兴儿随将柳湘莲的事说了一遍。
凤姐道:“这个人还算造化高,省了当那出名儿的忘八。
”因又问道:“没了别的事了么?”兴儿道:“别的事奴才不知道。
奴才刚才说的字字是实话,一字虚假,奶奶问出来只管打死奴才,奴才也无怨的。
”\par
凤姐低了一回头,便又指着兴儿说道:“你这个猴儿崽子就该打死。
这有什么瞒着我的?你想着瞒了我,就在你那糊涂爷跟前讨了好儿了,你新奶奶好疼你。
我不看你刚才还有点怕惧儿,不敢撒谎,我把你的腿不给你砸折了呢。
”说着喝声“起去”。
兴儿磕了个头,才爬起来,退到外间门口,不敢就走。
凤姐道:“过来,我还有话呢。
”兴儿赶忙垂手敬听。
凤姐道:“你忙什么,新奶奶等着赏你什么呢?”兴儿也不敢抬头。
凤姐道:“你从今日不许过去。
我什么时候叫你,你什么时候到。
迟一步儿,你试试!出去罢。
”兴儿忙答应几个“是”,退出门来。
凤姐又叫道:“兴儿!”兴儿赶忙答应回来。
凤姐道:“快出去告诉你二爷去,是不是啊?”兴儿回道:“奴才不敢。
”凤姐道:“你出去提一个字儿,隄防你的皮!”兴儿连忙答应着才出去了。
凤姐又叫:“旺儿呢?”旺儿连忙答应着过来。
凤姐把眼直瞪瞪的瞅了两三句话的工夫,才说道:“好旺儿,很好,去罢!外头有人提一个字儿,全在你身上。
”旺儿答应着也出去了。
\par
凤姐便叫倒茶。
小丫头子们会意,都出去了。
这里凤姐才和平儿说:“你都听见了?这才好呢。
”平儿也不敢答言,只好陪笑儿。
凤姐越想越气,歪在枕上只是出神,忽然眉头一皱,计上心来,便叫:“平儿来。
”平儿连忙答应过来。
凤姐道:“我想这件事竟该这么着才好。
也不必等你二爷回来再商量了。
”未知凤姐如何办理,下回分解。
