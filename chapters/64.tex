\chapter{幽淑女悲题五美吟 \quad 浪荡子情遗九龙珮}
\qi{此一回紧接贾敬灵柩进城,原当铺叙宁府丧仪之盛,但上回秦氏病故,凤姐理丧,已描写殆尽,若仍极力写去,不过加倍热闹而已。
故书中于迎灵送殡极忙乱处,却只闲闲数笔带过。
忽插入钗玉评诗、琏尤赠珮一段闲雅[风流]文字来,正所谓“急脉缓受”也。
\zhu{急脉缓受:“受”通“授”。中医上指遇来势急猛的脉象时,稳缓地授药调治。
比喻用缓和的办法来对付急迫的事情。
}}\par
题曰:\par
深闺有奇女,绝世空珠翠。
\zhu{空:徒然,白白地。
这两句的意思是她才貌绝世,幽居深闺,虽有珠翠可增容色,也是枉然。
}情痴苦泪多,未惜颜憔悴。
\par
哀哉千秋魂,薄命无二致。
\zhu{千秋魂:指此回中黛玉所作《五美吟》中西施等古代的五个“有才色的女子”。
}\ping{“薄命无二致”中的薄命人是谁呢?本回主要人物有黛玉,五美吟中的五个古代女子,也有尤氏姐妹。
若以为指的是古代五女子都是“薄命”并“无二致”则与所咏人物不尽符合。
至少,红拂不能算薄命;何况书中也明言这些古代女儿的命运,有的是“令人可喜、可羡、可悲、可叹”。
通过脂评和正文,可以知道黛玉和尤氏姐妹都是薄命人。
}嗟彼桑间人,
\zhu{
桑间人:放荡淫逸之人的意思,这里指的是尤氏姐妹。
《礼记·乐记》:“桑间濮上之音,亡国之音也。
” 《汉书·地理志》:“卫地有桑间、濮上之阻,男女亦亟聚会,声色生焉。
故俗称郑、卫之音。
”后多用“桑间”以称淫风。
《诗经》里面的郑风、卫风表达爱情的作品多,郑卫之风成了淫词艳曲的代名词。
}
好丑非其类。
\ping{虽然尤氏姊妹与黛玉就“薄命”而言,也“无二致”,但一则淫佚,一则贞静,显然不可同日而语,即“好丑非其类”。
}\par
\hop
话说贾蓉见家中诸事已妥,连忙赶至寺中,回明贾珍。
于是连夜分派各项执事人役,并预备一切应用幡杠等物。
\zhu{幡:音“翻”,挑起来直着挂的长条形旗子。
杠:较粗的棍子。
}择于初四日卯时请灵柩进城,
\zhu{卯时:早上五点到七点。}
一面使人知会诸位亲友。
\par
是日,丧仪炫耀,宾客如云,自铁槛寺至宁府,夹路而观者,何啻万数。
\zhu{啻:音“赤”,但、只、仅。
常用于疑问词或否定词之后。
如“不啻”、“何啻”。
}也有羡慕的,也有嗟叹的。
又有一等半瓶醋的读书人,说是“丧礼与其奢易莫若俭戚”的,\zhu{丧礼与其奢易莫若俭戚:丧礼与其奢侈而缺乏真情,不如俭朴而衷心悲戚。
语出《论语·八佾》:“礼,与其奢也,宁俭;丧,与其易也,宁戚。
”易:轻慢;怠弛。
这里是指缺乏真情实意。
}一路纷纷议论不一。
至未申时方到,
\zhu{未时:下午一点到三点。申时:下午三点到五点。}
将灵柩停放正室之内。
供奠举哀已毕,亲友渐次散回,只剩族中人分理迎宾送客等事。
近亲只有邢大舅等未去。
贾珍贾蓉此时为礼法所拘,不免在灵旁藉草枕块,\zhu{
藉草枕块:睡干草,枕土块。
这是古时居父母丧的礼节。
《仪礼·既夕礼》贾公彦疏有“寝以苫,以块枕头”语。
苫:音“山”,草。
}恨苦居丧。
人散后,仍乘空寻他小姨厮混。
宝玉亦每日在宁府穿孝,至晚人散,方回园里。
凤姐身体未愈,虽不能时常在此,或遇开坛诵经、亲友上祭之日,亦扎挣过来,\zhu{扎挣:勉强支持。
}相帮尤氏料理料理。
\zhu{相帮:帮助。
}\par
一日,供毕早饭,因此时天气尚长,贾珍等连日劳倦,不免在灵旁假寐。
宝玉见无客至,遂欲回家看视黛玉,因先回至怡红院中。
进入门来,只见院中寂静,悄无人声,有几个老婆子与小丫头们在回廊下取便乘凉,也有睡卧的,也有坐着打盹的。
宝玉也不去惊动。
只有四儿看见,连忙上前来打帘子。
将掀起时,只见芳官自内带笑跑出,几乎与宝玉撞个满怀。
一见宝玉,方含笑站住,说道:“你怎么来了?你快与我拦住晴雯,他要打我呢。
”一语未了,只听得屋内咭溜咕噜的乱响,不知是何物撒了一地。
随后晴雯赶来骂道:“我看你这小蹄子往那里去!输了不叫打。
宝玉又不在家,我看谁来救你!”宝玉连忙拦住,笑道:“你妹子小,不知怎么得罪了你,看我的分上饶他罢。
”晴雯也不想宝玉此时回来,乍一见,不觉好笑,遂笑说道:“芳官竟是狐狸精变的,就是会拘神遣将的,符咒也没有这样快。
”\ping{芳官最后真的是被王夫人当作狐狸精赶出怡红院了。
}又笑道:“就是你真请了神来,我也不怕。
”遂夺手仍要捉拿芳官。
芳官早已藏在宝玉身后。
\par
宝玉遂一手拖了晴雯,一手携了芳官,进入屋内。
看时,只见西边炕上麝月、秋纹、碧痕、紫绡等正在那里抓子儿赢瓜子呢。
\zhu{抓子儿赢瓜子:抓子儿,一种女孩子的游戏,“手五丸,且掷、且拾、且承,曰抓子儿”(《帝京景物略·春场》正月条)。
“丸”可用小石子、桃杏核,或小布口袋内装沙子或碎米之类,常不止五个。
玩时有一定规则,以抓接又快又准者为胜。
“赢瓜子”是一种罚约,输者要被赢家用指甲盖儿弹脑门或打手心。
}却是芳官输与晴雯,芳官不肯叫打,跑了出去。
晴雯因赶芳官,将怀内的子儿撒了一地。
宝玉欢喜道:“如此长天,我不在家,正恐你们寂寞,吃了饭睡觉,睡出病来,大家寻件事顽笑消遣甚好。
”因不见袭人,又问道:“你袭人姐姐呢?”晴雯道:“袭人么?越发道学了,独自一个在屋里面壁呢。
\zhu{面壁:因达摩坐禅,面对墙壁,所以佛家打坐又叫面壁。
}这好一会我们没进去,不知他作什么呢,一些声气也听不见。
你快瞧瞧去罢,或者此时参悟了也未可定。
”\par
宝玉听说,一面笑,一面走至里间。
只见袭人坐在近窗的床上,手中拿着一根灰色绦子,正在那里打结子呢。
见宝玉进来,连忙站起,笑道:“晴雯这东西编派我什么呢?我因要赶着打完这结子,没工夫和他们瞎闹,因哄他们道:‘你们玩去罢,趁着二爷不在家,我要在这里静坐一坐,养一养神。
’他就编派了许多混话,什么‘面壁了’‘参禅了’的,等一会我不撕他那嘴!”\par
宝玉笑着挨近袭人坐下,瞧他打结子,问道:“这么长天,你也该歇息歇息,或和他们玩去,要不,瞧瞧林妹妹去也好。
怪热的,打这个那里使?”袭人道:“我见你带的扇套还是那年东府里蓉大奶奶的事情上做的。
那个青东西除族中或亲友家夏天有丧事方带得着,一年遇着带一两遭,平常又不犯做。
如今那府里有事,这是要过去天天带的,所以我赶着另作一个。
等打完了结子,给你换下那旧的来。
你虽然不讲究这个,若叫老太太回来看见,又该说我们躲懒,连你穿带之物都不经心了。
”宝玉笑道:“这真难为你想得到。
只是也不可过于赶,热着了,倒是大事。
”说着,芳官早托了一杯凉水内新湃的茶来。
\zhu{湃:音“拔”,用冰或凉水浸泡果品或饮料等使之变冷。}
因宝玉素昔秉赋柔脆,虽暑月不敢用冰,只以新汲井水将茶连壶浸在盆内,不时更换,取其凉意而已。
宝玉就芳官手内吃了半盏,遂向袭人道:“我来时已吩咐了茗烟,若珍大哥那边有要紧人客来时,令彼即来通禀;若无甚要事,我就不过去了。
”说毕,遂出了房门,又回头向碧痕等道:“如有事,往林姑娘处来找我。
”于是一径往潇湘馆来看黛玉。
\par
将走过沁芳桥,只见雪雁领着两个老婆子,手中都拿着菱藕瓜果之类。
宝玉忙问雪雁道:“你们姑娘从来不大吃这些凉东西的,拿这些瓜果何用?莫非是要请那位姑娘、奶奶么?”雪雁笑道:“我告诉你,可不许你对姑娘说去。
”宝玉点头应允。
雪雁便命两个婆子:“先将瓜果送去交与紫鹃姐姐。
他要问我,你就说我做什么呢,就来。
”那婆子答应着去了。
雪雁方说道:“我们姑娘这两日方觉身上好些了。
今日饭后,三姑娘来,会着要瞧二奶奶去,姑娘也没去。
又不知想起甚么来,自己伤感了一会,题笔写了好些,不知是诗啊词啊。
叫我传瓜果去时,又听叫紫鹃将屋内摆着的小琴桌上的陈设搬下来,将桌子挪在外间当地,又叫将那龙文鼒
\zhu{鼒:音“资”,小鼎。
}
\qi{子之切,\zhu{子之切:一种古代表音的方法,即“反切”:用二字来表示一个字的读音。
如「东,德红切」,「德」代表「东」的声母,「红」代表「东」的韵母。
}小鼎也。
}放在桌上,等瓜果来时听用。
若说是请人呢,不犯先忙着把个炉摆出来;若说点香呢,我们姑娘素日屋内除摆新鲜花儿、木瓜、佛手之类,又不大喜熏香;就是点香,亦当点在常坐卧之处。
难道是老婆子们把屋子熏臭了,要拿香熏熏不成?究竟连我也不知何故。
”说毕,便连忙去了。
\par
宝玉这里,不由得低头细想,心内道:“据雪雁说来,必有原故。
若是同那一位姊妹们闲坐,亦不必如此先设馔具。
或者是姑爹、姑妈的忌辰,但我记得每年到此日期,老太太都吩咐另外整理肴馔,送去与林妹妹私祭,此时已过。
大约是因七月为瓜果之节,家家都上秋祭的坟,林妹妹有感于心,所以在私室自己奠祭,取《礼记》‘春秋荐其时食’之意,\zhu{春秋荐其时食:每逢春秋祭祀,向祖先进献时鲜食品。
《礼记·中庸》:“春秋,修其祖庙,陈其宗器,设其裳衣,荐其时食。
”}也未可定。
但我此刻走去,见林妹妹伤感,必极力劝解,又怕他烦恼郁结于心;若竟不去,又恐他过于伤感,无人劝止;两件皆足致疾。
莫若先到凤姐姐处一看,在彼稍坐即回。
如若见林妹妹伤感,再设法开解,既不至使其过悲,哀痛稍申,亦不至抑郁致病。
”想毕,遂出了园门,一径到凤姐处来。
\par
正有许多执事婆子们回事毕,纷纷散出。
凤姐儿正倚着门和平儿说话呢。
一见了宝玉,笑道:“你回来了么?我才吩咐了林之孝家的,叫他使人告诉跟你的小厮,若没什么事,趁便请你回来歇息歇息。
再者那里人多,你那里禁得住那些气味。
不想恰好你倒来了。
”宝玉笑道:“多谢姐姐记挂。
我也因今日没事,又见姐姐这两日没往那府里去,不知身上可大愈否,所以回来看视看视。
”凤姐道:“左右也不过是这样,三日好两日不好的。
老太太、太太不在家,这些大娘们,嗳,那一个是安分的!每日不是打架,就拌嘴,连赌博偷盗的事情都闹出来了两三件了。
虽说有三姑娘帮着办理,他又是个没出阁的姑娘。
也有好叫他知道的,也有对他说不得的事,也只好强扎挣着罢了。
总不得心静一会。
别说想病好,求其不添也就罢了。
”宝玉道:“虽如此说,姐姐还要保重身体,少操些心才是。
”说毕,又说了些闲话,别过凤姐,一直往园中走来。
\par
进了潇湘馆的院门看时,只见炉袅残烟,奠馀玉醴。
\zhu{醴:音“李”, 甜酒,甘泉。
玉醴:美酒,甘纯的泉水。
}紫鹃正看着人往里搬桌子,收陈设呢。
宝玉便知已经祭完了,走入屋内,只见黛玉面向里歪着,病体恹恹,\zhu{恹恹:音“烟烟”,困倦或忧郁的样子。
}大有不胜之态。
紫鹃连忙说道:“宝二爷来了。
”黛玉方慢慢的起来,含笑让坐。
宝玉道:“妹妹这两天可大好些了?气色倒觉静些,只是为何又伤心了?”黛玉道:“可是你没的说了,好好的我多早晚又伤心了?”宝玉笑道:“妹妹脸上现有哭泣之状,如何还哄我呢。
只是我想妹妹素日本来多病,凡事当各自宽解,不可过作无益之悲。
若作践坏了身子,将来使我……”说到这里,觉得以下的话有些难说,连忙咽住。
只因他虽说和黛玉自小一处长大,情投意合,又愿同生死,却只是心中领会,从来未曾当面说出。
况兼黛玉心重,每每因说话造次,得罪了他,致彼哭泣。
今日原为的是来劝解黛玉,不想把话来说造次了,接不下去,心中一急,又怕黛玉恼他。
又想一想自己的心实在是为好,因而转急为悲,早已滚下泪来。
黛玉起先原恼宝玉说话不论轻重,如今见此光景,心有所感,本来素昔爱哭,此时亦不免无言对泣。
\par
却说紫鹃端了茶来,打量他二人不知又为何事角口,因说道:“姑娘才身上好些,宝二爷又来怄气来了,到底是怎么样?”宝玉一面拭泪,笑道:“谁敢怄妹妹了!”一面搭讪着起来闲步。
只见砚台底下微露一纸角,不禁伸手拿起。
黛玉忙要起身来夺,已被宝玉揣在怀内,笑央道:“好妹妹!赏我看看罢。
”黛玉道:“不管什么,来了就混翻。
”\par
一语未了,只见宝钗走来,笑道:“宝兄弟要看什么?”宝玉因未见上面是何言词,又不知黛玉心中如何,未敢造次回答,却望着黛玉笑。
黛玉一面让宝钗坐,一面笑说道:“我曾见古史中有才色的女子,终身遭际,令人可喜、可羡、可悲、可叹者甚多。
今日饭后无事,因择出数人,胡乱凑几首诗,以寄感慨。
可巧探丫头来会我瞧凤姐姐去,我因身上懒懒的,没同他去。
适才做了五首,一时困倦起来,撂在那里,不想二爷来了,就瞧见了。
其实给他看也倒没有什么,但只我嫌他是不是的写了给人看去。
”\zhu{是不是:动不动、总是。
}宝玉忙道:“我多早晚给人看来呢?昨日那把扇子,原是我爱那几首白海棠的诗,所以我自己用小楷写了,不过为的是拿在手中看着便易。
我岂不知闺阁中诗词字迹是轻易往外传诵不得的?自从你说了,我总没拿出园子去。
”宝钗道:“林妹妹这虑得也是。
你既写在扇子上,偶然忘记了,拿在书房里去,被相公们看见了,岂有不问是谁做的呢。
倘或传扬开了,反为不美。
自古道‘女子无才便是德’,总以贞静为主,女工还是第二件。
其馀诗词之类,不过是闺中游戏,原可以会,可以不会。
咱们这样人家的姑娘,倒不要这些才华的名誉。
”因又笑向黛玉道:“拿出来给我看看无妨,只不叫宝兄弟拿出去就是了。
”黛玉笑道:“既如此说,连你也可以不必看了。
”\ping{宝钗刚说女孩子不必写诗,所以黛玉讽刺宝钗既然如此,作为女孩子的宝钗也不必看诗了。
}又指着宝玉笑道:“他早已抢了去了。
”宝玉听了,方自怀内取出,凑至宝钗身旁,一同细看。
只见写道:\par
\hop
西 施\par
一代倾城逐浪花,吴宫空自忆儿家。
\zhu{上句意思是一代美人随着浪花消逝了。
关于西施的结局,其说不一,较流行者有二:一说吴亡后她与范蠡同游五湖,见唐代陆广微《吴地记》引《越绝书》逸文;一说是沉水而死,见《墨子·亲士篇》。
这里用的是后一说。
倾城:代指绝世美女。
《汉书·孝武李夫人传》载李延年歌曰:“北方有佳人,绝世而独立,一顾倾人城,再顾倾人国,宁不知倾城与倾国,佳人难再得!”下句意为西施已经死了,吴宫里的人空自想念你。
儿家:你,指西施。
}\par
效颦莫笑东村女,头白溪边尚浣纱。
\zhu{这二句意思是,西施虽美,已如流水逝去,且莫笑东邻那个效颦的丑女,她却能溪边浣纱到白头。
效颦:《庄子·天运篇》说,西施邻家有一丑女,因见西施皱起眉头很好看,也学着捧心皱眉,结果显得更丑。
后人根据这类传说敷衍成了“东施效颦”的故事。
}\par
\ping{
黛玉嗟叹“一代倾城”的西施如江水东流,浪花消逝,徒然令人怀念,其命运之不幸,远在白头浣纱的“东村女”之上。
这是写她寄身贾府,虽有知己体贴,但预感病体难久的悲哀。
}
\par
\hop
虞 姬\par
肠断乌骓夜啸风,虞兮幽恨对重瞳。
\zhu{
骓[zhuī]:毛色黑白相间的马。
乌骓:项羽所骑战马名。
上句意思是听到乌骓迎风夜啸而痛断肝肠。
虞兮:指虞姬,项羽的侍妾。
重瞳:眼中有两个瞳孔。
这里代指项羽。
《史记·项羽本纪》:“又闻项羽亦重瞳子。”
虞姬:西楚霸王项羽的爱姬,常随项羽出征。
项羽被围垓下,夜闻四面楚歌,知汉军已近,乃悲歌慷慨,自作诗曰:“力拔山兮气盖世,时不利兮骓不逝。
骓不逝兮可奈何?虞兮虞兮奈若何?”虞姬和之。
见《史记·项羽本纪》。
}\par
黥彭甘受他年醢,饮剑何如楚帐中!\zhu{这两句意谓与其像黥布、彭越那样他年甘受醢刑,何如虞姬那样自刎于楚帐之中。
黥:音“擎”,指黥布,本是项羽的部将,降汉后随刘邦破楚,最后因谋反为刘邦所杀。
见《史记》。
彭:指彭越,最初起兵巨野泽中,曾助刘邦击昌邑,及归汉,封梁王。
因有人告越谋反,刘邦先降越为庶人,后诛而醢之,以其醢遍赐诸侯。
见《史记》。
醢:音“海”,本是肉、鱼等制成的酱,这里指把人制成肉酱的醢刑。
饮剑:用剑自刎。
唐代张守节《史记正义》引《楚汉春秋》记载的虞姬和项羽歌中有“大王意气尽,贱妾何聊生”之句,后人遂衍成虞姬自刎于楚帐的故事。
}\par
\ping{
黛玉鄙薄反复无常,苟且求荣、甘心得到耻辱下场的黥布、彭越,觉得不如虞美人的“饮剑”于楚帐,是借此寄托她自己“质本洁来还洁去,强于污淖陷渠沟”的志愿。
}
\par
\hop
昭 君
\par
\zhu{昭君:王昭君,即明妃。
因避晋文帝司马昭讳,改“昭”为“明”。
}\foot{据列藏本,馀本均作“明妃”。
按下文宝钗评论时亦称“昭君”。
}\par
绝艳惊人出汉宫,红颜薄命古今同。
\zhu{出汉宫:指昭君出嫁于南匈奴呼韩邪单于事。
}\par
君王纵使轻颜色,予夺权何畀画工?\zhu{下句意思是决定取舍的权力为什么要交给画工呢!予夺:赐予和剥夺;取与舍。
畀:音“毕”,交给。
传说汉元帝后宫人多,不能遍见,叫画工先画了像,然后看像选见。
宫人多向画工行贿而昭君不肯,所以像被丑化,不得召见。
匈奴求亲,元帝选昭君前去,临行召见,方知昭君很美,但已无法挽回,遂杀掉毛延寿等许多画工泄愤。
见《西京杂记》。
}\par
\ping{黛玉讥刺汉元帝大权旁落,听命于画工,表现了自己不肯听人摆布的独立性格。}
\par
\hop
绿 珠
\par
\zhu{绿珠:晋代石崇的侍妾。
《晋书·石崇传》:崇有妓曰绿珠,美而艳,善吹笛。孙秀使人求之,崇勃然曰:“绿珠吾所爱,不可得也!”
秀怒,矫诏(诈称皇帝的命令)收(捕)崇。崇正宴于楼上,介士(武士)到门,崇谓绿珠曰:“我今为尔得罪!”
绿珠泣曰:“当效死于官前。”因自投于楼下而死。
}\par
瓦砾明珠一例拋,何曾石尉重娇娆!\zhu{这两句意谓石崇把明珠(喻绿珠)当作瓦砾一样地抛弃,对姣艳美好的绿珠何尝看重过呢?石尉:即石崇,他曾作过散骑常侍、侍中、南蛮校尉等官。
}\par
都缘顽福前生造,更有同归慰寂寥。
\zhu{这两句意谓只因石崇有前生造就的“顽福”,所以虽然他并不真正看重绿珠,却得到了绿珠的真情,不仅生前供他玩乐,而且在被捕受戮时为他殉情同死,可与他作伴,使他在地府不至过于寂寞。
顽福:义同“痴福”,即旧时俗语所谓“痴人有痴福”之意,愚笨的人往往有好的福分,这里应该指石崇本不该得到却得到了的意外之福,即不看重绿珠却得到了绿珠的真情。
另一种说法,顽福是指前生修来的厚福。
同归:同死。
}\par
\ping{
黛玉惋惜绿珠而对石崇有微词,以为石崇生前珠玉绮罗之宠,抵不得绿珠临危以死相报,又可见其在爱情上重在意气相感,精神上有默契。
}
\par
\hop
红 拂
\par
\zhu{红拂:唐代杜光庭《虬髯客传》(虬髯:音“求然”,蜷曲的须髯。
)的女主人公,姓张,初为隋朝大臣杨素的侍女,后私奔李靖。
她在杨家时手执红拂(掸灰尘的用具),见李靖时又自称“红拂妓”。
}\par
长揖雄谈态自殊,美人巨眼识穷途。
\zhu{上句指李靖以布衣见杨素时长揖不拜,仪态洒脱,言谈自若,雄辩服人,和一般卑躬屈节的进谒者不同。
长揖:拱手。
下句意谓红拂眼光锐敏,能在李靖不得志时看出他将来一定有大作为。
}\par
尸居馀气杨公幕,岂得羁縻女丈夫!\zhu{尸居馀气:本形容人之将死。
这里借以极言杨素老朽不堪,无所作为。
《虬髯客传》红拂说杨素“彼尸居馀气,不足畏也。
”縻:系牛的绳索。
羁縻:引申为牵制、束缚。
}\par
\ping{
黛玉钦佩红拂卓识敢为,能不受相府权势和封建礼教的“羁糜”,更突出地表现了她大胆追求自由幸福的生活理想的封建叛逆思想。
}
\par
\hop
宝玉看了,赞不绝口,又说道:“妹妹这诗,恰好只做了五首,何不就命名曰《五美吟》。
”于是不容分说,便提笔写在后面。
\qi{《五美吟》与后《十独吟》对照。
\zhu{
《十独吟》:在本书遗失的后半部分,大概是表达古史上十个独处的女子如寡妇、弃妇、尼姑等的愁怨,另一种说法是分咏书中的十个女子的命运。
}
}宝钗亦说道:“做诗不论何题,只要善翻古人之意。
若要随人脚踪走去,纵使字句精工,已落第二义,\zhu{第二义:第二等;第二流。
}究竟算不得好诗。
即如前人所咏昭君之诗甚多,有悲挽昭君的,有怨恨延寿的,又有讥汉帝不能使画工图貌贤臣而画美人的,纷纷不一。
后来王荆公复有‘意态由来画不成,当时枉杀毛延寿’;\zhu{王荆公:即王安石,曾被封为荆国公。
}永叔有‘耳目所见尚如此,万里安能制夷狄’。
\zhu{永叔:欧阳修的字。
}二诗俱能各出己见,不袭前人。
今日林妹妹这五首诗,亦可谓命意新奇,别开生面了。
”\par
仍欲往下说时,只见有人回道:“琏二爷回来了。
适才外间传说,往东府里去了好一会了,想必就回来的。
”宝玉听了,连忙起身,迎至大门以内等待。
恰好贾琏自外下马进来。
于是宝玉先迎着贾琏跪下,口中给贾母、王夫人等请了安,又给贾琏请了安。
二人携手走了进来。
只见李纨、凤姐、宝钗、黛玉、迎、探、惜等早在中堂等候,一一相见已毕。
因听贾琏说道:“老太太明日一早到家,一路身体甚好。
今日先打发了我来回家看视,明日五更,仍要出城迎接。
”说毕,众人又问了些路途的景况。
因贾琏是远路适归,\zhu{适:刚才,刚刚。
}遂大家别过,让贾琏回房歇息。
一宿晚景,不必细述。
\par
至次日饭时前后,果见贾母、王夫人等到来。
众人接见已毕,略坐了一坐,吃了一杯茶,便领了王夫人等人过宁府中来。
只听见里面哭声震天,却是贾㻞、贾珖送贾母到家,即过这边来了。
当下贾母进入里面,早有贾赦、贾琏率领族中人哭着迎了出来。
他父子一边一个挽了贾母,走至灵前,又有贾珍、贾蓉跪着,扑入贾母怀中痛哭。
贾母暮年人,见此光景,亦搂了珍、蓉等痛哭不已。
贾赦、贾琏在旁苦劝,方略略止住。
又转至灵右,见了尤氏婆媳,不免又相持大痛一场。
哭毕,众人方上前一一请安问好。
贾珍因贾母才回家来,未得歇息,坐在此间看着,未免要伤心,遂再三求贾母回家,王夫人等亦再三相劝。
贾母不得已,方回来了。
\par
果然,年迈的人禁不住风霜伤感,至夜间,便觉头闷身酸,鼻塞声重。
连忙请了医生来诊脉下药,足足的忙乱了半夜一日。
幸而发散得快,未曾传经,\zhu{传经:中医术语,人体外感风寒通过经络传至全身叫“传经”。
}至三更天,些须发了点汗,脉静身凉,大家方放了心。
至次日仍服药调理。
又过了数日,乃贾敬送殡之期,贾母犹未大愈,遂留宝玉在家侍奉。
凤姐因未曾甚好,亦未去。
其馀贾赦、贾琏、邢夫人、王夫人等率领家人仆妇,都送至铁槛寺,至晚方回。
贾珍、尤氏并贾蓉仍在寺中守灵,等过百日后,方扶柩回籍。
家中仍托尤老娘并二姐、三姐照管。
\par
却说贾琏素日既闻尤氏姐妹之名,恨无缘得见。
近因贾敬停灵在家,每日与二姐、三姐相识已熟,不禁动了垂涎之意。
况知与贾珍、贾蓉等素有聚麀之诮,\zhu{聚麀:指父子共占一个女子的禽兽行为。
麀:音“优”,母鹿。
}因而乘机百般撩拨,眉目传情。
\ping{树倒猢狲散,人先变成了猢狲,树才会倒吧。
}那三姐却只是淡淡相对,只有二姐也十分有意,但只是眼目众多,无从下手。
贾琏又怕贾珍吃醋,不敢轻动,只好二人心领神会而已。
此时出殡以后,贾珍家下人少,除尤老娘带领二姐、三姐并几个粗使的丫鬟、老婆子在正室居住外,其馀婢妾都随在寺中。
外面仆妇,不过晚间巡更,日间看守门户,白日无事,亦不进里面去。
所以贾琏便欲趁此下手,遂托相伴贾珍为名,亦在寺中住宿,又时常借着替贾珍料理家务,不时至宁府中来勾搭二姐。
\par
一日,有小管家俞禄来回贾珍道:“前者所用棚杠孝布并请杠人青衣,\zhu{棚杠:搭建丧棚(即为举行丧祭而搭盖的棚)所用的材料。
孝布:办丧事所用的白布。
杠人:殡葬时抬运灵柩的人。
也作“杠夫”。
青衣:即皂服,黑色衣着,旧时地位低下的人所穿,后作为贱役人等的代称,如称婢女、吹鼓手和衙役等。
}共使银一千两,除给银五百两外,仍欠五百两。
昨日两处买卖人俱来催讨,奴才特来讨爷的示下。
”贾珍道:“你向库上去领就是了,这又何必来问我。
”俞禄道:“昨日已曾向库上去领,但只是老爷宾天以后,各处支领甚多,所剩还要预备百日道场及庙寺中用度,此时竟不能发给。
所以奴才今日特来回爷,或者爷内库里暂且发给,或者挪借何项,吩咐了奴才好办。
”贾珍笑道:“你还当是先呢,有银子放着不使。
你无论那里暂且借了给他罢。
”\ping{现在为了五百两欠银就要东挪西凑,然而之前秦可卿葬礼的时候,为了给贾蓉捐个官,就直接花了一千二百两。
}俞禄笑回道:“若说一二百,还可以巴结,这四五百两,一时那里办得来!”贾珍想了一想,向贾蓉道:“你问你娘去,昨日出殡以后,有江南甄家送来打祭银五百两,\zhu{打祭银:奠仪,即送给死者家属以代祭品的银钱。
}未曾交到库上去,你先要了来,给他去罢。
”贾蓉答应了,连忙过这边来,回了尤氏,复转来回他父亲道:“昨日那项银子已使了二百两,下剩的三百两,令人送至家中,交与老娘收了。
”贾珍道:“既然如此,你就带了他去,向你老娘要了出来交给他。
再也瞧瞧家中有事无事,问你两个姨娘好。
下剩的,俞禄先借了添上罢。
”\par
贾蓉与俞禄答应了,方欲退出,只见贾琏走了进来。
俞禄忙上前请了安。
贾琏便问何事,贾珍一一告诉了。
贾琏心中想道:“趁此机会,正可至宁府寻二姐。
”一面遂说道:“这有多大事,何必向人借去。
昨日我方得了一项银子,还没有使呢,莫若给他添上,岂不省事?”贾珍道:“如此甚好。
你就吩咐了蓉儿,一并令他取去。
”贾琏忙道:“这必得我亲身取去。
再我这几日没回家了,还要给老太太、老爷、太太们请请安去。
再到大哥那边查查家人们有无生事,也给亲家太太请请安。
”贾珍笑道:“只是又劳动你老二,我心不安。
”贾琏也笑道:“自家兄弟,这又何妨。
”贾珍又吩咐贾蓉道:“你跟了你叔叔去,也到那边给老太太、老爷、太太们请安,说我和你娘都请安,打听打听老太太身上可大安了,还服药呢没有?”贾蓉一一答应了,跟随贾琏出来,带了几个小厮,骑上马,一同进城。
\par
在路叔侄闲话。
贾琏有心,便提到尤二姐,因夸说如何标致,如何做人好,举止大方,言语温柔,无一处不令人可敬可爱,“人人都说你婶子好,据我看那里及你二姨一零儿呢。
”贾蓉揣知其意,便笑道:“叔叔既这么爱他,我给叔叔作媒,说了做二房何如?”贾琏笑道:“敢是好呢。
只怕你婶子不依,再也怕你老娘不愿意。
况且我听见说,你二姨已有了人家了。
”贾蓉道:“这都无妨。
我二姨、三姨都不是我老爷养的,原是我老娘带了来的。
听见说我老娘在那一家时,就把我二姨许给皇庄张家,指腹为婚。
后来张家遭了官司,败落了,我老娘又自那家嫁了出来,如今这十数年,两家音信不通。
我老娘时常抱怨,要与他家退婚,我父亲也要将二姨转聘。
只等有了好人家,不过令人找着张家,给他数两银子,写上一张退婚的字儿。
想张家穷极了的人,见了银子,有什么不依的。
再他也知道咱们这样的人家,也不怕他不依。
又是叔叔这样人说了做二房,我管保我老娘和我父亲都愿意。
倒只是婶子那里却难。
”\par
贾琏听到这里,心花都开了,那里还有什么话说,只是一味呆笑而已。
贾蓉又想了一想,笑道:“叔叔若有胆量,依我的主意行去,管保无妨,不过多花上几个钱。
”贾琏忙道:“有何主意,快些说来,我没有不依的。
”贾蓉道:“叔叔回家,一点声色也别露。
等我回明了我父亲,向我老娘说妥,然后在咱府后方近左右,买上一所房子及应用家伙什物,再拨两窝子家下人过去伏侍。
择了日子,人不知,鬼不觉,娶了过去,嘱咐家人不许走漏风声。
嫂子在里面住着,深宅大院,那里就得知道了。
叔叔两下里住着,过个一年半载,即或闹出来,不过挨上老爷一顿骂。
叔叔只说婶子总不生育,原是为子嗣起见,所以私自在外面作成此事。
就是婶子,见生米做成熟饭,也只得罢了。
再求一求老太太,没有不完的事。
”\par
自古道“欲令智昏”,贾琏只顾贪图二姐美色,听了贾蓉一篇话,遂为计出万全,\zhu{为:认为。
}将现今身上有服,并停妻再娶,严父妒妻种种不妥之处,皆置之度外了。
却不知贾蓉亦非好意,素日因同他两个姨娘有情,只因贾珍在内,不能畅意。
如今若是贾琏娶了,少不得在外居住,趁贾琏不在时,好去鬼混之意。
贾琏那里意想及此,遂向贾蓉致谢道:“好侄儿,你果然能够说成了,我买两个绝色的丫头谢你。
”说着,已至宁府门首。
贾蓉说道:“叔叔进去,向我老娘要出银子来,就交给俞禄罢。
我先给老太太请安去。
”贾琏含笑点头道:“老太太跟前,别提我和你一同来的。
”贾蓉道:“知道。
”又附耳向贾琏道:“今日要遇见二姨,可别性急了,闹出事来,往后倒难办了。
”贾琏笑道:“少胡说!你快去罢。
我在这里等你。
”于是贾蓉自去给贾母请安。
\par
贾琏进入宁府,早有家人头儿率领家人等请安,一路围随至厅上。
贾琏一一的问了些话,不过塞责而已,便命家人散去,独自往里面走来。
\par
原来贾琏、贾珍素日亲密,又是弟兄,本无可避忌之人,自来是不等通报的。
于是走至上房,早有廊下伺候的老婆子打起帘子,让贾琏进去。
贾琏进入房中一看,只见南边炕上只有尤二姐带着两个丫鬟一处做活,却不见尤老娘与三姐。
贾琏忙上前问好相见。
尤二姐亦含笑让坐,贾琏便靠东边板壁坐了,\zhu{板壁:分隔房间的木板墙。
}仍将上首让与二姐,\zhu{上首:佛说法时,于听众中推居首位者,称为“上首”。
后泛指寺院首座。
这里指上座,即位次较尊的一边。
}寒温毕,贾琏笑问道:“亲家太太和三妹妹那里去了。
怎么不见?”尤二姐笑道:“才有事往后头去了,也就来的。
”此时,伺候的丫鬟因倒茶去,无人在跟前,贾琏便睨视二姐一笑。
\zhu{睨[nì]:斜着眼看。}
二姐亦低了头,只含笑不理。
贾琏又不敢造次动手动脚,因见二姐手中拿着一条拴着荷包的手巾摆弄,便搭讪着往腰内摸了摸,说道:“槟榔荷包也忘记带了来,妹妹有槟榔,赏我一口吃。
”二姐道:“槟榔倒有,只是我的槟榔从来不给人吃。
”\ping{嚼槟榔易患口腔癌。
}\par
贾琏便笑着,欲近身来拿。
二姐怕人看见不雅,便连忙一笑,撂了过来。
贾琏接在手中,都倒了出来,拣了半块吃剩下的,撂在口中吃了,又将剩下的都揣了起来。
刚要把荷包亲身送过去,只见两个丫鬟倒了茶来。
贾琏一面接了茶吃茶,一面暗将自己带的一个汉玉九龙佩解了下来,拴在手绢上,趁丫鬟回头时,仍撂了过去。
二姐亦不去拿,只装看不见,仍坐着吃茶。
只听后面一阵帘子响,却是尤老娘、三姐带着两个小丫头自后面走来。
贾琏送目与二姐,令其拾取,这尤二姐亦只是不理。
贾琏不知二姐何意,甚是着急,只得迎上来与尤老娘、三姐相见。
一面又回头看二姐时,只见二姐笑着,没事人似的,再又看一看手巾,已不知那里去了,贾琏方放了心。
\par
于是大家归坐后,叙了些闲话。
贾琏说道:“大嫂子说,前日有一包银子交给亲家太太收起来了,今日因要还人,大哥令我来取。
再也看看家里有事无事。
”尤老娘听了,连忙使二姐拿钥匙去取银子。
这里贾琏又说道:“我也要给亲家太太请请安,瞧瞧二位妹妹。
亲家太太脸面倒好,只是二位妹妹在我们家里受委屈。
”尤老娘笑道:“咱们都是至亲骨肉,说那里的话。
在家里也是住着,在这里也是住着。
不瞒二爷说,我们家里自从先夫去世,家计也着实艰难了,全亏了这里姑爷帮助。
如今姑爷家里有了这样大事,我们不能别的出力,白看一看家还有什么委屈了的呢。
”\zhu{白:单单,只是。
}正说着,二姐已取了银子来,交与尤老娘。
尤老娘便递与贾琏。
贾琏叫一个小丫头叫了一个老婆子来,吩咐他道:“你把这个交给俞禄,叫他拿过那边去等我。
”老婆子答应了出去。
\par
只听得院内是贾蓉的声音说话。
须臾进来,给他老娘、姨娘请了安,又向贾琏笑道:“才刚老爷还问叔叔呢,说是有什么事情要使唤。
原要使人到寺里去叫,我回老爷说,叔叔就来。
老爷还吩咐我,路上遇着叔叔叫快去呢。
”贾琏听了,忙要起身,又听贾蓉和他老娘说道:“那一次我和老太太说的,我父亲要给二姨说的姨爹,就和我这叔叔的面貌身量差不多儿。
老太太说好不好?”一面说着,又悄悄的用手指着贾琏,和他二姨努嘴。
二姐倒不好意思说什么,只见三姐笑骂道:“坏透了的小猴儿崽子!没了你娘的说了,等我撕他那嘴!”一面说着,便赶了过来。
贾蓉早笑着跑了出去,贾琏也笑着辞了出来。
走至厅上,又吩咐了家人们不可耍钱吃酒等话;又悄悄的央贾蓉,回去急速和他父亲说。
一面便带了俞禄过来,将银子添足,交给他拿去;一面自己见他父亲,给贾母去请安,不提。
\par
却说贾蓉见俞禄跟了贾琏去取银子,自己无事,便仍回至里面,和他两个姨娘嘲戏一回,方起身。
至晚到寺,见了贾珍,回道:“银子已经交给俞禄了。
老太太已大愈了,如今已经不服药了。
”说毕,又趁便将路上贾琏要娶尤二姐做二房之意说了。
又说如何在外面置房子住,不使凤姐知道,“此时总不过为的是子嗣艰难起见,为的是二姨是见过的,亲上做亲,比别处不知道的人家说了来的好。
所以二叔再三央我对父亲说。
”只不说是他自己的主意。
\par
贾珍想了想,笑道:“其实倒也罢了。
只不知你二姨心中愿意不愿意。
明日你先去和你老娘商量,叫你老娘问准了你二姨,再作定夺。
”于是又教了贾蓉一篇话,便走过来,将此事告诉了尤氏。
尤氏却知此事不妥,因而极力劝止。
无奈贾珍主意已定,\ping{贾蓉替贾琏和尤二姐撮合,是为了方便自己去和尤氏姐妹鬼混,可能贾珍也是这样想的。
}素日又是顺从惯了的,况且他与二姐本非一母,不便深管,因而也只得由他们闹去了。
\ping{尤氏姐妹和贾珍妻子尤氏是异父异母的“亲”姐妹,并无血缘关系。
异父:“我二姨、三姨都不是我老爷养的,原是我老娘带了来的”;
异母:“他与二姐本非一母”。
贾蓉并非尤氏所生,和尤氏没血缘关系;尤氏姐妹虽然名义上是贾蓉的姨,但是实际上没有一点血缘关系。
}\par
至次日一早,果然贾蓉复进城来见他老娘,将他父亲之意说了,又添上许多话,说贾琏做人如何好,目今凤姐身子有病,已是不能好的了,暂且买了房子,在外住着,过个一年半载,只等凤姐一死,便接了二姨进去做正室。
又说他父亲此时如何聘,贾琏那边如何娶,如何接了你老人家养老,往后三姨也是那边应了替聘,说得天花乱坠,不由得尤老娘不肯。
况且素日全亏贾珍周济,此时又是贾珍作主替聘,
\zhu{聘:女子出嫁。}
一切妆奁不用自己置买,贾琏又是年轻公子,比张华胜强十倍,遂连忙过来合二姐商议。
二姐又是水性的人,在先已合姐夫不妥,又时常怨恨当时错许张华,使后来终身失所,今见贾琏有情,况且是姐夫将他聘嫁,有何不肯,亦便点头应允。
当下回复了贾蓉,贾蓉回了他父亲。
\par
次日,便请了贾琏到寺中来,贾珍当面告诉了他尤老娘应允之事。
贾琏自是喜出望外,又感谢贾珍、贾蓉父子不尽。
于是三人商议,使人看房子、打首饰,给二姐置买妆奁及新房中应用床帐等物。
不多几日,早将诸事办妥。
已于宁荣街后二里远近小花枝巷内买定一所房子,共二十馀间。
又买了几个小丫头。
贾珍又给了一房家人,叫鲍二夫妻两口,以备二姐过去时伏侍。
\zhu{
凤姐生日那天,鲍二媳妇与贾琏私会被捉奸后上吊而死。这个后来的鲍二老婆应是鲍二后来又娶的媳妇。
}
又使人将张华父子找来,逼着与尤老娘写了退婚书。
\par
且说张华之祖,原当皇庄,后来死了。
至张华父亲时,仍充此役,因与尤老娘前夫相好,所以将张华与二姐指腹为婚。
后来不料遭了官司,败落家产,弄得衣食不周,那里还娶得媳妇。
尤老娘又自那家嫁了出来,两家有十数年音信不通。
今被贾府家人唤来,逼他与二姐退婚,心中虽不愿意,无奈惧怕贾珍等势力,不敢不依,只得写了一张退婚文约。
尤老娘与银十两家去,不提。
\par
这里贾琏见诸事已妥,遂择了初三黄道吉日,娶二姐过门。
未知如何,下回分解。
正是:\par
只为同枝贪色欲,\zhu{同枝:贾琏与贾珍是堂兄弟,与贾蓉是叔侄,这就是所谓“同枝”。
}致教连理起戈矛。
\zhu{连理:两株树不同根而枝干结合在一起的叫“连理枝”,比喻夫妻恩爱。
这里指贾琏和凤姐的夫妻关系。
白居易《长恨歌》就有“在天愿作比翼鸟,在地愿为连理枝”的诗句。
起戈矛:贾琏偷娶尤二姐,后来凤姐知道后大闹,发展到害死尤二姐,贾琏发誓要为尤二姐报仇,也就是所谓的“起戈矛”了。
}\par
\qi{总评:五首新诗何所居,\zhu{何所居:有何深意?}颦儿应自日欷歔。
\zhu{欷歔:音“西虚”,悲泣抽噎的样子。
}柔肠一段千般结,岂是寻常望雁鱼。
\zhu{雁鱼:书信,古人有大雁传书、鱼腹藏信的故事。
这里的意思可能是黛玉盼望远方宝玉的书信。
}\hang
五百年风流债,一见了偏作怪。
你贪我爱自难休,天巧姻缘浑无奈。
\hang
父母者于子女间,莫失教训说前缘。
防微之处休弛纵,严厉才能真爱怜。
\ping{这四句诗可能是针对尤老娘不能管束尤二姐与贾琏偷情而发,
尤老娘只想着过上好日子,不知自尊自重自爱,更无心思教导女儿。女儿受到潜移默化的影响,学会了用自己的脸蛋儿和身体来换得安逸的生活。她听任女儿们与贾珍父子纠葛牵扯。
正是因为她这位不称职的母亲,想靠女儿的姿色来养老,害得两个如花的女儿先后惨死,自己也是竹篮打水一场空。
}}

\dai{127}{宝玉宝钗共读黛玉所写五美吟}
\dai{128}{浪荡子情遗九龙珮}
\sun{p64-1}{贾蓉调情怀里告饶,幽淑女悲题五美吟}{图右侧:贾蓉与二位姨娘调笑,见了尤二姐说:“我们父亲正想你呢。
”尤二姐便红了脸,顺手拿起一个熨斗来,搂头就打,吓的贾蓉抱着头滚到怀里告饶。
图左侧:宝玉来看望黛玉,宝钗也来了,二人共赏黛玉新作的《五美吟》,赞叹不已。
}