\chapter{听曲文宝玉悟禅机\quad 制灯谜贾政悲谶语}
\qi{禅理偏成曲调,灯谜巧隐谶言。
其中冷暖自寻看,昼夜因循暗转。
\zhu{
本回通过宝玉悟禅,暗伏宝玉抛弃宝钗出家为僧的后文。贾家众人所作灯谜又暗伏了其悲剧结局。
这种故事布局包含着全书经常点示的盛衰荣辱变幻无常的思想,所以说“其中冷暖自寻看,昼夜因循暗转”。
“冷暖”和“昼夜”都是比喻兴与败、荣与衰。
}
}\par
话说贾琏听凤姐儿说有话商量,因止步问是何话。
凤姐道:“二十一是薛妹妹的生日,\geng{好!}你到底怎么样呢?”贾琏道:“我知道怎么样!你连多少大生日都料理过了,这会子倒没了主意?”凤姐道:“大生日料理,不过是有一定的则例在那里。
如今他这生日,大又不是,小又不是,所以和你商量。
”\geng{有心机人在此。
\zhu{
凤姐在挪用月钱私放高利贷、承揽张家退亲诉讼官司等大事上恣意妄为,不跟贾琏商量,却在过生日这样的芥微小事给贾琏当家作主的虚假面子。
}
}贾琏听了,低头想了半日道:“你今儿糊涂了。
现有比例,那林妹妹就是例。
往年怎么给林妹妹过的,如今也照依给薛妹妹过就是了。
”\geng{比例引的极是。
无怪贾政委以家务也。
}凤姐听了,冷笑道:“我难道连这个也不知道?我原也这么想定了。
但昨儿听见老太太说,问起大家的年纪生日来,听见薛大妹妹今年十五岁,虽不是整生日,也算得将笄之年。
\zhu{笄:音“机”,用金属、玉石、骨角等物制成的别头发用的一种簪子。
古代女子十五岁而始戴笄,表示成年,可以许嫁。
故后来称女子到了成年为“及笄”或“将笄之年”。
}老太太说要替他作生日。
想来若果真替他作,自然比往年与林妹妹的不同了。
”贾琏道:“既如此,比林妹妹的多增些。
”凤姐道:“我也这么想着,所以讨你的口气。
我若私自添了东西,你又怪我不告诉明白你了。
”贾琏笑道:“罢,罢,这空头情我不领。
你不盘察我就够了,我还怪你!”说着,一径去了,不在话下。
\geng{一段题纲写得如见如闻,且不失前篇惧内之旨。
最奇者黛玉乃贾母溺爱之人也,不闻为作生辰,却去特意与宝钗,实非人想得着之文也。
此书通部皆用此法,瞒过多少见者,余故云不写而写是也。
}\geng{将薛、林作甄玉、贾玉,看书则不失执笔人本旨矣。
丁亥夏。
畸笏叟。
}\ping{全书只见薛宝钗过生日,而不见林黛玉过生日,一种说法是薛宝钗被贾母当作客人,而林黛玉被当作自己人,证据是在本回写有“就在贾母上房排了几席家宴酒席,并无一个外客,只有薛姨妈、史湘云、宝钗是客,馀者皆是自己人。
”,对于客人出于客套办了生日宴席,而且宝钗过十五岁生日,属于相对比较重要的生日,所以贾母才大操大办。
但是这种说法问题在于,本书浓墨重彩地在六十二回和六十三回写了贾宝玉过生日的盛况,并且第七十回也写了探春过生日}
\ping{后文的甄宝玉(真宝玉)和贾宝玉(假宝玉)互为补充衬托,这里把“薛、林作甄玉、贾玉”,可能是说薛宝钗和林黛玉类似于真假宝玉之间的这种关系,两人合二为一,互为补充,即“钗黛合一”。
证据一是第五回在同一首诗里写了薛宝钗和林黛玉两个人的判词,可见钗黛两人在作者心目中具有同样的地位,并列红楼梦第一女主角,两人身上凝结着虽有差异但互为补充的女性美好品质;证据二是四十二回脂评如下“钗、玉名虽二个,人却一身,此幻笔也。
今书至三十八回时,已过三分之一有馀,故写是回,使二人合而为一。
请看黛玉逝后宝钗之文字,便知余言不谬矣”,可见作者可能分拆一个写作原型的两个不同性格分别为黛玉和宝钗,故有“名虽二个,人却一身”的总结,即两人的现实原型是一个人。
证据三是根据故事情节的发展,宝钗黛玉从互相吃醋对立的情敌,逐渐成为知心姐妹,直至第四十五回,两人“金兰契互剖金兰语”,彻底冰释前嫌,钗黛从对立的两人到统一的姐妹,反映的是“钗黛合一”的过程。
}
\par
且说史湘云住了两日,因要回去。
贾母因说:“等过了你宝姐姐的生日,看了戏再回去。
”史湘云听了,只得住下。
又一面遣人回去,将自己旧日作的两色针线活计取来,
\zhu{色:品类;种类。活计:女红[gōng],即旧时指妇女所做的缝纫、刺绣、纺织一类的劳动或这类劳动的成品。}
为宝钗生辰之仪。
\par
谁想贾母自见宝钗来了,喜他稳重和平,\geng{四字评倒黛玉,是以特从贾母眼中写出。
}正值他才过第一个生辰,便自己蠲资二十两,\zhu{蠲:通“捐”。}\geng{写出太君高兴,世家之常事耳。
}\geng{前看凤姐问作生日数语甚泛泛,至此见贾母蠲资,方知作者写阿凤心机无丝毫漏笔。
己卯冬夜。
}唤了凤姐来,交与他置酒戏。
凤姐凑趣笑道:“一个老祖宗给孩子们作生日,\geng{家常话,却是空中楼阁,陡然架起。
}不拘怎样,谁还敢争,又办什么酒戏。
既高兴要热闹,就说不得自己花上几两。
巴巴的找出这霉烂的二十两银子来作东西,这意思还叫我赔上。
果然拿不出来也罢了,金的、银的、圆的、扁的,压塌了箱子底,\geng{小科诨解颐,\zhu{
诨[hùn]:指古代戏曲中逗笑的台词。
科诨:插科打诨的简称,指穿插在戏曲中令人发笑的滑稽动作和对话。
解颐:脸上露出笑容(颐:面颊)。
}却为借当伏线。
\zhu{借当:第七十二回,贾琏因资金周转困难求鸳鸯偷出贾母的“金银家伙”来。}
壬午九月。
}只是勒掯我们。
\zhu{勒掯[kèn]:强行索讨、敲诈,刁难。
}举眼看看,谁不是儿女?难道将来只有宝兄弟顶了你老人家上五台山不成?\zhu{顶了你老人家上五台山:旧俗出殡,主丧的“孝子”在灵前头顶铭旌,持幡领路,叫作“顶灵”。
这里的“顶”,即“顶灵”。
五台山:在山西省五台县,是我国古代佛教“圣地”之一。
这里因为不好直说死,就用“上五台山”暗喻“死后登仙成佛”。
}那些梯己只留于他,\zhu{梯己:意即私人的、贴心的。
私蓄亦可称作“梯己”。
}我们如今虽不配使,也别苦了我们。
这个够酒的?够戏的?”说的满屋里都笑起来。
贾母亦笑道:“你们听听这嘴!我也算会说的,怎么说不过这猴儿。
你婆婆也不敢强嘴,
\zhu{强嘴:现在一般写作“犟嘴”。}
你和我\bang\bang 的。
”凤姐笑道:“我婆婆也是一样的疼宝玉,我也没处去诉冤,倒说我强嘴。
”说着,又引着贾母笑了一回,\geng{正文在此一句。
}贾母十分喜悦。
\par
到晚间,众人都在贾母前,定昏之馀,
\zhu{
定昏:子女对父母早上问安叫“省”,晚上服侍就寝叫“定”。
见《礼记·曲礼上》:“凡为人子之礼,冬温而夏清(音“净”),昏定而晨省。
}
大家娘儿姊妹等说笑时,贾母因问宝钗爱听何戏,爱吃何物等语。
宝钗深知贾母年老人,喜热闹戏文,爱吃甜烂之食,便总依贾母往日素喜者说了出来。
\geng{看他写宝钗,比颦儿如何?}贾母更加欢悦。
次日便先送过衣服玩物礼去,王夫人、凤姐、黛玉等诸人皆有随分不一,不须多记。
\par
至二十一日,就贾母内院中搭了家常小巧戏台,\geng{另有大礼所用之戏台也,侯门风俗断不可少。
}定了一班新出小戏,昆弋两腔皆有。
\zhu{昆弋(弋音“异”)两腔:两种戏曲声腔。
昆腔即昆山腔,起源于江苏昆山县。
用昆腔唱的词曲(包括南曲和北曲)叫作昆曲。
弋腔,即弋阳腔,起源于江西弋阳县。
}\geng{是贾母好热闹之故。
}就在贾母上房排了几席家宴酒席,\geng{是家宴,非东阁盛设也。
\zhu{
东阁:向东小门。《汉书·公孙弘传》:「数年至宰相封侯,于是起客馆,开东阁以延贤人,与参谋议。」
引申为款待宾客之地。
}
非世代公子再想不及此。
}并无一个外客,只有薛姨妈、史湘云、宝钗是客,馀者皆是自己人。
\geng{将黛玉亦算为自己人,奇甚!}\ping{林黛玉和薛宝钗都不姓贾,也不是贾家的媳妇,从父系家族的角度看,都不算是贾家自己人。
}这日早起,宝玉因不见林黛玉,\geng{又转至黛玉文字,人不可少也。
}便到他房中来寻,只见林黛玉歪在炕上。
宝玉笑道:“起来吃饭去,就开戏了。
你爱看那一出?我好点。
”林黛玉冷笑道:“你既这样说,你特叫一班戏来,拣我爱的唱给我看。
这会子犯不上跐着人借光儿问我。
”\zhu{跐[cǐ]:蹬踏,也可引申为仰仗。
}\geng{好听之极,令人绝倒。
}宝玉笑道:“这有什么难的。
明儿就这样行,也叫他们借咱们的光儿。
”一面说,一面拉起他来,携手出去吃了饭。
\par
点戏时,贾母一定先叫宝钗点。
宝钗推让一遍,无法,只得点了一折《西游记》。
\geng{是顺贾母之心也。
}贾母自是欢喜,然后便命凤姐点。
凤姐亦知贾母喜热闹,更喜谑笑科诨,\zhu{
诨[hùn]:指古代戏曲中逗笑的台词。
科诨:插科打诨的简称,指穿插在戏曲中令人发笑的滑稽动作和对话。
}\geng{写得周到,想得奇趣,实是必真有之。
}便点了一出《刘二当衣》。
\geng{凤姐点戏,脂砚执笔事,今知者寥寥矣,不怨夫?\ping{“脂砚执笔”令人费解,可能的解释是,脂砚(脂砚斋)代替作者曹雪芹执笔写了凤姐点戏这段情节。
从脂评可以看出,脂砚斋和曹雪芹有着很多的共同经历,发出很多物是人非的感慨,和作者关系密切。
因此,脂砚斋评论之余参与一点创作也是合乎情理的。
}}\geng{前批“知者寥寥”,今丁亥夏只剩朽物一枚,宁不悲乎!}贾母果真更又喜欢,然后便命黛玉点。
\geng{先让凤姐点者,是非待凤先而后玉也。
盖亦素喜凤嘲笑得趣之故,今故命彼点,彼亦自知,并不推让,承命一点,便合其意。
此篇是贾母取乐,非礼筵大典,故如此写。
}黛玉因让薛姨妈王夫人等。
贾母道:“今日原是我特带着你们取笑,咱们只管咱们的,别理他们。
我巴巴的唱戏摆酒,为他们不成?他们在这里白听白吃,已经便宜了,还让他们点呢!”说着,大家都笑了。
黛玉方点了一出。
\geng{不提何戏,妙!盖黛玉不喜看戏也。
正是与后文“妙曲警芳心”留地步,
\zhu{妙曲警芳心:第二十三回。}
正见此时不过草草随众而已,非心之所愿也。
}然后宝玉、史湘云、迎、探、惜、李纨等俱各点了,接出扮演。
\par
至上酒席时,贾母又命宝钗点。
宝钗点了一出《鲁智深醉闹五台山》。
\zhu{《鲁智深醉闹五台山》,又叫《山门》或《醉打山门》,演《水浒》中鲁智深打死恶霸郑屠后,在五台山出家避难,因不守佛门清规,破戒醉酒,大闹寺院山门,被他师父智真长老打发离山的故事。
}
宝玉道:“只好点这些戏。
”宝钗道:“你白听了这几年的戏,那里知道这出戏的好处,排场又好,词藻更妙。
”宝玉道:“我从来怕这些热闹。
”宝钗笑道:“要说这一出热闹,你还算不知戏呢。
\geng{是极!宝钗可谓博学矣,不似黛玉只一《牡丹亭》便心身不自主矣。
真有学问如此,宝钗是也。
\ping{钗粉黛黑。
}}
你过来,我告诉你,这一出戏热闹不热闹。
——是一套北《点绛唇》,铿锵顿挫,韵律不用说是好的了,只那词藻中有一支《寄生草》,\zhu{在《山门》中为鲁智深拜别师父时所唱。
}填的极妙,你何曾知道。
”宝玉见说的这般好,便凑近来央告:“好姐姐,念与我听听。
”宝钗便念道:\par
\hop
漫揾英雄泪,\zhu{漫:随意,不经意。
揾:音“问”,揩拭。
}相离处士家。
\zhu{处士:不做官的隐居之士。
这里指七宝村的赵员外。
鲁智深打死郑屠后,先在他家避难,后因走漏风声,只得离开那里去五台山出家当和尚。
}谢慈悲剃度在莲台下。
\zhu{剃度:佛家语。
指佛教徒剃去须发,接受戒条,出家为僧的仪式。
佛教认为这是超度人们脱离生死苦难之始,故称“剃度”。
莲台:也叫“莲华台”,即佛像所坐的莲花状台座。
}没缘法转眼分离乍。
\zhu{缘法:缘分;机缘。
乍:这里是仓促之意。
}赤条条来去无牵挂。
那里讨烟蓑雨笠卷单行?\zhu{烟蓑雨笠:即蓑衣斗笠。
卷单行:即离寺而去。
游方僧入寺寄寓,须先将衣钵袋挂在僧堂的钩上,得到住持的许可,才能住下,这种手续叫“挂褡”,也叫“挂单”。
离寺就叫“卷单”。
“单”即僧人的执照。
}一任俺芒鞋破钵随缘化!\zhu{芒鞋:草鞋。
随缘化:即随机缘而求人布施,这里有随遇而安的意思。
化:即化缘,指僧徒向人劝募乞讨。
}\geng{此阕出自《山门》传奇。
\zhu{
阕[què]:量词。歌曲或词一首或词一段叫一阕。
}
近之唱者将“一任俺”改为“早辞却”,无理不通之甚。
必从“一任俺”三字,则“随缘”二字方不脱落。
\zhu{脱落:(文字)遗漏。}
}\par
\hop
宝玉听了,喜的拍膝画圈,
\zhu{画圈:意思不太明确,是用手画圈,还是脑袋在画圈呢?程乙本改为“摇头”。}
称赏不已,又赞宝钗无书不知,林黛玉道:“安静看戏罢,还没唱《山门》,
\zhu{《山门》:即宝钗点的《鲁智深醉闹五台山》。}
你倒《妆疯》了。
”\zhu{《妆疯》:北曲折子戏,演唐代尉迟敬德因不肯挂帅出征而假装疯病的故事。
本元代无名氏杂剧《功臣宴敬德不伏老》第三折,俗称《妆疯》。
}\geng{趣极!今古利口莫过于优伶。
此一诙谐,优伶亦不得如此急速得趣,可谓才人百技也。
一段醋意可知。
}说的湘云也笑了。
于是大家看戏。
\par
至晚散时,贾母深爱那作小旦的与一个作小丑的,\zhu{生是扮演男性人物。
旦是扮演女性人物的。
净是花脸的化妆,脸上都要勾画五颜六色的脸谱。
表演动作较为夸张,唱念嗓音宽粗洪亮,扮演的都是性格豪爽、刚强,或鲁莽、勇猛,或奸诈、凶残的男性人物。
丑角的化妆,一般都在鼻梁上勾一块粉白,所扮演的多是性格诙谐或品行不正的人物,有忠有奸,有愚有贤,有善有恶,性格各异。
}因命人带进来,细看时益发可怜见。
\geng{是贾母眼中之见、心内之想。
}因问年纪,那小旦才十一岁,小丑才九岁,大家叹息一回。
\ping{这体现了古代戏子地位低,后文湘云说戏子像黛玉导致生气方不突然。}
贾母令人另拿些肉果与他两个,又另外赏钱两串。
凤姐笑道:“这个孩子扮上活像一个人,\geng{明明不叫人说出。
}你们再看不出来。
”宝钗心里也知道,便只一笑,不肯说。
\geng{宝钗如此。
}宝玉也猜着了,亦不敢说。
\geng{不敢少。
}史湘云接着笑道:“倒像林妹妹的模样儿。
”\geng{口直心快,无有不可说之事。
}\geng{事无不可对人言。
}
\geng{湘云、探春二卿,正“事无不可对人言”之性。
丁亥夏。
畸笏叟。
}宝玉听了,忙把湘云瞅了一眼,使个眼色。
众人却都听了这话,留神细看,都笑起来了,说果然不错。
一时散了。
\par
晚间,湘云更衣时,便命翠缕把衣包打开收拾,都包了起来。
翠缕道:“忙什么,等去的日子再包不迟。
”湘云道:“明儿一早就走。
在这里作什么?——看人家的鼻子眼睛,什么意思!”\geng{此是真恼,非颦儿之恼可比,然错怪宝玉矣。
亦不可不恼。
}宝玉听了这话,忙赶近前拉他说道:“好妹妹,你错怪了我。
林妹妹是个多心的人。
别人分明知道,不肯说出来,也皆因怕他恼。
谁知你不防头就说了出来,\zhu{不防头:冒失,不留神、不经意。
}他岂不恼你。
我是怕你得罪了他,所以才使眼色。
你这会子恼我,不但辜负了我,而且反倒委曲了我。
若是别人,那怕他得罪了十个人,与我何干呢。
”湘云摔手道:
\zhu{摔手:甩动手臂。}
“你那花言巧语别哄我。
我也原不如你林妹妹,别人说他,拿他取笑都使得,只我说了就有不是。
我原不配说他。
他是小姐主子,我是奴才丫头,得罪了他,使不得!”宝玉急的说道:“我倒是为你,反为出不是来了。
我要有外心,\geng{玉兄急了。
}立刻就化成灰,叫万人践踹!”\geng{千古未闻之誓,恳切尽情。
宝玉此刻之心为如何?}湘云道:“大正月里,少信嘴胡说。
\geng{回护石兄。
}这些没要紧的恶誓、散话、歪话,\zhu{散话:闲话。
}说给那些小性儿、行动爱恼的人,会辖治你的人\geng{此人为谁?}听去!别叫我啐你。
”说着,一径至贾母里间,忿忿的躺着去了。
\par
宝玉没趣,只得又来寻黛玉。
刚到门槛前,黛玉便推出来,将门关上。
宝玉又不解其意,在窗外只是吞声叫“好妹妹”。
黛玉总不理他。
宝玉闷闷的垂头自审。
袭人早知端的,当此时断不能劝。
\geng{宝玉在此时一劝必崩了,袭人见机甚妙。
}那宝玉只是呆呆的站在那里。
\par
黛玉只当他回房去了,便起来开门,只见宝玉还站在那里。
黛玉反不好意思,不好再关,只得抽身上床躺着。
宝玉随进来问道:“凡事都有个原故,说出来,人也不委曲。
好好的就恼了,终是什么原故起的?”林黛玉冷笑道:“问的我倒好,我也不知为什么原故。
我原是给你们取笑的,——拿我比戏子取笑!”\ping{古代戏曲演员地位低下。
}宝玉道:“我并没有比你,我并没笑,为什么恼我呢?”黛玉道:“你还要比?你还要笑?你不比不笑,比人比了笑了的还利害呢!”\geng{可谓“官断十条路”是也。
\zhu{
断:断案、审案。
官断十条路:旧指官员断案的标准不一,往往多方揣测上司意图,或看送礼,或看背景,所以判决有多种结果,外人很难推测。
这条批语的意思大概是,黛玉“审理”宝玉的“案子”,难以预料其结果。无论宝玉怎么说,都能找到宝玉的错。
}
}宝玉听说,无可分辩,不则一声。
\geng{何便无言可辩?真令人不解。
前文湘云方来,“正言弹妒意”一篇中,颦、玉角口,后收至“褂子”一篇,\zhu{“褂子”一篇:第二十回,宝玉黛玉角口,宝玉脱掉青肷披风,黛玉叹道:“回来伤了风,又该饿着吵吃的了。
”}余已注明不解矣。
回思自心、自身是玉、颦之心,则洞然可解,否则无可解也。
身非宝玉,则有辩有答;若是宝玉,则再不能辩不能答。
何也?总在二人心上想来。
}\geng{此书如此等文章多多不胜枚举,机括神思自从天分而有。
其毛锥写人口气传神摄魄处,\zhu{毛锥:毛笔。
}怎不令人拍案称奇叫绝!丁亥夏。
畸笏叟。
}\par
黛玉又道:“这一节还恕得。
再你为什么又和云儿使眼色?这安的是什么心?莫不是他和我顽,他就自轻自贱了?他原是公侯的小姐,我原是贫民的丫头,他和我顽,设若我回了口,岂不他自惹人轻贱呢。
\ping{黛玉湘云皆觉着对方身份更高贵,敏感于自己的身份,这种自卑可能是因为二人都是小小年纪而失去父母,无父母庇佑。
这里两人怄气是暂时的,两个孤儿同命相怜,在第七十六回“凹晶馆联诗悲寂寞”中互相抚慰不幸的身世。
}是这主意不是?这却也是你的好心,只是那一个偏又不领你这好情,一般也恼了。
\zhu{一般:一样,同样。
}\geng{颦儿自知云儿恼,用心甚矣!}你又拿我作情,倒说我小性儿,\geng{颦儿却又听见,用心甚矣!}行动肯恼。
\zhu{肯:表示时常、易于。
}你又怕他得罪了我,我恼他。
我恼他,与你何干?他得罪了我,又与你何干?”\geng{问的却极是,但未必心应。
若能如此,将来泪尽夭亡已化乌有,世间亦无此一部《红楼梦》矣。
}\geng{神工乎,鬼工乎?文思至此尽矣。
丁亥夏。
畸笏。
}\par
宝玉见说,方才与湘云私谈,他也听见了。
细想自己原为他二人,怕生隙恼,方在中调和,不想并未调和成功,反已落了两处的贬谤。
正合着前日所看《南华经》上,有“巧者劳而智者忧,无能者无所求,饱食而遨游,泛若不系之舟”,\zhu{“巧者……之舟”语出《庄子·列御寇》。
意思是心灵手巧的人总是辛苦劳碌,聪明智慧的人总是多思多虑,而无挂无碍的人则什么也不追求,吃饱了就任兴漫游,好像没有缆绳拴住的小船,自由自在地随水漂流。
无能:在这里是自忘其能,无所挂碍的意思。
}又曰“山木自寇,
\zhu{
“山木自寇”:语出《庄子·人间世》。意思是山中的树木因成材而招人来砍伐。
}
\geng{按原注:“山木,漆树也。
精脉自出,岂人所使之?故云‘自寇’,言自相戕贼也。”
\zhu{这条批语的意思是,山木是自己长成的,并非人力所为。因自己长大成材而被砍伐,所以说是“自相戕贼”。}
}源泉自盗”等语。
\zhu{
“源泉自盗”似从《庄子·山木》中“甘井先竭”之句化出,意思是源泉之水因甘美而惹人来盗饮。
}\geng{源泉味甘,然后人争取之,自寻干涸也,亦如山木意,皆寓人智能聪明多知之害也。
前文无心云看《南华经》,不过袭人等恼时,无聊之甚,偶以释闷耳。
殊不知用于今日,大解悟大觉迷之功甚矣。
市徒见此必云:前日看的是外篇《胠箧》,
\zhu{
胠:音“区”,从旁打开(器物)。
箧:音“妾”,小箱子。
}
如何今日又知若许篇?然则彼时只曾看外篇数语乎?想其理,自然默默看过几篇,适至外篇,故偶触其机,方续之也。
若云只看了那几句便续,则宝玉彼时之心是有意续《庄子》,并非释闷时偶续之也。
且更有见前所续,则曰续的不通,更可笑矣。
试思宝玉虽愚,岂有安心立意与庄叟争衡哉?且宝玉有生以来,此身此心为诸女儿应酬不暇,眼前多少现成有益之事尚无暇去做,岂忽然要分心于腐言糟粕之中哉?可知除闺阁之外,并无一事是宝玉立意作出来的。
大则天地阴阳,小则功名荣枯,以及吟篇琢句,皆是随分触情。
偶得之,不喜;失之,不悲。
若当作有心,谬矣。
只看大观园题咏之文,已算平生得意之句得意之事矣,然亦总不见再吟一句,再题一事,据此可见矣。
然后可知前夜是无心顺手拈了一本《庄子》在手,且酒兴\sout{醮醮}[醺醺],芳愁默默,顺手不计工拙,草草一续也。
若使顺手拈一本近时鼓词,或如“钟无艳赴会”“齐太子走国”等草野风邪之传,
\zhu{
钟无艳赴会:战国时无盐(脂批为“艳”字)邑有女名钟离春,貌极丑,年四十犹未嫁,自谒齐宣王,
陈述齐国面临的四点危难,宣王善其言,并纳为后,尽反旧时所为,从钟离春教而齐国以安。
齐太子走国:春秋战国时齐国有几个太子走国(跑到别的国家),如齐孝公昭、齐桓公小白等。
齐桓公小白为齐襄公之弟,因襄公无道而出奔莒,后襄公被弑,乃归国即位。鲍叔牙荐管仲,桓公任以为相。齐国渐强大,为春秋五霸之首。
}
必亦续之矣。
观者试看此批,然后谓余不谬。
所以可恨者,彼夜却不曾拈了《山门》一出传奇。
若使《山门》在案,彼时拈着,又不知于《寄生草》后续出何等超凡入圣大觉大悟诸语录来。
}\geng{黛玉一生是聪明所误,宝玉是多事所误。
多事者,情之事也,非世事也。
多情曰多事,亦宗《庄》笔而来,盖余亦偏矣,可笑。
阿凤是机心所误,宝钗是博识所误,湘云是自爱所误,袭人是好胜所误,皆不能跳出庄叟言外,悲亦甚矣。
再笔。
\zhu{
庄叟言:指的是前文所述庄子《南华经》“巧者劳而智者忧,无能者无所求”、“山木自寇,源泉自盗”等语。
人皆知有用之用,却不知无用之用也。例如树木因“有材”而遭到人们的砍伐,源泉之水因为甘美而惹人来盗饮。
在庄子看来,“巧者”、“智者”、“山木”、“源泉”都是属于“有材”的,因为其“有材”反招致不幸,只有“无能者”可以平安无事。
宝玉想起庄子的文句,是感慨自己因为多事惹恼了黛玉和湘云。
对于造成这些女子的悲惨命运的原因,曹雪芹可能试图从庄子的“无材”的观点加以解释,他认为这些秉承了天地灵气的女儿们正是为“有材”而累。
}
}因此越想越无趣。
再细想来,目下不过这两个人,尚未应酬妥协,将来犹欲为何?\geng{看他只这一笔,写得宝玉又如何用心于世道。
言闺中红粉尚不能周全,何碌碌僭欲治世待人接物哉?\zhu{碌碌:平庸无能。
僭[jiàn]:超越本分,过分。
}视闺中自然如儿戏,视世道如虎狼矣,谁云不然?}想到其间也无庸分辩回答,自己转身回房来。
\geng{颦儿云“与你何干”,宝玉如此一回则曰“与我何干”可也。
口虽未出,心已悟矣,但恐不常耳。
若常存此念,无此一部书矣。
看他下文如何转折。
}林黛玉见他去了,便知回思无趣,赌气去了,一言也不曾发,不禁自己越发添了气,\geng{只此一句又勾起波浪。
去则去,来则来,又何气哉?总是断不了这根孽肠,忘不了这个祸害,既无而又有也。
}便说道:“这一去,一辈子也别来,也别说话。
”
\ping{现在的气话,也是预示未来的谶语。宝玉现在一去,犹能回来,但未来某时一去,就来不及回来再说话了。}
\par
宝玉不理,\geng{此是极心死处,将来如何?}回房躺在床上,只是瞪瞪的。
袭人深知原委,不敢就说,\geng{一说必崩。
}\qi{一说就恼。
}只得以他事来解释,
\zhu{解释:劝解疏通。}
因说道:“今儿看了戏,又勾出几天戏来。
宝姑娘一定要还席的。
”\zhu{还席:受人邀宴后,设酒席回请对方。
}
宝玉冷笑道:“他还不还,管谁什么相干。
”\geng{大奇大神之文。
此“相干”之语仍是近文与颦儿之语之“相干”也。
上文未说,终存于心,却于宝钗身上发泄。
素厚者唯颦、云,今为彼等尚存此心,况于素不契者有不直言者乎?
\zhu{契:符合;投合。}
情理笔墨,无不尽矣。
}袭人见这话不是往日的口吻,因又笑道:“这是怎么说?好好的大正月里,娘儿们姊妹们都喜喜欢欢的,你又怎么这个形景了?”宝玉冷笑道:“他们娘儿们姊妹们欢喜不欢喜,也与我无干。
”\geng{先及宝钗,后及众人,皆一颦之祸流毒于众人。
\ping{这个批书人对黛玉仇怨太大了。}
宝玉之心仅有一颦乎?}袭人笑道:“他们既随和,你也随和,岂不大家彼此有趣。
”宝玉道:“什么是‘大家彼此’!他们有‘大家彼此’,我是‘赤条条来去无牵挂’。
”\geng{拍案叫好!当此一发,西方诸佛亦来听此棒喝,参此语录。
}谈及此句,不觉泪下。
\geng{还是心中不静、不了、斩不断之故。
}袭人见此光景,不肯再说。
宝玉细想这句趣味,不禁大哭起来,\geng{此是忘机大悟,\zhu{忘机:道家语,意为消除机巧之心。
用以指淡泊清净,忘却世俗烦庸,与世无争。
}世人所谓疯癫是也。
}翻身起来至案,遂提笔立占一偈云:\zhu{占:随口成文。
偈(音“记”):梵语音译词“偈陀(Gatha)”的简称。佛经中的唱词;泛指与佛教有关或带有佛教色彩的诗作。
}\par
\hop
你证我证,心证意证。
\par
是无有证,斯可云证。
\par
无可云证,是立足境。
\geng{已悟已觉,是好偈矣。
}\geng{宝玉悟禅亦由情,读书亦由情,读《庄》亦由情。
可笑。
}\par
\zhu{证:印证;证验。
佛教用语中又作领悟解。
此偈用意双关,既是谈禅,也是说情。
就谈禅来说,无求于身外,不要证验,才谈得上参悟禅机,证得上乘。
禅宗是宣扬极端主观唯心论的。它认为宇宙间的一切都随人心的生灭而生灭,佛就在每个人的心中,天堂地狱也在每个人的心里,毋须向外追求,不必求外界证验,因为万境皆空,本无证验可言。
就说情来说,其大意是:彼此都想从对方得到感情的印证而频添烦恼;看来只有到了灭绝情意,无须再证验时,方谈得上感情上的彻悟;到了万境归空,什么都无可证验之时,才是真正的立足之境。
后文黛玉所续二句“无立足境,是方干净”的意思是:连立足之境也没有,那才是真止的干净了。
此偈既是禅理,也是谶语。后来,宝玉流落在外,讯息杳不可闻,以至他最终“悬崖撒手”,与世缘断绝,
都应了“无可云证”的话;而黛玉所说的“无立足境”,则是为她泪尽夭亡作谶。
}\par
\hop
写毕,自虽解悟,又恐人看此不解,\geng{自悟则自了,又何用人亦解哉?此正是犹未正觉大悟也。
}因此亦填一支《寄生草》,也写在偈后。
\geng{此处亦续《寄生草》。
余前批云不曾见续,今却见之,是意外之幸也。
盖前夜《庄子》是道悟,此日是禅悟,天花散漫之文也。
}自己又念一遍,自觉无挂碍,中心自得,便上床睡了。
\geng{前夜已悟,今夜又悟,二次翻身不出,故一世堕落无成也。
不写出曲文何辞,却留于宝钗眼中写出,是交代过节也。
\zhu{
过节:度过节日;在节日庆贺作乐;嫌隙纠纷。
“过节”用在此处,可能是指随分从时的宝钗与萌生离世参禅念头的宝玉的矛盾。
从宝玉“弃而为僧”的结局可以反推,宝玉会再次萌生离世出家的念头,这时候宝钗没能劝解成功。
}}\par
谁想黛玉见宝玉此番果断而去,故以寻袭人为由,来视动静。
\geng{这又何必?总因慧刀不利,未斩毒龙之故也。
\ping{剪不断的牵挂,藕断丝连。
}大都如此,叹叹!}袭人笑回:“已经睡了。
”黛玉听说,便要回去。
袭人笑道:“姑娘请站住,有一个字帖儿,瞧瞧是什么话。
”说着,便将方才那曲子与偈语悄悄拿来,递与黛玉看。
黛玉看了,知是宝玉一时感忿而作,不觉可笑可叹,\geng{是个善知觉。
何不趁此大家一解,齐证上乘,\zhu{
证:印证;证验。
佛教用语中又作领悟解。
上乘:上等而高妙的境界。
也可能是指上乘佛教。
}甘心堕落迷津哉?
\zhu{迷津:佛教指使人迷惘的境界;泛指错误的道路、方向。}
}便向袭人道:“作的是玩意儿,无甚关系。
”\geng{黛玉说“无关系”,将来必无关系。
}\ping{黛玉已经成竹在胸,想好了如何让宝玉回心转意的方法。
}\geng{余正恐颦、玉从此一悟则无妙文可看矣。
不想颦儿视之为漠然,更曰“无关系”,可知宝玉不能悟也。
余心稍慰。
盖宝玉一生行为,颦知最确,故余闻语则信而又信,不必宝玉而后证之方信也。
}\geng{余云恐他二人一悟则无妙文可看,然欲为开我怀,为醒我目,却愿他二人永堕迷津,生出孽障,余心甚不公矣。
世云损人利己者,余此愿是矣。
试思之,可发一笑。
今自呈于此,亦可为后人一笑,以助茶前酒后之兴耳。
而今后天地间岂不又添一趣谈乎?凡书皆以趣谈读去,其理自明,其趣自得矣。
}说毕,便携了回房去,与湘云同看。
\geng{却不同湘云分崩,有趣!}次日又与宝钗看。
宝钗看其词\geng{出自宝钗目中,正是大关键处。
\ping{暗示宝钗和宝玉婚后,宝玉再次因为某些变故而参禅悟道想要出家的时候,也是宝钗来劝解阻拦。
}}曰:\par
\hop
无我原非你,从他不解伊。
肆行无碍凭来去。
茫茫着甚悲愁喜,纷纷说甚亲疏密。
从前碌碌却因何,到如今回头试想真无趣!\geng{看此一曲,试思作者当日发愿不作此书,却立意要作传奇,
\zhu{传奇:明清两代以演唱南曲为主的长篇戏曲。著名的有《牡丹亭》《长生殿》《桃花扇》等。}
则又不知有如何词曲矣。
}\par
\zhu{此曲为宝玉对“你证我证”一偈的解释。
意亦双关,以谈禅之名谈情。
“无我原非你”:没有我也就没有你,你我相互依存,意取《庄子·齐物论》:“非彼无我,非我无所取。
”原意是,没有它(自然)就没有我,没有我也就没有东西来体现它了。
“从他不解伊”:任凭他人不理解你好了。
从:任凭。
伊:你。
“肆行无碍凭来去”:自己何妨随心所欲地自由行动呢。
}\par
\ping{
宝玉参禅,与上一回宝玉读《庄》,都是为后来佚稿中宝玉在贾家败落后而一度“悬崖撒手”出家的情节作“千里伏线”。
因为全书有两个超现实的象征人物一僧一道,所以第二十一回和第二十二回中宝玉也是分别从“道”和“禅”两个层面“觉悟”。
}
\par
\hop
看毕,又看那偈语,又笑道:“这个人悟了。
都是我的不是,都是我昨儿一支曲子惹出来的。
这些道书禅机最能移性。
\zhu{道书:这里指道家的书或用道家思想写的文字。
禅机:即佛门禅宗所宣扬的“妙谛”(真理)。
}\geng{拍案叫绝!此方是大悟彻语录,非宝卿不能谈此也。
}明儿认真说起这些疯话来,存了这个意思,都是从我这一只曲子上来,我成了个罪魁了。
”说着,便撕了个粉碎,递与丫头们说:“快烧了罢。
”黛玉笑道:“不该撕,等我问他。
你们跟我来,包管叫他收了这个痴心邪话。
”\ping{这里是宝玉的第一次领悟禅机萌发出家的想法,宝钗使用暴力的手段试图强迫宝玉回心转意,而黛玉则善于开导,从而使得宝玉主动放弃了出家的念头。
根据脂评,宝玉的最后结局是“弃而为僧”,可以想象宝玉肯定有第二次出家的念头,可能那时候真正懂宝玉的黛玉不在了,宝钗依旧采取了暴力强迫的手段,劝说没有奏效,最终宝玉还是毅然决然的出家了。
}\par
三人果然都往宝玉屋里来。
一进来,黛玉便笑道:“宝玉,我问你:至贵者是‘宝’,至坚者是‘玉’。
尔有何贵?尔有何坚?”\geng{拍案叫绝!大和尚来答此机锋,想亦不能答也。
非颦儿,第二人无此灵心慧性也。
}宝玉竟不能答。
三人拍手笑道:“这样钝愚,还参禅呢。
”\zhu{参禅:也叫悟禅,就是通过静心思虑,排除杂念的方法,来参悟佛教的“真理”。
}黛玉又道:“你那偈末云:‘无可云证,是立足境。
’固然好了,只是据我看,还未尽善。
我再续两句在后。
”因念云:“无立足境,是方干净。
”\geng{拍案叫绝!此又深一层也。
亦如谚云:“去年贫,只立锥;今年贫,锥也无。
”其理一也。
}
宝钗道:“实在这方悟彻。
当日南宗六祖惠能,\geng{用得妥当之极!}\zhu{据传,南天竺人菩提达摩于五世纪初来中国传播禅法,建立了早期禅宗,被推为禅宗东土始祖。
达摩传慧可(二祖),慧可传僧粲(三祖),僧粲传道信(四祖),道信传弘忍(五祖)。
弘忍死后,禅宗分成南北两宗。
北宗以神秀为六祖,南宗以惠能为六祖。
惠能,一作慧能,中国佛教禅宗的实际创立者。
初投弘忍门下当“行者”,在碓房里舂米。
\zhu{
碓[duì]:舂米的工具,由石臼和杵组成,用杵连续捣动盛放于石臼中的稻谷或糙米使之去壳或皮。
}
因作“菩提本非树”一偈,得到弘忍的赏识,便将禅法秘授与他,并付予法衣,史称南宗六祖。
后来中国佛教史上的禅宗和思想史上的禅学,一般指南宗和惠能的禅学。
}初寻师至韶州,闻五祖弘忍在黄梅,他便充役火头僧。
\zhu{火头僧:在厨房干活的僧人。
}五祖欲求法嗣,\zhu{法嗣:佛门宗派传法的继承人。
隋唐以后的佛教宗派,模仿世俗的封建宗法制度,把师徒之间的传承关系看作像父子继承的关系一祥,故仿宗法制度的方式建立自己的谱系,即所谓“法裔”、“法嗣”的制度。
}令徒弟诸僧各出一偈。
上座神秀说道:\zhu{上座神秀:神秀,少年出家,投禅宗五祖弘忍门下,是个博学的和尚。
弘忍生前,是寺中的上座(僧职,地位仅次于住持);弘忍死后,成为禅宗北派的创始人,史称北宗六祖。
}‘身是菩提树,心如明镜台,时时勤拂拭,莫使有尘埃。
’\zhu{菩提树:常绿乔木,据《西域记》载即毕钵罗树,相传释迦在此树下悟道成佛。
“菩提”为梵文音译,其义为“觉悟”。
尘:接触的对象。佛教将心和感官接触的对象分成色、声、香、味、触、法(指心所对的境)六尘。
若任由眼、耳、鼻、舌、身、意追逐六尘,心就会充满着烦恼。也称为「六处」。
认为它“坋(坋音“笨”,尘埃)污净心”。
这首偈代表了禅宗北宗的主张,认为人自身虽有佛性.但因受尘世杂念搅扰,必须通过坐禅,逐渐修炼,方能领悟,因称“渐悟”。
}彼时惠能在厨房碓米,\zhu{碓米:即舂米,用杵在石臼中捣米。
}听了这偈,说道:‘美则美矣,了则未了。
’因自念一偈曰:‘菩提本非树,明镜亦非台,本来无一物,何处惹尘埃?’\zhu{这首偈代表了禅宗南宗“顿悟”的主张,同北宗的“渐悟”相对。
即认为所谓“觉悟”不是外在的而只要向内心寻求,因此主张不用诵经、坐禅、布施,只要体会佛经的精神,主观上顿时觉悟,便可立地成佛。
发展下去,南宗逐渐占据了主流,成了禅宗正统。
佛教发展到禅宗已逐渐走向它自身的反面,因为它本身潜伏着从理论上导致破坏宗教的倾向:既然“菩提本非树,明镜亦非台”,
“无佛、无众生”,“本来无一物”,那么这种佛教教义的本身也是虚妄的。这样,它就埋藏下了毁灭它自己的炸弹。
}五祖便将衣钵传他。
\zhu{衣钵:佛家师徒间传承授受的法器。
衣:指袈裟。
钵:指僧人盛饭食器。
典出《传灯录》,谓禅宗五祖弘忍将衣钵传与六祖惠能。
后世本此,凡师徒相传,都称传承衣钵。
}\geng{出《语录》。
\zhu{《语录》:《六祖大师法宝坛经》,简称《六祖坛经》,亦称《坛经》。惠能死后,弟子们记录他的思想言行编撰成集。}
总写宝卿博学宏览,胜诸才人;颦儿却聪慧灵智,非学力所致——皆绝世绝伦之人也。
宝玉宁不愧杀!}今儿这偈语,亦同此意了。
只是方才这句机锋,\zhu{机锋:禅宗僧徒认为用语言文字正面表述禅理是不可能的,主张“以心传心,不立文字”,因而多用比喻或隐语来试探,让人猜度、印证,甚至一言不发,靠某些特定的动作来表达心意。
凡此都称之为机锋。
}尚未完全了结,这便丢开手不成?”黛玉笑道:“彼时不能答,就算输了,这会子答上了也不为出奇。
只是以后再不许谈禅了。
连我们两个所知所能的,你还不知不能呢,还去参禅呢。
”宝玉自己以为觉悟,不想忽被黛玉一问,便不能答,宝钗又比出“语录”来,\zhu{语录:古代一种语体文。
起源于唐代,僧徒用当时通俗口语记载其师的传授,叫“语录”。
到了宋代,儒家讲学风起,门生弟子仿照佛家语录体将其师言论直录成书,语录体遂大行于世。
}此皆素不见他们能者。
自己想了一想:“原来他们比我的知觉在先,尚未解悟,我如今何必自寻苦恼。
”\geng{前以《庄子》为引,故偶续之。
又借颦儿诗一鄙驳,兼不写着落,以为瞒过看官矣。
\zhu{
颦儿诗:指第二十一回,黛玉题宝玉续庄子文后的诗:
无端弄笔是何人?作践南华《庄子因》。不悔自己无见识,却将丑语怪他人。
因此诗之后没有再提及,所以说“不写着落”,“瞒过看官”。
}
此回用若许曲折,仍用老庄引出一偈来,再续一《寄生草》,可为大觉大悟矣。
以之上承果位,\zhu{
果位:佛教修行所证之果的地位,如小乘佛教有七个果位,分别是地狱、饿鬼、畜生、阿修罗、人、天、阿罗汉。大乘佛教在小乘佛教的果位上增加菩萨、佛两个果位。
这句话的“之上”是指宝玉续《庄子·胠箧》、作参禅偈、填《寄生草》解偈这些参禅悟道的举动,“承”字令人费解,在这里用作动词大概是达到,取得的意思。
}以后无书可作矣。
却又轻轻用黛玉一问机锋,又续偈言二句,并用宝钗讲五祖六祖问答二实偈子,使宝玉无言可答,仍将一大善知识,\zhu{
大善知识:指禅宗高僧。敦煌本《坛经》:“若不能自悟者,须觅大善知识示道见性。何名大善知识?
解最上乘法,直示正路,是大善知识。”
}始终跌不出警幻幻榜中,作下回若干书。
真有机心游龙不测之势,安得不叫绝?且历来小说中万写不到者。
己卯冬夜。
}想毕,便笑道:“谁又参禅,不过一时顽话罢了。
”说着,四人仍复如旧。
\geng{轻轻抹去也。
“心净难”三字不谬。
}\ping{宝玉不过被姊妹挤兑几句,便有了这么一出,真是天生向佛。
}\par
忽然人报,娘娘差人送出一个灯谜儿,命你们大家去猜,猜着了每人也作一个进去。
四人听说忙出去,至贾母上房。
只见一个小太监,拿了一盏四角平头白纱灯,专为灯谜而制,上面已有一个,众人都争看乱猜。
小太监又下谕道:“众小姐猜着了,不要说出来,每人只暗暗的写在纸上,一齐封进宫去,娘娘自验是否。
”宝钗等听了,近前一看,是一首七言绝句,并无甚新奇,口中少不得称赞,只说难猜,故意寻思,其实一见就猜着了。
宝玉、黛玉、湘云、探春\geng{此处透出探春,正是草蛇灰线,后文方不突然。
}四个人也都解了,各自暗暗的写了半日。
一并将贾环,贾兰等传来,一齐各揣机心都猜了,\geng{写出猜谜人形景,看他偏于两次\sout{戒}[禅]机后,\zhu{禅机:佛门禅宗所宣扬的“妙谛”(真理)。
}写此机心机事,足见用意至深至远。
}写在纸上。
然后各人拈一物作成一谜,恭楷写了,挂在灯上。
\par
太监去了,至晚出来传谕:“前娘娘所制,俱已猜着,惟二小姐与三爷猜的不是。
\geng{迎春、贾环也。
交错有法。
}小姐们作的也都猜了,不知是否。
”说着,也将写的拿出来。
也有猜着的,也有猜不着的,都胡乱说猜着了。
太监又将颁赐之物送与猜着之人,每人一个宫制诗筒,\zhu{诗筒:装诗歌草稿用的竹筒。
}
\geng{诗筒,身边所佩之物,以待偶成之句草录暂收之,其归至窗前不致有忘也。
或茜牙成,或琢香屑,或以绫素为之不一,想来奇特事,从不知也。
\zhu{
诗筒有丝绸制的软囊,但也有参以嵌象牙、填香等工艺的硬质材料做成者。
}
}一柄茶筅,\zhu{茶筅(筅音“险”):用竹子做的洗涤茶具的刷箒。
}\geng{破竹如帚,以净茶具之积也。
}\geng{二物极微极雅。
}独迎春、贾环二人未得。
迎春自为顽笑小事,并不介意,\geng{大家小姐。
}贾环便觉得没趣。
\ping{贾环在赵姨娘的教育下,自卑没有底气,何时都难大气。
}且又听太监说:“三爷说的这个不通,娘娘也没猜,叫我带回问三爷是个什么。
”众人听了,都来看他作的什么,写道是:\par
\hop
大哥有角只八个,二哥有角只两根。
\par
大哥只在床上坐,二哥爱在房上蹲。
\geng{可发一笑,真环哥之谜。
}
\geng{诸卿勿笑,难为了作者摹拟。
}\par
\hop
众人看了,大发一笑。
贾环只得告诉太监说:“一个枕头,一个兽头。
”\zhu{兽头:古代建筑塑在屋檐角上的两角怪兽。
}
\geng{亏他好才情,怎么想来?}太监记了,领茶而去。
\zhu{领茶:饮茶(接受贾府的款待)。}
\par
贾母见元春这般有兴,自己越发喜乐,便命速作一架小巧精致围屏灯来,
\zhu{围屏灯:指外形呈多面体,像围屏可开可合的灯。}
设于当屋,命他姊妹各自暗暗的作了,写出来粘于屏上,然后预备下香茶细果以及各色玩物,为猜着之贺。
贾政朝罢,见贾母高兴,况在节间,晚上也来承欢取乐。
设了酒果,备了玩物,上房悬了彩灯,请贾母赏灯取乐。
上面贾母、贾政、宝玉一席,下面王夫人、宝钗、黛玉、湘云又一席,迎、探、惜三个又一席。
地下婆娘丫鬟站满。
李宫裁、王熙凤二人在里间又一席。
\geng{细致。
}贾政因不见贾兰,便问:“怎么不见兰哥?”\geng{看他透出贾政极爱贾兰。
}地下婆娘忙进里间问李氏,李氏起身笑着回道:“他说方才老爷并没去叫他,他不肯来。
”婆娘回复了贾政。
众人都笑说:“天生的牛心古怪。
”贾政忙遣贾环与两个婆娘将贾兰唤来。
贾母命他在身旁坐了,抓果品与他吃。
大家说笑取乐。
\par
往常间只有宝玉长谈阔论,今日贾政在这里,便惟有唯唯而已。
\zhu{唯唯[wéi wéi]:恭敬应诺之词。
}
\geng{写宝玉如此。
非世家曾经严父之训者,断写不出此一句。
}馀者湘云虽系闺阁弱女,却素喜谈论,今日贾政在席,也自缄口禁言。
\geng{非世家经明训者,断不知此一句。
写湘云如此。
}黛玉本性懒与人共,原不肯多语。
\geng{黛玉如此。
与人多话则不肯,岂得与宝玉话\sout{更}[便]多哉?}宝钗原不妄言轻动,便此时亦是坦然自若。
\geng{瞧他写宝钗,真是又曾经严父慈母之明训,又是世府千金,自己又天性从礼合节,前三人之长并归一身。
前三人\sout{向}[尚]有捏作之态,故唯宝钗一人作坦然自若,亦不见逾规越矩也。
}故此一席虽是家常取乐,反见拘束不乐。
\geng{非世家公子断写不及此。
想近时之家,纵其儿女哭笑索饮,长者反以为乐,其无礼不法,何如是耶!}贾母亦知因贾政一人在此所致之故,\geng{这一句又明补出贾母亦是世家明训之千金也,不然断想不及此。
}酒过三巡,便撵贾政去歇息。
贾政亦知贾母之意,撵了自己去后,好让他们姊妹兄弟取乐的。
贾政忙陪笑道:“今日原听见老太太这里大设春灯雅谜,故也备了彩礼酒席,特来入会。
何疼孙子孙女之心,便不略赐以儿子半点?”\geng{贾政如此,余亦泪下。
}\ping{中年人撒娇,让人无端心疼。
}
贾母笑道:“你在这里,他们都不敢说笑,没的倒叫我闷。
你要猜谜时,我便说一个你猜,猜不着是要罚的。
”贾政忙笑道:“自然要罚。
若猜着了,也是要领赏的。
”贾母道:“这个自然。
”说着便念道:\par
\hop
猴子身轻站树梢。
\geng{所谓“树倒猢狲散”是也。
}打一果名。
\par
\zhu{贾母灯谜:“站树梢”,义同“立枝”,“立”“荔”谐音,故谜底为“荔枝”。
“荔枝”又与“离枝”谐音,故脂批说此谜的寓意在暗示树倒猢狲散。
}\par
\hop
贾政已知是荔枝,\geng{的是贾母之谜。
}便故意乱猜别的,罚了许多东西,然后方猜着,也得了贾母的东西。
然后也念一个与贾母猜,念道:\par
\hop
身自端方,体自坚硬。
虽不能言,有言必应。
\geng{好极!的是贾老之谜,包藏贾府祖宗自身。
\zhu{贾府祖宗:指曹玺,即曹雪芹曾祖。
这个谜语的谜底也可以理解为是“玺”(皇帝之印)。
}“必”字隐“笔”字。
妙极,妙极!}打一用物。
\par
\zhu{贾政灯谜:谜底是砚。
“必”与“笔”谐音,谜面即“有言必(笔)应”。
又“砚”“验”谐音,意寓贾母等人所作灯谜中的谶语,必将得到应验。
}\par
\hop
说毕,便悄悄的说与宝玉。
宝玉意会,又悄悄的告诉了贾母。
贾母想了想,\geng{太君身份。
}果然不差,便说:“是砚台。
”贾政笑道:“到底是老太太,一猜就是。
”回头说:“快把贺彩送上来。
”地下妇女答应一声,大盘小盘一齐捧上。
贾母逐件看去,都是灯节下所用所顽新巧之物,甚喜,遂命:“给你老爷斟酒。
”宝玉执壶,迎春送酒。
贾母因说:“你瞧瞧那屏上,都是他姊妹们做的,再猜一猜我听。
”贾政答应,起身走至屏前,只见头一个写道是:\par
\hop
能使妖魔胆尽摧,身如束帛气如雷。
\par
一声震得人方恐,回首相看已化灰。
\geng{此元春之谜。
才得侥幸,\zhu{侥幸:才选凤藻宫。
}
奈寿不长,可悲哉!}\par
\zhu{元春灯谜:此谜是元春得宠和短寿的形象写照。
迷信传说爆竹能驱除鬼魅,故云妖魔胆摧。
首二句喻元春为妃后身价百倍,声势烜赫。
后两句暗示元春昙花一现,贾府好景不长。
}\par
\hop
贾政道:“这是炮竹嗄。
\zhu{嗄:现在一般写作“啊”。}
”宝玉答道:“是。
”贾政又看道:\par
\hop
天运人功理不穷,有功无运也难逢。
\par
因何镇日纷纷乱,只为阴阳数不同。
\geng{此迎春一生遭际,惜不得其夫何!}\ping{暗示了迎春和孙绍祖的婚姻悲剧。
}\par
\zhu{迎春灯谜:此谜隐寓迎春一生的遭际。
天运:即天数。
人功:算盘上的珠子,要靠人去拨,故曰人功。
“镇”通“整”。镇日:整天。
阴阳:指上下排算珠,兼指男女、夫妻。
数:运数,命运。
}\par
\hop
贾政道:“是算盘。
”迎春笑道:“是。
”又往下看是:\par
\hop
阶下儿童仰面时,清明妆点最堪宜。
\par
游丝一断浑无力,莫向东风怨别离。
\geng{此探春远适之谶也。
\zhu{适:女子出嫁。
}使此人不远去,将来事败,诸子孙不致流散也,悲哉伤哉!}\par
\zhu{探春灯谜:此谜是以断线风筝暗示探春远嫁不归。
“清明”句,说清明时节是最适宜放风筝的好时光,与第五回判词中“清明涕送江边望”之句参看,实点出佚稿中她出嫁的季节。
游丝:本指春天昆虫吐出飘荡在空中的飞丝,这里指放风筝的线。
浑:全。
}\par
\hop
贾政道:“这是风筝。
”探春笑道:“是。
”又看道是:\par
\hop
前身色相总无成,不听菱歌听佛经。
\par
莫道此生沉黑海,性中自有大光明。
\geng{此惜春为尼之谶也。
公府千金至缁衣乞食,
\zhu{缁[zī]:黑色。}
宁不悲夫!}\lie{此是惜春之作。
}\par
\zhu{惜春灯谜:此谜暗示惜春为尼的归宿。
“前身”句,意为前世因迷恋尘世色相,未能修成正果。
色相:佛教用语,一切有形质、颜色、相貌可见的东西,都叫色相。
无成:没有悟道成佛。
不听菱歌:即看破红尘。
乐府诗中菱歌莲曲,内容多属男女情歌。
沉黑海:投身佛门与人间繁华欢乐绝缘,从世人来看,就像沉入漆黑的海底。
性:佛性。
大光明:佛祖释迦牟尼曾称大光明王,后以此代指佛。
}
\geng{此后破失,俟再补\foot{按:庚本本回至此止(列本同),后另页书“暂记宝钗制谜……”。
以下补文有不同版本,均为后人所补,兹据戚本(蒙、舒本略同)。
甲辰本补文将“更香”谜归于黛玉,另补入宝钗“竹夫人”谜和宝玉“镜”谜,情节甚不合理,不再附录。
}。
}
\geng{暂记宝钗制谜云:朝罢谁携两袖烟,琴边衾里总无缘。
晓筹不用鸡人报,五夜无烦侍女添。
焦首朝朝还暮暮,煎心日日复年年。
光阴荏苒须当惜,风雨阴晴任变迁。
此回未成而芹逝矣,叹叹!丁亥夏。
畸笏叟。
\ping{批书人和作者的关系很近,作者透露给批书人草稿内容。
本书后几十回的书稿可能并没有写出来过,只有作者零散的手稿,而作者尚未完成就去世了。
}}\par
\hop
贾政道:“这是佛前海灯嗄。
”惜春笑答道:“是海灯。
”\zhu{佛前海灯:即长明灯,供于寺庙佛像前,灯内大量贮油,中燃一焰,长年不灭。
}\par
贾政心内沉思道:“娘娘所作爆竹,此乃一响而散之物。
迎春所作算盘,是打动乱如麻。
探春所作风筝,乃飘飘浮荡之物。
惜春所作海灯,一发清净孤独。
今乃上元佳节,如何皆作此不祥之物为戏耶?”心内愈思愈闷,因在贾母之前,不敢形于色,只得仍勉强往下看去。
只见后面写着七言律诗一首,却是宝钗所作,随念道:\par
\hop
朝罢谁携两袖烟,琴边衾里总无缘。
\par
晓筹不用鸡人报,五夜无烦侍女添。
\par
焦首朝朝还暮暮,煎心日日复年年。
\par
光阴荏苒须当惜,风雨阴晴任变迁。
\par
\zhu{宝钗灯谜:谜底是更香,(更香:古时为夜间打更制造的一种线香,每燃完一支恰是一更,故称更香。
)用以暗示薛宝钗以后孤凄寡居的结局。
“朝罢”句,从杜甫“朝罢香烟携满袖”句化出,暗寓荣华过后,两手空空之意。
“琴边”句,意谓更香与弹琴时的鼎炉之香和熏被褥之香均无关,意寓宝钗同琴瑟和谐的夫妻生活终究是没有缘分的。
晓筹:代指早晨的时刻。
筹:这里指古代计时用的竹筹或铜筹。
鸡人:古代宫中头戴“绛帻”(帻:音“则”,头巾。
绛帻:红布头巾,象征雄鸡鸡冠)专职司晨报晓的卫士。
五夜:即五更。
“焦首”二句:以香从头上点燃和由外向内燃烧的情状,喻人苦恼忧煎的生活。
荏苒:音“忍染”,时间渐渐过去的样子。
}\par
\hop
贾政看完,心内自忖道:“此物还倒有限。
只是小小之人作此词句,更觉不祥,皆非永远福寿之辈。
”想到此处,愈觉烦闷,大有悲戚之状,因而将适才的精神减去十分之八九,只垂头沉思。
\par
贾母见贾政如此光景,想到或是他身体劳乏亦未可定,又兼之恐拘束了众姊妹不得高兴顽耍,即对贾政云:“你竟不必猜了,去安歇罢。
让我们再坐一会,也好散了。
”贾政一闻此言,连忙答应几个“是”字,又勉强劝了贾母一回酒,方才退出去了。
回至房中只是思索,翻来覆去竟难成寐,不由伤悲感慨,不在话下。
\par
且说贾母见贾政去了,便道:“你们可自在乐一乐罢。
”一言未了,早见宝玉跑至围屏灯前,指手画脚,满口批评,这个这一句不好,那一个破的不恰当,
\zhu{破:揭穿,使现出真相,这里指给出谜面对应的谜底。}
如同开了锁的猴子一般。
\ping{之前大家所作的谜语都是暗示未来悲剧结局的谶语,一笔突然转入现实的欢脱,一扫阴霾。
}宝钗便道:“还像适才坐着,大家说说笑笑,岂不斯文些儿。
”凤姐自里间忙出来插口道:“你这个人,就该老爷每日令你寸步不离方好。
适才我忘了,为什么不当着老爷,撺掇叫你也作诗谜儿。
若果如此,怕不得这会子正出汗呢。
”说的宝玉急了,扯着凤姐儿,扭股儿糖似的只是厮缠。
贾母又与李宫裁并众姊妹说笑了一会,也觉有些困倦起来。
听了听已是漏下四鼓,\zhu{漏:指“铜壶滴漏”,古代计时器,代指时间。
由于古代报更使用击鼓方式,故又以鼓指代更,四鼓即四更。四更天即丑时,即凌晨一点到三点。
}命将食物撤去,赏散与众人,随起身道:“我们安歇罢。
明日还是节下,该当早起。
明日晚间再玩罢。
”且听下回分解。
\par
\qi{总评:作者具菩提心,捉笔现身说法,每于言外警人再三再四。
而读者但以小说古词目之,则大罪过。
其先以庄子为引,及偈曲句作醒悟之语,以警觉世人。
犹恐不入,再以灯谜伸词致意,
\zhu{伸:同“申”,申述;说明。}
自解自叹,以不成寐为言,其用心之切之诚,读者忍不留心而慢忽之耶?}
\dai{043}{宝钗生日看戏}
\dai{044}{制灯谜贾政悲谶语}
\sun{p22-1}{荣国府宝钗做生辰,听曲文宝玉悟禅机}{时值宝钗十五岁生日这天,贾母内院搭了小巧戏台,就在贾母上房摆了几桌家宴酒席,合家庆贺。
听戏时宝钗点了一出《山门》, 宝玉讥她“只好点这些热闹戏”,宝钗笑宝玉“不知戏”,告诉他说这出音律好,词藻更好,又随口念了“寄生草”曲文。
宝玉听了,喜得拍膝摇头称妙。
黛玉却在一旁撇嘴,道:“安静些看戏吧!还没唱《山门》,你就《妆疯》了。
”说得湘云也笑了。
}