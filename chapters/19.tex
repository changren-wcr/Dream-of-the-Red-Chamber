\chapter[情切切良宵花解语\quad 意绵绵静日玉生香]{情切切良宵花解语\quad 意绵绵静日玉生香\foot{此回己、庚本虽已分出,但缺回目,据诸本补。
该回目各本一致,且本回己、庚本批语中多次提及,当系作者原拟。
}}
%在标题中使用花哨格式{}(比如添加脚注),但是在其他地方如页眉目录保持简单格式[]
%\chapter{情切切良宵花解语\quad 意绵绵静日玉生香}
\zhu{花解语:从“解语花”一词来。
解语花:即善解人意的、会说话的花。
解:会、能的意思。
王仁裕《开元天宝遗事》记述唐玄宗时太液池有数枝千叶白莲盛开,帝与贵戚共赏,指杨贵妃对左右说:“争如我解语花。
”\zhu{争如:怎如。
}后常用以比喻美人。
这里“花”指花袭人。
}
\par
\qi{彩笔辉光若转环,心情魔态几千般。
写成浓淡兼深浅,活现痴人恋恋间。
}\par
\chen{此回写出宝玉闲闯书房偷看袭人,笔意随机跳脱。
复又袭人将欲赎身,揣情讽谏,以及宝玉在黛玉房中寻香嘲笑,文字新奇,传奇之中殊所罕见。
原本评注过多,未免旁杂,反扰正文。
今删去,以俟后之观者凝思入妙,愈显作者之灵机耳\foot{此批显系后人所加。
但因其说明了甲辰本删除批语的理由,对了解后出抄本(列藏、舒序、杨藏、郑藏本等)因何均为白文本,有其认识价值。
姑存之。
}。
}\par
话说贾妃回宫,次日见驾谢恩,并回奏归省之事,龙颜甚悦,又发内帑彩缎金银等物,以赐贾政及各椒房等员,\ji{补还一句,细。
方见省亲不独贾家一门也。
}\zhu{椒房:汉代后妃住的宫室用花椒和泥涂壁,取其温暖有香气;又因花椒结实多,兼有希求多子之意。
后以椒房代指后妃居处或后妃。
}不必细说。
\par
且说荣宁二府中因连日用尽心力,真是人人力倦,各各神疲,又将园中一应陈设动用之物收拾了两三天方完。
第一个凤姐事多任重,别人或可偷安躲静,独他是不能脱得的;二则本性要强,不肯落人褒贬,只扎挣着与无事的人一样。
\zhu{扎挣:勉强支持。
}\ji{伏下病源。
}第一个宝玉是极无事最闲暇的。
\ping{家族诸人,尤其是凤姐劳累,而宝玉闲散,忽然可恨。
}
偏这日一早,袭人的母亲又亲来回过贾母,接袭人家去吃年茶,
\zhu{年茶:指年节聚会吃的果茶。}
晚间才得回来。
\ji{一回一回各生机轴,总在人意想之外。
}因此,宝玉只和众丫头们掷骰子赶围棋作戏。
\ji{写出正月光景。
}正在房内顽的没兴头,忽见丫头们来回说:“东府珍大爷来请过去看戏、放花灯。
”宝玉听了,便命换衣裳。
才要去时,忽又有贾妃赐出糖蒸酥酪来;
\zhu{
糖蒸酥酪:
下文李嬷嬷有这样的话:“……如今我吃他一碗牛奶,他就生气了?……”
可知是牛乳加糖和粉(米粉、面粉)一类甜食。
另一种说法是,牛奶加酒酿作引子发酵而成的特殊酸奶。
}
\ji{总是新正妙景。
}宝玉想上次袭人喜吃此物,便命留与袭人了。
自己回过贾母,过去看戏。
\par
谁想贾珍这边唱的是《丁郎认父》、《黄伯央大摆阴魂阵》,更有《孙行者大闹天宫》、《姜子牙斩将封神》等类的戏文。
\ji{真真热闹。
}\zhu{《丁郎认父》演明代杜文学(一说高文举)被严嵩迫害流落湖广,入赘胡丞相府中,他与前妻所生之子丁郎在街上相遇不敢相认,后入府说明原委才得父子相认。
《黄伯央大摆阴魂阵》讲燕将乐毅的师父黄伯央下山布迷魂阵围困齐将孙膑的事,剧情十分热闹。
《孙行者大闹天宫》、《姜子牙斩将封神》分别为敷演《西游记》和《封神演义》的故事。
}
倏尔神鬼乱出,忽又妖魔毕露,甚至于扬幡过会,\zhu{幡:音“翻”,挑起来直着挂的长条形旗子。
过会:中国民间习俗之一,类似于庙会的一种集体游艺活动。
}号佛行香,\zhu{号佛:口宣佛号,即大声念佛。
}锣鼓喊叫之声远闻巷外。
\ji{形容刻薄之至,弋阳腔能事毕矣。
\zhu{
弋阳腔:南戏四大声腔(海盐腔、余姚腔、弋阳腔、昆山腔)之一。形成于江西弋阳,故称。
}
}
\ji{
阅至此则有如耳内喧哗、目中离乱,后文至隔墙闻“袅晴丝”数曲,则有如魂随笛转、魄逐歌销。
形容一事,一事毕真,石头是第一能手矣。
}满街之人个个都赞:“好热闹戏,别人家断不能有的。
”\ji{必有之言。
}宝玉见繁华热闹到如此不堪的田地,只略坐了一坐,便走开各处闲耍。
先是进内去和尤氏和丫鬟姬妾说笑了一回,便出二门来。
尤氏等仍料他出来看戏,遂也不曾照管。
贾珍、贾琏、薛蟠等只顾猜枚行令,\zhu{猜枚:又称猜拳、划拳、豁拳、拇战或酒拳,是一种酒席上的游戏,饮酒时常以此助兴,属酒令的一种。
最初是把席上的果品或棋子握在拳中,让人猜测其数目之多寡、单双或颜色,输了的要罚酒,是谓猜枚。
后来演化成一种猜手指数目的游戏,是谓豁拳。
豁字有张开义,那么字面上,豁拳就是张手出指的意思。
而“划拳”等都是豁拳的讹写。
传统豁拳的方法是:参与的两个人同时各出一手(一般为右手),靠握紧拳头或伸出不同数目的手指比出从零到五六种不同的数目,在出手的同时喊出一个口令,对应最小为〇最大为十的一个数。
若喊出的口令所代表的数目与两人所出的手指数的总和相同,即为猜中。
若两人皆猜错或皆猜中,则视为平手,继续出拳呼令,直到一方猜中一方猜错时,决出胜负。
猜错者要罚酒。
豁拳所呼的口令皆是一些以数字开头或跟数字有关的,象征友谊和讨吉利的熟语,例如“一条龙”、“哥俩好”、“五魁首”等。
}百般作乐,也不理论,纵一时不见他在座,只道在里边去了,故也不问。
至于跟宝玉的小厮们,那年纪大些的,知宝玉这一来了,必是晚间才散,因此偷空也有去会赌的,也有往亲友家去吃年茶的,更有或嫖或饮的,都私散了,待晚间再来;那小些的,都钻进戏房里瞧热闹去了。
\par
宝玉见一个人没有,因想“这里素日有个小书房,名……,内曾挂着一轴美人,极画的得神。
今日这般热闹,想那里自然冷静,那美人也自然是寂寞的,须得我去望慰他一回。
”\ji{极不通极胡说中写出绝代情痴,宜乎众人谓之疯傻。
}\meng{天生一段痴情,所谓“情不情”也。
\zhu{情不情:施加情感于无情之物。
}}想着,便往书房里来。
刚到窗前,闻得房内有呻吟之韵。
宝玉倒唬了一跳:敢是美人活了不成?\ji{又带出小儿心意,一丝不落。
}乃乍着胆子,舔破窗纸,向内一看,那轴美人却不曾活,却是茗烟按着一个女孩子,也干那警幻所训之事。
宝玉禁不住大叫:“了不得!”一脚踹进门去,将那两个唬开了,抖衣而颤。
\par
茗烟见是宝玉,忙跪求不迭。
宝玉道:“青天白日,这是怎么说。
\ji{开口便好。
}珍大爷知道,你是死是活?”一面看那丫头,虽不标致,倒还白净,些微亦有动人处,羞的面红耳赤,低首无言。
宝玉跺脚道:“还不快跑!”\ji{此等搜神夺魄至神至妙处,只在囫囵不解中得。
}一语提醒了那丫头,飞也似去了。
宝玉又赶出去叫道:“你别怕,我是不告诉人的。
”\ji{活宝玉,移之他人不可。
}急的茗烟在后叫:“祖宗,这是分明告诉人了!”宝玉因问:“那丫头十几岁了?”茗烟道:“大不过十六七岁了。
”宝玉道:“连他的岁属也不问问,\zhu{岁属:岁数与属相。
}别的自然越发不知了。
可见他白认得你了。
可怜,可怜!”\ji{按此书中写一宝玉,其宝玉之为人,是我辈于书中见而知有此人,实未目曾亲睹者。
又写宝玉之发言,每每令人不解;宝玉之生性,件件令人可笑;不独不曾于世上亲见这样的人,即阅今古所有之小说传奇中,亦未见这样的文字。
于颦儿处更为甚。
其囫囵不解之中实可解,可解之中又说不出理路。
合目思之,却如真见一宝玉,真闻此言者,移至第二人万不可,亦不成文字矣。
余阅《石头记》中至奇至妙之文,全在宝玉颦儿至痴至呆、囫囵不解之语中,其诗词、雅谜、酒令、奇衣、奇食、奇文等类固他书中未能,然在此书中评之,犹为二着。
}又问:“名字叫什么?”茗烟大笑道:“若说出名字来话长,真真新鲜奇文\foot{“奇文”二字,据文意当为批语。
庚本后人墨眉:“奇文句似应作注。
”},竟是写不出来的。
\ji{若都写得出来,何以见此书中之妙?脂砚。
}据他说,他母亲养他的时节做了梦,\zhu{养:这里指生孩子。
}\ji{又一个梦,只是随手成趣耳。
}梦见得了一匹锦,上面是五色富贵不断头卍字的花样,\ji{千奇百怪之想,所谓“牛溲马渤皆至乐也,\zhu{牛溲马渤:溲音“搜”三声,牛溲是牛尿(一说车前草),马渤即马勃是一种菌类,都可做药用,比喻虽然微贱但是有用的东西。
}鱼鸟昆虫皆妙文也”,天地间无一物不是妙物,无一物不可不成文,
\zhu{无一物不可不成文:否定层数过多,可能应该是“无一物不成文”。}
但在人意舍取耳。
此皆信手拈来随笔成趣,大游戏、大慧悟、大解脱之妙文也。
}所以他的名字叫作卍儿。
”\ji{音万。
}\zhu{卍:音“万”,梵字,义同“万”,为印度的一种吉祥标相。
}宝玉听了笑道:“真也新奇,想必他将来有些造化。
”说着,沉思一会。
\par
茗烟因问:“二爷为何不看这样的好戏?”宝玉道:“看了半日,怪烦的,出来逛逛,就遇见你们了。
这会子作什么呢?”茗烟嘻嘻\foot{原作“\xixi\xixi”,\geng{\xixi,音希。
\xixi\xixi,笑貌。
}按:“\xixi\xixi”音义同“嘻嘻”,今予统一。
后第五十回“只见宝玉笑\xixi\xixi 掮了一枝红梅进来”,仿此不另注。
}笑道:“这会子没人知道,我悄悄的引二爷往城外逛逛去,一会子再往这里来,他们就不知道了。
”\ji{茗烟此时只要掩饰方才之过,故设此以悦宝玉之心。
}宝玉道:“不好,仔细花子拐了去。
\zhu{花子:这里指诓骗小孩的拐子。
亦称“拍花子”。
}便是他们知道了,又闹大了,不如往熟近些的地方去,还可就来。
”茗烟道:“熟近地方,谁家可去?这却难了。
”宝玉笑道:“依我的主意,咱们竟找你花大姐姐去,瞧他在家作什么呢。
”\ji{妙!宝玉心中早安了这着,但恐茗烟不肯引去耳。
恰遇茗烟私行淫媾,为宝玉所胁,故以城外引以悦其心,宝玉始说出往花家去。
非茗烟适有罪所胁,万不敢如此私引出外。
别家子弟尚不敢私出,况宝玉哉?况茗烟哉?文字榫楔,细极!
}茗烟笑道:“好,好!倒忘了他家。
”又道:“若他们知道了,说我引着二爷胡走,要打我呢?”\ji{必不可少之语。
}宝玉笑道:“有我呢。
”茗烟听说,拉了马,二人从后门就走了。
\par
幸而袭人家不远,不过一半里路程,展眼已到门前。
茗烟先进去,叫袭人之兄花自芳。
\ji{随姓成名,随手成文。
}彼时袭人之母接了袭人与几个外甥女儿、\ji{一树千枝,一源万派,无意随手,伏脉千里。
}几个侄女儿来家,正吃果茶。
听见外面有人叫“花大哥”,花自芳忙出去看时,见是他主仆两个,唬的惊疑不止,连忙抱下宝玉来,\ping{显出宝玉还是孩子。
}在院内嚷道:“宝二爷来了!”别人听见还可,袭人听了,也不知为何,忙跑出来迎着宝玉,一把拉着问:“你怎么来了?”宝玉笑道:“我怪闷的,来瞧瞧你作什么呢。
”袭人听了,才放下心来,\ji{精细周到。
}嗐了一声,\zhu{嗐:音“害”,表示不满,惋惜或懊悔。
}笑\ji{转至“笑”字,妙甚!}道:“你也忒胡闹了,\ji{该说,说得是。
}可作什么来呢!”一面又问茗烟:“还有谁跟来?”\ji{细。
}茗烟笑道:“别人都不知,就只我们两个。
”袭人听了,复又惊慌,\ji{是必有之神理,非特故作顿挫。
}
说道:“这还了得!倘或碰见了人,或是遇见了老爷,街上人挤车碰,马轿纷纷的,若有个闪失,也是顽得的!你们的胆子比斗还大。
都是茗烟调唆的,回去我定告诉嬷嬷们打你。
”\ji{该说,说的更是。
脂研。
}茗烟撅了嘴道:“二爷骂着打着,叫我引了来,这会子推到我身上。
我说别来罢,不然我们还去罢。
”\ji{茗烟贼。
}花自芳忙劝:“罢了,已是来了,也不用多说了。
只是茅檐草舍,又窄又脏,爷怎么坐呢?”\par
袭人之母也早迎了出来。
袭人拉了宝玉进去。
宝玉见房中三五个女孩儿,见他进来,都低了头,羞惭惭的。
花自芳母子两个百般怕宝玉冷,又让他上炕,又忙另摆果桌,又忙倒好茶。
\ji{连用三“又”字,上文一个“百般”,神理活现。
脂砚。
}袭人笑道:“你们不用白忙,\ji{妙!不写袭卿忙,正是忙之至。
若一写袭人忙,便是庸俗小派了。
}我自然知道。
果子也不用摆,也不敢乱给东西吃。
”\ji{如此至微至小中便带出[世]家常情,他书写不及此。
}
\meng{至敬至情。
}一面说,一面将自己的坐褥拿了铺在一个杌子上,
\zhu{杌(音“物”):小凳子。
}
宝玉坐了;用自己的脚炉垫了脚,向荷包内取出两个梅花香饼儿来,\zhu{梅花香饼儿:用香料粉末做成的梅花状小饼,可以佩带,也可焚烧。
}又将自己的手炉掀开焚上,仍盖好,放与宝玉怀内;然后将自己的茶杯斟了茶,送与宝玉。
\ji{叠用四“自己”字,写得宝袭二人素日如何亲洽如何尊荣,此时一盘托出。
盖素日身居侯府绮罗锦绣之中,其安富尊荣之宝玉,亲密浃洽、
\zhu{浃[jiā]洽:
1、深入沾润,遍及。《史记·礼乐志》:“于是教化浃洽。”颜师古注:“浃,彻也;恰,沾也。”
2、融洽、和洽。如:情意浃洽。
}
勤慎委婉之袭人,是分所应当不必写者也。
今于此一补,更见其二人平素之情义,且暗透此回中所有母女兄长欲为赎身角口等未到之过文。
}彼时他母兄已是忙另齐齐整整摆上一桌子果品来。
袭人见总无可吃之物,\ji{补明宝玉自幼何等娇贵。
以此一句留与下部后数十回“寒冬噎酸虀,\zhu{
噎[yē]:食物等塞住喉咙。
虀:音“基”,“齑”的繁体字,切碎的姜、葱、蒜等。
}雪夜围破毡”等处对看,可为后生过分之戒。
叹叹!}因笑道:“既来了,没有空去之理,好歹尝一点儿,也是来我家一趟。
”\ji{得意之态,是才与母兄较争以后之神理。
最细。
}说着,便拈了几个松子穰,\zhu{穰:同“瓤”。
}\ji{唯此品稍可一拈,别品便大错了。
}吹去细皮,用手帕托着送与宝玉。
\par
宝玉看见袭人两眼微红,粉光融滑,\ji{八字画出才收泪之一女儿,是好形容,且是宝玉眼中意中。
}因悄问袭人:“好好的哭什么?”袭人笑道:“何尝哭,才迷了眼揉的。
”因此便遮掩过了。
\ji{伏下后文所补未到多少文字。
\zhu{后文会揭晓,袭人母兄要把袭人从贾府中赎身出来,袭人不情愿而哭。}
}当下宝玉穿着大红金蟒狐腋箭袖,
\zhu{箭袖:原为便于射箭穿的窄袖衣服,这里指男子穿的一种服式。}
外罩石青貂裘排穗褂。
\zhu{
石青:淡灰青色。
排穗:排缀在衣服下面边缘的彩穗。
}
袭人道:“你特为往这里来又换新服,他们\ji{指晴雯麝月等。
}就不问你往那去的?”\ji{必有是问。
}\ji{阅此则又笑尽小说中无故家常穿红挂绿、绮绣绫罗等语,自谓是富贵语,究竟反是寒酸话。
}宝玉笑道:“珍大哥请过去看戏换的。
”袭人点头。
又道:“坐一坐就回去罢,这个地方不是你来的。
”宝玉笑道:“你就家去才好呢,我还替你留着好东西呢。
”\geng{生员切己之事。
\zhu{
生员:科举时代指在国学或州、县学读书的学生;
明清时指经过本省各级考试录取进入府、州、县学的学生,通称秀才。
金圣叹《贯华堂第六才子书西厢记》第二本第二折·请宴【上小楼】:
“秀才们闻道请,似得了将军令,先是五脏神愿随鞭镫。”
批:“又嘲戏生员切己事情。”
脂砚斋的很多评点模式和术语,都是从金圣叹那里借用来的。
“生员”应该是指批书人自己,批书人看到书中宝玉和袭人的暧昧关系,想到自己在现实中也有类似的经历。
}
}袭人悄笑道:“悄悄的,叫他们听着什么意思。
”\ji{想见二人素日情长。
}\meng{追魂。
}一面又伸手从宝玉项上将通灵玉摘了下来,向他姊妹们笑道:“你们见识见识。
时常说起来都当希罕,\meng{不可少之文。
}
恨不能一见,今儿可尽力瞧了。
再瞧什么希罕物儿,也不过是这么个东西。
”\ji{行文至此,固好看之极,且勿论按此言固是袭人得意之语,盖言你等所稀罕不得一见之宝,我却常守常见,视为平物。
然余今窥其用意之旨,则是作者借此,正为贬玉原非大观者也。
\ping{“贬玉原非大观者也”令人费解,可能是贬低这块玉不过是大荒山的一块石头罢了,不是珍贵之物,也可能是贬低贾宝玉虽然集万千宠爱于一身,但是不过是一个纨绔子弟罢了。
这里也许是用通灵宝玉代指贾宝玉,以上两个意思兼而有之。
}}说毕,递与他们传看了一遍,仍与宝玉挂好。
\geng{自“一把拉住”至此诸形景动作,
\zhu{前文并无“一把拉住”字样,而是有“一把拉着问”字样。}
袭卿有意微露绛芸轩中隐事也。
\zhu{隐事:指宝玉和袭人过于亲密乃至初试云雨之事。}
}又命他哥哥去或雇一乘小轿,或雇一辆小车,送宝玉回去。
花自芳道:“有我送去,骑马也不妨了。
”\geng{只知保重耳。
\zhu{这条批语的意思是,花自芳只知道注意宝玉的安全,却不知道要避免宝玉被人看到。袭人立刻提醒花自芳这一点。}
}袭人道:“不为不妨,为的是碰见人。
”\ji{细极!}\par
花自芳忙去雇了一顶小轿来,众人也不敢相留,只得送宝玉出去。
袭人又抓果子与茗烟,又把些钱与他买花炮放,教他:“不可告诉人,连你也有不是。
”\meng{细密。
}一直送宝玉至门前,看着上轿,放下轿帘。
花、茗二人牵马跟随。
来至宁府街,茗烟命住轿,向花自芳道:“须等我同二爷还到东府里混一混,才好过去的,不然人家就疑惑了。
”花自芳听说有理,忙将宝玉抱出轿来,送上马去。
宝玉笑说:“倒难为你了。
”\geng{公子口气。
}于是仍进后门来。
俱不在话下。
\par


却说宝玉自出了门,他房中这些丫鬟们都越性恣意的顽笑,也有赶围棋的,也有掷骰抹牌的,磕了一地瓜子皮。
偏奶母李嬷嬷拄拐进来请安,瞧瞧宝玉,见宝玉不在家,丫头们只顾玩闹,十分看不过。
\ji{人人都看不过,独宝玉看得过。
}因叹道:“只从我出去了,不大进来,你们越发没个样儿了,\ji{说得是,原该说。
}别的妈妈们越不敢说你们了。
\ji{补明好!宝玉虽不吃乳,岂无伴从之媪妪哉?\zhu{媪妪:媪音“奥”三声,妪音“玉”,都是对年老妇女的称呼,也作为妇女的通称。
}}那宝玉是个丈八的灯台——照见人家,照不见自家的。
\ji{用俗语入,妙!}只知嫌人家脏,这是他的屋子,由着你们糟蹋,越不成体统了。
”\ji{所以为今古未有之一宝玉。
}这些丫头们明知宝玉不讲究这些,二则李嬷嬷已是告老解事出去的了,\ji{调侃入微,妙妙!}如今管他们不着。
因此只顾顽,并不理他。
那李嬷嬷还只管问“宝玉如今一顿吃多少饭”、“什么时辰睡觉”等语。
\ji{可叹!}丫头们总胡乱答应。
有的说:“好一个讨厌的老货!”\geng{实在有的。
}\meng{入神。
}\par
李嬷嬷又问道:“这盖碗里是酥酪,\zhu{盖碗:一种上有盖、下有托,中有碗的汉族茶具。
又称“三才碗”、“三才杯”,盖为天、托为地、碗为人,暗含天地人和之意。
}怎不送与我去?我就吃了罢。
”说毕,拿匙就吃。
\ji{写龙钟奶母,便是龙钟奶母。
}\ping{为老不尊。
}一个丫头道:“快别动!那是说了给袭人留着的,\ji{过下无痕。
}回来又惹气了。
\ji{照应茜雪枫露茶前案。
\zhu{第八回,茜雪把宝玉留着吃的枫露茶给李嬷嬷吃了,导致宝玉大发雷霆。}
}你老人家自己承认,别带累我们受气。
”\ji{这等话语声口,必是晴雯无疑。
}李嬷嬷听了,又气又愧,便说道:“我不信他这样坏了。
别说我吃了一碗牛奶,就是再比这个值钱的,也是应该的。
难道待袭人比我还重?难道他不想想怎么长大了?我的血变的奶,吃的长这么大,如今我吃他一碗牛奶,他就生气了?我偏吃了,看怎么样!你们看袭人不知怎样,那是我手里调理出来的毛丫头,什么阿物儿!”\zhu{阿物儿:如同说“东西”、“家伙”(指人),是一种轻蔑的口气。
}\ji{虽暂委屈唐突袭卿,然亦怨不得李媪。
\zhu{丫头说袭人吃得李嬷嬷吃不得,打击了李嬷嬷的自尊心,怨不得李嬷嬷骂袭人“阿物儿”。}
}一面说,一面赌气将酥酪吃尽。
又一丫头笑道:“他们不会说话,怨不得你老人家生气。
宝玉还时常送东西孝敬你老去,岂有为这个不自在的。
”\ji{听这声口,必是麝月无疑。
}李嬷嬷道:“你们也不必妆狐媚子哄我,\zhu{妆狐媚子:用狐狸精善迷人来比喻献媚讨好。
}打量上次为茶撵茜雪的事我不知道呢。
\ji{照应前文,又用一“撵”,屈杀宝玉,然李媪心中口中毕肖。
}明儿有了不是,我再来领!”说着,赌气去了。
\ji{过至下回。
}\par
少时,宝玉回来,命人去接袭人。
只见晴雯躺在床上不动,\ji{娇态已惯。
}宝玉因问:“敢是病了?再不然输了?”秋纹道:“他倒是赢的。
谁知李老太太来了,混输了,他气的睡去了。
”宝玉笑道:“你别和他一般见识,由他去就是了。
”说着,袭人已来,彼此相见。
袭人又问宝玉何处吃饭,多早晚回来,又代母妹问诸同伴姊妹好。
一时换衣卸妆。
宝玉命取酥酪来,丫鬟们回说:“李奶奶吃了。
”宝玉才要说话,袭人便忙笑说道:“原来是留的这个,多谢费心。
前儿我吃的时候好吃,吃过了好肚子疼,足的吐了才好。
他吃了倒好,搁在这里倒白糟蹋了。
\ji{与前文应失手碎钟遥对,
\zhu{
第八回,茜雪把宝玉留着吃的枫露茶给李嬷嬷吃了,宝玉大怒,摔碎茶杯。
袭人以倒茶滑倒失手砸了钟子为借口遮掩过去,避免事态激化扩大。
}
通部袭人皆是如此,一丝不错。
}我只想风干栗子吃,你替我剥栗子,我去铺炕。
”\ji{必如此方是。
}\par
宝玉听了信以为真,方把酥酪丢开,取栗子来,自向灯前检剥。
一面见众人不在房中,乃笑问袭人道:“今儿那个穿红的是你什么人?”\ji{若是见过女儿之后没有一段文字便不是宝玉,亦非《石头记》矣。
}袭人道:“那是我两姨妹子。
”宝玉听了,赞叹了两声。
\ji{这一赞叹又是令人囫囵不解之语,只此便抵过一大篇文字。
}袭人道:“叹什么?\ji{只一“叹”字便引出“花解语”一回来。
}我知道你心里的缘故,想是说他那里配红的。
”\ji{补出宝玉素喜红色,这是激语。
}宝玉笑道:“不是,不是。
那样的不配穿红的,谁还敢穿。
\ji{活宝玉。
}我因为见他实在好的很,怎么也得他在咱们家就好了。
”\ji{妙谈妙意。
}袭人冷笑道:“我一个人是奴才命罢了,难道连我的亲戚都是奴才命不成?定还要拣实在好的丫头才往你家来?”\ji{妙答。
宝玉并未说“奴才”二字,袭人连补“奴才”二字最是劲节,怨不得作此语。
}
宝玉听了,忙笑道:“你又多心了。
我说往咱们家来,必定是奴才不成?\ji{勉强,如闻。
}说亲戚就使不得?”\ji{更勉强。
}\meng{这样妙文,何处得来?非目见身行,岂能如此的确?}袭人道:“那也搬配不上。
\zhu{搬配:门第相当、身分相配。}
”\ji{说的是。
}宝玉便不肯再说,只是剥粟子。
袭人笑道:“怎么不言语了?想是我才冒撞冲犯了你?明儿赌气花几两银子买他们进来就是了。
”\ji{总是故意激他。
}宝玉笑道:“你说的话,怎么叫我答言呢。
我不过是赞他好,正配生在这深堂大院里,没的我们这种浊物\ji{妙号!后文又曰“须眉浊物”之称,今古未有之一人始有此今古未有之妙称妙号。
}倒生在这里。
”\zhu{没的:无端,平白无故。
}\ji{这皆是宝玉心中意中确实之念,非前勉强之词,所以谓今古未有之一人耳。
听其囫囵不解之言,察其幽微感触之心,审其痴妄委婉之意,皆今古未见之人,亦是今古未见之文字。
说不得贤,说不得愚,说不得不肖,说不得善,说不得恶,说不得光明正大,说不得混账恶赖,说不得聪明才俊,说不得庸俗平凡,说不得好色好淫,说不得情痴情种,恰恰只有一颦儿可对,令他人徒加评论,总未摸着他二人是何等脱胎、何等心臆、何等骨肉。
余阅此书,亦爱其文字耳,实亦不能评出此二人终是何等人物。
后观《情榜》评曰“宝玉情不情”,“黛玉情情”,\zhu{情情:施加情感于有情之人。
}此二评自在评痴之上,亦属囫囵不解,妙甚!}
袭人道:“他虽没这造化,倒也是娇生惯养的呢,我姨爹姨娘的宝贝。
如今十七岁,各样的嫁妆都齐备了,明年就出嫁。
”\geng{所谓不入耳之言也。
}\par
宝玉听了“出嫁”二字,不禁又嗐了两声。
\ji{宝玉心思另是一样,余前评可见。
}正不自在,又听袭人叹道:\ji{袭人亦叹,自有别论。
}“只从我来这几年,姊妹们都不得在一处。
如今我要回去了,他们又都去了。
”宝玉听这话内有文章,\ji{余亦如此。
}不觉吃一惊,\ji{余亦吃惊。
}忙丢下栗子,问道:“怎么,你如今要回去了?”袭人道:“我今儿听见我妈和哥哥商议,教我再耐烦一年,明年他们上来,就赎我出去的呢。
”\ji{即余今日犹难为情,况当日之宝玉哉?}宝玉听了这话,越发怔了,因问:“为什么要赎你?”袭人道:“这话奇了!我又比不得是你这里的家生子儿,\zhu{家生子儿:按清代法律,家奴子女世代为奴,永远服役。
}
一家子都在别处,独我一个人在这里,怎么是个了局?”\ji{说得极是。
}
宝玉道:“我不叫你去也难。
”\ji{是头一句驳,故用贵公子声口,无理。
}袭人道:“从来没这道理。
便是朝廷宫里,也有个定例,或几年一选,几年一入,也没有个长远留下人的理,别说你了!”\ji{一驳,更有理。
}\par
宝玉想一想,果然有理。
\ji{自然。
}又道:“老太太不放你也难。
”\ji{第二层仗祖母溺爱,更无理。
}袭人道:“为什么不放?我果然是个最难得的,或者感动了老太太、太太,\ji{宝玉并不提王夫人,袭人偏自补出,周密之至!}必不放我出去的,设或多给我们家几两银子,留下我,容或有之;其实我也不过是个平常的人,\meng{此等语言便是袭卿心事。
}比我强的多而且多。
自我从小儿来了,跟着老太太,先伏侍了史大姑娘几年,\ji{百忙中又补出湘云来,真是七穿八达,得空便入。
}如今又伏侍了你几年。
如今我们家来赎,正是该叫去的,只怕连身价也不要,\zhu{身价:赎身钱。
}就开恩叫我去呢。
若说为伏侍的你好,不叫我去,断然没有的事。
那伏侍的好,是分内应当的,\geng{这却是真心话。
}不是什么奇功。
我去了,仍旧有好的来了,不是没了我就不成事。
”\ji{再一驳,更精细更有理。
}\meng{反敲。
}宝玉听了这些话,竟是有去的理,无留的理,\ji{自然。
}心内越发急了,\ji{原当急。
}
因又道:“虽然如此说,我只一心留下你,不怕老太太不和你母亲说。
多多给你母亲些银子,他也不好意思接你了。
”\ji{急心肠,故入于霸道。
无理。
}\meng{三字入神。
}袭人道:“我妈自然不敢强。
且漫说和他好说,\zhu{漫说:同“漫言”、“漫道”,相当于“不要说”。
}
又多给银子;就便不和他好说,一个钱也不给,安心要强留下我,他也不敢不依。
但只是咱们家从没干过这倚势仗贵霸道的事。
这比不得别的东西,因为你喜欢,加十倍利弄了来给你,那卖的人不得吃亏,可以行得。
如今无故平空留下我,于你又无益,反叫我们骨肉分离,这件事,老太太、太太断不肯行的。
”\ji{三驳,不独更有理,且又补出贾府自家慈善宽厚等事。
}宝玉听了,思忖半晌,\ji{正是思忖只有去理实无留理。
}\zhu{忖:音“村”三声,思量,揣度。
}乃说道:“依你说,你是去定了?”\ji{自然。
}袭人道:“去定了。
”\geng{口气像极。
}\ping{袭人一席话,可谓循循善诱,推波助澜,先把宝玉的情绪带到足够的高位,再逐渐达成自己的目的。
}宝玉听了,自思道:“谁知这样一个人,这样薄情无义。
”\ji{余亦如此见疑。
}乃叹道:“早知道都是要去的,\ji{“都是要去的”,妙!可谓触类旁通,活是宝玉。
}\meng{上古至今及后世有情者同声一哭!}我就不该弄了来,临了剩了我一个孤鬼儿。
”\ji{可谓见首知尾,活是宝玉。
\zhu{
见首知尾:从宝玉这时候对袭人离开的不舍与感伤,可以看出宝玉将来对大家终将离散产生的孤独与苦痛。
}
}说着,便赌气上床睡去了。
\ji{又到无可奈何之时了。
}\par
原来袭人在家,听见他母兄要赎他回去,\ji{补前文。
}他就说至死也不回去的。
又说:“当日原是你们没饭吃,就剩我还值几两银子,若不叫你们卖,没有个看着老子娘饿死的理。
\ji{补出袭人幼时艰辛苦状,与前文之香菱、后文之晴雯大同小异,自是又副十二钗中之冠,故不得不补传之。
}
\geng{孝女,义女。
}如今幸而卖到这个地方,\ji{可谓不幸中之幸。
}吃穿和主子一样,又不朝打暮骂。
况且如今爷虽没了,你们却又整理的家成业就,复了元气。
若果然还艰难,把我赎出来,再多掏澄几个钱,
\zhu{
掏澄[tāo chéng]:捞取。
}
\ping{从贾府赎身之后,再高价卖出,一买一卖赚取差价。
}也还罢了,\geng{孝女,义女。
}其实又不难了。
这会子又赎我作什么?权当我死了,\geng{可怜可怜!}再不必起赎我的念头!”\geng{我也要哭。
}\meng{同心同志,更觉幸遇。
}
因此哭闹了一阵。
\ji{以上补在家今日之事,与宝玉问哭一句针对。
}\par
他母兄见他这般坚执,自然必不出来的了。
况且原是卖倒的死契,\zhu{卖倒的死契:指旧社会买卖人口所立的一种字据,其载明永远不能赎取者叫“死契”。
“卖倒”即“卖定”、“卖死”、不可变更的意思。
}
明仗着贾宅是慈善宽厚之家,不过求一求,只怕身价银一并赏了,这是有的事呢。
\ji{又夹带出贾府平素施为来,与袭人口中针对。
}二则,贾府中从不曾作践下人,只有恩多威少的。
\ji{伏下多少后文。
}\meng{铁槛寺凤卿受赂,令人怅恨。
}且凡老少房中所有亲侍的女孩子们,更比待家下众人不同,平常寒薄人家的小姐,也不能那样尊重的。
\ji{又伏下多少后文。
先一句是传中陪客,此一句是传中本旨。
}因此,他母子两个也就死心不赎了。
\ji{既如此,何得袭人又作前语以愚宝玉?不知何意,且看后文。
}次后忽然宝玉去了,他二个又是那般景况,\ji{一件闲事一句闲文皆无,警甚。
}他母子二人心下更明白了,越发石头落了地,而且是意外之想,彼此放心,再无赎念了。
\ji{一段情结。
脂砚。
}\zhu{袭人之母亲自去接袭人回家,一方面是为当初因贫困卖掉女儿的行为赎罪,另一方面是为了袭人未来考虑。
袭人母兄最初打算赎出袭人,以自由身出嫁,或许能成为正妻。
但是在袭人眼中看来,卖过一次自己的母兄是值得提防的人,在情感上不可依赖,甚至说出了“权当我死了”这样绝情的狠话,袭人认为母兄这次也不是单纯的为了自己未来考虑,可能还是要“再多掏澄几个钱”,把自己的婚姻当作交易的筹码。
和家庭关系淡薄、孑然一身的袭人可能认为,与其回家被动接受母兄的安排,前途未卜,不如继续留在豪门,基于和宝玉的亲密关系,主动争取成为姨娘嫁给宝玉。
袭人和宝玉已经建立了亲密的关系,也让母兄看到了他们关系的不一般,袭人母兄可能一方面为了赎罪尊重袭人自己的意见,不去贾府赎人;另一方面也是看到能嫁到公府之家做妾,可能比“再多掏澄几个钱”对自己家更有帮助。
}\par
如今且说袭人自幼见宝玉性格异常,\ji{四字好!所谓“说不得好,又说不得不好”也。
}其淘气憨顽自是出于众小儿之外,更有几件千奇百怪口不能言的毛病儿。
\ji{只如此说更好。
所谓“说不得聪明贤良,说不得痴呆愚昧”也。
}近来仗着祖母溺爱,父母亦不能十分严紧拘管,更觉放荡弛纵,\ji{四字妙评。
脂砚。
}任性恣情,\ji{四字更好。
亦不涉于恶,亦不涉于淫,亦不涉于骄,不过一味任性耳。
}最不喜务正。
\ji{这还是小儿同病。
}每欲劝时,料不能听,今日可巧有赎身之论,故先用骗词,以探其情,以压其气,然后好下箴规。
\zhu{箴(箴音“针”)规:规劝;告诫。
}\ji{原来如此。
}\meng{以此法游刃,有何不可解之牛?}今见他默默睡去了,知其情有不忍,气已馁堕。
\zhu{馁[něi]:丧失勇气。如“气馁”。}
\ji{不独解语,亦且有智。
}自己原不想栗子吃的,只因怕为酥酪又生事故,亦如茜雪之茶等事,\ji{可谓贤而多智术之人。
}是以假以栗子为由,混过宝玉不提就完了。
于是命小丫头子们将栗子拿去吃了,自己来推宝玉。
只见宝玉泪痕满面,\ji{正是无可奈何之时。
}\meng{不知何故,我亦掩涕。
}袭人便笑道:“这有什么伤心的,你果然留我,我自然不出去了。
”宝玉见这话有文章,\ji{宝玉不愚。
}便说道:“你倒说说,我还要怎么留你,我自己也难说了。
”\ji{二人素常情义。
}
袭人笑道:“咱们素日好处,再不用说。
但今日你安心留我,不在这上头。
我另说出两三件事来,你果然依了我,就是你真心留我了,刀搁在脖子上,我也是不出去的了。
”\meng{以此等心,行此等事,昭昭苍天,岂无明见。
}\par
宝玉忙笑道:“你说,那几件?我都依你。
好姐姐,好亲姐姐,\ji{叠二语,活见从纸上走一宝玉下来,如闻其呼、如见其笑。
}别说两三件,就是两三百件,我也依。
\ji{“两三百”不成话,却是宝玉口中。
}只求你们同看着我,守着我,等我有一日化成了飞灰,\ji{脂砚斋所谓“不知是何心思,始得口出此等不成话之至奇至妙之话”,诸公请如何解得,如何评论?}\ji{所劝者正为此,偏于劝时一犯,妙甚!}——飞灰还不好,灰还有形有迹,还有知识。
\ji{灰“还有知识”,奇之不可胜言矣!余则谓人尚无知识者多多。
}等我化成一股轻烟,风一吹便散了的时候,你们也管不得我,我也顾不得你们了。
\meng{人人皆以宝玉为痴,孰不知世人比宝玉更痴。
}那时凭我去,我也凭你们爱那里去就去了。
”\ji{是聪明,是愚昧,是小儿淘气?余皆不知,只觉悲感难言,奇瑰愈妙。
}
话未说完,急的袭人忙握他的嘴,说:“好好的,正为劝你这些,倒更说的狠了。
”宝玉忙说道:“再不说这话了。
”\geng{只说今日一次。
呵呵,玉兄,玉兄,你到底哄的那一个?}袭人道:“这是头一件要改的。
”宝玉道:“改了。
再要说,你就拧嘴。
还有什么?”\par
袭人道:“第二件,你真喜读书也罢,假喜也罢,\geng{新鲜,真新鲜!}
只是在老爷跟前或在别人跟前,你别只管批驳诮谤,\zhu{诮:音“翘”,责备。
}只作出个喜读书的样子来,\ji{宝玉又诮谤读书人?恨此时不能一见如何诮谤。
}\geng{所谓“开方便门”。
《法华经》:“开方便门,示真实相。”
凡用善巧、权宜的方式宣讲佛法,使人容易信解,都称为“开方便门”。
}也教老爷少生些气,\geng{大家听听,可是丫鬟说的话。
}在人前也好说嘴。
他心里想着,我家代代读书,只从有了你,不承望你不喜读书,已经他心里又气又愧了。
而且背前背后乱说那些混话,凡读书上进的人,你就起个名字叫作‘禄蠹’;\zhu{禄蠹:禄:古代官吏的俸禄。
蠹:音“杜”,蛀虫。
禄蠹用以讽刺那些热衷功名利禄的人。
}\ji{二字从古未见,新奇之至!难怨世人谓之可杀,余却最喜。
}\ping{年轻人和天生富贵者总爱嘲笑厌恶经营俗务追求名利者,大概是因为不懂生存之苦,只有年岁渐长或者自幼微寒者,才能丢下浅薄的傲慢。
若是通悟者,此时对红尘中的挣扎过客,心中所存应该是悲悯了。
}又说只除‘明明德’外无书,\zhu{明明德:语出《大学》。
前一个“明”字作动词,彰明、发扬的意思;后一个“明”字修饰“德”,“明德”即所谓至德、完美的德行。
}都是前人自己不能解圣人之书,便另出己意,混编纂出来的。
\ji{宝玉目中犹有“明明德”三字,心中犹有“圣人”二字,又素日皆作如是等语,宜乎人人谓之疯傻不肖。
}这些话,怎么怨得老爷不气、不时时打你?叫别人怎么想你?”宝玉笑道:“再不说了。
那原是那小时不知天高地厚,信口胡说,如今再不敢说了。
\ji{又作是语,说不得不乖觉,\zhu{乖觉:机警,聪敏。
}然又是作者瞒人之处也。
}还有什么?”\par
袭人道:“再不可毁僧谤道,\ji{一件,是妇女心意。
}调脂弄粉。
\ji{二件,若不如此,亦非宝玉。
}还有更要紧的一件,\ji{忽又作此一语。
}再不许吃人嘴上擦的胭脂了,\ji{此一句是闻所未闻之语,宜乎其父母严责也。
}与那爱红的毛病儿。
”宝玉道:“都改,都改。
再有什么,快说。
”袭人笑道:“再也没有了。
只是百事检点些,不任意任情的就是了。
\ji{总包括尽矣。
其所谓“花解语”者,大矣!不独冗冗为儿女之分也。
\zhu{冗:音“容”三声,繁忙。
儿女:男女。
}}你若果都依了,便拿八人轿也抬不出我去了。
”宝玉笑道:“你在这里长远了,不怕没八人轿你坐。
\zhu{八个人抬的娶亲大花轿,旧时的结婚讲究明媒正娶,由夫家用轿迎娶是其主要内容。
现多用来指请的态度诚恳,仪式隆重。
袭人所说“拿八人轿也抬不出我去了”意思是不论母兄如何诚恳地请求袭人赎身,自己也不会离开了。
宝玉所说“不怕没八人轿你坐”暗示将来要娶袭人。
袭人知道按照礼法自己最多成为姨娘,宝玉所言的礼仪是迎娶正妻所用的,故说“没有那个道理”。
}
”袭人冷笑道:“这我可不希罕的。
有那个福气,没有那个道理。
纵坐了,也没甚趣。
”\ji{调侃不浅,然在袭人能作是语,实可爱可敬可服之至,所谓“花解语”也。
}\geng{“花解语”一段,乃袭卿满心满意将玉兄为终身得靠,千妥万当,故有是。
余阅至此,余为袭卿一叹。
丁亥春。
畸笏叟。
}
\meng{真正逼人。
}
\ping{根据脂评,袭人最终没能嫁给贾宝玉,心事终成泡影。
}\par
二人正说着,只见秋纹走进来,说:“快三更了,该睡了。
方才老太太打发嬷嬷来问,我答应睡了。
”宝玉命取表来\ji{照应前凤姐之文。
}
看时,果然针已指到亥正,
\zhu{亥正:晚上十点钟。}
\ji{表则是表的写法,前形容自鸣钟则是自鸣钟,各尽其神妙。
}方从新盥漱,\zhu{从新:重新。
}宽衣安歇,不在话下。
\par
至次日清晨,袭人起来,便觉身体发重,头疼目胀,四肢火热。
先时还扎挣的住,次后捱不住,\zhu{捱:同“挨”,熬,撑。
}只要睡着,因而和衣躺在炕上。
\geng{过下引线。
}宝玉忙回了贾母,传医诊视,说道:“不过偶感风寒,吃一两剂药疏散疏散就好了。
”开方去后,令人取药来煎好,刚服下去,命他盖上被渥汗,
\zhu{渥:同“焐”,音“物”,用热的东西接触凉的使变暖。}
宝玉自去黛玉房中来看视。
\ji{为下文留地步。
}\par
彼时黛玉自在床上歇午,丫鬟们皆出去自便,满屋内静悄悄的。
宝玉揭起绣线软帘,进入里间,只见黛玉睡在那里,忙走上来推他道:“好妹妹,\ji{才住了“好姐姐”,又闻“好妹妹”,大约宝玉一日之中一时之内,此六个字未曾暂离口角。
妙甚!}才吃了饭,又睡觉。
”将黛玉唤醒。
\ji{若是别部书中写,此时之宝玉一进来,便生不轨之心,突萌苟且之念,更有许多贼形鬼状等丑态邪言矣。
此却反推唤醒他,毫不在意,所谓“说不得淫荡”是也。
}黛玉见是宝玉,因说道:“你且出去逛逛,我前儿闹了一夜,今儿还没有歇过来,\ji{补出娇怯态度。
}浑身酸疼。
”宝玉道:“酸疼事小,睡出来的病大。
我替你解闷儿,混过困去就好了。
”\ji{宝玉又知养身。
}黛玉只合着眼,说道:“我不困,只略歇歇儿,你且别处去闹会子再来。
”宝玉推他道:“我往那里去呢,见了别人就怪腻的。
”\ji{所谓只有一颦可对,亦属怪事。
}\par
黛玉听了,嗤的一声笑道:“你既要在这里,那边去老老实实的坐着,咱们说话儿。
”宝玉道:“我也歪着。
”黛玉道:“你就歪着。
”宝玉道:“没有枕头,\ji{缠绵秘密入微。
}咱们在一个枕头上。
”\ji{更妙!渐逼渐近,所谓“意绵绵”也。
}黛玉道:“放屁!\geng{如闻。
}外头不是枕头?拿一个来枕着。
”宝玉出至外间,看了一看,回来笑道:“那个我不要,也不知是那个脏婆子的。
”黛玉听了,睁开眼,\ji{睁眼。
}起身\ji{起身。
}
笑\ji{笑。
}道:“真真你就是我命中的‘天魔星’!\zhu{天魔星:天魔:佛家语,印度古代传说中四魔之一,即“他化自在天魔”,为魔界之主,常率众魔扰人身心、障碍佛法、破坏善事。
这里是缠人的“冤家”的意思。
}\ji{妙语,妙之至!想见其态度。
}请枕这一个。
”说着,将自己枕的推与宝玉,又起身将自己的再拿了一个来,自己枕了,二人对面倒下。
\par
黛玉因看见宝玉左边腮上有钮扣大小的一块血渍,便欠身凑近前来,以手抚之细看,\ji{想见其缠绵态度。
}又道:“这又是谁的指甲刮破了?”\ji{妙极!补出素日。
}宝玉侧身,一面躲,\geng{对“推醒”看。
}一面笑道:“不是刮的,只怕是才刚替他们淘漉胭脂膏子,
\zhu{淘漉[lù]:洗涤、荡除污垢。}
蹭上了一点儿。
”\ji{遥与后文平儿于怡红院晚妆时对照。
\zhu{第四十四回:喜出望外平儿理妆。}
}说着,便找手帕子要揩拭。
\zhu{揩[kāi]:用手、布等擦拭。}
黛玉便用自己的帕子替他揩拭了,\ji{想见情之脉脉,意之绵绵。
}口内说道:“你又干这些事了。
\ji{又是劝戒语。
}干也罢了,\ji{一转,细极!这方是颦卿,不比别人一味固执死劝。
}必定还要带出幌子来。
便是舅舅看不见,别人看见了,又当奇事新鲜话儿去学舌讨好儿,\ji{补前文之未到,伏后文之线脉。
}
吹到舅舅耳朵里,又该大家不干净惹气。
”\ji{“大家”二字,何妙之至、神之至、细腻之至!乃父责其子,纵加以笞楚,
\zhu{
笞(音“痴”):竹板。
楚:荆条。
都是打人的工具。
这里作动词用。
笞楚:即鞭打;抽打。
}
何能使“大家不干净”哉?今偏“大家不干净”,则知贾母如何管孙责子迁怒于众,及自己心中多少抑郁,难堪难禁,\sout{代}[载]忧\sout{代}[载]痛,一齐托出。
}\par
宝玉总未听见这些话,\ji{可知昨夜“情切切”之语亦属行云流水。
}
\geng{一句描写[宝]玉,刻骨刻髓,至已尽矣。
壬午春。
}只闻得一股幽香,却是从黛玉袖中发出,闻之令人醉魂酥骨。
\ji{却像似淫极,然究竟不犯一些淫意。
}
宝玉一把便将黛玉的袖子拉住,要瞧笼着何物。
黛玉笑道:“冬寒十月,\geng{口头语,犹在寒冷之时。
}谁带什么香呢。
”宝玉笑道:“既然如此,这香是那里来的?”黛玉道:“连我也不知道。
\ji{正是。
按谚云:“人在气中忘气,鱼在水中忘水。
”余今续之曰:“美人忘容,花则忘香。
”此则黛玉不知自骨肉中之香同。
}想必是柜子里头的香气,衣服上熏染的也未可知。
”\ji{有理。
}宝玉摇头道:“未必。
这香的气味奇怪,不是那些香饼子、香毬子、
\zhu{毬:同“球”,球形或近似球形的东西。}
香袋子的香。
”\ji{自然。
}黛玉冷笑\ji{冷笑便是文章。
}道:“难道我也有什么‘罗汉’‘真人’给我些香不成?便是得了奇香,也没有亲哥哥亲兄弟弄了花儿、朵儿、霜儿、雪儿替我炮制。
\ji{活颦儿,一丝不错。
}\ping{遥指宝钗冷香丸。
}
我有的是那些俗香罢了!”\par
宝玉笑道:“凡我说一句,你就拉上这么些,不给你个利害,也不知道,从今儿可不饶你了。
”说着翻身起来,将两只手呵了两口,\ji{活画。
}\meng{情景如画。
}便伸手向黛玉膈肢窝内两胁下乱挠。
\zhu{胁:胸部的两侧。
}黛玉素性触痒不禁,宝玉两手伸来乱挠,便笑的喘不过气来,口里说:“宝玉!你再闹,我就恼了。
”\ji{如见如闻。
}宝玉方住了手,笑问道:“你还说这些不说了?”黛玉笑道:“再不敢了。
”一面理鬓\ji{画。
}笑道:“我有奇香,你有‘暖香’没有?”\ji{奇问。
}\par
宝玉见问,一时解不来,\ji{一时原难解,终逊黛卿一等,正在此等处。
}
因问:“什么‘暖香’?”黛玉点头叹笑道:\ji{画。
}“蠢才,蠢才!你有玉,人家就有金来配你;\ping{点出宝玉宝钗之间“金玉良缘”的传闻。
}人家有‘冷香’,\zhu{宝钗的冷香丸。
}你就没有‘暖香’去配?”宝玉方听出来。
\ji{的是颦儿,活画。
然这是阿颦一生心事,故每不禁自及之。
}宝玉笑道:“方才求饶,如今更说狠了。
”说着,又去伸手。
黛玉忙笑道:“好哥哥,我可不敢了。
”宝玉笑道:“饶便饶你,只把袖子我闻一闻。
”说着,便拉了袖子笼在面上,闻个不住。
黛玉夺了手道:“这可该去了。
”宝玉笑道:“去,不能。
咱们斯斯文文的躺着说话儿。
”说着,复又倒下。
黛玉也倒下,用手帕子盖上脸。
\ji{画。
}宝玉有一搭没一搭的说些鬼话,\ji{先一总。
}黛玉只不理。
宝玉问他几岁上京,路上见何景致古迹,扬州有何遗迹故事、土俗民风。
黛玉只不答。
\par
宝玉只怕他睡出病来,\ji{原来只为此故,不暇旁人嘲笑,
\zhu{不暇:没有空闲;顾不过来。}
所以放荡无忌处不特此一件耳。
}便哄他道:“嗳哟!\geng{像个说故事的。
}你们扬州衙门里有一件大故事,你可知道?”黛玉见他说的郑重,且又正言厉色,只当是真事,因问:“什么事?”宝玉见问,便忍着笑顺口诌道:\zhu{诌:音“周”,信口胡说,编瞎话。
}\geng{又哄我看书人。
}“扬州有一座黛山,山上有个林子洞。
”黛玉笑道:“这就扯谎,自来也没听见这山。
”\geng{山名洞名,颦儿已知之矣。
}宝玉道:“天下山水多着呢,你那里知道这些不成?等我说完了,\geng{不先了此句,可知此谎再诌不完的。
}你再批评。
”黛玉道:“你且说。
”宝玉又诌道:“林子洞里原来有群耗子精。
那一年腊月初七日,老耗子升座议事,\ji{耗子亦能升座且议事,自是耗子有赏罚有制度矣。
何今之耗子犹穿壁啮物,其升座者置而不问哉?}因说:‘明日乃是腊八,世上人都熬腊八粥。
如今我们洞中果品短少,\geng{难道耗子也要腊八粥吃?一笑。
}须得趁此打劫些来方妙。
’\ji{议的是这事,宜乎为鼠矣。
}乃拔令箭一枝,遣一能干的小耗\ji{原来能于此者便是小鼠。
}前去打听。
一时小耗回报:‘各处察访打听已毕,惟有山下庙里果米最多。
’\ji{庙里原来最多,妙妙!}老耗问:‘米有几样?果有几品?’小耗道:‘米豆成仓,不可胜记。
果品有五种:一红枣,二栗子,三落花生,
\zhu{落花生:也说花生。}
四菱角,
\zhu{菱角:菱的通称。}
五香芋。
’老耗听了大喜,即时点耗前去。
乃拔令箭问:‘谁去偷米?’一耗便接令去偷米。
又拔令箭问:‘谁去偷豆?’又一耗接令去偷豆。
然后一一的都各领令去了。
\geng{玉兄也知琐碎,以抄近为妙。
}只剩了香芋一种,因又拔令箭问:‘谁去偷香芋?’只见一个极小极弱的小耗\geng{玉兄,玉兄,唐突颦儿了!}应道:‘我愿去偷香芋。
’老耗并众耗见他这样,恐不谙练,
\zhu{谙[ān]练:熟习;明达老练。}
且怯懦无力,都不准他去。
小耗道:‘我虽年小身弱,却是法术无边,口齿伶俐,机谋深远。
\ji{凡三句暗为黛玉作评,讽得妙!}此去管比他们偷的还巧呢。
’众耗忙问:‘如何比他们巧呢?’小耗道:‘我不学他们直偷。
\geng{不直偷,可畏可怕。
}我只摇身一变,也变成个香芋,\meng{作意从此透露。
}滚在香芋堆里,使人看不出,听不见,却暗暗的用分身法搬运,\geng{可怕可畏。
}渐渐的就搬运尽了。
岂不比直偷硬取的巧些?’\ji{果然巧,而且最毒。
直偷者可防,此法不能防矣。
可惜这样才情、这样学术却只一耗耳。
}众耗听了,都道:‘妙却妙,只是不知怎么个变法?你先变个我们瞧瞧。
’小耗听了,笑道:‘这个不难,等我变来。
’说毕,摇身说‘变’,竟变了一个最标致美貌的一位小姐。
\geng{奇文怪文。
}众耗忙笑说:‘变错了,变错了。
原说变果子的,如何变出小姐来?’\ji{余亦说变错了。
}小耗现形笑道:‘我说你们没见世面,只认得这果子是香芋\foot{以上七处“香芋”,依甲辰本,他本均作“香玉”。
按:宝玉此处编故事打趣黛玉,用的是果品“香芋”谐音影射林小姐“香玉”,如直书“香玉”,则非但没有一种果品叫“香玉”,文字也嫌直露。
所以现在已出各主要整理本均依甲辰本作“香芋”,而不采用有多数版本支持的“香玉”。
},却不知盐课林老爷的小姐才是真正的“香玉”呢。
’”\ji{前面有“试才题对额”,故紧接此一篇无稽乱话,前无则可,此无则不可,盖前系宝玉之懒为者,此系宝玉不得不为者。
世人诽谤无碍,奖誉不必。
\ping{庄子《逍遥游》:举世而誉之而不加劝,举世而非之而不加沮。
}}\ping{这里贾宝玉把林黛玉比作“香玉”,前一回贾元春省亲时修改“红香绿玉”为“怡红快绿”,删除了“香玉”两字,可能是暗示了元春对于黛玉的态度。
变身为黛玉的小耗子是“极小极弱的”,“却是法术无边,口齿伶俐,机谋深远”,这也可能是暗示了黛玉身体虽弱,但是聪敏伶俐。
}\par
黛玉听了,翻身爬起来,按着宝玉笑道:“我把你烂了嘴的!我就知道你是编我呢。
”说着,便拧的宝玉连连央告,说:“好妹妹,饶我罢,再不敢了!我因为闻你香,忽然想起这个故典来。
”黛玉笑道:“饶骂了人,
\zhu{饶:不仅。
}
还说是故典呢。”
\geng{“玉生香”是要与“小恙梨香院”对看,愈觉生动活泼,且前以黛玉,后以宝钗,特犯不犯,\zhu{犯:重复。
特犯不犯:第一个“犯”,指的是写了两个女孩,确实有重复之嫌。
第二个“犯”,指的是虽然写了两个女孩,但是行文内容并不重复,每个女孩都有自己的人物特点和故事情节,并非数量上的简单堆砌。
}好看煞!丁亥春。
畸笏叟。
}\par
一语未了,只见宝钗走来,\ji{妙!}\geng{不犯梨香院。
}\ping{宛如天边来,奇了,只要宝玉和黛玉在一起时,宝钗必然出现;宝玉和宝钗一起时,黛玉也必然出现,真是孽缘。
}笑问:“谁说故典呢?我也听听。
”黛玉忙让坐,笑道:“你瞧瞧有谁!他饶骂了人,还说是故典。
”宝钗笑道:“原来是宝兄弟,怨不得他,他肚子里的故典原多。
\ji{妙讽。
}只是可惜一件,\ji{妙转。
}凡该用故典之时,他偏就忘了。
\ji{更妙!}有今日记得的,前儿夜里的芭蕉诗就该记得。
眼面前的倒想不起来,别人冷的那样,他急的只出汗。
\ji{与前“拭汗”二字针对,不知此书何妙至如此,有许多妙谈妙语、机锋诙谐,各得其时,各尽其理,前梨香院黛玉之讽则偏而趣,此则正而趣,二人真是对手,两不相犯。
\zhu{犯:重复。
}}这会子偏又有记性了。
”黛玉听了笑道:“阿弥陀佛!到底是我的好姐姐。
你一般也遇见对子了。
可知一还一报,不爽不错的。
”\zhu{爽:违背,不合。
引申为过失,差错。
}刚说到这里,只听宝玉房中一片声嚷,吵闹起来。
正是——\par
\qi{总评:若知宝玉真性情者,当留心此回。
其与袭人何等留连,其于画美人事,何等古怪。
其遇茗烟事何等怜惜,其于黛玉何等保护。
再袭人之痴忠,画人之惹事,
\zhu{画人之惹事:指宝玉望慰小书房内一轴美人画,恰好撞见茗烟苟且之事。}
茗烟之屈奉,黛玉之痴情,千态万状,笔力劲尖,有水到渠成之象,无微不至。
真画出一个上乘智慧之人,入于魔而不悟,甘心堕落。
且影出诸魔之神通,
\zhu{这句评语令人费解。}
亦非泛泛,有势不能轻登彼岸之形。
凡我众生掩卷自思,或于身心少有补益。
小子妄谈,诸公莫怪。
}
\dai{037}{宝玉在袭人家}
\dai{038}{意绵绵静日玉生香}
\sun{p19-1}{茗烟偷情理亏带宝玉去袭人家,意绵绵静日玉生香}{图右侧和上侧中部:宝玉在小书房内撞见茗烟与丫鬟偷情,宝玉令茗烟带他到袭人家,两人便从后门溜出。
图左侧:宝玉去黛玉房中探视,闻到从黛玉袖中发出一股幽香,一把拉住黛玉的衣袖,要瞧瞧笼着何物。又拉了袖子笼在面上,闻个不住。
正说笑间宝钗走了进来。
}