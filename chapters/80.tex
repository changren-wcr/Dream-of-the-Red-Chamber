\chapter[薛文龙悔娶河东狮 \quad 贾迎春误嫁中山狼]
{薛文龙悔娶河东狮 \quad 贾迎春误嫁中山狼\foot{按:列藏本第七十九回包含了诸本第七十九和第八十回的全部内容,应为原稿面貌。
庚辰本虽已分回但第八十回缺回目,因第七十九回回目已概括了两回内容,本回不采用后人所拟的回目。
}}

\qi{叙桂花妒用实笔,叙孙家恶用虚笔;叙宝玉卧病是省笔,叙宝玉烧香是停笔。
}\par
话说金桂听了,将脖项一扭,嘴唇一撇,\geng{画出一个悍妇来。
}鼻孔里哧哧两声,\geng{真真追魂摄魄之笔。
}拍着掌冷笑道:“菱角花谁闻见香来着?若说菱角香了,正经那些香花放在那里?可是不通之极!”香菱道:“不独菱角花,就连荷叶莲蓬,都是有一股清香的。
但他那原不是花香可比,若静日静夜或清早半夜细领略了去,那一股香比是花儿都好闻呢。
就连菱角、鸡头、\zhu{鸡头:指鸡头米,芡实之俗称。
芡是一种水生植物,其果仁可食。
}苇叶、芦根得了风露,那一股清香,就令人心神爽快的。
”\geng{说的出便是慧心人,何况菱卿哉?}金桂道:“依你说,那兰花桂花倒香的不好了?”\geng{又陪一个兰花,一则是自高声价,二则是诱人犯法。
}香菱说到热闹头上,忘了忌讳,便接口道:“兰花桂花的香,又非别花之香可比。
”一句未完,金桂的丫鬟名唤宝蟾者,忙指着香菱的脸儿说道:“要死,要死!你怎么真叫起姑娘的名字来!”香菱猛省了,\zhu{省:音“醒”,明白,领悟。
}
反不好意思,忙陪笑赔罪说:“一时说顺了嘴,奶奶别计较。
”金桂笑道:“这有什么,你也太小心了。
但只是我想这个‘香’字到底不妥,意思要换一个字,不知你服不服?”香菱忙笑道:“奶奶说那里话,此刻连我一身一体俱属奶奶,何得换一名字反问我服不服,叫我如何当得起。
奶奶说那一个字好,就用那一个。
”金桂笑道:“你虽说的是,只怕姑娘多心,说:‘我起的名字,反不如你?你能来了几日,就驳我的回了。
’”香菱笑道:“奶奶有所不知,当日买了我来时,原是老奶奶使唤的,故此姑娘起得名字。
后来我自伏侍了爷,就与姑娘无涉了。
如今又有了奶奶,益发不与姑娘相干。
况且姑娘又是极明白的人,如何恼得这些呢。
”金桂道:“既这样说,‘香’字竟不如‘秋’字妥当。
菱角菱花皆盛于秋,岂不比‘香’字有来历些。
”香菱道:“就依奶奶这样罢了。
”自此后遂改了秋字,宝钗亦不在意。
\ping{人如其名,改名暗示香菱命运的跌落。
}\ping{宝钗不在意,可能是因为,四面八方都是忧患,也顾不上小小香菱一个了。
}\par
只因薛蟠天性是“得陇望蜀”的,如今得娶了金桂,又见金桂的丫鬟宝蟾有三分姿色,举止轻浮可爱,便时常要茶要水的故意撩逗他。
宝蟾虽亦解事,只是怕着金桂,不敢造次,且看金桂的眼色。
金桂亦颇觉察其意,想着:“正要摆布香菱,无处寻隙,如今他既看上了宝蟾,如今且舍出宝蟾去与他,他一定就和香菱疏远了,我且乘他疏远之时,便摆布了香菱。
那时宝蟾原是我的人,也就好处了。
”打定了主意,伺机而发。
\par
这日薛蟠晚间微醺,又命宝蟾倒茶来吃。
薛蟠接碗时,故意捏他的手。
宝蟾又乔装躲闪,连忙缩手。
两下失误,豁啷一声,茶碗落地,泼了一身一地的茶。
薛蟠不好意思,佯说宝蟾不好生拿着。
宝蟾说:“姑爷不好生接。
”金桂冷笑道:“两个人的腔调儿都够使了。
别打量谁是傻子。
”薛蟠低头微笑不语,宝蟾红了脸出去。
一时安歇之时,金桂便故意的撵薛蟠别处去睡,“省得你馋痨饿眼。
”\zhu{馋痨饿眼:形容贪欲特别强烈的样子。
}薛蟠只是笑。
金桂道:“要作什么和我说,别偷偷摸摸的不中用。
”薛蟠听了,仗着酒盖脸,\zhu{盖脸:北京一带的方言,遮盖、掩饰羞容。
}
便趁势跪在被上拉着金桂笑道:“好姐姐,你若要把宝蟾赏了我,你要怎样就怎样。
你要人脑子也弄来给你。
”金桂笑道:“这话好不通。
你爱谁,说明了,就收在房里,省得别人看着不雅。
我可要什么呢。
”薛蟠得了这话,喜的称谢不尽,是夜曲尽丈夫之道,\zhu{曲尽:竭尽。
}\geng{“曲尽丈夫之道”,奇闻奇语。
}奉承金桂。
次日也不出门,只在家中厮奈,\zhu{厮奈:混日子,意同今北京话之“泡着”。
厮:相,义同厮守、厮混的“厮”。
奈:耐、捱的意思。
}越发放大了胆。
\par
至午后,金桂故意出去,让个空儿与他二人。
薛蟠便拉拉扯扯的起来。
宝蟾心里也知八九,也就半推半就,正要入港。
\zhu{入港:男女发生性关系的隐晦表达。
}谁知金桂是有心等候的,料必在难分之际,便叫丫头小舍儿过来。
原来这小丫头也是金桂从小儿在家使唤的,因他自幼父母双亡,无人看管,便大家叫他作小舍儿,专作些粗笨的生活。
\zhu{生活:活儿;工作。}
\geng{铺叙小舍儿首尾,忙中又点“薄命”二字,与痴丫头遥遥作对。
\zhu{痴丫头:指第七十三回出场的傻大姐,专作粗活,体肥面阔,心性愚顽,又叫他作“痴丫头”。
}}金桂如今有意独唤他来吩咐道:“你去告诉秋菱,到我屋里将手帕取来,不必说我说的。
”\geng{金桂坏极!所以独使小舍为此。
}小舍儿听了,一径寻着香菱说:“菱姑娘,奶奶的手帕子忘记在屋里了。
你去取来送上去岂不好?”香菱正因金桂近日每每的折挫他,不知何意,百般竭力挽回不暇。
\geng{总为痴心人一哭。
}听了这话,忙往房里来取。
不防正遇见他二人推就之际,一头撞了进去,自己倒羞的耳面飞红,忙转身回避不迭。
那薛蟠自为是过了明路的,\zhu{过了明路的:事经公开,毋需躲闪。
}除了金桂,无人可怕,所以连门也不掩,今见香菱撞来,故也略有些惭愧,还不十分在意。
无奈宝蟾素日最是说嘴要强的,今遇见了香菱,便恨无地缝儿可入,忙推开薛蟠,一径跑了,口内还恨怨不迭,说他强奸力逼等语。
薛蟠好容易圈哄的要上手,却被香菱打散,不免一腔兴头变作了一腔恶怒,都在香菱身上,不容分说,赶出来啐了两口,骂道:“死娼妇,你这会子作什么来撞尸游魂!”\zhu{撞尸游魂:义同“撞尸”,骂人到处乱跑。
}\ping{薛蟠自己偷情,被香菱撞到了,反骂香菱“撞尸”,这里的“尸”不就是自己吗?我骂我自己。
}香菱料事不好,三步两步早已跑了。
薛蟠再来找宝蟾,已无踪迹了,于是恨的只骂香菱。
\par
至晚饭后,已吃得醺醺然,洗澡时不防水略热了些,烫了脚,便说香菱有意害他,赤条精光赶着香菱踢打了两下。
香菱虽未受过这气苦,既到此时,也说不得了,只好自悲自怨,各自走开。
\zhu{各自:各方自己;个人自己,这里是第二个意思。
}\par
彼时金桂已暗和宝蟾说明,今夜令薛蟠和宝蟾在香菱房中去成亲,命香菱过来陪自己先睡。
先是香菱不肯,金桂说他嫌脏了,再必是图安逸,怕夜里劳动伏侍,又骂说:“你那没见世面的主子,见一个,爱一个,把我的人霸占了去,又不叫你来。
到底是什么主意,想必是逼我死罢了。
”薛蟠听了这话,又怕闹黄了宝蟾之事,忙又赶来骂香菱:“不识抬举!再不去便要打了!”香菱无奈,只得抱了铺盖来。
金桂命他在地下铺睡。
香菱无奈,只得依命。
刚睡下,便叫倒茶,一时又叫捶腿,如是一夜七八次,总不使其安逸稳卧片时。
那薛蟠得了宝蟾,如获珍宝,一概都置之不顾。
恨的金桂暗暗的发恨道:“且叫你乐这几天,等我慢慢的摆布了来,那时可别怨我!”一面隐忍,一面设计摆布香菱。
\par
半月光景,忽又装起病来,只说心疼难忍,四肢不能转动。
\geng{半月工夫,诸计安矣。
}请医疗治不效,众人都说是香菱气的。
闹了两日,忽又从金桂的枕头内抖出纸人来,上面写着金桂的年庚八字,有五根针钉在心窝并四肢骨节等处。
于是众人反乱起来,当作新闻,先报与薛姨妈。
薛姨妈先忙手忙脚的,薛蟠自然更乱起来,立刻要拷打众人。
金桂笑道:“何必冤枉众人,大约是宝蟾的镇魇法儿。
”\zhu{镇魇:迷信。
指驱使鬼神加害于人的法术。
}\geng{恶极!坏极!}薛蟠道:“他这些时并没多空儿在你房里,何苦赖好人。
”\geng{正要老兄此句。
}金桂冷笑道:“除了他还有谁,莫不是我自己不成!虽有别人,谁可敢进我的房呢。
”薛蟠道:“香菱如今是天天跟着你,他自然知道,先拷问他就知道了。
”金桂冷笑道:“拷问谁,谁肯认?依我说竟装个不知道,大家丢开手罢了。
横竖治死我也没什么要紧,乐得再娶好的。
若据良心上说,左不过你三个多嫌我一个。
”\zhu{左不过:反正,只不过,无非。
}说着,一面痛哭起来。
\par
薛蟠更被这一席话激怒,顺手抓起一根门闩来,\geng{与前要打死宝玉遥遥一对。
\zhu{第三十四回,薛蟠被误认为是宝玉挨打的罪魁祸首,一气之下抓起门闩要打死宝玉。
}}一径抢步找着香菱,不容分说便劈头劈面打起来,一口咬定是香菱所施。
香菱叫屈,薛姨妈跑来禁喝说:“不问明白,你就打起人来了。
这丫头伏侍了你这几年,那一点不周到,不尽心?他岂肯如今作这没良心的事!你且问个清浑皂白,再动粗卤。
”\zhu{粗卤:同“粗鲁”。
粗暴、鲁莽。
}金桂听见他婆婆如此说着,怕薛蟠耳软心活,便益发嚎啕大哭起来,一面又哭喊说:“这半个多月把我的宝蟾霸占了去,不容他进我的房,唯有秋菱跟着我睡。
我要拷问宝蟾,你又护到头里。
你这会子又赌气打他去。
治死我,再拣富贵的标致的娶来就是了,何苦作出这些把戏来!”薛蟠听了这些话,越发着了急。
薛姨妈听见金桂句句挟制着儿子,百般恶赖的样子,十分可恨。
无奈儿子偏不硬气,已是被他挟制软惯了。
如今又勾搭上丫头,被他说霸占了去,他自己反要占温柔让夫之礼。
这魇魔法究竟不知谁作的,实是俗语说的“清官难断家务事”,此事正是公婆难断床帏事了。
因此无法,只得赌气喝骂薛蟠说:“不争气的孽障!骚狗也比你体面些!谁知你三不知的把陪房丫头也摸索上了,叫老婆说嘴霸占了丫头,什么脸出去见人!也不知谁使的法子,也不问青红皂白,好歹就打人。
我知道你是个得新弃旧的东西,白辜负了我当日的心。
他既不好,你也不许打,我立即叫人牙子来卖了他,\zhu{人牙子:即人贩子。
旧时称买卖的中间经纪人为“牙子”,即掮客。
}你就心净了。
”说着,命香菱“收拾了东西跟我来”,一面叫人“去,快叫个人牙子来,多少卖几两银子,拔去肉中刺,眼中钉,大家过太平日子。
”\par
薛蟠见母亲动了气,早也低下头了。
金桂听了这话,便隔着窗子往外哭道:“你老人家只管卖人,不必说着一个扯着一个的。
我们很是那吃醋拈酸容不下人的不成,怎么‘拔出肉中刺,眼中钉’?是谁的钉,谁的刺?但凡多嫌着他,也不肯把我的丫头也收在房里了。
”薛姨妈听说,气的身战气咽道:
\zhu{战:通“颤”,发抖。}
“这是谁家的规矩?婆婆这里说话,媳妇隔着窗子拌嘴。
亏你是旧家人家的女儿!满嘴里大呼小喊,说的是些什么!”薛蟠急的跺脚说:“罢哟,罢哟!看人听见笑话。
”金桂意谓一不作,二不休,越发发泼喊起来了,说:“我不怕人笑话!你的小老婆治我害我,我倒怕人笑话了!再不然,留下他,就卖了我。
谁还不知道你薛家有钱,行动拿钱垫人,\zhu{拿钱垫人:意即行贿。
}又有好亲戚挟制着别人。
你不趁早施为,还等什么?嫌我不好,谁叫你们瞎了眼,三求四告的跑了我们家作什么去了!这会子人也来了,金的银的也赔了,略有个眼睛鼻子的也霸占去了,该挤发我了!”一面哭喊,一面滚揉,自己拍打。
薛蟠急的说又不好,劝又不好,打又不好,央告又不好,只是出入咳声叹气,\zhu{咳[hāi]:
咳声叹气:因忧愁、烦闷或痛苦而发出叹息声。
}抱怨说运气不好。
\geng{果然不差。
}\par
当下薛姨妈早被薛宝钗劝进去了,只命人来卖香菱。
宝钗笑道:“咱们家从来只知买人,并不知卖人之说。
妈可是气的糊涂了,倘或叫人听见,岂不笑话。
哥哥嫂子嫌他不好,留下我使唤,我正也没人使呢。
”薛姨妈道:“留着他还是淘气,不如打发了他倒干净。
”宝钗笑道:“他跟着我也是一样,横竖不叫他到前头去。
从此断绝了他那里,也如卖了一般。
”香菱早已跑到薛姨妈跟前痛哭哀求,只不愿出去,情愿跟着姑娘,薛姨妈也只得罢了。
\ping{香菱被夏金桂改名为秋菱,“宝钗亦不在意”。
此时反而替香菱出头,有点矛盾。
}\par
自此以后,香菱果跟随宝钗去了,把前面路径竟一心断绝。
虽然如此,终不免对月伤悲,挑灯自叹。
本来怯弱,虽在薛蟠房中几年,皆由血分中有病,\zhu{血分:指经血。
}是以并无胎孕。
今复加以气怒伤感,内外折挫不堪,竟酿成干血之症,\zhu{干血之症:即中医所说的“干血痨”,妇科病。
主要症状有面目暗黑、肌肉消瘦干枯、潮热盗汗、口干颧红(颧:音“全”,眼睛下边两腮上面突出的部分)、月经涩少或闭经。
}日渐羸瘦作烧,\zhu{羸:音“雷”,弱。
羸瘦:瘦弱。
}饮食懒进,请医诊视服药亦不效验。
那时金桂又吵闹了数次,气的薛姨妈母女惟暗自垂泪,怨命而已。
薛蟠虽曾仗着酒胆挺撞过两三次,持棍欲打,那金桂便递与他身子随意叫打;这里持刀欲杀时,便伸与他脖项。
薛蟠也实不能下手,只得乱闹了一阵罢了。
如今习惯成自然,反使金桂越发长了威风,薛蟠越发软了气骨。
虽是香菱犹在,却亦如不在的一般,虽不能十分畅快,就不觉的碍眼了,且姑置不究。
\par
如今又渐次寻趁宝蟾。
宝蟾却不比香菱的情性,最是个烈火干柴,既和薛蟠情投意合,便把金桂忘在脑后。
\lie{妙!所谓天理还报不爽。
\zhu{爽:违背,不合。
引申为过失,差错。
}}\ping{夏金桂引狼入室,引火上身。
}近见金桂又作践他,他便不肯服低容让半点。
先是一冲一撞的拌嘴,后来金桂气急了,甚至于骂,再至于打。
他虽不敢还言还手,便大撒泼性,拾头打滚,\zhu{拾头:用头去撞。
}寻死觅活,昼则刀剪,夜则绳索,无所不闹。
薛蟠此时一身难以两顾,惟徘徊观望于二者之间,十分闹的无法,便出门躲在外厢。
\zhu{外厢:外面。
}金桂不发作性气,有时欢喜,便纠聚人来斗纸牌、掷骰子作乐。
又生平最喜啃骨头,每日务要杀鸡鸭,将肉赏人吃,只单以油炸焦骨头下酒。
吃的不奈烦或动了气,便肆行海骂,说:“有别的忘八粉头乐的,\zhu{忘八:即“王八”,乌龟或鳖的俗称,骂人的话,指妻子有外遇的男人。
粉头:娼妓。
有时用来辱骂青年女子。
}我为什么不乐!”薛家母女总不去理他。
薛蟠亦无别法,惟日夜悔恨不该娶这搅家星罢了,都是一时没了主意。
\geng{补足本题。
}于是宁荣二宅之人,上上下下,无有不知,无有不叹者。
\par
此时宝玉已过了百日,出门行走。
亦曾过来见过金桂,“举止形容也不怪厉,一般是鲜花嫩柳,与众姊妹不差上下的人,焉得这等样情性,可为奇之至极”。
\geng{别书中形容妒妇,必曰“黄发黧面”,\zhu{黧:音“离”,黑黄相杂。
}岂不可笑。
}因此心下纳闷。
这日与王夫人请安去,又正遇见迎春奶娘来家请安,说起孙绍祖甚属不端,“姑娘惟有背地里淌眼抹泪的,只要接了来家散诞两日”。
\zhu{散诞:也作散旦、散淡、散荡。舒散、优闲的意思。
}
王夫人因说:“我正要这两日接他去,只因七事八事的都不遂心,\geng{草蛇灰线,后文方不见突然。
}所以就忘了。
前儿宝玉去了,回来也曾说过的。
\geng{补明。
}明日是个好日子,就接去。
”正说着,贾母打发人来找宝玉,说:“明儿一早往天齐庙还愿。
”\zhu{天齐庙:即东岳庙,唐代曾封东岳泰山神为天齐王。
还愿:求神保佑的人实践对神许下的报酬,如祭祀、慈善、捐献。
}宝玉如今巴不得各处去逛逛,听见如此,喜的一夜不曾合眼,盼明不明的。
\par
次日一早,梳洗穿戴已毕,随了两三个老嬷嬷坐车出西城门外天齐庙来烧香还愿。
这庙里已是昨日预备停妥的。
宝玉天生性怯,不敢近狰狞神鬼之像。
这天齐庙本系前朝所修,极其宏壮。
如今年深岁久,又极其荒凉。
里面泥胎塑像皆极其凶恶,是以忙忙的焚过纸马钱粮,\zhu{纸马:旧俗用于祭祀时供焚化的纸糊的人、车、马等造型,也指供焚化的印有神像的纸片。
焚钱粮:又名“烧包袱”,用纸糊的口袋,内装金、银箔纸折叠成的元宝,祭鬼神时焚烧。
}便退至道院歇息。
一时吃过饭,众嬷嬷和李贵等人围随宝玉到处散诞顽耍了一回。
宝玉困倦,复回至静室安歇。
众嬷嬷生恐他睡着了,便请当家的老王道士来陪他说话儿。
这老王道士专意在江湖上卖药,弄些海上方治人射利,\zhu{
海上方:旧时传说,东海(一说在渤海)之中的蓬莱、方丈、瀛洲三神山上,有不死之药。
后人遂称民间验方,秘方为“海上方”,意谓从东海神仙处求得的灵验药方。
射:追求,攫取。
}这庙外现挂着招牌,丸散膏丹,\zhu{丸散膏丹:
是中药方剂的四种剂型。丸是将配方中的若干药碾成细粉,加适量粘合剂如药汁、水、蜜、面糊、米糊等做成圆形颗粒。
散是将配方中的药物锉成细末或碾成细粉。
膏分两种:内服者将药物煎熬多次,去渣取汁浓缩,加冰糖或蜂蜜等而成;外用者,将药物加油类煎炼去滓成膏,直接涂敷或摊制膏药贴敷病部。
丹原指用金、石等药炼制的成药,如硫化汞炼制后升华的晶体;后世把部分精制的丸、散、锭等也称为丹。
}色色俱备,亦长在宁荣两宅走动熟惯,都与他起了个浑号,唤他作“王一贴”,言他的膏药灵验,只一贴百病皆除之意。
当下王一贴进来,宝玉正歪在炕上想睡,李贵等正说“哥儿别睡着了”,厮混着。
看见王一贴进来,都笑道:“来的好,来的好。
王师父,你极会说古记的,\zhu{古记:值得凭吊纪念的旧时景物事迹叫“古记儿”,在这里义近故事、传说。
}说一个与我们小爷听听。
”王一贴笑道:“正是呢。
哥儿别睡,仔细肚里面筋作怪。
”说着,满屋里人都笑了。
\geng{王一贴又与张道士遥遥一对,特犯不犯。
\zhu{犯:重复。
特犯不犯:第一个“犯”,指的是写了两个道士,确实有重复之嫌。
第二个“犯”,指的是虽然写了两个道士,但是行文内容并不重复,每个道士都有自己的人物特点,并非数量上的简单堆砌。
}}\par
宝玉也笑着起身整衣。
王一贴喝命徒弟们快泡好酽茶来。
\zhu{
酽:音“雁”,酒、茶等味厚。
}
茗烟道:“我们爷不吃你的茶,连这屋里坐着还嫌膏药气息呢。
”王一贴笑道:“没当家花花的,\zhu{没当家花花的:这里是“不敢”、“罪过”的意思。
“花花”是词尾,无义。
“家”读轻声。
“没当家”亦作“不当家”、“不当价”,明代刘侗、于奕正《帝京景物略·春场》:“不当价,如吴语云罪过。
”}膏药从不拿进这屋里来的。
知道哥儿今日必来,头三五天就拿香熏了又熏的。
”宝玉道:“可是呢,天天只听见你的膏药好,到底治什么病?”王一贴道:“哥儿若问我的膏药,说来话长,其中细理,一言难尽。
共药一百二十味,君臣相济,\zhu{君臣佐使:
原指君主、臣僚、僚佐、使者四种人分别起着不同的作用,后指中药处方中的各味药的不同作用。
出自《神农本草经》:“上药一百二十种为君,主养命;中药一百二十种为臣,主养性;下药一百二十种为佐使,主治病;用药须合君臣佐使。”
这句话的意思应该是君药臣药相互配合。
}宾主得宜,温凉兼用,贵贱殊方。
内则调元补气,开胃口,养荣卫,\zhu{养荣卫:又叫“养营卫”。
中医把人体中饮食所化的精气和功能分为“营”和“卫”。
“营”指充盈于内、生化血液、营养周身的作用;“卫”指捍卫于外、抗御病邪、保卫肌表的作用。
营与卫互为影响,如果外邪自表而入,就会出现营卫失和的症状。
}宁神安志,去寒去暑,化食化痰;外则和血脉,舒筋络,出死肌,生新肉,去风散毒。
其效如神,贴过的便知。
”宝玉道:“我不信一张膏药就治这些病。
我且问你,倒有一种病可也贴的好么?”王一贴道:“百病千灾,无不立效。
若不见效,哥儿只管揪着胡子打我这老脸,拆我这庙何如?只说出病源来。
”宝玉笑道:“你猜,若你猜的着,便贴的好了。
”王一贴听了,寻思一会,笑道:“这倒难猜,只怕膏药有些不灵了。
”宝玉命李贵等:“你们且出去散散。
这屋里人多,越发蒸臭了。
”李贵等听说,且都出去自便,只留下茗烟一人。
这茗烟手内点着一枝梦甜香,\geng{与前文一照。
\zhu{在第三十七回咏海棠和第七十回咏柳絮,点燃的梦甜香用来作为计时的工具,限定香燃尽前写出作品来。
}}宝玉命他坐在身旁,却倚在他身上。
王一贴心有所动,\geng{四字好。
万端生于心,心邪则意邪。
}便笑嘻嘻走近前来,悄悄的说道:“我可猜着了。
想是哥儿如今有了房中的事情,要滋助的药,可是不是?”话犹未完,茗烟先喝道:“该死,打嘴!”宝玉犹未解,\geng{“未解”妙!若解则不成文矣。
}忙问:“他说什么?”茗烟道:“信他胡说。
”唬的王一贴不敢再问,只说:“哥儿明说了罢。
”\par
宝玉道:“我问你,可有贴女人的妒病方子没有?”\lie{千古奇文奇语,仍归缩结至上半回正文,细密如此。
}王一贴听说,拍手笑道:“这可罢了。
不但说没有方子,就是听也没有听见过。
”宝玉笑道:“这样还算不得什么。
”王一贴又忙道:“这贴妒的膏药倒没经过,倒有一种汤药或者可医,只是慢些儿,不能立竿见影的效验。
”宝玉道:“什么汤药,怎么吃法?”王一贴道:“这叫做‘疗妒汤’:用极好的秋梨一个,二钱冰糖,一钱陈皮,水三碗,梨熟为度,每日清早吃这么一个梨,吃来吃去就好了。
”宝玉道:“这也不值什么,只怕未必见效。
”王一贴道:“一剂不效吃十剂,今日不效明日再吃,今年不效吃到明年。
横竖这三味药都是润肺开胃不伤人的,甜丝丝的,又止咳嗽,又好吃。
吃过一百岁,人横竖是要死的,死了还妒什么!那时就见效了。
”\geng{此科诨一收,\zhu{
诨:音“混”,指古代戏曲中逗笑的台词。
科诨:插科打诨的简称,指穿插在戏曲中令人发笑的滑稽动作和对话。
}
方为奇趣之至。
}说着,宝玉茗烟都大笑不止,骂“油嘴的牛头”。
王一贴笑道:“不过是闲着解午盹罢了,有什么关系。
说笑了你们就值钱。
实告你们说,连膏药也是假的。
我有真药,我还吃了作神仙呢。
有真的,跑到这里来混?”\geng{寓意深远,在此数语。
}正说着,吉时已到,请宝玉出去焚化钱粮散福。
\zhu{焚钱粮:又名“烧包袱”,用纸糊的口袋,内装金、银箔纸折叠成的元宝,祭鬼神时焚烧。
}功课完毕,\zhu{功课:佛教语。
指每日按时诵经念佛等事。
}方进城回家。
\par
那时,迎春已来家好半日,孙家的婆娘媳妇等人已待过晚饭,打发回家去了。
迎春方哭哭啼啼的在王夫人房中诉委曲,\ping{迎春不向继母邢夫人诉苦,反而向王夫人诉苦。
邢夫人这个名义上的母亲若有似无。
}说孙绍祖“一味好色,好赌酗酒,家中所有的媳妇丫头将及淫遍。
略劝过两三次,便骂我是‘醋汁子老婆拧出来的’。
\geng{奇文奇骂。
为迎春一哭。
}\geng{恨薛蟠何等刚霸,偏不能以此语及金桂,使人忿忿。
此书中全是不平,又全是意外之料。
}又说老爷曾收着他五千银子,不该使了他的。
如今他来要了两三次不得,他便指着我的脸说道:‘你别和我充夫人娘子,你老子使了我五千银子,把你准折卖给我的。
\zhu{准折:抵销,弥补,抵偿。
}好不好,打一顿撵在下房里睡去。
当日有你爷爷在时,希图上我们的富贵,赶着相与的。
\ping{在交往中,人人都说是对方上赶着巴结自己。
}论理我和你父亲是一辈,如今强压我的头,卖了一辈。
又不该作了这门亲,倒没的叫人看着赶势利似的。
’”\geng{不通,可笑。
遁辞如闻。
\zhu{遁辞:指理屈辞穷或不愿吐露真意时,用来支吾搪塞的话。
}}一行说,一行哭的呜呜咽咽,连王夫人并众姊妹无不落泪。
王夫人只得用言语解劝说:“已是遇见了这不晓事的人,可怎么样呢。
想当日你叔叔也曾劝过大老爷,不叫作这门亲的。
大老爷执意不听,一心情愿,到底作不好了。
我的儿,这也是你的命。
”迎春哭道:“我不信我的命就这么不好!从小儿没了娘,幸而过婶子这边过了几年心净日子,如今偏又是这么个结果!”\par
王夫人一面解劝,一面问他随意要在那里安歇。
迎春道:“乍乍的离了姊妹们,只是眠思梦想。
二则还记挂着我的屋子,还得在园里旧房子里住得三五天,死也甘心了。
不知下次还可能得住不得住了呢!”王夫人忙劝道:“快休乱说。
不过年轻的夫妻们,闲牙斗齿,亦是万万人之常事,何必说这丧话。
”仍命人忙忙的收拾紫菱洲房屋,命姊妹们陪伴着解释,\zhu{解释:劝解疏通。
}又吩咐宝玉:“不许在老太太跟前走漏一些风声,倘或老太太知道了这些事,都是你说的。
”宝玉唯唯的听命。
\zhu{唯唯:音“委委”,恭敬应诺之词。
}迎春是夕仍在旧馆安歇。
众姊妹等更加亲热异常。
\par
一连住了三日,才往邢夫人那边去。
先辞过贾母及王夫人,然后与众姊妹分别,更皆悲伤不舍。
还是王夫人薛姨妈等安慰劝释,方止住了过那边去。
\geng{凡迎春之文皆从宝玉眼中写出。
前“悔娶河东狮”是实写,“误嫁中山狼”出迎春口中可为虚写,以虚虚实实变幻体格,各尽其法。
}又在邢夫人处住了两日,就有孙绍祖的人来接去。
迎春虽不愿去,无奈惧孙绍祖之恶,只得勉强忍情作辞了。
邢夫人本不在意,也不问其夫妻和睦,家务烦难,只面情塞责而已。
\zhu{面情:犹情面,私人间的情分和面子。用在这里强调是表面工夫、虚情假意。
塞责:对自己应尽的责任敷衍了事。
}终不知端的,且听下回分解。
\par
\qi{总评:此文一为择婿者说法,一为择妻者说法。
择婿者必以得人物轩昂、家道丰厚、荫袭公子为快,择妻者必以得容貌艳丽、妆奁富厚、子女盈门为快,\zhu{子女:年青女子,即陪嫁丫鬟侍女;男女奴隶,即陪嫁的仆人。
}殊不知“以貌取人,失之子羽”。
\zhu{以貌取人,失之子羽:根据外貌来判别人的品质才能。
出自司马迁《史记·仲尼弟子列传》:“吾以言取人,失之宰予,以貌取人,失之子羽。
”孔子有许许多多弟子,其中有一个名叫宰予的,能说会道, 利口善辩。
他开始给孔子的印象不错,但后来渐渐地露出了真相:既无仁德又十分懒惰;大白天不读书听讲,躺在床上睡大觉。
为此,孔子骂他是“朽木不可雕”。
孔子的另一个弟子,叫澹台灭明,字子羽,是鲁国人,比孔子小三十九岁.子羽的体态和相貌很丑陋,想要事奉孔子。
孔子开始认为他资质低下,不会成才。
但他从师学习后,回去就致力于修身实践, 处事光明正大,不走邪路;不是为了公事,从不去会见公卿大夫。
后来,子羽游历到长江,跟随他的弟子有三百人,声誉很高,各诸侯国都传诵他的名字。
孔子听说了这件事,感慨他说:“我只凭言辞判断人品质能力的好坏,结果对宰予的判断就错了;我只凭相貌判断人品质能力的好坏,结果对子羽的判断又错了。
”}试看桂花夏家、指挥孙家,何等可羡可乐。
卒至迎春含悲,薛蟠贻恨,
\zhu{贻[yí]:留下;遗留。}
可慨也夫!\zhu{
也夫:语气助词。
表感叹。
}}
\dai{159}{夏金桂和薛姨妈隔窗拌嘴}
\sun{p80-1}{美香菱屈受贪夫棒}{图左侧:夏金桂装病,并暗示薛蟠是因为香菱针扎纸人咒自己,薛蟠不问青红皂白,抓起门闩暴打香菱。
薛姨妈跑来喝止,并骂薛蟠,和夏金桂拌嘴。
宝钗来劝薛姨妈。
图右侧:宝玉听说薛家的事情,对鲜花嫩柳的夏金桂这样的情性,感到纳闷。
}
\sun{p80-2}{王道士胡诌妒妇方}{图右侧:宝玉随两三个老嬷嬷到天齐庙烧香还愿。
众嬷嬷恐其睡去,便请王道士陪他说笑。
那道士专以丸散膏丹牟利,巧舌如簧吹嘘他的膏药如何包治百病。
宝玉听得有趣,问治女人妒病的药方。
王道士胡诌了疗妒汤,引得宝玉大笑。
图左侧:迎春受委屈回娘家诉苦,和姊妹们相见。
}