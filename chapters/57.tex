\chapter{慧紫鹃情辞试忙玉 \quad 慈姨妈爱语慰痴颦}
\qi{作者发无量愿,欲演出真情种,性地圆光,\zhu{
性地:禀性,性情。
圆光:佛教谓菩萨头顶上的圆轮金光。
}
遍示三千。
\zhu{三千:指三千大千世界。}
遂滴泪为墨,研血成字,画一幅大慈大悲图。
}\par
话说宝玉听王夫人唤他,忙至前边来,原来是王夫人要带他拜甄夫人去。
宝玉自是欢喜,忙去换衣服,跟了王夫人到那里。
见其家中形景,自与荣宁不甚差别,或有一二稍盛者。
细问,果有一宝玉。
甄夫人留席,竟日方回,宝玉方信。
因晚间回家来,王夫人又吩咐预备上等的席面,定名班大戏,请过甄夫人母女。
后二日,他母女便不作辞,\zhu{不作辞:可能是“作辞”,多了一个“不”字。
}回任去了,无话。
\par
这日宝玉因见湘云渐愈,然后去看黛玉。
正值黛玉才歇午觉,宝玉不敢惊动,因紫鹃正在回廊上手里做针黹,便来问他:“昨日夜里咳嗽可好了?”紫鹃道:“好些了。
”宝玉笑道:“阿弥陀佛!宁可好了罢。
”\zhu{宁可:表示希望期待。
}
紫鹃笑道:“你也念起佛来,真是新闻!”宝玉笑道:“所谓‘病笃乱投医’了。
”一面说,一面见他穿着弹墨绫薄棉袄,
\zhu{弹墨:以纸剪镂空图案覆于织品上,用墨色或其它颜色弹或喷成各种图案花样。}
外面只穿着青缎夹背心,
\zhu{夹:有面有里,中间不衬垫絮类。}
宝玉便伸手向他身上摸了一摸,说:“穿这样单薄,还在风口里坐着,看天风馋,\zhu{风馋:形容春天的风对衣衫单薄者的侵袭,容易使人受寒致病。
}时气又不好,你再病了,越发难了。
”紫鹃便说道:“从此咱们只可说话,别动手动脚的。
一年大二年小的,叫人看着不尊重。
打紧的那起混帐行子们背地里说你,\zhu{打紧:要紧;重要。
行(音“杭”)子:贬称自己所不喜爱的东西或人。
}你总不留心,还只管和小时一般行为,如何使得。
姑娘常常吩咐我们,不叫和你说笑。
你近来瞧他远着你还恐远不及呢。
”说着便起身,携了针线进别房去了。
\par
宝玉见了这般景况,心中忽浇了一盆冷水一般,只瞅着竹子,发了一回呆。
因祝妈正来挖笋修竿,
\zhu{竿[gān]:竹竿。}
便怔怔的走出来,一时魂魄失守,心无所知,随便坐在一块山石上出神,不觉滴下泪来。
直呆了五六顿饭工夫,千思万想,总不知如何是可。
偶值雪雁从王夫人房中取了人参来,
\ping{王夫人对黛玉也不是完全冷漠无情。}
从此经过,忽扭项看见桃花树下石上一人手托着腮颊出神,不是别人,却是宝玉。
\ji{画出宝玉来,却又不画阿颦,何等笔力!}\ji{偏不从鹃写,却写一雁,更奇是仍归写鹃。
}雪雁疑惑道:“怪冷的,他一个人在这里作什么?春天凡有残疾的人都犯病,敢是他犯了呆病了?”\ji{写娇憨女儿之心,何等新巧!}
一边想,一边便走过来蹲下笑道:“你在这里作什么呢?”宝玉忽见了雪雁,便说道:“你又作什么来找我?你难道不是女儿?他既防嫌,不许你们理我,你又来寻我,倘被人看见,岂不又生口舌?你快家去罢了。
”雪雁听了,只当是他又受了黛玉的委屈,只得回至房中。
\par
黛玉未醒,将人参交与紫鹃。
紫鹃因问他:“太太做什么呢?”雪雁道:“也歇中觉,所以等了这半日。
姐姐你听笑话儿:我因等太太的工夫,和玉钏儿姐姐坐在下房里说话儿,谁知赵姨奶奶招手儿叫我。
我只当有什么话说,原来他和太太告了假,出去给他兄弟伴宿坐夜,\zhu{伴宿:丧家在出殡的前一夜全家整宿守灵不睡,叫“伴宿”,又称“坐夜”。
}明儿送殡去,跟他的小丫头子小吉祥儿没衣裳,要借我的月白缎子袄儿。
我想他们一般也有两件子的,
\zhu{一般:一样,同样。}
往脏地方儿去恐怕弄脏了,自己的舍不得穿,故此借别人的。
借我的弄脏了也是小事,只是我想,他素日有些什么好处到咱们跟前,所以我说了:‘我的衣裳簪环都是姑娘叫紫鹃姐姐收着呢。
如今先得去告诉他,还得回姑娘呢。
姑娘身上又病着,更费了大事,误了你老出门,不如再转借罢。
’”紫鹃笑道:“你这个小东西倒也巧。
你不借给他,你往我和姑娘身上推,叫人怨不着你。
他这会子就下去了,还是等明日一早才去?”雪雁道:“这会子就去的,只怕此时已去了。
”紫鹃点点头。
雪雁道:“姑娘还没醒呢,是谁给了宝玉气受,坐在那里哭呢。
”紫鹃听了,忙问在那里。
雪雁道:“在沁芳亭后头桃花底下呢。
”\par
紫鹃听说,忙放下针线,又嘱咐雪雁好生听叫:“若问我,答应我就来。
”说着,便出了潇湘馆,一径来寻宝玉,走至宝玉跟前,含笑说道:“我不过说了那两句话,为的是大家好,你就赌气跑了这风地里来哭,作出病来唬我。
”宝玉忙笑道:“谁赌气了!我因为听你说的有理,我想你们既这样说,自然别人也是这样说,将来渐渐的都不理我了,我所以想着自己伤心。
”紫鹃也便挨他坐着。
宝玉笑道:“方才对面说话你尚走开,这会子如何又来挨我坐着?”紫鹃道:“你都忘了?几日前你们姊妹两个正说话,赵姨娘一头走了进来,——我才听见他不在家,所以我来问你。
正是前日你和他才说了一句‘燕窝’就歇住了,总没提起,\zhu{发生在第五十二回。
}我正想着问你。
”宝玉道:“也没什么要紧。
不过我想着宝姐姐也是客中,既吃燕窝,又不可间断,若只管和他要,也太托实。
\zhu{托实:实心眼儿,含有不识相的意思。
}虽不便和太太要,我已经在老太太跟前略露了个风声,只怕老太太和凤姐姐说了。
我告诉他的,竟没告诉完了他。
如今我听见一日给你们一两燕窝,这也就完了。
”紫鹃道:“原来是你说了,这又多谢你费心。
我们正疑惑,老太太怎么忽然想起来叫人每一日送一两燕窝来呢?这就是了。
”宝玉笑道:“这要天天吃惯了,吃上三二年就好了。
”紫鹃道:“在这里吃惯了,明年家去,那里有这闲钱吃这个。
”\par
宝玉听了,吃了一惊,忙问:“谁?往那个家去?”\ji{这句不成话,细读细嚼,方有无限神情滋味。
}紫鹃道:“你妹妹回苏州家去。
”宝玉笑道:\ji{“笑”字奇甚。
}“你又说白话。
苏州虽是原籍,因没了姑父姑母,无人照看,才就了来的。
\zhu{就:接近,靠近,趋向。
}明年回去找谁?可见是扯谎。
”\ji{此论极是不介意。
}\ping{人不愿意相信的事情,自己就会想一堆理由去反驳。
}紫鹃冷笑道:“你太看小了人。
你们贾家独是大族人口多的,除了你家,别人只得一父一母,房族中真个再无人了不成?我们姑娘来时,原是老太太心疼他年小,虽有叔伯,不如亲父母,故此接来住几年。
大了该出阁时,自然要送还林家的。
终不成林家的女儿在你贾家一世不成?
\ping{嫁入贾家,即可一世。}
林家虽贫到没饭吃,也是世代书宦人家,断不肯将他家的人丢在亲戚家,落人的耻笑。
所以早则明年春天,迟则秋天。
这里纵不送去,林家亦必有人来接的。
\ping{这可能暗示了林黛玉的死亡时间。
林家既然已经绝户了,那么接黛玉的大概是林家的鬼魂了吧,目的地不是苏州原籍,而是阴间。
}前日夜里姑娘和我说了,叫我告诉你:将从前小时顽的东西,有他送你的,叫你都打点出来还他。
他也将你送他的打叠了在那里呢。
”宝玉听了,便如头顶上响了一个焦雷一般。
紫鹃看他怎样回答,只不作声。
忽见晴雯找来说:“老太太叫你呢,谁知道在这里。
”紫鹃笑道:“他这里问姑娘的病症。
我告诉了他半日,他只不信。
你倒拉他去罢。
”说着,自己便走回房去了。
\par
晴雯见他呆呆的,一头热汗,满脸紫胀,忙拉他的手,一直到怡红院中。
袭人见了这般,慌起来,只说时气所感,热汗被风扑了。
无奈宝玉发热事犹小可,更觉两个眼珠儿直直的起来,口角边津液流出,皆不知觉。
给他个枕头,他便睡下;扶他起来,他便坐着;倒了茶来,他便吃茶。
众人见他这般,一时忙起来,又不敢造次去回贾母,先便差人出去请李嬷嬷。
\par
一时李嬷嬷来了,看了半日,问他几句话也无回答,用手向他脉门摸了摸,嘴唇人中上边着力掐了两下,\zhu{人中:人体穴位名,在鼻下唇上正中之凹痕处。
对此穴扎针或用指甲力掐,可急救昏迷不醒。
}掐的指印如许来深,竟也不觉疼。
李嬷嬷只说了一声“可了不得了”,“呀”的一声便搂着放声大哭起来。
急的袭人忙拉他说:“你老人家瞧瞧,可怕不怕?且告诉我们去回老太太、太太去。
你老人家怎么先哭起来?”李嬷嬷捶床捣枕说:“这可不中用了!我白操了一世心了!”袭人等以他年老多知,所以请他来看,如今见他这般一说,都信以为实,也都哭起来。
\par
晴雯便告诉袭人,方才如此这般。
袭人听了,便忙到潇湘馆来,见紫鹃正伏侍黛玉吃药,也顾不得什么,便走上来问紫鹃道:“你才和我们宝玉说了些什么?你瞧他去,你回老太太去,我也不管了!”说着,便坐在椅上。
黛玉忽见袭人满面急怒,又有泪痕,举止大变,便不免也慌了,忙问怎么了。
袭人定了一回,哭道:“不知紫鹃姑奶奶说了些什么话,那个呆子眼也直了,手脚也冷了,话也不说了,李妈妈掐着也不疼了,已死了大半个了!\ji{奇极之语。
从急怒姣憨口中描出不成话之话来,方是千古奇文。
五\sout{字}[句]是一口气来的。
}连李妈妈都说不中用了,那里放声大哭。
只怕这会子都死了!”黛玉一听此言,李妈妈乃是经过的老妪,说不中用了,可知必不中用。
哇的一声,将腹中之药一概呛出,抖肠搜肺,炽胃扇肝的痛声大嗽了几阵,一时面红发乱,目肿筋浮,\zhu{筋:称皮下的静脉管,青筋。
浮:在表面上的。
筋浮:类似于青筋暴起,皮下静脉血管因生气愤怒而变得很明显。
}喘的抬不起头来。
紫鹃忙上来捶背,黛玉伏枕喘息半晌,推紫鹃道:“你不用捶,你竟拿绳子来勒死我是正经!”紫鹃哭道:“我并没说什么,不过是说了几句顽话,他就认真了。
”袭人道:“你还不知道他,那傻子每每顽话认了真。
”黛玉道:“你说了什么话,趁早儿去解说,他只怕就醒过来了。
”紫鹃听说,忙下了床,同袭人到了怡红院。
\par
谁知贾母王夫人等已都在那里了。
贾母一见了紫鹃,眼内出火,骂道:“你这小蹄子,和他说了什么?”紫鹃忙道:“并没说什么,不过说几句顽话。
”谁知宝玉见了紫鹃,方“嗳呀”了一声,哭出来了。
众人一见,方都放下心来。
贾母便拉住紫鹃,只当他得罪了宝玉,所以拉紫鹃命他打。
谁知宝玉一把拉住紫鹃,死也不放,说:“要去连我也带了去。
”众人不解,细问起来,方知紫鹃说“要回苏州去”一句顽话引出来的。
贾母流泪道:“我当有什么要紧大事,原来是这句顽话。
”又向紫鹃道:“你这孩子素日最是个伶俐聪敏的,你又知道他有个呆根子,平白的哄他作什么?”薛姨妈劝道:“宝玉本来心实,可巧林姑娘又是从小儿来的,他姊妹两个一处长了这么大,比别的姊妹更不同。
这会子热剌剌的说一个去,\zhu{剌剌:音“辣辣”,语助词,表示加重语气。
热剌剌:形容极亲热、极热情。
}别说他是个实心的傻孩子,便是冷心肠的大人也要伤心。
这并不是什么大病,老太太和姨太太只管万安,\zhu{万安:安心,放心。
}吃一两剂药就好了。
”\par
正说着,人回林之孝家的单大良家的都来瞧哥儿来了。
贾母道:“难为他们想着,叫他们来瞧瞧。
”宝玉听了一个“林”字,便满床闹起来说:“了不得了,林家的人接他们来了,快打出去罢!”贾母听了,也忙说:“打出去罢。
”又忙安慰说:“那不是林家的人。
林家的人都死绝了,没人来接他的,你只放心罢。
”宝玉哭道:“凭他是谁,除了林妹妹,都不许姓林的!”贾母道:“没姓林的来,凡姓林的我都打走了。
”一面吩咐众人:“以后别叫林之孝家的进园来,你们也别说‘林’字。
好孩子们,你们听我这句话罢!”众人忙答应,又不敢笑。
一时宝玉又一眼看见了十锦格子上陈设的一只金西洋自行船,\zhu{十锦:即“什锦”,由多种原料制成或多种花样拼成的。
格子:架子,放置器物的木器。木架上分不同形状的许多层小格,格内可放入各种器皿、用具。也作“槅子”。
自行船:第六十七回也有“自行人”之说,这里的“自行”应该是一种设有机件而能运动的机械装置。
}便指着乱叫说:“那不是接他们来的船来了,\zhu{接他们来的船来了:第一个“来”应该是“去”、“走”,或者删去。
}湾在那里呢。
”\zhu{湾:系泊,在水湾处停泊。
}贾母忙命拿下来。
袭人忙拿下来,宝玉伸手要,袭人递过,宝玉便掖在被中,笑道:“可去不成了!”一面说,一面死拉着紫鹃不放。
\ping{宝玉因黛玉“假离开”而失心迷乱,为后文黛玉“真离开”时的情景埋下伏笔。}
\par
一时人回大夫来了,贾母忙命快进来。
王夫人、薛姨妈、宝钗等暂避里间,贾母便端坐在宝玉身旁。
王太医进来见许多的人,忙上去请了贾母的安,拿了宝玉的手诊了一回。
那紫鹃少不得低了头。
王大夫也不解何意,起身说道:“世兄这症乃是急痛迷心。
古人曾云:‘痰迷有别。
有气血亏柔,饮食不能熔化痰迷者;有怒恼中痰裹而迷者;有急痛壅塞者。
’此亦痰迷之症,\zhu{痰迷之症:中医术语,亦称“痰迷心窍”,由痰阻经络、孔窍引起神志昏迷的一种病症。
亦泛指“中风”病。
}系急痛所致,不过一时壅蔽,
\zhu{壅[yōng]蔽:遮蔽;阻塞。}
较诸痰迷似轻。
”贾母道:“你只说怕不怕,谁同你背医书呢。
”王太医忙躬身笑说:“不妨,不妨。
”贾母道:“果真不妨?”王太医道:“实在不妨,都在晚生身上。
”贾母道:“既如此,请到外面坐,开药方。
若吃好了,我另外预备好谢礼,叫他亲自捧了送去磕头;若耽误了,打发人去拆了太医院大堂。
”王太医只躬身笑说:“不敢,不敢。
”他原听了说“另具上等谢礼命宝玉去磕头”,故满口说“不敢”,竟未听见贾母后来说拆太医院之戏语,犹说“不敢”,贾母与众人反倒笑了。
一时,按方煎了药来服下,果觉比先安静。
无奈宝玉只不肯放紫鹃,只说他去了便是要回苏州去了。
贾母王夫人无法,只得命紫鹃守着他,另将琥珀去伏侍黛玉。
\par
黛玉不时遣雪雁来探消息,这边事务尽知,自己心中暗叹。
幸喜众人都知宝玉原有些呆气,自幼是他二人亲密。
如今紫鹃之戏语亦是常情,宝玉之病亦非罕事,因不疑到别事去。
\par
晚间宝玉稍安,贾母王夫人等方回房去。
一夜还遣人来问讯几次。
李奶母带领宋嬷嬷等几个年老人用心看守,紫鹃、袭人、晴雯等日夜相伴。
有时宝玉睡去,必从梦中惊醒,不是哭了说黛玉已去,便是有人来接。
每一惊时,必得紫鹃安慰一番方罢。
彼时贾母又命将祛邪守灵丹及开窍通神散各样上方秘制诸药,按方饮服。
次日又服了王太医药,渐次好起来。
宝玉心下明白,因恐紫鹃回去,故有时或作佯狂之态。
紫鹃自那日也着实后悔,如今日夜辛苦,并没有怨意。
袭人等皆心安神定,因向紫鹃笑道:“都是你闹的,还得你来治。
也没见我们这呆子听了风就是雨,往后怎么好。
”暂且按下。
\par
因此时湘云之症已愈,天天过来瞧看,见宝玉明白了,便将他病中狂态形容了与他瞧,引的宝玉自己伏枕而笑。
原来他起先那样竟是不知的,如今听人说还不信。
无人时紫鹃在侧,宝玉又拉他的手问道:“你为什么唬我?”紫鹃道:“不过是哄你顽的,你就认真了。
”宝玉道:“你说的那样有情有理,如何是顽话。
”紫鹃笑道:“那些顽话都是我编的。
林家实没了人口,纵有也是极远的。
族中也都不在苏州住,各省流寓不定。
纵有人来接,老太太必不放去的。
”宝玉道:“便老太太放去,我也不依。
”紫鹃笑道:“果真的你不依?只怕是口里的话。
你如今也大了,连亲也定下了,过二三年再娶了亲,你眼里还有谁了?”\par
宝玉听了,又惊问:“谁定了亲?定了谁?”紫鹃笑道:“年里我听见老太太说,要定下琴姑娘呢。
不然那么疼他?”宝玉笑道:“人人只说我傻,你比我更傻。
不过是句顽话,他已经许给梅翰林家了。
果然定下了他,我还是这个形景了?先是我发誓赌咒砸这劳什子,你都没劝过,说我疯的?刚刚的这几日才好了,你又来怄我。
”一面说,一面咬牙切齿的,又说道:“我只愿这会子立刻我死了,把心迸出来你们瞧见了,然后连皮带骨一概都化成一股灰,——灰还有形迹,不如再化一股烟,——烟还可凝聚,人还看见,须得一阵大乱风吹的四面八方都登时散了,这才好!”一面说,一面又滚下泪来。
紫鹃忙上来握他的嘴,替他擦眼泪,又忙笑解说道:“你不用着急。
这原是我心里着急,故来试你。
”宝玉听了,更又诧异,问道:“你又着什么急?”紫鹃笑道:“你知道,我并不是林家的人,我也和袭人鸳鸯是一伙的,偏把我给了林姑娘使。
偏生他又和我极好,比他苏州带来的还好十倍,一时一刻我们两个离不开。
我如今心里却愁,他倘或要去了,我必要跟了他去的。
我是合家在这里,我若不去,辜负了我们素日的情常;\zhu{情常:情分。
}若去,又弃了本家。
所以我疑惑,故设出这谎话来问你,谁知你就傻闹起来。
”宝玉笑道:“原来是你愁这个,所以你是傻子。
从此后再别愁了。
我只告诉你一句趸话:\zhu{趸:整数,整批。
趸话:概括的话。
}活着,咱们一处活着;不活着,咱们一处化灰化烟。
如何?”紫鹃听了,心下暗暗筹画。
\par
忽有人回:“环爷兰哥儿问候。
”宝玉道:“就说难为他们,我才睡了,不必进来。
”婆子答应去了。
紫鹃笑道:“你也好了,该放我回去瞧瞧我们那一个去了。
”宝玉道:“正是这话。
我昨日就要叫你去的,偏又忘了。
我已经大好了,你就去罢。
”紫鹃听说,方打叠铺盖妆奁之类。
宝玉笑道:“我看见你文具里头有三两面镜子,你把那面小菱花的给我留下罢。
我搁在枕头旁边,睡着好照,明儿出门带着也轻巧。
”紫鹃听说,只得与他留下。
先命人将东西送过去,然后别了众人,自回潇湘馆来。
\par
林黛玉近日闻得宝玉如此形景,未免又添些病症,多哭几场。
今见紫鹃来了,问其原故,已知大愈,仍遣琥珀去伏侍贾母。
夜间人定后,紫鹃已宽衣卧下之时,悄向黛玉笑道:“宝玉的心倒实,听见咱们去就那样起来。
”黛玉不答。
紫鹃停了半晌,自言自语的说道:“一动不如一静。
我们这里就算好人家,别的都容易,最难得的是从小儿一处长大,脾气情性都彼此知道的了。
”黛玉啐道:“你这几天还不乏,趁这会子不歇一歇,还嚼什么蛆。
”紫鹃笑道:“倒不是白嚼蛆,我倒是一片真心为姑娘。
替你愁了这几年了,无父母无兄弟,谁是知疼着热的人?趁早儿老太太还明白硬朗的时节,作定了大事要紧。
俗语说‘老健春寒秋后热’,\zhu{老健春寒秋后热:春日渐暖,虽寒也不会持久;立秋之后即转凉,再热也是暂时的。
这里用春寒秋热比喻老年人的健康不易常保。
}倘或老太太一时有个好歹,那时虽也完事,只怕耽误了时光,还不得趁心如意呢。
\ping{谶语,贾母死后宝黛结合希望渺茫。
}公子王孙虽多,那一个不是三房五妾,今儿朝东,明儿朝西?要一个天仙来,也不过三夜五夕,也丢在脖子后头了,甚至于为妾为丫头反目成仇的。
若娘家有人有势的还好些,若是姑娘这样的人,有老太太一日还好一日,若没了老太太,也只是凭人去欺负了。
所以说,拿主意要紧。
姑娘是个明白人,岂不闻俗语说:‘万两黄金容易得,知心一个也难求’。
”黛玉听了,便说道:“这丫头今儿不疯了?怎么去了几日,忽然变了一个人。
我明儿必回老太太退回去,我不敢要你了。
”紫鹃笑道:“我说的是好话,不过叫你心里留神,并没叫你去为非作歹,何苦回老太太,叫我吃了亏,又有何好处?”说着,竟自睡了。
黛玉听了这话,口内虽如此说,心内未尝不伤感,待他睡了,便直泣了一夜,至天明方打了一个盹儿。
次日勉强盥漱了,吃了些燕窝粥,便有贾母等亲来看视了,又嘱咐了许多话。
\par
目今是薛姨妈的生日,自贾母起,诸人皆有祝贺之礼。
黛玉亦早备了两色针线送去。
\zhu{色:品类;种类。}
是日也定了一本小戏请贾母王夫人等,独有宝玉与黛玉二人不曾去得。
至散时,贾母等顺路又瞧他二人一遍,方回房去。
次日,薛姨妈家又命薛蝌陪诸伙计吃了一天酒,连忙了三四天方完备。
\par
因薛姨妈看见邢岫烟生得端雅稳重,且家道贫寒,是个钗荆裙布的女儿,便欲说与薛蟠为妻。
因薛蟠素习行止浮奢,又恐糟塌人家的女儿。
\ping{知子莫若母,母亲都有自知之明,嫌弃自己不成器的儿子。
}
正在踌躇之际,忽想起薛蝌未娶,看他二人恰是一对天生地设的夫妻,因谋之于凤姐儿。
凤姐儿叹道:“姑妈素知我们太太有些左性的,这事等我慢谋。
”因贾母去瞧凤姐儿时,凤姐儿便和贾母说:“薛姑妈有件事求老祖宗,只是不好启齿的。
”贾母忙问何事,凤姐便将求亲一事说了。
贾母笑道:“这有什么不好启齿?这是极好的事。
等我和你婆婆说了,怕他不依?”因回房来,即刻就命人来请邢夫人过来,硬作保山。
\zhu{保山:指媒人。
}
邢夫人想了一想:薛家根基不错,且现今大富,薛蝌生得又好,且贾母硬作保山,将计就计便应了。
贾母十分喜欢,忙命人请了薛姨妈来。
二人见了,自然有许多谦辞。
邢夫人即刻命人去告诉邢忠夫妇。
他夫妇原是此来投靠邢夫人的,如何不依,早极口的说妙极。
贾母笑道:“我爱管个闲事,今儿又管成了一件事,不知得多少谢媒钱?”薛姨妈笑道:“这是自然的。
纵抬了十万银子来,只怕不希罕。
但只一件,老太太既是主亲,
\zhu{主亲:指主持婚事的人。}
还得一位才好。
”贾母笑道:“别的没有,我们家折腿烂手的人还有两个。
”说着,便命人去叫过尤氏婆媳二人来。
贾母告诉他原故,彼此忙都道喜。
贾母吩咐道:“咱们家的规矩你是尽知的,从没有两亲家争礼争面的。
如今你算替我在当中料理,也不可太啬,也不可太费,把他两家的事周全了回我。
”尤氏忙答应了。
薛姨妈喜之不尽,回家来忙命写了请帖补送过宁府。
尤氏深知邢夫人情性,本不欲管,无奈贾母亲自嘱咐,只得应了。
惟有忖度邢夫人之意行事。
薛姨妈是个无可无不可的人,倒还易说。
这且不在话下。
\par
如今薛姨妈既定了邢岫烟为媳,合宅皆知。
邢夫人本欲接出岫烟去住,贾母因说:“这又何妨,两个孩子又不能见面,就是姨太太和他一个大姑,一个小姑,又何妨?况且都是女儿,正好亲香呢。
”邢夫人方罢。
\par
蝌岫二人前次途中皆曾有一面之遇,大约二人心中也皆如意。
只是邢岫烟未免比先时拘泥了些,不好与宝钗姊妹共处闲语;又兼湘云是个爱取笑的,更觉不好意思。
幸他是个知书达礼的,虽有女儿身分,还不是那种佯羞诈愧一味轻薄造作之辈。
宝钗自见他时,见他家业贫寒,二则别人之父母皆年高有德之人,独他父母偏是酒糟透之人,于女儿分中平常;\zhu{分中:犹分内,本分以内。
这句话的意思大概是,邢岫烟的父母对女儿应尽的分内之事也做得很一般。
}邢夫人也不过是脸面之情,亦非真心疼爱;且岫烟为人雅重,迎春是个有气的死人,连他自己尚未照管齐全,如何能照管到他身上,凡闺阁中家常一应需用之物,或有亏乏,无人照管,他又不与人张口,宝钗倒暗中每相体贴接济,也不敢与邢夫人知道,亦恐多心闲话之故耳。
如今却出人意料之外奇缘作成这门亲事。
岫烟心中先取中宝钗,然后方取薛蝌。
有时岫烟仍与宝钗闲话,宝钗仍以姊妹相呼。
\par
这日,宝钗因来瞧黛玉,恰值岫烟也来瞧黛玉,二人在半路相遇。
宝钗含笑唤他到跟前,二人同走至一块石壁后,宝钗笑问他:“这天还冷的很,你怎么倒全换了夹的?”\zhu{夹的:即夹衣,有面有里,中间不衬垫絮类的衣服。
}岫烟见问,低头不答。
宝钗便知道又有了原故,因又笑问道:“必定是这个月的月钱又没得。
凤丫头如今也这样没心没计了。
”岫烟道:“他倒想着不错日子给,因姑妈打发人和我说,一个月用不了二两银子,叫我省一两给爹妈送出去,要使什么,横竖有二姐姐的东西,能着些儿搭着就使了。
姐姐想,二姐姐也是个老实人,也不大留心,我使他的东西,他虽不说什么,他那些妈妈丫头,那一个是省事的,那一个是嘴里不尖的?我虽在那屋里,却不敢很使他们,过三天五天,我倒得拿出钱来给他们打酒买点心吃才好。
因一月二两银子还不够使,如今又去了一两。
前儿我悄悄的把绵衣服叫人当了几吊钱盘缠。
”宝钗听了,愁眉叹道:“偏梅家又合家在任上,后年才进来。
若是在这里,琴儿过去了,好再商议你这事。
离了这里就完了。
如今不先定了他妹妹的事,也断不敢先娶亲的。
\ping{为什么必须妹妹宝琴先嫁,哥哥薛蝌才能娶亲呢?四大家族最近这些年不断衰弱,先前定下的与梅家的亲事,薛家衰落后,从梅家角度讲可能变得不合适了,因此宝琴才要小小年纪进京“备嫁”,可能也存着怕对方悔婚的意思。
第五十回,贾母又亲嘱惜春:“……第一要紧把昨日琴儿和丫头梅花,照模照样,一笔别错,快快添上。
”“丫头梅花”在这部书中是可有可无的人物,这里着重强调,应该是点出宝琴和梅翰林之子的婚约。
至于贾母为何要如此做,可能是薛家衰败之后,梅翰林家想要退亲,这样做是要挽救和巩固宝琴和梅翰林之子的婚约。
而且邢岫烟的家世更是普通,若她嫁入,明显削弱薛家的地位,梅家更会轻视,宝琴的婚事无疑雪上加霜。
}如今倒是一件难事。
再迟两年,又怕你熬煎出病来。
等我和妈再商议,有人欺负你,你只管耐些烦儿,千万别自己熬煎出病来。
不如把那一两银子明儿也越性给了他们,倒都歇心。
你以后也不用白给那些人东西吃,他尖刺让他们去尖刺,很听不过了,各人走开。
倘或短了什么,你别存那小家儿女气,只管找我去。
并不是作亲后方如此,你一来时咱们就好的。
便怕人闲话,你打发小丫头悄悄的和我说去就是了。
”岫烟低头答应了。
\par
宝钗又指他裙上一个碧玉珮问道:“这是谁给你的?”岫烟道:“这是三姐姐给的。
”宝钗点头笑道:“他见人人皆有,独你一个没有,怕人笑话,故此送你一个。
这是他聪明细致之处。
但还有一句话你也要知道,这些妆饰原出于大官富贵之家的小姐,你看我从头至脚可有这些富丽闲妆?然七八年之先,我也是这样来的,如今一时比不得一时了,所以我都自己该省的就省了。
将来你这一到了我们家,这些没有用的东西,只怕还有一箱子。
咱们如今比不得他们了,总要一色从实守分为主,不比他们才是。
”岫烟笑道:“姐姐既这样说,我回去摘了就是了。
”宝钗忙笑道:“你也太听说了。
这是他好意送你,你不佩着,他岂不疑心。
我不过是偶然提到这里,以后知道就是了。
”岫烟忙又答应,又问:“姐姐此时那里去?”宝钗道:“我到潇湘馆去。
你且回去把那当票叫丫头送来,\zhu{当票:当铺付给贷款人作为凭证的票据。
当铺是旧时以典当进行高利剥削的一种行业,收取抵押品,发放贷款.利率很高,若过期不赎,抵押品即归当铺所有。
}我那里悄悄的取出来,晚上再悄悄的送给你去,早晚好穿,不然风扇了事大。
但不知当在那里了?”岫烟道:“叫作‘恒舒典’,是鼓楼西大街的。
”宝钗笑道:“这闹在一家去了。
伙计们倘或知道了,好说‘人没过来,衣裳先过来’了。
”岫烟听说,便知是他家的本钱,也不觉红了脸一笑,二人走开。
\par
宝钗就往潇湘馆来。
正值他母亲也来瞧黛玉,正说闲话呢。
宝钗笑道:“妈多早晚来的?我竟不知道。
”薛姨妈道:“我这几天连日忙,总没来瞧瞧宝玉和他。
所以今儿瞧他二个,都也好了。
”黛玉忙让宝钗坐了,因向宝钗道:“天下的事真是人想不到的,怎么想的到姨妈和大舅母又作一门亲家。
”薛姨妈道:“我的儿,你们女孩家那里知道,自古道:‘千里姻缘一线牵’。
管姻缘的有一位月下老人,预先注定,暗里只用一根红丝把这两个人的脚绊住,凭你两家隔着海,隔着国,有世仇的,也终久有机会作了夫妇。
这一件事都是出人意料之外,凭父母本人都愿意了,或是年年在一处的,以为是定了的亲事,若月下老人不用红线拴的,再不能到一处。
\ping{宝黛年年在一处都以为确定的亲事最后“心事终虚化”,月老红线非人力所能及。}
比如你姐妹两个的婚姻,此刻也不知在眼前,也不知在山南海北呢。
”宝钗道:“惟有妈,说动话就拉上我们。
”一面说,一面伏在他母亲怀里笑说:“咱们走罢。
”黛玉笑道:“你瞧,这么大了,离了姨妈他就是个最老道的,\zhu{老道:这里义同“老到”,老成练达的意思。
}见了姨妈他就撒娇儿。
”薛姨妈用手摩弄着宝钗,叹向黛玉道:“你这姐姐就和凤哥儿在老太太跟前一样,有了正经事就和他商量,没了事幸亏他开开我的心。
我见了他这样,有多少愁不散的。
”\par
黛玉听说,流泪叹道:“他偏在这里这样,分明是气我没娘的人,故意来刺我的眼。
”宝钗笑道:“妈瞧他轻狂,倒说我撒娇儿。
”薛姨妈道:“也怨不得他伤心,可怜没父母,到底没个亲人。
”又摩娑黛玉笑道:“好孩子别哭。
你见我疼你姐姐你伤心了,你不知我心里更疼你呢。
你姐姐虽没了父亲,到底有我,有亲哥哥,这就比你强了。
我每每和你姐姐说,心里很疼你,只是外头不好带出来的。
你这里人多口杂,说好话的人少,说歹话的人多,不说你无依无靠,为人作人配人疼,只说我们看老太太疼你了,我们也洑上水去了。
”\zhu{洑上水:洑:音“父”,游泳。
洑上水:游向上游,比喻巴结有权势的人。
}黛玉笑道:“姨妈既这么说,我明日就认姨妈做娘,姨妈若是弃嫌不认,便是假意疼我了。
”薛姨妈道:“你不厌我,就认了才好。
”宝钗忙道:“认不得的。
”黛玉道:“怎么认不得?”宝钗笑问道:“我且问你,我哥哥还没定亲事,为什么反将邢妹妹先说与我兄弟了,是什么道理?”黛玉道:“他不在家,或是属相生日不对,所以先说与兄弟了。
”宝钗笑道:“非也。
我哥哥已经相准了,只等来家就下定了,也不必提出人来,我方才说你认不得娘,你细想去。
”说着,便和他母亲挤眼儿发笑。
\par
黛玉听了,便也一头伏在薛姨妈身上,说道:“姨妈不打他我不依。
”薛姨妈忙也搂他笑道:“你别信你姐姐的话,他是顽你呢。
”宝钗笑道:“真个的,妈明儿和老太太求了他作媳妇,岂不比外头寻的好?”黛玉便够上来要抓他,口内笑说:“你越发疯了。
”薛姨妈忙也笑劝,用手分开方罢。
又向宝钗道:“连邢女儿我还怕你哥哥糟踏了他,所以给你兄弟说了。
别说这孩子,我也断不肯给他。
前儿老太太因要把你妹妹说给宝玉,偏生又有了人家,不然倒是一门好亲。
前儿我说定了邢女儿,老太太还取笑说:‘我原要说他的人,谁知他的人没到手,倒被他说了我们的一个去了。
’虽是顽话,细想来倒有些意思。
我想宝琴虽有了人家,我虽没人可给,难道一句话也不说。
\ping{贾母在薛宝琴和薛蝌两个人的婚姻上都给予了支持,薛姨妈不得不做出姿态表示报答,即在宝玉的婚事上,从支持王夫人(宝钗),转而支持贾母(黛玉)。}
我想着,你宝兄弟老太太那样疼他,他又生的那样,若要外头说去,断不中意。
不如竟把你林妹妹定与他,岂不四角俱全?”\zhu{四角俱全:完美无缺的意思。
}林黛玉先还怔怔的,听后来见说到自己身上,便啐了宝钗一口,红了脸,拉着宝钗笑道:\ping{“笑”字奇甚。
}“我只打你!你为什么招出姨妈这些老没正经的话来?”宝钗笑道:“这可奇了!妈说你,为什么打我?”紫鹃忙也跑来笑道:“姨太太既有这主意,为什么不和太太说去?”薛姨妈哈哈笑道:“你这孩子,急什么,想必催着你姑娘出了阁,你也要早些寻一个小女婿去了。
”紫鹃听了,也红了脸,笑道:“姨太太真个倚老卖老的起来。
”说着,便转身去了。
黛玉先骂:“又与你这蹄子什么相干?”后来见了这样,也笑起来说:“阿弥陀佛!该,该,该!也臊了一鼻子灰去了!”薛姨妈母女及屋内婆子丫鬟都笑起来。
婆子们因也笑道:“姨太太虽是顽话,却倒也不差呢。
到闲了时和老太太一商议,姨太太竟做媒保成这门亲事是千妥万妥的。
”薛姨妈道:“我一出这主意,老太太必喜欢的。
”\ping{薛姨妈和宝钗黛玉的闺房私语竟然是当着众婆子丫鬟的面说的,贾府里确实没啥隐私可言。
从此可以看出,宝黛结合从贾母到下人是有着广泛共识的,看起来十拿九稳,才更突出了最后“心事终虚化”的无奈。
}\par
一语未了,忽见湘云走来,手里拿着一张当票,口内笑道:“这是个帐篇子?”黛玉瞧了,也不认得。
地下婆子们都笑道:“这可是一件奇货,这个乖可不是白教人的。
”\zhu{乖:灵巧,聪明,伶俐。
}宝钗忙一把接了,看时,就是岫烟才说的当票,忙折了起来。
薛姨妈忙说:“那必定是那个妈妈的当票子失落了,回来急的他们找。
那里得的?”湘云道:“什么是当票子?”众人都笑道:“真真是个呆子,连个当票子也不知道。
”薛姨妈叹道:“怨不得他,真真是侯门千金,而且又小,那里知道这个?那里去有这个?便是家下人有这个,他如何得见?别笑他呆子,若给你们家的小姐们看了,也都成了呆子。
”众婆子笑道:“林姑娘方才也不认得,别说姑娘们。
此刻宝玉他倒是外头常走出去的,只怕也还没见过呢。
”薛姨妈忙将原故讲明。
湘云黛玉二人听了方笑道:“原来为此。
人也太会想钱了,姨妈家的当铺也有这个不成?”众人笑道:“这又呆了。
‘天下老鸹一般黑’,
\zhu{老鸹[guā]:乌鸦的俗称。}
岂有两样的?”薛姨妈因又问是那里拾的?湘云方欲说时,宝钗忙说:“是一张死了没用的,不知那年勾了帐的,香菱拿着哄他们顽的。
”薛姨妈听了此话是真,也就不问了。
一时人来回:“那府里大奶奶过来请姨太太说话呢。
”薛姨妈起身去了。
\par
这里屋内无人时,宝钗方问湘云何处拾的。
湘云笑道:“我见你令弟媳的丫头篆儿悄悄的递与莺儿。
莺儿便随手夹在书里,只当我没看见。
我等他们出去了,我偷着看,竟不认得。
知道你们都在这里,所以拿来大家认认。
”黛玉忙问:“怎么,他也当衣裳不成?既当了,怎么又给你去?”宝钗见问,不好隐瞒他两个,遂将方才之事都告诉了他二人。
黛玉便说“兔死狐悲,物伤其类”,不免感叹起来。
史湘云便动了气说:“等我问着二姐姐去!我骂那起老婆子丫头一顿,给你们出气何如?”说着,便要走。
宝钗忙一把拉住,笑道:“你又发疯了,还不给我坐着呢。
”黛玉笑道:“你要是个男人,出去打一个报不平儿。
你又充什么荆轲聂政,\zhu{荆轲:曾谋刺秦始皇,未成被杀。
聂政:曾代人报仇,刺死韩相侠累后自杀。
俱见《史记·剌客列传》。
历来都被看作重义轻生、为报效知己而慷慨赴死的人物。
}真真好笑。
”湘云道:“既不叫我问他去,明儿也把他接到咱们苑里一处住去,岂不好?”宝钗笑道:“明日再商量。
”说着,人报:“三姑娘四姑娘来了。
”三人听了,忙掩了口不提此事。
要知端的,且听下回分解。
\par
\qi{总评:写宝玉黛玉呼吸相关,不在字里行间,全从无字句处,运鬼斧神工之笔,摄魄追魂,令我哭一回、叹一回,浑身都是呆气。
\hang
写宝钗岫烟相叙一段,真有英雄失路之悲,真有知己相逢之乐。
时方午夜,灯影幢幢,\zhu{幢幢:音“床床”,形容影子摇晃。
}读书至此,掩卷出户,见星月依稀,寒风微起,默立阶除良久。
\zhu{阶除:台阶。
}}
\dai{113}{慧紫鹃情辞试忙玉}
\dai{114}{慈姨妈爱语慰痴颦}
\sun{p57-1}{慧紫鹃情辞试忙玉}{图右下:宝玉来见黛玉,在回廊遇到紫鹃,见其衣着单薄,禁不住伸手摸了一下,紫鹃劝其自重。
图上侧:宝玉受了冷落,不觉呆坐沁芳亭后头桃花底下的石上流泪。
紫鹃听雪雁说了后,找到宝玉促膝相谈,谎称黛玉要回苏州家去,以试其情。
}
\sun{p57-2}{宝玉不忍分别犯病,慈姨妈爱语慰痴颦}{图右下:宝玉害怕黛玉回苏州家去,犯了痴症,贾母急忙来看望。
宝玉见了紫鹃,拉住紫鹃不放。
图左上:宝钗来见黛玉,正值母亲也在。
黛玉见宝钗躺在她母亲怀里撒娇,流下泪来。
薛姨妈忙也把黛玉搂在怀里安慰,并说起了婚姻大事。
}