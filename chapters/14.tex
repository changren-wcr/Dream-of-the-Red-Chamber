\chapter{林如海捐馆扬州城\quad 贾宝玉路谒北静王}
\zhu{
捐:弃。
馆:房舍。
捐馆:又称“捐舍”、“捐馆舍”,死亡的讳称。
}
\par
\jia{凤姐用彩明,因自己识字不多,且彩明系未冠之童。
\hang
写凤姐之珍贵,写凤姐之英气,写凤姐之声势,写凤姐之心机,写凤姐之骄大。
\hang
昭儿回,并非林文、琏文,是黛玉正文。
\zhu{
昭儿传送了林如海逝世的消息,对情节的推动有着重要的作用。
林如海去世使得黛玉常住贾府,所以脂批说“是黛玉正文”。
}
\hang
牛,丑也。
清,属水,子也。
柳拆卯字。
彪拆虎字,寅字寓焉。
陈即辰。
翼火为蛇,巳字寓焉。
马,午也。
魁拆鬼,鬼,金羊,未字寓焉。
侯、猴同音,申也。
晓鸣,鸡也,酉字寓焉。
石即豕,亥字寓焉。
其祖曰守业,即守夜也,犬字寓焉。
此所谓十二支寓焉。
\zhu{这条评语是对送殡人名的解析,该评语后文还会出现,注解置于评语再次出现时。}
\hang
路谒北静王,是宝玉正文。
}\par
\qi{家书一纸千金重,勾引难防嘱下人。
任你无双肝胆烈,多情念起自眉颦。
}\par
诗云:……\par
\hop
话说宁国府中都总管来升闻得里面委请了凤姐,
\zhu{都总管:始于五代,本为军队之最高统帅。这里乃指旧时王公官僚府邸之奴仆总头目。}
因传齐同事人等说道:“如今请了西府里琏二奶奶管理内事。
倘或他来支取东西或是说话,我们须要比往日小心些。
每日大家早来晚散,宁可辛苦这一个月,过后再歇着,不要把老脸面丢了。
\geng{此是都总管的话头。
}那是个有名的烈货,脸酸心硬,一时恼了不认人的。
”众人都道:“有理。
”又有一个笑道:“论理,我们里面也须得他来整治整治,\geng{伏线在二十板之误差妇人。
}都特不像了。
”\ping{连家中下人都觉得不像话,宁府主人是如何惫懒?}正说着,只见来旺媳妇拿了对牌,来领取呈文、京榜纸札,\zhu{呈文、京榜:都是纸的名称。
呈文纸是一种质地较结实、价钱较便宜的纸,旧时书写呈文及商店簿记多用之;因其含有麻质,又称麻呈文。
京榜是一种比较高级的榜纸,因其规格适宜于向京城销售,故称京榜。
纸札:也作“纸扎”。
这里是“纸张”的意思。
札:古代无纸,字写在小木板上,叫“札”。
}
票上批着数目。
众人连忙让坐倒茶,一面命人按数取纸来抱着,同来旺媳妇一路行来,至仪门口,方交与来旺媳妇自己抱进去了。
\par
凤姐即命彩明定造簿册。
\jia{宁府如此大家,阿凤如此身份,岂有使贴身丫头与家里男人答话交事之理呢?此作者忽略之处。
}\geng{彩明系未冠小童,阿凤便于出入使令者。
老兄并未前后看明是男是女,乱加批驳。
可笑。
}  
\geng{且明写阿凤不识字之故。
壬午春。
}即时传来升媳妇,兼要家口花名册来查看,又限于明日一早传齐家人媳妇进来听差等语。
大概点了一点数目单册,\jia{已有成见。
}问了来升媳妇几句话,便坐了车回家。
一宿无话。
\par
至次日,卯正二刻便过来了。
\zhu{
卯正二刻:早上六点半。
}
那宁国府中婆娘媳妇闻得到齐,只见凤姐正与来升媳妇分派,众人不敢擅入,只在窗外听觑。
\jia{传神之笔。
}
\zhu{觑:音“去”,眯着眼注视。}
只听凤姐与来升媳妇道:“既托了我,我就说不得要讨你们嫌了。
\jia{先站地步。
}我可比不得你们奶奶好性儿,由着你们去,再不要说你们这府里‘原是这样的’,\jia{此话听熟了。
一叹!}\meng{“不要说‘原是这样’的话”,破尽痼弊根底。
}\ping{下属排挤空降领导的标准语录。
}这如今可要依着我,\jia{婉转得妙!}行错我半点儿,管不得谁是有脸的,谁是没脸的,一例现清白处治!”说着,便吩咐彩明念花名册,按名一个一个的唤进来看视。
\geng{量才而用之意。
}\par
一时看完了,便又吩咐道:“这二十个分作两班,一班十个,每日在里头单管人来客往倒茶,别的事不用他们管。
这二十个也分两班,每日单管本家亲戚茶饭,别的事也不用他们管。
这四十个人也分作两班,单在灵前上香添油、挂幔守灵、供饭供茶、随起举哀,\zhu{随起举哀:这里指分派奴仆随同死者亲眷一起号哭。
举哀本是孝眷的事,但旧时有钱人家为了装潢门面,也令奴仆或专门雇人来一同哭丧,以示悲痛。
}别的事也不与他们相干。
这四个人单在内茶房收管杯碟茶器,若少一件,便叫他四个人描赔。
\zhu{描赔:照原样赔偿。
描:照底样描摹。
}这四个人单管酒饭器皿,少一件,也是他四个人描赔。
这八个人单管监收祭礼。
这八个人单管各处灯油、蜡烛、纸札,我总支了来,交与你八个,然后按我的定数再往各处去分派。
这三十个每日轮流各处上夜,照管门户,监察火烛,打扫地方。
这下剩的按着房屋分开,某人守某处,某处所有桌椅、古董起,至于痰盒掸帚,一草一苗,或丢或坏,就和守这处的人算帐描赔。
\ping{集体负责等于没人负责,责任到人,承包到人。
}来升家的每日揽总查看,或有偷懒的,赌钱吃酒的,打架拌嘴的,立刻来回我。
你有徇情,经我查出,三四辈子的老脸就顾不成了。
如今都有了定规,以后那一行乱了,只和那一行说话。
素日跟我的,随身自有钟表,不论大小事,我是皆有一定的时辰。
横竖你们上房里也有时辰钟。
\zhu{
北宋时开始将每个时辰分为“初”、“正”两部分,分十二时辰为二十四,称“小时”。
子时从当天夜里十一点到第二天凌晨一点,子初从夜里十一点开始到零点,子正从零点开始到凌晨一点。
以此类推,子丑寅卯辰巳午未申酉戌亥代表一天中的十二个时辰。
清代正式规定一昼夜为九十六刻,每个时辰八刻,又区分为上四刻和下四刻:初初刻、初一刻、初二刻、初三刻、正初刻、正一刻、正二刻、正三刻;正初刻又叫初四刻,下一时辰之初初刻又叫正四刻。
}
卯正二刻我来点卯,\zhu{卯正二刻:早上六点半。}巳正吃早饭,\zhu{巳正:上午十点钟。}凡有领牌、回事的,只在午初刻。
\zhu{午初刻:上午十一点。}
戌初烧过黄昏纸,\zhu{戌初:晚上七点。黄昏纸:旧时有丧人家,每天按一定时间在灵前烧纸钱。
日落黄昏时烧的那一次,叫“黄昏纸”。
}我亲到各处查一遍,回来上夜的交明钥匙。
第二日还是卯正二刻过来。
说不得咱们大家辛苦这几日,\jia{是协理口气,好听之至!}\geng{所谓先礼后兵是也。
}事完,你们家大爷自然赏你们。
”\geng{滑贼,好收煞。
}\par
说毕,又吩咐按数发与茶叶、油烛、鸡毛掸子、笤帚等物,一面又搬取家伙:桌围、椅搭、坐褥、毡席、痰盒、脚踏之类。
一面交发,一面提笔登记,某人管某处,某人领某物,开得十分清楚。
众人领了去,也都有了投奔,不似先时只拣便宜的做,剩下苦差没个招揽。
各房中也不能趁乱失迷东西。
便是人来客往,也都安静了,不比先前正摆茶又去端饭,正陪举哀又顾接客。
如这些无头绪、荒乱、推托、偷闲、窃取等弊,次日一概都蠲了。
\zhu{蠲:音“捐”,减去,免除。
}\par
凤姐儿见自己威重令行,心中十分得意。
\ping{凤姐热爱整理庶务,是典型管理人才。
}因见尤氏犯病,贾珍又过于悲哀,\ping{作者时刻不忘提醒读者贾珍对秦可卿的感情不寻常。
}不大进饮食,自己每日从那府里煎了各色细粥、精致小菜,命人送来劝食。
\geng{写凤之心机。
}贾珍也另外吩咐每日送上等菜到抱厦内,单与凤姐吃。
\geng{写凤之珍贵。
}那凤姐不畏勤劳,\qi{不畏勤劳者,一则任专而易办,一则技痒而莫遏。
士为知己者死。
不过勤劳,有何可畏?}天天于卯正二刻就过来点卯理事,\geng{写凤之英勇。
}独在抱厦内起坐,不与众妯娌合群,
\zhu{妯娌:音“轴里”,兄弟之妻相互的称呼。}
便有堂客来往,也不迎会。
\geng{写凤之骄大。
}\par
这日,正五七正五日上,
\zhu{五七:即第五个“七”。人死后,每七天为一周期,请僧道祭祀超度,称为“七”。通常经七个“七”才掩灵停祭。}
那应佛僧正开方破狱,\zhu{应佛僧:也叫“应付僧”、“应赴僧”,专门支应佛事的和尚。
开方破狱:民间习俗在人死亡后邀僧尼、道士大作超度亡灵的活动之一种。
开方(放):即开度。
《愚贤经》卷六:“唯愿如来当见哀愍,暂下开度。
”破狱:即诵念《破地狱偈文》以拯救亡灵出地狱得解脱而往生,见《宗镜录》卷九。
}传灯照亡,\zhu{传灯照亡:旧时迷信,认为人死后走向冥途,黑暗无边,而佛法能破除黑暗,犹如明灯。
因此于人将死时在脚后燃灯以照亡灵,故云“传灯照亡”。
}
参阎君,\zhu{参:古代下级见上级叫“参”。
阎君:传说中的阎罗,即地狱中的鬼王,为梵语Yama的音译。
}拘都鬼,\zhu{拘:拘留,拘禁。
都鬼:鬼域中的头头。
}延请地藏王,\zhu{地藏王:菩萨名,他“安忍不动如大地,静虑深密如秘藏”,故名地藏。
据佛教传说,他于释迦既灭之后,弥勒未生之前,在“人天地狱”之中救苦救难。
}开金桥,\zhu{开金桥:迷信传说,“善人”死后鬼魂所走的是金桥。
为死者开金桥,使他来世能“托生”于福禄之地。
}引幢幡;\zhu{幢幡:音“床番”,都是旗子一类的东西。
幢:竿头安装宝珠,竿身饰以锦帛的旗子。
幡:音“翻”,挑起来直着挂的长条形旗子。
}那道士们正伏章申表,\zhu{伏章申表:道士斋醮时俯首屈身恭读表章。
斋醮:道教设坛祭祷的一种仪式。即供斋醮神,借以求福免灾。其法为清心洁身,筑坛设供,书表章以祷神灵。
这里章与表皆系向上帝奏告的文书。
}
朝三清,\zhu{三清:道教合称该教的最高境界“玉清”、“上清”、“太清”为“三清”;也称居住在其中的“玉清元始天尊”、“上清灵宝天尊”、“太清太上老君”三位尊神为“三清”。
}叩玉帝;\zhu{玉帝:即玉皇大帝,是道教所尊奉的最高天神。
}禅僧们行香,放焰口,\zhu{放焰口:和尚替丧事人家念“焰口经”及施舍饮食于众鬼神,为饿鬼超度为死者祈福的迷信活动。
焰口:据佛教传说,地狱中的饿鬼,腹大如山,喉细似针,一切饮食到了口边即化为火炭,故称“焰口”。
}拜水忏;\zhu{拜水忏:和尚念“水忏经”来为死者祈求免除冤孽灾祸的迷信活动。
水忏:又叫慈悲水忏,佛教经文之一。
据说是唐代悟达禅师遇异僧用水替他洗好人面疮后,他为报恩而作。
}又有十三众青年尼僧,
\zhu{尼:佛教指出家修行的女子。}
搭绣衣,靸红鞋,
\zhu{靸:音“洒”,穿鞋时把鞋后帮踩在脚后跟下,拖着走。}
在灵前默诵接引诸咒,\zhu{接引咒:接引死者至“极乐世界”的咒语。
}十分热闹。
\geng{如此写得可叹可笑。
}那凤姐必知今日人客不少,在家中歇宿一夜,至寅正,
\zhu{寅正:早上四点钟。}
平儿便请起来梳洗。
及收拾完备,更衣盥手,吃了两口奶子糖粳粥,漱口已毕,已是卯正二刻了。
\zhu{卯正二刻:早上六点半。}
来旺媳妇率领诸人伺候已久。
凤姐出至厅前,上了车,前面打了一对明角灯,\zhu{明角灯:又叫羊角灯,灯罩用羊角胶制成,半透明,能防风雨。
}大书“荣国府”三个大字,款款来至宁府。
大门上门灯朗挂,\zhu{朗:明朗。
}两边一色戳灯照如白昼,\zhu{戳灯:又名高灯。
是一种竖在地上的灯笼,有长柄,可插在底座上,也可扛着行走。
}白茫茫穿孝仆从两边侍立。
请车至正门上,小厮等退去,众媳妇上来揭起车帘。
凤姐下了车,一手扶着丰儿,两个媳妇执着手把灯罩,簇拥着凤姐进来。
宁府诸媳妇迎来请安接待。
凤姐缓缓走入会芳园中登仙阁灵前,一见了棺材,那眼泪恰似断线珍珠滚将下来。
\ping{此时怕是凤姐人生巅峰时刻,夫妻温存,才尽其用,执掌一府丧白之事。
可执掌的却是好姐妹的葬礼,一开始就滚下来,事业巅峰是珍惜之人的葬礼,到底悲凉底色。
}院中许多小厮垂手伺候烧纸。
凤姐吩咐得一声:“供茶,烧纸。
”只听得一棒锣鸣,诸乐齐奏,\geng{谁家行事?宁不堕泪!}早有人端过一张大圈椅来,放在灵前,凤姐坐了,放声大哭。
于是里外男女上下,见凤姐出声,都忙接声嚎哭。
\par
一时贾珍、尤氏遣人来劝,凤姐方才止住。
来旺媳妇献茶漱口毕,凤姐方起身,别过族中诸人,自入抱厦内来,按名查点,各项人数都已到齐,只有迎送亲客上的一人未到。
\geng{须得如此,方见文章妙用。
余前批非谬。
}即命传到。
那人已张惶愧惧。
凤姐冷笑\jia{凡凤姐恼时,偏偏用“笑”字,是章法。
}道:“我说是谁误了,原来是你!\geng{四字有神,是有名姓上等人口气。
}你原比他们有体面,所以才不听我的话。
”那人道:“小的天天来的早,只有今日醒了觉得早些,因又睡迷了,来迟了一步,求奶奶饶过这次。
”正说着,只见荣国府中的王兴媳妇来了,\jia{惯起波澜,惯能忙中写闲,又惯用曲笔,又惯综错,真妙!}\geng{偏用这等闲文间住。
}在前面探头。
\par
凤姐且不发放这人,\geng{的是凤姐作\sout{仿}[派]。
}却先问:“王兴媳妇作什么?”王兴媳妇巴不得先问他完了事,连忙进来说:“领牌取线,打车轿网络。
”\zhu{车轿网络:车轿上用丝线编织成的网状装饰品。
}\geng{是丧事中用物,闲闲写却。
}说着,将个帖儿递上去。
凤姐命彩明念道:“大轿两顶,小轿四顶,车四辆,共用大小络子若干根,用珠儿线若干斤。
”凤姐听了,数目相合,便命彩明登记,取荣府对牌掷下。
王兴家的去了。
\par
凤姐方欲说话时,只见荣府四个执事人进来,都是要支取东西领牌来的。
凤姐命彩明要了帖儿念过,听了共四件,凤姐因指两件说道:“这两件开销错了,再算清了来取。
”\geng{好看煞,这等文字。
}说着掷下帖子来。
那二人扫兴而去。
\par
凤姐因见张材家的在旁,\geng{又一顿挫。
}因问:“你有什么事?”张材家的忙取帖儿回说道:“就是方才车轿围做成,领取裁缝工银若干两。
”凤姐听了,便收了帖子,命彩明登记,待王兴家的交过牌,得了买办的回押,相符,然后方与张材家的去领。
一面又命念那一个,是为宝玉外书房完竣,支买纸料糊裱。
\geng{却从闲中又引出一件关系文字乎?}凤姐听了,即命收帖儿登记,待张材家的缴清,又发与这人去了。
\par
凤姐便说道:“明儿他也睡迷了,后儿我也睡迷了,\jia{接上文,一点痕迹俱无,且是仍与方才诸人说话神色口角。
}\geng{接的紧,且无痕迹,是山断云连法也。
}将来都没有人了。
本来要饶你,只是我头一次宽了,下次人就难管,不如开发的好。
\zhu{开发:发落,处置。}
”登时放下脸来,喝命:“带出去,打二十大板!”一面又掷下宁府对牌:“出去说与来升,革他一月银米!”众人听了,又见凤姐眉立,\geng{二字如神。
}知是恼了,不敢怠慢,拖人的出去拖人,执牌传谕的忙去传谕。
那人身不由己,已拖出去挨了二十大板,还要进来叩谢。
凤姐道:“明儿再有误的打四十,后日的六十,有不怕打的只管误!”说着,吩咐:“散了罢。
”窗外众人听说,方各自执事去了。
彼时荣国、宁国两处执事领牌交牌的人来往不绝,那抱愧被打之人含羞去了,这才知道凤姐的利害。
\jia{又伏下文,非独为阿凤之威势费此一段笔墨。
}
众人不敢偷安,自此兢兢业业,\geng{收拾得好。
}执事保守,不在话下。
\ping{典型杀鸡儆猴。
}\par
如今且说宝玉\geng{忙中闲笔。
}因见今日人众,恐秦钟受了委曲,因默与他商议,要同他往凤姐处来坐。
秦钟道:“他的事多,况且不喜人去,咱们去了,他岂不烦腻?”\jia{纯是体贴人情。
}宝玉道:“他怎好腻我们,不相干,只管跟我来。
”说着,便拉了秦钟,直至抱厦。
凤姐才吃饭,见他们来了,便笑道:“好长腿子,快上来罢。
”宝玉道:“我们偏了。
”\zhu{偏了:谦词,占先、僭越之意。
这里是表示自己已经吃过了的客气话。
}
\geng{家常戏言,毕肖之至!}凤姐道:“在这边外头吃的,还是那边吃的?”宝玉道:“这边同那些浑人\jia{奇称。
试问谁是清人?}吃什么!原是那边,我们两个同老太太吃了来的。
”一面归座。
\par
凤姐吃毕饭,就有宁国府中的一个媳妇来领牌,为支取香灯事。
凤姐笑道:“我算着你们今日该来支取,总不见来,想是忘了。
这会子到底来取,要忘了,自然是你们包出来,都便宜了我。
”那媳妇笑道:“何尝不是忘了,\jia{此妇亦善迎合。
}\geng{下人迎合凑趣,毕真。
}方才想起来。
再迟一步,也领不成了。
”说罢,领牌而去。
\par
一时登记交牌。
秦钟因笑道:“你们两府里都是这牌,倘或别人私弄一个,支了银子跑了,怎样?”\geng{小人语。
}凤姐笑道:“依你说,都没王法了。
”宝玉道:“怎么咱们家没人来领牌子做东西?”\geng{写不理家务公子之语。
}凤姐道:“人家来领的时候,你还做梦呢。
\geng{言甚是也。
}
我且问你,你们这夜书多早晚才念呢?”\geng{补前文之未到。
}
\zhu{夜书:夜里读书,所以需要为宝玉建设书房。}
宝玉道:“巴不得这如今就念才好,他们只是不快收拾出书房来,这也没法。
”凤姐笑道:“你请我一请,包管就快了。
”宝玉道:“你要快也不中用。
他们该作到那里的,自然就有了。
”凤姐笑道:“便是他们作,也得要东西去,搁不住我不给对牌是难的。
”宝玉听说,便猴\geng{诗中知有炼字一法,不期于《石头记》中多得其妙。
}向凤姐身上立刻要牌,\zhu{猴:这里作动词用,形容像猴子一样屈身攀抱、纠缠。
}说:“好姐姐,给出牌子来,叫他们要东西去。
”凤姐道:“我乏的身上生疼,还搁的住你揉搓。
你放心罢,今儿才领了纸裱糊去了。
他们该要的,还等叫去呢,可不傻了?”宝玉不信,凤姐便叫彩明查册子与宝玉看了。
\par
正闹着,人回:“苏州去的人昭儿来了。
”\jia{接得好!}凤姐急命唤进来。
昭儿打千请安。
凤姐儿便问:“回来做什么?”昭儿道:“二爷打发回来的。
林姑老爷是九月初三日巳时没的。
\zhu{巳时:上午九点。}
\jia{颦儿方可长居荣府之文。
}二爷带了林姑娘\geng{暗写黛玉。
}同送林姑老爷的灵到苏州,大约赶年底就回来了。
二爷打发小的来报个信请安,讨老太太示下,还瞧瞧奶奶家里好,叫把大毛衣服带几件去。
”凤姐道:“你见过别人了没有?”昭儿道:“都见过了。
”说毕,连忙退出。
凤姐向宝玉笑道:“你林妹妹可在咱们家住长了。
”\geng{此系无意中之有意,妙!}宝玉道:“了不得!想来这几日他不知哭的怎么样呢!”说着蹙眉长叹。
\par
凤姐见昭儿回来,因当着人未及细问贾琏,心中自是记挂。
待要回去,争奈事情繁杂,一时去了恐有延迟失误,惹人笑话。
少不得耐到晚上回来,复命昭儿进来,细问一路平安信息。
连夜打点大毛衣服,和平儿亲自检点包裹,再细细追想所需\meng{“追想所需”四字,写尽能事者之所以[为]能事者之底蕴。
}何物,一并包藏交付。
又细细吩咐昭儿“在外好生小心伏侍,不要惹你二爷生气;时时劝他少吃酒,别勾引他认得浑账女人,\jia{切心事耶?},回来打折你的腿”\jia{此一句最要紧。
}等语。
赶乱完了,天已四更将尽,纵睡下,又走了困,\geng{此为病源伏线。
后文方不突然。
}不觉又是天明鸡唱,忙梳洗过宁府中来。
\par
那贾珍因见发引日近,
\zhu{
发引:出殡时,送丧人牵着引索作前导,把灵柩从停放的地方运出,叫发引。
引:牵引灵柩的索子。
}
亲自坐了车,带了阴阳司吏,往铁槛寺来踏看寄灵所在。
又一一嘱咐住持色空,\zhu{住持:主持寺庙事务的和尚,取常住护持的意思。
后道教亦袭用之。
}好生预备新鲜陈设,多请名僧,以备接灵使用。
色空忙看晚斋。
贾珍也无心茶饭,因天晚不得进城,就在净空处\foot{净空处,己、庚等本或作“净室”,\zhu{净室:一名“净住舍”,意为清静安住之所。
后用以称和尚的住室。
}意思是一样的。
净空当然不是指和尚的法号。
}胡乱歇了一夜。
次日早,便进城料理出殡之事,一面又派人先往铁槛寺,连夜另外修饰停灵之处,并厨茶等项接灵人口。
\par
里面凤姐见日期在限\foot{在限,列、舒本作“在即”,当系后改。
},也预先逐细分派料理,一面又派荣府中车轿人从跟王夫人送殡,又顾自己送殡去占下处。
目今正值缮国公诰命亡故,王、邢二夫人又去打祭送殡;西安郡王妃华诞送寿礼;\zhu{华诞:旧时对别人生日的敬称。
}镇国公诰命生了长男预备贺礼;又有胞兄王仁连家眷回南,一面写家信禀叩父母并带往之物;又有迎春染疾,每日请医服药,看医生启帖、\zhu{启帖:陈述事情的帖子。
}症源、药案\foot{“看医生启帖、症源、药案”九字,除杨本外,诸本均存,疑为草稿误衍。
杨本删去是对的。
}等事,亦难尽述。
又兼发引在迩,因此忙的凤姐茶饭也没工夫吃得,坐卧不能清净。
\geng{总得好。
}刚到了荣府,宁府的人又跟到荣府;既回到宁府,荣府的人又找到宁府。
\ping{过度劳累,埋下病根。
}凤姐见如此,心中倒十分欢喜,并不偷安推托,恐落人褒贬,因此日夜不暇,筹画得十分的整肃。
于是合族上下无不称赞者。
\par
这日伴宿之夕,\zhu{伴宿:丧家在出殡的前一夜全家整宿守灵不睡,叫“伴宿”,又称“坐夜”。
}里面两班小戏并耍百戏的,\zhu{百戏:古代乐舞杂技表演的总称,这里专指“杂技”。
}与亲朋、堂客伴宿,尤氏犹卧于内寝,一应张罗款待,都是凤姐一人周全承应。
合族中虽有许多妯娌,但或有羞口的,或有羞脚的,或有不惯见人的,或有惧贵怯官的,种种之类,都不及凤姐举止舒徐,言语慷慨,珍贵宽大;因此也不把众人放在眼内,挥霍指示,任其所为,目若无人。
\jia{写秦氏之丧,却只为凤姐一人。
}一夜中灯明火彩,客送官迎,那百般热闹自不用说的。
至天明,吉时已到,一班六十四名青衣请灵,\zhu{青衣:即皂服,黑色衣着,旧时地位低下的人所穿,后作为贱役人等的代称,如称婢女、吹鼓手和衙役等,这里指抬灵柩的舁夫(舁:音“鱼”,抬举、扛抬。
舁夫:轿夫;抬棺者。
)请灵:扛抬灵柩。
}前面铭旌上大书“奉天洪建兆年不易之朝\geng{“兆年不易之朝,永治太平之国”,奇甚妙甚!
\zhu{
评语中的“永治太平之国”在本回没有出现,应该是指第十三回僧道对坛榜文:
“世袭宁国公冢孙妇、防护内廷御前侍卫龙禁尉贾门秦氏恭人之丧。四大部州至中之地,奉天承运太平之国……”。
}
}诰封一等宁国公冢孙妇、\zhu{铭旌:同“明旌”,也叫“旌铭”,又简称“铭”。
旧时丧仪用具,绛帛粉书,上写死者官衔、姓名,用竹竿挑起,竖在灵前右方。
“奉天”隐射明太祖朱元璋独创的敬语“奉天承运皇帝,诏曰”。
“洪建”是太祖朱元璋“洪武”年号和惠帝朱允炆“建文”年号的缩略。
一说,“洪建”指洪武皇帝朱元璋建立。
“奉天洪建兆年不易之朝”在1792年出版的程甲本中删除,可能是因为出版者发现了这些字的言外之意,为了免祸而删去。
}防护内廷紫禁道御前侍卫龙禁尉、享强寿贾门秦氏恭人之灵柩”。
\zhu{享强寿:强寿是指国家强盛永久。
书中秦氏的铭旌上写“享强寿”,转意为寿命终于强健之年,意同“强死”。
《左传》文公十年载:楚国一巫者对楚成王和子玉、子西说:“三君皆将强死。
”唐代孔颖达疏:“强,健也,无病而死,谓被杀也。
”后晋楚城濮之战,子玉、子西战败,子玉自杀,子西“缢而悬绝”。
据此,强死或暗含讥贬,即指被杀或自缢之类。
}一应执事陈设,皆系现赶着新做出来的,一色光艳夺目。
宝珠自行未嫁女之礼外,摔丧驾灵,
\zhu{摔丧驾灵:旧日出殡,将起动棺材时,先由主丧孝子在灵前摔碎瓦盆一只,叫做“摔丧”,也称“摔盆”。
主丧孝子亲自抬扶灵柩或牵引灵车叫做“驾灵”。
后来,主丧孝子只在灵柩前领路,也称“驾灵”。
}
十分哀苦。
\par
那时,官客送殡的,有镇国公牛清之孙现袭一等伯牛继宗、
\zhu{
从魏晋时代开始,中国的世袭制度被进一步区分为世袭罔替和普通世袭。
世袭罔替即世袭次数无限、而且承袭者承袭被承袭者的原有爵位。
普通世袭是世袭次数有限、而且每承袭一次,承袭者只能承袭较被承袭者的原有爵位低一级的爵位。
到了宋代,世袭罔替基本被取消,更出现了不能被继承的终身爵。
}
理国公柳彪之孙现袭一等子柳芳、齐国公陈翼之孙世袭三品威镇将军陈瑞文、治国公马魁之孙世袭三品威远将军马尚、修国公侯晓明之孙世袭一等子侯孝康;缮国公诰命亡故,其孙石光珠守孝不曾来得。
\geng{牛,丑也。
清,属水,子也。
柳拆卯字。
彪拆虎字,寅字寓焉。
陈即辰。
翼火为蛇;
\zhu{
《本草·蛇》:“蛇在禽为翼火,天文星象居南方,在卦为巽风”。
}
巳字寓焉。
马,午也。
魁拆鬼,鬼,金羊,
\zhu{鬼宿,鬼金羊,二十八宿之一,南方七宿第二宿。}
未字寓焉。
侯、猴同音,申也。
晓鸣,鸡也,酉字寓焉。
石即豕,亥字寓焉。
其祖曰守业,即守夜也,犬字寓焉。
此所谓十二支寓焉。
\zhu{
天干与地支,合称干支。
天干有十个,分别是甲、乙、丙、丁、戊、己、庚、辛、壬、癸,又称十天干。
另外,地支有十二个,分别是子、丑、寅、卯、辰、巳、午、未、申、酉、戌、亥,人称十二地支。
例如我们计算年,一个天干搭配一个地支,这样算一年。从甲子开始,接着乙丑、丙寅、丁卯……,最后又回到甲子,
根据排列组合计算,需经过六十年,这也就是俗称的「一甲子六十年」。
天干与地支都可搭配阴阳与五行。“清,属水,子也”说的是五行中的“水”对应地支中的“子”。
}
}这六家与荣宁二家,当日所称“八公”的便是。
馀者更有南安郡王之孙、西宁郡王之孙、忠靖侯史鼎、平原侯之孙世袭二等男蒋子宁、定城侯之孙世袭二等男兼京营游击谢鲸、\zhu{游击:官名,游击将军的简称。
从汉代起,历代多有设置,但官阶及职权各不相同。
}襄阳侯之孙世袭二等男戚建辉、景田侯之孙五城兵马司裘良。
\zhu{五城兵马司:明、清时代,在京都设五城兵马司,掌管中、东、西、南、北五城巡缉盗贼,平治街道,稽查囚犯及防火等事。
}馀者锦乡伯公子韩奇、神威将军公子冯紫英、陈也俊、卫若兰等诸王孙公子,不可枚数。
堂客算来亦有十来顶大轿,三四十顶小轿,连家下大小轿车辆,不下百十馀乘。
连前面各色执事、陈设、百耍,浩浩荡荡,一带摆三四里远。
\par
走不多时,路旁彩棚高搭,\zhu{彩棚:这里指丧棚,即为举行丧祭而搭盖的棚。
}设席张筵,和音奏乐,俱是各家路祭。
\zhu{路祭:旧时出殡时,亲友在灵柩经过的路上设供致祭,叫“路祭”。
}
第一座是东平王府祭棚,第二座是南安郡王祭棚,第三座是西宁郡王祭棚,第四座是北静郡王祭棚。
原来这四王当日惟北静王功高,及今子孙犹袭王爵。
现今北静王水溶年未弱冠,\zhu{弱冠:古时男子二十岁行加冠礼,表示已经成人,但还未到壮年,故称“弱冠”。
}生得形容秀美,情性谦和。
近闻宁国府冢孙妇告殂,\zhu{殂:音“粗”二声,死亡。
}因想当日彼此祖父相遇之情,同难同荣,未以异姓相视,因此不以王位自居,上日也曾探丧上祭,如今又设路奠,命麾下各官在此伺候。
\zhu{麾下:犹言部下。
麾:音“挥”,古代指挥军队用的旗帜。
}自己五更入朝,公事已毕,便换了素服,坐大轿鸣锣张伞而来,至棚前落轿。
手下各官两旁拥侍,军民人众不得往还。
\par
一时,只见宁府大殡浩浩荡荡、压地银山一般从北而至。
\geng{数字道尽声势。
壬午春。
畸笏老人。
}早有宁府开路传事人看见,连忙回去报与贾珍。
贾珍急命前面驻扎,同贾赦、贾政三人连忙迎来,以国礼相见。
水溶在轿内欠身含笑答礼,仍以世交称呼接待,并不妄自尊大。
贾珍道:“犬妇之丧,累蒙郡驾下临,\zhu{累:音“类”,烦劳,劳累。
}荫生辈何以克当?”\zhu{荫生:明、清时代依靠先辈的馀荫而取得监生资格的人叫“荫生”。
荫:音“印”,受先人恩惠庇护的意思。
克:能够。
当:担当,承担。
}水溶笑道:“世交之谊,何出此言。
”遂回头命长府官主祭代奠。
\zhu{长府官:这里当指王府的长史。
始设于汉代;清代亲王、世子、郡王府中各置长史一人,统帅府属官员,总管全府事务。
}贾赦等一旁还礼毕,复身又来谢恩。
\par
水溶十分谦逊,因问贾政道:“那一位是衔玉而诞者?\geng{忙中闲笔,点缀玉兄,方不失正文中之正人。
作者良苦。
壬午春。
畸笏。
}几次要见一见,都为杂冗所阻,想今日是来的,何不请来一会?”贾政听说,忙回去,急命宝玉脱去孝服,领他前来。
那宝玉素日就曾听得父兄亲友人等说闲话时,常赞水溶是个贤王,\meng{宝玉见北静王水溶,是为后文伏线。
}且生得才貌双全,风流潇洒,每不以官俗国体所缚。
每思相会,只是父亲拘束严密,无由得会,今见反来叫他,自是欢喜。
一面走,一面早瞥见那水溶坐在轿内,好个仪表人才。
不知近看时又是怎样,下回便知。
\par
\geng{此回将大家丧事详细剔尽,如见其气概,如闻其声音,丝毫不错,作者不负大家后裔。
\hang
写秦死之盛,贾珍之奢,实是却写得一个凤姐。
}\par
\qi{总评:大抵事之不理,法之不行,多因偏于爱恶,幽柔不断。
请看凤姐无私,犹能整齐丧事。
况丈夫辈受职于庙堂之上,倘能奉公守法,一毫不苟,承上率下,何有不行?}
\dai{027}{凤姐打罚迟到奴仆}
\dai{028}{可卿出殡}
\sun{p13-3}{王熙凤协理宁国府}{图右下:宁国府总管来升得知贾珍已经委请了凤姐主事, 要众人小心伺候。
图左上:凤姐即命来升拿来家口花名册查看,限次日一早家人媳妇进府听差。
次日凤姐升座分派众人执事,制度分明,上下秩序井然。
见自已威重令行,凤姐十分得意。
}
\sun{p14-1}{秦可卿出殡,贾宝玉路谒北静王}{出殡之日,宁府大殡浩浩荡荡,压地银山般从北而至,大小轿子车辆不下百十余乘,加上各色执事陈设,接连不断足有三四里远。
走不多时,一路彩棚高搭,设席张筵, 和音奏乐,俱是各家路祭。
一时只见贾珍、贾赦、贾政三人急忙赶前去,原来北静王念及当日彼此祖父有相与之情,公事一毕,便换了素服,乘轿来祭。
小王爷见了他们三人,因问:“哪一位是衔玉而诞者,久欲一见为快,何不请来!”贾政忙传话令宝玉前来谒见。
}