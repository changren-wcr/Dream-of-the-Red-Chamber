\chapter{开夜宴异兆发悲音 \quad 赏中秋新词得佳谶}
\geng{乾隆二十一年五月初七日对清。
}\geng{缺中秋诗,俟雪芹。
\hang
〼 〼 〼 开夜宴 \quad 发悲音\hang
〼 〼 〼 赏中秋 \quad 得佳谶}\par
\qi{贾珍居长,不能承先启后,丕振家风,\zhu{丕:音“披”,大。
}兄弟问柳寻花,父子呼幺喝六,贾氏宗风,其坠地矣。
安得不发先灵一叹!}\par
话说尤氏从惜春处赌气出来,正欲往王夫人处去。
跟从的老嬷嬷们因悄悄的回道:“奶奶且别往上房去。
才有甄家的几个人来,还有些东西,不知是作什么机密事。
奶奶这一去恐不便。
”尤氏听了道:“昨日听见你爷说,看邸报甄家犯了罪,现今抄没家私,调取进京治罪。
怎么又有人来?”老嬷嬷道:“正是呢。
才来了几个女人,气色不成气色,慌慌张张的,想必有什么瞒人的事情也是有的。
”\par
尤氏听了,便不往前去,仍往李氏这边来了。
\geng{前只有探春一语,过至此回又用尤氏略为陪点,且轻轻淡染出甄家事故,此画家\sout{来}[谓]落墨之法也。
\zhu{“谓”,原作“来”,疑在传抄过程中,“谓”音讹为“未”,“未”又再形讹为“来”。}
}恰好太医才诊了脉去。
李纨近日也略觉精爽了些,
\zhu{精爽:犹言神清气爽。}
拥衾倚枕,坐在床上,正欲一二人来说些闲话。
因见尤氏进来不似往日和蔼可亲,只呆呆的坐着。
李纨因问道:“你过来了这半日,可在别屋里吃些东西没有?只怕饿了。
”命素云瞧有什么新鲜点心拣了来。
尤氏忙止道:“不必,不必。
你这一向病着,那里有什么新鲜东西。
\ping{尤氏的话,很不给李纨面子,显得病人就分不到新鲜东西似的。
}况且我也不饿。
”李纨道:“昨日他姨娘家送来的好茶面子,\zhu{茶面子:将面粉炒熟,吃时用开水冲调,可加入各种作料。
}倒是对碗来你喝罢。
”说毕,便吩咐人去对茶。
尤氏出神无语。
跟来的丫头媳妇们因问:“奶奶今日中晌尚未洗脸,这会子趁便可净一净好?”尤氏点头。
李纨忙命素云来取自己的妆奁。
素云一面取来,一面将自己的胭粉拿来,笑道:“我们奶奶就少这个。
奶奶不嫌脏,这是我的,能着用些。
”\zhu{能着:犹言“耐着”、“忍着”,引申为“将就着”。
}李纨道:“我虽没有,你就该往姑娘们那里取去。
怎么公然拿出你的来。
幸而是他,若是别人,岂不恼呢。
”尤氏笑道:“这又何妨。
自来我凡过来,\zhu{自来:从来,原来,历来。
}谁的没使过,今日忽然又嫌脏了?”一面说,一面盘膝坐在炕沿上。
银蝶上来忙代为卸去腕镯戒指,又将一大袱手巾盖在下截,将衣裳护严。
小丫鬟炒豆儿捧了一大盆温水走至尤氏跟前,只弯腰捧着。
银蝶笑道:“说一个个没机变的,说一个葫芦就是一个瓢。
奶奶不过待咱们宽些,在家里不管怎样罢了,你就得了意,不管在家出外,当着亲戚也只随着便了。
”尤氏道:“你随他去罢,横竖洗了就完事了。
”炒豆儿忙赶着跪下。
尤氏笑道:“你们家下大小的人只会讲外面假礼假体面,究竟作出来的事都够使的了。
”\geng{按尤氏犯七出之条,
\zhu{
七出:旧时七种休妻的条件。一为无子,二为淫佚,三为不事舅姑,四为口舌,五为盗窃,六为妒忌,七为恶疾。
}
不过只是“过于从夫”四字,此世间妇人之常情耳。
其心术慈厚宽顺竟可出于阿凤之上,特用此明犯七出之人从公一论,\zhu{明犯七出之人:虽然“过于从夫”并不符合七出的条件,但是根据语境这里应该是指尤氏。
}可知贾宅中暗犯七出之人亦不少。
似明犯者反可宥,恐其饰己非而扬人恶者,阴昧僻谲之流,实不能容于世者也。
}\geng{此为打草惊蛇法,实写邢夫人也。
}李纨听如此说,便知他已知道昨夜的事,因笑道:“你这话有因,谁作事究竟够使了?”尤氏道:“你倒问我!你敢是病着死过去了!”\par
一语未了,只见人报:“宝姑娘来了。
”忙说快请时,宝钗已走进来。
尤氏忙擦脸起身让坐,因问:“怎么一个人忽然走来,别的姊妹都怎么不见?”宝钗道:“正是我也没有见他们。
只因今日我们奶奶身上不自在,家里两个女人也都因时症未起炕,\zhu{时症:一时流行的传染病。
}别的靠不得,我今儿要出去伴着老人家夜里作伴儿。
要去回老太太、太太,我想又不是什么大事,且不用提,等好了我横竖进来的,所以来告诉大嫂子一声。
”李纨听说,只看着尤氏笑。
尤氏也只看着李纨笑。
\ping{抄检大观园没有抄检亲戚的,但是得到风声的宝钗,为了明哲保身,准备离开,俩人心知肚明,才相对一笑。
}一时尤氏盥沐已毕,大家吃面茶。
\zhu{面茶:一种小吃。
面粉加油炒熟,再加入开水冲或煮成糊状,吃时加糖或麻酱、椒盐等。
}
李纨因笑道:“既这样,且打发人去请姨娘的安,问是何病。
我也病着,不能亲自来的。
好妹妹,你去只管去,我自打发人去到你那里去看屋子。
你好歹住一两天还进来,别叫我落不是。
”宝钗笑道:“落什么不是呢,这也是通共常情,你又不曾卖放了贼。
\zhu{卖放:受贿私放。
}依我的主意,也不必添人过去,竟把云丫头请了来,你和他住一两日,\zhu{你和他:这里的“和”可能应该是“让”,否则语义不通,李纨住在稻香村,并不会去蘅芜苑。
}岂不省事。
”尤氏道:“可是史大妹妹往那里去了?”宝钗道:“我才打发他们找你们探丫头去了,叫他同到这里来,我也明白告诉他。
”\par
正说着,果然报:“云姑娘和三姑娘来了。
”大家让坐已毕,宝钗便说要出去一事,探春道:“很好。
不但姨妈好了还来的,就便好了不来也使得。
”尤氏笑道:“这话奇怪,怎么撵起亲戚来了?”探春冷笑道:“正是呢,有叫人撵的,不如我先撵。
亲戚们好,也不在必要死住着才好。
咱们倒是一家子亲骨肉呢,一个个不像乌眼鸡,恨不得你吃了我,我吃了你!”尤氏忙笑道:“我今儿是那里来的晦气,偏都碰着你姊妹们的气头儿上了。
”探春道:“谁叫你赶热灶来了!”\zhu{赶热灶:赶在火头上,赶在发脾气的时候。
}因问:“谁又得罪了你呢?”因又寻思道:“惜丫头不犯罗唣你,\zhu{不犯:犯不着,不值得。
罗唣:即“啰唣”,音“罗造”,骚扰,吵闹。
}却是谁呢?”尤氏只含糊答应。
探春知他畏事不肯多言,因笑道:“你别装老实了。
除了朝廷治罪,没有砍头的,你不必畏头畏尾。
实告诉你罢,我昨日把王善保家那老婆子打了,我还顶着个罪呢。
不过背地里说我些闲话,难道他还打我一顿不成!”宝钗忙问因何又打他,探春悉把昨夜怎的抄检,怎的打他,一一说了出来。
尤氏见探春已经说了出来,便把惜春方才之事也说了出来。
探春道:“这是他的僻性,孤介太过,我们再傲不过他的。
”\zhu{傲:强;拗;弯。
}又告诉他们说:“今日一早不见动静,打听凤辣子又病了。
我就打发我妈妈出去打听王善保家的是怎样。
回来告诉我说,王善保家的挨了一顿打,大太太嗔着他多事。
”尤氏李纨道:“这倒也是正理。
”探春冷笑道:“这种掩饰谁不会作,且再瞧就是了。
”尤氏李纨皆默无所答。
\ping{作为外来媳妇的李纨和尤氏,即使感受到贾府内部的问题也是不敢如探春一般直接 撕开表面和平的。
}一时估着前头用饭,湘云和宝钗回房打点衣衫,不在话下。
\par
尤氏等遂辞了李纨,往贾母这边来。
贾母歪在榻上,王夫人说甄家因何获罪,如今抄没了家产,回京治罪等语。
贾母听了正不自在,恰好见他姊妹来了,因问从那里来的?可知凤姐妯娌两个的病今日怎样?尤氏等忙回道:“今日都好些。
”贾母点头叹道:“咱们别管人家的事,且商量咱们八月十五日赏月是正经。
”\geng{贾母已看破狐悲兔死,\zhu{狐悲兔死:即“兔死狐悲”,比喻因同类的死亡而感到悲伤。
也作“狐兔之悲”、“狐死兔悲”、“狐死兔泣”。
这里的同类指甄家。
《敦煌变文集·燕子赋》:“叨闻狐死兔悲,物伤其类;四海尽为兄弟,何更同臭味!”
}故不改\sout{已}[色],聊\sout{未}[为]自遣耳。
\zhu{
聊:姑且,暂且。
自遣:发抒排遣自己的感情。
}}\ping{贾母作为家族领导显然不能任家人陷入低落的情绪,还是要说些喜庆事提振士气。
}王夫人笑道:“都已预备下了。
不知老太太拣那里好,只是园里空,夜晚风冷。
”贾母笑道:“多穿两件衣服何妨,那里正是赏月的地方,岂可倒不去的。
”\par
说话之间,早有媳妇丫鬟们抬过饭桌来,王夫人尤氏等忙上来放箸捧饭。
贾母见自己的几色菜已摆完,另有两大捧盒内捧了几色菜来,便知是各房另外孝敬的旧规矩。
贾母因问:“都是些什么?上几次我就吩咐,如今可以把这些蠲了罢,\zhu{蠲:音“捐”,减去,免除。
}你们还不听。
如今比不得在先辐辏的时光了。
”\zhu{
辐:“福”,车轮上连接轮圈和毂的直条。
毂:音“古”,车轮的中心部位。
辏:音“凑”,聚集。
辐辏:形容人物聚集像轮辐集中于毂。
“辐辏的时光”意即“兴盛的时光”之意。
}鸳鸯忙道:“我说过几次,都不听,也只罢了。
”王夫人笑道:“不过都是家常东西。
今日我吃斋没有别的。
那些面筋豆腐老太太又不大甚爱吃,只拣了一样椒油莼齑酱来。
”\zhu{莼齑(音“纯机”)酱:用莼菜捣碎腌成的小菜。
莼菜:水生,嫩叶可作菜肴。
齑:调味用的姜、蒜等碎末儿。
}贾母笑道:“这样正好,正想这个吃。
”鸳鸯听说,便将碟子挪在跟前。
宝琴一一的让了,方归坐。
贾母便命探春来同吃。
探春也都让过了,便和宝琴对面坐下。
待书忙去取了碗来。
鸳鸯又指那几样菜道:“这两样看不出是什么东西来,大老爷送来的。
这一碗是鸡髓笋,\zhu{鸡髓笋:鸡骨髓和竹笋为原料的菜。
}是外头老爷送上来的。
”\zhu{外头老爷:应该是指宁国府的贾珍。
}一面说,一面就只将这碗笋送至桌上。
贾母略尝了两点,便命:“将那两样着人送回去,就说我吃了。
以后不必天天送,我想吃自然来要。
”媳妇们答应着,仍送过去,不在话下。
\ping{贾赦送来的两样东西,鸳鸯竟然都没送到桌上,贾母让人把那两样送回去,以后也不让再送了。
第四十六回,鸳鸯女誓绝鸳鸯偶,鸳鸯拒绝了贾赦的纳妾请求,不待见贾赦,甚至贾母也不待见自己的大儿子。
}\par
贾母因问:“有稀饭吃些罢了。
”尤氏早捧过一碗来,说是红稻米粥。
贾母接来吃了半碗,便吩咐:“将这粥送给凤哥儿吃去,”又指着“这一碗笋和这一盘风腌果子狸给颦儿宝玉两个吃去,\zhu{风腌果子狸:一种名贵菜肴。
果子狸又名“花面狸”,狸之一种,体小如猫,嗜食谷类和果子等物,肉味鲜美。
风腌:经过腌制风干的。
}那一碗肉给兰小子吃去。
”又向尤氏道:“我吃了,你就来吃了罢。
”尤氏答应,待贾母漱口洗手毕,贾母便下地和王夫人说闲话行食。
\zhu{行食:饭后活动,借以帮助消化。
}尤氏告坐。
\zhu{告坐:同“告座”,上级或长辈让下级或晚辈坐,下级或晚辈谦让或道谢后坐下。
}探春宝琴二人也起来了,笑道:“失陪,失陪。
”尤氏笑道:“剩我一个人,大排桌的吃不惯。
”贾母笑道:“鸳鸯琥珀来趁势也吃些,又作了陪客。
”尤氏笑道:“好,好,好,我正要说呢。
”贾母笑道:“看着多多的人吃饭,最有趣的。
”又指银蝶道:“这孩子也好,也来同你主子一块来吃,等你们离了我,再立规矩去。
”尤氏道:“快过来,不必装假。
”
\zhu{装假:装出假象,掩盖本相;赴宴时矜持客套,不多吃。}
贾母负手看着取乐。
因见伺候添饭的人手内捧着一碗下人的米饭,尤氏吃的仍是白粳米饭,贾母问道:“你怎么昏了,盛这个饭来给你奶奶。
”那人道:“老太太的饭吃完了。
今日添了一位姑娘,所以短了些。
”鸳鸯道:“如今都是可着头做帽子了,要一点儿富馀也不能的。
”王夫人忙回道:“这一二年旱涝不定,田上的米都不能按数交的。
这几样细米更艰难了,所以都可着吃的多少关去,\zhu{关:领取。
}生恐一时短了,买的不顺口。
”贾母笑道:“这正是‘巧媳妇做不出没米的粥’来。
”众人都笑起来。
鸳鸯道:“既这样,你就去把三姑娘的饭拿来添也是一样,就这样笨。
”尤氏笑道:“我这个就够了,也不用取去。
”鸳鸯道:“你够了,我不会吃的。
”地下的媳妇们听说,方忙着取去了。
\geng{总伏下文。
\zhu{贾府财政危机为后文树倒猢狲散埋下伏笔。}
}一时王夫人也去用饭。
\par
这里尤氏直陪贾母说话取笑。
\zhu{直:副词。
径直,一直;特意。
}到起更的时候,贾母说:“黑了,过去罢。
”尤氏方告辞出来。
走至大门前上了车,银蝶坐在车沿上。
众媳妇放下帘子来,便带着小丫头们先直走过那边大门口等着去了。
因二府之门相隔没有一箭之路,每日家常来往不必定要周备,况天黑夜晚之间回来的遭数更多,所以老嬷嬷带着小丫头,只几步便走了过来。
两边大门上的人都到东西街口,早把行人断住。
尤氏大车上也不用牲口,只用七八个小厮挽环拽轮,轻轻的便推拽过这边阶矶上来。
于是众小厮退过狮子以外,众嬷嬷打起帘子,银蝶先下来,然后搀下尤氏来。
大小七八个灯笼照的十分真切。
尤氏因见两边狮子下放着四五辆大车,便知系来赴赌之人所乘,遂向银蝶众人道:“你看,坐车的是这样,骑马的还不知有几个呢。
马自然在圈里拴着,咱们看不见。
也不知道他娘老子挣下多少钱与他们,这么开心儿。
”一面说,一面已到了厅上。
贾蓉之妻带领家下媳妇丫头们,也都秉烛接了出来。
尤氏笑道:“成日家我要偷着瞧瞧他们,\zhu{家:一作“价”,语尾助词,无义。
成日家:一天到晚,终日里。
}也没得便。
今儿倒巧,就顺便打他们窗户跟前走过去。
”众媳妇答应着,提灯引路,又有一个先去悄悄的知会伏侍的小厮们不要失惊打怪。
于是尤氏一行人悄悄的来至窗下,只听里面称三赞四,耍笑之音虽多,\geng{妙!先画赢家。
}又兼有恨五骂六,忿怨之声亦不少。
\geng{妙!又画输家。
}\par
原来贾珍近因居丧,每不得游玩旷朗,又不得观优闻乐作遣。
\zhu{优:古代指以表演乐舞或杂戏为职业的人,后来泛指戏曲演员。
}无聊之极,便生了个破闷之法。
日间以习射为由,请了各世家弟兄及诸富贵亲友来较射。
因说:“白白的只管乱射,终无裨益,
\zhu{裨[bì]:增补;益处。裨益[bìyì]:好处。}
不但不能长进,而且坏了式样,必须立个罚约,赌个利物,大家才有勉力之心。
”因此在天香楼下箭道内立了鹄子,\zhu{鹄:音“谷”,箭靶。
}皆约定每日早饭后来射鹄子。
贾珍不肯出名,便命贾蓉作局家。
\zhu{局家:某些牌戏或赌博中每一局的主持人。}
这些来的皆系世袭公子,人人家道丰富,且都在少年,正是斗鸡走狗,\zhu{斗鸡走狗:使鸡相斗,使狗赛跑。形容游手好闲,终日嬉戏,不务正业。}问柳评花的一干游荡纨绔。
\zhu{问柳评花:比喻狎妓。}
因此大家议定,每日轮流作晚饭之主,——每日来射,不便独扰贾蓉一人之意。
于是天天宰猪割羊,屠鹅戮鸭,好似临潼斗宝一般,\zhu{临潼斗宝:其事不见史载,春秋时秦穆公设谋邀请十七国诸侯至临潼赴会,各出传国之宝比斗。
在这里是取其夸富斗奢、争强赌胜之意。
}都要卖弄自己家的好厨役好烹炮。
\zhu{炮:音“袍”,烧烤。
}不到半月工夫,贾赦贾政听见这般,不知就里,反说这才是正理,文既误矣,武事当亦该习,况在武荫之属。
\zhu{武荫:因武功而得到封荫。
}两处遂也命贾环、贾琮、宝玉、贾兰等四人于饭后过来,跟着贾珍习射一回,方许回去。
\par
贾珍志不在此,再过一二日便渐次以歇臂养力为由,晚间或抹抹骨牌,赌个酒东而已,\zhu{东:东道主的略称,主人。
在社交或商业活动中接待客人或顾客的人。
古时主位在东,宾位在西,所以主人称东。
}至后渐次至钱。
如今三四月的光景,竟一日一日赌胜于射了,公然斗叶掷骰,\zhu{斗叶:斗纸牌,也称“叶子戏”,赌博的一种。
}放头开局,\zhu{放头:这里是聚赌、作头家的意思。
抽头:向赢钱的赌徒抽取一部分的利益给提供赌博场所的人。
也称为“拈头”。
头家:聚赌抽头的人。
聚赌抽头所得的钱叫头儿钱。
}夜赌起来。
家下人借此各有些进益,巴不得的如此,所以竟成了势了。
外人皆不知一字。
近日邢夫人之胞弟邢德全也酷好如此,故也在其中。
又有薛蟠,头一个惯喜送钱与人的,见此岂不快乐。
邢德全虽系邢夫人之胞弟,却居心行事大不相同。
这个邢德全只知吃酒赌钱,眠花宿柳为乐,手中滥漫使钱,待人无二心,好酒者喜之,不饮者则不去亲近,无论上下主仆皆出自一意,并无贵贱之分,因此都唤他“傻大舅”。
薛蟠是早已出名的呆大爷。
今日二人皆凑在一处,都爱“抢新快”爽利,\zhu{抢新快:六个骰子,按一定的点色组合,定出分数,进行比赛,分多者胜。
}便又会了两家,在外间炕上“抢新快”。
别的又有几家在当地下大桌上打公番。
\zhu{打公番:赌博的一种,按照一定的标准比赛色子的点数。
}里间又一起斯文些的,抹骨牌打天九。
\zhu{骨牌:又名“牙牌”或“牌九”。
一种用兽骨或竹、木、象牙等制的娱乐品,也用作赌具。
牌作长方形,一面雕圆点,其数多寡不一,又分红绿二色,象征天地及星辰布列之形。
后世之打麻将由此演变而来。
打天九:一种用骨牌赌博的游戏。以天牌(十点)与九点合配成十九点为最尊,因此叫“打天九”。
}此间伏侍的小厮都是十五岁以下的孩子,若成丁的男子到不了这里,故尤氏方潜至窗外偷看。
其中有两个十六七岁娈童以备奉酒的,\zhu{娈童:被当作女性玩弄的美男。
}都打扮的粉妆玉琢。
\par
今日薛蟠又输了一张,正没好气,幸而掷第二张完了,算来除翻过来倒反赢了,心中只是兴头起来。
贾珍道:“且打住,吃了东西再来。
”因问那两处怎样。
里头打天九的,也作了帐等吃饭。
打公番的未清,且不肯吃。
于是各不能顾,先摆下一大桌,贾珍陪着吃,命贾蓉落后陪那一起。
薛蟠兴头了,便搂着一个娈童吃酒,又命将酒去敬邢傻舅。
傻舅输家,没心绪,吃了两碗,便有些醉意,嗔着两个娈童只赶着赢家不理输家了,因骂道:“你们这起兔子,\zhu{兔子:娈童、男妓。
传说月中有兔,月为阴之精;或谓兔子雌雄同体,望月而孕。
因由“兔子”联想而及雌化男性,即不男不女、亦男亦女的变态的性格和体态特征。
}就是这样专洑上水。
\zhu{洑上水:洑:音“父”,游泳。
洑上水:游向上游,比喻巴结有权势的人。
}天天在一处,谁的恩你们不沾,只不过我这一会子输了几两银子,你们就三六九等了。
难道从此以后再没有求着我们的事了!”众人见他带酒,忙说:“很是,很是。
果然他们风俗不好。
”因喝命:“快敬酒赔罪。
”两个娈童都是演就的局套,\zhu{局套:糊弄、蒙混人的一套做法。
}忙都跪下奉酒,说:“我们这行人,师父教的不论远近厚薄,只看一时有钱有势就亲敬,便是活佛神仙,一时没了钱势了,也不许去理他。
况且我们又年轻,又居这个行次,求舅太爷体恕些我们就过去了。
”
\zhu{体恕:体谅宽恕。}
\geng{调侃,骂死世人不是骂。
}  \geng{此一段娈童语句太真,反不得其为钱为势之神,当改作委曲认罪语方妥。
}说着,便举着酒俯膝跪下。
邢大舅心内虽软了,只还故作怒意不理。
众人又劝道:“这孩子是实情话。
老舅是久惯怜香惜玉的,如何今日反这样起来?若不吃这酒,他两个怎样起来。
”邢大舅已撑不住了,便说道:“若不是众位说,我再不理。
”说着,方接过来一气喝干了。
又斟一碗来。
\par
这邢大舅便酒勾往事,醉露真情起来,乃拍案对贾珍叹道:“怨不的他们视钱如命。
多少世宦大家出身的,若提起‘钱势’二字,连骨肉都不认了。
老贤甥,昨日我和你那边的令伯母赌气,你可知道否?”贾珍道:“不曾听见。
”邢大舅叹道:“就为钱这件混帐东西。
利害,\zhu{利害:厉害。
}利害!”贾珍深知他与邢夫人不睦,每遭邢夫人弃恶,扳出怨言,\zhu{扳:音“搬”,拉,引。
}因劝道:“老舅,你也太散漫些。
若只管花去,有多少给老舅花的。
”邢大舅道:“老贤甥,你不知我邢家底里。
我母亲去世时我尚小,世事不知。
他姊妹三个人,只有你令伯母年长出阁,一分家私都是他把持带来。
如今二家姐虽也出阁,他家也甚艰窘,三家姐尚在家里,一应用度都是这里陪房王善保家的掌管。
我便来要钱,也非要的是你贾府的,我邢家家私也就够我花了。
无奈竟不得到手,所以有冤无处诉。
”\geng{“众恶之,必察也。
”\zhu{众恶之,必察也:《论语·卫灵公第十五》:子曰:「众恶之,必察焉;众好之,必察焉。
」意思是,众人都厌恶他,一定对他加以考察,众人都喜欢他,也一定对他加以考察,主旨是对于舆论必须分析考察,不可简单地从众。
}今邢夫人一人,贾母先恶之,恐贾母心偏,亦可解之。
若贾琏阿凤之怨,恐儿女之私,亦可解之。
若探春之怒,恐女子不识大而知小,亦可解之。
今又忽用乃弟一怨,吾不知将又何如矣。
}贾珍见他酒后叨叨,恐人听见不雅,连忙用话解劝。
\par
外面尤氏听得十分真切,乃悄向银蝶笑道:“你听见了?这是北院里大太太的兄弟抱怨他呢。
可怜他亲兄弟还是这样说,这就怨不得这些人了。
”因还要听时,正值打公番者也歇住了,要吃酒。
因有一个问道:“方才是谁得罪了老舅,我们竟不曾听明白,且告诉我们评评理。
”邢德全见问,便把两个娈童不理输的只赶赢的话说了一遍。
这一个年少的纨绔道:“这样说,原可恼的,怨不得舅太爷生气。
我且问你两个:舅太爷虽然输了,输的不过是银子钱,并没有输丢了鸡巴,怎就不理他了?”说着,众人大笑起来,连邢德全也喷了一地饭。
尤氏在外面悄悄的啐了一口,骂道:“你听听,这一起子没廉耻的小挨刀的,才丢了脑袋骨子,就胡唚嚼毛了。
\zhu{唚:音“沁”,骂人的话。
牲畜呕吐叫“唚”。
把别人说话比作牲畜呕吐,较骂人“胡说”更甚。
}再肏攮下黄汤去,\zhu{肏攮:肏:有贬义色彩,本义为“性交”,一般用于骂人或戏谑。
肏攮;粗话,指吃喝,含贬义或戏谑义。
黄汤:指酒(骂人喝酒时说)。
}还不知唚出些什么来呢。
”一面说,一面便进去卸妆安歇。
至四更时,贾珍方散,往配凤房里去了。
\par
次日起来,就有人回西瓜月饼都全了,只待分派送人。
贾珍吩咐配凤道:“你请你奶奶看着送罢,我还有别的事呢。
”配凤答应去了,回了尤氏,尤氏只得一一分派遣人送去。
一时配凤又来说:“爷问奶奶,今儿出门不出?说咱们是孝家,明儿十五过不得节,今儿晚上倒好,可以大家应个景儿,吃些瓜饼酒。
”尤氏道:“我倒不愿出门呢。
那边珠大奶奶又病了,凤丫头又睡倒了,我再不过去,越发没个人了。
况且又不得闲,应什么景儿。
”配凤道:“爷说了,今儿已辞了众人,直等十六才来呢,好歹定要请奶奶吃酒的。
”尤氏笑道:“请我,我没的还席。
”配凤笑着去了,一时又来笑道:“爷说,连晚饭也请奶奶吃,好歹早些回来,叫我跟了奶奶去呢。
”尤氏道:“这样,早饭吃什么?快些吃了,我好走。
”配凤道:“爷说早饭在外头吃,请奶奶自己吃罢。
”尤氏问道:“今日外头有谁?”配凤道:“听见说外头有两个南京新来的,倒不知是谁。
”\ping{本回前文:“才有甄家的几个人来,还有些东西,……看邸报甄家犯了罪,现今抄没家私,调取进京治罪。
……想必有什么瞒人的事情也是有的。
”}说话之间,贾蓉之妻也梳妆了来见过。
少时摆上饭来,尤氏在上,贾蓉之妻在下相陪,婆媳二人吃毕饭。
尤氏便换了衣服,仍过荣府来,至晚方回去。
\par
果然贾珍煮了一口猪,烧了一腔羊,馀者桌菜及果品之类,不可胜记,就在会芳园丛绿堂中,屏开孔雀,褥设芙蓉,带领妻子姬妾,先饭后酒,开怀赏月作乐。
将一更时分,真是风清月朗,上下如银。
贾珍因要行令,尤氏便叫配凤等四个人也都入席,下面一溜坐下,猜枚划拳,\zhu{
猜枚:又称猜拳、划拳、豁拳、拇战或酒拳,是一种酒席上的游戏,饮酒时常以此助兴,属酒令的一种。
最初是把席上的果品或棋子握在拳中,让人猜测其数目之多寡、单双或颜色,输了的要罚酒,是谓猜枚。
后来演化成一种猜手指数目的游戏,是谓豁拳。
豁字有张开义,那么字面上,豁拳就是张手出指的意思。
而“划拳”等都是豁拳的讹写。
传统豁拳的方法是:参与的两个人同时各出一手(一般为右手),靠握紧拳头或伸出不同数目的手指比出从零到五六种不同的数目,在出手的同时喊出一个口令,对应最小为〇最大为十的一个数。
若喊出的口令所代表的数目与两人所出的手指数的总和相同,即为猜中。
若两人皆猜错或皆猜中,则视为平手,继续出拳呼令,直到一方猜中一方猜错时,决出胜负。
猜错者要罚酒。
豁拳所呼的口令皆是一些以数字开头或跟数字有关的,象征友谊和讨吉利的熟语,例如“一条龙”、“哥俩好”、“五魁首”等。
}
饮了一回。
贾珍有了几分酒,益发高兴,便命取了一竿紫竹箫来,命配凤吹箫,文\sout{化}[鸳]唱曲,喉清嗓嫩,真令人魄醉魂飞。
唱罢复又行令。
那天将有三更时分,贾珍酒已八分。
大家正添衣饮茶,换盏更酌之际,忽听那边墙下有人长叹之声。
大家明明听见,都悚然疑畏起来。
\geng{余亦悚然疑畏。
}贾珍忙厉声叱吒,
\zhu{叱吒[chìzhà]:同“叱咤”。怒斥;发怒吆喝。}
问:“谁在那里?”连问几声,没有人答应。
尤氏道:“必是墙外边家里人也未可知。
”贾珍道:“胡说。
这墙四面皆无下人的房子,况且那边又紧靠着祠堂,\geng{奇绝神想,余更为之悚惧矣。
}焉得有人。
”一语未了,只听得一阵风声,竟过墙去了。
恍惚闻得祠堂内槅扇开阖之声。
\zhu{槅扇:在房屋内部作隔开用的一扇扇木板墙或纸壁,上部一般做成窗棂,糊纸或装玻璃。
也作“隔扇”。
阖:音“合”,关闭。
}只觉得风气森森,比先更觉凉飒起来,月色惨淡,也不似先明朗。
众人都觉毛发倒竖。
贾珍酒已醒了一半,只比别人撑持得住些,心下也十分疑畏,便大没兴头起来。
勉强又坐了一会子,就归房安歇去了。
次日一早起来,乃是十五日,带领众子侄开祠堂行朔望之礼,
\zhu{
朔望之礼:朔:初一。望:十五。
过去大家望族每逢初一、十五要到祠堂照例举行祭祖的礼仪,叫做“朔望之礼”。
}
细察祠内,都仍是照旧好好的,并无怪异之迹。
贾珍自为醉后自怪,也不提此事。
礼毕,仍闭上门,看着锁禁起来。
\geng{未写荣府庆中秋,却先写宁府开夜宴,未写荣府数尽,先写宁府异兆。
盖宁乃家宅,凡有关于吉凶者,故必先示之。
且列祖祠在此,岂无得而警乎?凡人先人虽远,\zhu{凡人:指人世间的人。
在这里指的是活着的子孙,和“先人”即死去的祖先呼应。
}然气运相关,必有之理也。
非宁府之祖独有感应也。
}\par
贾珍夫妻至晚饭后方过荣府来。
只见贾赦贾政都在贾母房内坐着说闲话,与贾母取笑。
贾琏、宝玉、贾环、贾兰皆在地下侍立。
贾珍来了,都一一见过。
说了两句话后,贾母命坐,贾珍方在近门小杌子上告了坐,\zhu{告坐:同“告座”,上级或长辈让下级或晚辈坐,下级或晚辈谦让或道谢后坐下。
}
警身侧坐。
\zhu{警身侧坐:恭敬拘谨地侧身而坐。
}贾母笑问道:“这两日你宝兄弟的箭如何了?”贾珍忙起身笑道:“大长进了,不但样式好,而且弓也长了一个力气。
”\zhu{一个力气:一“力气”在这里是古代拉弓用力的单位,“一个力气”也叫“一个劲”,相当于九斤十二两。
}贾母道:“这也够了,且别贪力,仔细努伤。
”\zhu{努伤:因过分用力而受伤。
}贾珍忙答应几个“是”。
贾母又道:“你昨日送来的月饼好,西瓜看着好,打开却也罢了。
”贾珍笑道:“月饼是新来的一个专做点心的厨子,我试了试果然好,才敢做了孝敬。
西瓜往年都还可以,不知今年怎么就不好了。
”贾政道:“大约今年雨水太勤之故。
”贾母笑道:“此时月已上了,咱们且去上香。
”说着,便起身扶着宝玉的肩,带领众人齐往园中来。
\par
当下园之正门俱已大开,吊着羊角大灯。
\zhu{羊角灯:又叫明角灯,用透明角质材料为罩的灯,半透明,能防风雨。
}嘉荫堂前月台上,焚着斗香,\zhu{斗香:又叫香斗,将香束捆扎攒聚堆成塔形,点燃顶上一股,即从上到下层层燃尽。
一斗香可燃一夜。
}秉着风烛,陈献着瓜饼及各色果品。
邢夫人等一干女客皆在里面久候。
真是月明灯彩,人气香烟,晶艳氤氲,不可形状。
地下铺着拜毯锦褥。
贾母盥手上香拜毕,于是大家皆拜过。
贾母便说:“赏月在山上最好。
”因命在那山脊上的大厅上去。
众人听说,就忙着在那里去铺设。
贾母且在嘉荫堂中吃茶少歇,说些闲话。
一时,人回:“都齐备了。
”贾母方扶着人上山来。
王夫人等因说:“恐石上苔滑,还是坐竹椅上去。
”贾母道:“天天有人打扫,况且极平稳的宽路,何必不疏散疏散筋骨。
”于是贾赦贾政等在前导引,又是两个老婆子秉着两把羊角手罩,\zhu{手罩:即手照,一种手持的灯笼。
羊角手罩:有半透明角质灯罩的灯笼。
}
鸳鸯、琥珀、尤氏等贴身搀扶,邢夫人等在后围随,从下逶迤而上,不过百馀步,至山之峰脊上,便是这座敞厅。
因在山之高脊,故名曰凸碧山庄。
于厅前平台上列下桌椅,又用一架大围屏隔作两间。
凡桌椅形式皆是圆的,特取团圆之意。
上面居中贾母坐下,左垂首贾赦、\zhu{左垂首:左侧。
}贾珍、贾琏、贾蓉,右垂首贾政、宝玉、贾环、贾兰,团团围坐。
只坐了半壁,下面还有半壁馀空。
贾母笑道:“常日倒还不觉人少,今日看来,还是咱们的人也甚少,算不得甚么。
\geng{未饮先感人丁,总是将散之兆。
}想当年过的日子,到今夜男女三四十个,何等热闹。
今日就这样,太少了。
待要再叫几个来,他们都是有父母的,家里去应景,不好来的。
如今叫女孩们来坐那边罢。
”于是令人向围屏后邢夫人等席上将迎春,探春,惜春三个请出来。
贾琏宝玉等一齐出坐,先尽他姊妹坐了,然后在下方依次坐定。
\par
贾母便命折一枝桂花来,命一媳妇在屏后击鼓传花。
若花到谁手中,饮酒一杯,罚说笑话一个。
\geng{不犯前几次饮酒。
}于是先从贾母起,次贾赦,一一接过。
鼓声两转,恰恰在贾政手中住了,\geng{奇妙!偏在政老手中,竟能使政老一谑,真大文章矣。
}只得饮了酒。
众姊妹弟兄皆你悄悄的扯我一下,我暗暗的又捏你一把,都含笑倒要听是何笑话。
\geng{余也要细听。
}
贾政见贾母喜悦,只得承欢。
方欲说时,贾母又笑道:“若说的不笑了,还要罚。
”贾政笑道:“只得一个,说来不笑,也只好受罚了。
”因笑道:“一家子一个人最怕老婆的。
”才说了一句,大家都笑了。
因从不曾见贾政说过笑话,所以才笑。
\geng{是极,摹神之至。
}贾母笑道:“这必是好的。
”贾政笑道:“若好,老太太多吃一杯。
”贾母笑道:“自然。
”贾政又说道:“这个怕老婆的人从不敢多走一步。
偏是那日是八月十五,到街上买东西,便遇见了几个朋友,死活拉到家里去吃酒。
不想吃醉了,便在朋友家睡着了,第二日才醒,后悔不及,只得来家赔罪。
他老婆正洗脚,说:‘既是这样,你替我舔舔就饶你。
’这男人只得给他舔,未免恶心要吐。
他老婆便恼了,要打,说:‘你这样轻狂!’唬得他男人忙跪下求说:‘并不是奶奶的脚脏。
只因昨晚吃多了黄酒,又吃了几块月饼馅子,所以今日有些作酸呢。
’”说的贾母与众人都笑了。
\geng{这方是贾政之谑,亦善谑矣。
}贾政忙斟了一杯,送与贾母。
贾母笑道:“既这样,快叫人取烧酒来,别叫你们受累。
”\ping{笑话里的男人喝了黄酒,而现在喝的恰好是黄酒。贾母让取烧酒,意思是别拖累你们回去舔老婆脚丫子。
}众人又都笑起来。
\par
于是又击鼓,便从贾政传起,可巧传至宝玉鼓止。
宝玉因贾政在坐,自是踧踖不安,\zhu{踧踖:音“促及”,恭敬而局促不安的样子。
}花偏又在他手内,因想:“说笑话倘或不发笑,又说没口才,连一笑话不能说,何况别的,这有不是。
若说好了,又说正经的不会,只惯油嘴贫舌,更有不是。
不如不说的好。
”\geng{实写旧日往事。
}
乃起身辞道:“我不能说笑话,求再限别的罢了。
”贾政道:“既这样,限一个‘秋’字,就即景作一首诗。
若好,便赏你,若不好,明日仔细。
”贾母忙道:“好好的行令,如何又要作诗?”贾政道:“他能的。
”贾母听说,“既这样就作。
”命人取了纸笔来,贾政道:“只不许用那些冰玉晶银彩光明素等样堆砌字眼,要另出己见,试试你这几年的情思。
”宝玉听了,碰在心坎上,遂立想了四句,向纸上写了,呈与贾政看,道是……贾政看了,点头不语。
贾母见这般,知无甚大不好,便问:“怎么样?”贾政因欲贾母喜悦,便说:“难为他。
只是不肯念书,到底词句不雅。
”贾母道:“这就罢了。
他能多大,定要他做才子不成!这就该奖励他,以后越发上心了。
”贾政道:“正是。
”因回头命个老嬷嬷出去吩咐书房内的小厮,“把我海南带来的扇子取两把给他。
”宝玉忙拜谢,仍复归座行令。
当下贾兰见奖励宝玉,他便出席也做一首,递与贾政看时,写道是……\zhu{本回开头脂评:缺中秋诗,俟雪芹。
}贾政看了喜不自胜,遂并讲与贾母听时,贾母也十分欢喜,也忙令贾政赏他。
于是大家归坐,复行起令来。
\par
这次在贾赦手内住了,只得吃了酒,说笑话。
因说道:“一家子一个儿子最孝顺。
偏生母亲病了,各处求医不得,便请了一个针灸的婆子来。
婆子原不知道脉理,只说是心火,如今用针灸之法,针灸针灸就好了。
这儿子慌了,便问:‘心见铁即死,如何针得?’婆子道:‘不用针心,只针肋条就是了。
’儿子道:‘肋条离心甚远,怎么就好?’婆子道:‘不妨事。
你不知天下父母心偏的多呢。
’”众人听说,都笑起来。
贾母也只得吃半杯酒,半日笑道:“我也得这个婆子针一针就好了。
”贾赦听说,便知自己出言冒撞,贾母疑心,忙起身笑与贾母把盏,以别言解释。
贾母亦不好再提,且行起令来。
\ping{贾赦暗中讽刺贾母偏心。}
\par
不料这次花却在贾环手里。
贾环近日读书稍进,其脾味中不好务正也与宝玉一样,故每常也好看些诗词,专好奇诡仙鬼一格。
今见宝玉作诗受奖,他便技痒,只当着贾政不敢造次。
如今可巧花在手中,便也索纸笔来立挥一绝与贾政。
\geng{偏\sout{立}[写]贾政戏谑,已是异文,而贾环作诗实奇中又奇之奇文也,总在人意料之外。
竟有人曰:“贾环如何又有好诗,似前言不搭后文矣。
”盖不可向说问。
\zhu{本条批语错别字多。这句话的意思大概是,不能赞同之前提到的这种说法。
}贾环亦荣公\sout{子}[之]正脉,虽少年顽劣,见今古小儿之常情耳。
读书岂无长进之理哉?况贾政之教是弟子,自已大觉疏忽矣。
\zhu{“弟子”应为“子弟”的错讹。
}若是贾环连一平仄也不知,岂荣府是寻常膏粱不知诗书之家哉?然后知宝玉之一种情思,正非有益之聪明,不得谓比诸人皆妙者也。
}贾政看了,亦觉罕异,只是词句终带着不乐读书之意,遂不悦道:“可见是弟兄了。
发言吐气总属邪派,将来都是不由规矩准绳,一起下流货。
妙在古人中有‘二难’,\zhu{二难:《世说新语·德行》载:东汉陈寔有长子元方,少子季方,元方和季方之子各论其父功德,争持不下,便去请问祖父,陈寔说,“元方难为兄,季方难为弟”,意思是兄弟二人才德俱优,难分高下。
这里反其意而用之。
}你两个也可以称‘二难’了。
只是你两个的‘难’字,却是作难以教训之‘难’字讲才好。
哥哥是公然以温飞卿自居,
\zhu{温飞卿:温庭筠的字,唐代诗人,才思敏捷,长于词赋、音乐,作品以秾艳华丽为特色。}
如今兄弟又自为曹唐再世了。
”\zhu{
曹唐,唐代诗人,字尧宾,曾为道士,作品以游仙诗居多。
}说的贾赦等都笑了。
贾赦乃要诗瞧了一遍,连声赞好,道:“这诗据我看甚是有气骨。
想来咱们这样人家,原不比那起寒酸,定要‘雪窗萤火’,\zhu{雪窗萤火:雪窗:冬夜借窗前雪光读书。
《初学记》卷二引《宋齐语》:“孙康家贫,常映雪读书。
”萤火:夏夜借囊中萤光读书。
《晋书·车胤传》:“胤恭勤不倦,博学多通。
家贫不常得油,夏月则练囊盛数十萤火以照书,以夜继日焉。
”}一日蟾宫折桂,方得扬眉吐气。
咱们的子弟都原该读些书,不过比别人略明白些,可以做得官时就跑不了一个官的。
何必多费了工夫,反弄出书呆子来。
\ping{贾政原本想读书科举入仕,但是其父死后皇帝赐官。贾赦在这里讽刺贾政白忙一场。}
所以我爱他这诗,竟不失咱们侯门的气概。
”因回头吩咐人去取了自己的许多玩物来赏赐与他。
因又拍着贾环的头,笑道:“以后就这么做去,方是咱们的口气,将来这世袭的前程定跑不了你袭呢。
”贾政听说,忙劝说:“不过他胡诌如此,那里就论到后事了。
”\ping{贾宝玉和贾兰做完诗后,贾母令贾政赏他俩,但是当贾环做完诗之后,贾母并没有说话。
这里体现了贾母对于贾环的冷淡,原因可能是由于贾环嫉妒宝玉,曾经故意碰到油灯要烫瞎宝玉,另外赵姨娘还曾经请马道婆害死宝玉和凤姐。
荣国府实际的管家是住在荣国府正室荣禧堂的贾政夫妇,贾赦夫妇反而住在偏室。
由此可见贾赦不得母亲待见,致使自己虽然是长子,但是大权旁落。
贾赦不得母亲待见的原因,第四十六回里凤姐说道:“……老太太常说,老爷如今上了年纪,作什么左一个小老婆右一个小老婆放在屋里,没的耽误了人家。
放着身子不保养,官儿也不好生作去,成日家和小老婆喝酒……”另外在第四十八回,贾赦为了夺得石呆子的一把扇子,致使人家坑家败业。
综上,贾赦耽于酒色,不务正业,所以不得贾母待见。
贾赦求娶贾母的丫鬟鸳鸯就是贾赦企图夺回大权的试探,但是失败了。
从此更加受到贾母的冷落。
贾赦刚讲完一个讽刺贾母偏心的笑话,从同样不得宠的贾环身上看到了自己的影子,同样觉得兄弟得宠自己失宠,所以要力挺贾环。
“世袭的前程定跑不了你袭”这句话的言外之意是,身为弟弟的贾政可以打破惯例,取代贾赦掌控荣国府,那么同样身为弟弟的贾环,也可以打破惯例,取代贾宝玉的位置。
}说着便斟上酒,又行了一回令。
\geng{便又轻轻抹去也。
}贾母便说:“你们去罢。
自然外头还有相公们候着,也不可轻忽了他们。
况且二更多了,你们散了,再让我们姑娘们多乐一回,好歇着了。
”贾赦等听了,方止了令,又大家公进了一杯酒,方带着子侄们出去了。
要知端详,再听下回。
\par
\qi{总评:下回有一篇极清雅文字,下幅有半篇极整齐文字,故先叙抢快摸牌,沉湎酒色为反振,有骏马下坡、鸷鸟将翔之势。
\hang
看聚赌一段,宛然“宵小群居终日图”,看赏月一段,又宛然“望族序齿燕毛录”,\zhu{序齿:按年龄长幼定先后次序。
燕毛:祭毕燕饮时,以须毛头发黑白的颜色别长幼的坐次。《中庸》:“燕毛,所以序齿也。”
}说火则热,而说冰则寒,文心故无所不可。
}
\dai{149}{开夜宴异兆发悲音}
\dai{150}{中秋夜宴,贾赦赞赏贾环诗作}
\sun{p75-1}{开夜宴异兆发悲音}{中秋前夜,贾珍在会芳园丛绿堂中,带领妻子姬妾,先饭后酒,开怀赏月作乐。
将一更时分,贾珍命配凤吹箫,文鸳唱曲。
将三更时分,忽听那边墙下有人长叹之声,都悚然疑畏起来。
}
\sun{p75-2}{赏中秋新词得佳谶}{中秋佳节,贾府团聚。
贾母率众子孙在凸碧山庄赏月。
贾母令击鼓传花讲笑话。
}