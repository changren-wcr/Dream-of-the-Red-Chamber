\chapter{宝钗借扇机带双敲\quad 龄官划蔷痴及局外}
\zhu{机带双敲:意近“一语双关”、“一箭双雕”,即机智地用一语同时触及两方,既敲了甲,也刺了乙。
}\par
\geng{借扇敲双玉,是写宝钗金蝉脱壳。
\ping{
本回中宝钗分别对宝玉和黛玉借不落迹象的语言敲打讽刺,求得了解脱,所以说“金蝉脱壳”。
}\hang
银钗画“蔷”字,是[写]痴女梦中说梦。
\hang
脚踢袭人,是断无是理,竟有是事。
}\par
话说林黛玉与宝玉角口后,也自后悔,但又无去就他之理,因此日夜闷闷,如有所失。
紫鹃度其意,乃劝道:“若论前日之事,竟是姑娘太浮躁了些。
别人不知宝玉那脾气,难道咱们也不知道的?为那玉也不是闹了一遭两遭了。
”黛玉啐道:“你倒来替人派我的不是。
我怎么浮躁了?”紫鹃笑道:“好好的,为什么又剪了那穗子?岂不是宝玉只有三分不是,姑娘倒有七分不是。
我看他素日在姑娘身上就好,皆因姑娘小性儿,常要歪派他,\zhu{歪派:故意找茬编派别人的意思。
}才这么样。
”\par
林黛玉正欲答话,只听院外叫门。
紫鹃听了一听,笑道:“这是宝玉的声音,想必是来赔不是来了。
”林黛玉听了道:“不许开门!”紫鹃道:“姑娘又不是了。
这么热天毒日头地下,晒坏了他如何使得呢!”口里说着,便出去开门,果然是宝玉。
一面让他进来,一面笑道:“我只当是宝二爷再不上我们这门了,谁知这会子又来了。
”宝玉笑道:“你们把极小的事倒说大了。
好好的为什么不来?我便死了,魂也要一日来一百遭。
妹妹可大好了?”紫鹃道:“身上病好了,只是心里气不大好。
”宝玉笑道:“我晓得有什么气。
”一面说着,一面进来,只见林黛玉又在床上哭。
\par
那林黛玉本不曾哭,听见宝玉来,由不得伤了心,止不住滚下泪来。
宝玉笑着走近床来,道:“妹妹身上可大好了?”林黛玉只顾拭泪,并不答应。
宝玉因便挨在床沿上坐了,一面笑道:“我知道妹妹不恼我。
但只是我不来,叫旁人看着,倒像是咱们又拌了嘴的似的。
若等他们来劝咱们,那时节岂不咱们倒觉生分了?不如这会子,你要打要骂,凭着你怎么样,千万别不理我。
”说着,又把“好妹妹”叫了几万声。
林黛玉心里原是再不理宝玉的,这会子见宝玉说别叫人知道他们拌了嘴就生分了似的这一句话,又可见得比人原亲近,因又撑不住哭道:“你也不用哄我。
从今以后,我也不敢亲近二爷,二爷也全当我去了。
”宝玉听了笑道:“你往那去呢?”林黛玉道:“我回家去。
”宝玉笑道:“我跟了你去。
”林黛玉道:“我死了。
”宝玉道:“你死了,我做和尚!”\ping{暗示宝玉出家结局。
}
林黛玉一闻此言,登时将脸放下来,问道:“想是你要死了,胡说的是什么!你家倒有几个亲姐姐亲妹妹呢,明儿都死了,你几个身子去作和尚?明儿我倒把这话告诉别人去评评。
”\ping{可能在那个时代,这样的话过于露骨了,是不合礼法的。
黛玉希望感受到宝玉的心意,但宝玉热烈表白的时候她又会被吓到。
黛玉可能需要的是不用语言表达的、无时不刻不存在的、心意相通的爱和安全感。
}\par
宝玉自知这话说的造次了,后悔不来,
\zhu{后悔不来:表示事情已经发生,无法追悔或补救。}
登时脸上红胀起来,低着头不敢则一声。
幸而屋里没人。
林黛玉直瞪瞪的瞅了他半天,气的一声儿也说不出来。
见宝玉憋的脸上紫胀,便咬着牙用指头狠命的在他额颅上戳了一下,哼了一声,咬牙说道:“你这——”刚说了两个字,便又叹了一口气,仍拿起手帕子来擦眼泪。
宝玉心里原有无限的心事,又兼说错了话,正自后悔;又见黛玉戳他一下,要说又说不出来,自叹自泣,因此自己也有所感,不觉滚下泪来。
要用帕子揩拭,不想又忘了带来,便用衫袖去擦。
林黛玉虽然哭着,却一眼看见了,见他穿着簇新藕合纱衫,\zhu{藕合:也作“藕荷”,形容颜色浅紫而微微发红。
}竟去拭泪,便一面自己拭着泪,一面回身将枕边搭的一方绡帕子拿起来,向宝玉怀里一摔,一语不发,仍掩面自泣。
宝玉见他摔了帕子来,忙接住拭了泪,\chen{写尽宝、黛无限心曲,假使圣叹见之,
\zhu{圣叹:指明清之际文学评论家金圣叹。}
正不知批出多少妙处。
}
又挨近前些,伸手拉了林黛玉一只手,笑道:“我的五脏都碎了,你还只是哭。
走罢,我同你往老太太跟前去。
”林黛玉将手一摔道:“谁同你拉拉扯扯的。
一天大似一天的,还这么涎皮赖脸的,
\zhu{涎[xián]皮赖脸:形容不顾别人厌恶,厚着脸皮跟人纠缠的样子。}
连个道理也不知道。
”\par
一句没说完,只听喊道:“好了!”宝林二人不防,都唬了一跳,回头看时,只见凤姐儿跳了进来,笑道:“老太太在那里抱怨天抱怨地,只叫我来瞧瞧你们好了没有。
我说不用瞧,过不了三天,他们自己就好了。
老太太骂我,说我懒。
我来了,果然应了我的话了。
也没见你们两个人有些什么可拌的,三日好了,两日恼了,越大越成了孩子了!有这会子拉着手哭的,昨儿为什么又成了乌眼鸡呢!\zhu{乌眼鸡:乌眼鸡好斗,形容人吵架,怒目而视。
}还不跟我走,到老太太跟前,叫老人家也放些心。
”说着拉了林黛玉就走。
林黛玉回头叫丫头们,一个也没有。
凤姐道:“又叫他们作什么,有我伏侍你呢。
”一面说,一面拉了就走。
宝玉在后面跟着出了园门。
到了贾母跟前,凤姐笑道:“我说他们不用人费心,自己就会好的。
老祖宗不信,一定叫我去说合。
我及至到那里要说合,谁知两个人倒在一处对赔不是了。
对哭对诉,倒像‘黄鹰抓住了鹞子的脚’,
\zhu{鹞[yào]:猛禽。}
两个都扣了环了,\zhu{扣了环:喻亲密不可分。
鹰鹞爪子相互紧扣,不易掰开。
}那里还要人去说合。
”说的满屋里都笑起来。
\ping{“说合”一语双关,既有劝人消气和解的意思,也有牵线搭桥促成婚姻的意思。
}\par
此时宝钗正在这里。
那林黛玉只一言不发,挨着贾母坐下。
宝玉没甚说的,便向宝钗笑道:“大哥哥好日子,偏生我又不好了,没别的礼送,连个头也不得磕去。
大哥哥不知我病,倒像我懒,推故不去的。
倘或明儿恼了,姐姐替我分辨分辨。
”宝钗笑道:“这也多事。
你便要去也不敢惊动,何况身上不好,弟兄们日日一处,要存这个心倒生分了。
”宝玉又笑道:“姐姐知道体谅我就好了。
”又道:“姐姐怎么不看戏去?”宝钗道:“我怕热,看了两出,热的很。
要走,客又不散。
我少不得推身上不好,就来了。
”宝玉听说,自己由不得脸上没意思,\ping{宝钗借口身上不好不去看戏,暗中讽刺宝玉不去薛蟠生日的理由——身体有恙,其实也是借口,宝玉其实并没有生病。
}只得又搭讪笑道:“怪不得他们拿姐姐比杨妃,原来也体丰怯热。
”宝钗听说,不由的大怒,待要怎样,又不好怎样。
回思了一回,脸红起来,便冷笑了两声,说道:“我倒像杨妃,只是没一个好哥哥好兄弟可以作得杨国忠的!”\ping{宝钗进京待选,是为了成为当代的“杨贵妃”,荫蔽家族。
但是由于自己的哥哥薛蟠惹上了人命官司,损害了薛家的名声,连累自己最终落选,所以宝钗说自己没一个好哥哥的意思是感叹自己哥哥薛蟠太不争气,甚至不如历史上评价不高的杨国忠;没一个好兄弟可能是同时讽刺宝玉。
这也是宝钗的一种“机带双敲”。
杨贵妃虽是艳绝群芳,但是作为儿媳妇嫁给公公唐玄宗,在野史中和安禄山有暧昧。
沉湎享乐的唐玄宗执政期间爆发安史之乱,被认为是红颜祸水的杨贵妃被当作了替罪羊而被缢死,红颜薄命香消玉殒,并不是端庄持重的薛宝钗所认可的。
被宝玉比作杨贵妃,所以会生气。
况且宝钗进宫选秀落选,现在被比作杨妃,自然觉得是被挖苦。
}\ping{宝钗进京待选,是为了追随表姐贾元春。
宝玉对宝钗的打趣,使得宝钗与杨贵妃发生了联系。
在第十八回,元妃省亲点戏第二出《乞巧》。
《乞巧》即清初洪升《长生殿》传奇中的一出。
剧本演唐玄宗与杨贵妃的悲剧故事。
由此贾元春和杨贵妃也发生了联系。
综上,杨贵妃,贾元春和薛宝钗三人的联系很紧密。
杨贵妃之兄杨国忠,对应到贾元春之兄弟贾宝玉等贾府子弟。
这里的宝钗的话,可以看作是贾元春对于像依仗着杨贵妃而飞扬跋扈的杨国忠那样依仗着自己而胡作非为的贾府子弟的愤怒。
}二人正说着,可巧小丫头靛儿因不见了扇子,
\zhu{靛[diàn]:深蓝。}
和宝钗笑道:“必是宝姑娘藏了我的。
好姑娘,赏我罢。
”宝钗指他道:“你要仔细!我和你顽过,你再疑我。
和你素日嘻皮笑脸的那些姑娘们跟前,你该问他们去。
”\ping{宝钗迁怒于丫鬟,实际上是针对宝玉,说宝玉不要向自己发泄和黛玉的情绪,应该向“和你素日嘻皮笑脸”的黛玉去。
}说的个靛儿跑了。
宝玉自知又把话说造次了,当着许多人,更比才在林黛玉跟前更不好意思,便急回身又同别人搭讪去了。
\par
林黛玉听见宝玉奚落宝钗,心中着实得意,才要搭言也趁势儿取个笑,不想靛儿因找扇子,宝钗又发了两句话,他便改口笑道:“宝姐姐,你听了两出什么戏?”宝钗因见林黛玉面上有得意之态,一定是听了宝玉方才奚落之言,遂了他的心愿,忽又见问他这话,便笑道:“我看的是李逵骂了宋江,后来又赔不是。
”宝玉便笑道:“姐姐通今博古,色色都知道,
\zhu{色色:样样。}
怎么连这一出戏的名字也不知道,就说了这么一串子。
这叫《负荆请罪》。
\zhu{《负荆请罪》:当指元代康进之《李逵负荆》杂剧,演二歹徒假冒宋江、鲁智深之名,抢去良家之女,李逵信以为真,大闹忠义堂,后辨明真相,李逵向宋江等负荆请罪。
“负荆请罪”成语本于《史记·廉颇蔺相如列传》,演为戏曲另名《完璧记》。
}
”宝钗笑道:“原来这叫作《负荆请罪》!你们通今博古,才知道‘负荆请罪’,我不知道什么是‘负荆请罪’!”\ping{讽刺宝玉向黛玉“负荆请罪”。
}一句话还未说完,宝玉林黛玉二人心里有病,听了这话早把脸羞红了。
凤姐于这些上虽不通达,但只见他三人形景,便知其意,便也笑着问人道:“你们大暑天,谁还吃生姜呢?”众人不解其意,便说道:“没有吃生姜。
”凤姐故意用手摸着腮,诧异道:“既没人吃生姜,怎么这么辣辣的?”宝玉黛玉二人听见这话,越发不好过了。
宝钗再要说话,见宝玉十分讨愧,
\zhu{讨愧:羞愧,惭愧。}
形景改变,也就不好再说,只得一笑收住。
别人总未解得他四个人的言语,因此付之流水。
\par
一时宝钗凤姐去了,林黛玉笑向宝玉道:“你也试着比我利害的人了。
谁都像我心拙口笨的,由着人说呢。
”宝玉正因宝钗多了心,自己没趣,又见林黛玉来问着他,越发没好气起来。
待要说两句,又恐林黛玉多心,说不得忍着气,无精打采一直出来。
\par
谁知目今盛暑之时,\zhu{目今:现在,当前。
}又当早饭已过,各处主仆人等多半都因日长神倦之时,宝玉背着手,到一处,一处鸦雀无闻。
从贾母这里出来,往西走过了穿堂,便是凤姐的院落。
到他们院门前,只见院门掩着。
知道凤姐素日的规矩,每到天热,午间要歇一个时辰的,进去不便,遂进角门,来到王夫人上房内。
只见几个丫头子手里拿着针线,却打盹儿呢。
王夫人在里间凉榻上睡着,金钏儿坐在旁边捶腿,也乜斜着眼乱恍。
\zhu{乜(音“咩”)斜 :眯着眼睛,斜眼看人。
恍:摇晃不定,恍荡。
}\par
宝玉轻轻的走到跟前,把他耳上带的坠子一摘,金钏儿睁开眼,见是宝玉。
宝玉悄悄的笑道:“就困的这么着?”金钏抿嘴一笑,摆手令他出去,仍合上眼。
宝玉见了他,就有些恋恋不舍的,悄悄的探头瞧瞧王夫人合着眼,便自己向身边荷包里带的香雪润津丹掏了出来,便向金钏儿口里一送。
金钏儿并不睁眼,只管噙了。
宝玉上来便拉着手,悄悄的笑道:“我明日和太太讨你,咱们在一处罢。
”金钏儿不答。
宝玉又道:“不然,等太太醒了我就讨。
”金钏儿睁开眼,将宝玉一推,笑道:“你忙什么!‘金簪子掉在井里头,有你的只是有你的’,\ping{暗伏金钏投井而死。
}连这句话语难道也不明白?我倒告诉你个巧宗儿,你往东小院子里拿环哥儿同彩云去。
”宝玉笑道:“凭他怎么去罢,我只守着你。
”只见王夫人翻身起来,照金钏儿脸上就打了个嘴巴子,指着骂道:“下作小娼妇,好好的爷们,都叫你教坏了。
”宝玉见王夫人起来,早一溜烟去了。
\ping{王夫人装睡,其实是钓鱼执法。
}\ping{宝玉一溜烟跑了,实在称不上有担当。
}\par
这里金钏儿半边脸火热,一声不敢言语。
登时众丫头听见王夫人醒了,都忙进来。
王夫人便叫玉钏儿:“把你妈叫来,带出你姐姐去。
”金钏儿听说,忙跪下哭道:“我再不敢了。
太太要打骂,只管发落,别叫我出去就是天恩了。
我跟了太太十来年,这会子撵出去,我还见人不见人呢!”
\ping{
被赶出来的丫头,表示她做了不道德的事情,在礼教社会,只有死路一条。
}
王夫人固然是个宽仁慈厚的人,从来不曾打过丫头们一下,今忽见金钏儿行此无耻之事,此乃平生最恨者,故气忿不过,打了一下,骂了几句。
\ping{王夫人火气这么大,可能是在金钏的身上看到了赵姨娘年轻时候勾引自己丈夫贾政的影子,自己年老色衰,丈夫移情别恋,都发泄在金钏身上。
}\ping{贾珠死了,王夫人的亲儿子只有宝玉,宝玉还有点慧根,王夫人盼着宝玉学好,不要耽于享乐,所以某种程度上讲她只需要一个让宝玉向上的媳妇,并不需要一个让宝玉幸福的媳妇,这种让宝玉耽于享乐的丫鬟更是越远越好了。
}虽金钏儿苦求,亦不肯收留,到底唤了金钏儿之母白老媳妇来领了下去。
那金钏儿含羞忍辱的出去,不在话下。
\par
且说那宝玉见王夫人醒来,自己没趣,忙进大观园来。
只见赤日当空,树阴合地,满耳蝉声,静无人语。
刚到了蔷薇花架,只听有人哽噎之声。
宝玉心中疑惑,便站住细听,果然架下那边有人。
如今五月之际,那蔷薇正是花叶茂盛之际,宝玉便悄悄的隔着篱笆洞儿一看,只见一个女孩子蹲在花下,手里拿着根绾头的簪子在地下抠土,\zhu{绾:音“碗”,系。
}一面悄悄的流泪。
宝玉心中想道:“难道这也是个痴丫头,又像颦儿来葬花不成?”因又自叹道:“若真也葬花,可谓‘东施效颦’,\zhu{东施效颦:颦:音“贫”,皱眉。
相传春秋时越国美女西施因病捧心皱眉,显得更美,邻女东施,如法仿效但却更丑,引起人们的讥笑。
见《庄子·天运》。
后遂以“东施效颦”喻不自量地模仿别人,效果适得其反。
}不但不为新特,且更可厌了。
”想毕,便要叫那女子,说:“你不用跟着那林姑娘学了。
”话未出口,幸而再看时,这女孩子面生,不是个侍儿,倒像是那十二个学戏的女孩子之内的,却辨不出他是生旦净丑那一个角色来。
宝玉忙把舌头一伸,将口掩住,自己想道:“幸而不曾造次。
上两次皆因造次了,颦儿也生气,宝儿也多心,如今再得罪了他们,越发没意思了。
”\par
一面想,一面又恨认不得这个是谁。
再留神细看,只见这女孩子眉蹙春山,眼颦秋水,面薄腰纤,袅袅婷婷,大有林黛玉之态。
宝玉早又不忍弃他而去,只管痴看。
只见他虽然用金簪划地,并不是掘土埋花,竟是向土上画字。
宝玉用眼随着簪子的起落,一直一画一点一勾的看了去,数一数,十八笔。
\zhu{十八笔:“蔷”的繁体字合十八笔。
}自己又在手心里用指头按着他方才下笔的规矩写了,猜是个什么字。
写成一想,原来就是个蔷薇花的“蔷”字。
宝玉想道:“必定是他也要作诗填词。
这会子见了这花,因有所感,或者偶成了两句,一时兴至恐忘,在地下画着推敲,也未可知。
且看他底下再写什么。
”一面想,一面又看,只见那女孩子还在那里画呢,画来画去,还是个“蔷”字。
再看,还是个“蔷”字。
里面的原是早已痴了,画完一个又画一个,已经画了有几十个“蔷”。
外面的不觉也看痴了,两个眼睛珠儿只管随着簪子动,心里却想:“这女孩子一定有什么话说不出来的大心事,才这样个形景。
外面既是这个形景,心里不知怎么熬煎。
看他的模样儿这般单薄,心里那里还搁的住熬煎。
可恨我不能替你分些过来。
”\par
伏中阴晴不定,扇云可致雨,忽一阵凉风过了,唰唰的落下一阵雨来。
宝玉看着那女子头上滴下水来,纱衣裳登时湿了。
宝玉想道:“这时下雨。
他这个身子,如何禁得骤雨一激!”因此禁不住便说道:“不用写了。
你看下大雨,身上都湿了。
”那女孩子听说倒唬了一跳,抬头一看,只见花外一个人叫他不要写了,下大雨了。
一则宝玉脸面俊秀;二则花叶繁茂,上下俱被枝叶隐住,刚露着半边脸,那女孩子只当是个丫头,再不想是宝玉,因笑道:“多谢姐姐提醒了我。
难道姐姐在外头有什么遮雨的?”一句提醒了宝玉,“嗳哟”了一声,才觉得浑身冰凉。
低头一看,自己身上也都湿了。
说声“不好”,只得一气跑回怡红院去了,心里却还记挂着那女孩子没处避雨。
\zhu{宝玉“情不情”,总是忘情于他人,不顾自己。}
\par
原来明日是端阳节,\zhu{端阳节:端午节。
}那文官等十二个女子都放了学,进园来各处顽耍。
可巧小生宝官、正旦玉官两个女孩子,
\zhu{小生:传统戏曲中生角的一种,主要扮演青年男子。正旦:青衣,戏曲中旦角的一种,扮演举止端庄的中青年女子。大都穿黑色衣衫,故称。}
正在怡红院和袭人顽笑,被大雨阻住。
大家把沟堵了,水积在院内,把些绿头鸭、花鸂鶒、\zhu{鸂鶒:音“溪赤”,水鸟,似鸳鸯。
}彩鸳鸯,捉的捉,赶的赶,缝了翅膀,放在院内顽耍,将院门关了。
袭人等都在游廊上嘻笑。
\par
宝玉见关着门,便以手扣门,里面诸人只顾笑,那里听见。
叫了半日,拍的门山响,里面方听见了,估谅着宝玉这会子再不回来的。
\zhu{估谅:即“估量”,估计。
}袭人笑道:“谁这会子叫门,没人开去。
”宝玉道:“是我。
”麝月道:“是宝姑娘的声音。
”晴雯道:“胡说!宝姑娘这会子做什么来。
”袭人道:“让我隔着门缝儿瞧瞧,可开就开,要不可开,叫他淋着去。
”说着,便顺着游廊到门前,往外一瞧,只见宝玉淋的雨打鸡一般。
袭人见了又是着忙又是可笑,忙开了门,笑的弯着腰拍手道:“这么大雨地里跑什么?那里知道爷回来了。
”宝玉一肚子没好气,满心里要把开门的踢几脚,及开了门,并不看真是谁,还只当是那些小丫头子们,便抬腿踢在肋上。
袭人“嗳哟”了一声。
宝玉还骂道:“下流东西们!我素日担待你们得了意,一点儿也不怕,越发拿我取笑儿了。
”\ping{平等待人的宝玉下线了,支配蛮横的宝玉上线了。
毕竟是贵族小少爷,他会超越阶级善待拥有纯粹美丽的人,也是一种对于自己宠物的赏玩。
当宠物让主人不满意的时候,主人会打;当美丽的丫鬟让主人不满意的时候,主人也会打。
}口里说着,一低头见是袭人哭了,方知踢错了,忙笑道:“嗳哟,是你来了!踢在那里了?”袭人从来不曾受过大话的,今儿忽见宝玉生气踢他一下,又当着许多人,又是羞,又是气,又是疼,真一时置身无地。
待要怎么样,料着宝玉未必是安心踢他,少不得忍着说道:“没有踢着。
还不换衣裳去。
”宝玉一面进房来解衣,一面笑道:“我长了这么大,今日是头一遭儿生气打人,不想就偏遇见了你!”袭人一面忍痛换衣裳,一面笑道:“我是个起头儿的人,不论事大事小事好事歹,自然也该从我起。
但只是别说打了我,明儿顺了手也打起别人来。
”宝玉道:“我才也不是安心。
”袭人道:“谁说你是安心了!素日开门关门,都是那起小丫头子们的事。
他们是憨皮惯了的,早已恨的人牙痒痒,他们也没个怕惧儿。
你当是他们,踢一下子,唬唬他们也好些。
才刚是我淘气,不叫开门的。
”\ping{根据“袭为钗副”的观点,即袭人是宝钗在怡红院的代理,这里宝玉脚踢袭人,可能也是刚才和宝钗拌嘴,对宝钗有怨气,所以发泄到了袭人身上。
另外这里也可能埋下宝玉对宝钗粗暴蛮横的伏笔。
}\par
说着,那雨已住了,宝官、玉官也早去了。
袭人只觉肋下疼的心里发闹,晚饭也不曾好生吃。
至晚间洗澡时脱了衣服,只见肋上青了碗大一块,自己倒唬了一跳,又不好声张。
一时睡下,梦中作痛,由不得“嗳哟”之声从睡中哼出。
宝玉虽说不是安心,因见袭人懒懒的,也睡不安稳。
忽夜间听得“嗳哟”,便知踢重了,自己下床悄悄的秉灯来照。
刚到床前,只见袭人嗽了两声,吐出一口痰来,“嗳哟”一声,睁开眼见了宝玉,倒唬了一跳道:“作什么?”宝玉道:“你梦里‘嗳哟’,必定踢重了。
我瞧瞧。
”袭人道:“我头上发晕,嗓子里又腥又甜,你倒照一照地下罢。
”宝玉听说,果然持灯向地下一照,只见一口鲜血在地。
宝玉慌了,只说:“了不得了!”袭人见了,也就心冷了半截。
要知端的,且听下回分解。
\par
\qi{总评:爱众不常,多情不寿;风月情怀,醉人如酒。
\zhu{
爱众不常:因情感太多而引发许多苦恼。
多情不寿:可能是指金钏儿被撵出去后不久跳井自杀之事。
风月情怀,醉人如酒:宝玉在这些情感矛盾中沉浮,正如饮美酒沉醉其中而不能自拔。
}
}
\dai{059}{宝钗借扇机带双敲}
\dai{060}{宝玉淋雨脚踢袭人}
\sun{p30-1}{痴情女情重愈斟情,贾宝玉戏语金钏}{图右侧:宝玉因张道士提亲,心中不大受用,黛玉偏偏又提起“好姻缘”,越发逆了己意,便赌气从颈上摘下玉来,狠命往地下一摔,道:“什么劳什子!我砸了你完事!”黛玉见他如此,早已哭起来,道:“何苦来!你砸那哑巴物件,有砸他的,不如来砸我!”心里一急,才吃的药,“哇”的一声吐了。
图左侧:宝玉来到王夫人处,见其睡着,遂与金钏调笑。
}