\chapter{秋爽斋偶结海棠社 \quad 蘅芜苑夜拟菊花题}
\ji{美人用别号,亦新奇花样,且韵且雅,呼去觉满口生香。
结社出自探春意,作者已伏下回“兴利除弊”之文也。
\zhu{第五十六回:敏探春兴利除宿弊 ,时宝钗小惠全大体。
}\hang
此回才放笔写诗、写词、作札,\zhu{札[zhá]:书信。
}看他诗复诗、词复词、札又札,总不相犯。
\hang
湘云,诗客也,前回写之。
其今才起社,后用不即不离闲人数语数折,仍归社中。
何巧活之笔如此?}\par
\qi{海棠名诗社,林史傲秋闺。
\zhu{“林史傲秋闺”而非“林薛傲秋闺”,说明本回诗社的冠军并非薛宝钗,而是林黛玉和史湘云。}
纵有才八斗,不如富贵儿。
\zhu{
这两句似乎是指八十回后家道没落,回忆往昔感叹本回海棠诗社风流文采、荣华富贵,寄寓一种盛衰变迁的感叹。}
}\par
这年贾政又点了学差,\zhu{学差:即“学政”,全称“提督学政”,朝廷派往各省掌管科举学校等事的官员。
}择于八月二十日起身。
是日拜过宗祠及贾母起身,宝玉诸子弟等送至洒泪亭。
\par
却说贾政出门去后,外面诸事不能多记。
单表宝玉每日在园中任意纵性的逛荡,真把光阴虚度,岁月空添。
这日正无聊之际,只见翠墨进来,手里拿着一副花笺送与他。
宝玉因道:“可是我忘了,才说要瞧瞧三妹妹去的,可好些了,你偏走来。
”翠墨道:“姑娘好了,今儿也不吃药了,不过是凉着一点儿。
”宝玉听说,便展开花笺看时,上面写道:\par
\hop
娣探谨奉\zhu{娣:音“弟”,女弟,义同“妹”。
}\par
二兄文几:\zhu{这句话的意思是,将书信恭敬地送到宝玉的书桌上。
}前夕新霁,\zhu{霁:音“记”,雨、雪停止,天气放晴。
}月色如洗,因惜清景难逢,讵忍就卧,\zhu{讵:音“句”,表示反问,相当于“难道”、“哪里”。
}时漏已三转,犹徘徊于桐槛之下,\zhu{槛:音“剑”,栏杆。
桐槛:可能是桐树下的栏杆,也可能是桐木做的栏杆。
}未防风露所欺,致获采薪之患。
\zhu{采薪之患:即“采薪之忧”,见《孟子·公孙丑下》,意思是有病不能打柴。
后用作自称有病的婉辞。
薪:柴草。
}昨蒙亲劳抚嘱,复又数遣侍儿问切,\zhu{切:音“妾”,恳切,深切。
}兼以鲜荔并真卿墨迹见赐,\zhu{真卿墨迹:唐代大书法家颜真卿(又称颜鲁公)的手迹。
}何痌瘝惠爱之深哉!\zhu{痌瘝:音“通关”,痌:痛;瘝:病。
古代帝王常用“痌瘝乃身”、“痌瘝在抱”一类的话表示其视民间疾苦犹如自身病痛。
在这里探春用以表示宝玉对自己生病的关切。
}今因伏几凭床处默之时,因思及历来古人中处名攻利敌之场,犹置一些山滴水之区,\zhu{些山滴水:供玩赏的小巧的盆景山水之类。
这里指园林泉石。
区:房屋一处叫一区,亦指房屋。
}远招近揖,投辖攀辕,\zhu{投辖攀辕:极言留客之殷切。
辖:音“霞”,穿在车轴头上使轮子不致脱落的零件,多用金属制成。
投辖:《汉书·陈遵传》记陈遵嗜酒好客,宴饮时常将客人的车辖投入井中,使客人不得离去。
辕:音“元”,车辕子,车前驾牲口的直木。
攀辕:牵挽住车辕子不让走。
《六帖》:“汉侯霸为临淮太守,被征,百姓攀辕卧辙,愿留期年。
”}务结二三同志盘桓于其中,或竖词坛,或开吟社,虽一时之偶兴,遂成千古之佳谈。
娣虽不才,窃同叨栖处于泉石之间,\zhu{窃:私下、内心之意,常用作表示个人意见的谦词。
叨:音“涛”,表示受之有愧,谦辞,类似于“忝”(忝音“舔”,有愧于,辱,谦辞,表示因辱没别人而有愧。
)}而兼慕薛林之技。
风庭月榭,惜未宴集诗人;帘杏溪桃,\zhu{帘杏:倒装,即“杏帘”,指“杏帘在望”。
第十七回,宝玉题名“杏帘在望”的地方,即为“稻香村”。
溪桃:倒装,即“桃溪”,指岸边桃花盛开的小溪。
}或可醉飞吟盏。
孰谓莲社之雄才,\zhu{莲社:东晋名僧慧远居庐山虎溪东林寺所结成的一个文社,因寺内有白莲,故称莲社。
见梁代释惠皎《高僧传》。
}
独许须眉;直以东山之雅会,\zhu{东山:在浙江会稽。
东晋时谢安曾隐居东山,常邀集友人在此邀游山水,吟诗作文。
见《晋书·谢安传》。
}让余脂粉。
若蒙棹雪而来,\zhu{棹雪而来:即乘兴而来。
《世说新语·任诞》记述王子猷冒雪“夜乘小船”访戴安道,刚到门口就回转了。
人家问他为什么,他说:“吾本乘兴而行,兴尽而返,何必见戴。
”棹:音“照”,船桨,这里作动词用,相当于“划”。
}娣则扫花以待。
\zhu{扫花以待:杜甫《客至》诗有“花径不曾缘客扫,蓬门今始为君开”的句子。
这里借用诗意表示主人待客的诚意。
}此谨奉。
\par
\hop
\chai{tanchun}{探春吟句}
宝玉看了,不觉喜的拍手笑道:“倒是三妹妹的高雅,我如今就去商议。
”\ping{诗社发起人,除了探春还能是谁呢?宝钗黛玉湘云客居,惜春迎春一向没有存在感,只有探春才自精明志自高。
}一面说,一面就走,翠墨跟在后面。
刚到了沁芳亭,只见园中后门上值日的婆子手里拿着一个字帖走来,见了宝玉,便迎上去,口内说道:“芸哥儿请安,在后门只等着,叫我送来的。
”宝玉打开看时,写道是:\par
\hop
不肖男芸恭请\par
父亲大人万福金安。
男思自蒙天恩,认于膝下,日夜思一孝顺,竟无可孝顺之处。
前因买办花草,上托大人金福,竟认得许多花儿匠,\ji{直欲喷饭,真好新鲜文字。
 }并认得许多名园。
因忽见有白海棠一种,不可多得。
故变尽方法,只弄得两盆。
大人若视男是亲男一般,\ji{皆千古未有之奇文,初读令人不解,思之则喷饭。
} 便留下赏玩。
因天气暑热,恐园中姑娘们不便,故不敢面见。
奉书恭启,并叩台安。
\zhu{台:旧时常用“台”作为对别人的敬称,如“兄台”。
}男芸跪书。
\qi{一笑。
} \chen{接连二启,字句因人而施,诚作者之妙。
}\par
\hop
宝玉看了,笑道:“独他来了,还有什么人?”婆子道:“还有两盆花儿。
”宝玉道:“你出去说,我知道了,难为他想着。
你便把花儿送到我屋里去就是了。
”一面说,一面同翠墨往秋爽斋来,只见宝钗、黛玉、迎春、惜春已都在那里了。
\ji{却因芸之一字工夫,已将诸艳请来,省却多少闲文。
不然必云如何请如何来,则必至有犯宝玉,\zhu{犯:重复。}终成重复之文矣。
}\par
众人见他进来,都笑说:“又来了一个。
”探春笑道:“我不算俗,偶然起个念头,写了几个帖儿试一试,谁知一招皆到。
”宝玉笑道:“可惜迟了,早该起个社的。
”黛玉道:“你们只管起社,可别算上我,我是不敢的。
”迎春笑道:“你不敢谁还敢呢。
”\ji{必得如此方是妙文。
若也如宝玉说兴头话,则不是黛玉矣。
}宝玉道:“这是一件正经大事,大家鼓舞起来,不要你谦我让的。
各有主意自管说出来大家平章。
\zhu{平章:品评;议论。
“平章”也是职官名,唐宋以同平章事为宰相之职,元置平章为丞相之副。
}\ji{“这是正经大事”已妙,且曰“平章”,更妙!的是宝玉口角。
}宝姐姐也出个主意,林妹妹也说个话儿。
”宝钗道:“你忙什么,人还不全呢。
”\ji{妙!宝钗自有主见,真不诬也。
}一语未了,李纨也来了,进门笑道:“雅的紧!要起诗社,我自荐我掌坛。
前儿春天我原有这个意思的。
我想了一想,我又不会作诗,瞎乱些什么,因而也忘了,就没有说得。
既是三妹妹高兴,我就帮你作兴起来。
”\zhu{作兴:举办,使之兴盛。
}\ji{看他又是一篇文字,分叙单传之法也。
}\ping{李纨可能是个被寡妇身份耽误了的热爱社交的人物,即使她会作诗,寡居之人当时怕是也不能起这个头。
}\par
黛玉道:“既然定要起诗社,咱们都是诗翁了,先把这些姐妹叔嫂的字样改了才不俗。
”\ji{看他写黛玉,真可人也。
}李纨道:“极是,何不大家起个别号,彼此称呼则雅。
\ji{未起诗社,先起别号。
}我是定了‘稻香老农’,再无人占的。
”\ji{最妙!一个花样。
}探春笑道:“我就是‘秋爽居士’罢。
”宝玉道:“‘居士’、‘主人’到底不恰,且又瘰赘。
\zhu{瘰赘:同“累赘”,繁复,多余,麻烦。
}
这里梧桐芭蕉尽有,或指梧桐芭蕉起个倒好。
”探春笑道:“有了,我最喜芭蕉,就称‘蕉下客’罢。
”众人都道别致有趣。
黛玉笑道:“你们快牵了他去,炖了脯子吃酒。
”众人不解。
黛玉笑道:“古人曾云‘蕉叶覆鹿’。
\zhu{蕉叶覆鹿:《列子·周穆王》记述郑国有个樵夫打死了一只鹿,恐人看见,急忙藏在隍(隍音“黄”,无水池)中,覆之以蕉(蕉同“樵”,木柴),那知过后忘了所藏的地方,便以为是一场梦。
后常用“蕉鹿”比喻世事变幻。
这里只是取蕉下有鹿的字面意思来打趣。
}他自称‘蕉下客’,可不是一只鹿了?快做了鹿脯来。
”众人听了都笑起来。
探春因笑道:“你别忙中使巧话来骂人,我已替你想了个极当的美号了。
”又向众人道:“当日娥皇女英洒泪在竹上成斑,故今斑竹又名湘妃竹。
\zhu{湘妃竹,又称斑竹。
产于湖南、广西,竹上有紫色斑点。
传说舜帝南巡,死于苍梧,其妃湘夫人追至,哭甚哀,以泪挥竹,故竹上斑点若泪痕。
见晋代张华《博物志》。
}如今他住的是潇湘馆,他又爱哭,将来他想林姐夫,那些竹子也是要变成斑竹的。
以后都叫他作‘潇湘妃子’就完了。
”\ping{湘妃竹掩映潇湘馆,里面住着潇湘妃子,暗示着湘妃哭舜,泪染斑竹的典故。
联系到第一回神瑛侍者(贾宝玉)日以甘露灌溉绛珠草(林黛玉),绛珠草遂得脱却草胎木质,得换人形,成为绛珠仙子。
绛珠仙子为了报答神瑛侍者的甘露之惠,追随神瑛侍者下世为人,把一生所有的眼泪还他。
这两种意象结合起来,可以知道作者以湘妃哭舜泪洒斑竹比喻黛玉还泪于宝玉。
}
大家听说,都拍手叫妙。
林黛玉低了头方不言语。
\ji{妙极趣极!所谓“夫人必自侮然后人侮之”,看因一谑便勾出一美号来,何等妙文哉!另一花样。
}\ping{第二十五回,凤姐借吃茶打趣黛玉,要让黛玉做宝玉的媳妇,黛玉便红了脸,一声儿也不言语,回过头去了。
李宫裁笑向宝钗道:“真真我们二婶子的诙谐是好的。
”林黛玉含羞笑道:“什么诙谐,不过是贫嘴贱舌讨人厌恶罢了。
”在这里黛玉没有激烈的反应,因为这里没有明说黛玉要嫁到贾府,还因为这里用到的湘妃泪洒斑竹的典故,带有悲剧色彩。
}李纨笑道:“我替薛大妹妹也早已想了个好的,也只三个字。
”惜春迎春都问是什么。
\ji{妙文!迎春惜春固不能答言,然不便置之不叙,故插他二人问。
试思近日诸豪宴集雄语伟辩之时,座上或有一二愚夫不敢接谈,然偏好问,亦真可厌之事。
}李纨道:“我是封他为‘蘅芜君’了,不知你们如何。
”探春笑道:“这个封号极好。
”宝玉道:“我呢?你们也替我想一个。
”\ji{必有是问。
}宝钗笑道:“你的号早有了,‘无事忙’三字恰当的很。
”\ji{真恰当,形容得尽。
}李纨道:“你还是你的旧号‘绛洞花王’就好。
”
\zhu{
贾宝玉绛洞花王这一别号在小说其他地方也未提到,只在第八回写宝王屋子里间门斗上,贴看”绛云轩“三个字,与“绛洞花王”的名号多少有些关系。
}
\ji{妙极!又点前文。
通部中从头至末,前文已过者恐去之冷落,使人忘怀,得便一点。
未来者恐来之突然,或先伏一线。
皆行文之妙诀也。
}宝玉笑道:“小时候干的营生,还提他作什么。
”\ji{赧言如闻,\zhu{赧:音“难”三声,(因羞愧等)脸色泛红。
}不知大时又有何营生。
\ping{联系上文,这句的“大时”可能应该是“小时”。
也可能这里是嘲一句他长大后也是什么都没干吧。
}}\ping{这个营生,结合“绛洞花王”的旧号,可能是指在绛芸轩用花制胭脂。
}探春道:“你的号多的很,又起什么。
我们爱叫你什么,你就答应着就是了。
”\ji{更妙!若只管挨次一个一个乱起,则成何文字?另一花样。
}宝钗道:“还得我送你个号罢。
有最俗的一个号,却于你最当。
天下难得的是富贵,又难得的是闲散,这两样再不能兼有,不想你兼有了,就叫你‘富贵闲人’也罢了。
”宝玉笑道:“当不起,当不起,倒是随你们混叫去罢。
”李纨道:“二姑娘四姑娘起个什么号?”迎春道:“我们又不大会诗,白起个号作什么?”\ji{假斯文、守钱虏来看这句。
}探春道:“虽如此,也起个才是。
”宝钗道:“他住的是紫菱洲,就叫他‘菱洲’;四丫头在藕香榭,就叫他‘藕榭’就完了。
”\par
李纨道:“就是这样好。
但序齿我大,\zhu{序齿:按年龄长幼定先后次序。
}你们都要依我的主意,管情说了大家合意。
我们七个人起社,我和二姑娘四姑娘都不会作诗,须得让出我们三个人去。
我们三个各分一件事。
”探春笑道:“已有了号,还只管这样称呼,不如不有了。
以后错了,也要立个罚约才好。
”李纨道:“立定了社,再定罚约。
我那里地方大,竟在我那里作社。
我虽不能作诗,这些诗人竟不厌俗客,我作个东道主人,我自然也清雅起来了。
若是要推我作社长,我一个社长自然不够,必要再请两位副社长,就请菱洲藕榭二位学究来,一位出题限韵,
\zhu{限韵:旧时作诗,限定只能在某一韵部中用韵,或在某一韵部中只能某几个字作韵脚,叫限韵。
}
一位誊录监场。
亦不可拘定了我们三个人不作,若遇见容易些的题目韵脚,我们也随便作一首。
你们四个却是要限定的。
若如此便起,若不依我,我也不敢附骥了。
”\zhu{附骥:古有“苍蝇附骥尾而致千里”的说法,见《史记·伯夷列传》司马贞索隐。
比喻依附他人而成名。
后常以“附骥”作为自谦之辞。
骥:好马,喻有才德的人。
}迎春惜春本性懒于诗词,又有薛林在前,听了这话便深合己意,二人皆说:“极是。
”探春等也知此意,见他二人悦服,也不好强,只得依了。
因笑道:“这话也罢了,只是自想好笑,好好的我起了个主意,反叫你们三个来管起我来了。
”宝玉道:“既这样,咱们就往稻香村去。
”李纨道:“都是你忙,今日不过商议了,等我再请。
”\ping{李纨扫大家的兴。
}宝钗道:“也要议定几日一会才好。
”探春道:“若只管会的多,又没趣了。
一月之中,只可两三次才好。
”宝钗点头道:“一月只要两次就够了。
拟定日期,风雨无阻。
除这两日外,倘有高兴的,他情愿加一社的,或情愿到他那里去,或附就了来,亦可使得,岂不活泼有趣。
”众人都道:“这个主意更好。
”\par
探春道:“只是原系我起的意,我须得先作个东道主人,方不负我这兴。
”李纨道:“既这样说,明日你就先开一社如何?”探春道:“明日不如今日,此刻就很好。
\ping{李纨抢过来诗社领导权后,扫大家的兴,推延诗社时间。
诗社原来的召集人探春努力挽回一点话语权,提出现在就举行头一次诗社,不能让大家白跑一趟。
}你就出题,菱洲限韵,\zhu{限韵:旧时作诗,限定只能在某一韵部中用韵,或在某一韵部中只能某几个字作韵脚,叫限韵。
}藕榭监场。
”迎春道:“依我说,也不必随一人出题限韵,竟是拈阄公道。
”李纨道:“方才我来时,看见他们抬进两盆白海棠来,倒是好花。
你们何不就咏起他来?”\ji{真正好题。
妙在未起诗社先得了题目。
}迎春道:“都还未赏,先倒作诗。
”宝钗道:“不过是白海棠,又何必定要见了才作。
古人的诗赋,也不过都是寄兴写情耳。
若都是等见了作,如今也没这些诗了。
”\ji{真诗人语。
}迎春道:“既如此,待我限韵。
”说着,走到书架前抽出一本诗来,随手一揭,这首竟是一首七言律,递与众人看了,都该作七言律。
迎春掩了诗,又向一个小丫头道:“你随口说一个字来。
”那丫头正倚门立着,便说了个“门”字。
迎春笑道:“就是门字韵,‘十三元’了。
头一个韵定要这‘门’字。
”说着,又要了韵牌匣子过来,抽出“十三元” 一屉, \zhu{近体诗所用的诗韵,共分一〇六韵部。
各部都以该韵部的第一个字作为此韵部的名称。
“十三元”即上平声中以“元”字起首的第十三韵部的简称,有“元”、“原”、“源”、“门”、“存”等。
“门字韵”就是用“十三元”韵部中的“门”字作韵。
在现代普通话中“门”与“元”并不协韵,是由于古今或不同地区读音变化之故。
把每个字做成小牌,按韵部分屉,置于一箱匣内,叫韵牌匣子。
}又命那小丫头随手拿四块。
那丫头便拿了“盆”“魂”“痕”“昏”四块来。
宝玉道:“这‘盆’‘门’两个字不大好作呢!”\par
待书一样预备下四份纸笔,便都悄然各自思索起来。
独黛玉或抚梧桐,或看秋色,或又和丫鬟们嘲笑。
\ji{看他单写黛玉。
}迎春又令丫鬟炷了一支“梦甜香”。
原来这“梦甜香”只有三寸来长,有灯草粗细,以其易烬,故以此烬为限,如香烬未成便要罚。
\ji{好香!专能撰此新奇字样。
}
一时探春便先有了,自提笔写出,又改抹了一回,递与迎春。
因问宝钗:“蘅芜君,你可有了?”宝钗道:“有却有了,只是不好。
”宝玉背着手,在回廊上踱来踱去,因向黛玉说道:“你听,他们都有了。
”黛玉道:“你别管我。
”宝玉又见宝钗已誊写出来,因说道:“了不得!香只剩了一寸了,我才有了四句。
”又向黛玉道:“香就完了,只管蹲在那潮地下作什么?”黛玉也不理。
宝玉道:“我可顾不得你了,好歹也写出来罢。
”说着也走在案前写了。
李纨道:“我们要看诗了,若看完了还不交卷是必罚的。
”宝玉道:“稻香老农虽不善作却善看,又最公道,\ji{理岂不公。
}你就评阅优劣,我们都服的。
”众人都道:“自然。
”于是先看探春的稿上写道是:\par
\hop
咏白海棠 \quad 限门盆魂痕昏\par
\zhu{限门盆魂痕昏:第一句的结尾一定要用到“门”,二四六八句结尾一定要是“盆、魂、痕、昏”这四个字。}
\par
斜阳寒草带重门,\zhu{寒草:经霜的衰草。
带:连接。
重门:一层层院门。
}苔翠盈铺雨后盆。
\par
玉是精神难比洁,雪为肌骨易消魂。
\par
芳心一点娇无力,\zhu{芳心:指女子的情意,这里喻花蕊。
}倩影三更月有痕。
\zhu{倩(音“欠”)影:俏丽的身影。
月有痕:指白海棠在月光下映出的投影。
痕:这里指影子。
}\par
莫谓缟仙能羽化,\zhu{
缟:音“稿”,白绢。
缟仙:白衣仙女。
羽化:道家称得道成仙飞升为“羽化”。
}多情伴我咏黄昏。
\par
\hop
大家看了,称赏一回,又看宝钗的:\par
\hop
珍重芳姿昼掩门,\zhu{珍重:加意爱惜。
“珍重”句:借白海棠自喻,极写豪门闺秀端庄矜持的仪态,故脂批说:“宝钗诗全是自写身分”。
}\ji{宝钗诗全是自写身份,讽刺时事。
只以品行为先,才技为末。
纤巧流荡之词、绮靡秾艳之语,一洗皆尽。
非不能也,屑而不为也。
最恨近日小说中,一百美人诗词语气,只得一个艳稿。
}自携手瓮灌苔盆。
\par
胭脂洗出秋阶影,\zhu{秋阶之上映有洗去红粉的白海棠淡雅的姿影。
}冰雪招来露砌魂。
\zhu{露水未干的台阶招来白海棠冰雪般素洁的精魂。
}\ji{看他清洁自厉,
\zhu{自厉:慰勉警戒自己。}
终不肯作一轻浮语。
}\par
淡极始知花更艳,\zhu{“淡极”句以花自赞,寄托了作者薛宝钗不慕浮华,守拙藏愚的追求。
}\ji{好极!高情巨眼能几人哉!\zhu{巨眼:比喻善于鉴别的眼力。
}正“鸟鸣山更幽”也。
}
愁多焉得玉无痕。
\zhu{本句的“玉”指代白海棠,后文宝玉的诗里有“晓风不散愁千点,宿雨还添泪一痕。
”这里也是以雨痕喻泪痕,全句意思是,愁苦满怀的白海棠,怎么能够不留下哭泣的泪痕呢?可能借白海棠的愁苦涕泣,影射宝玉黛玉两人的愁苦涕泣,后文黛玉的诗里有“秋闺怨女拭啼痕”,宝玉的诗里有“宿雨还添泪一痕”。}\ji{看他讽刺林宝二人,省手。
\zhu{省手:这个词令人费解,可能是指这句诗一语双关,节省笔墨,顺手以海棠之雨痕喻宝玉黛玉之泪痕。}
}\par
欲偿白帝凭清洁,\zhu{白帝:古代神话传说中五天帝之一,掌管西方之神,五行属白,季节属秋,故常以白帝代指秋天。
}\ji{看他收到自己身上来,是何等身份。
}不语婷婷日又昏。
\zhu{婷婷:形容女子姿态窈窕美丽,这里指白海棠花。
}\par
\hop
李纨笑道:“到底是蘅芜君。
”说着又看宝玉的,道是:\par
\hop
秋容浅淡映重门,\zhu{秋容:指白海棠素淡的姿容。
据“五行”之说秋色属白,故借秋以喻素白。
}七节攒成雪满盆。
\zhu{七节:形容海棠枝节繁多。
攒:丛聚。
}\par
出浴太真冰作影,\zhu{太真:杨贵妃的号。
唐玄宗曾赐她沐浴华清池,又曾以海棠睡未足喻贵妃醉态。
}捧心西子玉为魂。
\zhu{“捧心西子”指西施“捧心而颦”的病态美。
}\zhu{二句均借古代美人喻白海棠。
}\par
晓风不散愁千点,\zhu{愁千点:指枝上盛开的朵朵白花,若含无限哀愁。
}\ji{这句直是自己一生心事。
}宿雨还添泪一痕。
\ji{妙在终不忘黛玉。
}\zhu{据脂批,“晓风”句宝玉借以自况。
“宿雨”句喻黛玉。
}\par
独倚画栏如有意,清砧怨笛送黄昏。
\zhu{如有意:像有所思虑。
}\ji{宝玉再细心作,只怕还有好的。
只是一心挂着黛玉,故平妥不警也。
}\zhu{清砧(砧音“针”):指清冷的捣衣声,古时妇女为远人作寒衣多于秋夜将衣捣平,故砧声多用以表达妇女秋夜捣衣怀念远人的意境。
怨笛:哀怨幽咽的笛声。
}\zhu{这两句把白海棠喻为独守空闺思念情郎的女子。
}\par
\hop
大家看了,宝玉说探春的好,李纨才要推宝钗这诗有身分,因又催黛玉。
黛玉道:“你们都有了。
”说着提笔一挥而就,掷与众人。
李纨等看他写道是:\par
\hop
半卷湘帘半掩门,\zhu{湘帘:湘妃竹做的帘子。
}\ji{且不说花,且说看花的人,起得突然别致。
}碾冰为土玉为盆。
\ji{妙极!料定他自与别人不同。
}\par
\hop
看了这句,宝玉先喝起彩来,只说“从何处想来!”又看下面道:\par
\hop
偷来梨蕊三分白,借得梅花一缕魂。
\par
\hop
众人看了也都不禁叫好,说“果然比别人又是一样心肠。
”又看下面道是:\par
\hop
月窟仙人缝缟袂,\zhu{月窟:月宫。
袂:衣袖。
缟袂:代指白绢做的衣服。
}秋闺怨女拭啼痕。
\ji{虚敲旁比,真逸才也。
且不脱落自己。
}\par
娇羞默默同谁诉,倦倚西风夜已昏。
\ji{看他终结道自己,一人是一人口气。
逸才仙品固让颦儿,温雅沉着终是宝钗。
今日之作宝玉自应居末。
}\par
\hop
众人看了,都道是这首为上。
李纨道:“若论风流别致,自是这首;若论含蓄浑厚,终让蘅稿。
”\ping{喜欢宝钗诗的含蓄浑厚,也符合李纨的志趣。
}探春道:“这评的有理,潇湘妃子当居第二。
”李纨道:“怡红公子是压尾,你服不服?”宝玉道:“我的那首原不好了,这评的最公。
”\ji{话内细思,则似有不服先评之意。
}又笑道:“只是蘅潇二首还要斟酌。
”李纨道:“原是依我评论,不与你们相干,再有多说者必罚。
”
\ping{李纨好霸道,大搞诗社一言堂。}
宝玉听说,只得罢了。
李纨道:“从此后我定于每月初二、十六这两日开社,出题限韵都要依我。
这其间你们有高兴的,你们只管另择日子补开,那怕一个月每天都开社,我只不管。
只是到了初二、十六这两日,是必往我那里去。
”宝玉道:“到底要起个社名才是。
”探春道:“俗了又不好,特新了,刁钻古怪也不好。
可巧才是海棠诗开端,就叫个海棠社罢。
虽然俗些,因真有此事,也就不碍了。
”说毕大家又商议了一回,略用些酒果,方各自散去。
也有回家的,也有往贾母王夫人处去的。
当下别人无话。
\ji{一路总不大写薛、林兴头,可见他二人并不着意于此。
}\ji{不写薛、林,正是大手笔,独他二人长于诗,必使他二人为之则板腐矣。
全是错综法。
}\par
\ping{李纨在这场赛诗会中的表现很不平常。
本回前文一:宝钗道:“你忙什么,人还不全呢。
”一语未了,李纨也来了。
前文二:李纨笑道:“我替薛大妹妹也早已想了个好的,也只三个字。
”惜春迎春都问是什么。
李纨道:“我是封他为‘蘅芜君’了。
”前文三:李纨道:“就是这样好。
但序齿我大,你们都要依我的主意。
”前文四:(探春)笑道:“这话也罢了,只是自想好笑,好好的我起了个主意,反叫你们三个来管起我来了。
”本回前文五:李纨道:“都是你忙,今日不过商议了,等我再请。
”宝钗道:“也要议定几日一会才好。
”前文六:探春道:“只是原系我起的意,我须得先作个东道主人,方不负我这兴。
”李纨道:“既这样说,明日你就先开一社如何?”探春道:“明日不如今日,此刻就很好。
”前文七:李纨道:“方才我来时,看见他们抬进两盆白海棠来,倒是好花。
你们何不就咏起他来?”迎春道:“都还未赏,先倒作诗。
”宝钗道:“不过是白海棠,又何必定要见了才作。
古人的诗赋,也不过都是寄兴写情耳。
若都是等见了作,如今也没这些诗了。
”前文八:众人看了,都道是这首为上。
李纨道:“若论风流别致,自是这首;若论含蓄浑厚,终让蘅稿。
”综合以上各处线索,李纨属于诗社的不速之客(前文一),依靠自己年龄大从诗社发起人探春手中“篡夺”了诗社领导权(前文三),引起了诗社发起人的不满(前文四),还扫大家的兴,多次推延本该今天举行的首次诗社(前文五、六),在诗社中和宝钗互动频繁(前文一、二、五、七、八)。
由此可以推测,李纨青春守寡,孤儿寡母,槁木死灰一般,并没有被探春邀请参加诗社这种孩子的娱乐活动,宝钗临时通知了李纨也来参加(前文一),所以李纨对自己不被小姑子邀请而感到怨愤,对惦记自己的宝钗而心存感激,主动帮宝钗起别号(前文二),可能提前给宝钗透题(前文七),最后在大家都觉得黛玉的诗最好的时候,力挺薛宝钗成为首届诗社的冠军(前文八)。
薛宝钗也在支持李纨,让大家等李纨来(前文一),帮助确定诗社时间(前文五),在迎春质疑李纨要求大家不见白海棠而赋诗的时候给李纨打圆场(前文七)。
}\par
且说袭人\ji{忽然写到袭人,真令人不解。
看他如何终此诗社之文。
}因见宝玉看了字贴儿便慌慌张张的同翠墨去了,也不知是何事。
后来又见后门上婆子送了两盆海棠花来。
袭人问是那里来的,婆子便将宝玉前一番缘故说了。
袭人听说便命他们摆好,让他们在下房里坐了,自己走到自己房内秤了六钱银子封好,又拿了三百钱走来,都递与那两个婆子道:“这银子赏那抬花来的小子们,这钱你们打酒吃罢。
”那婆子们站起来,眉开眼笑,千恩万谢的不肯受,见袭人执意不收,方领了。
袭人又道:“后门上外头可有该班的小子们?”婆子忙应道:“天天有四个,原预备里面差使的。
姑娘有什么差使,我们吩咐去。
”袭人笑道:“有什么差使?今儿宝二爷要打发人到小侯爷家与史大姑娘送东西去,可巧你们来了,顺便出去叫后门小子们雇辆车来。
回来你们就往这里拿钱,不用叫他们又往前头混碰去。
”婆子答应着去了。
\par
袭人回至房中,拿碟子盛东西与史湘云送去,\ji{线头却牵出,观者犹不理会。
\zhu{袭人特地给湘云送过去她喜欢的缠丝白玛瑙碟子。}
}\ji{不知是何碟何物,令人犯思度。
}却见槅子上碟槽空着。
\ji{妙极,细极!因此处系依古董式样抠成槽子,故无此件此槽遂空。
若忘却前文,此句不解。
}
因回头见晴雯、秋纹、麝月等都在一处做针黹,
\zhu{
黹:音“旨”,缝纫,刺绣。
针黹:旧时妇女针线活儿的统称。
也叫“女红(红:音“工”)。
}
袭人问道:“这一个缠丝白玛瑙碟子那去了?”
\zhu{缠丝白玛瑙:一种有丝纹的白玛瑙,比较珍贵。}
众人见问,都你看我我看你,都想不起来。
半日,晴雯笑道:“给三姑娘送荔枝去的,还没送来呢。
”袭人道:“家常送东西的家伙也多,巴巴的拿这个去。
”晴雯道:“我何尝不也这样说。
他说这个碟子配上鲜荔枝才好看。
\ji{自然好看,原该如此。
可恨今之有一二好花者,不肯像景而用。
\zhu{像:随;依顺。
}}我送去,三姑娘见了也说好看,叫连碟子放着,就没带来。
你再瞧,那槅子尽上头的一对联珠瓶还没收来呢。
”\zhu{联珠瓶:疑即为“双联瓶”,指两个等大的圆形并联(或叠联)的瓷瓶。
一说指饰有联珠纹样的瓶子,“联珠”是以圆珠联串,作为一种连续纹样的带状条饰。
}\par
秋纹笑道:“提起瓶来,我又想起笑话。
我们宝二爷说声孝心一动,也孝敬到二十分。
因那日见园里桂花,折了两枝,原是自己要插瓶的,忽然想起来说,这是自己园里的才开的新鲜花,不敢自己先顽,巴巴的把那一对瓶拿下来,亲自灌水插好了,叫个人拿着,亲自送一瓶进老太太,又进一瓶与太太。
谁知他孝心一动,连跟的人都得了福了。
可巧那日是我拿去的。
老太太见了这样,喜的无可无不可,见人就说:‘到底是宝玉孝顺我,连一枝花儿也想的到。
别人还只抱怨我疼他。
’你们知道,老太太素日不大同我说话的,有些不入他老人家的眼的。
那日竟叫人拿几百钱给我,说我可怜见的,生的单柔。
这可是再想不到的福气。
几百钱是小事,难得这个脸面。
及至到了太太那里,太太正和二奶奶、赵姨奶奶、周姨奶奶好些人翻箱子,找太太当日年轻的颜色衣裳,不知给那一个。
一见了,连衣裳也不找了,且看花儿。
又有二奶奶在旁边凑趣儿,夸宝玉又是怎么孝敬,又是怎样知好歹,有的没的说了两车话。
当着众人,太太自为又增了光,堵了众人的嘴。
太太越发喜欢了,现成的衣裳就赏了我两件。
衣裳也是小事,年年横竖也得,却不像这个彩头。
”
\zhu{彩头:好运气;也指获得的奖品、赏物。}
\par
晴雯笑道:“呸!没见世面的小蹄子!
\zhu{小蹄子:骂年轻女孩或婢女的话。}
那是把好的给了人,挑剩下的才给你,你还充有脸呢。
”秋纹道:“凭他给谁剩的,到底是太太的恩典。
”晴雯道:“要是我,我就不要。
若是给别人剩下的给我,也罢了。
一样这屋里的人,难道谁又比谁高贵些?把好的给他,剩下的才给我,我宁可不要,冲撞了太太,我也不受这口软气。
”秋纹忙问:“给这屋里谁的?我因为前儿病了几天,家去了,不知是给谁的。
好姐姐,你告诉我知道知道。
”晴雯道:“我告诉了你,难道你这会退还太太去不成?”秋纹笑道:“胡说。
我白听了喜欢喜欢。
\zhu{白:单单,只是。
}那怕给这屋里的狗剩下的,我只领太太的恩典,也不犯管别的事。
”众人听了都笑道:“骂的巧,可不是给了那西洋花点子哈巴儿了。
”\ping{晴雯话语里的“他”就指的是袭人,晴雯点火,秋纹助攻,最后众人一笑,可见王夫人给袭人姨娘待遇,引发了怡红院所有丫鬟的不满与吃醋。
}袭人笑道:“你们这起烂了嘴的!得了空就拿我取笑打牙儿。
\zhu{打牙:说闲话。}
一个个不知怎么死呢。
”秋纹笑道:“原来姐姐得了,我实在不知道。
我陪个不是罢。
”袭人笑道:“少轻狂罢。
你们谁取了碟子来是正经。
”\ji{看他忽然夹写女儿喁喁一段,\zhu{喁:音“鱼”,模拟小声说话的声音。
}总不脱落正事。
所谓此书一回是两段,两段中却有无限事体,或有一语透至下回者,或有反补上回者,错综穿插,从不一气直起直泻至终为了。
}\ping{袭人此时并不真的发火,一方面袭人脾气好,另一方面估计袭人觉得得宝玉和王夫人宠信在手,别人的嘲讽都当是无关紧要的嫉妒了。
}麝月道:“那瓶得空儿也该收来了。
老太太屋里还罢了,太太屋里人多手杂。
别人还可以,赵姨奶奶一夥的人见是这屋里的东西,\zhu{夥:同“伙”。
}又该使黑心弄坏了才罢。
太太也不大管这些,不如早些收来正经。
”晴雯听说,便掷下针黹道:“这话倒是,等我取去。
”秋纹道:“还是我取去罢,你取你的碟子去。
”晴雯笑道:“我偏取一遭儿去。
是巧宗儿你们都得了,难道不许我得一遭儿?”麝月笑道:“通共秋丫头得了一遭儿衣裳,那里今儿又巧,你也遇见找衣裳不成。
”晴雯冷笑道:“虽然碰不见衣裳,或者太太看见我勤谨,一个月也把太太的公费里分出二两银子来给我,也定不得。
”说着,又笑道:“你们别和我装神弄鬼的,什么事我不知道。
”\ping{王夫人把袭人的待遇提高到姨娘的标准,引发了晴雯的嫉妒。
}一面说,一面往外跑了。
秋纹也同他出来,自去探春那里取了碟子来。
\par
袭人打点齐备东西,叫过本处的一个老宋妈妈来,\ji{“宋”,送也。
随事生文,妙!}向他说道:“你先好生梳洗了,换了出门的衣裳来,如今打发你与史姑娘送东西去。
”那宋嬷嬷道:“姑娘只管交给我,有话说与我,我收拾了就好一顺去的。
”袭人听说,便端过两个小掐丝盒子来。
\zhu{
掐丝:一种手工艺,在器物花纹的边缘嵌上金银等线条,或用金银丝等直接嵌成花纹。
另一种解释,“嵌”或为“粘焊”。
}
先揭开一个,里面装的是红菱和鸡头两样鲜果;\zhu{鸡头:指鸡头米,芡实之俗称。
芡是一种水生植物,其果仁可食。
}又那一个是一碟子桂花糖蒸新栗粉糕。
又说道:“这都是今年咱们这里园里新结的果子,宝二爷送来与姑娘尝尝。
再前日姑娘说这玛瑙碟子好,姑娘就留下顽罢。
\ji{妙!隐这一件公案。
余想袭人必要玛瑙碟子盛去,何必娇奢轻发如是耶?\zhu{娇奢:即“骄奢”。
轻发:轻率行动。
}固有此一案,则无怪矣。
}\ping{袭人和湘云的感情真的很不错。
}这绢包儿里头是姑娘上日叫我作的活计,姑娘别嫌粗糙,能着用罢。
替我们请安,替二爷问好就是了。
”宋嬷嬷道:“宝二爷不知还有什么说的,姑娘再问问去,回来又别说忘了。
”袭人因问秋纹:“方才可见在三姑娘那里?”秋纹道:“他们都在那里商议起什么诗社呢,又都作诗。
想来没话,你只去罢。
”宋嬷嬷听了,便拿了东西出去,另外穿戴了。
袭人又嘱咐他:“从后门出去,有小子和车等着呢。
”宋妈去后,不在话下。
\par
宝玉回来,先忙着看了一回海棠,至房内告诉袭人起诗社的事。
袭人也把打发宋妈妈与史湘云送东西去的话告诉了宝玉。
宝玉听了,拍手道:“偏忘了他。
我自觉心里有件事,只是想不起来,亏你提起来,正要请他去。
这诗社里若少了他还有什么意思。
”袭人劝道:“什么要紧,不过玩意儿。
他比不得你们自在,家里又作不得主儿。
告诉他,他要来又由不得他;不来,他又牵肠挂肚的,没的叫他不受用。
”宝玉道:“不妨事,我回老太太打发人接他去。
”正说着,宋妈妈已经回来,回复道生受,\zhu{生受:这里是道谢语,难为、有劳的意思。
}与袭人道乏,又说:“问二爷作什么呢,我说和姑娘们起什么诗社作诗呢。
史姑娘说,他们作诗也不告诉他去,急的了不的。
”宝玉听了立身便往贾母处来,立逼着叫人接去。
贾母因说:“今儿天晚了,明日一早再去。
”宝玉只得罢了,回来闷闷的。
\par
次日一早,便又往贾母处来催逼人接去。
直到午后,史湘云才来,宝玉方放了心,见面时就把始末原由告诉他,又要与他诗看。
李纨等因说道:“且别给他诗看,先说与他韵。
他后来,先罚他和了诗:若好,便请入社;若不好,还要罚他一个东道再说。
”史湘云道:“你们忘了请我,我还要罚你们呢。
就拿韵来,我虽不能,只得勉强出丑。
容我入社,扫地焚香我也情愿。
”众人见他这般有趣,越发喜欢,都埋怨昨日怎么忘了他,遂忙告诉他韵。
史湘云一心兴头,等不得推敲删改,一面只管和人说着话,心内早已和成,即用随便的纸笔录出,\ji{可见越是好文字,不管怎样就有了。
越用工夫,越讲究笔墨,终成涂鸦。
}先笑说道:“我却依韵和了两首,\ji{更奇!想前四律已将形容尽矣,一首犹恐重犯,不知二首又从何处着笔。
}好歹我却不知,不过应命而已。
”说着递与众人。
众人道:“我们四首也算想绝了,再一首也不能了。
你倒弄了两首,那里有许多话说,必要重了我们。
”一面说,一面看时,只见那两首诗写道:\par
\hop
其一\par
神仙昨日降都门,\zhu{都门:即京都。
}\ji{落想便新奇,不落彼四套。
\zhu{彼四套:指代前文探春,宝钗,宝玉,黛玉做的四首诗。
}}种得蓝田玉一盆。
\ji{好!“盆”字押得更稳,不落彼四套。
}\zhu{蓝田:陕西省蓝田县,山中自古产白玉,称蓝田玉。
这里以喻白海棠。
种玉:晋代干宝《搜神记》载,雒阳人杨伯雍,居无终山,山高八十里,上无水,他担水设义浆于其上,供过路人渴饮。
三年之后,遇一仙人来饮,送他一斗石子,叫他种在山上有石处,说:“玉当生其中”,“汝后当得好妇”。
杨依言种石,后于种玉之处,挖出白璧五双,以之聘得富家徐氏女。
}\par
自是霜娥偏爱冷,\zhu{霜娥:即青女,神话中司霜雪的女神。
}\ji{又不脱自己将来形景。
}非关倩女亦离魂。
\zhu{关:关系,涉及。
倩女离魂:见唐代陈玄祐《离魂记》。
故事写张镒的女儿倩娘与表兄王宙相爱,张镒却将倩娘另许他人,王宙愤而远行,途中倩娘忽连夜追至,两人遂一同出走。
五年后,他们回家看望父母,这时房中久病的倩娘迎出,与归来的倩娘合为一体。
原来跟王宙出走的是倩娘的魂灵。
}\ping{上句可能暗示了湘云也如同霜娥那样清冷。
下句应该和第三十一回的标题“因麒麟伏白首双星”一起来看。
史湘云可能和同样有麒麟的卫若兰结为夫妻,但是因故夫妻离散,如天上的牛郎星和织女星那样天各一方。
倩女离魂的故事也是说的两个恋人在肉体上被强行分开,不得相见,为了能够在一起而魂灵出窍的故事。
湘云的这句诗更加印证了所推测的湘云的最终结局。
}\par
秋阴捧出何方雪,\zhu{秋阴:即秋云。
阴:密云。
}\ji{拍案叫绝!压倒群芳在此一句。
}雨渍添来隔宿痕。
\par
却喜诗人吟不倦,岂令寂寞度朝昏。
\ji{真好!}\par
\hop
其二\par
蘅芷阶通萝薜门,\zhu{蘅:杜蘅;芷:白芷;均为香草。
萝:松萝;薜:薜荔[bìlì];皆蔓生植物。
}也宜墙角也宜盆。
\ji{更好!}\par
花因喜洁难寻偶,人为悲秋易断魂。
\par
玉烛滴干风里泪,\zhu{玉烛:白色蜡烛。
本句以燃着的白蜡烛来比喻在秋风中摇曳的白海棠。
}晶帘隔破月中痕。
\zhu{晶帘:水晶帘。
本句是说从水晶帘内看月色中白海棠的姿影更显得朦胧模糊。
}\par
幽情欲向嫦娥诉,\zhu{幽情:深藏在内心的衷情。
}无奈虚廊夜色昏。
\zhu{虚廊:寂静的长廊。
}\ji{二首真可压卷。
诗是好诗,文是奇奇怪怪之文,总令人想不到忽有二首来压卷。
}\par
\hop
众人看一句,惊讶一句,看到了,赞到了,都说:“这个不枉作了海棠诗,真该要起海棠社了。
”史湘云道:“明日先罚我个东道,就让我先邀一社可使得?”众人道:“这更妙了。
”因又将昨日的与他评论了一回。
\par
至晚,宝钗将湘云邀往蘅芜苑安歇去。
湘云灯下计议如何设东拟题。
宝钗听他说了半日,皆不妥当,\ji{却于此刻方写宝钗。
}因向他说道:“既开社,便要作东。
虽然是顽意儿,也要瞻前顾后,又要自己便宜,
\zhu{便宜[biànyí]:适宜;便利。}
又要不得罪了人,然后方大家有趣。
你家里你又作不得主,一个月通共那几串钱,你还不够盘缠呢。
这会子又干这没要紧的事,你婶子听见了,越发抱怨你了。
况且你就都拿出来,做这个东道也是不够。
难道为这个家去要不成?还是往这里要呢?”一席话提醒了湘云,倒踌蹰起来。
宝钗道:“这个我已经有个主意。
我们当铺里有个伙计,他家田上出的很好的肥螃蟹,前儿送了几斤来。
现在这里的人,从老太太起连上园里的人,有多一半都是爱吃螃蟹的。
前日姨娘还说要请老太太在园里赏桂花吃螃蟹,因为有事还没有请呢。
你如今且把诗社别提起,只管普通一请。
等他们散了,咱们有多少诗作不得的。
我和我哥哥说,要几篓极肥极大的螃蟹来,再往铺子里取上几坛好酒,再备上四五桌果碟,岂不又省事又大家热闹了。
”湘云听了,心中自是感服,极赞他想的周到。
宝钗又笑道:“我是一片真心为你的话。
你千万别多心,想着我小看了你,咱们两个就白好了。
你若不多心,我就好叫他们办去的。
”湘云忙笑道:“好姐姐,你这样说,倒多心待我了。
凭他怎么糊涂,连个好歹也不知,还成个人了?我若不把姐姐当亲姐姐一样看,上回那些家常话烦难事也不肯尽情告诉你了。
”宝钗听说,便叫一个婆子来:“出去和大爷说,依前日的大螃蟹要几篓来,明日饭后请老太太姨娘赏桂花。
你说大爷好歹别忘了,我今儿已请下人了。
”\ji{必得如此叮咛,阿呆兄方记得。
}那婆子出去说明,回来无话。
\par
这里宝钗又向湘云道:“诗题也不要过于新巧了。
你看古人诗中那些刁钻古怪的题目和那极险的韵了,\zhu{险韵:诗韵各韵部因所属字数的多少不同,特别是容易用来押韵的字多少不同,有宽韵、窄韵和险韵之分;险韵即指最难押的韵部。
此外,虽用宽韵作诗,而偏择取其中难用的或生僻的字押韵,也叫用险韵。
好用险韵作诗的人,常常借此炫耀自己作诗的本领。
}若题过于新巧,韵过于险,再不得有好诗,终是小家气。
诗固然怕说熟话,更不可过于求生,只要头一件立意清新,自然措词就不俗了。
究竟这也算不得什么,还是纺绩针黹是你我的本等。
\zhu{本等:本分,指分内应作或应有的事。}
一时闲了,倒是于你我深有益的书看几章是正经。
”\par
湘云只答应着,因笑道:“我如今心里想着,昨日作了海棠诗,我如今要作个菊花诗如何?”宝钗道:“菊花倒也合景,只是前人太多了。
”湘云道:“我也是如此想着,恐怕落套。
”宝钗想了一想,说道:“有了,如今以菊花为宾,以人为主,竟拟出几个题目来,都是两个字:一个虚字,一个实字,实字便用‘菊’字,虚字就用通用门的。
\zhu{门:类别,门类。
}如此又是咏菊,又是赋事,前人也没作过,也不能落套。
赋景咏物两关着,\zhu{关:关系,涉及。
}又新鲜,又大方。
”湘云笑道:“这却很好。
只是不知用何等虚字才好。
你先想一个我听听。
”宝钗想了一想,笑道:“《菊梦》就好。
”湘云笑道:“果然好。
我也有一个,《菊影》可使得?”宝钗道:“也罢了。
只是也有人作过,若题目多,这个也夹的上。
我又有了一个。
”湘云道:“快说出来。
”宝钗道:“《问菊》如何?”湘云拍案叫妙,因接说道:“我也有了,《访菊》如何?”宝钗也赞有趣,因说道:“越性拟出十个来,写上再来。
”说着,二人研墨蘸笔,湘云便写,宝钗便念,一时凑了十个。
湘云看了一遍,又笑道:“十个还不成幅,越性凑成十二个便全了,也如人家的字画册页一样。
”宝钗听说,又想了两个,一共凑成十二。
又说道:“既这样,越性编出他个次序先后来。
”湘云道:“如此更妙,竟弄成个菊谱了。
”宝钗道:“起首是《忆菊》;忆之不得,故访,第二是《访菊》;访之既得,便种,第三是《种菊》;种既盛开,故相对而赏,第四是《对菊》;相对而兴有馀,故折来供瓶为玩,第五是《供菊》;既供而不吟,亦觉菊无彩色,第六便是《咏菊》;既入词章,不可不供笔墨,第七便是《画菊》;既为菊如是碌碌,\zhu{碌碌:忙碌的样子。
}究竟不知菊有何妙处,不禁有所问,第八便是《问菊》;菊如解语,使人狂喜不禁,第九便是《簪菊》;如此人事虽尽,犹有菊之可咏者,《菊影》《菊梦》二首续在第十第十一;末卷便以《残菊》总收前题之盛。
这便是三秋的妙景妙事都有了。
”\par
湘云依说将题录出,又看了一回,又问“该限何韵?”宝钗道:“我平生最不喜限韵的,分明有好诗,何苦为韵所缚。
咱们别学那小家派,只出题不拘韵。
原为大家偶得了好句取乐,并不为此而难人。
”湘云道:“这话很是。
这样大家的诗还进一层。
但只咱们五个人,这十二个题目,难道每人作十二首不成?”宝钗道:“那也太难人了。
将这题目誊好,都要七言律,明日贴在墙上。
他们看了,谁作那一个就作那一个。
有力量者,十二首都作也可;不能的,一首不成也可。
高才捷足者为尊。
若十二首已全,便不许他后赶着又作,罚他就完了。
”湘云道:“这倒也罢了。
”二人商议妥贴,方才息灯安寝。
要知端的,且听下回分解。
\par
\qi{总评:薛家女子何贞侠,总因富贵不须夸。
发言行事何其嘉,居心用意不狂奢。
世人若肯平心度,便解云、钗两不暇。
\zhu{暇:闲暇。
云、钗两不暇:湘云和宝钗两人忙着“蘅芜苑夜拟菊花题”。
另一种解释,形容繁多,令读者来不及欣赏,应付不过来。
}}
\dai{073}{秋爽斋偶结海棠社}
\dai{074}{蘅芜苑夜拟菊花题}
\sun{p37-1}{贾芸寄书送花宝玉,秋爽斋偶结海棠社}{图上侧:翠墨送来探春的花笺,邀其商议结社之事,宝玉大喜,跟着翠墨往秋爽斋来,行至沁芳亭,只见园中后门上值日的婆子手里拿着一个贾芸关于送白海棠花的字帖走来。
图左下:来到秋爽斋,只见大家都到了,商量如何起诗社,又都起了诗人雅号。
接着,由李纨出题,迎春限韵,惜春监场,作起诗来。
}