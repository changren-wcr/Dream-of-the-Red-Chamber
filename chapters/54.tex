\chapter{史太君破陈腐旧套 \quad 王熙凤效戏彩斑衣}
\geng{首回楔子内云“古今小说千部共成一套”云云,\zhu{楔子:音“蝎子”,戏曲、小说的引子。
一般在篇首,用以点明、补充正文。
}犹未泄真。
今借老太君一写,是劝后来胸中无机轴之诸君子不可动笔作书。
\hang
凤姐乃太君之要紧陪堂,\zhu{陪堂:帮闲,陪客。
}今题“斑衣戏彩”是作者酬我阿凤之劳,特贬贾珍琏辈之无能耳。
}\par
\qi{积德于今到子孙,都中旺族首吾门。
可怜立业英雄辈,遗脉谁知祖父恩。
}\par
却说贾珍贾琏暗暗预备下大簸箩的钱,听见贾母说“赏”,他们也忙命小厮们快撒钱。
只听满台钱响,贾母大悦。
\par
二人遂起身,小厮们忙将一把新暖银壶捧在贾琏手内,随了贾珍趋至里面。
贾珍先至李婶席上,躬身取下杯来,回身,贾琏忙斟了一盏;然后便至薛姨妈席上,也斟了。
二人忙起身笑说:“二位爷请坐着罢了,何必多礼。
”于是除邢王二夫人,满席都离了席,俱垂手旁侍。
贾珍等至贾母榻前,因榻矮,二人便屈膝跪了。
贾珍在先捧杯,贾琏在后捧壶。
虽止二人奉酒,那贾环弟兄等,却也是排班按序,一溜随着他二人进来,见他二人跪下,也都一溜跪下。
宝玉也忙跪下了。
史湘云悄推他笑道:“你这会又帮着跪下作什么?有这样,你也去斟一巡酒岂不好?”宝玉悄笑道:“再等一会子再斟去。
”说着,等他二人斟完起来,方起来。
又与邢夫人王夫人斟过来。
贾珍笑道:“妹妹们怎么样呢?”\zhu{贾珍的意思是,自己是不是还要依次给众姊妹斟酒。
}贾母等都说:“你们去罢,他们倒便宜些。
”\zhu{便宜:音“变仪”,方便,顺当。
}说了,贾珍等方退出。
\par
当下天未二鼓,
\zhu{二鼓:晚九点到十一点。}
戏演的是《八义》中《观灯》八出。
\zhu{《八义》即《八义记》,明代徐元所作传奇剧本,据元杂剧《赵氏孤儿》改编,描写春秋时晋国赵盾一家与屠岸贾之间矛盾斗争的故事。
剧中有八个“义士”为赵盾一家出力效命,故称“八义记”。
《观灯》即《八义记》第五出《宴赏元宵》。写赵盾之子赵朔与妻子赵国公主一同观赏元宵节灯市景象。该出有一些滑稽表演,故场面比较“热闹”。
《八义记》不适合在年节演,作者在此时暗示贾家的繁华后面已经危机四伏。
}正在热闹之际,宝玉因下席往外走。
贾母因说:“你往那里去!外头爆竹利害,仔细天上掉下火纸来烧了。
”宝玉回说:“不往远去,只出去就来。
”贾母命婆子们好生跟着。
于是宝玉出来,只有麝月秋纹并几个小丫头随着。
\par
贾母因说:“袭人怎么不见?他如今也有些拿大了,单支使小女孩子出来。
”王夫人忙起身笑回道:“他妈前日没了,因有热孝,\zhu{热孝:俗称新遭父母丧事为热孝。
}不便前头来。
”贾母听了点头,又笑道:“跟主子却讲不起这孝与不孝。
若是他还跟我,难道这会子也不在这里不成?皆因我们太宽了,有人使,不查这些,竟成了例了。
”凤姐儿忙过来笑回道:“今儿晚上他便没孝,那园子里也须得他看着,灯烛花炮最是耽险的。
\zhu{耽:承受,担负。
耽险:担风险、危险。
}这里一唱戏,园子里的人谁不偷来瞧瞧。
他还细心,各处照看照看。
况且这一散后宝兄弟回去睡觉,各色都是齐全的。
若他再来了,众人又不经心,散了回去,铺盖也是冷的,茶水也不齐备,各色都不便宜,所以我叫他不用来,只看屋子。
散了又齐备,我们这里也不耽心,\zhu{耽心:担心。
}又可以全他的礼,岂不三处有益。
老祖宗要叫他,我叫他来就是了。
”\ping{王夫人把袭人看作宝玉未来的妾,是半个主人,替袭人没来而辩解的时候也是基于袭人这个“主人”身份来说的。
而贾母依旧把袭人看作奴仆,反对王夫人对袭人的刻意抬高。
最后王熙凤的解释是先承认袭人的奴仆身份,不来是为了做好后勤保障,而不是因为热孝在身,“全他的礼”只是客观上的效果,而不是目的。
}\par
贾母听了这话,忙说:“你这话很是,比我想的周到,快别叫他了。
但只他妈几时没了,我怎么不知道。
”凤姐笑道:“前儿袭人去亲自回老太太的,怎么倒忘了。
”贾母想了一想笑说:“想起来了。
我的记性竟平常了。
”\ping{一方面是贾母精神不济,身体状况下降,一方面表明袭人最开始的靠山贾母已经不把她放心上。
}众人都笑说:“老太太那里记得这些事。
”贾母因又叹道:“我想着,他从小儿伏侍了我一场,又伏侍了云儿一场,末后给了一个魔王宝玉,亏他魔了这几年。
他又不是咱们家的根生土长的奴才,没受过咱们什么大恩典。
他妈没了,我想着要给他几两银子发送,\zhu{发送:殡葬死者。
}也就忘了。
”凤姐儿道:“前儿太太赏了他四十两银子,也就是了。
”\par
贾母听说,点头道:“这还罢了。
正好鸳鸯的娘前儿也死了,我想他老子娘都在南边,我也没叫他家去走走守孝,如今叫他两个一处作伴儿去。
”又命婆子将些果子、菜馔、点心之类与他两个吃去。
琥珀笑说:“还等这会子呢,他早就去了。
”说着,大家又吃酒看戏。
\ping{同样都是母亲去世不来服侍,贾母对鸳鸯和袭人的态度差别很大,默许鸳鸯,敲打袭人,其意图还是提醒王夫人不要过度抬高袭人的地位,不认可王夫人让袭人成为宝玉未来的妾的安排。
}\par
且说宝玉一径来至园中,众婆子见他回房,便不跟去,只坐在园门里茶房里烤火,和管茶的女人偷空饮酒斗牌。
宝玉至院中,虽是灯光灿烂,却无人声。
麝月道:“他们都睡了不成?咱们悄悄的进去唬他们一跳。
”于是大家蹑足潜踪的进了镜壁一看,\zhu{镜壁:室内安装有玻璃的间墙。
宝玉和麝月可能通过镜子看到了袭人和鸳鸯。
}只见袭人和一人二人对面都歪在地炕上,那一头有两三个老嬷嬷打盹。
\par
宝玉只当他两个睡着了,才要进去,忽听鸳鸯叹了一声,说道:“可知天下事难定。
论理你单身在这里,父母在外头,每年他们东去西来,没个定准,想来你是不能送终的了,偏生今年就死在这里,你倒出去送了终。
”袭人道:“正是。
我也想不到能够看父母回首。
\zhu{回首:死亡的讳语。
}太太又赏了四十两银子,这倒也算养我一场,我也不敢妄想了。
”宝玉听了,忙转身悄向麝月等道:“谁知他也来了。
我这一进去,他又赌气走了,\ping{第四十六回,鸳鸯女誓绝鸳鸯偶,鸳鸯誓死不嫁贾赦,也不要嫁贾家子弟甚至所有的男人,便从此躲着宝玉。
}不如咱们回去罢,让他两个清清静静的说一回。
袭人正一个闷着,他幸而来的好。
”说着,仍悄悄的出来。
\par
宝玉便走过山石之后去站着撩衣,麝月秋纹皆站住背过脸去,口内笑说:“蹲下再解小衣,
\zhu{小衣:贴身内裤。}
仔细风吹了肚子。
”后面两个小丫头子知是小解,忙先出去茶房预备去了。
\ping{宝玉随地小便。
}这里宝玉刚转过来,只见两个媳妇子迎面来了,问是谁,秋纹道:“宝玉在这里,你大呼小叫,仔细唬着罢。
”那媳妇们忙笑道:“我们不知道,大节下来惹祸了。
姑娘们可连日辛苦了。
”说着,已到了跟前。
\ping{打狗也要看主人,奴才地位主子决定。
}\par
麝月等问:“手里拿的是什么?”媳妇们道:“是老太太赏金、花二位姑娘吃的。
”秋纹笑道:“外头唱的是《八义》,没唱《混元盒》,\zhu{《混元盒》:明末清初的一部神魔剧,内容荒诞不经。
其中有金花圣母娘娘同张真人斗法的情节,“混元盒”是张真人的一件法宝。
下文的“金花娘娘”是借戏中人物打趣“金(鸳鸯)、花(袭人)二位姑娘”。
}
那里又跑出‘金花娘娘’来了。
”宝玉笑命:“揭起来我瞧瞧。
”秋纹麝月忙上去将两个盒子揭开。
两个媳妇忙蹲下身子,\geng{细腻之极!一部大观园之文,皆若食肥蟹,至此一句,则又三月于镇江江上啖出网之鲜鲥矣。
}宝玉看了两盒内都是席上所有的上等果品菜馔,点了一点头,迈步就走。
麝月二人忙胡乱掷了盒盖,跟上来。
宝玉笑道:“这两个女人倒和气,会说话,他们天天乏了,倒说你们连日辛苦,倒不是那矜功自伐的。
”\zhu{矜功自伐:居功自夸。
矜:音“今”,认为自己了不起。
伐:自我夸耀。
}麝月道:“这好的也很好,那不知礼的也太不知礼。
”宝玉笑道:“你们是明白人,耽待他们是粗笨可怜的人就完了。
”\zhu{耽:承受,担负。
耽待:原谅。
}一面说,一面来至园门。
\par
那几个婆子虽吃酒斗牌,却不住出来打探,见宝玉来了,也都跟上了。
来至花厅后廊上,只见那两个小丫头一个捧着小沐盆,一个搭着手巾,又拿着沤子壶在那里久等。
\zhu{沤(音“偶”,四声)子:一种润肤的油脂香蜜。
}秋纹先忙伸手向盆内试了一试,说道:“你越大越粗心了,那里弄的这冷水。
”小丫头笑道:“姑娘瞧瞧这个天,我怕水冷,巴巴的倒的是滚水,这还冷了。
”\par
正说着,可巧见一个老婆子提着一壶滚水走来。
小丫头便说:“好奶奶,过来给我倒上些。
”那婆子道:“哥哥儿,这是老太太泡茶的,劝你走了舀去罢,那里就走大了脚。
”秋纹道:“凭你是谁的,你不给?我管把老太太茶吊子倒了洗手。
”那婆子回头见是秋纹,忙提起壶来就倒。
秋纹道:“够了。
你这么大年纪也没个见识,谁不知是老太太的水!要不着的人就敢要了。
”婆子笑道:“我眼花了,没认出这姑娘来。
”宝玉洗了手,那小丫头子拿小壶倒了些沤子在他手内,宝玉沤了。
秋纹麝月也趁热水洗了一回,沤了,跟进宝玉来。
\par
宝玉便要了一壶暖酒,也从李婶薛姨妈斟起,二人也让坐。
贾母便说:“他小,让他斟去,大家倒要干过这杯。
”说着,便自己干了。
邢王二夫人也忙干了,让他二人。
薛李也只得干了。
贾母又命宝玉道:“连你姐姐妹妹一齐斟上,不许乱斟,都要叫他干了。
”宝玉听说,答应着,一一按次斟了。
\par
至黛玉前,偏他不饮,拿起杯来,放在宝玉唇上边,宝玉一气饮干。
黛玉笑说:“多谢。
”宝玉替他斟上一杯。
凤姐儿便笑道:“宝玉,别喝冷酒,仔细手颤,明儿写不得字,拉不得弓。
”
\zhu{第八回:薛姨妈忙道:“这可使不得,吃了冷酒,写字手打飐(颤)儿。”}
宝玉忙道:“没有吃冷酒。
”凤姐儿笑道:“我知道没有,不过白嘱咐你。
”然后宝玉将里面斟完,只除贾蓉之妻是丫头们斟的。
复出至廊上,又与贾珍等斟了。
坐了一回,方进来仍归旧坐。
\par
一时上汤后,又接献元宵来。
贾母便命将戏暂歇歇:“小孩子们可怜见的,也给他们些滚汤滚菜的吃了再唱。
”又命将各色果子元宵等物拿些与他们吃去。
\par
一时歇了戏,便有婆子带了两个门下常走的女先儿进来,\zhu{女先儿:女盲艺人。
旧时习惯称算命和说书唱曲的盲艺人为“先儿”。
}放两张杌子在那一边命他坐了,\zhu{杌(音“物”):小凳子。
}将弦子琵琶递过去。
贾母便问李薛听何书,他二人都回说:“不拘什么都好。
”贾母便问:“近来可有添些什么新书?”那两个女先儿回说道:“倒有一段新书,是残唐五代的故事。
”贾母问是何名,女先儿道:“叫做《凤求鸾》。
”贾母道:“这一个名字倒好,不知因什么起的,先大概说说原故,若好再说。
”女先儿道:“这书上乃说残唐之时,有一位乡绅,本是金陵人氏,名唤王忠,曾做过两朝宰辅,如今告老还家,膝下只有一位公子,名唤王熙凤。
”\par
众人听了,笑将起来。
贾母笑道:“这重了我们凤丫头了。
”媳妇忙上去推他,“这是二奶奶的名字,少混说。
”贾母笑道:“你说,你说。
”女先生忙笑着站起来,说:“我们该死了,不知是奶奶的讳。
”凤姐儿笑道:“怕什么,你们只管说罢,重名重姓的多呢。
”\par
女先生又说道:“这年王老爷打发了王公子上京赶考,那日遇见大雨,进到一个庄上避雨。
谁知这庄上也有个乡绅,姓李,与王老爷是世交,便留下这公子住在书房里。
这李乡绅膝下无儿,只有一位千金小姐。
这小姐芳名叫作雏鸾,琴棋书画,无所不通。
”贾母忙道:“怪道叫作《凤求鸾》。
不用说,我猜着了,自然是这王熙凤要求这雏鸾小姐为妻。
”女先儿笑道:“老祖宗原来听过这一回书。
”众人都道:“老太太什么没听过!便没听过,也猜着了。
”\par
贾母笑道:“这些书都是一个套子,左不过是些佳人才子,\zhu{左不过:反正,只不过,无非。
}最没趣儿。
把人家女儿说的那样坏,还说是佳人,编的连影儿也没有了。
开口都是书香门第,父亲不是尚书就是宰相,生一个小姐必是爱如珍宝。
这小姐必是通文知礼,无所不晓,竟是个绝代佳人。
只一见了一个清俊的男人,不管是亲是友,便想起终身大事来,父母也忘了,书礼也忘了,鬼不成鬼,贼不成贼,那一点儿是佳人?便是满腹文章,做出这些事来,也算不得是佳人了。
比如男人满腹文章去作贼,难道那王法就说他是才子,就不入贼情一案不成?可知那编书的是自己塞了自己的嘴。
再者,既说是世宦书香大家小姐都知礼读书,连夫人都知书识礼,便是告老还家,自然这样大家人口不少,奶母丫鬟伏侍小姐的人也不少,怎么这些书上,凡有这样的事,就只小姐和紧跟的一个丫鬟?你们白想想,那些人都是管什么的,可是前言不答后语?”\par
众人听了,都笑说:“老太太这一说,是谎都批出来了。
”贾母笑道:“这有个原故:编这样书的,有一等妒人家富贵,或有求不遂心,所以编出来污秽人家。
再一等,他自己看了这些书看魔了,他也想一个佳人,所以编了出来取乐。
何尝他知道那世宦读书家的道理!别说他那书上那些世宦书礼大家,如今眼下真的,拿我们这中等人家说起,也没有这样的事,别说是那些大家子。
可知是诌掉了下巴的话。
\zhu{诌:信口胡说,编瞎话。
}所以我们从不许说这些书,丫头们也不懂这些话。
这几年我老了,他们姊妹们住的远,我偶然闷了,说几句听听,他们一来,就忙歇了。
”李薛二人都笑说:“这正是大家的规矩,连我们家也没这些杂话给孩子们听见。
”
\ping{贾母先听故事梗概是为了给孩子进行内容审查。}
\par
凤姐儿走上来斟酒,笑道:“罢,罢,酒冷了,老祖宗喝一口润润嗓子再掰谎。
这一回就叫作《掰谎记》,就出在本朝本地本年本月本日本时,老祖宗一张口难说两家话,花开两朵,各表一枝,是真是谎且不表,再整那观灯看戏的人。
老祖宗且让这二位亲戚吃一杯酒看两出戏之后,再从昨朝话言掰起如何?”他一面斟酒,一面笑说,未曾说完,众人俱已笑倒。
两个女先儿也笑个不住,都说:“奶奶好刚口。
\zhu{刚口:说书艺人用语,意为言词爽利动听。
这里意近“口才”。
}奶奶要一说书,真连我们吃饭的地方也没了。
”薛姨妈笑道:“你少兴头些,外头有人,比不得往常。
”凤姐儿笑道:“外头的只有一位珍大爷。
我们还是论哥哥妹妹,从小儿一处淘气了这么大。
这几年因做了亲,我如今立了多少规矩了。
便不是从小儿的兄妹,便以伯叔论,那《二十四孝》上‘斑衣戏彩’,\zhu{《二十四孝》上‘斑衣戏彩’:《二十四孝》是元代郭居业编的一本宣扬封建孝道的书,共收二十四个“孝子”的故事,“斑衣戏彩”即其中之一,也称“老莱娱亲”。
写七十岁的老莱子穿上色彩斑斓的衣裳,拿着玩具学儿童嬉戏,以使双亲欢娱。
}他们不能来‘戏彩’引老祖宗笑一笑,我这里好容易引的老祖宗笑了一笑,多吃了一点儿东西,大家喜欢,都该谢我才是,难道反笑话我不成?”贾母笑道:“可是这两日我竟没有痛痛的笑一场,倒是亏他,才一路笑的我心里痛快了些,我再吃一钟酒。
”吃着酒,又命宝玉:“也敬你姐姐一杯。
”凤姐儿笑道:“不用他敬,我讨老祖宗的寿罢。
”说着,便将贾母的杯拿起来,将半杯剩酒吃了,将杯递与丫鬟,另将温水浸的杯换了一个上来。
于是各席上的杯都撤去,另将温水浸着待换的杯斟了新酒上来,然后归坐。
\par
女先儿回说:“老祖宗不听这书,或者弹一套曲子听听罢。
”贾母便说道:“你们两个对一套《将军令》罢。
”\zhu{对:两者相对,两人在一起。
将军令:乐曲名,原为军中发令时所用鼓吹之曲,后仿其调制成乐曲。
}二人听说,忙和弦按调拨弄起来。
贾母因问:“天有几更了。
”众婆子忙回:“三更了。
”贾母道:“怪道寒浸浸的起来。
”早有众丫鬟拿了添换的衣裳送来。
王夫人起身笑说道:“老太太不如挪进暖阁里地炕上倒也罢了。
这二位亲戚也不是外人,我们陪着就是了。
”贾母听说,笑道:“既这样说,不如大家都挪进去,岂不暖和?”王夫人道:“恐里间坐不下。
”贾母笑道:“我有道理。
如今也不用这些桌子,只用两三张并起来,大家坐在一处挤着,又亲香,又暖和。
”众人都道:“这才有趣。
”说着,便起了席。
众媳妇忙撤去残席,里面直顺并了三张大桌,另又添换了果馔摆好。
贾母便说:“这都不要拘礼,只听我分派你们就坐才好。
”说着便让薛李正面上坐,自己西向坐了,叫宝琴、黛玉、湘云三人皆紧依左右坐下,向宝玉说:“你挨着你太太。
”于是邢夫人王夫人之中夹着宝玉,宝钗等姊妹在西边,挨次下去便是娄氏带着贾菌,尤氏李纨夹着贾兰,下面横头便是贾蓉之妻。
\zhu{横头:正面两侧的位置,或长方形物体较短两侧的位置。
}贾母便说:“珍哥儿带着你兄弟们去罢,我也就睡了。
”\par
贾珍等忙答应,又都进来。
贾母道:“快去罢!不用进来,才坐好了,又都起来。
你快歇着,明日还有大事呢。
”贾珍忙答应了,又笑说:“留下蓉儿斟酒才是。
”贾母笑道:“正是忘了他。
”贾珍答应了一个“是”,便转身带领贾琏等出来。
二人自是欢喜,便命人将贾琮贾璜各自送回家去,便邀了贾琏去追欢买笑,不在话下。
\par
这里贾母笑道:“我正想着虽然这些人取乐,竟没一对双全的,就忘了蓉儿。
这可全了,蓉儿就合你媳妇坐在一处,倒也团圆了。
”因有媳妇回说开戏,贾母笑道:“我们娘儿们正说的兴头,又要吵起来。
况且那孩子们熬夜怪冷的,也罢,叫他们且歇歇,把咱们的女孩子们叫了来,就在这台上唱两出给他们瞧瞧。
”媳妇听了,答应了出来,忙的一面着人往大观园去传人,一面二门口去传小厮们伺候。
小厮们忙至戏房将班中所有的大人一概带出,只留下小孩子们。
\ping{这里可能是作者写错或者抄手抄错,根据上下文,带出的是文官等小孩子去唱戏,而不是大人。
可能应该是“将班中所有的小孩子们一概带出,只留下大人”。
}\par
一时,梨香院的教习带了文官等十二个人,从游廊角门出来。
婆子们抱着几个软包,\zhu{软包:演出时只带简单的服装道具的布包。
}因不及抬箱,估料着贾母爱听的三五出戏的彩衣包了来。
婆子们带了文官等进去见过,只垂手站着。
贾母笑道:“大正月里,你师父也不放你们出来逛逛。
你等唱什么?刚才八出《八义》闹得我头疼,咱们清淡些好。
你瞧瞧,薛姨太太这李亲家太太都是有戏的人家,不知听过多少好戏的。
这些姑娘们都比咱们家姑娘见过好戏,听过好曲子。
如今这小戏子又是那有名玩戏家的班子,虽是小孩子们,却比大班还强。
咱们好歹别落了褒贬,少不得弄个新样儿的。
叫芳官唱一出《寻梦》,只提琴至管箫合,\zhu{提琴:书中此处指胡琴。
}笙笛一概不用\foot{此句疑有错夺。
有人断为“只提琴,至管箫合笙笛一概不用。
”语气更顺畅,但与后文薛姨妈说的“从没见用箫管的”矛盾。
暂保留目前的标点。
}。
”文官笑道:“这也是的,我们的戏自然不能入姨太太和亲家太太姑娘们的眼,不过听我们一个发脱口齿,\zhu{发脱口齿:唱戏时的发声吐字。
}再听一个喉咙罢了。
”贾母笑道:“正是这话了。
”李婶薛姨妈喜的都笑道:“好个灵透孩子,他也跟着老太太打趣我们。
”贾母笑道:“我们这原是随便的顽意儿,又不出去做买卖,所以竟不大合时。
”说着又道:“叫葵官唱一出《惠明下书》,\zhu{《惠明下书》:《西厢记》第二本第二折,惠明和尚持张生的书信投送蒲关,请白马将军杜确前来普救寺解围的情节。
}也不用抹脸。
只用这两出叫他们听个疏异罢了。
\zhu{疏异:这里即新鲜别致之意。
}若省一点力,我可不依。
”\par
文官等听了出来,忙去扮演上台,先是《寻梦》,\zhu{《寻梦》:《牡丹亭》的第十二出,写杜丽娘在梦中与柳梦梅欢会后,次日在花园中循迹重温梦境的情节。
}次是《下书》。
众人都鸦雀无闻,薛姨妈因笑道:“实在亏他,戏也看过几百班,从没见用箫管的。
”贾母道:“也有,只是像方才《西楼·楚江晴》一支,多有小生吹箫和的。
这大套的实在少,这也在主人讲究不讲究罢了。
这算什么出奇?”指湘云道:“我像他这么大的时节,他爷爷有一班小戏,偏有一个弹琴的凑了来,即如《西厢记》的《听琴》,《玉簪记》的《琴挑》,《续琵琶》的《胡笳十八拍》,\zhu{《听琴》是《西厢记》第二本,写崔莺莺月夜听张生弹琴而知音会意的情景。
《玉簪记》是明代高濂编写的传奇,描写尼姑陈妙常和书生潘必正结合的故事;《琴挑》是该剧第十六出《寄弄》演出本的名目。
剧情为:潘必正暂住女贞观中,偶遇道姑陈妙常,顿生爱慕,故借弹琴之机,奏《雉朝飞》曲,以情挑之。陈妙常为之心动,但惧于戒规,不得不假意发怒,逐其离去。
《续琵琶》是曹雪芹的祖父曹寅撰写的传奇,今存抄本,描写汉末蔡邕的女儿蔡文姬因汉室衰亡,战争迭起而被掳入匈奴,十二年后被曹操赎回,又不得不与亲子生离。其第二十七出《制拍》表现蔡文姬写作和弹奏《胡笳十八拍》,倾诉自己一生的遭遇和心情。
}竟成了真的了,比这个更如何?”\ping{文官等人唱的时候用箫管伴奏,使得贾母想起了自己小时候听戏的时候,《听琴》、《琴挑》这些有“琴”元素的剧本有琴来伴奏。
}
众人都道:“这更难得了。
”贾母便命个媳妇来,吩咐文官等叫他们吹一套《灯月圆》。
\zhu{《灯月圆》:不详,可能系作者自创。
曲牌中有《人月圆》之名,或由此变化而来。
}媳妇领命而去。
\par
当下贾蓉夫妻二人捧酒一巡,凤姐儿因见贾母十分高兴,便笑道:“趁着女先儿们在这里,不如叫他们击鼓,咱们传梅,行一个‘春喜上眉梢’的令如何?”\zhu{春喜上眉梢:即“击鼓传梅”的雅称,“梅”“眉”谐音。
将“传梅”说成“喜上眉(梅)梢”,是为讨吉利的口彩。
}贾母笑道:“这是个好令,正对时对景。
”忙命人取了一面黑漆铜钉花腔令鼓来,\zhu{花腔:花腔鼓,鼓框上绘有花纹的鼓。
}与女先儿们击着,席上取了一枝红梅。
贾母笑道:“若到谁手里住了,吃一杯,也要说个什么才好。
”凤姐儿笑道:“依我说,谁像老祖宗要什么有什么呢。
我们这不会的,岂不没意思。
依我说也要雅俗共赏,不如谁输了谁说个笑话罢。
”众人听了,都知道他素日善说笑话,最是他肚内有无限的新鲜趣谈。
今儿如此说,不但在席的诸人喜欢,连地下伏侍的老小人等无不欢喜。
那小丫头子们都忙出去,找姐唤妹的告诉他们:“快来听,二奶奶又说笑话儿了。
”众丫头子们便挤了一屋子。
于是戏完乐罢。
贾母命将些汤点果菜与文官等吃去,便命响鼓。
那女先儿们皆是惯的,或紧或慢,或如残漏之滴,或如迸豆之疾,或如惊马之乱驰,或如疾电之光而忽暗。
其鼓声慢,传梅亦慢;鼓声疾,传梅亦疾。
恰恰至贾母手中,鼓声忽住。
大家呵呵一笑,贾蓉忙上来斟了一杯。
众人都笑道:“自然老太太先喜了,我们才托赖些喜。
”贾母笑道:“这酒也罢了,只是这笑话倒有些个难说。
”众人都说:“老太太的比凤姐儿的还好还多,赏一个我们也笑一笑儿。
”\par
贾母笑道:“并没什么新鲜发笑的,少不得老脸皮子厚的说一个罢了。
”因说道:“一家子养了十个儿子,娶了十房媳妇。
惟有第十个媳妇最聪明伶俐,心巧嘴乖,公婆最疼,成日家说那九个不孝顺。
这九个媳妇委屈,便商议说:‘咱们九个心里孝顺,只是不像那小蹄子嘴巧,所以公公婆婆老了,只说他好,这委屈向谁诉去?’大媳妇有主意,便说道:‘咱们明儿到阎王庙去烧香,和阎王爷说去,问他一问,叫我们托生人,为什么单单的给那小蹄子一张乖嘴,我们都是笨的。
’众人听了都喜欢,说这主意不错。
第二日便都到阎王庙里来烧了香,九个人都在供桌底下睡着了。
九个魂专等阎王驾到,左等不来,右等也不到。
正着急,只见孙行者驾着筋斗云来了,看见九个魂便要拿金箍棒打,唬得九个魂忙跪下央求。
孙行者问原故,九个人忙细细的告诉了他。
孙行者听了,把脚一跺,叹了一口气道:‘这原故幸亏遇见我,等着阎王来了,他也不得知道的。
’九个人听了,就求说:‘大圣发个慈悲,我们就好了。
’孙行者笑道:‘这却不难。
那日你们妯娌十个托生时,可巧我到阎王那里去的,因为撒了泡尿在地下,你那小婶子便吃了。
你们如今要伶俐嘴乖,有的是尿,再撒泡你们吃了就是了。
’”说毕,大家都笑起来。
\par
凤姐儿笑道:“好的,幸而我们都笨嘴笨腮的,不然也就吃了猴儿尿了。
”尤氏娄氏都笑向李纨道:“咱们这里谁是吃过猴儿尿的,别装没事人儿。
”薛姨妈笑道:“笑话儿不在好歹,只要对景就发笑。
”说着又击起鼓来。
小丫头子们只要听凤姐儿的笑话,便俏俏的和女先儿说明,以咳嗽为记。
须臾传至两遍,刚到了凤姐儿手里,小丫头子们故意咳嗽,女先儿便住了。
\par
众人齐笑道:“这可拿住他了。
快吃了酒说一个好的,别太逗的人笑的肠子疼。
”凤姐儿想了一想,笑道:“一家子也是过正月半,合家赏灯吃酒,真真的热闹非常,祖婆婆、太婆婆、婆婆、媳妇、孙子媳妇、重孙子媳妇、亲孙子、侄孙子、重孙子、灰孙子、滴滴搭搭的孙子、孙女儿、外孙女儿、姨表孙女儿、姑表孙女儿,……嗳哟哟,真好热闹!”众人听他说着,已经笑了,都说:“听数贫嘴,又不知编派那一个呢?”尤氏笑道:“你要招我,我可撕你的嘴。
”凤姐儿起身拍手笑道:“人家费力说,你们混,我就不说了。
”贾母笑道:“你说你说,底下怎么样?”凤姐儿想了一想,笑道:“底下就团团的坐了一屋子,吃了一夜酒就散了。
”众人见他正言厉色的说了,别无他话,都怔怔的还等下话,只觉冰冷无味。
\par
史湘云看了他半日,凤姐儿笑道:“再说一个过正月半的。
几个人抬着个房子大的炮仗往城外放去,引了上万的人跟着瞧去。
有一个性急的人等不得,便偷着拿香点着了。
只听‘噗哧’一声,众人哄然一笑都散了。
这抬炮仗的人抱怨卖炮仗的捍的不结实,\zhu{捍:坚实。
这里应该是捆绑起来使得坚实。
}没等放就散了。
”湘云道:“难道他本人没听见响?”凤姐儿道:“这本人原是聋子。
”众人听说,一回想,不觉一齐失声都大笑起来。
又想着先前那一个没完的,问他:“先一个怎么样?也该说完。
”凤姐儿将桌子一拍,说道:“好罗唆,到了第二日是十六日,年也完了,节也完了,我看着人忙着收东西还闹不清,那里还知道底下的事了。
”众人听说,复又笑将起来。
凤姐儿笑道:“外头已经四更,依我说,老祖宗也乏了,咱们也该‘聋子放炮仗——散了’罢。
”\ping{偏于极热闹极富丽之处道此不祥之谶语。
现在团圆一屋子似乎很多人呢,但是散的快,甚至可能还没放就散了,这就是急转直下之前的一点点谶语,热闹相聚转眼就是四散零落。
炮仗无声散了的细节,对应首回“谨防佳节元宵夜,便是烟消火灭时。”
}尤氏等用手帕子握着嘴,笑的前仰后合,指他说道:“这个东西真会数贫嘴。
”贾母笑道:“真真这凤丫头越发贫嘴了。
”一面说,一面吩咐道:“他提炮仗来,咱们也把烟火放了解解酒。
”\par
贾蓉听了,忙出去带着小厮们就在院内安下屏架,将烟火设吊齐备。
\zhu{设吊:布置、吊挂。
}
这烟火皆系各处进贡之物,虽不甚大,却极精巧,各色故事俱全,夹着各色花炮。
林黛玉禀气柔弱,不禁毕驳之声,\zhu{毕驳:拟声词。
}贾母便搂他在怀中。
薛姨妈搂着湘云。
湘云笑道:“我不怕。
”宝钗等笑道:“他专爱自己放大炮仗,还怕这个呢。
”王夫人便将宝玉搂入怀内。
凤姐儿笑道:“我们是没有人疼的了。
”尤氏笑道:“有我呢,我搂着你。
也不怕臊,你这孩子又撒娇了,听见放炮仗,吃了蜜蜂儿屎的,今儿又轻狂起来。
”凤姐儿笑道:“等散了,咱们园子里放去。
我比小厮们还放的好呢。
”\par
说话之间,外面一色一色的放了又放,又有许多的满天星、九龙入云、一声雷、飞天十响之类的零碎小爆竹。
放罢,然后又命小戏子打了一回“莲花落”,\zhu{莲花落:曲艺的一种,也叫“莲花乐”、“落(音“涝”)子”。
原为行乞卖唱者所唱,后出现专业艺人;演唱内容多为民间传说,打竹板按节拍伴奏,因而说“打了一回莲花落”。
}撒了满台钱,命那孩子们满台抢钱取乐。
又上汤时,贾母说道:“夜长,觉的有些饿了。
”凤姐儿忙回说:“有预备的鸭子肉粥。
”贾母道:“我吃些清淡的罢。
”凤姐儿忙道:“也有枣儿熬的粳米粥,
\zhu{粳[jīng]:稻之不黏者。}
预备太太们吃斋的。
”贾母笑道:“不是油腻腻的就是甜的。
”凤姐儿又忙道:“还有杏仁茶,只怕也甜。
”贾母道:“倒是这个还罢了。
”说着,又命人撤去残席,外面另设上各种精致小菜。
大家随便随意吃了些,用过漱口茶,方散。
\par
十七日一早,又过宁府行礼,伺候掩了宗祠,收过影像,方回来。
此日便是薛姨妈家请吃年酒。
十八日便是赖大家,十九日便是宁府赖升家,二十日便是林之孝家,二十一日便是单大良家,二十二日便是吴新登家。
这几家,贾母也有去的,也有不去的,也有高兴直待众人散了方回的,也有兴尽半日一时就来的。
凡诸亲友来请或来赴席的,贾母一概怕拘束不会,自有邢夫人、王夫人、凤姐儿三人料理。
连宝玉只除王子腾家去了,馀者亦皆不会,只说贾母留下解闷。
所以倒是家下人家来请,贾母可以自便之处,方高兴去逛逛。
闲言不提,且说当下元宵已过——\par
\qi{总评:读此回者凡三变。
不善读者徒赞其如何演戏、如何行令、如何挂花灯、如何放爆竹,目眩耳聋,应接不暇。
少解读者,赞其座次有伦、巡酒有度,从演戏渡至女先,从女先渡至凤姐,从凤姐渡至行令,从行令渡至放花爆:脱卸下来,井然秩然,一丝不乱。
会读者须另具卓识,单着眼史太君一席话,将普天下不近理之“奇文”、不近情之“妙作”一齐抹倒。
是作者借他人酒杯,消自己傀儡,\zhu{傀儡:可能是“块垒”的错讹。
块垒:亦作“块礨”,亦作“块磊”,积砌成堆的土块。
泛指郁积之物,比喻胸中郁结的愁闷或气愤。
出自《世说新语·任诞》:“阮籍胸中垒块,故须酒浇之。
”}画一幅行乐图,铸一面菱花镜,\zhu{
菱花镜:古代铜镜。
六角形或背面刻有菱花者名菱花镜。
}为全部总评。
噫!作者已逝,圣叹云亡,\zhu{圣叹:金圣叹,明末清初文学批评家,曾批改《水浒传》《西厢记》。
云:句首、句中、句末语气词。
}愚不自量,辄拟数语,知我罪我,其听之矣。
}
\dai{107}{贾母批评才子佳人陈腐旧套}
\dai{108}{击鼓传梅讲笑话}
\sun{p54-1}{听戏饮酒传梅行令}{三更时分,贾母提议在里间暖阁地炕上摆一行桌子,留下女眷和小辈,挤着坐在一起。
遣人将梨香院的戏班找来唱戏。
梨香院的教习带了文官等十二个人来唱戏,婆子们抱着几个软包,包了贾母爱听的三五出戏的彩衣。
凤姐又提议击鼓行令罚酒讲笑话,众人欣允。
}