\chapter{呆霸王调情遭苦打 \quad 冷郎君惧祸走他乡}
\qi{不是同人,且莫浪作知心语。
\zhu{薛蟠把柳湘莲误认为自己的同道中人,有龙阳之好。
}似假如真,事事应难许。
\zhu{柳湘莲假装答应,真假难辨。
}着紧温存,白雪阳春曲。
\zhu{柳湘莲装出一副“着紧温存”的样子,骗薛蟠说到另一处去喝酒玩乐。
所谓“白雪阳春曲”即说知音难遇, “阳春”和“白雪”是战国时期楚国的高级歌曲,楚国宋玉《对楚王问》: “客有歌于郢中者,其始曰《下里》、《巴人》,国中属而和者数千人……其为《阳春》、《白雪》,国中属而和者不过数十人。
”}谁堪比?船上要离,未解奸侠起。
\zhu{要离是战国时吴国的刺客,为吴王阖闾刺杀政敌庆忌,要离骗得庆忌信任,乘其不备在船上刺杀了庆忌。
这里说柳湘莲就像要离骗庆忌一样,甜言蜜语把薛蟠骗到野外痛打。
骗似“奸”而实为“侠”,故曰“奸侠”,第一回《好了歌解注》中有“训有方,保不定日后作强梁”,有针对性的脂批说“柳湘莲一干人”,可见柳湘莲是符合“奸侠”之称呼的。
}}\par
话说王夫人听见邢夫人来了,连忙迎了出去。
邢夫人犹不知贾母已知鸳鸯之事,正还要来打听信息,进了院门,早有几个婆子悄悄的回了他,他方知道。
待要回去,里面已知,又见王夫人接了出来,\ping{王夫人赶紧把邢夫人请进来,承接贾母的怒火,自己就可以暂时解脱。
}少不得进来,先与贾母请安,贾母一声儿不言语,自己也觉得愧悔。
凤姐儿早指一事回避了。
鸳鸯也自回房去生气。
薛姨妈王夫人等恐碍着邢夫人的脸面,也都渐渐的退了。
邢夫人且不敢出去。
\par
贾母见无人,方说道:“我听见你替你老爷说媒来了。
你倒也三从四德,\zhu{三从四德:施于妇女的封建礼教。
从:服从;听从。
三从:未嫁从父,既嫁从夫,夫死从子。
见《仪礼·丧服·子夏传》。
德:道德规范。
四德:妇德(品德),妇言(辞令),妇容(仪态),妇功(女工)。
见《周礼·天官·九嫔》。
}只是这贤慧也太过了!你们如今也是孙子儿子满眼了,你还怕他,劝两句都使不得,还由着你老爷性儿闹。
”邢夫人满面通红,回道:“我劝过几次不依。
老太太还有什么不知道呢,我也是不得已儿。
”贾母道:“他逼着你杀人,你也杀去?如今你也想想,你兄弟媳妇本来老实,又生得多病多痛,上上下下那不是他操心?你一个媳妇虽然帮着,也是天天丢下笆儿弄扫帚。
\zhu{笆: 用树枝、荆条、竹篾等编制的片状物。
丢下笆儿弄扫帚:俗语,搁下这样,又做那样,事情总也做不完,忙不过来。
}凡百事情,我如今都自己减了。
他们两个就有一些不到的去处,\zhu{他们两个:凤姐和王夫人。
}有鸳鸯,那孩子还心细些,我的事情他还想着一点子,该要去的,他就要了来,该添什么,他就度空儿告诉他们添了。
\zhu{度:音“夺”,计算,估计。
}鸳鸯再不这样,他娘儿两个,里头外头,大的小的,那里不忽略一件半件,我如今反倒自己操心去不成?还是天天盘算和你们要东西去?我这屋里有的没的,剩了他一个,年纪也大些,我凡百的脾气性格儿他还知道些。
\zhu{凡百:表示概括的词,全部,所有的意思。
}
二则他还投主子们的缘法,也并不指着我和这位太太要衣裳去,又和那位奶奶要银子去。
\ping{“指着”的意思应该是指靠依仗,这里说明了鸳鸯自己虽然作为贾母的代理人,形式上拥有贾母的权威,但是并不会以权谋私。
}所以这几年一应事情,他说什么,从你小婶和你媳妇起,\zhu{小婶:王夫人。
媳妇:王熙凤。
邢夫人对于王夫人和王熙凤的称呼,来源于儿子贾琏对于她们的称呼。
“小婶”类似于“小姑”。
}以至家下大大小小,没有不信的。
所以不单我得靠,连你小婶媳妇也都省心。
我有了这么个人,便是媳妇和孙子媳妇有想不到的,我也不得缺了,也没气可生了。
这会子他去了,你们弄个什么人来我使?你们就弄他那么一个真珠的人来,\zhu{真珠:珍珠。
}不会说话也无用。
我正要打发人和你老爷说去,他要什么人,我这里有钱,叫他只管一万八千的买,就只这个丫头不能。
留下他伏侍我几年,就比他日夜伏侍我尽了孝的一般。
你来的也巧,你就去说,更妥当了。
”\par
说毕,命人来:“请了姨太太、你姑娘们来说个话儿。
才高兴,怎么又都散了!”丫头们忙答应着去了。
众人忙赶的又来。
只有薛姨妈向丫鬟道:“我才来了,又作什么去?你就说我睡了觉了。
”\ping{薛姨妈全家是凭借王夫人的关系住在贾府,而贾母刚刚痛斥薛姨妈依靠的王夫人,使得薛姨妈脸上无光。
}那丫头道:“好亲亲的姨太太,姨祖宗!我们老太太生气呢,你老人家不去,没个开交了,只当疼我们罢。
你老人家嫌乏,我背了你老人家去。
”薛姨妈道:“小鬼头儿,你怕些什么?不过骂几句完了。
”说着,只得和这小丫头子走来。
贾母忙让坐,又笑道:“咱们斗牌罢。
姨太太的牌也生,咱们一处坐着,别叫凤姐儿混了我们去。
”薛姨妈笑道:“正是呢,老太太替我看着些儿。
就是咱们娘儿四个斗呢,还是再添个呢?”王夫人笑道:“可不只四个。
”
\zhu{可不只四个:断句为“可不”、“只四个”。}
\geng{老实人言语。
}凤姐儿道:“再添一个人热闹些。
”贾母道:“叫鸳鸯来,叫他在这下手里坐着。
姨太太眼花了,咱们两个的牌都叫他瞧着些儿。
”凤姐儿叹了一声,向探春道:“你们知书识字的,倒不学算命!”探春道:“这又奇了。
这会子你倒不打点精神赢老太太几个钱,又想算命。
”凤姐儿道:“我正要算算今儿该输多少呢,我还想赢呢!你瞧瞧,场子没上,左右都埋伏下了。
”说的贾母薛姨妈都笑起来。
\par
一时,鸳鸯来了,便坐在贾母下手,鸳鸯之下便是凤姐儿。
铺下红毡,洗牌告幺,\zhu{告幺:斗牌时,洗完牌,由头家掷骰子,或每人先翻一张牌,点数最小的人先拿牌。
因“幺”(“一”)点次序最先,故称这种按点起牌叫“告幺”。
}五人起牌。
斗了一回,鸳鸯见贾母的牌已十严,\zhu{十严:亦云“得等”。
即斗牌时牌已配齐,只等所需的最后一张牌出现,便可放牌获胜,谓之“十严”。
}只等一张二饼,便递了暗号与凤姐儿。
凤姐儿正该发牌,便故意踌躇了半晌,笑道:“我这一张牌定在姨妈手里扣着呢。
我若不发这一张,再顶不下来的。
”薛姨妈道:“我手里并没有你的牌。
”凤姐儿道:“我回来是要查的。
”薛姨妈道:“你只管查。
你且发下来,我瞧瞧是张什么。
”凤姐儿便送在薛姨妈跟前。
薛姨妈一看是个二饼,便笑道:“我倒不稀罕他,只怕老太太满了。
”凤姐儿听了,忙笑道:“我发错了。
”贾母笑的已掷下牌来,说:“你敢拿回去!谁叫你错的不成?”凤姐儿道:“可是我要算一算命呢。
这是自己发的,也怨埋伏!”贾母笑道:“可是呢,你自己该打着你那嘴,问着你自己才是。
”又向薛姨妈笑道:“我不是小器爱赢钱,原是个彩头儿。
”薛姨妈笑道:“可不是这样,那里有那样糊涂人说老太太爱钱呢?”凤姐儿正数着钱,听了这话,忙又把钱穿上了,\zhu{把钱穿上:旧时使用的制钱,中有方孔,为便于携带和保存,多用绳穿起来。
}向众人笑道:“够了我的了。
竟不为赢钱,单为赢彩头儿。
我到底小器,输了就数钱,快收起来罢。
”贾母规矩是鸳鸯代洗牌,因和薛姨妈说笑,不见鸳鸯动手,贾母道:“你怎么恼了,连牌也不替我洗。
”鸳鸯拿起牌来,笑道:“二奶奶不给钱。
”贾母道:“他不给钱,那是他交运了。
”便命小丫头子:“把他那一吊钱都拿过来。
”\zhu{一吊钱:旧时制钱一个叫一文,一千文叫一吊,也叫一串。
各地并不一致,也有一百文作一吊的。
}小丫头子真就拿了,搁在贾母旁边。
凤姐儿笑道:“赏我罢,我照数儿给就是了。
”薛姨妈笑道:“果然是凤丫头小器,不过是顽儿罢了。
”凤姐听说,便站起来,拉着薛姨妈,回头指着贾母素日放钱的一个木匣子笑道:“姨妈瞧瞧,那个里头不知顽了我多少去了。
这一吊钱顽不了半个时辰,那里头的钱就招手儿叫他了。
只等把这一吊也叫进去了,牌也不用斗了,老祖宗的气也平了,又有正经事差我办去了。
”话说未完,引的贾母众人笑个不住。
偏有平儿怕钱不够,又送了一吊来。
凤姐儿道:“不用放在我跟前,也放在老太太的那一处罢。
一齐叫进去倒省事,不用做两次,叫箱子里的钱费事。
”贾母笑的手里的牌撒了一桌子,推着鸳鸯,叫:“快撕他的嘴!”\par
平儿依言放下钱,也笑了一回,方回来。
至院门前遇见贾琏,问他:“太太在那里呢?老爷叫我请过去呢。
”平儿忙笑道:“在老太太跟前呢,站了这半日还没动呢。
趁早儿丢开手罢。
老太太生了半日气,这会子亏二奶奶凑了半日趣儿,才略好了些。
”贾琏道:“我过去只说讨老太太的示下,十四往赖大家去不去,好预备轿子的。
又请了太太,又凑了趣儿,岂不好?”平儿笑道:“依我说,你竟不去罢。
合家子连太太宝玉都有了不是,这会子你又填限去了。
”\zhu{填限:也作“填馅”,代人受过、白白充当牺牲品的意思。
}贾琏道:“已经完了,难道还找补不成?况且与我又无干。
二则老爷亲自吩咐我请太太的,这会子我打发了人去,倘或知道了,正没好气呢,指着这个拿我出气罢。
”说着就走。
平儿见他说得有理,也便跟了过来。
\par
贾琏到了堂屋里,便把脚步放轻了,往里间探头,只见邢夫人站在那里。
凤姐儿眼尖,先瞧见了,使眼色儿不命他进来,又使眼色与邢夫人。
邢夫人不便就走,只得倒了一碗茶来,放在贾母跟前。
贾母一回身,贾琏不防,便没躲伶俐。
贾母便问:“外头是谁?倒像个小子一伸头。
”凤姐儿忙起身说:“我也恍惚看见一个人影儿,让我瞧瞧去。
”一面说,一面起身出来。
贾琏忙进去,陪笑道:“打听老太太十四可出门?好预备轿子。
”贾母道:“既这么样,怎么不进来?又作鬼作神的。
”贾琏陪笑道:“见老太太玩牌,不敢惊动,不过叫媳妇出来问问。
”贾母道:“就忙到这一时,等他家去,你问多少问不得?那一遭儿你这么小心来着!又不知是来作耳报神的,\zhu{耳报神:比喻喜好私下传递消息的人。
}也不知是来作探子的,鬼鬼祟祟的,倒唬了我一跳。
什么好下流种子!你媳妇和我顽牌呢,还有半日的空儿,你家去再和那赵二家的商量治你媳妇去罢!”\ping{贾母的儿子贾赦好色要纳鸳鸯为妾,使得贾母联系到,前不久贾赦的儿子贾琏也是如此,在凤姐生日的时候偷情鲍二家的,儿孙都不争气。
骂贾琏下流种子,其实也是在骂贾琏的父亲贾赦。
}说着,众人都笑了。
鸳鸯笑道:“鲍二家的,老祖宗又拉上赵二家的。
”贾母也笑道:“可是,我那里记得什么抱着背着的,提起这些事来,不由我不生气!我进了这门子作重孙子媳妇起,到如今我也有了重孙子媳妇了,连头带尾五十四年,\ping{这可真是多年媳妇熬成婆,职场升级记。
}
凭着大惊大险千奇百怪的事,也经了些,从没经过这些事。
还不离了我这里呢!”\par
贾琏一声儿不敢说,忙退了出来。
平儿站在窗外悄悄的笑道:“我说着你不听,到底碰在网里了。
”正说着,只见邢夫人也出来,贾琏道:“都是老爷闹的,如今都搬在我和太太身上。
”邢夫人道:“我把你没孝心雷打的下流种子!人家还替老子死呢,白说了几句,你就抱怨了。
你还不好好的呢,这几日生气,仔细他捶你。
”贾琏道:“太太快过去罢,叫我来请了好半日了。
”说着,送他母亲出来过那边去。
\par
邢夫人将方才的话只略说了几句,贾赦无法,又含愧,自此便告病,且不敢见贾母,只打发邢夫人及贾琏每日过去请安。
只得又各处遣人购求寻觅,终究费了八百两银子买了一个十七岁的女孩子来,名唤嫣红,收在屋内。
不在话下。
\par
这里斗了半日牌,吃晚饭才罢。
此一二日间无话。
\par
展眼到了十四日,黑早,\zhu{黑早:尚未天亮仍旧漆黑的早晨。
}赖大的媳妇又进来请。
贾母高兴,便带了王夫人薛姨妈及宝玉姊妹等,到赖大花园中坐了半日。
那花园虽不及大观园,却也十分齐整宽阔,泉石林木,楼阁亭轩,也有好几处惊人骇目的。
外面厅上,薛蟠、贾珍、贾琏、贾蓉并几个近族的,很远的也没来,贾赦也没来。
赖大家内也请了几个现任的官长并几个世家子弟作陪。
因其中有柳湘莲,薛蟠自上次会过一次,已念念不忘。
又打听他最喜串戏,\zhu{串戏:即扮演戏剧。
非职业演员参加演戏也叫串戏,或称“客串”。
}
且串的都是生旦风月戏文,
\zhu{生:戏曲里的一个行当,扮演男子。旦:戏曲里的一个行当,扮演女性。}
不免错会了意,误认他作了风月子弟,正要与他相交,恨没有个引进,这日可巧遇见,竟觉无可不可。
\zhu{无可不可:犹言不知如何是好。
形容情绪激动至极。
形容感激、喜悦的样子。
}且贾珍等也慕他的名,酒盖住了脸,\zhu{盖脸:北京一带的方言,遮盖,掩饰羞容。
酒盖住了脸:酒壮怂人胆,本来没脸不好意思,但是喝酒之后,酒“盖住”了脸,就不觉得不好意思了。
}就求他串了两出戏。
下来,移席和他一处坐着,问长问短,说此说彼。
\par
那柳湘莲原是世家子弟,读书不成,父母早丧,素性爽侠,不拘细事,酷好耍枪舞剑,赌博吃酒,以至眠花卧柳,吹笛弹筝,无所不为。
因他年纪又轻,生得又美,不知他身分的人,却误认作优伶一类。
那赖大之子赖尚荣与他素习交好,故他今日请来作陪。
不想酒后别人犹可,独薛蟠又犯了旧病。
他心中早已不快,得便意欲走开完事,无奈赖尚荣死也不放。
赖尚荣又说:“方才宝二爷又嘱咐我,才一进门虽见了,只是人多不好说话,叫我嘱咐你散的时候别走,他还有话说呢。
你既一定要去,等我叫出他来,你两个见了再走,与我无干。
”说着,便命小厮们到里头找一个老婆子,悄悄告诉“请出宝二爷来。
”那小厮去了没一盏茶时,果见宝玉出来了。
赖尚荣向宝玉笑道:“好叔叔,把他交给你,我张罗人去了。
”说着,一径去了。
\par
宝玉便拉了柳湘莲到厅侧小书房中坐下,问他这几日可到秦钟的坟上去了。
\geng{忽提此人使我堕泪。
近几回不见提此人,自谓不表矣。
乃忽于此处柳湘莲提及,所谓“方以类聚,物以群分”也。
\zhu{方以类聚:义同“物以类聚”,谓同类事物相聚一处。
《易经·系辞上》:“天尊地卑,乾坤定矣。
卑高以陈,贵贱位矣。
动静有常,刚柔断矣。
方以类聚,物以群分,吉凶生矣。
在天成象,在地成形,变化见矣。
”方以类聚中的“方”指万物之性情。
}}湘莲道:“怎么不去?前日我们几个人放鹰去,\zhu{放鹰:这里是打猎的别称。
猎人出猎,常放出驯养的猎鹰捕取猎物。
}离他坟上还有二里,我想今年夏天的雨水勤,恐怕他的坟站不住。
我背着众人,走去瞧了一瞧,果然又动了一点子。
回家来就便弄了几百钱,第三日一早出去,雇了两个人收拾好了。
”宝玉道:“怪道呢,上月我们大观园的池子里头结了莲蓬,我摘了十个,叫茗烟出去到坟上供他去,回来我也问他可被雨冲坏了没有。
他说不但不冲,且比上回又新了些。
我想着,不过是这几个朋友新筑了。
我只恨我天天圈在家里,一点儿做不得主,行动就有人知道,不是这个拦就是那个劝的,能说不能行。
虽然有钱,又不由我使。
”湘莲道:“这个事也用不着你操心,外头有我,你只心里有了就是。
眼前十月一,我已经打点下上坟的花消。
\zhu{花消:花销。
}你知道我一贫如洗,家里是没的积聚,纵有几个钱来,随手就光的,不如趁空儿留下这一分,省得到了跟前扎煞手。
”\zhu{扎煞手:也作“扎撒手”, 扎煞,即双手张开的样子,指遇到难处没有办法。
}宝玉道:“我也正为这个要打发茗烟找你,你又不大在家,知道你天天萍踪浪迹,没个一定的去处。
”湘莲道:“这也不用找我。
这个事不过各尽其道。
眼前我还要出门去走走,外头逛个三年五载再回来。
”宝玉听了,忙问道:“这是为何?”柳湘莲冷笑道:“你不知道我的心事,等到跟前你自然知道。
我如今要别过了。
”宝玉道:“好容易会着,晚上同散岂不好?”湘莲道:“你那令姨表兄还是那样,再坐着未免有事,不如我回避了倒好。
”宝玉想了一想,道:“既是这样,倒是回避他为是。
只是你要果真远行,必须先告诉我一声,千万别悄悄的去了。
”说着便滴下泪来。
柳湘莲道:“自然要辞的。
你只别和别人说就是。
”说着便站起来要走,又道:“你们进去,不必送我。
”一面说,一面出了书房。
\par
刚至大门前,早遇见薛蟠在那里乱嚷乱叫说:“谁放了小柳儿走了!”柳湘莲听了,火星乱迸,恨不得一拳打死,复思酒后挥拳,又碍着赖尚荣的脸面,只得忍了又忍。
薛蟠忽见他走出来,如得了珍宝,忙趔趄着上来一把拉住,笑道:“我的兄弟,你往那里去了?”湘莲道:“走走就来。
”薛蟠笑道:“好兄弟,你一去都没兴了,好歹坐一坐,你就疼我了。
凭你有什么要紧的事,交给哥,你只别忙,有你这个哥,你要做官发财都容易。
”湘莲见他如此不堪,心中又恨又愧,早生一计,便拉他到避人之处,笑道:“你真心和我好,假心和我好呢?”\ping{红楼梦里人要报复都山路十八弯啊,凤姐教训贾瑞是先曲意迎合然后再请君入瓮,这里也是,细究起来都是面对色迷迷的不喜欢的追求者,说到底都是觉得癞蛤蟆怎配吃天鹅肉。
}薛蟠听这话,喜的心痒难挠,乜斜着眼忙笑道:\zhu{乜(音“咩”)斜 :眯着眼睛,斜眼看人。
}“好兄弟,你怎么问起我这话来?我要是假心,立刻死在眼前!”湘莲道:“既如此,这里不便。
等坐一坐,我先走,你随后出来,跟到我下处,咱们替另喝一夜酒。
\zhu{替另:北京方言,另外;重新。}
我那里还有两个绝好的孩子,\zhu{绝好的孩子:这里指男妓,也称“相公”或“相姑”。
}从没出门。
\zhu{从没出门:大概是指这两个孩子都是处男。}
你可连一个跟的人也不用带,到了那里,伏侍的人都是现成的。
”薛蟠听如此说,喜得酒醒了一半,说:“果然如此?”湘莲道:“如何!人拿真心待你,你倒不信了!”薛蟠忙笑道:“我又不是呆子,怎么有个不信的呢!既如此,我又不认得,你先去了,我在那里找你?”湘莲道:“我这下处在北门外头,你可舍得家,城外住一夜去?”薛蟠笑道:“有了你,我还要家做什么!”湘莲道:“既如此,我在北门外头桥上等你。
咱们席上且吃酒去。
你看我走了之后你再走,他们就不留心了。
”薛蟠听了,连忙答应。
于是二人复又入席,饮了一回。
那薛蟠难熬,只拿眼看湘莲,心内越想越乐,左一壶右一壶,并不用人让,自己便吃了又吃,不觉酒已八九分了。
\par
湘莲便起身出来,瞅人不防去了,至门外,命小厮杏奴:“先家去罢,我到城外就来。
”说毕,已跨马直出北门,桥上等候薛蟠。
没顿饭时工夫,只见薛蟠骑着一匹大马,远远的赶了来,张着嘴,瞪着眼,头似拨浪鼓一般不住左右乱瞧。
及至从湘莲马前过去,只顾望远处瞧,不曾留心近处,反踩过去了。
湘莲又是笑,又是恨,便也撒马随后赶来。
薛蟠往前看时,渐渐人烟稀少,便又圈马回来再找,不想一回头见了湘莲,如获奇珍,忙笑道:“我说你是个再不失信的。
”湘莲笑道:“快往前走,仔细人看见跟了来,就不便了。
”说着,先就撒马前去,薛蟠也紧紧跟来。
\par
湘莲见前面人迹已稀,且有一带苇塘,便下马,将马拴在树上,向薛蟠笑道:“你下来,咱们先设个誓,日后要变了心,告诉人去的,便应了誓。
”薛蟠笑道:“这话有理。
”连忙下了马,也拴在树上,便跪下说道:“我要日久变心,告诉人去的,天诛地灭!”一语未了,只听“嘡”的一声,
\zhu{嘡[tāng]:拟声词,模拟敲锣、撞钟等的声音。}
颈后好似铁锤砸下来,只觉得一阵黑,满眼金星乱迸,身不由己,便倒下来。
湘莲走上来瞧瞧,知道他是个笨家,不惯捱打,只使了三分气力,向他脸上拍了几下,登时便开了果子铺。
\zhu{开了果子铺:比喻脸上被打得青紫红肿,像陈列着五颜六色果品的果子铺一般。
}薛蟠先还要挣挫起来,\zhu{挣挫:勉强支撑,也作“挣扎”。
}又被湘莲用脚尖点了两点,仍旧跌倒,口内说道:“原是两家情愿,你不依,只好说,为什么哄出我来打我?”一面说,一面乱骂。
湘莲道:“我把你瞎了眼的,你认认柳大爷是谁!你不说哀求,你还伤我!我打死你也无益,只给你个利害罢。
”说着,便取了马鞭过来,从背至胫,\zhu{胫:音“静”,人的小腿,也泛指腿。
}打了三四十下。
薛蟠酒已醒了大半,觉得疼痛难禁,不禁有“嗳哟”之声。
湘莲冷笑道:“也只如此!我只当你是不怕打的。
”一面说,一面又把薛蟠的左腿拉起来,朝苇中泞泥处拉了几步,滚的满身泥水,又问道:“你可认得我了?”薛蟠不应,只伏着哼哼。
湘莲又掷下鞭子,用拳头向他身上擂了几下。
薛蟠便乱滚乱叫,说:“肋条折了。
我知道你是正经人,因为我错听了旁人的话了。
”湘莲道:“不用拉别人,你只说现在的。
”薛蟠道:“现在没什么说的。
不过你是个正经人,我错了。
”湘莲道:“还要说软些才饶你。
”薛蟠哼哼着道:“好兄弟。
”湘莲便又一拳。
薛蟠“嗳哟”了一声道:“好哥哥。
”湘莲又连两拳。
薛蟠忙“嗳哟”叫道:“好老爷,饶了我这没眼睛的瞎子罢!从今以后我敬你怕你了。
”湘莲道:“你把那水喝两口!”薛蟠一面听了,一面皱眉道:“那水脏得很,怎么喝得下去!”湘莲举拳就打。
薛蟠忙道:“我喝,喝。
”说着,只得俯头向苇根下喝了一口,犹未咽下去,只听“哇”的一声,把方才吃的东西都吐了出来。
湘莲道:“好脏东西,你快吃尽了饶你。
”薛蟠听了,叩头不迭道:“好歹积阴功饶我罢!这至死不能吃的。
”湘莲道:“这样气息,倒薰坏了我。
”说着丢了薛蟠,便牵马认镫去了。
\zhu{镫:挂在马鞍两旁的踏脚。
认镫:脚尖踏进马镫,在这里即“上马”的意思。
}这里薛蟠见他已去,方放下心来,后悔自己不该误认了人。
待要挣挫起来,无奈遍身疼痛难禁。
\par
谁知贾珍等席上忽然不见了他两个,各处寻找不见。
有人说:“恍惚出北门去了。
”薛蟠的小厮们素日是惧他的,他吩咐不许跟去,谁还敢找去?\geng{亦如秦法自误。
\zhu{秦法自误:即“作法自毙”,义同“自作自受”、“咎由自取”,据《史记·卷六八·商君列传》载,战国时期秦孝公任用商鞅治理国政,并接受变法主张,实施一连串改革,使得秦国逐渐走向富强,进而成为战国七雄之一。
由于商鞅执法非常严苛,得罪许多王亲贵戚。
孝公死后,惠公即位。
惠公原本就对变法持反对意见,再加上公子虔等人对商鞅怀恨在心,密告他有谋反意图,于是派遣官吏全力捉拿他到案。
商鞅潜逃到函谷关时,想要到旅舍投宿,旅舍主人不知道他就是商鞅,说:“这是商鞅制定的法令,让身分不明的人留宿旅舍,被查到了会遭到牵连而受罚。
”拒绝让他留宿。
商鞅叹气说:“唉!自己订立的法规,竟然把自己害到这种地步!”后来“作法自毙”这句成语就从这里演变而出,用来比喻自作自受。
典故也可能出自《战国策·燕策三》:“(荆轲)既取图奉之,发图,图穷而匕首见。
因左手拔秦王之袖,右持匕首揕抗之。
未至身,秦王惊,自引而起,绝袖。
拔剑,剑长,掺其室。
时怨急,剑坚,故不可立拔。
荆轲逐秦王,秦王还柱而走。
群臣惊愕,卒起不意,尽失其度。
而秦法,群臣侍殿上者,不得持尺兵。
诸郎中执兵,皆陈殿下,非有诏不得上。
方急时,不及召下兵,以故荆轲逐秦王,而卒惶急无以击轲,而乃以手共搏之。
”由于秦国法令规定群臣上殿不得带兵器,所以秦王面对突如其来的刺客荆轲,方寸大乱,险些丢了性命。
}}后来还是贾珍不放心,命贾蓉带着小厮们寻踪问迹的直找出北门,下桥二里多路,忽见苇坑边薛蟠的马拴在那里。
众人都道:“可好了!有马必有人。
”一齐来至马前,只听苇中有人呻吟。
大家忙走来一看,只见薛蟠衣衫零碎,面目肿破,没头没脸,遍身内外,滚的似个泥猪一般。
贾蓉心内已猜着九分了,忙下马令人搀了出来,笑道:“薛大叔天天调情,今儿调到苇子坑里来了。
必定是龙王爷也爱上你风流,要你招驸马去,你就碰到龙犄角上了。
”薛蟠羞的恨没地缝儿钻不进去,那里爬的上马去?贾蓉只得命人赶到关厢里雇了一乘小轿子,\zhu{关厢:指城门外的大街和附近地区。
}薛蟠坐了,一齐进城。
贾蓉还要抬往赖家去赴席,薛蟠百般央告,又命他不要告诉人,贾蓉方依允了,让他各自回家。
\zhu{各自:各方自己;个人自己,这里是第二个意思。
}贾蓉仍往赖家回复贾珍,并说方才形景。
贾珍也知为湘莲所打,也笑道:“他须得吃个亏才好。
”至晚散了,便来问候。
薛蟠自在卧房将养,\zhu{将养:保养,调养。
}推病不见。
\par
贾母等回来各自归家时,薛姨妈与宝钗见香菱哭得眼睛肿了。
\ping{香菱还挺喜欢薛大傻子啊。
}问其原故,忙赶来瞧薛蟠时,脸上身上虽有伤痕,并未伤筋动骨。
薛姨妈又是心疼,又是发恨,骂一回薛蟠,又骂一回柳湘莲,意欲告诉王夫人,遣人寻拿柳湘莲。
宝钗忙劝道:“这不是什么大事,不过他们一处吃酒,酒后反脸常情。
谁醉了,多挨几下子打,也是有的。
况且咱们家无法无天,也是人所共知的。
妈不过是心疼的缘故。
要出气也容易,等三五天哥哥养好了出的去时,那边珍大爷琏二爷这干人也未必白丢开了,自然备个东道,叫了那个人来,当着众人替哥哥赔不是认罪就是了。
如今妈先当件大事告诉众人,倒显得妈偏心溺爱,纵容他生事招人,今儿偶然吃了一次亏,妈就这样兴师动众,倚着亲戚之势欺压常人。
”薛姨妈听了道:“我的儿,到底是你想的到,我一时气糊涂了。
”宝钗笑道:“这才好呢。
他又不怕妈,又不听人劝,一天纵似一天,吃过两三个亏,他倒罢了。
”\par
薛蟠睡在炕上,痛骂柳湘莲,又命小厮们去拆他的房子,打死他,和他打官司。
薛姨妈禁住小厮们,只说柳湘莲一时酒后放肆,如今酒醒,后悔不及,惧罪逃走了。
薛蟠听见如此说了,要知端的——\par
\qi{总评:自斗牌一节,写贵家长上之尊重,卑幼之侍奉;遭打一节,写薛蟠之呆,湘莲之豪,薛母、宝钗之言,无不逼真。
}
\dai{093}{贾母、薛姨妈、王夫人、凤姐、鸳鸯斗牌,平儿给凤姐送钱}
\dai{094}{呆霸王调情遭苦打}
\sun{p47-1}{赖尚荣搭桥宝玉见湘莲,呆霸王调情遭苦打}{图右侧:赖大家因儿子升迁,大摆宴席庆贺,酒席上,薛蟠见了柳湘莲还误认他是风月子弟,还借酒调戏。
柳湘莲碍着赖尚荣与宝玉的情分,意欲走开完事,赖尚荣说宝玉还有话对你说呢,你两个见了再走。
宝玉出来后,赖尚荣向宝玉笑道:“好叔叔,把他交给你,我张罗人去了。
”说着,一径去了。
图左侧:柳湘莲告辞出来。
此时薛蟠追来纠缠,柳湘莲还假意亲近,将其骗到北门外僻静处,痛打了一顿。
}
