\section*{黛玉之死}

1、荣宁街(夜)\par
下弦月把微弱、朦胧的光洒在地上。一辆马车由远而近,马蹄声、车轮声打破了夜的宁静。 
字幕(叠):第二十四集:黛玉之死 

2、荣国府前角门\par
马车停住。从车上跳下一个人来,急步走到门前,伸手叩门。叩门声不太响,但显得慌乱而急促。 

3、门内\par
一个上夜的男仆提着灯笼走来,一边走,一边打着哈欠:“谁呀?深更半夜的!” 

4、门外\par
叩门的人把嘴凑近门缝:“急事儿,快,开了门再说!” 
门内传出含混不清的、不满的咕哝声,接着是一大串钥匙的叮当作响声。 叩门的人焦急地对着门缝:“快点儿!” 
门内传出开锁声、下门闩声。 叩门的人紧张地朝两边看了看。 
门“吱”的一声开了。 上夜的男仆举起灯笼朝来人脸上照了照,惊讶地:“哟!这不是史府的……” 
来人:“哎!小声点儿!” 男仆不解地看着来人。 
来人悄声:“快进去回话,就说史家来人,有要紧事儿立等着求见!” 
男仆愣了一下,为难地:“这个时候……” 来人:“快着吧!别耽误了事儿!” 
男仆迟疑着转身欲去。 来人:“慢着!……悄悄找个人去回琏二爷或是琏二奶奶,别惊动了旁人。” 
男仆点点头,瞥了一眼门外的马车,转身走去。 来人急步走到车旁,悄声地:“都下来吧,先把东西抬进去。” 
车把式和另外两个人急忙跳下车来,轻手轻脚地卸下车上大大小小的几口箱子。 

5、大观园前角门(晨)\par
榴花绽火,鸟鸣喈喈。紫鹃捧着一个檀木食盒走进角门。 

6、潇湘馆\par
黛玉渐渐止住咳嗽,喘息着把罗帕送给雪雁。雪雁接过罗帕一愣,急忙抖开。 
绣着翠竹的罗帕上湮着星星点点的血斑。雪雁害怕地:“姑娘……”黛玉脸色苍白,无力地倚在榻上。 
紫鹃进门。雪雁急忙迎上:“紫鹃姐姐!”紫鹃忙把食盒放在高脚几上,急步走到榻旁:“姑娘怎么了?” 
雪雁递过罗帕。紫鹃倒吸了一口凉气:“啊?!”紫鹃一把抢过罗帕。 
雪雁带着哭腔:“紫鹃姐姐……”紫鹃急忙用眼睛制止雪雁。背过身去,把罗帕揣在怀里。 
紫鹃:“雪雁,快去倒碗茶来!”雪雁应声急步走出。紫鹃轻声:“姑娘……”黛玉闭目不语。
紫鹃眼圈儿一红,急忙背过脸去,掏出自己的手帕擦擦眼睛。转过脸来强笑笑:
“……今儿一早儿去回老太太的话,老太太听说姑娘见好,今儿能起床了,高兴得什么似的!老太太说,
这两个多月家里头老不顺:姑娘病,老太太自己也病,偏偏琏二奶奶也是好一天坏一天的。这回好了,都顺过来了。”
黛玉慢慢睁开眼睛。紫鹃走到高脚几旁,打开食盒:“……这不,老太太才配了些人参养荣丸,让给姑娘拿来,
还有这些个燕窝,让姑娘千万仔细调养着。”雪雁托着茶盘走来。紫鹃接过茶盅:“姑娘……”黛玉轻轻摇头,看着紫鹃惨然一笑。 

7、王夫人房内\par
形容憔悴的王夫人斜倚着引枕坐在正面炕上。麝月立在一边。李纨携着贾兰进门。李纨:“请太太安。”
贾兰规规矩矩地:“给太太请安。”王夫人绽开一丝笑容。赵姨娘、贾环进门。贾环:“给太太请安。” 
赵姨娘:“给太太请安。”麝月惆怅地看着王夫人。王夫人看看贾环,又看看贾兰,半晌,轻轻吁了一口气,含着泪:“……都念书去吧……”
贾环、贾兰“是”了一声退下。王夫人对李纨、赵姨娘:“你们也去吧……” 
玉钏儿进门,径直走到王夫人面前,压低的声音里透着兴奋:“太太,跟宝二爷去的周瑞大爷回来了!” 
“什么?”王夫人一下子坐直了身子:“快叫他进来!” 

8、潇湘馆\par

廊下,紫鹃看着湮上血斑的罗帕暗暗垂泣。雪雁悄悄走来,轻轻拉了拉紫鹃的衣襟:“紫鹃姐姐,可怎么好呢?赶紧回老太太去吧。” 
紫鹃摇头。雪雁:“再不,就去回太太……”紫鹃摇头。雪雁:“那……” 
紫鹃把罗帕揣进怀里,用手背抹了抹眼睛:“雪雁,你好好照看着姑娘,我去去就来。”说着,走下台阶。 

9、王夫人房内\par
王夫人匆匆看着手中的书信,惊异地:“怎么?宝玉没跟着老爷去?” 
周瑞微躬着身子站在下面:“老爷说,大凡侯门公府一代不如一代,总是安富尊荣的缘故。
咱们是武荫之家,祖宗故事,子孙多有不如。如今良机难得,正好去领略一下汉关烽火之地、海域悲笳之声。所以……” 
王夫人:“这么说,宝玉……去西海沿子了?”周瑞:“是,老爷叩请北静王爷一路上护庇着……”
王夫人一下子失去了控制,眼泪簌簌流下,拍着炕桌道:“这怎么行!这怎么行!老爷真是老糊涂了,万一路上有个闪失,还叫我活不活!”
说着,“呜”地一下哭了起来。周瑞不知所措地站着。麝月含泪看着王夫人,轻轻叹了口气。

10、凤姐院\par
紫鹃匆匆转过粉油大影壁,在院门外停步。院门紧闭。紫鹃眼睛里闪过一丝疑问,犹豫了一下,上前叩门。 
门开了一条缝,伸出一个小丫鬟的脑袋。紫鹃:“有事回二奶奶。”小丫鬟悄声:“等等,我去回一声。” 
门“吱”地一声关上了。紫鹃焦急地在院门外徘徊。院门开了一扇,平儿快步走出:“紫鹃!”
紫鹃急忙用手背擦了擦眼睛。平儿:“怎么?”紫鹃:“有事回二奶奶。” 
平儿笑笑:“这会子不行。”紫鹃:“怎么?”平儿悄声:“出大事儿了!”紫鹃疑问地看着平儿。 
平儿:“史大姑娘家给抄了!”紫鹃吃惊地:“啊?!” 
平儿:“昨儿夜里史家来人,带了些东西来,说要存放在咱们府里。二爷不在家,二奶奶就作主收下了。
今儿一早二爷回来知道了,这会子正为这件事儿生气呢!”紫鹃:“那史大姑娘……”
平儿摇摇头:“不知道,怕是……再也来不了了。”紫鹃的泪水一下涌满了眼眶。 
平儿轻轻叹了口气。紫鹃:“老太太知道了?”平儿摇摇头:“二奶奶吩咐过,怎么也得瞒着老太太。”

11、凤姐房内\par
贾琏气急败坏地:“就算该瞒着老太太,可这么大个事儿,总得讨两位太太的示下吧?
你怎么就这么大胆,随便作主就把东西收下了?!”
凤姐冷笑一声:“二爷发什么虚呢?以往比这大得多的事也不是没经过!那年蓉儿媳妇死,东府里用的那块板,
不是坏了事的义忠亲王的吗?用了就用了,也没见怎么着!头年儿江南甄家抄了,不是也有东西往这儿存吗?” 
贾琏:“先前就是有人告咱们谋反都不怕,可眼下不行了!”凤姐:“眼下怎么了?” 
贾琏:“怎么了?说话就有几档子事儿:前儿里头透出信儿来,说有人弹劾老爷外任亏空,主上脸色就不好看,
亏着有三妹妹和番的功劳,才算罢议了;再有,昨儿平安州节度派心腹人来,说我去了几次平安州,有人知道了,
要弹劾咱们家结交外官呢!如今都察院可都换了忠顺王爷的人,没碴儿还找碴儿呢!你倒好,给人家个辫子抓!”
凤姐:“史家可是老太太的娘家!依你说咱们就撒手不管了?”贾琏:“就是管,也得商量个管法!”
凤姐:“商量?跟谁商量?二爷这阵子帮人家撕掳案子,不是今儿一早儿才回来的吗?大老爷这些事是从来不问的,
二老爷在任上,头些日子我们王家遭事儿,我叔叔刚升了内阁大学士,还没到任就殁了,太太心里正不好受呢……。
我跟谁商量?总不能去找老太太吧?”
贾琏语塞,一跺脚“嗐”了一声,坐在圈椅上。凤姐:“好,我这会子就去回大太太,请她的示下,总行了吧?” 
贾琏看了看凤姐,欲言又止。凤姐朝房门外:“平儿!” 

12、凤姐院门外\par
“哎!”平儿朝门内答应了一声,回过头来对紫鹃:“二奶奶喊我呢,我先去了,有什么事儿,你回头再来吧。”
平儿对紫鹃苦笑笑,转身急步走入门内。紫鹃下意识地摸了摸揣罗帕的地方,欲喊:“平……”门“吱”地一声关上了。 

13、府内通道\par
一辆翠幄青绸车缓缓而来。凤姐端坐车内,抬眼朝车外望去。远处,几个丫鬟、婆子聚在一起正议论着什么,
听见车铃声,急忙散开,垂手侍立。凤姐微微皱眉。 

14、荣国府前角门\par
几个男仆坐在门口的大板凳上,嘁嘁喳喳地悄声议论着:“哎,听说了吗?史家是为什么事儿抄的?” 
“还不是吃了江南甄家的挂络儿!”“不这么简单,听说义忠亲王一档子旧案,如今又翻出来了。” 
“嗐!就凭外任亏空一条,也够上治罪了,还用得着扯上这么多!”
“说来说去呀,还是得罪了人!”“哎,那你说会不会再连带上咱们府里?” 
“这话可就难说了……”“唉,躲还躲不开呢,可昨儿夜里咱们这位这位当家奶奶……”
“哎哎哎……”翠幄青绸车出角门。男仆们连忙站起来。凤姐朝车外瞥了一眼。男仆们用异样的眼光偷觑着凤姐。 

15、贾赦院仪门\par
凤姐下车,款款走入仪门。一群丫头、婆子正围在一起,见凤姐走来,急忙散开,垂手侍立。 

16、邢夫人正房\par
凤姐沿游廊走至正房门前。房内传出马道婆的声音:“……阿弥陀佛!不是五个箱子,是七个箱子!”凤姐一惊,停步。 

17、邢夫人正房内\par
邢夫人脸上露出惊异的神态。“……七个箱子,一点儿不会错!今儿一大早儿,赵姨奶奶巴巴地打发环哥儿到……”
马道婆伸出两个指头:“到她院里看的!”邢夫人:“哦?这么大个事儿,她怎么不跟谁说一声就己做了主了?” 

18、门外\par
凤姐咬了咬牙,抬脚欲进房内。传出马道婆的冷笑声:“阿弥陀佛!不瞒着点儿还行?”凤姐退回原处。 

19、房内\par 
邢夫人:“怎么?”马道婆:“太太想想,史家抄了,存在这儿的东西还指望拿回去?还不是眼下谁收下了,将来就便宜了谁!”

20、门外\par
凤姐摇晃了一下,赶紧扶住门框。马道婆的声音:“听赵姨奶奶说,江南甄家抄家时候存放在这儿的东西,
早弄到她王家去了!”凤姐眼睛红红的,象要滴出血来。一个丫鬟蹭过来,被凤姐狠狠地瞪了回去。

21、房内\par
邢夫人气哼哼地坐着。马道婆越说越起劲:“阿弥陀佛!从来不肯舍个灯油钱!……为几两银子,人命都敢害……”
“嘭”地一声,凤姐推门而入。邢夫人、马道婆吃惊地看着凤姐。凤姐面色惨白,指着马道婆,颤抖着:“你……你……” 

22、薛家后院·宝钗房内\par
宝钗“唿”地站起来:“什么?你说什么?”紫鹃含泪不语,从怀里掏出罗帕递给宝钗。
宝钗一把接过罗帕,惊异地看着上面的星星血斑。紫鹃忍住哭泣,求助地看了看坐在炕上的薛姨妈。 
薛姨妈朝宝钗伸出微微打颤的手。宝钗怔怔地递过罗帕。大滴的泪水落在罗帕上。 
薛姨妈抹着眼睛叹了口气,拾起头来看着紫鹃:“你没去回你们大奶奶?” 
紫鹃摇了摇头:“……大奶奶每日过太太那边内书房,去看着兰哥儿念书……” 
宝钗哽咽着:“妈……”薛姨妈:“……我过去看看。”说着,要站起来。 
宝钗:“妈先别去吧,没的再惊动了老太太、姨妈,倒不好了。……还是我跟着紫鹃去吧。” 
薛姨妈:“……也好。”宝钗用绢子擦了擦眼睛,对紫鹃强笑了笑:“走吧。”
说着,又指了指紫鹃的眼睛:“别让她看出来了。” 
突然,前院传来一阵哭闹声,夹杂着“乒乒乓乓”的摔砸器皿声。三个人都愣住了。 
紧接着一阵“呼通呼通”的脚步声,几个婆子丫鬟惊惶失措地跑进门来,七嘴八舌地:“太太,不好了!” 
“出大事了!”“太太……”薛姨妈强作镇定:“……怎么了?” 
一个婆子哆哆嗦嗦地:“……太太,才刚冯府的大爷派人送急信儿来,说咱们家大爷让……让巡检衙门给……给锁了去了!” 
薛姨妈:“为了……什么事?”婆子:“说是……争一个戏子,把仇都尉的公子给……”薛姨妈、宝钗紧张地看着婆子。 
婆子:“给……打死了!”薛姨妈一下瘫倒在炕上。宝钗:“妈!”众人乱作一团:“太太!”“太太……”
前院传来夏金桂的哭喊声:"……杀千刀的……有人养没人管的……” 

23、大观园前角门\par
紫鹃抽泣着走进角门。守园婆子闻声迎上来,诧异地:“紫鹃姑娘,这是怎么了?”紫鹃停步,呜呜大哭。 
守园婆子不知所措地:“紫鹃姑娘,紫鹃姑娘……”紫鹃渐渐止住哭泣,抬眼望去。 
园内空无人迹,曲港叠石隐约可见,蜂群蝶阵忙碌于红花绿柳之中……。 

24、潇湘馆(夜)\par 
烛火荧荧。紫鹃木然而坐,怔怔地看着微微闪动着的火苗。几声微咳,打破了夜的寂静。 
紫鹃俯过身去,看着躺在床上的黛玉,用轻得几乎听不见的声音:“姑娘……”黛玉的眉尖动了动。 
(幻觉)很远很远的地方有人在喊:“姑娘……”黛玉睁开眼睛。阳光格外明媚,四周开满了各种颜色的奇花异卉。 
仿佛从天边飘来一阵仙乐。紫鹃兴奋地跑来:“姑娘,你看谁来了!”风神飘逸的宝玉笑吟吟地出现在面前! 
黛玉惊喜地迎上前去。宝玉拉起黛玉的一只手。黛玉眼圈儿一红,轻轻抽回手,背过身去。 
宝玉托着一串鹡鸰香珠送到黛玉手里。黛玉朝香珠瞥了一眼,甩手扔在地下。 
宝玉珍重地解开衣领,露出袄襟上挂着的一个荷包。黛玉感动地看着宝玉,破涕为笑。 
宝玉扣好衣领。黛玉把藏在身后的手一下子举到宝玉眼前,手里托着一只精美的香袋。 
宝玉惊喜地接过香袋。晴雯、探春、香菱、芳官、迎春、惜春、宝钗、湘云、紫鹃、袭人等笑着跑来,把宝玉和黛玉团团围住。大家手拉着手围着宝玉和黛玉转起圈来。 
宝玉握着香袋,深情地看着黛玉。突然“轰隆”一声巨响,四周变得漆黑,晴雯、探春等人一个也不见了。
黛玉惊惶地一把抓住宝玉的手。“轰隆”又一声巨响,宝玉也不见了。 
(幻觉完)黛玉“唿”地一下从床上坐起来,惊叫着:“宝玉!宝玉——”紫鹃慌忙搂住黛玉:“姑娘,姑娘!” 
黛玉撒然觉来,余怖未消,惶惶四顾。窗外霹雷电闪,大雨滂沱而下…… 

25、凤姐院外(晨)\par 
宿雨初霁,空气显得格外清新;檐角树梢滴落的水珠,在阳光的照射下粼粼闪烁。
平儿匆匆自院内走出,小心地绕过地上的积水,朝远处走去。 

26、贾母房内\par
王夫人心神不定地坐在一个楠木凳上。贾母斜靠在榻上。王夫人:“……说了这半天话,老太太该歇歇了。” 
贾母笑嘻嘻地:“不妨事的,你再坐一会子,我还有话跟你说呢。” 
鸳鸯站在榻旁,给王夫人使了一个眼色:“今儿老太太高兴,还说想要起来走走呢?” 
贾母:“这不,凤丫头好了;昨儿紫鹃来说,林丫头也见好了;都顺过来了,我也没病了!” 
鸳鸯凑趣地:“过几天,等林姑娘大安了,该摆儿桌酒,庆贺庆贺。” 
贾母:“嗯,你提醒得很是,可是有两个多月没热闹热闹了。……到时候,你们还得提醒着我,想着把湘云丫头也接了来!”
王夫人和鸳鸯交换了一个不安的眼神。贾母一眼瞥见,看看王夫人,又看了看鸳鸯,疑问地:“嗯?” 
鸳鸯眼睛里闪过一丝慌乱。王夫人连忙岔过去:“……老太太刚才……要留我说什么话?” 
贾母抬了抬身子,刚要说话。平儿急步走入房内:“请老太太安!请太太安!我们奶奶请太太赶快过去,有要紧事讨太太的示下。”
王夫人:“哦?”贾母急切地:“什么要紧事?敢是……琏儿又欺负凤丫头了?” 
平儿看了看王夫人:“……”王夫人给平儿使了个眼色。平儿会意,急忙点头,“……对,是琏二爷……” 
贾母对王夫人:“你快去吧。传我的话,琏儿那个下作黄子要是再跟混帐女人算计着治他媳妇,我可就不依他了!”
王夫人起身:“是是,我这就去。”平儿陪笑着:“老太太放心吧,太太去了就没事了。” 
贾母笑笑:“那就去吧。”王夫人、平儿退出房门。贾母收起了笑容,忧心忡忡地看着鸳鸯叹了口气:
“我知道,一定是出大事儿了,都瞒着我,我只当是没看见罢了。” 鸳鸯支吾着:“老太太……”贾母老泪潸然。

27、凤姐房内\par
王夫人跟着平儿进门。凤姐两边太阳穴上贴着“依弗那”,从炕上挣起来:“太太……”
王夫人意外地:“怎么你又……,快别起来吧!”凤姐含着泪:“我不妨事的,请太太坐。”
王夫人坐下,惊疑地看看平儿,又看看凤姐。凤姐:“……才刚孙家派人送讣闻来……”王夫人:“谁家?” 
凤姐:“孙绍祖,孙家。”王夫人紧张地盯着凤姐。凤姐哽咽着:“……说迎春二妹妹……死了……” 
王夫人一下愣住了,半晌,嘴角微微搐动着,泪水渐渐涌满了眼眶,喃喃地:“……我就知道这孩子……活不长……” 
凤姐:“可……怎么回老太太呢?”王夫人轻轻摇头,“啪哒啪哒”地掉着眼泪:“……这事儿……瞒不住……”

28、贾母房内\par
贾母拍着榻沿子,老泪纵横:“……多大个年纪……还不满十七呢!……离了我眼前才几天!”
王夫人站在一边无言唏嘘。贾赦、邢夫人站在另一边嘿然垂首。
贾母指着贾赦、邢夫人:“……我只朝你们要人!……当爹当妈的,不为孩子想想,生把个小命儿给……断送了……”

29、贾母房门外\par
紫鹃揉着眼睛走上台阶。鸳鸯急忙拦住:“哎哎紫鹃!”紫鹃带着哭腔:“我找老太太!” 
鸳鸯着急地:“这会子不行,正……”紫鹃:“我不管!我找老太太!”说着就要进门。
鸳鸯一把拉住:“紫鹃!”从房内传出贾母的声音:“外头是谁?”紫鹃挣开鸳鸯:“老太太!” 

30、贾母房内\par
紫鹃哭着进门,“扑通”跪在地下:“老太太……”贾母抹抹眼睛:“紫鹃?” 
紫鹃“呜”地一声:“老太太,林姑娘……”贾母颤声:“怎么了?” 
紫鹃从怀里掏出湮着血斑的罗帕,膝行至贾母榻前,哀哀地哭着说不出话。 
贾母颤抖着接过罗帕、吃惊地看了又看,泪水止不住簌簌流下。 
贾母心疼地使劲攥着罗帕,另一只手点着众人,泣不成声地:“……都瞒着我吧!都瞒着我吧……。要是林丫头再有个三长两短……我也不活了……”

31、潇湘馆(傍晚)\par 
满天的火烧云给竹梢儿、檐角镶上了金黄色和暗红色的边儿,晚风驱赶着酷暑的燠热,一阵阵的蝉声仿佛接力般地嘶鸣着。
黛玉斜倚在当门而设的湘妃榻上,怔怔地看着手里的一只剪破了的香袋儿。 
(闪回)黛玉赌气拿起剪刀,把一只才做了一半,却十分精巧的香袋儿“嚓”地一下剪破了。
黛玉轻轻抚弄着香袋,眼圈儿一红。(幻觉声音)宝玉:“好妹妹,明儿另替我做个香袋儿吧……” 
(幻觉声音)黛玉:“那要看我高兴不高兴了。”黛玉的眼泪“唰”地涌出了眼眶,两手使劲握着香袋儿,按在胸前。 
紫鹃用小漆茶盘托着药碗走到榻旁,悄悄看了看闭目啜泣的黛玉,轻轻地:“姑娘该吃药了。” 
黛玉摇着头哭出声来。紫鹃小心地把托盘搁在旁边的矮脚几上,忙俯身搂着黛玉的肩膀,一面用手帕轻轻地擦去黛玉面颊上的泪水,
一面娓娓劝说着,“……姑娘快别哭了,才刚听太医说,姑娘本没什么病,都是哭伤了心。姑娘想想,这眼睛里头能有多少泪水,怎么禁得起天天哭的?”
黛玉“呜”地一声,哭得更加伤心。

32、王夫人院(傍晚)\par 
王夫人送着薛姨妈和宝钗从房内走出。王夫人:“……妹妹放宽心吧,先捎个信儿给蟠儿,叫他在里头别胡说,外头有琏儿帮着撕掳……”
王夫人说一句,薛姨妈答应一句。同喜、同贵上前搀着走出穿堂。 

33、潇湘馆(傍晚)\par
落日余辉里,一高一低的蝉声还在不知疲倦地嘶鸣着。
黛玉已经止住了哭泣,正无力地倚在榻上。紫鹃收拾起空药碗刚要离去。 
黛玉:“紫鹃……”紫鹃停步回头。黛玉:“……给我找点儿针线来……” 
紫鹃不解地看着黛玉,刚要发问。雪雁快步走进门来,“姑娘,姨太太和宝姑娘来了!”接着转身打起湘帘。 
薛姨妈和宝钗进门。黛玉欲起身:“姨妈,姐姐……”薛姨妈忙上前把黛玉轻轻按在榻上:“我的儿,快别动。”就势坐在榻边。
雪雁把一个凳子往前挪了挪:“宝姑娘请坐。”说完接过紫鹃手里的托盘退下。
黛玉悄悄地把香袋儿塞在身子下面。宝钗含笑:“听说妹妹……好多了?”黛玉微微一笑,点了点头。 
薛姨妈轻轻摩挲着黛玉消瘦、憔悴的脸庞,慈爱地:“这阵子为你大哥哥的事,家里头乱糟糟的,没得空来看你。”
黛玉看着薛姨妈,眼圈儿一红。宝钗默默地观察着黛玉的气色。雪雁端上茶来。薛姨妈、宝钗接过茶盅,放在旁边的矮脚几上。 
紫鹃拿来两柄精致的两面绣的纨扇,递给薛姨妈和宝钗。宝钗接过执纨扇细看了看,站起身来笑嘻嘻地:
“紫鹃,我托你找的花样子呢?”说着悄悄给紫鹃使了个眼色。紫鹃会意,笑了笑:“姑娘自己过来挑挑吧。”说着朝屋门走去。 
宝钗就势跟着紫鹃走出屋门。 

34、院内\par
余霞散绮。紫鹃引着宝钗沿曲折游廊走来。紫鹃边走边回头看,打量着说话房内听不见了,才停下脚步。 
宝钗悄声:“太医怎么说?”紫鹃眼圈儿一红:“太医说……脉象不好,说病是从忧伤思虑上起的,别的还在其次,最要紧的是忌哭。
若能够几百的事都不动心,就好了。可……姑娘想想,她什么时候断过眼泪的?更何况……”
说着低下头去,哽咽着:“……打从……三姑娘走,哪天不哭几回?再这么下去,岂不是要……” 
宝钗含着泪忙制止紫鹃:“可别胡思乱想的!怎么想个法子治病是正经。” 
紫鹃拾起头来,直直地看着宝钗,半晌,徐徐开口:“治病治根儿。姑娘是个明白人,这些年了,不会不知道她的那块心病。
要想除了她这病根儿也不难。只是……都在姨太太身上……”宝钗:“这话……怎么讲?” 
紫鹃:“……姑娘只想想那年姨太太在我们这里曾说起的……‘四角俱全’的话……”宝钗怔怔地看着紫鹃。 
(闪回)薛姨妈一边摩挲着依偎在自己怀里的黛玉,一边笑吟吟地看着宝钗:“……我想着,你宝兄弟老太太那样疼他,
他又生得那样儿,若要外头说去,断不中意。不如竞把你林妹妹定给他,岂不四角俱全?” 
紫鹃笑着跑来:“姨太太既有这主意,为什么不和太太说去?”薛姨妈呵呵笑着:“我一出这主意,老太太必定欢喜的。” 
(闪回完)“宝姑娘!”雪雁沿曲折游廓匆匆走来:“姨太太要回去了,说黑了天路不好走。” 
宝钗看了看紫鹃,欲言又止。紫鹃期待地看着宝钗。宝钗慢慢转身,沿游廊走去。淡淡的暮色悄悄地渗入了庭院…… 

35、潇湘馆(夜)\par 
细密的雨丝打在院内的芭蕉叶上,发出均匀的“沙沙沙”声。卧房内,黛玉半坐半靠在榻上,手里拿着那只剪破了的香袋,用针线缝着破口。 
紫鹃坐在一边,怔怔地看着黛玉。黛玉缝好一针,停下来看着香袋出了一会儿神,不知不觉地又垂下泪来…… 

36、潇湘馆(秋)\par 
半空里,一行大雁冉冉南飞。紫鹃收回惆怅的目光,慢慢走进房内。黛玉倚在榻上,托着缝好一半的香袋暗暗啜泣…… 

37、潇湘馆(冬)\par
窗外银镶素嵌、瑞雪飘飘。地下火盆里炭火熊熊。黛玉靠在床上,默默地接过紫鹃分好的各色丝线,在香袋上比着颜色。 几滴泪水落在香袋上…… 

38、潇湘馆(春)\par
香袋被剪破的几处巧妙地绣上了节节翠竹,泪水滴落在翠竹上,无声无息地湮开了,又消失了。
病体支离的黛玉止住了咳嗽,喘息着把目光从香袋上移开,落在刚刚进门的紫鹃手上。
紫鹃匆匆把擎在手上的几枝桃花插进花瓶,忙走过来给黛玉轻轻拭去额上的虚汗和面颊上的泪水。 
黛玉喘息甫定,无力地靠在紫鹃身上,哽咽着:“我梦见……宝玉……出事了……”紫鹃劝慰着:“姑娘别乱猜疑,梦哪有灵验的?” 
黛玉摇着头抽噎着:“……翻了船……掉在水里……昏天黑地…。”紫鹃强忍着眼泪,把黛玉楼在怀里:“姑娘!姑娘……” 
黛玉抑制不住地呜咽着。紫鹃强笑笑,哄孩子似地摩挲着黛玉:“那……我给姑娘圆圆梦。……姑娘是快天亮才睡着的,后半夜是反梦!
……人人都说梦见水是吉利事!……昏天黑地……地不就是岸么?宝二爷不是坐船走的么?船靠了岸……这可是个吉兆,保不定是……
二爷要回来了!”紫鹃说着兴奋起来:“对了!‘昏天’!姑娘这个梦敢是应在……婚事上头?……姑娘快别哭了,这梦可是个……”
黛玉剧烈地咳嗽起来。紫鹃忙把床头叠着的一方罗帕拿给黛玉。黛玉用罗帕捂着嘴,咳嗽了一阵,闭目喘息着,把罗帕移开。 
紫鹃下意识地连忙接过罗帕,背过脸去偷偷瞥了一眼,不由得脸上一寒。罗帕上又湮上了星星点点的血斑。 
紫鹃赶紧把罗帕揣起来,泪水“唰”地涌满了眼眶,强掩饰着:“……这梦可是个……好……” 

39、王夫人院\par

贾琏匆匆走过穿堂,沿着回廊走进王夫人房内。 

40、王夫人房内\par 

贾琏兴奋地:“喜事儿,太太!”愁云满面的王夫人心不在焉地看着贾琏。贾琏上前一步:“宝兄弟要回来了!” 
王夫人一下子坐了起来:“什么?”贾琏:“今儿去部里办事,听说刚接着北静王爷三百里邮传,说两位王爷奉旨查边,余下的人冬底就已经回返了!” 
王夫人猝然听闻,惊喜不可言状,半晌,吩咐贾琏:“……快去回老太太,让老太太高兴高兴!”贾琏嘴里答应着,脚下没动地方。 
王夫人:“快去!”贾琏犹豫着:“……太太,还有一件事……”王夫人:“说!”贾琏:“……薛家大兄弟的案子有些麻烦,昨儿已经把挂在户部的职名褫革了去了……” 
王夫人吃惊地看着贾琏。贾琏:“……忠顺府插了一手,怕要……坏事。”王夫人:“……姨太太……得着信儿了么?” 

41、薛家后院正房内\par
薛姨妈哽咽着:“……我已经……知道了……”王夫人含着泪,爱莫能助地看着薛姨妈。玉钏儿和麝月立在一边,惊异地四面环顾。 
房内一片狼藉,到处是碎瓷片,桌椅板凳东倒西歪,撕破了的窗纱门帘摊在地上……玉钏儿给同喜使了一个眼色,同喜会意,二人悄悄走出房门。 

42、院内\par 
玉钏儿回头看了看房门:“这是怎么了?” 同喜悄声:“我们那位搅家精大奶奶,一听说大爷的事不好,就闹起来了。
再加上宝蟾那个坏蹄子百般撩拨,今儿一早儿大闹了一场,回娘家去了。这不,临走临走,又砸了个乱七八糟!” 

43、房内\par 
薛姨妈抹着眼泪:“……这个孽障!到底作出祸来!饶着坑了自己不算,还带累了他妹子……” 
王夫人:“怎么?宝丫头……”薛姨妈:“这不,把祖宗的荫封给丢了,宝丫头待选的事,不也就……”说着,又哽咽起来。 
王夫人含着泪叹了口气。麝月若有所思地看着王夫人。 

44、潇湘馆\par
晨光熹微。阵阵叩门声传来。紫鹃嘴里咬着一根簪子,双手结着头发闻声走出。
院内,草色苔痕一片新绿,竹叶上的宿露在晨曦中闪着幽光。 
紫鹃走到院门前站住,用簪子随便簪住发髻,随后打开院门。 
鸳鸯提着一个用丝绵垫儿包着的砂罐儿站在门外。 
紫鹃惊喜地:“鸳鸯姐姐!怎么这么早来了?” 
鸳鸯进门,边走边拉起紫鹃的一只手,笑嘻嘻地:“好个懒丫头,睡到这会子才起!怪不得手皮子养得这么细!
朱子治家格言上说:‘黎明即起,洒扫庭除’,敢情你什么都……” 
紫鹃抽出手来轻轻打了鸳鸯一下:“小声点儿,林姑娘还没起来呢!” 
鸳鸯伸了伸舌头,悄悄走进房门,顺手把砂罐放在高脚几上。 
紫鹃看了看砂罐,疑问地:“这是……?” 
鸳鸯:“参汤,老太太吃的,让送过来一半儿给林姑娘,早起空肚子吃最好。要不,我这么一大早儿跑来!” 
紫鹃:“哎呀!林姑娘身子这么弱,前儿太医还说不易多用参呢,怕一下子补过了。眼下吃着参茸丸,再吃参汤,要是把火气吃上来,那可就……” 
鸳鸯抿嘴儿一笑:“你知道什么!这是西洋进贡的白参,不比咱们常日里用的参。这种参是凉性的,最能养阴清火、生津,大暑天里吃也不怕的。宫里还用这个治阴虚上火、咳嗽、咯血呢!” 
紫鹃:“哎呀,我们姑娘吃这个再合适不过了!……我去回老太太,索性多要几棵来,慢慢煎着吃。” 
鸳鸯冷笑一声:“多要几棵?说得轻巧!老太太也只有这么几棵,也就是林姑娘吧,老太太还能从嘴里省出几口来给她吃,别人,哼!” 
紫鹃叹了口气。鸳鸯:“哎,有件事儿,告诉你,你可别乱说。” 
紫鹃:“什么事儿?”鸳鸯悄悄走到内室门口,掀起帘子,朝里面看了看。 
黛玉安详地合目而卧。鸳鸯压低了声音:“老太太要给宝玉议婚了!” 
紫鹃一惊:“什么?” 鸳鸯:“昨天巴巴地找了太太去,就是为这个事儿。” 
紫鹃:“太太的意思是……?” 鸳鸯:“还不得听老太太的!” 
紫鹃:“那老太太……?” 鸳鸯:“老太太心里早有了。” 
紫鹃急切地:“谁?” 鸳鸯抿嘴儿一笑。 
紫鹃央求着,“好姐姐,快说吧,急死人了。” 
鸳鸯故作惊讶地:“哎?宝玉的事儿,麝月急嘛还差不多,你急哪门子的?” 
紫鹃一把抓住鸳鸯,作出要往胳肢窝儿里捣的样子,咬着牙:“我把你这个坏透了的蹄子……,说不说?” 
鸳鸯屏住气笑着:“别别……你放手,我说。” 紫鹃松开手。 
鸳鸯整了整衣服:“这事老太太可是只跟我说过,你得发誓不说出去才行。” 
紫鹃:“快说吧。” 鸳鸯朝内室瞥了一眼,俯在紫鹃耳边说了句什么。 
紫鹃眼睛一亮:“真的?” 鸳鸯笑嘻嘻地点了点头。 
紫鹃含泪看着内室的门帘,喃喃地:“阿弥陀佛,总算是……”忽然又若有所觉地看着鸳鸯:“怎么……这会子议起这件事来了?” 
鸳鸯:“是因为……宝二爷说话就回来。” 紫鹃:“真的?!” 
内室里“啪”地一声响。 紫鹃急忙擦擦眼睛掀开门帘看了看。 
黛玉向内侧卧,枕边的一本书落在地上。 紫鹃轻轻放下门帘,回头向鸳鸯黯然一笑。 

45、王夫人院内(晚)\par 
麝月心事重重地在廊下徘徊,不时地看看正房门口。 
帘子一动,玉钏儿挑着灯笼出门,朝穿堂外走去。 
王夫人在房内来回走动着,身影映在纸窗上。 
麝月俯思片刻,慢慢走向房门。 

46、房内\par 
王夫人双眉紧锁,愀然踱步。 麝月悄悄进门。 
王夫人瞥了麝月一眼,停下脚步。 麝月迟疑地:“……太太,有句话……不知当说不当说……” 
王夫人注视着麝月:“……说吧。” 麝月:“……二爷说话就回来了,虽说出了一趟远门,先前的脾性,怕也难大改。……如今可是大了,说话行事,都不拿着当小爷待了。我们做下人的,万一眼错不见,有个提醒不到的,岂不辜负了太太的苦心?” 
王夫人暗暗点头。 麝月觑着王夫人的脸色:“……虽说林姑娘、宝姑娘都是知礼守训的,现又不在一处住着了,自然……生不出什么闲话来,可……” 
一道阴影掠过王夫人的眉梢。麝月:“……二爷房里新补送来的几个小丫头子,看上去本份,谁知道究竞怎么样呢?……不如赶早定一门亲事,二爷也就安心了。” 
王夫人:“我的儿,可不是正为着这件事犯愁呢?” 麝月:“那太太何不找老太太说说?快着定了,大家就都不用老悬着心了。” 
王夫人:“昨儿……老太太已经提了。” 麝月紧张地:“老太太……怎么说?” 
王夫人惆怅地看了麝月一眼,叹了口气:“……老太太说,要个亲上做亲的,打小一块儿厮混过来的,脾气性格合得来的,模样周正的。……你想想,这还有谁?” 
麝月:“……不说……宝姑娘待选才人赞善的事……不成了么?” 
王夫人似乎被提醒了,直直地看着麝月,半晌,叹了口气:“可老太太……” 
麝月试探地:“要不……让琏二奶奶帮着太太再跟老太太说说?” 
“她?”王夫人苦笑着摇摇头:“怕和我想的……” 
麝月陪笑着:“太太别着急,我想……二爷的婚事非别人可比的。譬如先前的二姑娘、以后的环儿三爷,只要讨了老太太的示下就行了。二爷的事,恐怕还得请宫里娘娘的旨意吧?” 
王夫人眼睛一亮。 麝月思忖着:“若是娘娘……,不过,娘娘的旨意,得太太亲自进宫去问问才好。” 
王夫人怔怔地看着麝月。 (闪回)袭人含泪进言:“只是……好歹留着麝月,这些年,我留心看着,只有她……能替我……长长远远地伺候二爷……” 
王夫人一把拉住麝月的手,哽咽着:“我的儿,事到如今,我也只有你这么个说心腹话的人了。”麝月眼圈儿一红。 

47、贾母院(晚)\par 
玉钏儿正房内走出,接过一个婆子手里的灯笼:“谢谢嬷嬷。” 

48、贾母院穿堂\par 
两个上夜的婆子提着灯笼转过墙角,边走边悄声闲话: 
“……我要是太太,我就要薛姑娘,断不能要林姑娘!” 
“就是!好好的也得一天哭几回,何况又是个病秧子,谁知道哪天就……” 
玉钏儿刚好走出穿堂,听见婆子的话,一愣,立刻放轻了脚步跟在后面。 
“打来到这府里,见了下人就从没有答理过,哪象薛姑娘那么和气!”
“那你说为什么琏二奶奶总撺掇老太太定林姑娘呢?”“你怎么也知道?” 
“你寻思只你有耳报神,我就没有?”两个婆子嘻嘻哈哈地笑起来。 
玉钏儿屏声敛气地跟在后面。 “这还不明白?琏二奶奶正经是大太太那边儿的,却在这边儿当着家。你想想,若是这边儿娶了个强似她的,那她这个家还能当得长远?你当她真向着林姑娘呢!” 
“哎哟我的娘!只怕连老太太都让她给蒙了!” 
“那可不是!……实诚人倒是有一个,硬在老太太面前替林姑娘保媒呢!” 
“你说的是……”“薛家姨太太呗!……只可惜这个媒保得早了些,把个薛姑娘给耽误了。” 
“这话怎么讲?”“嗐!这不没过几天,薛姑娘待选的事就算完了么!要早知道不成,这倒是一门好亲呢!” 
玉钏儿只顾偷听,不提防脚下一绊,不由地“哎哟”了一声。两个婆子慌忙回头:“谁?” 

49、大观园·沁芳桥(晨)\par 
芳草萋萋,飞絮濛濛。 宝钗带着莺儿缓缓下桥。 

50、潇湘馆院外\par 
紫鹃沿着墙外小路走向院门,忽见宝钗、莺儿远远走来,忙迎上去。 
紫鹃:“姑娘好!怎么这会子得空来了?” 宝钗含笑:“你这是从哪儿回来?” 
紫鹃:“才大奶奶来看我们姑娘,我送了几步。” 
“林妹妹怎么样了?”宝钗问着话,没有停步,不时轻轻拨开路旁伸过来的新竹嫩枝。 
紫鹃不无欣喜地:“这儿日忽然好起来了,也能睡会子觉了,有时候还能起来走走呢!只是……咳嗽的遍数反多了些。” 
宝钗有些意外地:“哦?” 紫鹃:“说起来,还得谢谢姑娘和姨太太呢,若不是姨太太保媒,林姑娘的病还能有好的日子?一定是姑娘催着姨太太……” 
宝钗正色打断紫鹃的话:“别胡乱说了!……林妹妹的终身,自有老太太做主。我一个女孩儿家,是该管这些事的么?……林妹妹也断不会因为这个病的……” 
紫鹃惊愕地看着宝钗。 宝钗低下头去:“……你要真为林妹妹好,就别老引着她想这些……” 

51、黛玉房内 \par
一张雪浪笺上疏密合度地写满了蝇头小楷,娇喘微微的黛玉正手搦湘管,在后面落下最后一行字迹:右录药余偶得十独吟十首。 
雪雁捧着一只注满清水的笔洗走来,轻轻放在案上。黛玉才搁下笔,就连连咳嗽起来。 
宝钗等进门。 紫鹃忙走过去,边给黛玉摩挲着后背,边轻声回话:“宝姑娘来了。” 
黛玉渐渐止住咳嗽,用罗帕拭着咳嗽出来的泪水,要站起来。 
宝钗过来轻轻按住:“快别起来!”接着瞥了一眼案上的雪浪笺,不无埋怨地:“看你,才好了些,又弄这个!等大安了,有多少诗不能做?非得这会子伤这些个精神!” 
黛玉赧然一笑。 宝钗随手拿起雪浪笺来,匆匆浏览一过,轻轻叹了口气:“颦儿颦儿!你非要把个心呕出来才算完么?” 
黛玉的眼睛里又含上了一汪清泪。 

52、王夫人院\par
贾琏神色惶惶地沿回廊走来,后面跟着风尘仆仆、面带伤痕的李贵。 
周瑞家的忙迎上来:“二爷……”贾琏:“太太在么?” 
周瑞家的惊异地看了一眼李贵:“今儿不是逢二么?太太才刚大妆了,要进宫请侯呢。” 
贾琏、李贵急忙朝正房走去。 

53、王夫人房内 \par
大妆的王夫人从内室走出。贾琏、李贵进门。贾琏凄惶地:“太太……” 
王夫人惊愕地看着李贵:“……李贵?” 
李贵“扑通”跪在地下,涕泗交流,头连连碰着地,迸出一声:“太太——” 

54、沁芳溪畔\par 
柳丝摇漾,落英缤纷,碧清的溪水潺湲逝去。 
宝钗、紫鹃一左一右,搀扶着羸弱的黛玉缓慢地走着。 
莺儿攥着一把五颜六色的不知名的野花,紧跟在三个人的后面。 
宝钗关切地:“行吗?还是回去吧。” 
黛玉莞尔一笑:“姐姐不是说,让我多出来走走吗?” 
宝钗:“病才好些了,别累着,以后日子多着呢。” 
黛玉:“……也好,我在这里歇会子,紫鹃替我送送。” 
紫鹃迟疑着:“姑娘一个人……”黛玉:“我正想一个人坐坐,去吧,不妨事的。” 
紫鹃:“那……我就来。” 宝钗、紫鹃搀着黛玉坐在一块石头上。 

55、花冢\par
微微隆起的花冢上嫩草如茵。
黛玉摇摇晃晃地走来,倚在一块山石上,怔怔地望着冢。 
(闪回)落红阵阵。黛玉提着绢袋、宝玉兜着衣襟一前一后嬉笑着跑来。 
花冢里的绢袋被不断落入的花瓣埋住…… (闪回完) 
花冢青青。黛玉慢慢坐下,泪水溢出眼眶。 
远处,小红慌慌张张地走来。黛玉下意识地往树丛后面藏了藏。 
紫鹃从小山坡上跑下来,拦住小红:“小红!上哪儿去?” 
小红吃了一惊:“哦,紫鹃!”接着一把拉住紫鹃,紧张地:“前头乱了套了,老太太、太太都昏死过去了……” 
紫鹃惊愕地:“什么?”黛玉在树丛后面睁大了眼睛侧耳倾听。 
小红:“……二奶奶打发我赶紧找大奶奶去商量怎么办呢!” 紫鹃:“出了什么事?” 
“跟着宝二爷去的李贵回来了……”小红稍顿了一下,紧张地嘱咐紫鹃:“可千万不敢让林姑娘知道了!” 
黛玉打了个哆嗦。紫鹃急切地:“你快说吧!” 
小红:“李贵说,他们的船快进内河的时候,忽然遇了海盗,一炮先把桅杆打断了,船动不了,海盗涌上船来,见人杀人,见货抢货,李贵头上挨了一锤,昏过去了……” 
紫鹃害怕地抓住小红:“那宝二爷……?”小红:“等李贵醒过来,已经打劫空了。船上、水里到处是死人……” 
黛玉两眼发直,双手死死攥住身旁的嫩草。小红哽咽着:“……只找到了宝二爷的……那块玉……” 
黛玉眼前一黑,昏倒在花冢旁。青青芳草,在微风中抖动。 

56、园中路上\par 
天空里,回荡着紫鹃空旷的呼喊声:“林姑娘——林姑娘——”紫鹃满面泪痕,踉跄着走来。 

57、府内甬道\par
哭声震天。男女仆人纷攘杂沓地来来往往。 

58、园中山坡\par
空旷的呼喊声……紫鹃吃力地攀援…… 

59、王夫人院\par
哭声震天。凤姐、李纨带着一群仆妇涌进正房。 

60、园中竹桥\par
空旷的呼喊声…… 紫鹃摇晃着走过…… 

61、贾母院\par 
哭声震天。凤姐、李纨带着一群仆妇踊进正房。 

62、园中树丛\par
空旷的呼喊声……紫鹃艰难地穿行…… 

63、荣禧堂\par 
哭声震天。贾赦、贾珍、贾蓉等一群人涌进。贾琏慌忙迎上。 

64、园中正殿\par
空旷的呼喊声……紫鹃呆呆地站着。 
巍峨的琳宫俯视着紫鹃,金辉兽面、彩焕螭头傲然峭立,匾额上“顾恩思义”四个大字熠熠闪光。 紫鹃哽咽着:“林姑娘……” 

65、怡红院门外\par 
一把大大的铜锁挂在院门上。黛玉面色雪白、目光炯炯,摇摇晃晃地走来。 
院门外满径荒草。黛玉含笑上前叩门:“宝玉……宝玉……开门……宝玉……” 
院门紧锁,院内寂无人声。黛玉喃喃地:“病了?……不会的……怎么不开门……睡下了?……” 
黛玉使劲拍门,高声:“晴雯……晴雯……是我……还不开门么?……晴雯……” 
紫鹃远远走来,见状停步,掩面恸哭。黛玉全然不觉,使劲拍着门上的铜锁,高声叫着。“宝玉……晴雯……晴雯……宝玉……” 
紫鹃哭着跑过来:“姑娘……”黛玉叹了口气,喃喃地:“想是……都不在家……” 
紫鹃一把拉住黛玉:“姑娘……”黛玉直直地看着紫鹃,诧异地:“怎么?你来了……哭什么?谁欺负你了?” 
紫鹃:“姑娘……”黛玉微微一笑:“我来找宝玉……他让晴雯给我送去两块旧帕子……可……都不在……” 
紫鹃拉着黛玉:“姑娘,咱们回去吧。”黛玉若有所思地:“回去?……是该回去了,……回去……”黛玉说着,径自撤步离去。 

66、园中路上\par 
黛玉步履轻捷,飞快地走着。紫鹃紧紧跟在后面。 

67、沁芳桥\par 
黛玉翩若惊鸿,飘然而下。紫鹃紧紧跟在后面。 

68、潇湘馆\par 
黛玉快步走到院门口推门而入,紫鹃跟上搀扶。紫鹃抹了一把眼泪:“总算到家了。” 
黛玉停步,看看紫鹃,又看看院内,身子往下一沉,张嘴喷出一口鲜血。 

69、黛玉房内(夜)\par 
夜,静得出奇。一点昏暗的烛光,仿佛是凝固在烛插上的荧火。黛玉昏睡在床榻上,枕边、被头、领口、袖沿……血渍斑斑。 
紫鹃失神地坐在床边,手里紧攥着一块带血的罗帕。 

70、黛玉房内(晨)\par 
华膏耗尽,蜡炬成灰。房内仍是死一般的寂静。黛玉昏睡在床上。 
(幻觉)宝玉、黛玉执手相看泪眼,黛玉的心声:“……你只管你,你好我自好……你失……我自失……” 
(闪回)少女们清脆的笑声里,黛玉伸手从签筒里掣出一根象牙花名签子。默默地念着:“莫怨东风当自嗟……” 
黛玉口眼微动,极度虚弱地吁出一口气:“……莫怨东风……当自嗟……”紫鹃失神地看着黛玉。 
雪雁搀着黛玉的乳母王嬷嬷颤巍巍地走进来。王嬷嬷唏嘘着:“……紫鹃姑娘,……预备着……移床吧……” 
紫鹃仿佛没有听见,死死地盯着黛玉:“……姑娘醒过来了!……姑娘醒过来了!姑娘!姑娘……” 

71、院内\par
一个婆子持帚打扫满院落花。雪雁走过来,轻轻拉拉婆子的衣襟。婆子抬头看了看雪雁。 
雪雁悄声:“嬷嬷,都说姑娘没指望了,可……怎么忽然醒过来了不说,精神也好了,还一定要换到榻上去躺着,是不是……要好了?” 
婆子歪着头朝房内看了看,叹了口气,又低头继续扫着落花。 

72、房内\par
几上放着一叠诗稿,最上面是写着《十独吟》的那张雪浪笺。黛玉躺在傍几而设的湘妃榻上,一只手软软地搭在胸前,手里捏着那只补缀精巧的香袋。 
紫鹃托着两块写满字迹的旧帕子走到榻前,轻轻地抖开:“姑娘看看,是这个么?”黛玉点了点头,另一只手无力地抬了抬。 
紫鹃忙把诗帕递在黛玉手里。黛玉颤抖着接过诗帕。诗帕上墨迹斑斑。 
(闪回)晴雯笑嘻嘻地:“二爷让我送手帕子来给姑娘。” 
(闪回)晴雯笑嘻嘻地:“不是新的,就是家常旧的。”
黛玉双目紧闭,泪水涌出。(闪回)黛玉在帕子上走笔题诗。 
黛玉攥紧诗帕,泪水簌簌流下。雪雁用托盘托着药碗走来,悄声:“姑娘……” 
黛玉摇了摇头,睁开眼睛,嘴唇动了动。紫鹃凑近:“姑娘要什么?” 
黛玉的嘴里发出微弱的声音:“……笼一盆火来……”紫鹃:“姑娘要是觉着冷,就多盖一件,还是别笼火吧,那炭气怕姑娘受不了。” 
黛玉摇摇头:“……快去……”紫鹃直起身子吩咐雪雁:“去……笼火……”雪雁忙转身退下。 
黛玉示意紫鹃坐在自己身旁。紫鹃轻轻坐在榻沿上。黛玉拉起紫鹃的一只手,放在自己的胸口上,轻轻地摩挲着。 
紫鹃:“姑娘……”黛玉深情地看着紫鹃,凄然一笑:“……好妹妹……你我姐妹一场……眼看就要……。……原想着……能够同始同终的,可……。你白替我……操了这些年的心了……” 
紫鹃看着黛玉,哽咽着说不出话。黛玉吃力地抬起一只手,轻轻抹去紫鹃腮边的泪水:“……好妹妹……我化成了灰……也忘不了你的……” 
紫鹃失声恸哭。黛玉:“……好妹妹……别哭……我还有事没了呢,……哪能立刻就……” 
紫鹃强止住哭泣,抬起头来看着黛玉。雪雁端着火盆进门,轻轻放在地下。 
黛玉歪过头去看了看火盆,示意紫鹃挪近一些。紫鹃走过去,挪了挪火盆。 
黛玉摇摇头,伸手指指榻旁。紫鹃犹豫着把火盆挪过来。黛玉挣扎着要坐起来。 
紫鹃忙走到另一边,轻轻扶起黛玉,用自己的身子把黛玉戗住。黛玉托起诗帕,死命看了一眼,泪水“唰”地涌出,喃喃地:“……宝玉……等等我……”一撒手,把诗帕扔进火里。 
雪雁惊叫了一声,忙跑过来。紫鹃一下搂紧了黛玉:“姑娘!” 
雪雁从火里一把抓出燃烧着的诗帕,丢在地上,三脚两脚踩灭了,蹲下一看,已经成了一堆黑灰。雪雁“哇”地一声哭起来。 
黛玉微喘着,看着雪雁,轻轻摇了摇头。紫鹃抽泣着,搂着黛玉的双手慢慢松开:“雪雁……去吧……” 
雪雁抹着眼泪慢慢退出房门。黛玉颤抖着抓起香袋,哽咽着:“……宝玉,……等等我……”撒手丢进火盆。 
一页页诗稿被丢进火盆。火苗伴着黑烟腾起…… 

73、旷野(黄昏)\par 
斜阳古道,阴霾渐起。茗烟纵马狂奔而来。 

74、城门(晚)\par 
风灯闪烁,星雨飘零。茗烟打马跑进城门。 

75、荣国府·荣禧堂(晚)\par
烛火通明。贾赦、贾珍、贾琏、贾蓉等相对嘿然。忽然一阵“呼通呼通”的脚步声。 
几个男仆拥着茗烟闯进房门,一叠连声地喊着:“大老爷!”“大爷!”“二爷!” 
贾赦等人吃了一惊。茗烟“扑通”跪在中间。贾赦等人愕然瞠目,不约而同地:“茗烟?!” 
茗烟:“给大老爷、大爷、二爷、小爷们请安!宝二爷回来了!打发小的先回来报信儿!” 
“什么?!”几个人倏地一齐站了起来。贾琏一步跨过去,劈胸抓住茗烟,一把拎了起来:“……不是说……宝兄弟……” 
茗烟结结巴巴地:“我们那天……正等死……不想……被人救了……”贾赦、贾珍、贾蓉等人不约而同地围上来,惊问:“谁?!” 
茗烟:“打头的象是……”贾琏猛一跺脚:“快着回老太太、太太去吧!救命要紧!等会子再问不迟!” 
众人如梦方醒,“呼啦”一下拥出门去。 

76、潇湘馆院外(雨夜)\par
一声撕心裂肺的哭喊声蓦地传出:“姑娘……你倒是等等……”

77、黛玉房内\par
黑暗中,紫鹃的身影跪在榻旁,拼命地摇晃着软软地横在榻上的黛玉:“……你听见了么姑娘,……宝玉……回来了!……”

78、大观园正门\par 
阳光灿烂。园门缓缓开启。消瘦、憔悴的宝玉毫无表情地出现在门外。 

79、园中路上\par
一阵阵红雨般的落花,自空中飘洒而下。宝玉毫无表情地走着…… 

80、潇湘馆\par 
依旧是凤尾森森、龙吟细细,仿佛一切都不曾在这里发生过。宝玉象往常一样地进门。 
架上鹦鹉:“紫鹃,快打帘子,宝玉来了!紫鹃,快打帘子,宝玉来了……” 
紫鹃从房内走出,默默地打起帘子。宝玉进门。 

81、黛玉房内\par 
梁间燕子细语呢喃。空空的榻……宝玉缓缓走进内室。空空的床…… 
宝玉轻轻地坐在床榻边,良久,对着床头,柔声细气地:“……妹妹,我回来了……” 
歌声起: 
一个是阆苑仙葩,一个是美玉无瑕。若说没奇缘,今生偏又遇着他;若说有奇缘,如何心事终虚化?一个枉自嗟呀,一个空劳牵挂。
一个是水中月,一个是镜中花。想眼中能有多少泪珠儿,怎经得秋流到冬尽,春流到夏!
歌声里,宝玉娓娓诉说着、诉说着…… 