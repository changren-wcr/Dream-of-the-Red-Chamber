\chapter{贾夫人仙逝扬州城\quad 冷子兴演说荣国府}
\zhu{演说:铺演陈说。
《北史·熊安生传》:“问所疑,安生皆一一演说,咸就其根本。
”}
\jia{此回亦非正文,本旨只在冷子兴一人,即俗谓“冷中出热,无中生有”也。
其演说荣府一篇者,盖因族大人多,若从作者笔下一一叙出,尽一二回不能得明,则成何文字?故借用冷子一人,略出其大半,使阅者心中,已有一荣府隐隐在心,然后用黛玉、宝钗等两三次皴染,则耀然于心中眼中矣。
此即画家三染法也。
未写荣府正人,先写外戚,是由远及近、由小至大也。
若使先叙出荣府,然后一一叙及外戚,又一一至朋友、至奴仆,其死板拮据之笔,岂作“十二钗”人手中之物也?今先写外戚者,正是写荣国一府也。
故又怕闲文赘累,开笔即写贾夫人已死,是特使黛玉入荣之速也。
通灵宝玉于士隐梦中一出,今于子兴口中一出,阅者已洞然矣。
然后于黛玉、宝钗二人目中极精极细一描,则是文章锁合处。
盖不肯一笔直下,有若放闸之水、燃信之爆,使其精华一泄而无馀也。
究竟此玉原应出自钗、黛目中,方有照应。
今预从子兴口中说出,实虽写而却未写。
观其后文可知。
此一回则是虚敲傍击之文,笔则是反逆隐回之笔。
}
\qi{以百回之大文,先以此回作两大笔以冒之,诚是大观。
世态人情,尽盘旋于其间,而一丝不乱,非具龙象力者,其孰能哉?}
诗云:\jia{只此一诗便妙极!此等才情,自是雪芹平生所长,余自谓评书,非关评诗也。
}\par
\hop
一局输赢料不真,香销茶尽尚逡巡。
\par
欲知目下兴衰兆,须问旁观冷眼人。
\jia{故用冷子兴演说。
}\par
\hop
却说封肃因听见公差传唤,忙出来陪笑启问。
那些人只嚷:“快请出甄爷来!”\jia{一丝不乱。
}封肃忙陪笑道:“小人姓封,并不姓甄。
只有当日小婿姓甄,今已出家一二年了,不知可是问他?”那些公人道:“我们也不知什么‘真’‘假’,\jia{点睛妙笔。
}因奉太爷之命来问。
他既是你女婿,便带了你去亲见太爷面禀,省得乱跑。
”说着,不容封肃多言,大家推拥他去了。
封家人各各惊慌,不知何兆。
\par
那天约有二更时分,只见封肃方回来,欢天喜地。
\jia{出自封肃口内,便省却多少闲文。
}众人忙问端的。
\zhu{端的:究竟、详情。
}他乃说道:“原来本府新升的太爷,姓贾名化,本胡州人氏,曾与女婿旧日相交。
\meng{世态精神,叠露于数语间。
}
方才在咱门前过去,因看见娇杏\jia{侥幸也。
}\jia{托言当日丫头回顾,故有今日,亦不过偶然侥幸耳,非真实得尘中英杰也。
非近日小说中满纸红拂紫烟之可比。
}\jia{余批重出。
余阅此书,偶有所得,即笔录之。
非从首至尾阅过复从首加批者,故偶有复处。
且诸公之批,自是诸公眼界;脂斋之批,亦有脂斋取乐处。
后每一阅,亦必有一语半言,重加批评于侧,故又有于前后照应之说等批。
}那丫头买线,所以他只当女婿移住于此。
我一一将原故回明,那太爷倒伤感叹息了一回。
又问外孙女儿,\jia{细。
}我说看灯丢了。
太爷说:‘不妨,我自使番役务必采访回来。
’\zhu{
番役:明清官衙中专司缉捕的差役,也称“番子”。《红楼梦》中泛指官衙中负责稽查缉捕的差役。
}\jia{为葫芦案伏线。
}说了一回话,临走倒送了我二两银子。
”\jia{此事最要紧。
}甄家娘子听了,不免心中伤感。
\jia{所谓“旧事凄凉不可闻”也。
}\ping{封肃欢天喜地的缘由竟为二两银子,至于那女婿和外孙女的踪迹,哪曾放心上。
甄家娘子失去丈夫女儿,岂仅是伤感,心有希望而不敢燃,怕又是一场空。
}一宿无话。
\par
至次日,早有雨村遣人送两封银子、四匹锦缎,答谢甄家娘子,\jia{雨村已是下流人物,看此,今之如雨村者亦未有矣。
}又寄一封密书与封肃,转托他向甄家娘子要那娇杏作二房。
\jia{谢礼却为此。
险哉,人之心也!}封肃喜得屁滚尿流,巴不得去奉承,便在女儿前一力撺掇成了,\zhu{撺掇(音“汆夺”):怂恿。
}\jia{一语道尽。
}乘夜只用一乘小轿,便把娇杏送进去了。
雨村欢喜自不必说。
\meng{知己相逢,得遂平生,一大快事。
}乃封百金赠封肃,\ping{雨村送两封银子、四匹锦缎,答谢甄家娘子,报答甄士隐的知遇之恩;封百金赠封肃作为纳妾谢礼。
两相比较,恩情就显得寂寥。
}外又谢甄家娘子许多物事,令其好生养赡,以待寻访女儿下落。
\jia{找前伏后。
}封肃回家无话。
\jia{士隐家一段小荣枯至此结住,所谓“真不去,假焉来”也!}\par
却说娇杏这丫鬟,便是那年回顾雨村者。
因偶然一顾,便弄出这段事来,亦是自己意料不到之奇缘。
\jia{注明一笔,更妥当。
}\meng{点出情事。
}
谁想他命运两济,\jia{好极!与英莲“有命无运”四字遥遥相映射。
莲,主也;杏,仆也。
今莲反无运,而杏则两全,可知世人原在运数,不在眼下之高低也。
此则大有深意存焉。
}不承望自到雨村身边,只一年便生了一子,又半载,雨村嫡妻忽染疾下世,\zhu{下世:此指死亡。}雨村便将他扶册作正室夫人了。
\ping{不同境遇下的人,有不同的幸福尺度。
某些人心中梦寐以求的幸福终点,在其他人看来只不过是自己的起点。
娇杏作为一个趁夜被主人卖给大官作妾的女子,不仅生存下来,而且还得子扶正,在物质上得到了满足,已经算是属于她的婚姻幸福了。
含着金汤勺出生的林黛玉,天生就不会为物质而发愁,娇杏的幸福终点只不过是她的起点罢了,才有机会去追求精神共鸣的伴侣。
}正是:\par
\hop
偶因一着错,\jia{妙极!盖女儿原不应私顾外人之谓。
}便为人上人。
\zhu{一着:下棋术语,一步棋谓之一着。
这里比喻人的一个行动。
女子私顾外人,是封建礼法所不允许的.故云“一着错”;但娇杏却因此由奴婢变为主了,成了“人上人”。
}\jia{更妙!可知守礼俟命者终为饿莩。
\zhu{莩:音”漂“三声。饿莩:饿死的人。《孟子·梁惠王上》:「民有饥色,野有饿莩。」也作「饿殍」。}
其调侃寓意不小。
}
\jia{从来只见集古、集唐等句,未见集俗语者。
此又更奇之至!}\par
\hop
原来,雨村因那年士隐赠银之后,他于十六日便起身入都。
至大比之期,不料他十分得意,已会了进士,选入外班,\zhu{会了进士,选入外班:指会试考中进士,分发外省任官。
进士分为三甲(三等),除一甲三名外.其馀进士再经“朝考”,录取的称庶吉士;没有录取的,经过候选的程序,分发各部或外省听候委用。
“班”是指官员补缺的班次。
}今已升了本府知府。
虽才干优长,未免有些贪酷之弊,且又恃才侮上,那些官员皆侧目而视。
\jia{此亦奸雄必有之理。
}不上一年,便被上司寻了个空隙,作成一本,参他“生情狡猾,\zhu{参:参究、稽考,引伸为控告、弹劾。
弹劾所用的文书,称“详参”。
}擅纂礼仪,\zhu{擅纂礼仪:擅纂:擅自纂集。
封建时代的礼制仪式,例由礼部掌管,官员擅自纂集,要受惩处。
}且沽清正之名,而暗结虎狼之属,致使地方多事,民命不堪”\jia{此亦奸雄必有之事。
}等语。
龙颜大怒,即批革职。
\meng{罪重而法轻,何其幸也。
}该部文书一到,本府官员无不喜悦。
那雨村心中虽十分惭恨,却面上全无一点怨色,仍是喜悦自若。
\jia{此亦奸雄必有之态。
}交代过公事,将历年做官积的些资本并家小人属送至原籍,安排妥协,\jia{先云“根基已尽”,故今用此四字,细甚!}却又自己担风袖月,游览天下胜迹。
\jia{已伏下至金陵一节矣。
}\par
那日,偶又游至维扬地面,\zhu{维扬:即扬州,今江苏省扬州市。
《尚书·禹贡》:“淮海惟扬州。
”“惟”,通“维”,后称“维扬”,本此。
}因闻得今岁鹾政点的是林如海。
\zhu{鹾(音“搓”)政:这里指朝廷派到地方管理盐务的官员,带原衔品级。
鹾:盐。
}这林如海姓林名海,表字如海。
\zhu{表字:又称字。}
\jia{盖云“学海文林”也。
总是暗写黛玉。
}乃是前科的探花,\zhu{探花:明、清科举制度,殿试取为第三名者称探花。
}今已升至兰台寺大夫\zhu{兰台寺大夫:作者沿古虚拟的官名。
兰台是汉朝宫内藏书的地方,由御史中丞主管,兼任纠察。
后因称主管弹劾的御史台为兰台,御史府也叫兰台寺,设官曰兰台史令},\jia{官制半遵古名亦好。
余最喜此等半有半无,半古半今,事之所无,理之必有,极玄极幻,荒唐不经之处。
}本贯姑苏\jia{十二钗正出之地,故用真。
}人氏,今钦点出为巡盐御史,到任方一月有馀。
\par
原来这林如海之祖,曾袭过列侯,今到如海,业经五世。
起初时,只封袭三世,因当今隆恩盛德,远迈前代,\jia{可笑近时小说中,无故极力称扬浪子淫女,临收结时,还必致感动朝廷,使君父同入其情欲之界,明遂其意,何无人心之至!不知彼作者有何好处,有何谢报,到朝廷廊庙之上,直将半生淫朽,秽渎睿聪,又苦拉君父作一干证护身符,强媒硬保,得遂其淫欲哉!}额外加恩,至如海之父,又袭了一代;至如海,便从科第出身。
\ping{由于爵位降等世袭,祖先在开国定鼎的功业荫庇子孙终有尽头,子孙必须另谋出路。
林如海是贵族后裔转型的成功案例,作为科举新秀,新科探花,走上了进入仕途的正道;相比于其他贫寒出身的同年,林如海毕竟还有贵族之间数代积累的关系网,正如第四回门子说到贾史王薛四大家族时所言:“这四家皆连络有亲,一损皆损,一荣皆荣,扶持遮饰,皆有照应的。
”林如海和贾敏结婚,可以同时利用贾家和林家两家的贵族人脉资源,再加上科举探花出身,仕途顺利步步高升也就不足为奇了。
}虽系钟鼎之家,\zhu{钟鼎之家:“钟鸣鼎食之家”的简称。
钟:乐器。
鼎:一种三足两耳的金属器皿,这里是指盛菜肴的食具。
贵族家庭宴享祭祀时,鸣钟列鼎。
后常用“钟鼎之家”代指贵族豪门。
}却亦是书香\jia{要紧二字,盖钟鼎亦必有书香方至美。
}之族。
只可惜这林家支庶不盛,子孙有限,虽有几门,却与如海俱是堂族而已,没甚亲支嫡派的。
\jia{总为黛玉极力一写。
}
\ping{世上没有完美的事。
}
今如海年已四十,只有一个三岁之子,偏又于去岁死了。
虽有几房姬妾,\jia{带写贤妻。
}奈他命中无子,亦无可如何之事。
今只有嫡妻贾氏,生得一女,乳名黛玉,\meng{绛珠初见。
}年方五岁。
夫妻无子,故爱女如珍,且又见他聪明清秀,\jia{看他写黛玉,只用此四字。
可笑近来小说中,满纸“天下无二”“古今无双”等字。
}便也欲使他读书识得几个字,不过假充养子之意,聊解膝下荒凉之叹。
\zhu{膝下荒凉:指没有子嗣。
膝下:指幼儿环绕于父母的膝下,后为子女代称。
见《孝经·圣治》。
}\jia{如此叙法,方是至情至理之妙文。
最可笑者,近小说中满纸班昭蔡琰、文君道韫。
\zhu{
班昭:东汉史学家班固之妹,博学,曾参与续《汉书》。
和帝时担任过宫廷教师,号称“大家(家:音“姑”)”, 故称“班姑”。
编有《女诫》七篇,历来奉为妇德的典范。
见《后汉书·曹世叔妻传》。
蔡琰:指蔡文姬,名琰,东汉文学家蔡邕之女,博学多才,精通音律,是历史上有名的“才女”。
见《后汉书·董祀妻传》。
文君:汉代卓王孙的女儿.新寡后“私奔”文学家司马相如,结为夫妇。
道韫:东晋谢安姪女,王凝之的妻子,聪颖有才辩,以诗著称。
}
}\par
雨村正值偶感风寒,病在旅店,将一月光景方渐愈。
一因身体劳倦,二因盘费不继,也正欲寻个合式之处,\zhu{合式:同“合适”。
}暂且歇下。
幸有两个旧友,亦在此境居住,\jia{写雨村自得意后之交识也。
}\jia{又为冷子兴作引。
}因闻得鹾政欲聘一西宾,\zhu{西宾:亦称西席。
古代以西为尊,宾客或教师的座位,居西面东(见梁章钜《称谓录》),故以西宾或西席为家庭教师或官僚幕客的代称。
}雨村便相托友力,谋了进去,且作安身之计。
\ping{若仅作安身之计,又岂需托友力“谋”,此乃贾雨村欲借助林如海的关系谋求东山再起的计划。
}妙在只一个女学生,并两个伴读丫鬟,这女学生年又极小,身体又极怯弱,工课不限多寡,故十分省力。
\par
堪堪又是一载的光阴,\zhu{堪堪:即“看看”,估量时间之辞,义近转眼。
}谁知女学生之母贾氏夫人一疾而终。
女学生侍汤奉药,守丧尽哀,\meng{先要使黛玉哭起。
}遂又将要辞馆别图。
林如海意欲令女守制读书,\zhu{守制:古人父母或祖父母死后,嫡长子或承重孙(长房嫡长孙)要守孝三年,须闭门读书.谢绝世务,称为“守制”。
《论语·阳货》:“夫三年之丧,天下之通丧也。
”}故又将他留下。
近因女学生哀痛过伤,本自怯弱多病的,\jia{又一染。
}触犯旧症,遂连日不曾上学。
\jia{上半回已终,写“仙逝”正为黛玉也。
故一句带过,恐闲文有妨正笔。
}\par

雨村闲居无聊,每当风日晴和,饭后便出来闲步。
这日,偶至郭外,\zhu{郭外:指城外郊区。
《孟子·公孙丑下》:“三里之城,七里之郭。
”}
意欲赏鉴那村野风光。
\jia{大都世人意料此,终不能此;不及彼者,而反及彼。
故特书意在村野风光,却忽遇见子兴一篇荣国繁华气象。
}忽信步至一山环水旋、茂林深竹之处,隐隐有座庙宇,门巷倾颓,墙垣朽败,门前有额,题着“智通寺”三字,\jia{谁为智者?又谁能通?一叹。
}门旁又有一副旧破的对联,曰:\par
\hop
身后有馀忘缩手,眼前无路想回头。
\jia{先为宁、荣诸人当头一喝,却是为余一喝。
}\ping{知足不辱,知止不殆,可以长久。
}\par
\hop
雨村看了,因想到:“这两句话,文虽浅,其意则深。
\jia{一部书之总批。
}
也曾游过些名山大刹,倒不曾见过这话头,其中想必有个翻过筋斗来的亦未可知,\zhu{翻过筋斗来的:比喻饱经世事动荡或遭受重大挫折后“看破世情”的人。
筋斗:通作“跟头”。
}\jia{随笔带出禅机,又为后文多少语录不落空。
}何不进去试试?”想着,走入看时,只有一个聋肿老僧在那里煮粥。
\jia{是雨村火气。
}雨村见了,便不在意。
\jia{火气。
\zhu{火气:指的是贾雨村的世俗烟火之气。}
}及至问他两句话,那老僧既聋且昏,\jia{是翻过来的。
}\jia{欲写冷子兴,偏闲闲有许多着力语。
}齿落舌钝,\jia{是翻过来的。
}
所答非所问。
\par
雨村不耐烦,便仍出来,\jia{毕竟雨村还是俗眼,只能识得阿凤、宝玉、黛玉等未觉之先,却不识得既证之后。
}\ping{莫非“聋肿老僧”即为宝玉“既证之后”?也许暗示了宝玉“悬崖撒手”,“弃而为僧”的结局。
}\jia{未出宁、荣繁华盛处,却先写一荒凉小境;未写通部入世迷人,却先写一出世醒人。
回风舞雪,倒峡逆波,别小说中所无之法。
}意欲到那村肆中沽酒三杯,\zhu{村肆:这里指乡村酒店。
}以助野趣。
于是款步行来,刚入肆门,只见座上吃酒之客有一人起身大笑,接了出来,口内说:“奇遇,奇遇!”雨村忙看时,此人是都中古董行中贸易的号冷子兴者,\jia{此人不过借为引绳,不必细写。
}\ping{冷子兴,冷:衰微凄冷,兴:兴盛繁荣。
冷子兴的名字似乎有冷眼看兴衰之意,其职业是古董贸易商,也具有“冷眼看兴衰”的条件。
因为一般处于上升时期的家族,经济宽裕,需要买古董来装潢台面。
而处于下降阶段的家族,经济窘迫,需要卖或者典当抵押古董来维持开支。
古董商作为直接和这些大家族接触的商人,通过买卖典当这些贸易往来,就能看出家族的兴衰变迁。
后文当贾府入不敷出周转不开的时候,贾琏要“把老太太查不着的金银家伙偷着运出一箱子来,暂押千数两银子支腾过去。
不上半年的光景,银子来了,我就赎了交还”,可以证明。
}旧日在都相识。
雨村最赞这冷子兴是个有作为大本领的人,\qi{不赞出则文不灵活,而冷子兴之谈吐似觉唐突矣。
}这子兴又借雨村斯文之名,故二人说话投机,最相契合。
\zhu{借:冷子兴是商人,贾雨村是文人,文人借商人的财,商人借文人的才,互为补充。}
雨村忙亦笑问:“老兄何日到此?弟竟不知。
今日偶遇,真奇缘也。
”子兴道:“去年岁底到家,今因还要入都,从此顺路找个敝友说一句话,\zhu{敝:音“闭”,谦辞,用于跟自己有关的事物。
}承他之情,留我多住两日。
我也无甚紧事,且盘桓两日,待月半时也就起身了。
今日敝友有事,我因闲步至此,且歇歇脚。
不期这样巧遇!”一面说,一面让雨村同席坐了,另整上酒肴来。
二人闲谈慢饮,叙些别后之事。
\jia{好!若多谈则累赘。
}
\meng{又抛一笔。
}\par
雨村因问:“近日都中可有新闻没有?”\jia{不突然,亦常问常答之言。
}
子兴道:“倒没有什么新闻,倒是老先生你贵同宗家,\zhu{同宗:按古代宗法制度,本指同出于一个远祖者为“同宗”,后用以泛称同族或同姓。
}\jia{雨村已无族中矣,何及此耶?看他下文。
}出了一件小小的异事。
”雨村笑道:“弟族中无人在都,何谈及此?”子兴笑道:“你们同姓,岂非同宗一族?”雨村问是谁家。
\par
子兴道:“荣国府贾府中,可也不玷辱了先生的门楣了?”\jia{刳小人之心肺,闻小人之口角。
}雨村笑道:“原来是他家。
若论起来,寒族人丁却不少,自东汉贾复以来,\zhu{贾复:东汉南阳冠军(今属河南邓县)人,曾任执金吾、左将军,封胶东侯。
见《后汉书·贾复传》。
}支派繁盛,各省皆有,\jia{此话纵真,亦必谓是雨村欺人语。
}\meng{如闻其声。
}谁能逐细考查?若论荣国一支,却是同谱。
但他那等荣耀,我们不便去攀扯,至今越发生疏难认了。
”子兴叹\jia{叹得怪。
}
道:“老先生休如此说。
如今这荣国两门,也都萧疏了,不比先时的光景。
”\jia{记清此句。
可知书中之荣府已是末世了。
}雨村道:“当日宁荣两宅的人口极多,如何就萧疏了?”\jia{作者之意原只写末世,此已是贾府之末世了。
}冷子兴道:“正是,说来也话长。
”雨村道:“去岁我到金陵地界,因欲游览六朝遗迹,那日进了石头城,\zhu{石头城:故址在今南京市。
三国时孙权所建。
后用以代指金陵或南京。
上文讲的六朝(吴、东晋、南朝的宋、齐、梁、陈),皆建都金陵。
}\jia{点睛神妙。
}从他老宅门前经过。
街东是宁国府,街西是荣国府,二宅相连,竟将大半条街占了。
大门前虽冷落无人,\jia{好!写出空宅。
}隔着围墙一望,里面厅殿楼阁,也还都峥嵘轩峻,就是后\jia{“后”字何不直用“西”字?}\jia{恐先生堕泪,故不敢用“西”字。
}一带花园子里,树木山石,也还都有蓊蔚洇润之气,\zhu{蓊(音“翁”)蔚洇(音“因”)润:茂盛润泽的样子。
}那里像个衰败之家?”\par
冷子兴笑道:“亏你是进士出身,原来不通!古人有云:‘百足之虫,死而不僵。
’\zhu{百足之虫,死而不僵:比喻大贵族官僚家庭,虽已衰败,但表面仍能维持某种繁荣的假象。
语见三国魏曹冏(音“窘”)《六代论》:“故语曰:‘百足之虫,死而不僵’,扶之者众也。
”百足之虫:指马陆、蜈蚣一类节肢动物。
僵:仆倒。
}如今虽说不及先年那样兴盛,较之平常仕宦之家,到底气象不同。
如今生齿日繁,\zhu{生齿:人口,古代把长出乳齿的男女登入户籍。
}事务日盛,主仆上下,安富尊荣者尽多,运筹谋画者无一,\jia{二语乃今古富贵世家之大病。
}其日用排场费用,又不能将就省俭,如今外面的架子虽未甚倒,\jia{“甚”字好!盖已半倒矣。
}内囊却也尽上来了。
\ping{家里钱袋空空,袋的内面翻上来,言经济困窘。
外面的架子只不过是强撑罢了,财源枯竭必然导致无以为继。
}\meng{世家兴败,寄口与人,诚可悲夫。
\zhu{这条评语的意思是,世家大族的兴衰,成为了旁人的谈资,确实令人感到可悲。}
}这还是小事,更有一件大事。
谁知这样钟鸣鼎食之家,翰墨诗书之族,\jia{两句写出荣府。
}
如今的儿孙,竟一代不如一代了!”\jia{文是极好之文,理是必有之理,话则极痛极悲之话。
}\ping{躺在祖先的功劳簿上不思进取,最后一定会坐吃山空。
},也纳罕道:“这样诗书之家,岂有不善教育之理?别家不知,只说这宁、荣二宅,是最教子有方的。
”\jia{一转有力。
}\par
子兴叹道:“正说的是这两门呢。
待我告诉你。
当日宁国公\jia{演。
}
与荣国公\jia{源。
}是一母同胞弟兄两个。
宁公居长,生了四个儿子。
\jia{贾蔷、贾菌之祖,不言可知矣。
}宁公死后,长子贾代化袭了官,\jia{第二代。
}
也养了两个儿子。
长名贾敷,至八九岁上便死了,只剩了次子贾敬袭了官,\jia{第三代。
}如今一味好道,只爱烧丹炼汞,\zhu{烧丹炼汞:道教以朱砂(丹)、水银(汞)等烧炼“仙药”(即所谓“金丹”)的一种方术,以此妄求飞升成仙,长生不死。
}\jia{亦是大族末世常有之事。
叹叹!}\meng{偏先从好神仙的苦处说来。
}馀者一概不在心上。
幸而早年留下一子,名唤贾珍,\jia{第四代。
}因他父亲一心想作神仙,把官倒让他袭了。
他父亲又不肯回原籍来,只在都中城外和道士们胡羼。
\zhu{
羼:音“颤”。《说文》:“羼,羊相厕(厕:置身于)也。
”引申为搀杂。
胡羼:犹言“鬼混”。
}这位珍爷也倒生了一个儿子,今年才十六岁,名叫贾蓉。
\jia{至蓉五代。
}如今敬老爹一概不管。
这珍爷那肯读书,只是一味高乐不已,\zhu{高乐:恣意寻欢作乐。
}把宁国府竟翻了过来,也没有人敢来管他。
\jia{伏后文。
}再说荣府你听,方才所说异事,就出在这里。
自荣公死后,长子贾代善袭了官,\jia{第二代。
}娶的金陵世勋史侯家的小姐\jia{因湘云,故及之。
}为妻,生了两个儿子:长子贾赦,次子贾政。
\jia{第三代。
}如今代善早已去世,太夫人\jia{记真,湘云祖姑史氏太君也。
\zhu{太君:古代官员的母亲的封号;后用来尊称对方的母亲。}
}
尚在。
长子贾赦袭着官。
次子贾政,自幼酷喜读书,祖父最疼。
原欲以科甲出身的,不料代善临终时遗本一上,皇上因恤先臣,即时令长子袭官外,问还有几子,立刻引见,遂额外赐了这政老爹一个主事之衔,\zhu{主事:清代六部之下设司,司的主管官是郎中,其副手是员外郎,再下就是主事。
下文的入部,指的是工部,主管建筑、水利诸事。
}
\jia{嫡真实事,非妄拟也。
}令其入部习学,如今现已升了员外郎了。
\jia{总是称功颂德。
}\ping{贾政本可以像自己的妹夫林如海那样从科举正道进入仕途,却偏偏因祖荫袭官,走了旁门左道,并不利于自己仕途的发展,依旧活在祖先功业的庞大阴影下。
贾政逼迫贾宝玉读书考功名,也可能是寄希望于儿子完成自己未尽的遗憾。
}这政老爹的夫人王氏,\jia{记清。
}头胎生的公子,名唤贾珠,十四岁进学,不到二十岁就娶了妻生了子,\jia{此即贾兰也。
至兰第五代。
}
一病死了。
\jia{略可望者即死,叹叹!}\ping{贾家各代人名字按字排辈,贾母这一代的男性都是代字辈——贾代化、贾代善。
下一代全都是文字辈,单名,都是文字偏旁——贾敷、贾敬、贾赦、贾政。
林黛玉的妈妈贾敏也是文字旁。
再下一代是玉字辈,名字都跟玉有关——贾珍、贾珠、贾宝玉、贾琏、贾瑞、贾环。
再下来一代是草字头,贾蓉,还有贾珠的儿子贾兰。
还有贾蔷,贾芹。
}第二胎生了一位小姐,生在大年初一,这就奇了;不想次年又生了一位公子,\jia{一部书中第一人却如此淡淡带出,故不见后来玉兄文字繁难。
}说来更奇:一落胎胞,嘴里便衔下一块五彩晶莹的玉来,上面还有许多字迹,\jia{青埂顽石已得下落。
}就取名叫作宝玉。
你道是新奇异事不是?”\chen{正是宁、荣二处支谱。
}\par
\zhu{“次年”还是“后来”?“次年”,各本均同,只有戚本、舒本改为“后来”。
从后文可以看出,元春比宝玉显然远不止大一岁。
如下文第十八回:\hang
……当日这贾妃未入宫时,自幼亦系贾母教养。
后来添了宝玉,贾妃乃长姊,宝玉为弱弟,贾妃之心上念母年将迈,始得此弟,是以怜爱宝玉,与诸弟待之不同。
且同随贾母,刻未暂离。
那宝玉未入学堂之先,三四岁时,已得贾妃手引口传,教授了几本书、数千字在腹内了。
其名分虽系姊弟,其情状有如母子。
\hang
但是,原著本来就存在宝玉年龄忽大忽小问题,是作者未最后修改完稿的结果(张爱玲《红楼梦魇》分析甚详,可参看)。
所以,个别的改动并不能彻底解决宝玉的年龄问题。
另外,也可以理解为,这是作者为表现冷子兴的信口开河而故意写错的,如此,就更不应该随便改动了。
因此,这里我们仍然保留“次年”,不从后改的“后来”。
\hang
除了宝玉,书中其他人物如黛玉、宝钗、贾兰等也存在年龄前后不一的情况。
有的本子鉴于第九回上学的贾兰明显大于第十八回极幼的贾兰,把前者改名贾蓝,变成另一人物,同样是不可取的。
}
\par
雨村笑道:“果然奇异。
只怕这人来历不小。
”子兴冷笑道:“万人皆如此说,因而乃祖母便先爱如珍宝。
\zhu{乃:第二人称称代词。相当于「你」、「你的」。如:「乃父」、「乃翁」。}
那年周岁时,政老爹便要试他将来的志向,便将那世上所有之物摆了无数,与他抓取。
\zhu{抓取:即“抓周”,俗名“试儿”。
婴儿满一周岁,家人陈列各种物品、用具,任其抓取以预测他未来的志向和前途。
见《颜氏家训·风操》。
}谁知他一概不取,伸手只把些脂粉钗环抓来。
政老爹便大怒了,说:‘将来酒色之徒耳!’因此便大不喜悦。
独那史老太君还是命根一样。
说来又奇,如今长了七八岁,虽然淘气异常,但其聪明乖觉处,百个不及他一个。
说起孩子话来也奇怪,他说:‘女儿是水作的骨肉,男人是泥作的骨肉。
\jia{真千古奇文奇情。
}我见了女儿,我便清爽;见了男人,便觉浊臭逼人。
’你道好笑不好笑?将来色鬼无疑了!”\jia{没有这一句,雨村如何罕然厉色,并后奇奇怪怪之论?}雨村罕然厉色忙止道:“非也!可惜你们不知道这人来历。
大约政老前辈也错以淫魔色鬼看待了。
若非多读书识事,加以致知格物之功,\zhu{致知格物:语出《大学》:“致知在格物,物格而后知至。
”致:推导。
格:推究。
}悟道参玄之力者,\zhu{悟道参玄:宗教用语。
领会和推究宗教中玄妙的道理。
}不能知也。
”\par
子兴见他说得这样重大,忙请教其端。
雨村道:“天地生人,除大仁大恶两种,馀者皆无大异。
若大仁者,则应运而生,大恶者,则应劫而生。
\zhu{应运而生、应劫而生:运:宋代象数学者邵雍《皇极经世书》中,以三十年为一世,十二世为一运,三十运为一会,十二会为一元。
这里指气数,吉祥和顺的时代气运。
劫:佛家用语。
梵文音译“劫波”之略,意为“远大时节"。
佛教认为,世界有周期性的生灭过程,它经历若干万年后,就要毁灭一次,重新开始,此一周期称为一“劫”。
每“劫”中还包括“成”、“住”、“坏”、“空”四个阶段。
到“坏劫”时,有水、火、风三灾出现.世界便归于毁灭.故后人又将“劫”引伸作灾难解,如后文“劫终之日”、“生关死劫”等,这里指时代的灾难、厄运。
这两句意思是说:(大圣大贤的人)是适应祥和的时代气运而生的;(大邪大恶的人)是应着灾难的时代气运而生的。
}运生世治,劫生世危。
尧、舜、禹、汤、文、武、周、召、孔、孟、董、韩、周、程、张、朱,\zhu{尧、舜:即唐尧虞舜,传说中原始社会的两个部落联盟领袖。
禹:夏禹夏代开国君主。
汤:成汤,商代开国君主。
文、武:周文王、周武王。
姬姓,周朝开国的两个君主。
周、召:即周公旦、召(音“邵”)公奭(音“试”),周武王的两个弟弟,也是辅佐他开国的两个大臣。
孔、孟:即孔丘、孟轲,两个儒家代表人物。
董:西汉经学大师董仲舒。
韩:唐代文学家韩愈。
周:北宋理学家周敦颐。
程:北宋理学家程颢(音“号”)、程颐兄弟。
张:北宋思想家张载。
朱:南宋理学家朱熹。
}皆应运而生者。
蚩尤、共工、桀、纣、始皇、王莽、曹操、桓温、安禄山、秦桧等,\zhu{蚩尤:传说中的上古部族首领,曾与黄帝战于“涿鹿之野”。
共工:传说中的“四凶”之一。
桀:夏朝末代君主。
纣:商朝末代君主。
始皇:即秦始皇。
王莽:西汉末年的大官僚贵族,篡位称帝.改国号为新。
曹操:即魏武帝,三国时政治家、军事家。
桓温:东晋时大司马,专擅朝政。
安禄山:胡人,唐玄宗时节度使,曾与史思明发起“安史之乱”。
秦桧:南宋高宗时宰相,历史上有名的奸臣。
}皆应劫而生者。
\jia{此亦略举大概几人而言。
}大仁者,修治天下;大恶者,挠乱天下。
清明灵秀,天地之正气,仁者之所秉也;残忍乖僻,天地之邪气,恶者之所秉也。
今当运隆祚永之朝,\zhu{运隆祚永:国运兴隆,皇位传世久远。
运:这里指国运。
祚(音“做”):皇位、国统。
}太平无为之世,清明灵秀之气所秉者,上至朝廷,下至草野,比比皆是。
所馀之秀气,漫无所归,遂为甘露,为和风,洽然溉及四海。
\zhu{洽(音“恰”)然:协和滋润的样子。
}彼残忍乖僻之邪气,不能荡溢于光天化日之中,遂凝结充塞于深沟大壑之内,偶因风荡,或被云摧,略有摇动感发之意,一丝半缕误而泄出者,偶值灵秀之气适过,正不容邪,邪复妒正,\jia{譬得好。
}
两不相下,亦如风水雷电,地中既遇,既不能消,又不能让,必至搏击掀发后始尽。
故其气亦必赋人,发泄一尽始散。
使男女偶秉此气而生者,在上则不能成仁人君子,下亦不能为大凶大恶。
\jia{恰极,是确论。
}置之于万万人中,其聪俊灵秀之气,则在万万人之上,其乖僻邪谬、不近人情之态,又在万万人之下。
\ping{世间大多数人皆是秉正邪两气所生,亦正亦邪,“上则不能成仁人君子,下亦不能为大凶大恶”,并没有极端脸谱化的“好人”、“坏人”之分。
}若生于公侯富贵之家,则为情痴情种,若生于诗书清贫之族,则为逸士高人,\meng{巧笔奇言,另开[生]面。
但此数语,恐误尽聪明后生者。
}纵再偶生于薄祚寒门,断不能为走卒健仆,甘遭庸人驱制驾驭,必为奇优名倡。
如前代之许由、陶潜、阮籍、嵇康、刘伶、王谢二族、顾虎头、陈后主、唐明皇、宋徽宗、刘庭芝、温飞卿、米南宫、石曼卿、柳耆卿、秦少游,近日之倪云林、唐伯虎、祝枝山,再如李龟年、黄幡绰、敬新磨、卓文君、红拂、薛涛、崔莺、朝云之流。
\zhu{许由:传说中的上古高士。
许由在河边,皇帝去找他,说听说你是贤良之士,想请你来做官。
许由觉得做官这种事听到都感觉脏,于是赶快用水洗耳朵,“颍水洗耳”成了一个典故。
陶潜:东晋大诗人。
阮籍:魏晋之际诗人。
嵇康:魏晋之际文学家。
刘伶:与阮、嵇皆为“竹林七贤”之一。
王谢二族:指东晋王导、谢安两家族。
顾虎头:顾恺之,字虎头。
东晋名画家。
陈后主:南朝陈末代皇帝陈叔宝。
唐明皇:即唐玄宗李隆基。
宋徽宗:北宋末代皇帝赵佶(佶:音“急”)。
刘庭芝:刘希夷,字庭芝,唐代诗人。
温飞卿:温庭筠,字飞卿,晚唐诗人。
米南宫:即米芾,北宋著名书画家。
石曼卿:石延年,字曼卿,北宋文学家。
耆音“齐“。柳耆卿:柳永,字耆卿,北宋著名词人。
秦少游:秦观,字少游,北宋著名词人。
倪云林:倪瓒(瓒:音“赞”),号云林子,元代山水画家。
唐伯虎:唐寅,字伯虎,明代画家、文学家。
祝枝山:祝允明,号枝山,明代书法家、文学家。
李龟年:唐玄宗时宫廷乐师。
黄幡绰(幡绰:音“番辍”):唐玄宗时艺人。
敬新磨:五代后唐庄宗时宫廷艺人。
卓文君:汉代卓王孙的女儿.新寡后“私奔”文学家司马相如,结为夫妇。
红拂:隋代越国公杨素的侍女,私奔李靖。
薛涛:唐代名妓。
崔莺:即《会真记》中的女主人公莺莺。
朝云:宋代钱塘名妓。
}此皆易地则同之人也。
”\par
子兴道:“依你说,‘成则王侯败则贼’\jia{《女仙外史》中论魔道已奇,此又非《外史》之立意,故觉愈奇。
}了。
\zhu{
冷子兴这里说的“成则王侯败则贼”,前后衔接过于突兀。
这里可能暗含了曹雪芹对至高无上的皇权思想的批判。
这句话可能有更现实的历史背景,这就是雍正即位的斗争。
康熙晚年,诸王子争位,各立党派,斗争非常激烈,
雍正即位后,对与他争位的诸王子,杀的杀,关的关。
这事离曹雪芹的时候,才不过二十多年,曹雪芹的家也是在这场斗争的余波中败落的。
这件事对于雍正来说,当然是”成则王侯“,但对于允禩、允禟等来说,那就是”败则贼“了。
无怪后来怡亲王府的抄本,要把“王侯”的“王”字改为“公”字,成为“成则公侯败则贼”了,因为他是亲历过这场斗争的。
}
”雨村道:“正是这意。
你还不知,我自革职以来,这两年遍游名省,也曾遇见两个异样孩子。
\jia{先虚陪一个。
}所以,方才你一说这宝玉,我就猜着了八九亦是这一派人物。
不用远说,只金陵城内,钦差金陵省体仁院总裁\jia{此衔无考,亦因寓怀而设,置而勿论。
}甄家,\zhu{钦差金陵省体仁院总裁:钦差:明清时由皇帝指派出外办理重大事情的官员,其中由皇帝特命并授予关防(关防:旧时政府机关或军队用的印信,多为长方形)者,权力更大,称“钦差大臣”。
体仁院总裁:作者虚拟的官衔。
}\jia{又一个真正之家,特与假家遥对,故写假则知真。
}你可知么?”子兴道:“谁人不知!这甄府和贾府就是老亲,又系世交。
两家来往,极其亲热的。
便在下也和他家来往非止一日了。
”\jia{说大话之走狗,毕真。
}雨村笑道:“去岁我在金陵,也曾有人荐我到甄府处馆。
我进去看其光景,谁知他家那等显贵,却是富而好礼之家,\jia{如闻其声。
}\jia{只一句便是一篇家传,与子兴口中是两样。
}倒是个难得之馆。
但这一个学生,虽是启蒙,\zhu{启蒙:启发蒙昧。
这里指旧时儿童开始上学读书,读物有《三字经》、《百家姓》之类。
}却比一个举业的还劳神。
\zhu{举业:指旧时科举应试,其读物有《四书》、《五经》之类。
}说起来更可笑,他说:‘必得两个女儿伴着我读书,我方能认得字,心里也明白,不然我自己心里糊涂。
’\jia{甄家之宝玉乃上半部不写者,故此处极力表明,以遥照贾家之宝玉。
凡写贾宝玉之文,则正为真宝玉传影。
}又常对跟他的小厮们说:‘这女儿两个字,极尊贵,极清净的,比那阿弥陀佛,\zhu{阿弥陀佛:梵文音译,习称“弥陀”。
意译为“无量寿”、“无量光”,为大乘佛教的佛名。
佛经说他是“极乐世界”的教主,净土宗宣称诵此名号,即可往西方“极乐世界”。
}元始天尊的这两个宝号还更尊荣无对的呢!\zhu{元始天尊:道教的尊神。
道经说他“生于太元之先”,故称“元始”。
他居于天界最高的“玉清”仙境,为“三清”之首。
}\jia{如何只以释、老二号为譬,略不敢及我先师儒圣等人?余则不敢以顽劣目之。
}
\zhu{
程高本篡改“比那阿弥陀佛,元始天尊的这两个宝号还更尊荣无对的呢”为“比那瑞兽珍禽、奇花异草更觉希罕尊贵的呢”,这充分暴露了篡改者把妇女当作花草玩物的剥削阶级腐朽思想。
}
你们这浊口臭舌,万不可唐突了这两个字,要紧!\meng{\sout{固}[故]作险笔,以为后文之伏线。
}但凡要说时,必须先用清水香茶漱了口才可,设若失错,便要凿牙穿腮等事。
’其暴虐浮躁,顽劣憨痴,种种异常。
只一放了学,进去见了那些女儿们,其温厚和平,聪敏文雅,\jia{与前八个字嫡对。
}竟又变了一个。
因此,他令尊也曾下死笞楚过几次,\zhu{
笞(音“痴”):竹板。
楚:荆条。
都是打人的工具。
这里作动词用。
笞楚:即鞭打;抽打。
}
无奈竟不能改。
每打的吃疼不过时,他便‘姐姐’‘妹妹’乱叫起来。
\jia{以自古未闻之奇语,故写成自古未有之奇文。
此是一部书中大调侃寓意处。
盖作者实因鹡鸰之悲、棠棣之威,\zhu{鹡鸰:音“急灵”,鸟,背部羽毛颜色纯一,中央尾羽比两侧的长,停息时尾上下摆动。
生活在水边,吃昆虫等。
在本书下次出现在第十五回,皇帝亲赐鹡鸰香念珠一串给北静王,北静王转赠给宝玉,宝玉转赠给黛玉,而黛玉不取。
“鹡鸰之悲”可能指回忆这件往事而伤悲。
棠棣:音“糖弟”,古书上说的一种植物。
在本书不再出现,可能也指代某件事情,由于原文散失而无法具体得知。
另一种解释是:
《诗经》常棣:“常棣之華,鄂不韡韡。
凡今之人,莫如兄弟。
死喪之威,兄弟孔懷。
原隰裒矣,兄弟求矣。
脊令在原,兄弟急難。
每有良朋,況也永歎。
”常棣:即棠棣,这种植物的特质就是,花开之后复再合拢,用来形容兄弟之间发生龃龉没关系,终究和好,就像花朵分开后又合上。
脊令:即鹡鸰。
据师旷《禽经》上说,鹡鸰这种鸟群体情深,一只离群,其他鸟就鸣叫以寻找同类。
历代皆用棠棣和鹡鸰来比喻兄弟手足情深。
}故撰此闺阁庭帏之传。
}后来听得里面女儿们拿他取笑:‘因何打急了只管唤姐妹做甚?莫不是求姐妹去讨情讨饶?你岂不愧些!’他回答的最妙。
他说:‘急疼之时,只叫姐姐、妹妹字样,或可解疼也未可知,因叫了一声,便果觉不疼了,遂得了秘方。
每疼痛之极,便连叫姐妹起来了。
’\meng{闲闲逗出无穷奇语,都只为下文。
}你说可笑不可笑?也因祖母溺爱不明,每因孙辱师责子,因此我就辞了馆出来。
如今在巡盐御史林家坐馆了。
你看,这等子弟,必不能守祖父之根基,从师长之规谏的。
只可惜他家几个好姊妹,都是少有的。
”\jia{实点一笔,余谓作者必有。
}\par
子兴道:“便是贾府中,现有的三个也不错。
政老爹之长女,名元\jia{“原”也。
}
春,现因贤孝才德,选入宫中作女史\jia{因汉以前例,妙!}
去了。
\zhu{女史:古代宫中女官名。
掌管王后的礼职。
见《周礼·天官·女史》。
后也成为尊贵、文雅女子的泛称。
}二小姐乃赦老爹前妻所出,名迎\jia{“应”也。
}春,三小姐乃政老爹之庶出,名探\jia{“叹”也。
}春,四小姐乃宁府珍爷之胞妹,名唤惜\jia{“息”也。
}
春。
\chen{贾敬之女。
}\ping{贾府四个姑娘:元春,迎春,探春,惜春,谐音“原应叹息”,暗含对其命运的叹息。
}因史老夫人极爱孙女,都跟在祖母这边一处读书,听得个个不错。
”\chen{复续前文未及,正词源三叠。
}雨村道:“更妙在甄家之风俗,女儿之名,亦皆从男子之名命字,不似别家另外用这些‘春’‘红’‘香’‘玉’等艳字的,何得贾府亦落此俗套?”\par
子兴道:“不然,只因现今大小姐是正月初一日所生,故名元春,馀者方从了‘春’字。
上一辈的,却也是从兄弟而来的。
\meng{黛玉之入\sout{宁}[荣]国府的根源,却藉他二人之口,下文便不费力。
}现有对证:目今你贵东家林公之夫人,\zhu{目今:现在,当前。
}即荣府中赦、政二公之胞妹,在家时名唤贾敏。
不信时,你回去细访可知。
”雨村拍案笑道:“怪道这女学生读至凡书中有‘敏’字,他皆念作‘密’字,\zhu{“敏”念“密”:古代有避讳之制,对君亲的名字,不能直读其音,直书其字。
必须改字、改音或省笔,以示敬避之意。
}每每如是;写字时遇着‘敏’字,又减一二笔,我心中就有些疑惑。
今听你说,是为此无疑矣。
怪道我这女学生言语举止另是一样,不与近日女子相同,度其母必不凡,方得其女,今知为荣府之孙,又不足罕矣。
可伤上月竟亡故了。
”子兴叹道:“老姊妹四个,这一个是极小的,又没了。
长一辈的姊妹,一个也没了。
只看这小一辈的,将来之东床如何呢。
”\zhu{东床:指女婿。
晋代太尉郗(音“希”)鉴派人到丞相王导家选女婿,王家的子弟都很矜持,惟独王羲之不以为意.坦腹躺在东床上吃东西。
郗鉴欣赏他这种“名士”风度,就选中了他。
见《世说新语·雅量》。
}\par
雨村道:“正是,方才说这政公,已有了一个衔玉之儿,\meng{灵玉却只一块,而宝玉有两个。
\zhu{宝玉有两个:贾宝玉和甄宝玉。}
情性如一,亦如六耳悟空之意耶?}又有长子所遗一个弱孙。
这赦老竟无一个不成?”子兴道:“政公既有玉儿之后,其妾后又生了一个,\jia{带出贾环。
}倒不知其好歹。
只眼前现有二子一孙,却不知将来如何。
若问那赦公,也有二子。
\meng{本家族谱记不清者甚多,偏是旁人说来,一丝不乱。
}长名贾琏,今已二十来往了。
亲上作亲,娶的就是政老爹夫人王氏之内侄女,\jia{另出熙凤一人。
}\zhu{内侄女:即侄女。}今已娶了二年。
这位琏爷身上现捐的是个同知,
\zhu{
同知:职官名。指正官之副。
凡主管一事而不授以正官之名,则谓之知某事,
如宋代不以枢密院使授人,则称为「知枢密院事」,副使则称为「同知」。
辽、金以后,沿此习惯,如府之主官称「知府」,而以府之佐官为「同知」。
}
也是不喜读书,于世路上好机变言谈去的,所以如今只在乃叔政老爷家住着,帮着料理些家务。
谁知自娶了他令夫人之后,倒上下无一人不称颂他夫人的,琏爷倒退了一射之地。
\zhu{一射之地:约当一百二十至一百五十步。
亦称“一箭道”。
}说模样又极标致,言谈又爽利,心机又极深细,竟是个男人万不及一的。
”\jia{未见其人,先已有照。
}\jia{非警幻案下而来为谁?}\par
雨村听了,笑道:“可知我前言不谬。
\jia{略一总住。
}你方才所说的这几个人,都只怕是那正邪两赋而来一路之人,未可知也。
”子兴道:“邪也罢,正也罢,只顾算别人家的帐,你也吃一杯酒才好。
”\meng{笔转如流,毫无沾滞。
}雨村道:“正是,只顾说话,竟多吃了几杯。
”子兴笑道:“说着别人家的闲话,正好下酒,\jia{盖云此一段话亦为世人茶酒之笑谈耳。
}即多几杯何妨。
”雨村向窗外看\jia{画。
}道:“天也晚了,仔细关了城。
我们慢慢进城再谈,未为不可。
”于是,二人起身,算还酒帐。
\jia{不得谓此处收得索然,盖原非正文也。
}\par

方欲走时,又听得后面有人叫道:“雨村兄,恭喜了!特来报个喜信的。
”\jia{此等套头,亦不得不用。
}雨村忙回头看时——\ji{语言太烦,令人不耐。
古人云“惜墨如金”,看此则视墨如土矣。
虽演至千万回亦可也。
}
\par
\qi{总评:先自写幸遇之情于前,而叙借口谈幻境之情于后。
世上不平事,道路口如碑。
虽作者之苦心,亦人情之必有。
\hang
雨村之遇娇杏,是此文之总冒,故在前。
冷子兴之谈,是事迹之总冒,故叙写于后。
冷暖世情,比比如画。
\hang
有情原比无情苦,生死相关总在心。
也是前缘天作合,何妨黛玉泪淋淋。
}
\dai{003}{贾夫人仙逝扬州城}
\dai{004}{冷子兴演说荣国府}
\sun{p2-1}{贾雨村求娶娇杏}{图右上:是日晚间,忽听许多公人乱嚷,却是传唤甄士隐。
封肃只好出来说明原委。
公人们为了交差,便推拥封肃而去。
一家人惊慌不已。
到了二更时分,封肃方回来。
原来这新任太爷便是贾雨村。
他在街上看到当年心仪的甄家丫鬟, 便来寻访恩人。
得知他走后甄家败落经过,甄士隐已无奈出家,各感伤心, 说了许多安慰话。
图右下:次日一早雨村派人送了银两,又修密书向甄家娘子要那娇杏做二房。
喜得封肃眉开眼笑,当夜用一乘小轿便将娇杏送进衙门里去了。
}
\sun{p2-2}{贾雨村教导林黛玉}{雨村在旧友引荐下来到巡盐御史林如海家做了西宾,教授年方五岁的黛玉。
}