%\chapter{金陵城起复贾雨村\quad 荣国府收养林黛玉}
\chapter[金陵城起复贾雨村\quad 荣国府收养林黛玉]{金陵城起复贾雨村\quad 荣国府收养\jia{二字触目凄凉之至!}林黛玉}
\zhu{起复:旧时官吏因事降革者,恢复原官、原衔叫“开复”;因父母之丧离职,守孝期满而复用者叫“起复”。
“起复”和“开复”在习惯用法上没有严格区别。
}
\par
\qi{我为你持戒,我为你吃斋;我为你百行百计不舒怀,我为你泪眼愁眉难解。
无人处,自疑猜,生怕那慧性灵心偷改。
\hang
宝玉通灵可爱,天生有眼堪穿。
\zhu{有眼堪穿:本回后文在形容贾宝玉的通灵宝玉的时候,有“落草时从他口里掏出,上头有现成的穿眼“的描述。}
万年幸一遇仙缘,从此春光美满。
随时喜怒哀乐,远却离合悲欢。
地久天长香影连,可意方舒心眼。
\hang
宝玉衔来,是补天之馀,落地已久,得地气收藏,因人而现。
其性质内阳外阴,其形体光白温润,天生有眼可穿,故名曰宝玉,将欲得者尽皆宝爱此玉之意也。
\hang
天地循环秋复春,生生死死旧重新。
君家着笔描风月,宝玉颦颦解爱人。
\zhu{颦颦:后文中宝玉要送给黛玉的字。}
}\par

却说雨村忙回头看时,不是别人,乃是当日同僚一案参革的号张如圭\jia{盖言“如鬼如蜮”也,
\zhu{蜮:音“玉”,传说中一种会害人的水中毒虫,形状似鳖,能含沙射人;第二种解释是一种食苗叶的害虫。}
亦非正人正言。
}者。
他本系此地人,革职后家居,今打听得都中奏准起复旧员之信,他便四下里寻情找门路,忽遇见雨村,故忙道喜。
\meng{此途宦境,描写的当。
}二人见了礼,张如圭便将此信告诉雨村,雨村自是欢喜,忙忙的叙了两句,\jia{画出心事。
}遂作别各自回家。
冷子兴听得此言,便忙献计,\jia{毕肖赶热灶者。
}令雨村央烦林如海,转向都中去央烦贾政。
雨村领其意,作别回至馆中,忙寻邸报看真确了。
\zhu{邸报:邸(音“底”):本为来朝诸侯王或上京办事官僚的居处,后用以泛称王侯和大官僚的府第。
邸报:又名“邸钞”、“宫门钞”,是邸中给诸藩官僚的书面报导,内容包括传钞的诏令奏章和其它新闻记事等,是我国最早的一种报纸。
起于汉代。
后世亦称政府官报为邸报。
}
\jia{细!}\ping{喜悦之态倒是感同身受。
}\par
次日,面谋之如海。
如海道:“天缘凑巧,因贱荆去世,\zhu{贱荆:荆:指“荆钗布裙”。
语出《列女传》。
旧时谦称自己的妻子为贱荆、拙荆、山荆等。
}都中家岳母念及小女无人依傍教育,前已遣了男女、\zhu{男女:指仆人。
}船只来接,因小女未曾大痊,故未及行。
此刻正思向蒙训教之恩未经酬报,遇此机会,岂有不尽心图报之理。
但请放心,弟已预为筹画至此,已修下荐书一封,转托内兄务为周全协佐,
\zhu{内兄:称谓。用以称妻子的哥哥。}
方可稍尽弟之鄙诚,即有所费用之例,弟于内兄信中已注明白,亦不劳尊兄多虑矣。
”\meng{要说正文故以此作引,且黛玉路中实无可托之人。
文笔逼切得宜。
}\ping{此时贾雨村需要拜托林如海谋求复职,林如海却没有摆出施恩于人的倨傲态度,而是把自己放到了需要报答贾雨村教导自己女儿的恩情的地位,从而把贾雨村求林如海办事转化为林如海要报恩于贾雨村,林如海可谓谦逊之至。
}雨村一面打躬,谢不释口,一面又问:“不知令亲大人现居何职?\jia{奸险小人欺人语。
}只怕晚生草率,不敢骤然入都干渎。
”\zhu{干渎(音“读”):冒犯的意思。
}\jia{全是假,全是诈。
}\meng{借雨村细密心思之语,容容易易转入正文,亦是宦途人之口头心头。
最妙!}如海笑道:“若论舍亲,与尊兄犹系同谱,乃荣公之孙。
大内兄现袭一等将军之职,名赦,字恩侯;二内兄名政,字存周,\jia{二名二字皆颂德而来,与子兴口中作证。
}现任工部员外郎,其为人谦恭厚道,大有祖父遗风,非膏粱轻薄仕宦之流,故弟方致书烦托。
否则不但有污尊兄之清操,即弟亦不屑为矣。
”\jia{写如海实写政老。
所谓此书有不写之写是也。
}\meng{作弊者每每偏能如此说。
}雨村听了,心下方信了昨日子兴之言,\ping{第二回“冷子兴演说荣国府”的时候,介绍了贾府的情况,按理说当时在场的贾雨村应该清楚。
贾雨村这时候还继续问林如海,一方面是怀疑古董商的话不全是可信的,想要确认一下,另一方面是害怕对方官不够大,办不成事。
由此可见贾雨村之周全细密。
}于是又谢了林如海。
如海乃说:“已择了出月初二日小女入都,尊兄即同路而往,岂不两便?”雨村唯唯听命,\zhu{唯唯:音“委委”,恭敬应诺之词。
}心中十分得意。
如海遂打点礼物并饯行之事,雨村一一领了。
\par
那女学生黛玉,身体大愈,原不忍弃父而往,无奈他外祖母致意务去,且兼如海说:“汝父年将半百,再无续室之意,且汝多病,年又极小,上无亲母教养,下无姊妹兄弟扶持,\jia{可怜!一句一滴血,一句一滴血之文。
}今依傍外祖母及舅氏姊妹去,正好减我顾盼之忧,何云不往?”黛玉听了,方洒泪拜别,\jia{实写黛玉。
}\meng{此一段是不肯使黛玉作弃父乐为远游者。
以此可见作者之心宝爱黛玉如己。
}随同奶娘及荣府几个老妇人登舟而去。
雨村另有一只船,带两个小童,依附黛玉而行。
\jia{老师依附门生,怪道今时以收纳门生为幸。
}\meng{细密如此,是大家风范。
}\par
\zhu{
    黛玉身体大愈?甲戌本作“大愈”,己卯、庚辰、杨藏、舒序及列藏本作“又愈”(列藏本系“既”    字涂改为“又”字),戚本、蒙本作“方愈”,甲辰本及程本则干脆删    去“黛玉身体大愈”数字。
\hang
    大概因为前文叙述林黛玉生病,那么此时病“方愈”看起来更顺理    成章吧,现在所见通行各新校本均依戚本、蒙本作“方愈”。
其实,仔    细推敲,就会发现选用“方愈”大有问题。
“那女学生黛玉身体方愈,    原不忍弃父而往……”给人的感觉就是,黛玉的不愿前往有两层顾虑,    首先是出于病刚好经不起远行,其次才是不忍离开父亲。
这就大大削弱    了作者笔下的林黛玉的孝心了。
蒙府本侧批:“此一段是不肯使黛玉作    弃父乐为远游者。
以此可见作者之心宝爱黛玉如己。
”甚是。
由此看来,    “方愈”当是出于戚本一系底本抄手的妄改,不应采用。
\hang
    那么,“大愈”和“又愈”又孰是呢?\hang
    由于“大”和“又”字型相近,存在抄误的可能性较大。
有人就认    为“大”是“又”的形讹,少数服从多数,原文应作“又愈”,正是作    者要暗示黛玉自来身体孱弱,又病又愈是经常的事;而就她的体质,说    “大愈”(完全康复)是不真实的。
\hang
    我的看法是:“又愈”的说法并不很通顺,在书中其他地方也没有    出现;而“大愈”一词却多次出现,是作者的习用词汇。
更重要的是前    文如海对雨村说他“岳母念及小女无人依傍教育,前已遣了男女船只来    接,因小女未曾大痊,故未及行。
”“大痊”“大愈”意思一样,前后    观照。
原先病未“大痊”,故未及行,现在“大愈”了,总得行了,却    “不忍弃父”,还是不愿意行。
这就突出表现了黛玉的父女深情。
可见    以“大愈”为是。
\hang
    至于“大愈”的“大”字,在这里只是表程度,“大愈”也就是    “好多了”的意思,并不违背事实。
后文第五十八回:\hang
    “宝玉……瞧黛玉益发瘦的可怜,问起来,比往日已算大愈了。
    黛玉见他也比先大瘦了,想起往日之事,不免流下泪来……”\hang
    这里“大愈”“大瘦”正是“好多了”“瘦多了”的意思。
\hang
    综上所述,此处异文应校为“大愈”。
}
\par
有日到了都中,\zhu{都中:这里说的“都中”包括京畿,即京城周围地区。
}\jia{繁中减笔。
}进入神京,\zhu{神京:即京城。
皇帝所居,故称神京。
}雨村先整了衣冠,\jia{且按下黛玉以待细写。
今故先将雨村安置过一边,方起荣府中之正文也。
}带了小童,\jia{至此渐渐好看起来也。
}拿着宗侄的名帖,\zhu{
宗侄:同宗族的侄辈。
名帖:即名片。
旧时在纸片上书写自己的姓名、籍贯、官职、爵位,拜访时,投以通名。
始行汉代,最早用削平的木条写上姓名里居。
两汉时叫“谒”,汉末叫“刺”,后代虽用纸制,亦相沿称“名刺”。
}\jia{此帖妙极,可知雨村的品行矣。
}\ping{贾雨村为了东山再起,不惜低头攀附认亲。
}
至荣府门前投了。
彼时贾政已看了妹丈之书,即忙请入相会。
见雨村相貌魁伟,言谈不俗,且这贾政最喜读书人,礼贤下士,拯溺济危,大有祖风,况又系妹丈致意,因此优待雨村,\jia{君子可欺[以]其方也,
\zhu{
「欺以其方」出于〔孟子‧万章上〕,文中载孟子的话说:「故君子可欺以其方,难罔以非其道。」
意思是说,君子心地光明坦荡,对事纯粹以是否合乎情理来判断,
不会猜疑人,所以容易受人用合情合理的说法欺骗;但是如果明显地违情悖理,就无法骗得了他了。
这是孟子弟子万章问,舜的弟弟象想要害死舜,但未成功,看见舜仍然好好的坐在家里,
于是骗舜说:「我是来看你的。」舜听了很高兴,相信了象的话,是否因为舜不知道象是害自己的?
孟子说:「舜不是不知道,只是因为爱弟弟,看见弟弟高兴就高兴而已。」
同时又举出一个例子,说有人送了子产一条鱼,子产不忍吃,叫管池沼的小吏送到池里去放生,
小吏却偷偷的烹来吃了,然后告诉子产说:「这鱼真是得其所哉了!」
子产轻易相信小吏的话,并不是他笨,而是听说游鱼得所就高兴。
所以用符合君子的观念来欺骗他,欺骗是可以被君子以为真的。「君子可以欺以其方」就是这个意思。}
况雨村正在王莽谦恭下士之时,
\zhu{
王莽谦恭下士之时:王莽为西汉末权臣,未篡汉前曾伪装谦恭下士。
白居易《放言五首·其三》:
赠君一法决狐疑,不用钻龟与祝蓍。
试玉要烧三日满,辨材须待七年期。
周公恐惧流言日,王莽谦恭未篡时。
向使当初身便死,一生真伪复谁知?
这条评语意在说明贾雨村犹如王莽一样,未得势时还在装作谦谦君子。
}
虽政老亦为所惑,在作者系指东说西也。
}更又不同,便竭力内中协助。
题奏之日,轻轻谋\jia{《春秋》字法。
\zhu{
《春秋》字法:
《春秋》是孔子根据鲁史撰修的编年体史书。
古代学者说它“以一字为褒贬”,含有“微言大义”。
后来就把文笔深隐曲折、意含褒贬叫“春秋笔法”。
}
}了一个复职候缺,不上两个月,金陵应天府缺出,便谋补\jia{《春秋》字法。
}了此缺,拜辞了贾政,择日到任去了。
\jia{因宝钗故及之,一语过至下回。
\zhu{这条评语的意思是,第四回薛宝钗的哥哥薛蟠强抢香菱为妾闹出人命,是金陵应天府的贾雨村从中谋划使其脱罪。}
}不在话下。
\meng{了结雨村。
}\ping{贾政只是工部主事,清代六部之下设司,司的主管官是郎中,其副手是员外郎,再下就是主事,所以贾政的官职并不是很高,但是却能够帮助贾雨村“轻轻”,也就是很容易地谋到应天巡抚这样的关键城市要职,可见贾政依赖的不是自己的而是女儿贾元春的地位。
(第四回:如今且说贾雨村,因补授了应天府,一下马,就有一件人命官司详至案下)}\par

且说黛玉自那日弃舟登岸时,\jia{这方是正文起头处。
此后笔墨,与前两回不同。
}便有荣国府打发了轿子并拉行李的车辆久候了。
这黛玉常听得\jia{三字细。
}\meng{以“常听见”等字省下多少笔墨。
}母亲说过,他外祖母家与别家不同。
他近日所见的这几个三等的仆妇,已是不凡了,何况今至其家。
因此步步留心,时时在意,不肯轻意多说一句话,多行一步路,\meng{颦颦故自不凡。
}生恐被人耻笑了他去。
\jia{写黛玉自幼之心机。
}\chen{黛玉自忖之语。
}\ping{黛玉心细如此,除了天性颖慧,可能还因为母亲早逝且父女分离,无人可真正依傍。
}自上了轿,进入城中,从纱窗向外瞧了一瞧,其街市之繁华,人烟之阜盛,自与别处不同。
\jia{先从街市写来。
}又行了半日,忽见街北蹲着两个大石狮子,三间兽头大门,门前列坐着十来个华冠丽服之人。
正门却不开,只有东西两角门有人出入。
正门之上有一匾,匾上大书“敕造宁国府”五个大字。
\zhu{敕造:奉皇帝之命建造。
敕(音“ 斥”):本为自上命下之词,南北朝以前,通用于长官对下属,长辈对晚辈,之后,则为皇帝发布诏令的专称。
}\jia{先写宁府,这是由东向西而来。
}黛玉想道:“这是外祖母之长房了。
”想着,又往西行,不多远,照样也是三间大门,方是荣国府了。
却不进正门,\meng{以下写\sout{宁}[荣]国府第,总借黛玉一双俊眼中传来。
非黛玉之眼,也不得如此细密周详。
}只进了西边角门。
那轿夫抬进去,走了一射之地,将转弯时,便歇下退出去了。
后面婆子们已都下了轿,赶上前来。
另换了三四个衣帽周全的十七八岁的小厮上来,复抬起轿子。
众婆子步下围随,至一垂花门前落下。
众小厮退出,众婆子上来打起轿帘,扶黛玉下轿。
\meng{以上写款项。
}林黛玉扶着婆子的手,进了垂花门,\zhu{垂花门:旧家宅院,进入大门之后,内院院门例有雕刻的垂花,倒悬于门额两侧,门上边盖有宫殿式的小屋顶,称垂花门。
}两边是抄手游廊,\zhu{抄手游廊:院门内两侧环抱的走廊。
}当中是穿堂,\zhu{穿堂:座落在前后两个院落之间可以穿行的厅堂。
}当地放着一个紫檀架子大理石的大插屏。
\zhu{大插屏:放在穿堂中的大屏风,除作装饰外,还可以遮蔽视线,以免进入穿堂,直见正房。
}转过插屏,小小三间内厅,厅后就是后面的正房大院。
正面五间上房,皆是雕梁画栋,两边穿山游廊厢房,\zhu{穿山游廊:山:指山墙,房子两侧的墙。
“穿山游廊”是从山墙开门接起的游廊。
厢房,指四合院中东西两边的房子。
}挂着各色鹦鹉、画眉等鸟雀。
\zhu{宅院结构示意图见页脚\foot{
\footPic{宅院结构示意图}{house2.jpg}{0.8}
}:三座房屋围合的民宅称为三合院,四座房屋围合则称为四合院,围合出一个庭院则称为一进,围合出两个庭院则称为两进,以此类推。
图片中的房屋是三进四合院,从大门到垂花门是第一进院,从垂花门到正房是第二进院,从正房到后罩房是第三进院。
大门通常设置于院落一角,也叫做角门。
大门左边的一排房屋因背对街巷而称为倒座房,是整个四合院中形制较低的房屋,往往作为客房或由佣人居住。
进入大门,便是第一进院,用于接待来客。
再向里走,便需要经过二门,这里是内外有别的分界线,一般男性客人到此止步,而内宅的女眷也遵循礼教不再外出,所谓“大门不出、二门不迈”。
有的二门两侧各有一根悬空的短柱,柱子下端雕刻成莲花花苞形状,这种二门又称垂花门。
二门内侧会摆放屏风等遮挡物,客人到来的时候视线被挡在二门之外,保护主人的隐私。
穿过二门,整个四合院的核心就会出现在眼前,即第二进院。
第二进院由正房、厢房等构成,正房两边的小屋为耳房,东西相对的房屋即东西厢房,厢房多为家中晚辈的居所,耳房格局很小,多用于储藏室、书房等辅助用房。
正房是一座院落中等级最高的建筑,前后往往四排柱子并设置外廊,空间宽敞、光线明亮,一般由长辈居住,体现明确的等级、辈分。
正房与厢房之间还会设置廊道,方便雨雪时行走。
院落到这里还没有结束,经过耳房旁边的一个狭小通道,正房后面还“藏”着一进院落即第三进院。
院中的房屋被称为后罩房,其私密性最强,往往用作厨房、仓库或由家中的女眷居住。
基础宅院升级为高级府邸有如下几种方式:一是可以增加纵向深度,从三进增加到四进或五进甚至更多;二是增加横向宽度,将四合院平行排列,扩展为多跨四合院,向西称为“西跨院”,向东称为“东跨院”。
三跨十三进院的恭王府是高级府邸的典型代表,甚至可以把紫禁城可以看做一个超级多跨四合院。
}台矶之上,坐着几个穿红着绿的丫鬟,一见他们来了,便忙都笑迎上来,说:“才刚老太太还念呢,可巧就来了。
”\jia{如见如闻,活现于纸上之笔。
好看煞!}于是三四人争着打起帘栊,\zhu{栊:窗户;房舍。
帘栊:竹帘与窗户,或窗户上的竹帘,这里显然是门上的帘子。
}\jia{真有是事,真有是事!}一面听得人回话:“林姑娘到了。
”\par
黛玉方进入房时,只见两个人搀着一位鬓发如银的老母迎上来,\jia{此书得力处,全是此等地方,所谓“颊上三毫”也。
\zhu{
颊上三毫:“颊上三毫”为中国古代绘画术语,语出晋代顾恺之:
“……颊上加三毛,观者觉神明殊胜。”意在说明,画人物若要传神,必须抓住主要特点,而笔墨不必多。
脂批用来指文中描写贾母出场,只用了一句话便极为传神,“鬓发如银”四字,更简洁地勾绘出贾母的神貌。
}
}黛玉便知是他外祖母。
方欲拜见时,早被他外祖母一把搂入怀中,“心肝儿肉”\qi{写尽天下疼女儿的神理。
}\meng{此一段文字是天性中流出,我读时不觉泪盈双袖。
}\ping{为林黛玉而哭,更为去世的贾敏而哭。
}叫着大哭起来。
\jia{几千斤力量写此一笔。
}当下地下侍立之人,无不掩面涕泣,\jia{旁写一笔,更妙!}黛玉也哭个不住。
\jia{自然顺写一笔。
}\meng{逼真。
}一时众人慢慢的解劝住了,黛玉方拜见了外祖母。
\jia{书中正文之人,却如此写出,却是天生地设章法,不见一丝勉强。
}此即冷子兴所云之史氏太君,贾赦、贾政之母也。
\jia{书中人目太繁,故明注一笔,使观者省眼。
}当下贾母一一指与黛玉:“这是你大舅母,\chen{邢氏。
}这是你二舅母,\chen{王氏。
}这是你先珠大哥的媳妇珠大嫂。
\chen{李纨。
}”黛玉一一拜见过。
贾母又说:“请姑娘们来。
今日远客才来,可以不必上学去了。
”众人答应了一声,便去了两个。
\par
不一时,只见三个奶嬷嬷并五六个丫鬟,簇拥着三个姊妹来了。
\jia{声势如现纸上。
}\jia{从黛玉眼中写三人。
}第一个肌肤微丰,\jia{不犯宝钗。
\zhu{犯:重复。}}
合中身材,腮凝新荔,鼻腻鹅脂,温柔沉默,观之可亲。
\jia{为迎春写照。
}
第二个削肩细腰,\jia{《洛神赋》中云“肩若削成”是也。
}长挑身材,鸭蛋脸面,俊眼修眉,顾盼神飞,文彩精华,见之忘俗。
\jia{为探春写照。
}第三个身量未足,形容尚小。
\jia{浑写一笔更妙!必个个写去则板矣。
可笑近之小说中有一百个女子,皆是如花似玉一副脸面。
}其钗环裙袄,\jia{是极。
}三人皆是一样的妆饰。
\jia{毕肖。
}\meng{欲画天尊,先画\sout{纵}[众]神。
如此,其天尊自当另有一番高山世外的景象。
}黛玉忙起身迎上来见礼,\zhu{见礼:见面行礼。
}\jia{此笔亦不可少。
}互相厮认过,大家归坐。
丫鬟们斟上茶来。
不过说些黛玉之母如何得病,如何请医服药,如何送死发丧。
不免贾母又伤感起来,\jia{妙!}\meng{层层不露,周密之至。
}因说:“我这些儿女,所疼者惟有你母,今日一旦先舍我去了,连面也不能一见,今见了你,我怎么不伤心!”说着,搂了黛玉在怀,又呜咽起来。
\meng{不禁我也跟他哭起。
}众人忙都宽慰解释,\zhu{解释:劝解疏通。
}方略略止住。
\jia{总为黛玉自此不能别往。
}\par
众人见黛玉年貌虽小,其举止言谈不俗,身体面庞虽怯弱不胜,\jia{写美人是如此笔仗,看官怎得不叫绝称赏!}却有一段自然风流态度,\jia{为黛玉写照。
众人目中,只此一句足矣。
}\jia{从众人目中写黛玉。
}
\jia{
草胎卉质,
\zhu{
草胎卉质:黛玉的前身是西方灵河岸上三生石畔的绛珠草。
}
岂能胜物耶?想其衣裙皆不得不勉强支持者也。
}便知他有不足之症。
\zhu{不足之症:中医病症名。
由身体虚弱引起。
如脾胃虚弱,叫中气不足;气血虚弱,叫正气不足。
}因问:“常服何药,如何不急为疗治?”黛玉笑道:“我自来是如此,从会吃饮食时便吃药,到今未断,请了多少名医修方配药,皆不见效。
那一年我才三岁时,听得说\jia{文字细如牛毛。
}来了一个癞头和尚,\jia{奇奇怪怪一至于此。
通部中假借癞僧、跛道二人点明迷情幻海中有数之人也。
非袭《西游》中一味无稽、至不能处便用观世音可比。
\zhu{
有数之人:西游记中的观世音现身并推动故事剧情发展的场景很多;
但是本书中癞僧、跛道现身的次数很少,受其点明的人仅是有限的几个人。
}
}说要化我去出家,我父母固是不从。
他又说:‘既舍不得他,只怕他的病一生也不能好的了。
若要好时,除非从此以后总不许见哭声,\qi{爱哭的偏写出有人不教哭。
}\meng{作者既以黛玉为绛珠化生,是要哭的了,反要使人先叫他不许哭。
妙!}除父母之外,凡有外姓亲友之人,一概不见,方可平安了此一世。
’\ping{林黛玉进贾府,见到的不都是除父母外的外姓亲友之人吗?按此说法,黛玉岂可平安了此一世。
}疯疯癫癫,说了这些不经之谈,\jia{是作书者自注。
}也没人理他。
如今还是吃人参养荣丸。
”\jia{人生自当自养荣卫。
\zhu{养荣卫:又叫“养营卫”。
中医把人体中饮食所化的精气和功能分为“营”和“卫”。
“营”指充盈于内、生化血液、营养周身的作用;“卫”指捍卫于外、抗御病邪、保卫肌表的作用。
营与卫互为影响,如果外邪自表而入,就会出现营卫失和的症状。
}}\jia{甄英莲乃副十二钗之首,却明写癞僧一点。
今黛玉为正十二钗之冠,反用暗笔。
盖正十二钗人或洞悉可知,副十二钗或恐观者忽略,故写极力一提,使观者万勿稍加玩忽之意耳。
}贾母道:“这正好,我这里正配丸药呢。
叫他们多配一料就是了。
”\jia{为后菖、菱伏脉。
\zhu{
这条评语的意思大概是,遗失的后文中贾府子弟贾菖、贾菱在贾府负责配药的差事,
黛玉的药就是他们两人配的。
}
}\par
一语未了,只听得后院中有人笑声\jia{懦笔庸笔何能及此!}说:“我来迟了,不曾迎接远客!”\jia{第一笔,阿凤三魂六魄已被作者拘定了,后文焉得不活跳纸上?此等文字非仙助即神助,从何而得此机括耶?\zhu{机括:弩上控制箭矢发射的机件。
泛指机械发动、开启的部分。
这里比喻心机、计谋。
}}\jia{另磨新墨,搦锐笔,
\zhu{
搦:音“诺”。握、持。
}
特独出熙凤一人。
未写其形,先使闻声,所谓“绣幡开,遥见英雄俺”也。
}黛玉纳罕道:“这些人个个皆敛声屏气,恭肃严整如此,这来者系谁,这样放诞无礼?”\jia{原有此一想。
}\meng{天下事,不可一概而论。
}心下想时,只见一群媳妇丫鬟围拥着一个人从后房门进来。
这个人打扮与众姑娘不同,彩绣辉煌,恍若神妃仙子:头上戴着金丝八宝攒珠髻,\zhu{金丝八宝攒珠髻:用金丝穿绕珍珠和镶嵌八宝(玛瑙、碧玉之类)制成的珠花的发髻。
攒:凑聚。
用金丝或银丝把珍珠穿扭成各种花样叫“攒珠花”。
}绾着朝阳五凤挂珠钗,\zhu{绾:音“碗”,系。
朝阳五凤挂珠钗:一种长钗.样子是一支钗上分出五股,每股一支凤凰,口衔一串珍珠。
}\jia{头。
}项上带着赤金盘螭璎珞圈,\zhu{螭(音“吃”):古代传说中的无角龙。
璎珞:联缀起来的珠玉。
圈:项圈。
}\jia{颈。
}裙边系着豆绿宫绦、双衡比目玫瑰佩,\zhu{佩:玉佩。
古代贵族佩带的玉器,常雕琢成各种形状。
比目:鱼名,传说这种鱼成双而行。
“比目玫瑰佩”是玫瑰色的玉片雕琢成双鱼形的玉佩。
衡:亦作“珩”(音“横”):佩玉上部的小横杠,用以系饰物。
}\jia{腰。
}身上穿着缕金百蝶穿花大红洋缎窄褃袄,\zhu{缕金百蝶穿花大红洋缎窄褃袄:指在大红洋缎的衣面上用金线绣成百蝶穿花图案的紧身袄。
褃(音“垦”,四声):上衣前后两幅在腋下合缝的部分。
}
\meng{大凡能事者,多是尚奇好异,不肯泛泛同流。
}外罩五彩刻丝石青银鼠褂,\zhu{五彩刻丝石青银鼠褂:石青色的衣面上有各种彩色刻丝、衣里是银鼠皮的褂子。
银鼠:状颇类鼬,耳小毛短,其色洁白,皮可御轻寒,极贵重。
刻丝:在丝织品上用丝平织成的图案,与凸出的绣花不同。
石青:淡灰青色。
}
下着翡翠撒花洋绉裙。
\zhu{翡翠撒花洋绉裙:翡翠:翠绿色。
撒花:在绸缎上用散碎小花点组成的花样或图案。
绉:音“宙”,皱纹。
洋绉:极薄而软的平纹春绸,微带自然皱纹。
}一双丹凤三角眼,两弯柳叶吊梢眉,\zhu{丹凤眼:用丹顶凤凰细长的眼睛来比照说明某人的眼睛形状,形容眼角向上微翘。
用“丹凤”一语暗示该人物具有男性的性格。
在中国传统文化中,“丹凤眼”一词常用于忠义豪爽的男性。
最典型的例证就是家喻户晓的《三国演义》中的关羽形象。
小说中说他生就“丹凤眼,卧蚕眉”。
在本回的黛玉口中得知王熙凤“自幼假充男儿教养的”,从而解释了王熙凤身上男性性格的来源。
三角眼:三角就是一般意义上的形状描绘,眼睛呈三角状,“丹凤眼”向上微翘的眼角是眼的外角,除此之外还有眼的内角和凸起的上缘。
若某人眼上缘凸得厉害,则与两角大体呈三角状,这当是“三角眼”的真正内涵。
此种形状大约近似于正放的钝角三角形,当然其顶角度数当是比较大的,呈三角眼的难看一面。
在文学作品中也大多是用“三角眼”来形容长相丑陋的。
如《水浒传》第三十七回:“宋江看那张横时,但见:七尺身躯三角眼,黄髯赤发红睛,浔阳江上有声名。
冲波如水怪,跃浪似飞鲸。
恶水狂风都不惧,蛟龙见处魂惊,天差列宿害生灵。
小孤山下住,船火号张横。
”古代麻衣相法认为“三角眼”为狡黠、狠毒、性巧、通变、邪淫之相。
“三角”二字刻画眼神的狡黠,狠毒。
柳叶吊梢眉:形容眉梢斜飞入鬓的样子。
}\meng{非如此眼,非如此眉,不得为熙凤。
作者读过麻衣相法。
}身量苗条,体格风骚,粉面含春威不露,丹唇未启笑先闻。
\jia{为阿凤写照。
}\jia{试问诸公:从来小说中可有写形追像至此者?}\meng{英豪本等。
}\ping{霸气外露。
}黛玉连忙起身接见。
贾母笑\jia{阿凤一至,贾母方笑,与后文多少笑字作偶。
}道:“你不认得他,他是我们这里有名的一个泼皮破落户儿,\zhu{泼皮破落户儿:原指没有正当生活来源的无赖。
这里形容凤姐泼辣,是戏谑的称谓。
}南省俗谓作‘辣子’,你只叫他‘凤辣子’就是。
”\jia{阿凤笑声进来,老太君打诨,虽是空口传声,却是补出一向晨昏起居,阿凤于太君处承欢应候一刻不可少之人,看官勿以闲文淡文也。
}黛玉正不知以何称呼,\meng{想黛玉此时神情,含浑可爱。
}只见众姊妹都忙告诉他道:“这是琏嫂子。
”黛玉虽不识,亦曾听见母亲说过,大舅贾赦之子贾琏,娶的就是二舅母王氏之内侄女,自幼假充男儿教养的,学名叫王熙凤。
\jia{奇想奇文。
以女子曰“学名”固奇,然此偏有学名的反倒不识字,不曰学名者反若假。
\zhu{
王熙凤文化程度不高。
第二十八回:凤姐命人取过笔砚纸来,向宝玉道:“大红妆缎四十匹,蟒缎四十匹,上用纱各色一百匹,金项圈四个。
”宝玉道:“这算什么?又不是帐,又不是礼物,怎么个写法?”凤姐道:“你只管写上,横竖我自己明白就罢了。}
}
黛玉忙陪笑见礼,以“嫂”呼之。
这熙凤携着黛玉的手,上下细细的打量了一回,\jia{写阿凤全部传神第一笔也。
}便仍送至贾母身边坐下,因笑道:“天下真有这样标致人物,我今儿才算见了!\jia{这方是阿凤言语。
若一味浮词套语,岂复为阿凤哉!}\jia{“真有这样标致人物”出自凤口,黛玉丰姿可知。
宜作史笔看。
}况且这通身的气派,竟不像老祖宗的外孙女儿,竟是个嫡亲的孙女,\jia{仍归太君,方不失《石头记》文字,且是阿凤身心之至文。
}怨不得老祖宗天天口头心头,一时不忘。
\jia{却是极淡之语,偏能恰投贾母之意。
}
\meng{以“真有”“怨不得”五字,写熙凤之口头,真是机巧异常,“怨不得”三字,愚弄了多少聪明特达者。
}只可怜我这妹妹这样命苦,\jia{这是阿凤见黛玉正文。
}
怎么姑妈偏就去世了!”\jia{若无这几句,便不是贾府媳妇。
}说着,便用帕拭泪。
贾母笑道:“我才好了,你倒来招我。
\jia{文字好看之极。
}你妹妹远路才来,身子又弱,也才劝住了,快再休提前话!”\jia{反用贾母劝,看阿凤之术亦甚矣。
}这熙凤听了,忙转悲为喜道:\ping{嬉笑怒骂收放自如。
}“正是呢!我一见了妹妹,一心都在他身上了,又是欢喜,又是伤心,竟忘记了老祖宗。
该打,该打!”又忙携黛玉之手,问:“妹妹几岁了?可也上过学?现吃什么药?在这里不要想家,想要什么吃的、什么顽的,只管告诉我,丫头老婆们不好了,也只管告诉我。
”一面又问婆子们:“林姑娘的行李东西可搬进来了?带了几个人来?\jia{当家的人本如此,毕肖!}\meng{三句话不离本行,职任在兹也。
}你们赶早打扫两间下房,让他们去歇歇。
”\par
说话时,已摆了茶果上来,熙凤亲为捧茶捧果。
\jia{总为黛玉眼中写出。
}\meng{熙凤后到,为有事,
\zhu{为:因为。}
写其劳能,先为筹画,写其机巧。
摇前映后之笔。
}
又见二舅母问他:“月钱放完了不曾?”\zhu{月钱:每月按身份等级发给家中上下人等供零用的钱。
}\jia{不见后文,不见此笔之妙。
\zhu{
关于“月钱”的对话并非闲笔,这条评语可以作证。
围绕“月钱”后文又发生了哪些矛盾和纠纷呢?
\hang
第三十六回王夫人问王熙凤,赵姨娘周姨娘的月钱是否都按数给他们,
并直言道:“前儿我恍惚听见有人抱怨,说短了一吊钱,是什么原故?”
王熙凤巧言解释后到廊檐上看见几个执事的媳妇便借机大发牢骚:
“我从今以后倒要干几样尅毒事了。抱怨给太太听,我也不怕。
……明儿一裹脑子扣的日子还有呢。如今裁了丫头的钱,就抱怨了咱们。
也不想一想是奴几,也配使两三个丫头!”
\hang
第三十九回袭人问平儿道:“这个月的月钱,连老太太和太太还没放呢,是为什么?”
平儿见问,忙转身至袭人跟前,见方近无人,才悄悄说道:“你快别问,横竖再迟几天就放了。”
袭人笑道:“这是为什么,唬得你这样?”平儿悄悄告诉他道:
“这个月的月钱,我们奶奶早已支了,放给人使呢。等别处的利钱收了来,凑齐了才放呢。……”
袭人道:“难道他还短钱使,还没个足厌?何苦还操这心。”
平儿笑道:“何曾不是呢。这几年拿着这一项银子,翻出有几百来了。
他的公费月例又使不着,十两八两零碎攒了放出去,只他这梯己利钱,一年不到,上千的银子呢。”
\hang
第五十五回秋纹又问平儿,“宝玉的月钱我们的月钱多早晚才领。”
\hang
小说以“月钱”为引线串起了一系列情节的发展,由王夫人问“月钱”牵出王熙凤迟发月钱——挪用月钱——克扣月钱——利用月钱放债,最后贾府遭祸可能和放债有关。
这些事件和情节集中表现了王熙凤的贪欲,为王熙凤由受宠到失宠最后“机关算尽”“哭向金陵”的悲剧命运埋下了伏笔,也为贾府的势败家亡埋下了一条重要的伏线。
\hang
这句问话还妙在虽含蓄但却又很好地体现出了人物之间的微妙关系。
从王夫人的问话中我们不难看出,她才是贾府真正的管理者,而王熙凤只不过是代她行使管家的权利而已,
对于月钱的不能及时发放王夫人应该有所耳闻,这轻轻的一问表明了她对月钱发放问题的关注和关心,
但王熙凤毕竟是她信得过的内侄女、贾母宠爱的孙媳妇,明知有问题但并没有一问到底,
聪明的王熙凤也深知信任的力量,面对王夫人的询问,她敷衍搪塞,转移话题,巧妙滑过。
}
}熙凤道:“月钱已放完了。
才刚带着人到后楼上找缎子,\jia{接闲文,是本意避繁也。
}找了这半日,也并没有见昨日太太说的那样,\jia{却是日用家常实事。
}想是太太记错了?”\meng{陪笔用得灵活,兼能形容熙凤之为人,妙心妙手,故有妙文妙口。
}王夫人道:“有没有,什么要紧。
”因又说道:“该随手拿出两个来,给你这妹妹去裁衣裳的,\jia{仍归前文。
妙妙!}等晚上想着叫人再去拿罢,可别忘了。
”熙凤道:“倒是我先料着了,知道妹妹不过这两日到的,我已预备下了,\jia{余知此缎阿凤并未拿出,此借王夫人之语机变欺人处耳。
若信彼果拿出预备,不独被阿凤瞒过,亦且被石头瞒过了。
}等太太回去过了目好送来。
”\jia{试看他心机。
}王夫人一笑,点头不语。
\jia{深取之意。
}\chen{很漏凤姐是个当家人。
}\par

当下茶果已撤,贾母命两个老嬷嬷带了黛玉去见两个母舅。
时贾赦之妻邢氏忙亦起身,笑道:“我带了外甥女过去,倒也便宜。
”\zhu{便(音“遍”)宜:这里是方便之意。
}\meng{以黛玉之来去候安之便,便将荣宁二府的势排描写尽矣。
}贾母笑道:“正是呢,你也去罢,不必过来了。
”邢夫人答应一个“是”字,遂带了黛玉与王夫人作辞,大家送至穿堂前。
出了垂花门,早有众小厮们拉过一辆翠幄青紬车来。
\zhu{翠幄:指用粗厚的绿色绸类作的轿车车帐。
青紬(紬即绸字):这里指用青色绸作的车帘。
}邢夫人携了黛玉坐上,\chen{未识黛卿能乘此否。
}众婆子们放下车帘,方命小厮们抬起,拉至宽处,方驾上驯骡,亦出了西角门,往东过了荣府正门,便入一黑油大门中,至仪门前方下来。
\zhu{仪门:旧时官衙、府第的大门之内的门,具装饰作用。
一说,旁门也可称仪门。
}众小厮退出,方打起车帘,邢夫人搀了黛玉的手,进入院中。
黛玉度其房屋院宇,必是荣府中之花园隔断过来的。
\jia{黛玉之心机眼力。
}进入三层仪门,果见正房厢庑游廊,\zhu{厢庑:庑音“五”,厢房,指四合院中东西两边的房子。
游廊:连接两个或几个独立建筑物的走廊。
}悉皆小巧别致,\meng{分别得历历,可想如见。
}不似方才那边轩峻壮丽,且院中随处之树木山石皆有。
\jia{为大观园伏脉。
试思荣府园今在西,后之大观园偏写在东,何不畏难之若此?}一时进入正室,早有许多盛妆丽服之姬妾丫鬟迎着。
邢夫人让黛玉坐了,一面命人到外面书房去请贾赦。
\jia{这一句都是写贾赦,妙在全是指东击西打草惊蛇之笔。
\zhu{这条评语的意思是,这里通过“盛妆丽服之姬妾丫鬟”,暗示贾赦的好色,为后文贾赦强娶鸳鸯埋下伏笔。}
若看其写一人即作此一人看,先生便呆了。
}一时人来回说:“老爷说了:‘连日身上不好,见了姑娘彼此倒伤心,\jia{追魂摄魄。
}\jia{余久不作此语矣,见此语未免一醒。
}暂且不忍相见。
\jia{若一见时,不独死板,且亦大失情理,亦不能有此等妙文矣。
}\meng{作者绣口锦心,见有见的亲切,不见有不见的亲切,直说横讲,一毫不爽。
}劝姑娘不要伤心想家,\meng{亦在情理之内。
}跟着老太太和舅母,即同家里一样。
姊妹们虽拙,大家一处伴着,亦可以解些烦闷。
\jia{赦老亦能作此语,叹叹!}
或有委屈之处,只管说得,不要外道才是。
’”黛玉忙站起来,一一听了。
再坐一刻,便告辞。
那邢夫人苦留吃过晚饭去,黛玉笑回道:“舅母爱惜赐饭,原不应辞,只是还要过去拜见二舅舅,恐领了赐去不恭,\jia{得体。
}\meng{黛玉之为人,必当有如此身分。
}异日再领,未为不可。
望舅母容谅。
”邢夫人听说,笑道:“这倒是了。
”\ping{邢夫人岂不知林黛玉仍需拜见二舅舅,这里苦留,是客气还是刻意使林黛玉留吃晚饭而不去拜见贾政,给黛玉留下不知礼数,不懂规矩的名声?}遂命两三个嬷嬷,用方才的车好生送了过去,于是黛玉告辞。
邢夫人送至仪门前,又嘱咐众人几句,\meng{又嘱咐了几句,方是舅母的本等。
}眼看着车去了方回来。
\par
一时黛玉进入荣府,下了车。
众嬷嬷引着,便往东转弯,穿过一个东西的穿堂,\jia{这一个穿堂是贾母正房之南者,凤姐处所通者则是贾母正房之北。
}向南大厅之后,仪门内大院落,上面五间大正房,两边厢房鹿顶耳房钻山,\zhu{两边厢房鹿顶耳房钻山:两边的厢房用钻山的方式与鹿顶的耳房相连接。
厢房:指四合院中东西两边的房子。
鹿顶:旧式四合院东西房和南北房连接转角的地方。
耳房:像耳朵一样位置在正房两侧的小房子。
钻山:指山墙上开门或开洞,与相邻的房子或游廊相接。
}四通八达,轩昂壮丽,比贾母处不同。
黛玉便知这方是正经正内室,\ping{大儿子贾赦住在偏室,二儿子贾政却住在正室,前任管家王夫人是贾政的妻子,而现任管家王熙凤是王夫人的侄女又是贾赦的儿媳妇,和两边都有关系。
贾赦外无实权,内不得母亲喜爱,某种意义上也是可怜。
}一条大甬路,\zhu{甬路:庭院中间的通道,多用砖石铺砌而成。
}直接出大门的。
进入堂屋中,抬头迎面先看见一个赤金九龙青地大匾,匾上写着斗大三个字,是“荣禧堂”,\jia{真是荣国府。
}后有一行小字:“某年月日,书赐荣国公贾源。
”又有“万几宸翰之宝”。
\zhu{万几宸翰之宝:这是皇帝印章上的文字。
几:同机。
“万机”即万事形容皇帝政务繁多,“日理万几”的意思。
宸(音“辰”):北宸,即北极星。
皇帝坐北朝南,故以北宸代指皇帝。
翰:墨迹、书法。
宸翰:皇帝的笔迹。
宝:皇帝的印玺。
}大紫檀雕螭案上,设着三尺来高青绿古铜鼎,悬着待漏随朝墨龙大画,\zhu{待漏随朝墨龙大画:待漏:封建时代大臣要在五更前到朝房里等待上朝的时刻。
漏:指“铜壶滴漏”,古代计时器,代指时间。
随朝:按照大臣的班列朝见皇帝。
墨龙大画:巨龙在云雾海潮中隐现的大幅水墨画。
因旧时以龙象征帝王,又画中之“潮”与朝见之“朝”谐音。
隐寓上朝陛见君王之意。
贵族家中悬挂此画以示身份地位之荣耀。
}一边是金蜼彝,\zhu{
蜼:音“伟”,一种长尾猿。
彝:音“夷”,古代青铜器中礼器的通称。
金蜼彝:原为有蜼形图案的青铜祭器,后作贵重陈设品。
}\jia{蜼,音垒。
周器也。
}一边是玻璃\hai。
\zhu{\hai(音“海”):盛酒器。
}\jia{\hai,音海。
盛酒之大器也。
}地下两溜十六张楠木交椅。
又有一副对联,乃是乌木联牌,
\zhu{
乌木联牌:指在乌木板上刻镂文字的对联。
乌木材质坚硬细密,色黑,有光泽,为高级优质木材。
这在当时是十分高级的装饰做法。
}
镶着錾银的字迹,\zhu{錾银:一种银雕工艺。
錾(音“赞”):雕刻。
}\jia{雅而丽,富而文}。
道是:\par
\hop
座上珠玑昭日月,堂前黼黻焕烟霞。
\zhu{珠玑:珍珠,兼喻诗文之美。
黼黻(音“府服”):古代官僚贵族礼服上绣的花纹。
两句形容座中人和堂上客的衣饰华贵:佩带的珠玉如日月般光彩照人,衣服的图饰如烟霞般绚丽夺目。
}\jia{实贴。
}\par
\hop
下面一行小字,道是:“同乡世教弟勋袭东安郡王穆莳拜手书。
\zhu{莳:音“十”。}
”\jia{先虚陪一笔。
}\par
原来王夫人时常居坐宴息,亦不在这正室,\jia{黛玉由正室一段而来,是为拜见政老耳,故进东房。
}只在这正室东边的三间耳房内。
\jia{若见王夫人,直写引至东廊小正室内矣。
\zhu{下一段老嬷嬷引黛玉到了东廊三间小正房内拜见王夫人。}
}于是老嬷嬷引黛玉进东房门来。
临窗大炕上铺着猩红洋罽,\zhu{罽(音“计”):毛织的毯子。
}正面设着大红金钱蟒靠背,石青金钱蟒引枕,\zhu{引枕:坐时搭扶胳膊的一种圆墩形的倚枕。
}秋香色金钱蟒大条褥。
\zhu{秋香色:淡黄绿色。
}两边设一对梅花式洋漆小几。
\zhu{洋漆:也称洋倭漆。指明朝宣德年间从日本传入的彩漆工艺。}
左边几上文王鼎、匙箸、香盒,\zhu{文王鼎:指周代的传国国鼎,此处说的是小型仿古香炉,内烧粉状檀香之类的香料。
匙著:拨弄香灰的用具。
香盒:盛香料的盒子。
}
右边几上汝窑美人觚\zhu{汝窑美人觚(音“孤”):宋代河南汝州窑烧制的一种仿古瓷器。
觚:古代盛酒器,长身细腰,形如美人,故称。
}——内插着时鲜花卉,并茗碗、唾壶等物。
地下面西一溜四张椅上,都搭着银红撒花椅搭,\zhu{
银红:把银朱加在粉红色的颜料里调和而成的颜色。
银朱:硫化汞的通称。无机化合物,鲜红色,粉末状,有毒。用作颜料和药品等。
椅搭:搭在椅上的一种长方形的绣花呢缎饰物。
}底下四副脚踏。
椅子两边,也有一对高几,几上茗碗花瓶俱备。
其馀陈设,自不必细说。
\jia{此不过略叙荣府家常之礼数,特使黛玉一识阶级座次耳,馀则繁。
}老嬷嬷们让黛玉炕上坐,炕沿上却也有两个锦褥对设,黛玉度其位次,便不上炕,只向东边椅子上坐了。
\jia{写黛玉心意。
}\ping{黛玉来到贾府,“步步留心,时时在意”,岂多余之举?老嬷嬷让黛玉炕上坐,而炕上显然是贾政和王夫人的位置,老嬷嬷岂不知贾府礼数,而故意陷害黛玉,诱使她违背礼数,成为笑柄,其用心险恶。
}本房内的丫鬟忙捧上茶来。
黛玉一面吃茶,一面打量这些丫鬟们,\meng{借黛玉眼写三等使婢。
}妆饰衣裙,举止行动,果亦与别家不同。
\par
茶未吃了,只见一个穿红绫袄、青缎掐牙背心的丫鬟\jia{金乎?玉乎?}
\zhu{金:王夫人的丫鬟金钏。玉:王夫人的丫鬟玉钏。}
走来笑说道:\zhu{绫:古代丝织物名。
掐牙:在衣服的滾边上再镶一条极细的滚边。
滚边:在衣服、布鞋等的边缘缝制的一种圆棱的边。
}“太太说,请姑娘到那边坐罢。
”\meng{唤去见,方是舅母,方是大家风范。
}老嬷嬷听了,于是又引黛玉出来,到了东廊三间小正房内。
\ping{黛玉见王夫人何其难也,此非王夫人刻意拿大摆谱耶?}
\ping{根据前一段开头的评语,程序繁琐的原因是先要在东房拜见贾政,然后再去东廊小正房拜见王夫人。}
正面炕上横设一张炕桌,桌上磊着书籍茶具,\zhu{磊(音“洛”):叠放。
}\jia{伤心笔,堕泪笔。
}
靠东壁面西设着半旧青缎靠背引枕。
王夫人却坐在西边下首,
\zhu{下首:同“下手”,位置较卑的一侧,就室内说,一般指靠外的或者靠右的(左右以人在室内而脸朝外时为准)。}
亦是半旧青缎靠背坐褥。
见黛玉来了,便往东让。
黛玉心中料定这是贾政之位。
\jia{写黛玉心到眼到,伧夫但云为贾府叙坐位,\zhu{伧:音“仓“,庸俗鄙贱。}岂不可笑?}\ping{此客气耶?故意使黛玉违背礼数耶?}因见挨炕一溜三张椅子上,也搭着半旧的\jia{三字有神。
}弹墨椅袱,\zhu{弹墨椅袱:以纸剪镂空图案覆于织品上,用墨色或其它颜色弹或喷成各种图案花样,叫弹墨。
椅袱:用棉、缎之类做成的椅套。
}\jia{此处则一色旧的,可知前正室中亦非家常之用度也。
可笑近之小说中,不论何处,则曰商彝周鼎、绣幕珠帘、孔雀屏、芙蓉褥等样字眼。
}黛玉便向椅上坐了。
\jia{近闻一俗笑语云:一庄农人进京回家,众人问曰:“你进京去可见些个世面否?”庄人曰:“连皇帝老爷都见了。
”众罕然问曰:“皇帝如何景况?”庄人曰:“皇帝左手拿一金元宝,右手拿一银元宝,马上捎着一口袋人参,行动人参不离口。
一时要屙屎了,连擦屁股都用的是鹅黄缎子,所以京中掏茅厕的人都富贵无比。
”试思凡稗官写富贵字眼者,悉皆庄农进京之一流也。
盖此时彼实未身经目睹,所言皆在情理之外焉。
}\jia{又如人嘲作诗者亦往往爱说富丽话,故有“胫骨变成金玳瑁,眼睛嵌作碧琉璃”之诮。
\zhu{
诮:音“翘”,责备,讥讽。
胫:音“净”,从膝盖到脚跟的部分,俗称为「小腿」。
玳瑁:龟类动物,甲壳可作酒器或装饰品,甚名贵。
}
余自是评《石头记》,非鄙薄前人也。
}王夫人再四携他上炕,他方挨王夫人坐了。
\ping{此客气耶?故意使黛玉违背礼数耶?}王夫人因说:“你舅舅今日斋戒去了,\zhu{斋戒:古人在祭祀、礼佛或举行隆重大典前,沐浴、吃素、静养一至三日,摒除杂念,以示诚敬,叫斋戒。
}\jia{点缀宦途。
}再见罢。
\jia{赦老不见,又写政老。
政老又不能见,是重不见重,犯不见犯。
作者惯用此等章法。
}只是有一句话嘱咐你:你三个姊妹倒都极好,以后一处念书认字学针线,或是偶一顽笑,都有尽让的。
\zhu{尽让:使别人占先,谦让。
}但我不放心的最是一件:\meng{王夫人嘱咐与邢夫人嘱咐,似同\sout{的}[而]迥异。
儿女累心,我欲代伊哭诉一面愁苦。
}我有一个孽根祸胎,\jia{四字是血泪盈面,不得已无奈何而下。
四字是作者痛哭。
}是这家里的‘混世魔王’,\jia{与“绛洞花王”为对看。
}\ping{似是贬言,实为爱语。
}今日因庙里还愿去了,\zhu{还愿:求神保佑的人实践对神许下的报酬,如祭祀、慈善、捐献。
}\jia{是富贵公子。
}尚未回来,晚间你看见便知。
你只以后不要睬他,你这些姊妹都不敢沾惹他的。
”\par
黛玉亦常听得母亲说过,二舅母生的有个表兄,乃衔玉而诞,顽劣异常,\jia{与甄家子恰对。
}极恶读书,\jia{是极恶每日“诗云”“子曰”的读书。
}
最喜在内帏厮混,\zhu{内帏:即内室,女子的居处。
帏:幕帐。
}外祖母又极溺爱,无人敢管。
\zhu{外祖母:黛玉的外祖母、贾宝玉的祖母,即贾母。}
今见王夫人如此说,便知说的是这表兄了。
\jia{这是一段反衬章法。
[与]黛玉心\sout{用}[内]猜度“蠢物”等句对\sout{着}[看]去,方不失作者本旨。
\zhu{本回后文:黛玉心中正疑惑着:“这个宝玉,不知是怎生个惫懒人物、懵懂顽童?倒不见那蠢物也罢了。}
}因陪笑道:“舅母说的,可是衔玉所生的这位哥哥?在家时亦曾听见母亲常说,\meng{有曾听得,所以闻言便知不必用心搜求了。
}这位哥哥比我大一岁,小名就唤宝玉,\jia{以黛玉道宝玉名,方不失正文。
}\ping{宝玉是小名,那大名为何?全书也没有提到,可能作者就是宝玉的原型,全书有作者自传性质,以自己的口吻叙事,称呼自己只用小名。
}虽\jia{“虽”字是有情字,宿根而发,勿得泛泛看过。
}极憨顽,说在姊妹情中极好的。
\meng{黛玉口中心中早中此。
\zhu{中:切合、符合、正对上。如:“中规中矩”、“中意”。}
}况我来了,自然只和姊妹同处,兄弟们自是别院另室的,\jia{又登开一笔,妙妙!}岂得去沾惹之理?”\meng{用黛玉反衬一句,更有深味。
}\ping{由后文可知,贾母把宝玉黛玉两人都安排在自己的屋子里,“一桌子吃饭,一床上睡觉”。
并非是”别院另室“。
}王夫人笑道:“你不知道原故。
他与别人不同,自幼因老太太疼爱,原系同姊妹一处娇养惯了的。
\jia{此一笔收回,是明通部同处原委也。
}若姊妹们有日不理他,他倒还安静些,纵然他没趣,不过出了二门,背地里拿着他的两三个小幺儿出气,\zhu{
幺(音“妖”):幼小。
小幺儿:身边使唤的小仆人。
}咕唧一会子就完了。
\jia{这可是宝玉本性真情,前四十九字迥异之批今始方知。
\zhu{
甲戌本第三回侧批所说“前四十九字迥异之批今始方知”,遍查各本第一、二、三回,都无四十九字评贾宝玉的批语。
甲戌本同回有一条眉批“不写黛玉眼中之宝玉,却先写黛玉心中已毕有一宝玉矣,幻妙之至。
只冷子兴口中之后余已极思欲一见,及今尚未得见,狡猾之至。”
共五十二字。在戚序本中,“毕”作“早”,文气通顺,缺“冷子兴”三字,刚好四十九字,
但有点没头没脑,这里录以备考。也许这四十九字的批语,被抄写者漏抄了。
}
盖小人口碑累累如是。
是是非非任尔口角,大都皆然。
}若这一日姊妹们和他多说一句话,他心里一乐,便生出多少事来。
所以嘱咐你别睬他。
他嘴里一时甜言蜜语,一时有天无日,一时又疯疯傻傻,只休信他。
”\ping{混世魔王会不会搅个天翻地覆呢?一笑。
}\par
黛玉一一的都答应着。
\jia{不写黛玉眼中之宝玉,却先写黛玉心中已早有一宝玉矣,幻妙之至!自冷子兴口中之后,余已极思欲一见,及今尚未得见,狡猾之至!}\meng{客居之苦,在有意无意中写来。
}只见一个丫鬟来回:“老太太那里传晚饭了。
”王夫人忙携了黛玉从后房门\jia{后房门。
}由后廊\jia{是正房后廊也。
}往西,出了角门,\jia{这是正房后西界墙角门。
}是一条南北宽夹道。
南边是倒座三间小小的抱厦厅,\zhu{倒座:“倒座”是与正房相对、朝向相反的房子。
抱厦厅:回绕堂屋(堂屋:正房居中的一间,也泛指正房)后面的侧室。
}北边立着一个粉油大影壁,\zhu{影壁:俗称照墙。
于门内或门外用作屏障或装饰。
}后有一半大门,小小一所房宇。
王夫人笑指向黛玉道:“这是你凤姐姐的屋子,回来你好往这里找他来,\meng{灵活无一漏空。
}少什么东西,你只管和他说就是了。
”这院门上也有\jia{二字是他处不写之写也。
}四五个才总角的小厮,\zhu{总角:儿童向上分开的两个发髻。
代指儿童时代。
}
都垂手侍立。
王夫人遂携黛玉穿过一个东西穿堂,\jia{这正是贾母正室后之穿堂也,与前穿堂是一带之屋,中一带乃贾母之下室也。
记清。
}便是贾母的后院了。
\jia{写得清,一丝不错。
}于是,进入后房门,已有多人在此伺候,见王夫人来了,方安设桌椅。
\jia{不是待王夫人用膳,是恐使王夫人有失侍膳之礼耳。
}贾珠之妻李氏捧饭,熙凤安箸,王夫人进羹。
\meng{大人家规矩礼法。
}
贾母正面榻上独坐,两边四张空椅,熙凤忙拉了黛玉在左边第一张椅上坐了,黛玉十分推让。
贾母笑道:“你舅母和嫂子们不在这里吃饭。
你是客,原应如此坐的。
”黛玉方告了座,\zhu{告座:同“告坐”,上级或长辈让下级或晚辈坐,下级或晚辈谦让或道谢后坐下。
}坐了。
贾母命王夫人坐了。
迎春姊妹三个告了座,方上来。
迎春便坐右手第一,探春左第二,惜春右第二。
旁边丫鬟执着拂尘、漱盂、巾帕。
\zhu{拂尘:形如马尾,后有持柄,用以拂拭尘土,或驱赶蝇蚊,俗称“蝇甩子”。
古时多用麈(麈:音“主”,一种似骆驼的鹿类动物)兽之尾制成,故又称麈尾。
}李、凤二人立于案旁布让。
\zhu{布让:宴席间向客人敬菜、劝餐叫布让。
}外间伺候之媳妇丫鬟虽多,却连一声咳嗽不闻。
寂然饭毕,各有丫鬟用小茶盘捧上茶来。
\meng{作者非身履其境过,不能如此细密完足。
}当日林如海教女以惜福养身,云饭后务待饭粒咽尽,过一时再吃茶,方不伤脾胃。
\jia{夹写如海一派书气,最妙!}今黛玉见了这里许多事情不合家中之式,不得不随的,少不得一一的改过来,\meng{幼而学,壮而行者常情。
有不得已,行权达变,多至于失守者,亦千古同慨,诚可悲夫。
}因而接了茶。
早见人又捧过漱盂来,黛玉也照样漱了口。
然后盥手毕,又捧上茶来,方是吃的茶。
\jia{总写黛玉以后之事,故只以此一件小事略为一表也。
}\jia{今看至此,故想日前所阅“王敦初尚公主,登厕时不知塞鼻用枣,敦辄取而啖之,早为宫人鄙诮多矣”。
今黛玉若不漱此茶,或饮一口,不为荣婢所诮乎?观此则知黛玉平生之心思过人。
}贾母便说:“你们去罢,让我们自在说话儿。
”王夫人听了,忙起身,又说了两句闲话,方引李、凤二人去了。
贾母因问黛玉念何书。
黛玉道:“只刚念了《四书》。
”\zhu{《四书》:《大学》、《中庸》、《论语》、《孟子》合称为《四书》,宋代朱熹把它们编在一起,作《四书章句集注》,故有此称。
是元、明、清三代科举考试的必读之书。
}\jia{好极!稗官专用“腹隐五车书”者来看。
}黛玉又问姊妹们读何书。
贾母道:“读的是什么书,不过是认得两个字,不是睁眼的瞎子罢了!”\ping{贾母不识字,证据在三十八回“(贾母)一面说,一面又看见柱上挂的黑漆嵌蚌的对子,命人念。
湘云念道……”,贾母自己是“睁眼瞎”,这里是懊悔自己没能读书识字,还是自嘲呢?后文贾府孙女才华虽不及薛林,但是也不可能只“认得两个字”,这里也有贾母在黛玉面前对贾府孙女的谦抑。
}\par
一语未了,只听院外一阵脚步响,\jia{与阿凤之来相映而不相犯。
}丫鬟进来笑道:“宝玉来了!”\jia{余为一乐。
}\meng{形容出姣养,神。
}黛玉心中正疑惑着:“这个宝玉,不知是怎生个惫懒人物、懵懂顽童?\zhu{惫懒:涎皮赖脸的意思。
涎[xián]皮赖脸:形容不顾别人厌恶,厚着脸皮跟人纠缠的样子。
}\jia{文字不反,不见正文之妙,似此应从《国策》得来。
\zhu{
《国策》:即《战国策》,为战国末年和秦汉间人所纂集,是一部记述战国时游说之士的策谋和言论的汇编。
脂批认为,黛玉疑惑宝玉这一句是先反后正,先抑后扬,先贬后褒,是从《战国策》中得来的。
}
}\meng{从黛玉口中故反一句,则下文更觉生色。
}倒不见那蠢物也罢了。
\zhu{“倒不见那蠢物也罢了”:此句疑非正文。
“蠢物”是叙述者的口气,用在此处有调侃的意味;黛玉从未见过宝玉就不想见他也不合情理。
但此句后已有批语,则它可能是早期批语甚或作者自批而混入正文的。
}”\jia{这蠢物不是那蠢物,却有个极蠢之物相待。
妙极!}心中正想着,忽见丫鬟话未报完,已进来了一个年轻\foot{甲戌本原作“轻年”,他本均作“年轻”。
按“轻年”与“年轻”义同,而书中他处多作“年轻”,故予统一。
庚本五十五回另一处“轻年”仿此。
}公子:头上戴着束发嵌宝紫金冠,齐眉勒着二龙抢珠金抹额,\zhu{紫金冠:把头发束扎在顶部的一种髻冠,上面插戴各种饰物或镶嵌珠玉。
抹额:围扎在额前,用以压发、束额。
二龙抢珠:是抹额上的装饰图案。
}穿一件二色金百蝶穿花大红箭袖,\zhu{二色金百蝶穿花大红箭袖:用两色金线绣成的百蝶穿花图案的大红窄袖衣服。
箭袖:原为便于射箭穿的窄袖衣服,这里指男子穿的一种服式。
}束着五彩丝攒花结长穗宫绦,\zhu{长穗宫绦:指系在腰间的绦带。
长穗:是绦带端部下垂的穗子。
五彩丝攒花结:用五彩丝攒聚成花朵的结子,指绦带上的装饰花样。
}外罩石青起花八团倭缎排穗褂,\zhu{石青:淡灰青色。
团:圆形团花。
因其凸出,故云“起花”。
倭缎:又称东洋缎。
排穗:排缀在衣服下面边缘的彩穗。
}登着青缎粉底小朝靴。
\zhu{青缎:黑色的缎子。
朝靴:古代百官穿的“乌皮履”。
这里指黑色缎面、白色厚底、半高筒的靴子。
}面若中秋之月,\jia{此非套“满月”,盖人生有面扁而青白色者,则皆可谓之秋月也。
用“满月”者不知此意。
}色如春晓之花。
\jia{“少年色嫩不坚牢”,以及“非夭即贫”之语,
\zhu{夭:音“腰”,未成年就短命早死。如:「夭折」、「夭寿」。}
余犹在心。
今阅至此,放声一哭。
}鬓如刀裁,眉如墨画,眼似桃瓣,睛若秋波。
虽怒时而若笑,即嗔视而有情。
\jia{真真写杀。
}项上金螭璎珞,
\zhu{
螭(音“吃”):古代传说中的无角龙。
璎珞:联缀起来的珠玉。
}
又有一根五色丝绦,系着一块美玉。
黛玉一见,\qi{写宝玉只是宝玉,写黛玉只是黛玉,从中用黛玉一惊宝玉之面善等字,文气自然笼就,要分开不得了。
}便吃一大惊,\jia{怪甚。
}心下想道:“好生奇怪,倒像在那里见过的一般,\meng{此一惊,方下文之留连缠绵,不为猛浪,不是淫邪。
}何等眼熟到如此!”\jia{正是。
想必在灵河岸上三生石畔曾见过。
}只见这宝玉向贾母请了安,\zhu{请安:即问安。
清代的请安礼节是,男子打千,即右膝半跪,较隆重时双膝跪下,女子双手扶左膝,右腿微屈,往下蹲身,口称“请某人安”。
}贾母便命:“去见你娘来。
”宝玉即转身去了。
一时回来,再看,已换了冠带:头上周围一转的短发,
\zhu{一转:即“一圈”。}
都结成了小辫,红丝结束,共攒至顶中胎发,
\zhu{胎发:小儿初生时的毛发。}
总编一根大辫,黑亮如漆,从顶至梢,一串四颗大珠,用金八宝坠角,\zhu{坠角:用于朝珠、床帐等下端起下垂作用的小装饰品,这里是指辫子梢部所坠的饰物。
}身上穿着银红撒花半旧大袄,仍旧带着项圈、宝玉、寄名锁、护身符等物,\zhu{寄名锁:旧时怕幼儿夭亡,给寺院或道观一定财物,让幼儿当“寄名”弟子,并在幼儿的项下系一小金锁,名“寄名锁”。
护身符:是从道观领来的一种符箓,带在身上,避祸免灾。
皆为迷信习俗。
符箓:亦作“符录”,音“符录”,道士巫师所画的一种图形或线条,相传可以役鬼神,辟病邪。
}下面半露松花撒花绫裤腿,\zhu{松花:偏黑的深绿色。
绫:古代丝织物名。
}
锦边弹墨袜,
\zhu{弹墨:以纸剪镂空图案覆于织品上,用墨色或其它颜色弹或喷成各种图案花样。}
厚底大红鞋。
越显得面如敷粉,唇若施脂;转盼多情,语言常笑。
天然一段风骚,全在眉梢;平生万种情思,悉堆眼角。
\meng{总是写宝玉,总是为下文留地步。
}看其外貌最是极好,却难知其底细。
后人有《西江月》二词,\zhu{《西江月》二词:这两首词用似贬实褒、寓褒于贬的手法揭示了贾宝玉的性格。
}批这宝玉极恰,\jia{二词更妙。
最可厌野史“貌如潘安”“才如子建”等语。
}其词曰:\par
\hop
无故寻愁觅恨,有时似傻如狂。
纵然生得好皮囊,\zhu{皮囊:一作“皮袋”,指人的躯壳。
}腹内原来草莽。
\zhu{草莽:丛生的杂草,喻不学无术。
} 潦倒不通世务,愚顽怕读文章。
\zhu{文章:此指四书五经及时文八股之类。
}行为偏僻性乖张,那管世人诽谤!\par
富贵不知乐业,\zhu{乐业:这里是满意、安于富贵的意思。
}贫穷难耐凄凉。
可怜辜负好韶光,于国于家无望。
 天下无能第一,古今不肖无双。
寄言纨袴与膏粱:\zhu{纨袴(袴同裤):代指富家子弟。
纨:素色细绢。
}莫效此儿形状!\jia{末二语最要紧。
只是纨绔膏粱,亦未必不见笑我玉卿。
可知能效一二者,亦必不是蠢然纨绔矣。
\zhu{
这条评语的意思是,指出《西江月》二词对宝玉明贬实褒,突出了宝玉异于纨绔膏粱的宝贵品质。
}
}\qi{纨袴膏粱,此儿形状有意思。
当设想其像,合宝玉之来历同看,方不被作者愚弄。
}\par
\hop
贾母因笑道:“外客未见,就脱了衣裳,还不去见你妹妹!”宝玉早已看见多了一个姊妹,便料定是林姑母之女,忙来作揖。
厮见毕归坐,\zhu{厮:互相。
}
细看形容,\jia{又从宝玉目中细写一黛玉,直画一美人图。
}与众各别:两弯似蹙非蹙罥烟眉,\zhu{罥烟眉:形容眉毛像一抹轻烟。
罥(音“绢”):挂的意思。
}\jia{奇眉妙眉,奇想妙想。
}一双似泣非泣含露目\foot{原作“两湾似蹙非蹙笼烟眉,一双似〼非〼〼〼〼”,后一句他本异文较多,己本作“似笑非笑含露目”,杨本作“似喜非喜含情目”,此据列本。
列本此语公认较佳,但也未必是作者原拟文字。
后文第二十三回写黛玉“竖起两道似蹙非蹙的眉,瞪了两只似睁非睁的眼”,可参看。
}。
\jia{奇目妙目,奇想妙想。
}态生两靥之愁,娇袭一身之病。
\zhu{
态:情态,风韵。
靥(音“夜”):面颊上的酒涡。
袭:承继,由……而来。
态生两靥之愁,娇袭一身之病:意思是妩媚的风韵生于含愁的面容,娇怯的情态出于孱弱的病体。
}泪光点点,娇喘微微。
闲静时如娇花照水,行动处似弱柳扶风。
\jia{至此八句是宝玉眼中。
}心较比干多一窍,\jia{此一句是宝玉心中。
}\jia{更奇妙之至!多一窍固是好事,然未免偏僻了,所谓“过犹不及”也。
}\meng{写黛玉,也是为下文留地步。
}
\zhu{
比干:商(殷)代纣王的叔父。
《史记·殷本纪》载:纣王淫乱,“比干曰:‘为人臣者,不得不以死争。
’乃强谏纣。
纣怒曰:‘吾闻圣人心有七窍。
’剖比干,观其心。
”古人认为心窍越多越有智慧。
这句极言林黛玉聪明颍悟。
}
病如西子胜三分。
\zhu{
西子:即西施,相传“西施病心而颦(颦:皱眉)”,益增妩媚。
见《庄子·天运》。
这句是说黛玉病弱娇美胜过西施。
}\jia{此十句定评,直抵一赋。
}\jia{不写衣裙妆饰,正是宝玉眼中不屑之物,故不曾看见。
黛玉之举止容貌,亦是宝玉眼中看、心中评。
若不是宝玉,断不能知黛玉是何等品貌。
}宝玉看罢,因笑\jia{黛玉见宝玉写一“惊”字,宝玉见黛玉写一“笑”字,一存于中,一发乎外,可见文于下笔必推敲的准稳,方才用字。
}道:\jia{看他第一句是何话。
}“这个妹妹我曾见过的。
”\jia{疯话。
与黛玉同心,却是两样笔墨。
观此则知玉卿心中有则说出,一毫宿滞皆无。
}贾母笑道:“可又是胡说,你又何曾见过他?”宝玉笑道:“虽然未曾见过他,然我看着面善,心里就算是旧相识,\jia{一见便作如是语,宜乎王夫人谓之疯疯傻傻也。
}
\meng{世人得遇相好者,每曰一见如故,与此一意。
}今日只作远别重逢,未为不可。
”\jia{妙极奇语,全作如是等语。
[焉]怪人谓曰痴狂。
}\jia{作小儿语瞒过世人亦可。
}
\zhu{
远别重逢:三生石畔,赤瑕宫神瑛侍者灌溉绛珠草。
后神瑛侍者下凡为贾宝玉,绛珠草下凡为林黛玉,此为宝黛二人的前世缘分。
}
贾母笑道:“更好,更好。
若如此,更相和睦了。
”\jia{亦是真话。
}宝玉便走近黛玉身边坐下,又细细打量一番,\jia{与黛玉两次打量一对。
}
\meng{姣惯处如画。
如此亲近,而黛玉之灵心巧性,能不被其缚住,反不是性理。
文从宽缓中写来,妙!}因问:“妹妹可曾读书?”\jia{自己不读书,却问到人,妙!}
黛玉道:“不曾读书,只上了一年学,些须认得几个字。
”\ping{前文:“贾母因问黛玉念何书。
黛玉道:‘只刚念了《四书》。
’黛玉又问姊妹们读何书。
贾母道:‘读的是什么书,不过是认得两个字,不是睁眼的瞎子罢了!’”,此处改口为“不曾读书,只上了一年学,些须认得几个字”,可见黛玉处处留心,当贾母说贾府孙女“只认得几个字”后,黛玉尽量要和大家一样而不是显示出不同,避免刻意卖弄学问之嫌。
这个对比还可能暗示了林家和贾家的差异,林家“虽系钟鼎之家,却亦是书香之族”,重视教育,为林黛玉聘请家庭教师,而贾家的子弟如贾珍“那肯读书,只是一味高乐不已”,这就导致了从重视教育的林家来到不那么重视教育的贾家的林黛玉有个逐渐适应的过程。
}宝玉又道:“妹妹尊名是那两个字?”黛玉便说了名。
宝玉又问表字,黛玉道:“无字。
”宝玉笑道:“我送妹妹一个妙字,莫若‘颦颦’二字极好。
”探春\jia{写探春。
}便问何出。
\meng{借问难说探春,
\zhu{难:音四声。问难:辩论诘问。}
以足后文。
}宝玉道:“《古今人物通考》上说:\zhu{《古今人物通考》:	从下文来看,可能是宝玉的杜撰。
}‘西方有石名黛,可代画眉之墨。
’况这林妹妹眉尖若蹙,用取这两个字,岂不两妙!”\meng{黛玉泪因宝玉,而宝玉赠曰颦颦,初见时亦定盟矣。
}探春笑道:“只恐又是你的杜撰。
”宝玉笑道:“除《四书》外,杜撰的太多,偏只我是杜撰不成?”\jia{如此等语,焉得怪彼世人谓之怪?只瞒不过批书者。
}又问黛玉:“可也有玉没有?”\jia{奇极怪极,痴极愚极,焉得怪人目为痴哉?}众人不解其语,黛玉便忖度着:“因他有玉,故问我也有无。
”\jia{奇之至,怪之至,又忽将黛玉亦写成一极痴女子,观此初会二人之心,则可知以后之事矣。
}因答道:“我没有那个。
想来那玉亦是一件罕物,岂能人人有的。
”宝玉听了,登时发作起痴狂病来,摘下那玉,就狠命摔去,\jia{试问石兄:此一摔,比在青埂峰下萧然坦卧何如?}骂道:“什么罕物,连人之高低不择,还说‘通灵’不‘通灵’呢!我也不要这劳什子了!”\zhu{劳什子:如同说“东西”、“玩意”,含有厌恶之意。
}吓的地下众人一拥争去拾玉。
贾母急的搂了宝玉道:“孽障!\jia{如闻其声,恨极语却是疼极语。
}你生气,要打骂人容易,何苦摔那命根子!”\jia{一字一千斤重。
}宝玉满面泪痕泣\jia{千奇百怪,不写黛玉泣,却反先写宝玉泣。
}道:“家里姐姐妹妹都没有,单我有,我就没趣,\meng{不是写宝玉狂,\sout{下}[亦]不是写贾母疼,总是要下种在黛玉心里,则下文写黛玉之近宝玉之由,作者苦心,妙妙。
}如今来了这么一个神仙似的妹妹也没有,可知这不是个好东西!”\meng{“不是冤家不聚头”第一场也。
}贾母忙哄他道:“你这妹妹原有这个来的,因你姑妈去世时,舍不得你妹妹,无法可处,遂将他的玉带了去了。
一则全殉葬之礼,\zhu{殉葬:古代把活人或器物随同死者埋在墓中叫“殉葬”。
}尽你妹妹之孝心,二则你姑妈之灵,亦可权作见了女儿之意。
因此他只说没有这个,不便自己夸张之意。
\meng{不如此说则不为姣养,文灵活之至。
}你如今怎比得他?还不好生慎重带上,仔细你娘知道了。
”说着,便向丫鬟手中接来,亲与他带上。
宝玉听如此说,想一想竟大有情理,也就不生别论了。
\jia{所谓小儿易哄,余则谓“君子可欺以其方”云。
}
\zhu{君子可欺以其方:参见本回前面的注释。}
\par
当下,奶娘来请问黛玉之房舍。
贾母说:“今将宝玉挪出来,同我在套间暖阁儿里面,\zhu{套间:与正房相连的两侧房间。
暖阁:是指在套间内再隔断成为小房间,内设炕褥,两边安有隔扇,上边有一横眉,形成床帐的样子,称“暖阁”。
}把你林姑娘暂安置碧纱橱里。
\zhu{
碧纱橱:装在房内起隔开作用的一扇一扇的木板墙,也称“隔扇”、“槅扇”。中间两扇平日可以开关,或加挂帘子帷帐,又叫“纱橱”、“纱厨”。
槅心部分常糊以绿纱,故称碧纱橱\foot{\footPic{碧纱橱}{bishachu.png}{0.8}}。
这里的“碧纱橱里”,是指以碧纱橱隔开的里间。
}等过了残冬,春天再与他们收拾房屋,另作一番安置罢。
”\meng{女死外孙女来,不得不令其近己,移疼女之心疼外孙女者当然。
}宝玉道:“好祖宗,\jia{跳出一小儿。
}我就在碧纱橱外的床上很妥当,何必又出来闹的老祖宗不得安静。
”贾母想了一想说:“也罢了。
”每人一个奶娘并一个丫头照管,\meng{小儿不禁情事,无违,下笔运用有法。
}馀者在外间上夜听唤。
一面早有熙凤命人送了一顶藕合色花帐,\zhu{藕合:也作“藕荷”,形容颜色浅紫而微微发红。
}并几件锦被缎褥之类。
\par
黛玉只带了两个人来:一个是自幼奶娘王嬷嬷,一个是十岁的小丫头,亦是自幼随身的,名唤雪雁。
\jia{新雅不落套,是黛玉之文章也。
}贾母见雪雁甚小,一团孩气,王嬷嬷又极老,料黛玉皆不遂心省力的,便将自己身边一个二等的丫头,名唤鹦哥\jia{妙极!此等名号方是贾母之文章。
最厌近之小说中,不论何处,满纸皆是红娘、小玉、嫣红、香翠等俗字。
}者与了黛玉。
\zhu{
鹦哥:这应该就是紫鹃了。
第八回紫鹃出场后有评语“鹦哥改名也。”
因为这个鹦哥,除了贾母刚给了黛玉,
在宝玉摔玉后说了句“林姑娘在这里伤心……”的话后,便没有露面。
在四十六回鸳鸯对平儿提到大观园里自小儿在一起的丫头有
“袭人、琥珀、素云、紫鹃、彩霞、玉钏儿、麝月、翠墨,跟了史姑娘去的翠缕,死了的可人和金钏,去了的茜雪”
而语不及鹦哥来判断,鹦哥应该就是紫鹃。
鹦哥为什么改名紫鹃呢?鹦哥这名字,恐有鹦鹉学舌,让她向史太君汇报黛玉情况的意思,
此正唐诗中宫人在“鹦鹉前头不敢言”之谓也。
而紫鹃其名,正为杜鹃啼血,以应潇湘妃子千般泪水滴成红血的意思吧。
}
外亦如迎春等例,每人除自幼乳母外,另有四个教引嬷嬷,\zhu{教引嬷嬷:清代皇子一落生,即有保母、乳母各八人;断乳后,增“谙达”,“凡饮食、言语、行步、礼节皆教之。
”(见《清裨类钞》)贵族家庭的“教引嬷嬷”,其职务与皇宫的“谙达”近似。
}除贴身掌管钗钏盥沐两个丫鬟外,另有五六个洒扫房屋来往使役的小丫鬟。
当下,王嬷嬷与鹦哥陪侍黛玉在碧纱橱内。
宝玉之乳母李嬷嬷,并大丫鬟名唤袭人\jia{奇名新名,必有所出。
}者,陪侍在外面大床上。
\par
原来这袭人亦是贾母之婢,本名珍珠。
\jia{亦是贾母之文章。
前鹦哥已伏下一鸳鸯,今珍珠又伏下一琥珀矣。
以下乃宝玉之文章。
}\meng{袭人之情性,不得不点染明白者,为后日旧案。
}贾母因溺爱宝玉,生恐宝玉之婢无竭力尽忠之人,素喜袭人心地纯良,克尽职任,遂与了宝玉。
\meng{贾母爱孙,锡以善人,
\zhu{锡:音“次”,赐。}
此诚为能爱人者,非世俗之爱也。
}宝玉因知他本姓花,又曾见旧人诗句上有“花气袭人”之句,\zhu{“花气袭人”句:全句为“花气袭人知骤暧”,见于宋代陆游诗《村居喜书》。
意思是花香扑人.知道天气骤然和暖了。
}遂回明贾母,即更名袭人。
这袭人亦有些痴处:\jia{只如此写又好极!最厌近之小说中,满纸“千伶百俐”“这妮子亦通文墨”等语。
}
\meng{世人有职任的,能如袭人,则天下幸甚。
}伏侍贾母时,心中眼中只有一个贾母,今与了宝玉,心中眼中又只有个宝玉。
只因宝玉性情乖僻,每每规谏,宝玉不听,心中着实忧郁。
\meng{我读至此,不觉放声大哭。
}\par
是晚,宝玉、李嬷嬷已睡了,他见里面黛玉和鹦哥犹未安息,他自卸了妆,悄悄进来,笑问:“姑娘怎么还不安息?”黛玉忙笑让:“姐姐请坐。
”袭人在床沿上坐了。
鹦哥笑道:“林姑娘正在这里伤心,\jia{可知前批不谬。
}自己淌眼抹泪\jia{黛玉第一次哭却如此写来。
}\jia{前文反明写宝玉之哭,今却反如此写黛玉,几被作者瞒过。
}
\jia{
这是第一次算还,不知下剩还该多少?
\zhu{
林黛玉的前世是绛珠仙子,受到贾宝玉的前世神瑛侍者的甘露灌溉之惠。
下凡后,林黛玉要用一生所有的眼泪报答偿还。
}
}
的说:‘今儿才来了,就惹出你家哥儿的狂病来,倘或摔坏那玉,岂不是因我之过!’\jia{所谓宝玉知己,全用体贴功夫。
}\meng{我也心疼,岂独颦颦!}因此便伤心,我好容易劝好了。
”袭人道:“姑娘快休如此,将来只怕比这个更奇怪的笑话儿还有呢!若为他这种行止,你多心伤感,只怕你伤感不了呢。
快别多心!”\meng{后百十回黛玉之泪,总不能出此二语。
}
\meng{
“月上窗纱人到阶,窗上影儿先进来”,笔未到而境先到矣。
}\chen{应知此非伤感,来还甘露水也。
}黛玉道:“姐姐们说的,我记着就是了。
究竟不知那玉是怎么个来历?上头还有字迹?”袭人道:“连一家也不知来历。
听得说,落草时从他口里掏出,\zhu{落草:“妇人分娩曰坐草”。
引申其义,小儿落生叫“落草”。
}上头有现成的穿眼。
\jia{癞僧幻术亦太奇矣。
}
\meng{天生带来美玉,有现成可穿之眼,岂不可爱,岂不可惜!}让我拿来你看便知。
”黛玉忙止道:“罢了,此刻夜深,明日再看也不迟。
”\jia{总是体贴,不肯多事。
}\meng{他天生带来的美玉,他自己不爱惜,遇知己替他爱惜,连我看书的人也着实心疼不了,不觉背人一哭,以谢作者。
}大家又叙了一回,方才安歇。
\par
次日起来,省过贾母,\zhu{省(音“醒”):家庭日常礼节。
子女对父母早上问安叫“省”,晚上服侍就寝叫“定”。
见《礼记·曲礼上》:“凡为人子之礼,冬温而夏清(音“净”),昏定而晨省。
”}因往王夫人处来,正值王夫人与熙凤在一处拆金陵来的书信看,又有王夫人之兄嫂处遣了两个媳妇来说话的。
黛玉虽不知原委,探春等却都晓得是议论金陵城中所居的薛家姨母之子、姨表兄薛蟠,倚财仗势,打死人命,现在应天府案下审理。
\meng{作者每用牵前摇后之笔。
}如今母舅王子腾得了信息,故遣人来告诉这边,意欲唤取进京之意。
\meng{㨄下文。
\zhu{
㨄:音“周“,方言,把重物从一侧或一端托起或上掀。 
}
}\par
\qi{总评:补不完的是离恨天,所馀之石岂非离恨石乎。
而绛珠之泪偏不因离恨而落,为惜其石而落。
可见惜其石必惜其人,其人不自惜,而知己能不千方百计为之惜乎?所以绛珠之泪至死不干,万苦不怨。
所谓“求仁而得仁,又何怨”,悲夫!}
\dai{005}{林黛玉弃舟登岸}
\dai{006}{林黛玉初到贾府}
\sun{p3-1}{雨村依附黛玉进京}{黛玉洒泪拜别父亲,随了奶娘及荣府中几个老妇登舟而去。
雨村另有船只,带了两个小童,依附黛玉而行。
}
\sun{p3-2}{林黛玉初到荣国府}{林黛玉见贾母、舅母和姊妹。
王熙凤未见其人先闻其声。
}
\sun{p3-3}{林黛玉拜见二舅舅}{黛玉又往二舅这边来,进入堂屋,大紫檀雕螭案上,设着三尺来高青绿古铜鼎,悬着待漏随朝墨龙大画,一边是金蜼彝,一边是玻璃\hai 。
}
\sun{p3-4}{宝黛初会,宝玉摔玉}{右侧是宝黛初会,宝玉摔玉。左侧可能是幼年宝玉,其项上系着一块美玉。}