\chapter{芦雪广争联即景诗 \quad 暖香坞雅制春灯谜}
\qi{此回着重在宝琴,却出色写湘云。
写湘云联句极敏捷聪慧,而宝琴之联句不少于湘云,可知出色写湘云,正所以出色写宝琴。
出色写宝琴者,全为与宝玉提亲作引也。
金针暗渡,
\zhu{金针暗渡:评点家用来评点小说、戏剧中的巧妙章法和构思。}
不可不知。
}\par
话说薛宝钗道:“到底分个次序,让我写出来。
”说着,便令众人拈阄为序。
起首恰是李氏。
\geng{一定要按次序,恰又不按次序,
\zhu{本回后文宝琴和湘云不管次序抢着联句。}
似脱落处而不脱落,文章歧路如此。
}然后按次各各开出。
凤姐儿说道:“既是这样说,我也说一句在上头。
”众人都笑说道:“更妙了!”宝钗便将稻香老农之上补了一个“凤”字,李纨又将题目讲与他听。
凤姐儿想了半日,笑道:“你们别笑话我。
我只有一句粗话,下剩的我就不知道了。
”众人都笑道:“越是粗话越好,你说了只管干正事去罢。
”凤姐儿笑道:“我想下雪必刮北风。
昨夜听见了一夜的北风,我有了一句,就是‘一夜北风紧’,可使得?”众人听了,都相视笑道:“这句虽粗,不见底下的,
这正是会作诗的起法。
不但好,而且留了多少地步与后人。
\zhu{地步:言语、行动可以回旋的地方。}
就是这句为首,稻香老农快写上续下去。
”凤姐和李婶平儿又吃了两杯酒,自去了。
这里李纨便写了:\par
\hop
一夜北风紧,\par
\hop
自己联道:\par
\hop
开门雪尚飘。
入泥怜洁白,\par
\hop
香菱道:\par
\hop
匝地惜琼瑶。
\zhu{匝:周;遍。
琼瑶:美玉。
入泥怜洁白,匝地惜琼瑶:意谓白雪落入污泥,犹如美玉抛撒遍地,令人怜惜。
}
有意荣枯草,\par
\hop
探春道:\par
\hop
无心饰萎苕。
\zhu{苕[tiáo]:芦苇的花穗。}
价高村酿熟,\zhu{
价高:指酒价高,雪大天寒,故酒价高涨。
村酿:即村酒。
}\par
\hop
李绮道:\par
\hop
年稔府粱饶。
\zhu{
稔[rěn],庄稼成熟。
年稔:年景好;收成好。
这里是说大雪之后将会有一个丰收的年景。
府粱:指官仓中的粮食。
饶:丰富。
}
葭动灰飞管,
\zhu{
葭:音“家”,芦苇。
灰飞管:古代预测节气,把芦苇茎里的薄膜烧制成灰,放入十二乐律的管内,把管放到密室中特制的内低外高的木案上,到了某一节气,相应律管里的葭灰就会自行飞动(见《后汉书·律历志上》)。
}
\par
\hop
李纹道:\par
\hop
阳回斗转杓。
\zhu{
阳回:阳气复回,说明已到“冬至”。
斗:指北斗星。
杓:音“标”,斗杓,北斗七星中第五、六、七颗星的总称,也叫“斗柄”。
由于地球的自转和公转,斗柄的指向和方位不断变动转换。
冬至这天,斗柄指向正北,阴极阳生,自此开始,斗柄即渐向东转,所以说是“阳回”。
葭动灰飞管,阳回斗转杓:这里是用冬至节代指雪天。
意谓乐律管里的葭灰飞动,斗柄已转,正是阴极阳回的冬至节气。
}
寒山已失翠,\par
\hop
岫烟道:\par
\hop
冻浦不闻潮。
易挂疏枝柳,
\zhu{这句指雪凝成冰,冰凌容易挂在稀疏的柳枝上。}
\par
\hop
湘云道:\par
\hop
难堆破叶蕉。
\zhu{这句意谓冬天枯败残破的芭蕉叶上,不易堆积落雪。}
麝煤融宝鼎,\zhu{麝煤:本为香墨的别名,也叫“麝墨”;这里指取暖用的优质木炭之类。
}\par
\hop
宝琴道:\par
\hop
绮袖笼金貂。
\zhu{这句意为笼两袖于貂皮中以御寒。}
光夺窗前镜,\par
\hop
黛玉道:\par
\hop
香粘壁上椒。
\zhu{
壁上椒:以椒涂壁,取其温暖有香气。
汉代后妃住的宫室用花椒和泥涂壁,取其温暖有香气;又因花椒结实多,兼有希求多子之意。
}
斜风仍故故,\zhu{
故故:屡屡。
这里指风吹阵阵。
}\par
\hop
宝玉道:\par
\hop
清梦转聊聊。
\zhu{
聊聊:略微;短暂。
这里指因天冷而梦境不长。
或为“寥寥”之误;寥寥:稀少的意思。
}
何处梅花笛?\zhu{
梅花笛:指吹奏《梅花落》的笛声。
这里又用落梅隐喻飞雪。
南唐李煜《清平乐·忆别》:“砌下落梅如雪乱,拂了一身还满。
”}\par
\hop
宝钗道:\par
\hop
谁家碧玉箫?鳌愁坤轴陷,\zhu{鳌:传说中的大海龟。
《淮南子·览冥训》:“往古之时,四极废,九州裂,天不兼覆,地不周载……于是女娲氏炼五色石以补苍天,断鳌足以立四极。
故有鳌负大地的说法。
”坤轴:即地轴,这里泛指大地。
鳌愁坤轴陷:巨鳌因怕大雪压坍大地而发愁。
}\par
\hop
李纨笑道:“我替你们看热酒去罢。
”宝钗命宝琴续联,只见湘云站起来道:\par
\hop
龙斗阵云销。
\zhu{
阵云销:浓云消散,表示龙战已毕。
龙斗阵云销:以玉龙战罢鳞片纷飞的景象,比喻大雪纷飞。
}
野岸回孤棹,\par
\hop
宝琴也站起道:\par
\hop
吟鞭指灞桥。
\zhu{
“吟鞭”句:意谓雪中行吟,诗思益增。
南宋尤袤《全唐诗话》:“相国郑綮[qìng]善诗。
或曰:‘相国近为新诗否?’对曰:‘诗思在灞桥风雪中驴子上,此何以得之?’灞桥:在今西安市东灞水上。
}
赐裘怜抚戍,\zhu{
“赐裘”句:意谓皇帝怜恤守边将士,赏给他们过冬衣裘。
}\par
\hop
湘云那里肯让人,且别人也不如他敏捷,都看他扬眉挺身的说道:\par
\hop
加絮念征徭。
\zhu{加絮念征徭:意谓制衣者惦念服徭役的征人寒冷,在衣中多加棉絮。}
坳垤审夷险,\zhu{坳:音“傲”,地低洼处;山间平地。
垤:音“迭”,小土堆。
审:详察。
夷:平坦;平安。
坳垤审夷险:意谓大雪铺平了洼坑和高坎儿,走路时需要细察路面的高低不平。
}\par
\hop
宝钗连声赞好,也便联道:\par
\hop
枝柯怕动摇。
\zhu{柯:音“科”,树枝。}
皑皑轻趁步,\par
\hop
黛玉忙联道:\par
\hop
剪剪舞随腰。
\zhu{
剪剪:形容风轻而带寒意。
剪剪舞随腰:以轻盈舞姿喻白雪的随风飞旋。
}
煮芋成新赏,\zhu{
煮芋:苏东坡赞其幼子苏过以山芋作玉糁羹,诗中有“香似龙涎仍酽白(纯白)”句,这里是用玉糁羹的“酽白”,以喻雪色之白。
酽:音“雁”,酒、茶等味厚,这里引申为颜色鲜艳。
}\par
\hop
一面说,一面推宝玉,命他联。
宝玉正看宝钗、宝琴、黛玉三人共战湘云,十分有趣,那里还顾得联诗,今见黛玉推他,方联道:\par
\hop
撒盐是旧谣。
\zhu{
撒盐:晋人谢道韫,聪明有才辩,某天大雪,韫叔谢安问:“白雪纷纷何所似?”韫堂兄谢朗答道:“撒盐空中差可拟。
”道韫曰:“未若柳絮因风起。
”谢安赞赏不已。
见《世说新语·言语》。
}
苇蓑犹泊钓,\zhu{
“苇蓑”句:用唐代柳宗元“孤舟蓑笠翁,独钓寒江雪”诗意。
}\par
\hop
湘云笑道:“你快下去,你不中用,倒耽搁了我。
”一面只听宝琴联道:\par
\hop
林斧不闻樵。
\zhu{樵[qiáo]:柴,打柴。}
伏象千峰凸,
\zhu{本句意谓山峰积雪如伏卧的白象。}
\par
\hop
湘云忙联道:\par
\hop
盘蛇一径遥。
\zhu{盘蛇一径遥:意谓雪地小路似盘曲的长蛇。}
花缘经冷结,
\zhu{
花:指雪花。
缘:因为;由于。
}
\par
\hop
宝钗与众人又忙赞好。
探春又联道:\par
\hop
色岂畏霜凋。
\zhu{
色:指雪花。
花缘经冷结,色岂畏霜凋:意谓雪花由于天冷才结聚而成,洁白的颜色哪里会因怕霜冻而消褪。
}
深院惊寒雀,
\zhu{本句意为大雪雀饥,噪声如惊。或谓积雪滑落使雀惊。}
\par
\hop
湘云正渴了,忙忙的吃茶,已被岫烟道:\par
\hop
空山泣老鸮。
\zhu{
泣老鸮:意谓雪光照得夜色如同白昼,怕光的鸱鸮(音“吃消”)因不能捕食而哀泣。
鸮:鸱鸮,即猫头鹰.昼伏夜出,常于夜间捕捉食物,鸣声惨厉。
}
阶墀随上下,\zhu{
墀:音“迟”,台阶。
}\par
\hop
湘云忙丢了茶杯,忙联道:\par
\hop
池水任浮漂。
照耀临清晓,\par
\hop
黛玉联道:\par
\hop
缤纷入永宵。
诚忘三尺冷,
\zhu{
三尺:剑。
雪中戍守,刀剑随身,尤觉寒冷。
}
\par
\hop
湘云忙笑联道:\par
\hop
瑞释九重焦。
\zhu{
瑞:指瑞雪。
九重:代指皇帝。
诚忘三尺冷,瑞释九重焦:意谓诚敬之心,使将士忘却了戍守的寒苦;雪兆丰年,可以消除皇帝的焦虑。
}
僵卧谁相问,\par
\zhu{
“僵卧”句:用“袁安卧雪”故事。
《后汉书·袁安传》注引《汝南先贤传》:“时大雪积地丈馀。
洛阳令身出案行(案行:官员巡视),见人家皆除雪出,有乞食者。
至袁安门,无有行路。
谓安已死,令人除雪入户,见安僵卧,问何以不出。
安曰:‘大雪人皆饿,不宜干人’(干:冒犯,冲犯,这里是打扰的意思)。
”
}\par
\hop
宝琴也忙笑联道:\par
\hop
狂游客喜招。
\zhu{
“狂游”句:用唐代王元宝雪天招客宴饮的故事。
王仁裕《开元天宝遗事》:“巨豪王元宝,每大雪则自所居至坊口扫雪开道,迎揖宾客饮宴,谓之暖寒会。”
}
天机断缟带,
\zhu{
天机:传说中天上织女的织机。
缟带:白色的丝带。
}
\par
\hop
湘云又忙道:\par
\hop
海市失鲛绡。
\zhu{
海市:即海市蜃楼,海上由于光线变化而出现的奇异幻景。
鲛绡(绡音“消”):传说南海中有鲛人,即人鱼,能织绡,后用以泛称薄纱。
天机断缟带,海市失鲛绡:用天上落下的缟带、海市移来的鲛绡喻雪的洁白美好。
}
\par
\hop
林黛玉不容他出,接着便道:\par
\hop
寂寞对台榭,\par
\hop
湘云忙联道:\par
\hop
清贫怀箪瓢。
\zhu{
箪(音“丹”)瓢:语本《论语·雍也》:“一箪食,一瓢饮。
”这是孔子称赞他的学生颜回不以贫困为忧的话。
箪:盛饭用的圆形竹器。
清贫怀箪瓢:意谓穷苦之士由于大雪封门饮食无着,连“箪食瓢饮”的清贫生活也怀念起来了。
}\par
\hop
宝琴也不容情,也忙道:\par
\hop
烹茶冰渐沸,\par
\hop
湘云见这般,自为得趣,又是笑,又忙联道:\par
\hop
煮酒叶难烧。
\par
\hop
黛玉也笑道:\par
\hop
没帚山僧扫,\zhu{没:音“莫”,淹没,掩埋。这句意为积雪很深,山僧扫雪时,雪可没帚。
}\par
\hop
宝琴也笑道:\par
\hop
埋琴稚子挑。
\zhu{这句意为雪深之极,琴童挑琴而行,琴亦几为雪所埋。一说琴被雪掩埋了,所以小书童赶紧把它挑起来。}
\par
\hop
湘云笑的弯了腰,忙念了一句,众人问:“到底说的什么?”湘云喊道:\par
\hop
石楼闲睡鹤,\par
\hop
黛玉笑的握着胸口,高声嚷道:\par
\hop
锦罽暖亲猫。
\zhu{锦罽(罽音“计”):织有文彩的毛毯。
}\par
\hop
宝琴也忙笑道:\par
\hop
月窟翻银浪,
\zhu{
银浪:月光。
这里代指雪波。
}
\par
\hop
湘云忙联道:\par
\hop
霞城隐赤标。
\zhu{
霞城:指赤城山,在浙江天台县北。
赤标:这里指赤城山的高峰。
月窟翻银浪,霞城隐赤标:上句以月光普照暗喻白雪遍地;下句用隐没赤标形容积雪深厚。
}
\par
\hop
黛玉忙笑道:\par
\hop
沁梅香可嚼,
\zhu{这句意为梅花浸雪而更显浓郁清香,可以咀嚼。}
\par
\hop
宝钗笑称好,也忙联道:\par
\hop
淋竹醉堪调。
\zhu{
淋竹:雪洒在竹子上。
醉:以酒浸物之意,形容雪水浸润过的竹子像酒渍过一样。
也可能是形容竹子纤弱东倒西歪似醉。
调:调弄,弹奏,湿润的竹子在雪压下发出的声音正好作为弹琴的参照。
宋代王禹偁《黄冈竹楼记》:“冬宜密雪,有碎玉声;宜鼓琴,琴调和畅。”
一说“醉”指人醉酒。
}
\par
\hop
宝琴也忙道:\par
\hop
或湿鸳鸯带,\zhu{鸳鸯带:有鸳鸯图案花纹的带子。
}\par
\hop
湘云忙联道:\par
\hop
时凝翡翠翘。
\zhu{翡翠翘:也称翠翘。
翘,首饰。
用翡翠鸟的羽毛粘在首饰上叫“点翠”,点翠的凤冠叫“翠翘”,但也可泛指点翠的其他首饰。
一说,翠翘本指翠鸟尾上的长羽,也称形似翠鸟尾上长羽的妇女首饰。
}\par
\hop
黛玉又忙道:\par
\hop
无风仍脉脉,\par
\hop
宝琴又忙笑联道:\par
\hop
不雨亦潇潇。
\par
\hop
湘云伏着已笑软了。
众人看他三人对抢,也都不顾作诗,看着也只是笑。
黛玉还推他往下联,又道:“你也有才尽之时。
我听听还有什么舌根嚼了!”湘云只伏在宝钗怀里,笑个不住。
宝钗推他起来道:“你有本事,把‘二萧’的韵全用完了,我才伏你。
”湘云起身笑道:“我也不是作诗,竟是抢命呢。
”众人笑道:“倒是你说罢。
”探春早已料定没有自己联的了,便早写出来,因说:“还没收住呢。
”李纹听了,接过来便联了一句道:\par
\hop
欲志今朝乐,\par
\hop
李绮收了一句道:\par
\hop
凭诗祝舜尧。
\par
\hop
李纨道:“够了,够了。
虽没作完了韵,剩的字若生扭用了,倒不好了。
”说着,大家来细细评论一回,独湘云的多,都笑道:“这都是那块鹿肉的功劳。
”\par
李纨笑道:“逐句评去都还一气,只是宝玉又落了第了。
”宝玉笑道:“我原不会联句,只好担待我罢。
”李纨笑道:“也没有社社担待你的。
又说韵险了,又整误了,又不会联句了,今日必罚你。
我才看见栊翠庵的红梅有趣,我要折一枝来插瓶。
可厌妙玉为人,我不理他。
\ping{讨厌妙玉啥呢?见人下菜碟?自诩孤高但是凡心偶炽?李纨是个被社会要求清心寡欲的人,可能看到装作清心寡欲的人不舒坦。
}如今罚你去取一枝来。
”众人都道这罚的又雅又有趣。
宝玉也乐为,答应着就要走。
湘云黛玉一齐说道:“外头冷得很,你且吃杯热酒再去。
”湘云早执起壶来,黛玉递了一个大杯,满斟了一杯。
湘云笑道:“你吃了我们的酒,你要取不来,加倍罚你。
”宝玉忙吃一杯,冒雪而去。
李纨命人好好跟着。
黛玉忙拦说:“不必,有了人反不得了。
”李纨点头说:“是。
”一面命丫鬟将一个美女耸肩瓶拿来,
\zhu{
耸肩瓶:此瓶与“美人觚”略似,相异处在于美人觚立面弧形曲线以不同弧度一直伸延至瓶口,线条柔缓悠长;
而美人耸肩瓶之立面弧形曲线则不直伸至瓶口便向内突折,看去颇美人耸肩之状,瓶口较小,高出“耸肩”。
}
贮了水准备插梅,因又笑道:“回来该咏红梅了。
”湘云忙道:“我先作一首。
”宝钗忙道:“今日断乎不容你再作了。
你都抢了去,别人都闲着,也没趣。
回来还罚宝玉,他说不会联句,如今就叫他自己作去。
”\geng{想此刻宝玉已到庵中矣。
}黛玉笑道:“这话很是。
我还有个主意,方才联句不够,莫若拣着联的少的人作红梅。
”宝钗笑道:“这话是极。
方才邢李三位屈才,\zhu{邢李三位:邢岫烟,李纹,李绮。
}
且又是客。
琴儿和颦儿云儿三个人也抢了许多,我们一概都别作,只让他三个作才是。
”李纨因说:“绮儿也不大会作,还是让琴妹妹作罢。
”宝钗只得依允,\geng{想此刻二玉已会,不知肯见赐否。
}又道:“就用‘红梅花’三个字作韵,每人一首七律。
邢大妹妹作‘红’字,你们李大妹妹作‘梅’字,琴儿作‘花’字。
”李纨道:“饶过宝玉去,我不服。
”湘云忙道:“有个好题目命他作。
”众人问何题目?湘云道:“命他就作‘访妙玉乞红梅’,岂不有趣?”众人听了,都说有趣。
\par
一语未了,只见宝玉笑嘻嘻掮了一枝红梅进来。
\zhu{掮:音“钱”,此指擎着、小心地把扶着。
}众丫鬟忙已接过,插入瓶内。
众人都笑称谢。
宝玉笑道:“你们如今赏罢,也不知费了我多少精神呢。
”说着,探春早又递过一钟暖酒来,众丫鬟走上来接了蓑笠掸雪。
各人房中丫鬟都添送衣服来,\geng{冬日午后景况。
}袭人也遣人送了半旧的狐腋褂来。
\zhu{狐腋:狐狸腋窝部位的皮。}
李纨命人将那蒸的大芋头盛了一盘,又将朱橘、黄橙、橄榄等物盛了两盘,命人带与袭人去。
湘云且告诉宝玉方才的诗题,又催宝玉快作。
宝玉道:“姐姐妹妹们,让我自己用韵罢,别限韵了。
”众人都说:“随你作去罢。
”\par
一面说一面大家看梅花。
原来这枝梅花只有二尺来高,旁有一横枝纵横而出,约有五六尺长,其间小枝分歧,或如蟠螭,\zhu{蟠:音“盘”,盘曲地伏着。
螭:音“吃”,古代传说中的无角龙。
}或如僵蚓,或孤削如笔,或密聚如林,花吐胭脂,香欺兰蕙,
\zhu{欺:压倒,胜过。}
\geng{一篇《红梅赋》。
}各各称赏。
谁知邢岫烟、李纹、薛宝琴三人都已吟成,各自写了出来。
众人便依“红梅花”三字之序看去,写道是:\par
\hop
咏红梅花\quad {\footnotesize 得“红”字} \quad 邢岫烟\zhu{得“红”字:多人一起作诗,先提出若干字为韵字,由大家自由选择或拈阉决定,叫做“分韵”。
分到某一韵字的人,在他的诗题下注明“得某字”,并用这个字所在韵部的字作韵脚。
}\par
桃未芳菲杏未红,冲寒先已笑东风。
\zhu{“冲寒”句:意谓红梅先于桃杏冲破寒冷笑向东风。
}\par
魂飞庾岭春难辨,
\zhu{
庾岭:即大庾岭,五岭之一,在江西、广东两省边境。
因岭上多梅花,又称梅岭。
这句意谓红梅花若开在庾岭,其景色就与春天很难区别了。
借“庾岭”点出梅花,借“春”点出色红。
}
霞隔罗浮梦未通。
\zhu{
罗浮:山名,在广东省。
这句意为红梅如霞隔断了去罗浮山之路,也就更谈不到做遇见梅花仙女之梦了。
梦:隋代赵师雄游罗浮山梦见梅花化为“淡妆素服”的美人与之欢宴歌舞的故事(见《龙城录》)。
一说“隔”、“未通”是因赵师雄所梦见的罗浮山梅花是淡色的,与所咏的红梅不同。
}\par
绿萼添妆融宝炬,缟仙扶醉跨残虹。
\zhu{绿萼:即绿萼梅,是梅中之花瓣萼蒂皆绿者,有人比之为仙女萼绿华。
缟仙:白衣仙子,这里代指白梅。
扶醉:醉须人扶。
残虹:尚未完全消失的彩虹。
这两句用萼绿仙女(绿梅)为添加红妆而熔化红烛,白衣仙子(白梅)带着醉颜跨上残虹来形容红梅。
“融宝炬”、“扶醉”、“跨残虹”均喻红色。
}\par
看来岂是寻常色,浓淡由他冰雪中。
\par
\hop
咏红梅花\quad {\footnotesize 得“梅”字} \quad 李纹\par
白梅懒赋赋红梅,逞艳先迎醉眼开。
\zhu{醉眼开:喻微开的红梅。
一说“迎醉眼开”意为先迎着我醉眼(沉醉于梅花或酒醉)开放。
}\par
冻脸有痕皆是血,酸心无恨亦成灰。
\zhu{冻脸:喻开放于冰雪严寒中的红梅。
酸心:梅花结子味酸,故云“酸心”。
}\par
误吞丹药移真骨,偷下瑶池脱旧胎。
\zhu{
上句意谓红梅是白梅误吞了仙丹换掉真骨化成。
下句意谓红梅是瑶池仙女偷下凡间,脱化旧胎而成。
瑶池:即阿母池,古代传说中西王母的住处。
}\par
江北江南春灿烂,
\zhu{春灿烂:因寒冬中的红梅色似春花,所以如此说,非实指春天。}
寄言蜂蝶漫疑猜。
\zhu{
漫:不要,莫。
意谓告诉蜂蝶,切莫误把红梅认作是桃杏,而疑猜是否已是桃李芳菲的暖春,眼下还正在冰雪严寒的季节。
}\par
\hop
咏红梅花\quad {\footnotesize 得“花”字} \quad 薛宝琴\par
疏是枝条艳是花,春妆儿女竞奢华。
\zhu{春妆:红妆。
下句暗喻红梅怒放,犹如少女秾妆,争艳斗妍。
}\par
闲庭曲槛无馀雪,流水空山有落霞。
\zhu{馀雪:代指白梅。
落霞:代指红梅。
这二句意谓无论庭院或山野,尽是红梅无白梅,运用了互文的技巧。
}\par
幽梦冷随红袖笛,
\zhu{红袖笛:这里代指红梅花。}
游仙香泛绛河槎。
\zhu{
泛,浮行,乘船。
绛河:即天河、银河。
这里用“绛”,意在点出“红”字,以喻红梅。
槎,音“查”,木筏,这里指神话传说中往来天上的“星槎”。
乘槎游仙的传说,见于《博物志》:银河与海相通,居海岛者,年年八月定期可见有木筏从水上来去。
有人便带了粮食,登上木筏而去,结果碰到了牛郎织女。
这两句意谓随着红袖少女的清冷笛声进入梦境;乘坐绛河的香筏游于仙界。
}\par
前身定是瑶台种,无复相疑色相差。
\zhu{瑶台:神仙所居之处。
色相:本佛家语,这里指红梅花的颜色和形状。
}\par
\hop
众人看了,都笑称赞了一番,又指末一首说更好。
宝玉见宝琴年纪最小,才又敏捷,深为奇异。
黛玉湘云二人斟了一小杯酒,齐贺宝琴。
宝钗笑道:“三首各有各好。
你们两个天天捉弄厌了我,如今捉弄他来了。
”李纨又问宝玉:“你可有了?”宝玉忙道:“我倒有了,才一看见那三首,又吓忘了,等我再想。
”湘云听了,便拿了一支铜火箸击着手炉,笑道:“我击鼓了,\zhu{击鼓:即“击鼓催诗”,是旧日多人一起作诗的活动形式之一,以鼓声的起止为限定的构思时间,到时作不出来或超过时间便要受罚。
击鼓亦时用击炉、击钵等代替。
钵[bō]:一种敞口器皿,像盆而较小较深,多为陶制。
}若鼓绝不成,又要罚的。
”宝玉笑道:“我已有了。
”黛玉提起笔来,说道:“你念,我写。
”湘云便击了一下笑道:“一鼓绝。
”宝玉笑道:“有了,你写吧。
”众人听他念道:\par
\hop
酒未开樽句未裁,\par
\hop
黛玉写了,摇头笑道:“起的平平。
”湘云又道“快着!”宝玉笑道:\par
\hop
寻春问腊到蓬莱。
\zhu{寻春问腊:以“春”点“红”,以“腊”点“梅”,寻春问腊的意思是乞红梅。
蓬莱:仙境,这里指妙玉所居栊翠庵。
}\par
\hop
黛玉湘云都点头笑道:“有些意思了。
”宝玉又道:\par
\hop
不求大士瓶中露,为乞嫦娥槛外梅。
\zhu{这里大士、嫦娥皆隐指妙玉。
大士:本为菩萨之称,这里指观世音。
传说观世音形为女身,手持净瓶,中有甘露,上插杨枝,以杨枝洒甘露,能解人间困厄。
槛[jiàn]:栏杆。
槛外:世外,这里指栊翠庵。
}\par
\hop
黛玉写了,又摇头道:“凑巧而已。
”湘云忙催二鼓,宝玉又笑道:\par
\hop
入世冷挑红雪去,离尘香割紫云来。
\zhu{这里是将栊翠庵当作仙佛境地,把从庵中采梅回来叫“入世”,去庵中求梅叫“离尘”。
梅称冷香,分嵌于两句中。
挑红雪、割紫云:均喻折红梅。
}\par
槎枒谁惜诗肩瘦,衣上犹沾佛院苔。
\zhu{槎枒:同“查牙”、“杈丫”,这里形容诗人骨瘦如柴。
诗肩瘦:指诗人瘦削高耸的肩头。
佛院:代指栊翠庵。
}\ping{宝玉“衣上犹沾佛院苔”,可能暗示了宝玉最终出家的结局。
}\par
\hop
黛玉写毕,湘云大家才评论时,又见几个丫鬟跑进来道:“老太太来了。
”众人忙迎出来。
大家又笑道:“怎么这等高兴!”说着,远远见贾母围了大斗篷,带着灰鼠暖兜,\zhu{暧兜:一种能够防风保暖的帽子。
}坐着小竹轿,打着青绸油伞,鸳鸯琥珀等五六个丫鬟,每人都是打着伞,拥轿而来。
李纨等忙往上迎,贾母命人止住说:“只在那里就是了。
”来至跟前,贾母笑道:“我瞒着你太太和凤丫头来了。
大雪地下坐着这个无妨,没的叫他们来踩雪。
”众人忙一面上前接斗篷,搀扶着,一面答应着。
贾母来至室中,先笑道:“好俊梅花!你们也会乐,我来着了。
”说着,李纨早命拿了一个大狼皮褥来铺在当中。
贾母坐了,因笑道:“你们只管顽笑吃喝。
我因为天短了,不敢睡中觉,抹了一回牌,想起你们来了,我也来凑个趣儿。
”李纨早又捧过手炉来,探春另拿了一副杯箸来,亲自斟了暖酒,奉与贾母。
贾母便饮了一口,问那个盘子里是什么东西。
众人忙捧了过来,回说是糟鹌鹑。
贾母道:“这倒罢了,撕一两点腿子来。
”李纨忙答应了,要水洗手,亲自来撕。
贾母又道:“你们仍旧坐下说笑我听。
”又命李纨:“你也坐下,就如同我没来的一样才好,不然我就去了。
”众人听了,方依次坐下,这李纨便挪到尽下边。
贾母因问作何事了,众人便说作诗。
贾母道:“有作诗的,不如作些灯谜,大家正月里好顽的。
”众人答应了。
说笑了一回,贾母便说:“这里潮湿,你们别久坐,仔细受了潮湿。
”因说:“你四妹妹那里暖和,我们到那里瞧瞧他的画儿,赶年可有了。
”众人笑道:“那里能年下就有了?只怕明年端阳有了。
”贾母道:“这还了得!他竟比盖这园子还费工夫了。
”\par
说着,仍坐了竹轿,大家围随,过了藕香榭,穿入一条夹道,东西两边皆有过街门,
\zhu{
过街门:内院直通街市的门道,这里指与外院相通的夹道门。
一说,通道两侧相对着开的门。
}
门楼上里外皆嵌着石头匾,如今进的是西门,向外的匾上凿着“穿云”二字,向里的凿着“度月”两字。
来至当中,进了向南的正门,贾母下了轿,惜春已接了出来。
从里边游廊过去,便是惜春卧房,门斗上有“暖香坞”三个字。
\geng{看他又写出一处,从起至末一笔一部之文也有,千万笔成一部之文也有,一二笔成一部之文也有。
如“试才”一回起若都说完,以后则索然无味,故留此几处以为后文之点染也。
此方活泼不板,眼目屡新。
}
早有几个人打起猩红毡帘,已觉温香拂脸。
\geng{各处皆如此,非独因“暖香”二字方有此景。
戏注于此,以博一笑耳。
}大家进入房中,贾母并不归坐,只问画在那里。
惜春因笑回:“天气寒冷了,胶性皆凝涩不润,画了恐不好看,故此收起来。
”贾母笑道:“我年下就要的。
你别托懒儿,快拿出来给我快画。
”一语未了,忽见凤姐儿披着紫羯绒褂,\zhu{羯:音“节”,被阉割的公羊,泛指羊。
}笑嘻嘻的来了,口内说道:“老祖宗今儿也不告诉人,私自就来了,要我好找。
”贾母见他来了,心中自是喜悦,便道:“我怕你们冷着了,所以不许人告诉你们去。
你真是个鬼灵精儿,到底找了我来。
以理,孝敬也不在这上头。
”凤姐儿笑道:“我那里是孝敬的心找了来?我因为到了老祖宗那里,鸦没雀静的,\geng{这四个字俗语中常闻,但不能落纸笔耳。
便欲写时,究竟不知系何四字,今如此写来,真是不可移易。
}问小丫头子们,他又不肯说,叫我找到园里来。
我正疑惑,忽然来了两三个姑子,我心里才明白。
我想姑子必是来送年疏,\zhu{疏:这里指焚化在神佛前的祭文、祝辞等,又称“疏头”。
旧时为了祈福消灾,请僧尼诵经,井指定诵经遍数,每诵一遍,便在预先准备好的疏头上印一个朱砂小红圈,年终送至施主家中,以备祭神礼佛时焚化,并照例得到施主的报酬。
这种每年送一次的疏头,叫做“年疏”。
}或要年例香例银子,老祖宗年下的事也多,一定是躲债来了。
我赶忙问了那姑子,果然不错。
我连忙把年例给了他们去了。
如今来回老祖宗,债主已去,不用躲着了。
已预备下希嫩的野鸡,请用晚饭去,再迟一回就老了。
”他一行说,众人一行笑。
\par
凤姐儿也不等贾母说话,便命人抬过轿子来。
贾母笑着,搀了凤姐的手,仍旧上轿,带着众人,说笑出了夹道东门。
一看四面粉妆银砌,
\zhu{粉妆银砌:用白粉装饰,用银砌成。形容庭园雪景。}
忽见宝琴披着凫靥裘站在山坡上遥等,
\zhu{凫靥裘:用野鸭子脸颊皮毛剪贴重叠作成的御寒外衣。}
身后一个丫鬟抱着一瓶红梅。
众人都笑道:“少了两个人,他却在这里等着,也弄梅花去了。
”贾母喜的忙笑道:“你们瞧,这山坡上配上他的这个人品,又是这件衣裳,后头又是这梅花,像个什么?”众人都笑道:“就像老太太屋里挂的仇十洲画的《双艳图》。
”\zhu{仇(音“求”)十洲:明代画家仇英的别号,擅画工笔仕女及山水,为明代四大画家之一。
}贾母摇头笑道:“那画的那里有这件衣裳?人也不能这样好!”一语未了,只见宝琴背后转出一个披大红猩毡的人来。
贾母道:“那又是那个女孩儿?”众人笑道:“我们都在这里,那是宝玉。
”贾母笑道:“我的眼越发花了。
”说话之间,来至跟前,可不是宝玉和宝琴。
\ping{宝玉黛玉、宝玉宝钗,这又来了宝玉宝琴。
}宝玉笑向宝钗黛玉等道:“我才又到了栊翠庵。
妙玉每人送你们一枝梅花,我已经打发人送去了。
”众人都笑说:“多谢你费心。
”\par
\chai{baoqin}{宝琴踏雪}
说话之间,已出了园门,来至贾母房中。
吃毕饭大家又说笑了一回。
忽见薛姨妈也来了,说:“好大雪,一日也没过来望候老太太。
今日老太太倒不高兴?正该赏雪才是。
”贾母笑道:“何曾不高兴!我找了他们姊妹们去顽了一会子。
”薛姨妈笑道:“昨日晚上,我原想着今日要和我们姨太太借一日园子,摆两桌粗酒,请老太太赏雪的,又见老太太安息的早。
我闻得女儿说,老太太心下不大爽,因此今日也没敢惊动。
早知如此,我正该请。
”贾母笑道:“这才是十月里头场雪,往后下雪的日子多呢,再破费不迟。
”薛姨妈笑道:“果然如此,算我的孝心虔了。
”凤姐儿笑道:“姨妈仔细忘了,如今先秤五十两银子来,交给我收着,一下雪,我就预备下酒,姨妈也不用操心,也不得忘了。
”贾母笑道:“既这么说,姨太太给他五十两银子收着,我和他每人分二十五两,到下雪的日子,我装心里不快,混过去了,姨太太更不用操心,我和凤丫头倒得了实惠。
”凤姐将手一拍,笑道:“妙极了,这和我的主意一样。
”众人都笑了。
贾母笑道:“呸!没脸的,就顺着竿子爬上来了!你不说姨太太是客,在咱们家受屈,我们该请姨太太才是,那里有破费姨太太的理!不这样说呢,还有脸先要五十两银子,真不害臊!”凤姐儿笑道:“我们老祖宗最是有眼色的,试一试,姨妈若松呢,拿出五十两来,就和我分。
这会子估量着不中用了,翻过来拿我做法子,\zhu{做法:即“作法”,方言,又作“扎筏子”、“扎罚子”等。
意为找岔头、找借口,拿某人“作法”,意即拿某人当作出气或立威的对象,以儆其馀。
}说出这些大方话来。
如今我也不和姨妈要银子,竟替姨妈出银子治了酒,请老祖宗吃了,我另外再封五十两银子孝敬老祖宗,算是罚我个包揽闲事。
这可好不好?”话未说完,众人已笑倒在炕上。
\par
贾母因又说及宝琴雪下折梅比画儿上还好,因又细问他的年庚八字并家内景况。
薛姨妈度其意思,大约是要与宝玉求配。
\ping{贾母喜欢宝琴,逼着王夫人认为干女儿,和宝玉是“兄妹”,不可能“求配”。}
薛姨妈心中固也遂意,只是已许过梅家了,因贾母尚未明说,自己也不好拟定,遂半吐半露告诉贾母道:“可惜这孩子没福,前年他父亲就没了。
他从小儿见的世面倒多,跟他父母四山五岳都走遍了。
他父亲是好乐的,各处因有买卖,带着家眷,这一省逛一年,明年又往那一省逛半年,所以天下十停走了有五六停了。
那年在这里,把他许了梅翰林的儿子,偏第二年他父亲就辞世了,他母亲又是痰症。
”
\zhu{痰症:中医病症名。痰症是指脏腑气血失和,水湿,津液凝结成痰所产生的各种病证。}
凤姐也不等说完,便嗐声跺脚的说:
\zhu{嗐:音“害”,表示不满,惋惜或懊悔。}
“偏不巧,我正要作个媒呢,又已经许了人家。
”贾母笑道:“你要给谁说媒?”\ping{宝琴已经认了宝玉的母亲王夫人为干妈,这样宝琴和宝玉就是兄妹了,当然不可能结合,此处薛姨妈误会了,贾母和王熙凤的说媒对象,肯定不是宝玉了。
另外,即使没有这一层干兄妹的关系,在那个时代,长辈也不会在两个说媒对象都在现场的情况下说媒。
倘若真的如此,宝玉宝琴肯定会感觉尴尬,联系到张道士给宝玉提亲之后黛玉的激烈反应,黛玉也会生气。
然而这一切都没有发生,由此可以推断,当场的所有人都知道宝琴的说媒对象肯定不是宝玉,至于到底是谁,可能是和贾家来往甚密的江南甄家里的贾宝玉的“影子”甄宝玉。
}凤姐儿说道:“老祖宗别管,我心里看准了他们两个是一对。
如今已许了人,说也无益,不如不说罢了。
”贾母也知凤姐儿之意,听见已有了人家,也就不提了。
大家又闲话了一会方散。
一宿无话。
\par
次日雪晴。
饭后,贾母又亲嘱惜春:“不管冷暖,你只画去,赶到年下,十分不能便罢了。
第一要紧把昨日琴儿和丫头梅花,照模照样,一笔别错,快快添上。
”\ping{第四十九回:薛蟠之从弟薛蝌,因当年父亲在京时已将胞妹薛宝琴许配都中梅翰林之子为婚,正欲进京发嫁。
“丫头梅花”在这部书中是可有可无的人物,这里着重强调,应该是点出宝琴和梅翰林之子的婚约。
至于贾母为何要如此做,可能是薛家衰败之后,梅翰林家想要退亲,这样做是要挽救和巩固宝琴和梅翰林之子的婚约。
}惜春听了虽是为难,只得应了。
一时众人都来看他如何画,惜春只是出神。
李纨因笑向众人道:“让他自己想去,咱们且说话儿。
昨儿老太太只叫作灯谜,回家和绮儿纹儿睡不着,我就编了两个‘四书’的。
他两个每人也编了两个。
”众人听了,都笑道:“这倒该作的。
先说了,我们猜猜。
”李纨笑道:“‘观音未有世家传’,打《四书》一句。
”湘云接着就说“在止于至善。
”\zhu{在止于至善:意谓德行达到最完美的境界。
语出《礼记·大学》。
至:极;顶峰。
}宝钗笑道:“你也想一想‘世家传’三个字的意思再猜。
”李纨笑道:“再想。
”黛玉笑道:“哦,是了。
是‘虽善无征’。
”\zhu{虽善无征:语出《礼记·中庸》:“上焉者虽善无征。
”意谓先王的礼制虽好,但无从证实。
征:征验;证实。
此语为“观音未有世家传”的谜底,“世家”当取《史记》之例(《史记》全书包括记历代帝王政绩的十二本纪、记诸侯勋贵兴亡的三十世家、记重要人物的言行事迹的七十列传等),观音至善,未有世家为传,无从征考。
一说,“征”作“纳征”解,是婚礼中男方向女方致送聘礼。
“虽善无征”别解之后,意思便是说,(观音)“虽”然是个“善”者,可是“无“人向她纳“征”,聘她为妻,她当然也便没有后代(世家传)了。
黛玉猜中谜底,可能暗示黛玉自己婚姻的悲剧。
}众人都笑道:“这句是了。
”李纨又道:“一池青草草何名。
”湘云忙道:“这一定是‘蒲芦也’。
\zhu{蒲芦也:语出《礼记·中庸》:“人道敏政,地道敏树,夫政也者,蒲芦也。”
朱熹集注:“敏,速也。蒲芦,沈括以为蒲苇是也。以人立政,犹以地种树,其成速矣。而蒲苇又易生之物,其成尤速也。言人存政举,其易如此。”
}再不是不成?”李纨笑道:“这难为你猜。
纹儿的是‘水向石边流出冷’,打一古人名。
”探春笑问道:“可是山涛?”\zhu{山涛:晋代诗人,字巨源,竹林七贤之一。
}李纹笑道:“是。
”李纨又道:“绮儿的是个‘萤’字,打一个字。
”众人猜了半日,宝琴笑道:“这个意思却深,不知可是花草的‘花’字?”李绮笑道:“恰是了。
”众人道:“萤与花何干?”黛玉笑道:“妙得很!萤可不是草化的?”\zhu{萤可不是草化的:“花”字可以拆成“草”字头和“化”两个字。
萤在夏季多就水草产卵,化蛹成长,古人误认为萤是由腐草本身变化而成。
《礼记·月令》:“季夏之月……腐草为萤。
”}
众人会意,都笑了说;“好!”宝钗道:“这些虽好,不合老太太的意思,不如作些浅近的物儿,大家雅俗共赏才好。
”众人都道:“也要作些浅近的俗物才是。
”湘云笑道:“我编了一支《点绛唇》,恰是俗物,你们猜猜。
”说着便念道:\par
\hop
溪壑分离,红尘游戏,真何趣?名利犹虚,后事终难继。
\par
\hop
众人不解,想了半日,也有猜是和尚的,也有猜是道士的,也有猜是偶戏人的。
\zhu{偶戏人:木偶人。
}宝玉笑了半日,道:“都不是,我猜着了,一定是耍的猴儿。
”湘云笑道:“正是这个了。
”众人道:“前头都好,末后一句怎么解?”湘云道:“那一个耍的猴子不是剁了尾巴去的?”
\zhu{
溪壑分离,红尘游戏:猴子多生活在山谷中、涧溪旁,被人捕住后,便离了山林,来到闹市,供人耍玩。
名利犹虚:指猴子穿衣戴帽,扮成文官武将的样子。
清代富察敦崇《燕京岁时记》:“耍猴儿者,木箱之内,藏有羽帽乌纱,猴手自启箱,戴而坐之,俨如官之排衙。
猴人口唱俚歌,抑扬可听,古称‘沐猴而冠’,殆指此也。”
暗中嘲讽世上热衷于功名利禄之辈,从他们套上名利的绳索的那一天起,也就像“耍的猴儿”一样,上蹿下跳地在扮演着滑稽的角色,他们洋洋得意于一时的高官厚禄,戏演完后免不了落得个“后事终难继”的下场。
后事终难继:湘云解说:“那一个耍的猴子不是剁了尾巴去的?”
宝玉猜中这个谜并非偶然。因为它句句适用于宝玉:神瑛侍者带看大荒山青埂峰的顽石,幻形入世,成了佩戴通灵玉的怡红公子,这不正是“溪壑分离,红尘游戏”吗?
“真何趣”的感慨与宝玉在《寄生草·解偈》一曲中所说的“到如今回头试想真无趣”的意思一样;
“名利犹虚”是宝玉蔑视仕途经济的叛逆思想;
“后事终难继”正应了宝玉“悬崖撒手”,弃家为僧的结局。这样,谜语就简括着宝玉一生的道路。
}
众人听了,都笑起来,说:“偏他编个谜儿也是刁钻古怪的。
”李纨道:“昨日姨妈说,琴妹妹见的世面多,走的道路也多,你正该编谜儿,正用着了。
你的诗且又好,何不编几个我们猜一猜?”宝琴听了,点头含笑,自去寻思。
宝钗也有了一个,念道:\par
\hop
镂檀锲梓一层层,
\zhu{
镂(音“漏”)、锲(音“妾”):都是雕刻的意思。
檀、梓(音“子”):都是质地比较坚硬的木材。
}
岂系良工堆砌成?\par
虽是半天风雨过,何曾闻得梵铃声!\zhu{梵铃:佛寺和宝塔檐角上悬挂的铜铃。
}\par
打一物。
\par
\ping{
前两句说谜底之物像一座玲珑的宝塔,层层叠叠,但它并不是工匠用砖石垒砌起来的,而是天然生成的。
后人猜谜底,有以为是树上松球的,因松球状如梵铃而无声。
宝钗的灯谜可能是对其性格、遭际的概括和感叹。前二句比喻她为人处世总能精细周全、八面玲珑以及美丽端庄的外表,并说这些都出于天性,并非是她有意做出的。
后二句暗示宝钗与宝玉的婚姻悲剧,说她与宝玉只是空做了一场名义上的夫妻。一说这里是借用唐玄宗与杨贵妃死别后于风雨中闻铃悲感事(白居易《长恨歌》:夜雨闻铃肠断声),寓宝钗与宝玉的生离。
}\par
\hop
众人猜时,宝玉也有了一个,念道:\par
\hop
天上人间两渺茫,
\zhu{本句意为天上地下相距遥远。}
琅玕节过谨隄防。
\zhu{
琅玕:音“郎甘”,本是青色的玉石,借以喻竹。
节:植物分枝长叶的地方,这里是竹节。
节过:竹子生长“拔节”之后。
隄防:即提防。
谨隄防:端正态度,集中精力注意倾听。
这句话与后两句相照应。
}\par
鸾音鹤信须凝睇,
\zhu{
鸾音鹤信:指仙界传来的消息。这可能是指人的死去,亦即所谓仙逝。
鸾和鹤在传说中都被看作“仙禽”,乘鸾鹤表示登仙。
凝睇:注视。
}
好把唏嘘答上苍。
\zhu{
唏嘘:叹息的声音。
上苍:青天。
}\par
\ping{
此首后人猜谜底,有以为是风筝的:风筝地下天上,望之渺茫,放线过竹竿时须防被挂住。后两句则形容其发声。
明人陈沂《询刍录·风筝》:“于鸢首以竹为笛,使风入作声如筝,俗呼风筝。”
宝玉的谜寓痛悼黛玉夭亡之意比较明显。首句用的就是南唐李煜《浪淘沙》词“别时容易见时难。流水落花春去也,天上人间”和白居易《长恨歌》“含情凝睇谢君王,一别音容两渺茫”中的词语和意思,暗示黛玉死后和宝玉天人两隔。
黛玉号“潇湘妃子”,所以借“琅玕”(青色的玉石,借以喻竹)来点她。
第二十六回写潇湘馆“凤尾(竹叶)森森,龙吟(风吹竹声)细细”;但到后半部,却景物全非,只见“落叶萧萧,寒烟漠漠”(脂评引佚稿中文字),一片荒凉。这也许就是“琅玕节过”的含义。黛玉所居之地如此,其中之人也必凋零,可能黛玉就是在“琅玕节过”即竹子生长“拔节”之后这个时间去世的。
借“鸾音鹤信”暗示黛玉仙逝,宝玉痛悼黛玉,据脂评说佚稿中亦有如《芙蓉女儿诔》那样大段文字,深以不能读到他“唏嘘答上苍”之词为憾。
}\par
\hop
黛玉也有了一个,念道是:\par
\hop
騄駬何劳缚紫绳?驰城逐堑势狰狞。
\zhu{騄駬:音“录耳”,马名,传说为周穆王八骏之一。
见《穆天子传》。
紫绳:缰绳。
驰城逐堑:奔驰过城池,跨越过沟渠。
狰狞:凶猛,骠勇。
}\par
主人指示风雷动,鳌背三山独立名。
\zhu{鳌背三山:古代传说,见于《列子》,渤海之东,有蓬莱、方丈、瀛洲三座神山,本随波往来,天帝害怕它们漂浮到西极去,就叫十五只巨鳌(大海龟)来背着它们。
}\par
\ping{
此首后人猜谜底,有以为是走马灯的。末句说灯节之鳌山。古时正月十五夜观灯,京都中所搭起的灯山,作鳌背神山形,上立千百种彩灯,亦称“鳌山”。
前两句说走马灯因灯烛燃烧形成气流自然转动,故曰不需要用绳索牵动。
后面“主”取“烛”的谐音,点明纸马要靠灯烛才能转动。
黛玉的谜中说千里马奔腾驰突,有不可羁勒之势。当喻黛玉才情横溢,口角锋芒,锐利无比,又不满封建礼教束缚。
“风雷动”或喻重大事变发生。声名独占鳌头,是对她的赞语也是谶语。因为海上“鳌背三山”终究是无法寻求的,即《长恨歌》中所谓“山在虚无缥缈间”是也。既然她是名列蓬莱的“世外仙姝”,在人间也就没有她的立足之地了,魂归仙界,复为三生石畔的绛珠仙草。
}\par
\hop
探春也有了一个,方欲念时,宝琴走过来笑道:“我从小儿所走的地方的古迹不少,我今拣了十个地方的古迹,作了十首怀古的诗。
诗虽粗鄙,却怀往事,又暗隐俗物十件,姐姐们请猜一猜。
”众人听了,都说:“这倒巧,何不写出来大家一看?”要知端的——\par
\qi{总评:诗词之俏丽、灯谜之隐秀不待言,须看他极整齐、极参差,愈忙迫愈安闲,一波一折路转峰回,一落一起山断云连,各人局度、各人情性都现。
至李纨主坛,而起句却在凤姐,李纨主坛,而结句却在最少之李绮,另是一样弄奇。
\hang
最爱他中幅惜春作画一段,似与本文无涉,而前后文之景色人物,莫不筋动脉摇,而前后文之起伏照应,莫不穿插映带。
文字之奇,难以言状。
\zhu{
大观园图上人和景,关联到图下的人和景。作画一事起源于第四十回刘姥姥的一句玩笑话,但是在后文中作画的事又多次被提起。
在本回特意交代了贾母让惜春在画上一定要“添上”“琴儿和丫头梅花”,即为对宝琴与梅翰林之子婚姻的支持,避免梅家因薛家败落而退婚。
}
}
\dai{099}{湘云击鼓,黛玉执笔,宝玉咏红梅}
\dai{100}{暖香坞雅制春灯谜}
\sun{p50-1}{宝琴宝玉踏雪折梅}{贾母上轿,带着众人,说笑出了夹道东门。
一看四面粉妆银砌,忽见宝琴披着凫靥裘站在山坡上遥等,身后一个丫鬟抱着一瓶红梅。
宝琴背后转出一个披大红猩毡的人来。
贾母道:“那又是那个女孩儿?”众人笑道:“我们都在这里,那是宝玉。
}