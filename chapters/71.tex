\chapter{嫌隙人有心生嫌隙 \quad 鸳鸯女无意遇鸳鸯}
\qi{叙贾母开寿诞,与宁府祭宗祠是一样手笔,俱为五凤裁诏体。
\zhu{凤诏、五色诏:天子的诏书。
裁:写作。
五色诏:皇帝的命令由官员写在剪裁整齐的五色纸上,故称五色诏。
凤诏:凤池(凤凰池)是古代禁苑中的池沼,为中书省所在地,中书省是决策机构,替皇帝起草诏书(门下省是审议机构;
尚书省是执行机构)。
王维《和贾至舍人早朝大明宫之作》诗:“朝罢须裁五色诏,佩声归到凤池头。”
}}\par
话说贾政回京之后,诸事完毕,赐假一月在家歇息。
因年景渐老,事重身衰,又近因在外几年,骨肉离异,今得晏然复聚于庭室,\zhu{晏然:悠闲安适的样子。
}自觉喜幸不尽。
一应大小事务一概益发付于度外,只是看书,闷了便与清客们下棋吃酒,或日间在里面母子夫妻共叙天伦庭闱之乐。
\zhu{天伦:自然的伦常关系,如父子、兄弟等。
庭闱:父母所住的厅房。
因用以称父母。
}\par
因今岁八月初三日乃贾母八旬之庆,又因亲友全来,恐筵宴排设不开,便早同贾赦及贾珍贾琏等商议,议定于七月二十八日起至八月初五日止荣宁两处齐开筵宴,宁国府中单请官客,\zhu{官客:男客人。
}荣国府中单请堂客,\zhu{堂客:旧时称妇女内眷为堂客。
}大观园中收拾出缀锦阁并嘉荫堂等几处大地方来作退居。
\zhu{退居:指供宾客临时休息的处所。
}二十八日请皇亲、驸马、王公、诸公主、郡主、王妃、国君、太君、夫人等,\zhu{国君、太君、夫人:是按官阶赐予臣下母、妻的封号。
}二十九日便是阁下、都府、督镇及诰命等,\zhu{阁下:指入阁办事的大学士。
阁:内阁,辅佐皇帝的中央最高机关。
都府:泛指军政将帅之府署的长官。
督镇:泛指各省督抚、总兵之类的长官和将帅。
诰命:本指皇帝赐爵授官的诏令,在此义同“命妇”,代指受皇帝封赠的贵妇人。
}三十日便是诸官长及诰命并远近亲友及堂客。
初一日是贾赦的家宴,初二日是贾政,初三日是贾珍贾琏,初四日是贾府中合族长幼大小共凑的家宴。
初五日是赖大林之孝等家下管事人等共凑一日。
自七月上旬,送寿礼者便络绎不绝。
礼部奉旨:钦赐金玉如意一柄,彩缎四端,\zhu{端:布帛长度单位。
}金玉环四个,帑银五百两。
\zhu{帤银:帤音“倘”,国库所藏之钱财。
}元春又命太监送出金寿星一尊,沉香拐一只,伽南珠一串,\zhu{伽(音“茄”)南珠:用伽南香(即沉香)制成的念珠。
}福寿香一盒,金锭一对,银锭四对,彩缎十二匹,玉杯四只。
馀者自亲王驸马以及大小文武官员之家凡所来往者,莫不有礼,不能胜记。
堂屋内设下大桌案,铺了红毡,将凡所有精细之物都摆上,请贾母过目。
贾母先一二日还高兴过来瞧瞧,后来烦了,也不过目,只说:“叫凤丫头收了,改日闷了再瞧。
”\par
至二十八日,两府中俱悬灯结彩,屏开鸾凤,褥设芙蓉,笙箫鼓乐之音,通衢越巷。
\zhu{衢:音“渠”,四通八达的大路。
}宁府中本日只有北静王、南安郡王、永昌驸马、乐善郡王并几个世交公侯应袭,荣府中南安王太妃、北静王妃并几位世交公侯诰命。
贾母等皆是按品大妆迎接。
大家厮见,\zhu{厮:互相。
}先请入大观园内嘉荫堂,茶毕更衣,方出至荣庆堂上拜寿入席。
大家谦逊半日,方才入席。
上面两席是南北王妃,下面依叙,\zhu{依叙:依次叙说;或者可能是“依序”的错讹。
}便是众公侯诰命。
左边下手一席,陪客是锦乡侯诰命与临昌伯诰命,右边下手一席,方是贾母主位。
邢夫人王夫人带领尤氏凤姐并族中几个媳妇,两溜雁翅站在贾母身后侍立。
林之孝赖大家的带领众媳妇都在竹帘外面侍候上菜上酒,周瑞家的带领几个丫鬟在围屏后侍候呼唤。
凡跟来的人,早又有人管待别处去了。
\par
一时台上参了场,\zhu{参了场:旧时喜庆祝寿等演戏时,演员在开场前须出台致贺,叫“参场”。
}台下一色十二个未留发的小厮侍候。
须臾,一小厮捧了戏单至阶下,先递与回事的媳妇。
这媳妇接了,才递与林之孝家的,用一小茶盘托上,挨身入帘来递与尤氏的侍妾配凤。
配凤接了才奉与尤氏。
尤氏托着走至上席,南安太妃谦让了一回,点了一出吉庆戏文,然后又谦让了一回,北静王妃也点了一出。
众人又让了一回,命随便拣好的唱罢了。
少时,菜已四献,汤始一道,跟来各家的放了赏。
大家便更衣复入园来,另献好茶。
\par
南安太妃因问宝玉,贾母笑道:“今日几处庙里念‘保安延寿经’,他跪经去了。
”\zhu{跪经:参加寺庙诵经的一种方式。
}又问众小姐们,贾母笑道:“他们姊妹们病的病,弱的弱,见人腼腆,所以叫他们给我看屋子去了。
有的是小戏子,传了一班在那边厅上陪着他姨娘家姊妹们也看戏呢。
”南安太妃笑道:“既这样,叫人请来。
”贾母回头命凤姐儿去把史、薛、林带来,“再只叫你三妹妹陪着来罢。
”\ping{贾母看重探春,不叫迎春,为后文邢夫人生气埋下伏笔。
}凤姐答应了,来至贾母这边,只见他姊妹们正吃果子看戏,宝玉也才从庙里跪经回来。
凤姐儿说了话。
宝钗姊妹与黛玉、探春、湘云五人来至园中,大家见了,不过请安问好让坐等事。
众人中也有见过的,还有一两家不曾见过的,都齐声夸赞不绝。
\qi{人非草木,见此数人,焉得不垂涎称妙?}其中湘云最熟,南安太妃因笑道:“你在这里,听见我来了还不出来,还只等请去。
我明儿和你叔叔算帐。
”因一手拉着探春,一手拉着宝钗,问几岁了,又连声夸赞。
因又松了他两个,又拉着黛玉宝琴,也着实细看,极夸一回。
又笑道:“都是好的,你不知叫我夸那一个的是。
”早有人将备用礼物打点出五分来:金玉戒指各五个,腕香珠五串。
南安太妃笑道:“你姊妹们别笑话,留着赏丫头们罢。
”五人忙拜谢过。
北静王妃也有五样礼物,馀者不必细说。
\par
吃了茶,园中略逛了一逛,贾母等因又让入席。
南安太妃便告辞,说身上不快,“今日若不来,实在使不得,因此恕我竟先要告别了。
”贾母等听说,也不便强留,大家又让了一回,送至园门,坐轿而去。
接着北静王妃略坐一坐也就告辞了。
馀者也有终席的,也有不终席的。
\par
贾母劳乏了一日,次日便不会人,一应都是邢夫人王夫人管待。
有那些世家子弟拜寿的,只到厅上行礼,贾赦、贾政、贾珍等还礼管待,至宁府坐席。
不在话下。
\par
这几日,尤氏晚间也不回那府里去,白日间待客,晚间陪贾母顽笑,又帮着凤姐料理出入大小器皿,以及收放赏礼事务,晚间在园内李氏房中歇宿。
这日晚间伏侍过贾母晚饭后,贾母因说:“你们也乏了,我也乏了,早些寻一点子吃的歇歇去。
明儿还要起早闹呢。
”尤氏答应着退了出来,到凤姐儿房里来吃饭。
凤姐儿在楼上看着人收送礼的新围屏,只有平儿在房里与凤姐儿叠衣服。
尤氏因问:“你们奶奶吃了饭了没有?”平儿笑道:“吃饭岂不请奶奶去的。
”尤氏笑道:“既这样,我别处找吃的去。
饿的我受不得了。
”说着就走。
平儿忙笑道:“奶奶请回来。
这里有点心,且点补一点儿,\zhu{点补:谓进食少量食品。
}回来再吃饭。
”尤氏笑道:“你们忙的这样,我园里和他姊妹们闹去。
”一面说,一面就走。
平儿留不住,只得罢了。
\par
且说尤氏一径来至园中,只见园中正门与各处角门\geng{伏下文。
\zhu{本回后文,司棋趁夜色和表弟偷情。
}}仍未关,犹吊着各色彩灯,因回头命小丫头叫该班的女人。
那丫鬟走入班房中,竟没一个人影,回来回了尤氏。
尤氏便命传管家的女人。
这丫头应了便出去,到二门外鹿顶内,\zhu{鹿顶:旧式四合院东西房和南北房连接转角的地方。
}乃是管事的女人议事取齐之所。
到了这里,只有两个婆子分菜果呢。
因问:“那一位奶奶在这里?东府奶奶立等一位奶奶,有话吩咐。
”这两个婆子只顾分菜果,又听见是东府里的奶奶,不大在心上,因就回说:“管家奶奶们才散了。
”小丫头道:“散了,你们家里传他去。
”婆子道:“我们只管看屋子,不管传人。
姑娘要传人再派传人的去。
”小丫头听了道:“嗳呀,嗳呀,这可反了!怎么你们不传去?你哄那新来了的,怎么哄起我来了!素日你们不传谁传去!这会子打听了梯己信儿,\zhu{梯己:意即私人的、贴心的。
私蓄亦可称作“梯己”。
}或是赏了那位管家奶奶的东西,你们争着狗颠儿似的传去的,不知谁是谁呢。
琏二奶奶要传,你们可也这么回?”这两个婆子一则吃了酒,二则被这丫头揭挑着弊病,便羞激怒了,因回口道:“扯你的臊!我们的事,传不传不与你相干!你不用揭挑我们,你想想,你那老子娘在那边管家爷们跟前比我们还更会溜呢。
什么‘清水下杂面,你吃我也见’的事,\zhu{清水下杂面,你吃我也见:也作“清水下杂面,你吃我看”,歇后语,意思是说你安的什么心我看得清清楚楚,杂面是一种以绿豆为主制成的面条,下在清水里煮时,面是面,水是水,分得很清楚,故歇后语的后句说“你吃我看见”。
}各家门,另家户,你有本事,排场你们那边人去。
\zhu{排场:同“排揎”(揎音“宣”),数落、责难的意思。
}我们这边,你们还早些呢!”丫头听了,气白了脸,因说道:“好,好,这话说的好!”一面转身进来回话。
\par
尤氏已早入园来,因遇见了袭人、宝琴、湘云三人同着地藏庵的两个姑子正说故事顽笑,尤氏因说饿了,先到怡红院,袭人装了几样荤素点心出来与尤氏吃。
两个姑子、宝琴、湘云等都吃茶,仍说故事。
那小丫头子一径找了来,气狠狠的把方才的话都说了出来。
尤氏听了,冷笑道:“这是两个什么人?”两个姑子并宝琴湘云等听了,生怕尤氏生气,忙劝说:“没有的事,必是这一个听错了。
”两个姑子笑推这丫头道:“你这孩子好性气,\zhu{性气:性情、脾气。
}那糊涂老嬷嬷们的话,你也不该来回才是。
咱们奶奶万金之躯,劳乏了几日,黄汤辣水没吃,\zhu{黄汤辣水:泛指吃的东西。
}咱们哄他欢喜一会还不得一半儿,说这些话做什么。
”袭人也忙笑拉出他去,说:“好妹子,你且出去歇歇,我打发人叫他们去。
”尤氏道:“你不要叫人,你去就叫这两个婆子来,到那边把他们家的凤儿叫来。
”袭人笑道:“我请去。
”尤氏道:“偏不要你去。
”两个姑子忙立起身来,笑道:“奶奶素日宽洪大量,今日老祖宗千秋,奶奶生气,岂不惹人谈论。
”宝琴湘云二人也都笑劝。
尤氏道:“不为老太太的千秋,我断不依。
且放着就是了。
”\par
说话之间,袭人早又遣了一个丫头去到园门外找人,可巧遇见周瑞家的,这小丫头子就把这话告诉周瑞家的。
周瑞家的虽不管事,因他素日仗着是王夫人的陪房,
\zhu{陪房:旧时富家女子的随嫁仆人。}
原有些体面,心性乖滑,\zhu{乖滑:机灵圆滑。
}专管各处献勤讨好,所以各处房里的主人都喜欢他。
他今日听了这话,忙的便跑入怡红院来,一面飞走,一面口内说:“气坏了奶奶了,可了不得!我们家里,如今惯的太不堪了。
偏生我不在跟前,若在跟前,且打给他们几个耳刮子,再等过了这几日算帐。
”尤氏见了他,也便笑道:“周姐姐你来,有个理你说说。
这早晚门还大开着,\zhu{早晚:偏向于“晚”。
}明灯亮烛,出入的人又杂,倘有不防的事,如何使得?因此叫该班的人吹灯关门。
谁知一个人芽儿也没有。
”周瑞家的道:“这还了得!前儿二奶奶还吩咐了他们,说这几日事多人杂,一晚就关门吹灯,不是园里人不许放进去。
今儿就没了人。
这事过了这几日,必要打几个才好。
”尤氏又说小丫头子的话。
周瑞家的道:“奶奶不要生气,等过了事,我告诉管事的打他个臭死。
只问他们,谁叫他们说这‘各家门各家户’的话!我已经叫他们吹了灯,关上正门和角门子。
”正乱着,只见凤姐儿打发人来请吃饭。
尤氏道:“我也不饿了,才吃了几个饽饽,\zhu{饽饽:北平方言。
指糕点或馒头一类的食品。
}请你奶奶自吃罢。
”\par
一时周瑞家的得便出去,便把方才的事回了凤姐,又说:“这两个婆子就是管家奶奶,时常我们和他说话,都似狠虫一般。
奶奶若不戒饬,\zhu{饬:音“赤”,整顿;告诫;命令。
}
大奶奶脸上过不去。
”凤姐道:“既这么着,记上两个人的名字,等过了这几日,捆了送到那府里凭大嫂子开发,
\zhu{开发:发落,处置。}
或是打几下子,或是他开恩饶了他们,随他去就是了,什么大事。
”周瑞家的听了,巴不得一声儿,素日因与这几个人不睦,出来了便命一个小厮到林之孝家传凤姐的话,立刻叫林之孝家的进来见大奶奶,一面又传人立刻捆起这两个婆子来,交到马圈里派人看守。
\par
林之孝家的不知有什么事,此时已经点灯,忙坐车进来,先见凤姐。
至二门上传进话去,丫头们出来说:“奶奶才歇了。
大奶奶在园里,叫大娘见了大奶奶就是了。
”林之孝家的只得进园来到稻香村,丫鬟们回进去,尤氏听了反过意不去,忙唤进他来,因笑向他道:“我不过为找人找不着因问你,你既去了,也不是什么大事,谁又把你叫进来,倒要你白跑一遭。
\ping{凤姐不知,周瑞擅自做主狐假虎威,捆人并叫林之孝进来。}
不大的事,已经撒开手了。
”林之孝家的也笑道:“二奶奶打发人传我,说奶奶有话吩咐。
”尤氏笑道:“这是那里的话,只当你没去,白问你。
这是谁又多事告诉了凤丫头,大约周姐姐说的。
家去歇着罢,没有什么大事。
”李纨又要说原故,尤氏反拦住了。
林之孝家的见如此,只得便回身出园去。
可巧遇见赵姨娘,姨娘因笑道:“嗳哟哟,我的嫂子!这会子还不家去歇歇,还跑些什么?”林之孝家的便笑说何曾不家去的,如此这般进来了。
又是个齐头故事。
\zhu{齐头故事:
“齐头故事”即“无头公案”。这里的“无头”指的是“没头绪”。
从正文文意看,林之孝家的在入夜时分受传唤忙忙的进园来,又说并无大事,打发她回去,因觉得莫名其妙,“又是个齐头故事”,无事而被传唤,令人摸不着头脑。
}赵姨娘原是好察听这些事的,且素日又与管事的女人们扳厚,\zhu{扳:同“攀”。扳厚:因扳关系而有交情。
}互相连络,好作首尾。
\zhu{首尾:意为呼应、照应。
好作首尾:串通一气。
}方才之事,已竟闻得八九,听林之孝家的如此说,便恁般如此告诉了林之孝家的一遍,\zhu{恁:音“嫩”,如此,这样,那。
}林之孝家的听了,笑道:“原来是这事,也值一个屁!开恩呢,就不理论,心窄些儿,也不过打几下子就完了。
”赵姨娘道:“我的嫂子,事虽不大,可见他们太张狂了些。
巴巴的传进你来,明明戏弄你,顽算你。
快歇歇去,明儿还有事呢,也不留你吃茶去。
”\par
说毕,林之孝家的出来,到了侧门前,就有方才两个婆子的女儿上来哭着求情。
林之孝家的笑道:“你这孩子好糊涂,谁叫你娘吃酒混说了,惹出事来,连我也不知道。
二奶奶打发人捆他,连我还有不是呢。
我替谁讨情去。
”这两个小丫头子才七八岁,原不识事,只管哭啼求告。
缠的林之孝家的没法,因说道:“糊涂东西!你放着门路不去,却缠我来。
你姐姐现给了那边太太作陪房费大娘的儿子,你走过去告诉你姐姐,叫亲家娘和太太一说,什么完不了的事!”一语提醒了一个,那一个还求。
林之孝家的啐道:“糊涂攮的!他过去一说,自然都完了。
没有个单放了他妈,又只打你妈的理。
”说毕,上车去了。
\par
这一个小丫头果然过来告诉了他姐姐,和费婆子说了。
这费婆子原是邢夫人的陪房,起先也曾兴过时,只因贾母近来不大作兴邢夫人,\zhu{不作兴:此为不待见、不喜欢、不抬举之意。
}所以连这边的人也减了威势。
凡贾政这边有些体面的人,那边各各皆虎视耽耽。
这费婆子常倚老卖老,仗着邢夫人,常吃些酒,嘴里胡骂乱怨的出气。
如今贾母庆寿这样大事,干看着人家逞才卖技办事,呼幺喝六弄手脚,心中早已不自在,指鸡骂狗,闲言闲语的乱闹。
这边的人也不和他较量。
如今听了周瑞家的捆了他亲家,越发火上浇油,仗着酒兴,指着隔断的墙\geng{细致之甚。
}大骂了一阵,便走上来求邢夫人,说他亲家并没什么不是,“不过和那府里的大奶奶的小丫头白斗了两句话,周瑞家的便调唆了咱家二奶奶捆到马圈里,等过了这两日还要打。
求太太——我那亲家娘也是七八十岁的老婆子——和二奶奶说声,饶他这一次罢。
”邢夫人自为要鸳鸯之后讨了没意思,后来见贾母越发冷淡了他,凤姐的体面反胜自己,且前日南安太妃来了,要见他姊妹,贾母又只令探春出来,迎春竟似有如无,自己心内早已怨忿不乐,只是使不出来。
又值这一干小人在侧,他们心内嫉妒挟怨之事不敢施展,便背地里造言生事,调拨主人。
先不过是告那边的奴才,后来渐次告到凤姐“只哄着老太太喜欢了他好就中作威作福,辖治着琏二爷,调唆二太太,把这边的正经太太倒不放在心上。
”后来又告到王夫人,说:“老太太不喜欢太太,都是二太太和琏二奶奶调唆的。
”邢夫人纵是铁心铜胆的人,妇女家终不免生些嫌隙之心,近日因此着实恶绝凤姐。
\ping{当贾琏偷娶尤二姐后,贾赦还赏给贾琏秋桐为妾,可能是贾赦邢夫人故意要气凤姐。
}今听了如此一篇话,也不说长短。
\zhu{长短:情由,情况。
}\par
至次日一早,见过贾母,众族中人到齐,坐席开戏。
贾母高兴,又见今日无远亲,都是自己族中子侄辈,只便衣常妆出来,堂上受礼。
当中独设一榻,引枕靠背脚踏俱全,\zhu{引枕:坐时搭扶胳膊的一种圆墩形的倚枕。
}自己歪在榻上。
榻之前后左右,皆是一色的小矮凳,宝钗、宝琴、黛玉、湘云、迎春、探春、惜春姊妹等围绕。
因贾㻞之母也带了女儿喜鸾,贾琼之母也带了女儿四姐儿,还有几房的孙女儿,大小共有二十来个。
贾母独见喜鸾和四姐儿生得又好,说话行事与众不同,心中喜欢,便命他两个也过来榻前同坐。
宝玉却在榻上脚下与贾母捶腿。
首席便是薛姨妈,下边两溜皆顺着房头辈数下去。
\zhu{房头:家族的分支。
}
帘外两廊都是族中男客,也依次而坐。
\par
先是那女客一起一起行礼,后方是男客行礼。
贾母歪在榻上,只命人说“免了罢”,早已都行完了。
然后赖大等带领众家人,从仪门直跪至大厅上,磕头礼毕,又是众家下媳妇,然后各房的丫鬟,足闹了两三顿饭时。
然后又抬了许多雀笼来,在当院中放了生。
贾赦等焚过了天地寿星纸,\zhu{天地:天地神灵。
}方开戏饮酒。
直到歇了中台,\zhu{中台:旧时演戏,开场时观众尚未到齐,例由次要演员先演开场戏;“中台”才由主要演员演出正本戏。
}贾母方进来歇息,命他们取便,因命凤姐儿留下喜鸾四姐儿顽两日再去。
凤姐儿出来便和他母亲说,他两个母亲素日都承凤姐的照顾,也巴不得一声儿。
他两个也愿意在园内顽耍,至晚便不回家了。
\par
邢夫人直至晚间散时,当着许多人陪笑和凤姐求情说:“我听见昨儿晚上二奶奶生气,\ping{邢夫人叫儿媳妇用敬称,体现邢夫人的不满和对王熙凤的疏远。
}打发周管家的娘子捆了两个老婆子,可也不知犯了什么罪。
论理我不该讨情,我想老太太好日子,发狠的还舍钱舍米,周贫济老,咱们家先倒折磨起老人家来了。
不看我的脸,权且看老太太,竟放了他们罢。
”说毕,上车去了。
凤姐听了这话,又当着许多人,又羞又气,一时抓寻不着头脑,
\ping{周瑞家的擅自捆人,连累凤姐。}
憋得脸紫涨,回头向赖大家的等笑道:\geng{又写笑,妙!凡凤真怒处必曰“笑”,丝丝不错。
}“这是那里的话。
昨儿因为这里的人得罪了那府里的大嫂子,我怕大嫂子多心,所以尽让他发放,并不为得罪了我。
这又是谁的耳报神这么快。
”\zhu{耳报神:比喻喜好私下传递消息的人。
}王夫人因问为什么事,凤姐儿笑将昨日的事说了。
尤氏也笑道:“连我并不知道。
你原也太多事了。
”凤姐儿道:“我为你脸上过不去,所以等你开发,不过是个礼。
就如我在你那里有人得罪了我,你自然送了来尽我。
\zhu{尽我:前文“我怕大嫂子多心,所以尽让他发放”,可知这里的“尽”是“任凭、纵使”的意思。
}凭他是什么好奴才,到底错不过这个礼去。
这又不知谁过去没的献勤儿,这也当作一件事情去说。
”王夫人道:“你太太说的是。
就是珍哥儿媳妇也不是外人,也不用这些虚礼。
老太太的千秋要紧,放了他们为是。
”说着,回头便命人去放了那两个婆子。
凤姐由不得越想越气越愧,不觉的灰心转悲,滚下泪来。
因赌气回房哭泣,又不使人知觉。
偏是贾母打发了琥珀来叫立等说话。
\zhu{立等:立着等候。
多指时间短暂。
}琥珀见了,诧异道:“好好的,这是什么原故?那里立等你呢。
”凤姐听了,忙擦干了泪,洗面另施了脂粉,方同琥珀过来。
\par
贾母因问道:“前儿这些人家送礼来的共有几家有围屏?”凤姐儿道:“共有十六家有围屏,十二架大的,四架小的炕屏。
\zhu{炕屏:陈设在炕上的一种小屏风。
}内中只有江南甄家\geng{好,一提甄事。
盖真事将显,假事将尽。
}一架大屏十二扇,大红缎子缂丝‘满床笏’,\zhu{缂丝:同“刻丝”。
我国特有的一种丝织工艺。
织造时,以细丝为经,彩色丝作纬,各色纬丝仅于图案花纹需要处与经丝交织,纬丝不贯串全幅,而经丝则纵贯织品。
满床笏:清代传奇剧,演唐郭子仪“七子八婿,富贵寿考”的故事。
}一面是泥金‘百寿图’的,\zhu{泥金:涂以金粉作底。
}是头等的。
还有粤海将军邬家一架玻璃的还罢了。
”贾母道:“既这样,这两架别动,好生搁着,我要送人的。
”凤姐儿答应了。
\ping{礼单必须留着,知道什么东西是谁送的,不能到时候又给人送回去。所以可能是江南甄家送的“满床笏”,下一次会送到南安太妃那里。}
\par
鸳鸯忽过来向凤姐儿面上只管瞧,引的贾母问说:\ping{凤姐不好向贾母直说自己受委屈,鸳鸯帮助凤姐引起贾母注意才有机会获得贾母支持。
}“你不认得他?只管瞧什么。
”鸳鸯笑道:“怎么他的眼肿肿的,所以我诧异,只管看。
”贾母听说,便叫进前来,也觑着眼看。
\zhu{觑:音“去”,眯着眼注视。
}凤姐笑道:“才觉的一阵痒痒,揉肿了些。
”鸳鸯笑道:“别又是受了谁的气了不成?”凤姐道:“谁敢给我气受,便受了气,老太太好日子,我也不敢哭的。
”贾母道:“正是呢。
我正要吃晚饭,你在这里打发我吃,剩下的你就和珍儿媳妇吃了。
你两个在这里帮着两个师傅替我拣佛豆儿,\zhu{拣佛豆儿、结寿缘:旧时生日,众人一面念佛,一面拣豆,叫“拣佛豆儿”;然后把佛豆煮熟,在街口分送行人,以求添寿,叫做“结寿缘”。
}你们也积积寿,前儿你姊妹们和宝玉都拣了,如今也叫你们拣拣,别说我偏心。
”说话时,先摆上一桌素的来。
两个姑子吃了,然后才摆上荤的,贾母吃毕,抬出外间。
尤氏凤姐儿二人正吃,贾母又叫把喜鸾四姐儿二人也叫来,跟他二人吃毕,洗了手,点上香,捧过一升豆子来。
两个姑子先念了佛偈,\zhu{偈(音“记”):梵文音译“偈陀[jì tuó]”或“伽陀[qié tuó]”之略,意译为颂。
一般为四句之韵文。
}然后一个一个的拣在一个簸箩内,每拣一个,念一声佛。
明日煮熟了,令人在十字街结寿缘。
贾母歪着听两个姑子又说些佛家的因果善事。
\par
鸳鸯早已听见琥珀说凤姐哭之事,又和平儿前打听得原故。
晚间人散时,便回说:“二奶奶还是哭的,那边大太太当着人给二奶奶没脸。
”贾母因问为什么原故,鸳鸯便将原故说了。
贾母道:“这才是凤丫头知礼处,难道为我的生日由着奴才们把一族中的主子都得罪了也不管罢。
这是太太素日没好气,不敢发作,所以今儿拿着这个作法子,\zhu{作法:就是树立某种标准,给别人立规矩,通过责骂、惩罚等手段处理某人立威,杀鸡儆猴,以儆其馀。
}明是当着众人给凤儿没脸罢了。
”正说着,只见宝琴等进来,也就不说了。
贾母因问:“你在那里来?”宝琴道:“在园里林姐姐屋里大家说话的。
”贾母忽想起一事来,忙唤一个老婆子来,吩咐他:“到园里各处女人们跟前嘱咐嘱咐,留下的喜姐儿和四姐儿虽然穷,也和家里的姑娘们是一样,大家照看经心些。
我知道咱们家的男男女女都是‘一个富贵心,两只体面眼’,\zhu{体面眼:即势利眼,只看得起有身份、有体面的人。
}未必把他两个放在眼里。
有人小看了他们,我听见可不依。
”婆子应了方要走时,鸳鸯道:“我说去罢。
他们那里听他的话。
”说着,便一径往园子来。
\par
先到稻香村中,李纨与尤氏都不在这里。
问丫鬟们,说“都在三姑娘那里呢。
”鸳鸯回身又来至晓翠堂,果见那园中人都在那里说笑。
见他来了,都笑说:“你这会子又跑来做什么?”又让他坐。
鸳鸯笑道:“不许我也逛逛么?”于是把方才的话说了一遍。
李纨忙起身听了,就叫人把各处的头儿唤了一个来。
令他们传与诸人知道。
不在话下。
\par
这里尤氏笑道:“老太太也太想的到,实在我们年轻力壮的人捆上十个也赶不上。
”李纨道:“凤丫头仗着鬼聪明儿,还离脚踪儿不远。
咱们是不能的了。
”鸳鸯道:“罢哟,还提凤丫头虎丫头呢,他也可怜见儿的。
虽然这几年没有在老太太、太太跟前有个错缝儿,暗里也不知得罪了多少人。
总而言之,为人是难作的:若太老实了没有个机变,公婆又嫌太老实了,家里人也不怕;若有些机变,未免又治一经损一经。
\zhu{治一经损一经:本中医术语,借喻顾此失彼、好了这头又坏了那头。
}
如今咱们家里更好,新出来的这些底下奴字号的奶奶们,一个个心满意足,都不知要怎么样才好,少有不得意,不是背地里咬舌根,就是挑三窝四的。
\zhu{调三窝四:搬弄口舌、挑拨是非。
也作“调三斡四”、“挑三豁四”、“挑三窝四”。
}我怕老太太生气,一点儿也不肯说。
不然我告诉出来,大家别过太平日子。
这不是我当着三姑娘说,老太太偏疼宝玉,有人背地里怨言还罢了,算是偏心。
如今老太太偏疼你,我听着也是不好。
这可笑不可笑?”探春笑道:“糊涂人多,那里较量得许多。
我说倒不如小人家人少,虽然寒素些,倒是欢天喜地,大家快乐。
我们这样人家人多,外头看着我们不知千金万金小姐,何等快乐,殊不知我们这里说不出来的烦难,更利害。
”\zhu{利害:厉害。
}\par
宝玉道:“谁都像三妹妹好多心。
事事我常劝你,总别听那些俗语,想那俗事,只管安富尊荣才是。
比不得我们没这清福,\zhu{比不得我们没这清福:比不得我们(的人),(他们)没这清福。
}该应浊闹的。
”尤氏道:“谁都像你,真是一心无挂碍,只知道和姊妹们顽笑,饿了吃,困了睡,再过几年,不过还是这样,一点后事也不虑。
”宝玉笑道:“我能够和姊妹们过一日是一日,死了就完了。
什么后事不后事。
”李纨等都笑道:“这可又是胡说。
就算你是个没出息的,终老在这里,难道他姊妹们都不出门的?”\zhu{出门:女子出嫁。
}尤氏笑道:“怨不得人都说他是假长了一个胎子,究竟是个又傻又呆的。
”宝玉笑道:“人事莫定,知道谁死谁活。
倘或我在今日明日,今年明年死了,也算是遂心一辈子了。
”众人不等说完,便说:“可是又疯了,别和他说话才好。
若和他说话,不是呆话就是疯话。
”喜鸾因笑道:“二哥哥,你别这样说,等这里姐姐们果然都出了阁,\zhu{出阁:古时称公主出嫁为“出阁”,今用于指女子出嫁。
}横竖老太太、太太也寂寞,我来和你作伴儿。
”李纨尤氏等都笑道:“姑娘也别说呆话,难道你是不出门的?这话哄谁。
”说的喜鸾低了头。
当下已是起更时分,\zhu{起更:俗称五更中的初更时分,约晚上七点左右。
}大家各自归房安歇,众人都且不提。
\par
且说鸳鸯一径回来,刚至园门前,只见角门虚掩,犹未上闩。
此时园内无人来往,只有该班的房内灯光掩映,微月半天。
\geng{是月初旬起更时也。
\zhu{初旬:上旬。}
}鸳鸯又不曾有个作伴的,也不曾提灯笼,独自一个,脚步又轻,所以该班的人皆不理会。
偏生又要小解,因下了甬路,寻微草处,\zhu{寻:沿着、顺着。
}行至一湖山石后大桂树阴下来。
\zhu{湖山石:用太湖石堆叠而成的假山。
}\geng{是八月,随笔点景。
}刚转过石后,只听一阵衣衫响,吓了一惊不小。
定睛一看,只见是两个人在那里,见他来了,便想往石后树丛藏躲。
鸳鸯眼尖,趁月色见准一个穿红裙子、梳鬅头、\zhu{
鬅:音“朋”,头发散乱的样子。
鬅头:一种发髻鬅松的女子发式。
}
高大丰壮身材的,\geng{是月下所见之像,故不写至容貌也。
}是迎春房里的司棋。
鸳鸯只当他和别的女孩子也在此方便,见自己来了,故意藏躲恐吓着耍,\geng{此见是女儿们常事,观书者自亦为如此。
\ping{这里的“为”,意思如果是“认为”,那么批书人不一定也干过这样的事,只是认为这是女儿们常事;如果是“做”,那么可以推断出批书人是一个女性,她曾经做过这样的事。
}}因便笑叫道:“司棋你不快出来,吓着我,我就喊起来当贼拿了。
这么大丫头了,没个黑家白日的只是顽不够。
”\zhu{没个黑家白日的:不分白天夜晚。
}这本是鸳鸯的戏语,叫他出来。
谁知他贼人胆虚,\geng{更奇,不知后为何事。
}只当鸳鸯已看见他的首尾了,\zhu{首尾:头和尾,引申指事情的经过始末。
这里指男女关系。
}生恐叫喊起来使众人知觉更不好,且素日鸳鸯又和自己亲厚不比别人,便从树后跑出来,一把拉住鸳鸯,便双膝跪下,只说:“好姐姐,千万别嚷!”鸳鸯反不知因何,忙拉他起来,笑问道:“这是怎么说?”司棋满脸红胀,又流下泪来。
鸳鸯再一回想,那一个人影恍惚像个小厮,心下便猜疑了八九,\geng{是聪敏女儿,妙!}自己反羞的面红耳赤,\geng{是娇贵女儿,笔笔皆到。
}又怕起来。
因定了一会,忙悄问:“那个是谁?”司棋复跪下道:“是我姑舅兄弟。
”鸳鸯啐了一口,道:“要死,要死。
”\geng{如见其面,如闻其声。
}
司棋又回头悄道:“你不用藏着,姐姐已看见了,快出来磕头。
”那小厮听了,只得也从树后爬出来,磕头如捣蒜。
鸳鸯忙要回身,司棋拉住苦求,哭道:“我们的性命,都在姐姐身上,只求姐姐超生要紧!”\zhu{超生:宽宥其生命。
常用于祈求他人怜悯救助。
}鸳鸯道:“你放心,我横竖不告诉一个人就是了。
”一语未了,只听角门上有人说道:“金姑娘已出去了,角门上锁罢。
”鸳鸯正被司棋拉住,不得脱身,听见如此说,便接声道:“我在这里有事,且略住手,我出来了。
”司棋听了,只得松手让他去了——\par
\qi{总评:叙一番灯火未息,门户未关。
叙一番赵姨失体,费婆憋气。
叙一番林家托大,周家献勤。
叙一番凤姐灰心,鸳鸯传信。
非为本文渲染,全为下文引逗,良工苦心,可谓惨淡经营。
\zhu{惨淡:苦费心思;经营:筹划。
惨淡经营:费尽心思辛辛苦苦地经营筹划。
后指在困难的境况中艰苦地从事某种事业。
}\hang
司棋事从鸳鸯误吓得来,是善周全处。
方与鸳鸯前后行景不致矛盾。
\zhu{行景:状况、情形。
}
一何精细如此。
}
\dai{141}{邢夫人上车前为老婆子和凤姐求情}
\dai{142}{鸳鸯女无意遇鸳鸯}
\sun{p71-1}{贾母八十大寿}{贾母八十寿辰,荣宁两府连日筵宴。
上面两席是南北王妃,其余客人依序坐下。
邢王二夫人带领尤氏,凤姐等雁翅般在贾母身后侍立。
周瑞家的带领丫鬟在围屏后听唤。
史湘云、宝钗姊妹、林黛玉、贾探春前来请安问好,南安太妃拉着连声夸赞。
}
\sun{p71-2}{邢夫人指责凤姐,鸳鸯撞见偷情事}{邢夫人当着许多人,陪笑和凤姐向犯事的老婆子求情说:“老太太好日子,发狠的还舍钱舍米,周贫济老,咱们家先倒折磨起老人家来了。
不看我的脸,权且看老太太,竟放了他们罢。
”弄得凤姐又羞又气。
鸳鸯去晓翠堂传贾母话,回途中,无意碰见司棋与人偷情,司棋忙跪下求鸳鸯保守秘密。
}