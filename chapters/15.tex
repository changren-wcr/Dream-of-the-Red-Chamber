\chapter{王熙凤弄权铁槛寺\quad 秦鲸卿得趣馒头庵}
\jia{宝玉谒北静王辞对神色,方露出本来面目,迥非在闺阁中之形景。
\hang
北静王问玉上字果验否,政老对以未曾试过,是隐却多少捕风捉影闲文。
\hang
北静王论聪明伶俐,又年幼时为溺爱所累,亦大得病源之语。
\hang
凤姐中火,\ping{“中火”令人费解。
有一种说法是,“中火”意为“打尖”,清人福格《听雨丛谈》卷十一记:“今人行役,于日中投店而饭,谓之打尖。
皆不喻其字义,或曰中途为住宿之间,乃误‘间’而为‘尖’也。
谨按《翠华寻幸》,谓中顿曰‘中火’。
又见宋元人小说,谓途中之餐曰‘打火’,自是因‘火’字而误为‘尖’也。
”}写纺线村姑,是宝玉闲花野景一得情趣。
\hang
凤姐另住,明明系秦、玉、智能幽事,却是为净虚钻营凤姐大大一件事作引。
\hang
秦、智幽情,忽写宝、秦事云:“不知算何账目,未见真切,不曾记得,此系疑案,不敢纂创。
”是不落套中,且省却多少累赘笔墨。
昔安南国使有题一丈红句云:“五尺墙头遮不得,留将一半与人看。
”}\par
\qi{欲显铮铮不避嫌,英雄每入小人缘。
鲸卿些子风流事,胆落魂销已可怜。
}\par
诗云:……\par
\hop
话说宝玉举目见北静王水溶头上戴着洁白簪缨银翅王帽,穿着江牙海水五爪坐龙白蟒袍,\zhu{江牙(江崖)海水图纹样式见页脚\foot{
\footPic{江牙(江崖)海水图纹样式}{jiangya.jpg}{0.6}}:由“海水纹”和“山崖纹”两个部分构成。
海水纹,又分为“平水”和“立水”。
平水纹指图案上方螺旋状卷曲的横线曲线。
立水纹表现为图案下方的并列的斜向排列的波浪线,比平水纹长且卷曲弧度小,俗称“水脚”。
在封建社会中,海水江崖纹中水纹的元素象征着江海湖泊,海水意指“海潮”,“潮”与“朝”同音,故成为官服之专用纹饰;山崖纹,也为姜芽,意为山头重叠,和姜的芽一样,象征着江山。
海水江崖纹暗含着上层统治者美好寓意,隐喻“江山一统”、“万世开平”和“国土永固”之意。
坐龙:盘成圆形的龙纹统称“团龙”,其中头部在上者称“升龙”,头部朝下者称“降龙”,头部呈正面者称“正龙”,头部呈侧面者称“坐龙”。
蟒袍:又名“花衣”,因袍上绣有蟒纹而得名。
明代官员非特赐不许擅服蟒袍。
清代皇子、亲王、郡王以下,文武八、九品官以上,凡遇典礼,皆穿蟒袍。
}系着碧玉红鞓带,\zhu{鞓:音“听”,皮革制成的带子。
}面如美玉,目似明星,真好秀丽人物。
宝玉忙抢上来参见,水溶连忙从轿内伸出手来挽住。
见宝玉戴着束发银冠,勒着双龙出海抹额,穿着白蟒箭袖,
\zhu{箭袖:原为便于射箭穿的窄袖衣服,这里指男子穿的一种服式。}
围着攒珠银带,面若春花,目如点漆。
\jia{又换此一句,如见其形。
}水溶笑道:“名不虚传,果然如‘宝’似‘玉’。
”因问:“衔的那宝贝在那里?”宝玉见问,连忙从衣内取了递与过去。
水溶细细的看了,又念了那上头的字,因问:“果灵验否?”贾政忙道:“虽如此说,只是未曾试过。
”水溶一面极口称奇道异,一面理好彩绦,亲自与宝玉戴上,\jia{钟爱之至。
}又携手问宝玉几岁,读何书。
宝玉一一答应。
\par
水溶见他言语清楚,谈吐有致,\geng{八字道尽玉兄,如此等方是玉兄正文写照。
壬午季春。
}\ping{这意思是平日犯浑厮混不过是不上心,正经认真无人能比?}一面又向贾政笑道:“令郎真乃龙驹凤雏,非小王在世翁前唐突,将来‘雏凤清于老凤声’,\zhu{“雏凤清于老凤声”:比喻儿子将胜过父亲。
}\jia{妙极!开口便是西昆体,\zhu{西昆体:中国宋代初期的宋诗与文学流派,因《西昆酬唱集》而得名。
}宝玉闻之,宁不刮目哉?}未可量也。
”贾政忙陪笑道:“犬子岂敢谬承金奖。
赖藩郡馀祯,\zhu{赖藩郡馀祯:犹言“托郡王的福”。
藩郡;这里指受封的郡王。
祯:吉祥。
}果如是言,亦荫生辈之幸矣。
”\geng{谦的得体。
}水溶又道:“只是一件,令郎如是资质,想老太夫人、夫人辈自然钟爱极矣;但吾辈后生,甚不宜钟溺,钟溺则未免荒失学业。
昔小王曾蹈此辙,想令郎亦未必不如是也。
若令郎在家难以用功,不妨常到寒第。
小王虽不才,却多蒙海上众名士凡至都者,未有不另垂青目,\zhu{垂青目:也作“垂青”,意思是用青眼(即正眼)看人,表示尊重、看得起。
《晋书·阮籍传》:“籍又能为青白眼。
见礼俗之士,以白眼对之。
及嵇喜来吊,籍作白眼,喜不怿而退。
喜弟康闻之,乃赍酒扶琴造焉,籍大悦,乃见青眼。
”}是以寒第高人颇聚。
令郎常去谈会谈会,则学问可以日进矣。
”贾政忙躬身答应。
\par
水溶又将腕上一串念珠卸了下来,递与宝玉道:“今日初会,仓促竟无敬贺之物,此系前日圣上亲赐鹡鸰香念珠一串,\zhu{鹡鸰:音“及灵”,据师旷《禽经》上说,鹡鸰这种鸟群体情深,一只离群,其他鸟就鸣叫以寻找同类。
历代皆用鹡鸰来比喻兄弟手足情深。
从前文对于贾宝玉和北静王的衣着的描写可以看到,他们俩人都是一身素白,不仅服饰上相似,北静王就好像贾宝玉的哥哥那样关心贾宝玉的教育问题,最后送了有兄弟之情寓意的鹡鸰香念珠。
}权为贺敬之礼。
”宝玉连忙接了,回身奉与贾政。
\geng{转出没调教。
\ping{
首先可以肯定,这里不能做“没调教”的贬义理解。证据如下:
回前批:“宝玉谒北静王辞对神色,方露出本来面目,迥非在闺阁中之形景。”
第五十六回:贾母也笑道:“……可知你我这样人家的孩子们,凭他们有什么刁钻古怪的毛病儿,见了外人,必是要还出正经礼数来的。……见人礼数竟比大人行出来的不错,使人见了可爱可怜,……”
由此可见,宝玉的礼数周到,不能理解为“没调教”。
对于这条评语的理解,一种说法是,宝玉回身奉与贾政,然后贾政和宝玉一起谢过这个举动,并没有贾政在旁调教,而是父子平时就守礼懂规矩,心照不宣,自然而然。
脂批应是在夸宝玉没有贾政当场调教,也照样行事妥当,不像有的孩子,谢恩领赏说什么话,行什么事,还要大人当场教他;
另一种说法是,“转出没调教”的断句为“转/出没调教”。
“出没”有指“进退”、“内外”或“人前人后”的意思。“出没调教”应该是说进退规矩、在内在外的调教或人前人后的礼数。而“转”,是脂砚斋在提醒:这里笔锋一转,开始体现贾家的家教了。
}}贾政与宝玉一齐谢过。
于是贾赦、贾珍等一齐上来请回舆,
\zhu{舆:车;轿子。}
水溶道:“逝者已登仙界,非碌碌你我尘寰中之人也。
小王虽上叩天恩,虚邀郡袭,岂可越仙輀而进也?”\zhu{輀:音“儿”,古时载运灵柩的车子。
}
贾赦等见执意不从,只得告辞谢恩回来,命手下掩乐停音,滔滔然将殡过完,\geng{有层次,好看煞。
}方让水溶回舆去了。
不在话下。
\par
且说宁府送殡,一路热闹非常。
刚至城门前,又有贾赦、贾政、贾珍等诸同僚属下各家祭棚接祭,一一的谢过,然后出城,竟奔铁槛寺大路行来。
彼时贾珍带贾蓉来到诸长辈前,让坐轿上马,因而贾赦一辈的各自上了车轿,贾珍一辈的也将要上马。
凤姐因记挂着宝玉,\jia{千百件忙事内不漏一丝。
}\geng{细心人自应如是。
}怕他在郊外纵性逞强,不服家人的话,贾政管不着这些小事,惟恐有个闪失,难见贾母,因此便命小厮来唤他。
宝玉只得来到他的车前。
凤姐笑道:“好兄弟,你是个尊贵人,女孩儿一样的人品,\jia{非此一句宝玉必不依,阿凤真好才情。
}别学他们猴在马上。
下来,咱们姐儿两个坐车,岂不好?”宝玉听说,便忙下了马,爬入凤姐车上,二人说笑前进。
\par
不一时,只见从那边两骑马压地飞来,\geng{有气有声,有形有影。
}离凤姐车不远,一齐蹿下来,扶车回说:“这里有下处,奶奶请歇息、更衣。
”\zhu{更衣:对大小便的雅称。
}
凤姐急命请邢夫人、王夫人的示下,\geng{有次序。
}那人回来说:“太太们说不用歇了,叫奶奶自便罢。
”凤姐听了,便命歇歇再走。
众小厮听了,一带辕马,岔出人群,往北飞走。
宝玉在车内急命请秦相公。
那时秦钟正骑马随着他父亲的轿,忽见宝玉的小厮跑来,请他去打尖。
\zhu{打尖:在旅途或劳作中间休息、饮食,俗称“打尖”。
}秦钟看时,只见凤姐儿的车往北而去,后面拉着宝玉的马,搭着鞍笼,便知宝玉同凤姐坐车,自己也便带马赶上来,同入一庄门内。
早有家人将众庄汉撵尽。
\zhu{家人:仆役。}
那村庄人家无多房舍,婆娘们无处回避,只得由他们去了。
那些村姑、庄妇见了凤姐、宝玉、秦钟的人品衣服、礼数款段,\zhu{款段:形容仪态举止从容舒缓的样子。
}岂有不爱看的?\par
一时凤姐进入茅堂,因命宝玉等先出去顽顽。
宝玉等会意,因同秦钟出来,带着小厮们各处游顽。
凡庄农动用之物,皆不曾见过。
\geng{真,毕真!}宝玉一见了锹、锄、镢、犁等物,皆以为奇,不知何项所使,其名为何。
\jia{凡膏粱子弟齐来着眼。
}\ping{这是真四体不勤五谷不分。
}小厮在旁一一的告诉了名色,
\zhu{名色:事物的名称、名字。}
说明原委。
\jia{也盖因未见之故也。
}宝玉听了,因点头叹道:“怪道古人诗上说:‘谁知盘中餐,粒粒皆辛苦’,正为此也。
”\jia{聪明人自是一喝即悟。
}
\geng{写玉兄正文总于此等处,作者良苦。
壬午季春。
}一面说,一面又至一间房前,只见炕上有个纺车,宝玉又问小厮们:“这又是什么?”小厮们又告诉他原委。
宝玉听说,便上来拧转作耍,自为有趣。
\ping{古代男耕女织,宝玉不去玩锹、锄、镢、犁这些男人用的农具,反而对女人用的纺车如此感兴趣,宝玉有一颗女儿心。
}只见一个约有十七八岁的村庄丫头跑了来乱嚷:“别动坏了!”\geng{天生地设之文。
}众小厮忙断喝拦阻,宝玉忙丢开手,陪笑说道:\geng{一“忙”字,二“陪笑”字,写玉兄是在女儿分上。
壬午季春。
}“我因为没见过这个,所以试他一试。
”那丫头道:“你们那里会弄这个,站开了,\jia{如闻其声,见其形。
}\geng{三字如闻。
}\meng{这丫头是技痒,是多情,是自己生活恐至损坏?\zhu{生活:工业,农业和手工业的劳动或者劳动工具,这里指纺车。
}宝玉此时一片心神,另有主张。
}我纺与你瞧。
”秦钟暗拉宝玉笑道:“此卿大有意趣。
”\geng{忙中闲笔;却伏下文。
}宝玉一把推开,笑道:“该死的!\jia{的是宝玉生性之言。
}
再胡说,我就打了!”\geng{玉兄身分本心如此。
}说着,只见那丫头纺起线来。
宝玉正要说话时,\geng{若说话,便不是《石头记》中文字也。
}只听那边老婆子叫道:“二丫头,快过来!”那丫头听见,丢下纺车,一径去了。
\par
宝玉怅然无趣。
\jia{处处点“情”,又伏下一段后文。
}只见凤姐打发人来叫他两个进去。
凤姐洗了手,换衣服,抖灰土,问他们换不换。
宝玉不换,只得罢了。
家下仆妇们将带着行路的茶壶茶杯、十锦屉盒、各样小食端来,\zhu{十锦:即“什锦”,由多种原料制成或多种花样拼成的。
}凤姐等吃过茶,待他们收拾完备,便起身上车。
外面旺儿预备下赏封,赏了本村主人,庄妇等来叩赏。
凤姐并不在意,宝玉却留心看时,内中并无二丫头。
\geng{妙在不见。
}一时上了车,出来走不多远,只见迎头二丫头怀里抱着他小兄弟,\geng{妙在此时方见,错综之妙如此!}同着几个小女孩子说笑而来。
宝玉恨不得下车跟了他去,料是众人不依的,少不得以目相送,争奈车轻马快,\jia{四字有文章。
人生离聚亦未尝不如此也。
}
一时展眼无踪。
\ping{这是宝玉和二丫头的一面之缘,今生可能永无再见之日。
}\par
走不多时,仍又跟上了大殡。
早有前面法鼓金铙,\zhu{
法鼓:佛教名词,寺院法器之一。法堂设二鼓,东北角之鼓,称“法鼓”,西北角之鼓,称“茶鼓”。
铙:音“挠”,打击乐器。与钹形制相似,只是中间隆起部分较小,发音较低,余音较长。
金铙:法器之一。
}幢幡宝盖,\zhu{
幢幡:音“床番”,都是旗子一类的东西。
幢:竿头安装宝珠,竿身饰以锦帛的旗子。
幡:音“翻”,挑起来直着挂的长条形旗子。
宝盖:佛教用物,圆筒形丝帛制品,是佛像上所置的饰物。
}铁槛寺接灵众僧齐至。
少时,到入寺中,另演佛事,重设香坛。
安灵于内殿偏室之中,宝珠安理寝室相伴。
外面贾珍款待一应亲友,也有扰饭的,也有不吃饭而辞的,一应谢过乏,从公侯伯子男一起一起的散去,至未末时分方才散尽了。
\zhu{未末:将近下午三点。}
里面的堂客,皆是凤姐张罗接待,先从显官诰命散起,也到晌午大错时方散尽了。
\zhu{晌午大错:晌午,正午;晌午错,正午已过;晌午大错,正午过去较久而未到下一个时辰。
}只有几个亲戚是至近的,等做过三日安灵道场方去。
那时邢、王二夫人知凤姐必不能回家,也便就要进城。
王夫人要带宝玉去,宝玉乍到郊外,那里肯回去,只要跟凤姐住着。
王夫人无法,只得交与凤姐便回来了。
\par
原来这铁槛寺原是宁荣二公当日修造,现今还是有香火地亩布施,\zhu{香火地亩布施:施舍地亩给寺庙,其收入作为香火之费。
}
以备京中老了人口,\zhu{老了:即“死了”。
旧俗讳说“死”字,故用“老”字代替。
}在此便宜寄放。
\zhu{便宜(便音“变”):方便合适。}
其中阴阳两宅俱已预备妥贴,\zhu{阴阳两宅:迷信说法,埋葬死人的墓地或寄放灵柩之处叫“阴宅”,活人居住的地方叫“阳宅”。
}
\jia{大凡创业之人,无有不为子孙深谋至细。
奈后辈仗一时之荣显,犹为不足,另生枝叶,虽华丽过先,奈不常保,亦足可叹,争及先人之常保其朴哉!近世浮华子弟齐来着眼。
}
好为送灵人口寄居。
\jia{祖宗为子孙之心细到如此!}\geng{《石头记》总于没要紧处闲三二笔,写正文筋骨。
看官当用巨眼,\zhu{巨眼:比喻善于鉴别的眼力。
}不为彼瞒过方好。
壬午季春。
}不想如今后辈人口繁盛,其中贫富不一,或性情参商,\zhu{性情参(参音“申”)商:性情各有不同,合不来的意思。
参商:“参”和“商”都是星宿名,属二十八宿。
因两星此出彼没,故常用来比喻两人分离不得见面。
参商也用以比喻人与人之间感情不和。
}\jia{所谓“源远水则浊,枝繁果则稀”。
余为天下痴心祖宗为子孙谋千年业者痛哭。
}有那家业艰难安分的,\jia{妙在艰难就安分,富贵则不安分矣。
}便住在这里了;有那尚排场有钱势的,只说这里不方便,一定另外或村庄或尼庵寻个下处,为事毕宴退之所。
\jia{真真辜负祖宗体贴子孙之心。
}即今秦氏之丧,族中诸人皆权在铁槛寺下榻,\zhu{槛:门槛,比喻生死界限。
唐代王梵志诗:“世无百年人,强作千年调;打铁作门限(槛),鬼见拍手笑。
”或云又喻富贵。
据《消夏闲记》:明万历间,凡家中有大厅者,即加门槛税,故俗称富贵人家叫“门槛人家”。
}独有凤姐嫌不方便,\jia{不用说,阿凤自然不肯将就一刻的。
}因而早遣人来和馒头庵的姑子净虚说了,\zhu{馒头:喻坟墓。
王梵志诗:“城外土馒头,馅食在城里;一人吃一个,莫嫌没滋味。
”又,宋代范成大《重九日行营寿藏之地》诗:“纵有千年铁门限,终须一个土馒头。
”}腾出两间房子来作下处。
\par
原来这馒头庵就是水月寺,因他庙里做的馒头好,就起了这个浑号,离铁槛寺不远。
\jia{前人诗云:“纵有千年铁门限,终须一个土馒头。
”是此意。
故“不远”二字有文章。
}当下和尚工课已完,奠过晚茶,贾珍便命贾蓉请凤姐歇息。
凤姐见还有几个妯娌陪着女亲,自己便辞了众人,带了宝玉、秦钟往水月庵来。
原来秦业年迈多病,\jia{伏一笔。
}不能在此,只命秦钟等待安灵罢了。
那秦钟便只跟着凤姐、宝玉,一时到了水月庵,净虚带领智善、智能两个徒弟出来迎接,大家见过。
凤姐等来至净室更衣净手毕,因见智能儿越发长高了,模样儿越发出息了,因说道:“你们师徒怎么这些日子也不往我们那里去?”净虚道:“可是,这几天都没工夫,因胡老爷府里产了公子,太太送了十两银子来这里,叫请几位师父念三日《血盆经》,\zhu{血盆经:佛经名,全称是《目连正教血盆经》,又叫《女人血盆经》。
旧时迷信,认为妇女产时出血不吉利,要请僧众念血盆经祈福消灾。
}忙的无个空儿,就无来请奶奶的安。
”\jia{虚陪一个胡姓,妙!言是糊涂人之所为也。
}\par
不言老尼陪着凤姐。
且说秦钟、宝玉二人正在殿上顽耍,因见智能过来,宝玉笑道:“能儿来了。
”秦钟道:“理那个东西作什么?”宝玉笑道:“你别弄鬼,那一日在老太太屋里,一个人没有,你搂着他作什么?这会子还哄我。
”\jia{补出前文未到处,细思秦钟近日在荣府所为可知矣。
}秦钟笑道:“这可是没有的话。
”\ping{秦钟和智能的地下见不得光的私情,在公开场合需要避嫌而刻意疏远。
}宝玉笑道:“有没有也不管你,你只叫住他倒碗茶来我吃,就丢开手。
”秦钟笑道:“这又奇了,你叫他倒去,还怕他不倒?何必要我说呢。
”宝玉道:“我叫他倒的是无情意的,不及你叫他倒的是有情意的。
”\jia{总作如是等奇语。
}秦钟只得说道:“能儿,倒碗茶来给我。
”那智能儿自幼在荣府走动,无人不识,因常与宝玉、秦钟顽笑。
他如今大了,渐知风月,便看上了秦钟人物风流,那秦钟也极爱他妍媚,二人虽未上手,却已情投意合了。
\jia{不爱宝玉,却爱秦钟,亦是各有情孽。
}今智能见了秦钟,心眼俱开,走去倒了茶来。
秦钟笑说:“给我。
”\jia{如闻其声。
}宝玉叫:“给我!”智能儿抿嘴笑道:“一碗茶也来争,我难道手里有蜜!”\jia{一语毕肖,如闻其语,观者已自酥倒,不知作者从何着想。
}宝玉先抢得了,吃着,方要问话,只见智善来叫智能去摆茶碟子,一时来请他两个去吃茶果点心。
他两个那里吃这些东西?坐一坐仍出来顽耍。
\par
凤姐也略坐片时,便回至净室歇息,老尼相送。
此时众婆娘媳妇见无事,皆陆续散了,自去歇息,跟前不过几个心腹常侍小婢,老尼便趁机说道:“我正有一事,要到府里求太太,先请奶奶一个示下。
”凤姐因问何事。
老尼道:“阿弥陀佛!\jia{开口称佛,毕肖。
可叹可笑!}只因当日我先在长安县内善才庵\jia{“才”字妙。
}内出家的时节,那时有个施主姓张,是大财主。
他有个女儿小名金哥,\jia{俱从“财”一字上发出。
}那年都往我庙里来进香,不想遇见了长安府府太爷的小舅子李衙内。
\zhu{衙内:唐代藩镇于所居州城之内又筑小城一重,作节度使的治所,前为节堂(办公的地方),后为私第,称做“牙城”。
并自募亲兵由牙内指挥使统帅,保护牙城。
五代及宋初,藩镇多用自己的子弟充当牙内指挥使,后遂称贵官之子弟为“衙内”。
衙:同“牙”。
}那李衙内一心看上,要娶金哥,打发人来求亲,不想金哥已受了原任守备的公子的聘礼。
\zhu{守备:明、清所置官名,掌管分守城堡或营务粮饷等事。
}张家若退亲,又怕守备不依,因此说有了人家。
谁知李公子执意不依,定要娶他女儿。
张家正无计策,两处为难。
不想守备家听了此信,也不管青红皂白,便来作践辱骂,说一个女儿许几家,偏不许退定礼,就要打官司告状起来。
\jia{守备一闻便\sout{问}[闹],
\zhu{
“问”字有“责问、追究”之义,但老尼说
“守备家听了此信,也不管青红皂白,便来作践辱骂”,
显然已超出“责问”的程度,兹依俞平伯辑评本校“问”为“闹”。
}
断无此理。
此不过张家惧府尹之势,必先退定礼,守备方不从,或有之。
此时老尼只欲与张家完事,故将此言遮饰,以便退亲,受张家之贿也。
}那张家急了,\jia{如何便急了,话无头绪,可知张家理缺。
此系作者巧摹老尼无头绪之语,莫认作者无头绪,正是神处奇处。
摹一人,一人必到纸上活现。
}只得着人上京来寻门路,赌气偏要退定礼。
\jia{如何?的是张家要与府尹攀亲!}我想如今长安节度云老爷与府上最契,可以求太太与老爷说声,打发一封书去,求云老爷和那守备说一声,不怕那守备不依。
若是肯行,张家连倾家孝敬,也都情愿。
”\jia{坏极,妙极!若与府尹攀了亲,何惜张财不能再得?小人之心如此,良民遭害如此!}\par
凤姐听了笑道:“这事倒不大,\jia{五字是阿凤心迹!}只是太太再不管这样的事。
”老尼道:“太太不管,奶奶也可以主张了。
”凤姐听说笑道:“我也不等银子使,也不作这样的事。
”\geng{口是心非,如闻已见。
}\ping{摆谱拿大,并且暗示索要好处费。
}
净虚听了,打去妄想,半晌叹\geng{一叹转出多少至恶不畏之文来。
}道:“虽如此说,张家已知我来求府里,如今不管这事,张家不知道没工夫管这事,不希罕他的谢礼,倒像府里连这点子手段也没有的一般。
”\geng{闺阁营谋说事,往往被此等语惑了。
}\ping{激将法。
}\par
凤姐听了这话,便发了兴头,说道:“你是素日知道我的,从来不信什么阴司地狱报应的,\geng{批书人深知卿有是心,叹叹!}凭是什么事,我说要行就行。
你叫他拿三千两银子来,我就替他出这口气。
”老尼听说,喜之不尽,\ping{凤姐讲自己不信因果报应,对面的出家人却喜之不尽,可见对于净虚来说,信仰只是用来打通关系,谋财嗜利的工具,她本身并不信。
凤姐刚料理完大事,体力疲惫却精神膨胀,此时心智最脆弱难守,才误入圈套。
}忙说:“有,有,有!这个不难。
”凤姐又道:“我比不得他们扯篷拉纤的图银子。
\geng{欺人太甚。
}这三千银子,不过是给打发说去的小厮作盘缠,使他赚几个辛苦钱,我一个钱也不要他的。
\geng{对如是之奸尼,阿凤不得不如是语。
}便是三万两,我此刻也拿的出来。
”\jia{阿凤欺人如此。
}老尼连忙答应,又说道:“既如此,奶奶明日就开恩也罢了。
”凤姐道:“你瞧瞧我忙的,那一处少了我?既应了你,自然快快的了结。
”老尼道:“这点子事,在别人跟前就忙的不知怎么样,若是奶奶跟前,再添上些也不够奶奶一发挥的。
\meng{“若是奶奶”等语,陷害杀无穷英明豪烈者。
誉而不喜,毁而不怒,或可逃此等术法。
}只是俗语说的‘能者多劳’,太太因大小事见奶奶妥贴,越性都推给奶奶了,奶奶也要保重金体才是。
”一路话奉承的凤姐越发受用了,也不顾劳乏,更攀谈起来。
\jia{总写阿凤聪明中的痴人。
}\ping{净虚干的事情既不净也不虚,既龌龊又势利。
}\par
\chai{fengjie}{凤姐弄权}
谁想秦钟趁黑无人,来寻智能。
刚到后面房中,只见智能独在房中洗茶碗,秦钟跑来便搂着亲嘴。
智能急的跺脚说:“这算什么呢!再这么我就叫唤了。
”秦钟求道:“好人,我已急死了。
你今儿再不依,我就死在这里。
”智能道:“你想怎样?除非等我出了这个牢坑,离了这些人,才依你。
”\ping{由此可见,出家人受苦地位低,惜春出家为尼,故入薄命司。
}秦钟道:“这也容易,只是远水救不得近渴。
”说着,一口吹了灯,满屋漆黑,将智能抱到炕上,就云雨起来。
\geng{此处写小小风流事,亦在人意外。
谁知为小秦伏线,大有根据。
}\geng{实表奸淫,尼庵之事如此。
壬午季春。
}那智能百般挣挫不起,又不好叫的,\geng{还是不肯叫。
}少不得依他了。
正在得趣,只见一人进来,将他二人按住,也不则声。
二人不知是谁,唬的不敢动一动。
只听那人嗤的一声,撑不住笑了,\geng{请掩卷细思此刻形景,真可喷饭。
历来风月文字可有如此趣味者?}二人听声,方知是宝玉。
秦钟连忙起身,抱怨道:“这算什么?”宝玉笑道:“你倒不依,咱们就叫喊起来。
”羞的智能趁黑地跑了。
\geng{若历写完,则不是《石头记》文字了,壬午季春。
}宝玉拉了秦钟出来道:“你可还和我强?”\meng{请问此等光景,是强是顺?一片儿女之态,自与凡常不同。
细极,妙极!}秦钟笑道:“好人,\geng{前以二字称智能,今又称玉兄,看官细思。
}你只别嚷的众人知道,你要怎么样我都依你。
”宝玉笑道:“这会子也不用说,等一会睡下,再细细的算帐。
”一时宽衣安歇的时节,凤姐在里间,秦钟、宝玉在外间,满地下皆是家下婆子,打铺坐更。
\zhu{坐更:夜间警卫。
}凤姐因怕通灵玉失落,便等宝玉睡下,命人拿来塞在自己枕边。
宝玉不知与秦钟算何帐目,未见真切,未曾记得,此系疑案,不敢纂创。
\jia{忽又作如此评断,似自相矛盾,却是最妙之文。
若不如此隐去,则又有何妙文可写哉?这方是世人意料不到之大奇笔。
若通部中万万件细微之事俱备,《石头记》真亦太觉死板矣。
故特用此二三件隐事,借石之未见真切,淡淡隐去,越觉得云烟渺茫之中,无限丘壑在焉。
}\ping{秦钟说:“你要怎么样我都依你”,宝玉说:“等一会睡下,再细细的算帐”。
作者对于当夜发生了什么欲言又止,可能两人云雨一番,但是考虑到“满地下皆是家下婆子,打铺坐更”,在众目睽睽之下两人出格的举动又不太可能。
}\par
一宿无话,至次日一早,便有贾母王夫人打发人来看宝玉,又命多穿两件衣服,无事宁可回去。
宝玉那里肯回去,又有秦钟恋着智能,调唆宝玉求凤姐再住一天。
凤姐想了一想:\jia{一想便有许多的好处。
真好阿凤!}凡丧仪大事虽妥,还有一半点小事未曾安插,可以指此再住一天,岂不又在贾珍跟前送了满情;二则又可以完净虚的那事;三则顺了宝玉的心,贾母听见,岂不欢喜?因有此三益,\jia{世人只云一举两得,独阿凤一举更添一。
}便向宝玉道:“我的事都完了,你要在这里逛,少不得越性辛苦一日罢了,明日可是定要走的了。
”\ping{凤姐卖人情,不说自己有事,只说自己是为了迁就宝玉才不得不再辛苦一日,多住一天。
}宝玉听说,千姐姐万姐姐的央求:“只住一日,明日必回去的。
”于是又住了一夜。
\par
凤姐便命悄悄将昨日老尼之事,说与来旺儿。
来旺儿心中俱已明白,急忙进城找着主文的相公,假托贾琏所嘱,修书一封,\jia{不细。
}连夜往长安县来,不过百里路程,两日工夫俱已妥协。
那节度使名唤云光,久欠贾府之情,这一点小事,岂有不允之理,给了回书,旺儿回来。
且不在话下。
\jia{一语过下。
}\par
却说凤姐等又过了一日,次日方别了老尼,着他三日后往府里去讨信。
\jia{过至下回。
}那秦钟与智能百般不忍分离,背地里多少幽期密约,俱不用细述,只得含泪而别。
凤姐又到铁槛寺中照望一番。
宝珠执意不肯回家,贾珍只得派妇女相伴。
后文再见。
\ping{在秦可卿葬礼上,秦钟智能逾越禁忌纵情宣淫,凤姐净虚逾越禁忌包揽诉讼。
}\par
\qi{总评:请看作者写势利之情,亦必因激动;写儿女之情,偏生含蓄不吐,可谓细针密缝。
其述说一段,言语形迹,无不逼真,圣手神文,敢不熏沐拜读?}
\dai{029}{王熙凤弄权铁槛寺}
\dai{030}{秦鲸卿得趣馒头庵}
\sun{p15-1}{宝凤同乘奔铁槛寺,秦鲸卿途遇贾宝玉}{图中部:出殡队伍出城,奔铁槛寺大路而来。
彼时长辈改乘轿前行,贾珍一辈的也将上马而行。
凤姐惦记着宝玉,怕他纵性闹出闪失,便令他弃马随她一同乘轿。
行至一处,凤姐得知这里有下处,下令歇歇再走,于是进了一庄户人家。
图右上:秦钟本随他父亲的轿子而行,忽见宝玉的小厮跑来请他去打尖,便策马赶来。
}
\sun{p15-2}{茅舍偶遇二丫头}{图左:宝玉见一间房内炕上有个纺车,越发以为稀奇。
听小厮说是纺线织布的,便上炕摇转作耍。
只见一个村庄丫头说道:“别动坏了!” 这才住手。
那丫头道:“你不会转,我转给你看。
”只见那丫头纺起线来,果然好看。
休息了一阵,便起身上车,来旺赏了那庄户人家。
走不多远,宝玉回头看见那丫头怀里抱着个小孩看着他们离去。
}
\sun{p15-3}{铁槛寺众僧迎灵柩}{大队人马终于来到铁槛寺,法鼓金铙,幢幡宝盖,好不热闹,只见寺中僧众早已排列路旁迎候。
到了寺中,另演佛事,重设香坛,安灵于内殿偏室后,做三日道场。
}
\sun{p15-4}{王熙凤弄权铁槛寺,秦鲸卿得趣馒头庵}{图右侧:凤姐辞别众人,带着宝玉、秦钟来到馒头庵。
静虚老尼带领智善、智能出来迎接,说些闲话。
图左下:当晚,老尼见凤姐身边只有几个心腹丫头,便趁机道明长安县张老爷家为女儿金哥婚事闹出官司一事,求凤姐帮助了断。
凤姐道:“我这人从来不信什么阴司地狱报应的。
你叫他拿三千两银子来,我就替他出这口气。
”老尼听了当下允诺。
随后凤姐命来旺连夜赶往长安县去找长安节度使云光。
图左上:秦钟趁夜黑无人找到智能,两人正在欢会之时被宝玉捉个正着。
}
