\chapter{闲取乐偶攒金庆寿 \quad 不了情暂撮土为香}
\qi{了与不了在心头,迷却原来难自由。
\zhu{原来:本源,最初,原本。
}如有如无谁解得,相生相灭第传流。
\zhu{第:只,仅仅。
}}\par
话说王夫人因见贾母那日在大观园不过着了些风寒,不是什么大病,请医生吃了两剂药也就好了,便放了心,因命凤姐来吩咐他预备给贾政带送东西。
正商议着,只见贾母打发人来请,王夫人忙引着凤姐儿过来。
王夫人又请问:“这会子可又觉大安些?”贾母道:“今日可大好了。
方才你们送来野鸡崽子汤,我尝了一尝,倒有味儿,又吃了两块肉,心里很受用。
”王夫人笑道:“这是凤丫头孝敬老太太的。
算他的孝心虔,不枉了素日老太太疼他。
”贾母点头笑道:“难为他想着。
若是还有生的,再炸上两块,咸浸浸的,吃粥有味儿。
那汤虽好,就只不对稀饭。
”凤姐听了,连忙答应,命人去厨房传话。
\par
这里贾母又向王夫人笑道:“我打发人请你来,不为别的。
初二是凤丫头的生日,上两年我原早想替他做生日,偏到跟前有大事,就混过去了。
今年人又齐全,料着又没事,咱们大家好生乐一日。
”\geng{贾母犹云“好生乐一日”,可见逐日虽乐,皆还不趁心也。
所以世人无论贫富,各有愁肠,终不能时时遂心如意。
此是至理,非不足语也。
}王夫人笑道:“我也想着呢。
既是老太太高兴,何不就商议定了?”贾母笑道:“我想往年不拘谁作生日,都是各自送各自的礼,这个也俗了,也觉生分的似的。
今儿我出个新法子,又不生分,又可取笑。
”\ping{人老了就变着花样的想要热闹。
}王夫人忙道:“老太太怎么想着好,就是怎么样行。
”贾母笑道:“我想着,咱们也学那小家子大家凑分子,\geng{原来凑分子是小家的事。
近见多少人家红白事一出,且筹算分子之多寡,不知何说。
}多少尽着这钱去办,你道好顽不好顽?”\geng{看他写与宝钗作生日后,又偏写与凤姐作生日。
阿凤何人也,岂不为彼之华诞大用一回笔墨哉?只是亏他如何想来,特写于宝钗之后,较姊妹胜而有馀;于贾母之前,较诸父母相去不远。
一部书中,若一个一个只管写过生日,复成何文哉?故起用宝钗,盛用阿凤,终用贾母,各有妙文,各有妙景。
馀者诸人,或一笔不写,或偶因一语带过,或丰或简,其情当理合,不表可知。
岂必谆谆死笔,按数而写众人之生日哉?}\geng{迥不犯宝钗。
}王夫人笑道:“这个很好,但不知怎么凑法?”贾母听说,益发高兴起来,忙遣人去请薛姨妈、邢夫人等,\meng{世家之长上多犯此等“办寿也要请人”毛病。
}
又叫请姑娘们并宝玉,那府里珍儿媳妇并赖大家的等有头脸管事的媳妇也都叫了来。
\zhu{有头脸:即“有头有脸”,此指有一定身分和体面的人。
}\par
众丫头婆子见贾母十分高兴,也都高兴,忙忙的各自分头去请的请,传的传,没顿饭的工夫,老的少的,上的下的,乌压压挤了一屋子。
只薛姨妈和贾母对坐,邢夫人王夫人只坐在房门前两张椅子上,宝钗姊妹等五六个人坐在炕上,宝玉坐在贾母怀前,地下满满的站了一地。
贾母忙命拿几个小杌子来,\zhu{杌(音“物”):小凳子。
}给赖大母亲等几个高年有体面的妈妈坐了。
贾府风俗,年高伏侍过父母的家人,比年轻的主子还有体面,所以尤氏凤姐儿等只管地下站着,那赖大的母亲等三四个老妈妈告个罪,
\zhu{告罪:谦词,用于向别人认错,表示歉意。}
都坐在小杌子上了。
\ping{贾政在三十三回说过:“自祖宗以来,皆是宽柔以待下人。
”这里贾母的行为验证了这一点。
但是书中也有被王夫人逼死的金钏和在凤姐和贾蓉的指示下被捆起来填马粪的焦大,这也是印证了贾政的话,老一辈是宽柔以待下人,但是年轻一代,已经不再如此。
地位高的人,只要给一些表面上的小恩小惠,就能收买人心。
}\par
贾母笑着把方才一席话说与众人听了。
众人谁不凑这趣儿?再也有和凤姐儿好的,有情愿这样的;有畏惧凤姐儿的,巴不得来奉承的:况且都是拿的出来的,所以一闻此言,都欣然应诺。
贾母先道:“我出二十两。
”薛姨妈笑道:“我随着老太太,也是二十两了。
”邢夫人王夫人笑道:“我们不敢和老太太并肩,自然矮一等,每人十六两罢了。
”尤氏李纨也笑道:“我们自然又矮一等,每人十二两罢。
”贾母忙和李纨道:“你寡妇失业的,那里还拉你出这个钱,我替你出了罢。
”\geng{必如是方妙。
}\ping{不管李纨到底有没有钱,她都必须接受贾母的好意,这确实是个寡妇人设问题,所以之前说她青春年代就如枯槁死灰一般,不论她本质是不是,社会的隐形规则都要求她必须是。
}凤姐忙笑道:“老太太别高兴,且算一算账再揽事。
老太太身上已有两分呢,这会子又替大嫂子出十二两,说着高兴,一会子回想又心疼了。
过后儿又说:‘都是为凤丫头花了钱。
’使个巧法子,哄着我拿出三四分子来暗里补上,我还做梦呢。
”说的众人都笑了。
贾母笑道:“依你怎么样呢?”\geng{又写阿凤一评,更妙。
若一笔直下,有何趣哉?}
凤姐笑道:“生日没到,我这会子已经折受的不受用了。
\zhu{折受:旧谓享受非份而折福叫“折受”。
这里用作谦词,是无福承受、于心不安的意思。
}我一个钱饶不出,惊动这些人实在不安,不如大嫂子这一分我替他出了罢了。
我到了那一日多吃些东西,就享了福了。
”邢夫人等听了,都说:“很是。
”贾母方允了。
\par
凤姐儿又笑道:“我还有一句话呢。
我想老祖宗自己二十两,又有林妹妹宝兄弟的两分子。
\ping{通过凤姐之口,可以看出,贾母把宝黛二人看作最亲近的心肝肉。
}姨妈自己二十两,又有宝妹妹的一分子,这倒也公道。
只是二位太太每位十六两,自己又少,又不替人出,这有些不公道。
老祖宗吃了亏了!”贾母听了,忙笑道:“倒是我的凤丫头向着我,这说的很是。
要不是你,我叫他们又哄了去了。
”凤姐笑道:“老祖宗只把他姐儿两个交给两位太太,一位占一个,派多派少,每位替出一分就是了。
”贾母忙说:“这很公道,就是这样。
”赖大的母亲忙站起来笑说道:“这可反了!我替二位太太生气。
在那边是儿子媳妇,\zhu{凤姐是邢夫人的儿媳妇。
}在这边是内侄女儿,\zhu{凤姐是王夫人的侄女。
}倒不向着婆婆姑娘,\zhu{
婆婆:指凤姐的婆婆邢夫人。
姑娘:姑姑,父亲的姐妹。指凤姐的姑姑王夫人。
}倒向着别人。
这儿媳妇成了陌路人,内侄女儿竟成了个外侄女儿了。
”\zhu{外侄女:相对于真正的侄女(即内侄女)来说,更加疏远一些的亲戚,可能是指配偶的侄女,或者本人的外甥女。
}说的贾母与众人都大笑起来了。
\geng{写阿凤全副精神,虽一戏,亦人想不到之文。
}\par
赖大之母因又问道:“少奶奶们十二两,我们自然也该矮一等了。
”贾母听说,道:“这使不得。
你们虽该矮一等,我知道你们这几个都是财主,果位虽低,\zhu{果位:修佛所达到的境界,这里指地位。
}钱却比他们多。
\geng{惊魂夺魄只此一句。
所以一部书全是老婆舌头,全是讽刺世事,反面春秋也。
所谓“痴子弟正照风月鉴”,若单看了家常老婆舌头,岂非痴子弟乎?}\ping{当富贵人家的奴仆,甚至比一般平民还要风光,后面会看到,赖嬷嬷家自己也有个园林,赖嬷嬷的孙子托贾家的关系,也当上了官。
}\ping{雍正废除贱民阶级,要不是成书在雍正即位后,可能赖嬷嬷后代还真不能当官。
}你们和他们一例才使得。
”众妈妈听了,连忙答应。
贾母又道:“姑娘们不过应个景儿,每人照一个月的月例就是了。
”又回头叫鸳鸯来,“你们也凑几个人,商议凑了来。
”鸳鸯答应着,去不多时带了平儿、袭人、彩霞等还有几个小丫鬟来,也有二两的,也有一两的。
贾母因问平儿:“你难道不替你主子作生日,还入在这里头?”平儿笑道:“我那个私自另外有了,这是官中的,
\zhu{官中:指大家庭所共有的。}
也该出一分。
”贾母笑道:“这才是好孩子。
”\par
凤姐又笑道:“上下都全了。
还有二位姨奶奶,\zhu{二位姨奶奶:赵姨娘和周姨娘。
}他出不出,也问一声儿,尽到他们是理。
不然,他们只当小看了他们了。
”\geng{纯写阿凤以衬后文。
}\ping{小丫鬟都出完了,才想到这两位姨娘,这地位和存在感也太低了。
是这二位太讨人嫌吗?不然丫鬟力争上游做了姨娘,这还不如小丫鬟了。
另外凤姐此处提起两个姨娘其实就是因为和赵姨娘有仇,周姨娘躺枪了……}贾母听了,忙说:“可是呢,怎么倒忘了他们!只怕他们不得闲儿,叫一个丫头问问去。
”说着,早有丫头去了,半日回来说道:“每位也出二两。
”
\ping{虽然是所谓的姨太太,可是份子钱基本和丫头一样。这个凑份子的名单,其实也是在讲这个家族中每个人地位的高低。}
贾母喜道:“拿笔砚来算明,共计多少。
”尤氏因悄骂凤姐道:“我把你这没足厌的小蹄子!这么些婆婆婶子来凑银子给你过生日,你还不足,又拉上两个苦瓠子作什么?”\zhu{
瓠:音“户”,葫芦的一种。
苦瓠子:喻“苦命人”。
}凤姐也悄笑道:“你少胡说,一会子离了这里,我才和你算账。
他们两个为什么苦呢?有了钱也是白填送别人,\ping{赵姨娘给钱给马道婆,让她施巫术害宝玉和凤姐。
}不如拘来咱们乐。
”\geng{纯写阿凤以衬后文,二人形景如见,语言如闻,真描画得到。
}\par
说着,早已合算了,共凑了一百五十两有馀。
贾母道:“一日戏酒用不了。
”尤氏道:“既不请客,酒席又不多,两三日的用度都够了。
头等,戏不用钱,省在这上头。
”贾母道:“凤丫头说那一班好,就传那一班。
”凤姐儿道:“咱们家的班子都听熟了,倒是花几个钱叫一班来听听罢。
”贾母道:“这件事我交给珍哥媳妇了。
越性叫凤丫头别操一点心,受用一日才算。
”\geng{所以特受用了,才有琏卿之变。
\zhu{下一回,在凤姐生日时贾琏偷情被抓。
}乐极生悲,自然之理。
}尤氏答应着。
又说了一回话,都知贾母乏了,才渐渐的都散出来。
\par
尤氏等送邢夫人王夫人二人散去,便往凤姐房里来商议怎么办生日的话。
凤姐儿道:“你不用问我,你只看老太太的眼色行事就完了。
”尤氏笑道:“你这阿物儿,
\zhu{阿物儿:如同说“东西”、“家伙”(指人),是一种轻蔑的口气。}
也忒行了大运了。
我当有什么事叫我们去,原来单为这个。
出了钱不算,还要我来操心,你怎么谢我?”凤姐笑道:“你别扯臊,\zhu{扯臊:胡扯。
}我又没叫你来,谢你什么!你怕操心?你这会子就回老太太去,再派一个就是了。
”尤氏笑道:“你瞧他兴的这样儿!\zhu{兴的:义近“宠的”。
}我劝你收着些儿好。
太满了就泼出来了。
”二人又说了一回方散。
\par
次日将银子送到宁国府来,尤氏方才起来梳洗,因问是谁送过来的,丫鬟们回说:“是林大娘。
”尤氏便命叫了他来。
丫鬟走至下房,叫了林之孝家的过来。
尤氏命他脚踏上坐了,一面忙着梳洗,一面问他:“这一包银子共多少?”林之孝家的回说:“这是我们底下人的银子,凑了先送过来。
老太太和太太们的还没有呢。
”正说着,丫鬟们回说:“那府里太太和姨太太打发人送分子来了。
”尤氏笑骂道:“小蹄子们,专会记得这些没要紧的话。
昨儿不过老太太一时高兴,故意的要学那小家子凑分子,你们就记得,到了你们嘴里当正经的说。
\meng{世家风调。
}还不快接了进来好生待茶,再打发他们去。
”丫鬟应着,忙接了进来,一共两封,连宝钗黛玉的都有了。
尤氏问还少谁的,林之孝家的道:“还少老太太、太太、姑娘们的和底下姑娘们的。
”尤氏道:“还有你们大奶奶的呢?”林之孝家的道:“奶奶过去,这银子都从二奶奶手里发,\meng{伏线。
}一共都有了。
”\par
说着,尤氏已梳洗了,命人伺候车辆。
一时来至荣府,先来见凤姐。
只见凤姐已将银子封好,正要送去。
尤氏问:“都齐了?”凤姐儿笑道:\geng{“笑”字就有神情。
}“都有了,快拿了去罢,丢了我不管。
”\meng{\sout{斗}[逗]起。
}
尤氏笑道:“我有些信不及,倒要当面点一点。
”说着果然按数一点,只没有李纨的一分。
\meng{点明题面。
}尤氏笑道:“我说你肏鬼呢,怎么你大嫂子的没有?”凤姐儿笑道:“那么些还不够使?短一分儿也罢了,等不够了我再给你。
”\geng{可见阿凤处处心机。
}尤氏道:“昨儿你在人跟前作人,今儿又来和我赖,这个断不依你。
我只和老太太要去。
”\ping{凤姐在贾母面前白赚好名声,不费一文钱。
}凤姐儿笑道:“我看你利害。
明儿有了事,我也‘丁是丁卯是卯’的,\zhu{丁是丁卯是卯:某个钉子一定要安在相应的铆处,不能有差错。
形容对事认真,毫不含糊。
另一种说法,丁为天干之一,卯为地支之一,有错就会影响农历推算。
又丁为物之凸出者,即榫头;卯为物之凹入者,即卯眼,二者若错就会安装不上。
表示做事认真、不马虎,含有不肯通融之意。
}你也别抱怨。
”尤氏笑道:“你一般的也怕。
\zhu{一般:一样。
}不看你素日孝敬我,我才是不依你呢。
”\meng{处处是世情作趣,处处是随笔埋伏。
}说着,把平儿的一分拿了出来,说道:“平儿,来!把你的收起去,等不够了,我替你添上。
”平儿会意,因说道:“奶奶先使着,若剩下了再赏我一样。
”尤氏笑道:“只许你那主子作弊,就不许我作情儿。
”\meng{请看。
}平儿只得收了。
尤氏又道:“我看着你主子这么细致,弄这些钱那里使去!使不了,明儿带了棺材里使去。
”\geng{此言不假,伏下后文短命。
尤氏亦能干事矣,惜不能劝夫治家,惜哉痛哉!}\par
一面说着,一面又往贾母处来。
先请了安,大概说了两句话,便走到鸳鸯房中和鸳鸯商议,只听鸳鸯的主意行事,何以讨贾母的喜欢。
二人计议妥当。
尤氏临走时,也把鸳鸯二两银子还他,\meng{请看世情。
可笑可笑!}说:“这还使不了呢。
”说着,一径出来,又至王夫人跟前说了一回话。
因王夫人进了佛堂,把彩云一分也还了他。
见凤姐不在跟前,一时把周、赵二人的也还了。
\meng{另是一番作用。
}他两个还不敢收。
\geng{阿凤声势亦甚矣。
}尤氏道:“你们可怜见的,那里有这些闲钱?凤丫头便知道了,有我应着呢。
”二人听说,千恩万谢的方收了。
\geng{尤氏亦可谓有才矣。
论有德比阿凤高十倍,惜乎不能谏夫治家,所谓“人各有当”也。
\zhu{当:这里应该是指担任的角色、所处的位置。
}此方是至理至情,最恨近之野史中,恶则无往不恶,美则无一不美,何不近情理之如是耶?}于是尤氏一径出来,坐车回家。
不在话下。
\ping{
电影《让子弹飞》中有台词:“县长上任,得巧立名目,拉拢豪绅,缴税捐款,他们交了,才能让百姓跟着交钱。得钱之后,豪绅的钱如数奉还,百姓的钱三七分成。”
}
\par
展眼已是九月初二日,园中人都打听得尤氏办得十分热闹,不但有戏,连耍百戏并说书的男女先儿全有,\zhu{男女先儿:男女盲艺人。
旧时习惯称算命和说书唱曲的盲艺人为“先儿”。
}\meng{剩笔,且影射能事不独熙凤。
}都打点取乐顽耍。
李纨又向众姊妹道:“今儿是正经社日,可别忘了。
\geng{看书者已忘,批书者亦已忘了,作者竟未忘,忽写此事,真忙中愈忙、紧处愈紧也。
}
宝玉也不来,想必他只图热闹,把清雅就丢开了。
”\geng{此独宝玉乎?亦骂世人。
余亦为宝玉忘了,不然何不来耶?}说着,便命丫鬟去瞧作什么,快请了来。
丫鬟去了半日,回说:“花大姐姐说,今儿一早就出门去了。
”\geng{奇文。
}众人听了,都诧异说:“再没有出门之理。
这丫头糊涂,不知说话。
”因又命翠墨去。
\par
一时翠墨回来说:“可不真出了门了。
说有个朋友死了,出去探丧去了。
”\geng{奇文。
信有之乎?花团锦簇之日偏如此写法。
}探春道:“断然没有的事。
凭他什么,再没今日出门之理。
你叫袭人来,我问他。
”刚说着,只见袭人走来。
李纨等都说道:“今儿凭他有什么事,也不该出门。
头一件,你二奶奶的生日,老太太都这等高兴,两府上下众人来凑热闹,他倒走了;\meng{因行文不肯平,下一反笔,则文语并奇,好看煞人。
}第二件,又是头一社的正日子,他也不告假,就私自去了!”袭人叹道:“昨儿晚上就说了,今儿一早起有要紧的事到北静王府里去,就赶回来的。
劝他不要去,他必不依。
今儿一早起来,又要素衣裳穿,想必是北静王府里的要紧姬妾没了,也未可知。
”李纨等道:“若果如此,也该去走走,只是也该回来了。
”说着,大家又商议:“咱们只管作诗,等他回来罚他。
”刚说着,只见贾母已打发人来请,便都往前头来了。
袭人回明宝玉的事,贾母不乐,便命人去接。
\par
原来宝玉心里有件私事,于头一日就吩咐茗烟:“明日一早要出门,备下两匹马在后门口等着,不要别一个跟着。
说给李贵,我往北府里去了。
倘或要有人找我,叫他拦住不用找,只说北府里留下了,横竖就来的。
”茗烟也摸不着头脑,只得依言说了。
今儿一早,果然备了两匹马在园后门等着。
天亮了,只见宝玉遍体纯素,从角门出来,一语不发跨上马,一弯腰,顺着街就颠下去了。
\zhu{颠:溜跑的意思。
}茗烟也只得跨马加鞭赶上,在后面忙问:“往那里去?”宝玉道:“这条路是往那里去的?”茗烟道:“这是出北门的大道。
出去了冷清清没有可顽的。
”宝玉听说,点头道:“正要冷清清的地方好。
”说着,越性加了鞭,那马早已转了两个弯子,出了城门。
茗烟越发不得主意,只得紧紧跟着。
\par
一气跑了七八里路出来,人烟渐渐稀少,宝玉方勒住马,回头问茗烟道:“这里可有卖香的?”茗烟道:“香倒有,不知是那一样?”宝玉想道:“别的香不好,须得檀、芸、降三样。
”\zhu{檀、芸、降:三种较为名贵的香。
檀:檀香,以檀香木制成。
芸:芸香,以芸香草制成。
降:降香,以降香木制成。
}茗烟笑道:“这三样可难得。
”宝玉为难。
茗烟见他为难,因问道:“要香作什么使?我见二爷时常小荷包有散香,何不找一找。
”一句提醒了宝玉,便回手向衣襟上拉出一个荷包来,摸了一摸,竟有两星沉速,\zhu{两星沉速:星:量词,小颗、小块。
沉速:沉香和速香。
这里是指两小块以沉香和速香合成的香料。
}心内欢喜:“只是不恭些。
”再想自己亲身带的,倒比买的又好些。
于是又问炉炭。
茗烟道:“这可罢了。
荒郊野外那里有?用这些何不早说,带了来岂不便宜。
”宝玉道:“糊涂东西,若可带了来,又不这样没命的跑了。
”\geng{奇奇怪怪不知为何,看他下文怎样。
}\par
茗烟想了半日,笑道:“我得了个主意,不知二爷心下如何?我想二爷不只用这个呢,只怕还要用别的。
这也不是事。
如今我们往前再走二里地,就是水仙庵了。
”宝玉听了忙问:“水仙庵就在这里?更好了,我们就去。
”说着,就加鞭前行,一面回头向茗烟道:“这水仙庵的姑子长往咱们家去,咱们这一去到那里,和他借香炉使使,他自然是肯的。
”茗烟道:“别说他是咱们家的香火,就是平白不认识的庙里,和他借,他也不敢驳回。
只是一件,我常见二爷最厌这水仙庵的,如何今儿又这样喜欢了?”宝玉道:“我素日因恨俗人不知原故,混供神混盖庙,这都是当日有钱的老公们和那些有钱的愚妇们听见有个神,就盖起庙来供着,也不知那神是何人,因听些野史小说,便信真了。
\geng{近闻刚丙庙,又有三教庵,以如来为尊,太上为次,先师为末,真杀有馀辜,所谓此书救世之溺不假。
}比如这水仙庵里面因供的是洛神,\zhu{洛神:三国魏曹植(字子建)曾作《洛神赋》,叙述他和想象中的洛水女神相会之事。
下文“翩若惊鸿”等都是《洛神赋》的句子。
}故名水仙庵,殊不知古来并没有个洛神,那原是曹子建的谎话,谁知这起愚人就塑了像供着。
今儿却合我的心事,故借他一用。
”\ping{第三十九回,刘姥姥胡诌的茗玉小姐死后成精幻化人形,贾宝玉信以为真,还要给她修盖庙。
按照宝玉自己的说法,宝玉自己也是信了刘姥姥的谎话,是一个愚人。
}\par
说着早已来至门前。
那老姑子见宝玉来了,事出意外,竟像天上掉下个活龙来的一般,忙上来问好,命老道来接马。
宝玉进去,也不拜洛神之像,却只管赏鉴。
虽是泥塑的,却真有“翩若惊鸿,婉若游龙”之态,“荷出绿波,日映朝霞”之姿。
\geng{妙极!用《洛神赋》赞洛神,本地风光,愈觉新奇。
}宝玉不觉滴下泪来。
老姑子献了茶。
宝玉因和他借香炉。
那姑子去了半日,连香供纸马都预备了来。
\zhu{香供:香和供品。
纸马:旧俗祭祀时供焚化的纸糊的人、车、马等造型,也指供焚化的印有神像的纸片。
}宝玉道:“一概不用。
”说着,命茗烟捧着炉出至后园中,拣一块干净地方儿,竟拣不出。
茗烟道:“那井台儿上如何?”宝玉点头,一齐来至井台上,将炉放下。
\geng{妙极之文。
宝玉心中拣定是井台上了,故意使茗烟说出,使彼不犯疑猜矣。
宝玉亦有欺人之才,盖不用耳。
}\par
茗烟站过一旁。
宝玉掏出香来焚上,含泪施了半礼,
\zhu{半礼:(身分地位高的对身分地位低的人行礼或答礼时)比较简单的仪式。}
\geng{奇文。
只云“施半礼”,终不知为何事也。
}回身命收了去。
茗烟答应,且不收,忙爬下磕了几个头,口内祝道:“我茗烟跟二爷这几年,二爷的心事,我没有不知道的,只有今儿这一祭祀没有告诉我,我也不敢问。
只是这受祭的阴魂虽不知名姓,想来自然是那人间有一、天上无双,极聪明极俊雅的一位姐姐妹妹了。
二爷心事不能出口,让我代祝:若芳魂有感,香魄多情,虽然阴阳间隔,既是知己之间,时常来望候二爷,未尝不可。
你在阴间保佑二爷来生也变个女孩儿,和你们一处相伴,再不可又托生这须眉浊物了。
”\ping{真是痴人,忽然觉得宝玉和不爱江山爱美人的英王爱德华八世、艺术家皇帝宋徽宗一样,就是阴差阳错,身处不得不得承担责任的位置,却不愿或者不能承担起那种重量。
}说毕,又磕几个头,才爬起来。
\geng{忽插入茗烟一篇流言,粗看则小儿戏语,亦甚无味。
细玩则大有深意,试思宝玉之为人岂不应有一极伶俐乖巧小童哉?此一祝亦如《西厢记》中双文降香,\zhu{双文:即《西厢记》里的崔莺莺,因莺莺的名字是用两个“莺”字叠成。
降香:泛指烧香朝拜。
}第三炷则不语,红娘则代祝数语,直将双文心事道破。
此处若写宝玉一祝,则成何文字?若不祝则成一哑迷,如何散场?故写茗烟一戏,直戏入宝玉心中,又发出前文,又可收后文,又写茗烟素日之乖觉可人,且衬出宝玉直似一个守礼待嫁的女儿一般,其素日脂香粉气不待写而全现出矣。
今看此回,直欲将宝玉当作一个极清俊羞怯的女儿,看茗烟则极乖觉可人之丫鬟也。
}\par
宝玉听他没说完,便撑不住笑了,\geng{方一笑,盖原可发笑,且说得合心,愈见可笑也。
}因踢他道:“休胡说,看人听见笑话。
”\geng{也知人笑,更奇。
}茗烟起来收过香炉,和宝玉走着,因道:“我已经和姑子说了,二爷还没用饭,叫他随便收拾了些东西,二爷勉强吃些。
我知道今儿咱们里头大排筵宴,热闹非常,二爷为此才躲了出来的。
横竖在这里清净一天,也就尽到礼了。
若不吃东西,断使不得。
”宝玉道:“戏酒既不吃,这随便素的吃些何妨。
”茗烟道:“这便才是。
还有一说,咱们来了,还有人不放心。
若没有人不放心,便晚了进城何妨?若有人不放心,二爷须得进城回家去才是。
第一老太太、太太也放了心,第二礼也尽了,不过如此。
就是家去了看戏吃酒,也并不是二爷有意,原不过陪着父母尽孝道。
二爷若单为了这个不顾老太太、太太悬心,就是方才那受祭的阴魂也不安生。
二爷想我这话如何?”宝玉笑道:“你的意思我猜着了,你想着只你一个跟了我出来,回来你怕担不是,所以拿这大题目来劝我。
\geng{亦知这个大,妙极!}我才来了,不过为尽个礼,再去吃酒看戏,并没说一日不进城。
这已完了心愿,赶着进城,大家放心,岂不两尽其道。
”\geng{这是大通的意见,\zhu{大通:通达事物之理。
}世人不及的去处。
}茗烟道:“这更好了。
”说着二人来至禅堂,果然那姑子收拾了一桌素菜,宝玉胡乱吃了些,茗烟也吃了。
\par
二人便上马仍回旧路。
茗烟在后面只嘱咐:“二爷好生骑着,这马总没大骑的,\zhu{大:表示程度深,这里是指骑得时间长。
}手里提紧着。
”\geng{看他偏不写凤姐那样热闹,却写这般清冷,真世人意料不到这一篇文字也。
}一面说着,早已进了城,仍从后门进去,忙忙来至怡红院中。
袭人等都不在房里,只有几个老婆子看屋子,见他来了,都喜的眉开眼笑,说:“阿弥陀佛,可来了!把花姑娘急疯了!上头正坐席呢,二爷快去罢。
”宝玉听说忙将素服脱了,自去寻了华服换上,问在什么地方坐席,老婆子回说在新盖的大花厅上。
\zhu{花厅:我国古建筑中供饮宴、观剧、会客等用的内厅,因其有别于正厅,大多建于园中,或另辟跨院建造(跨院:正院两旁的院子),四周湖石点缀,种植花木,富有园林气息,俗统称为花厅。
}\par
宝玉听说,一径往花厅来,耳内早已隐隐闻得歌管之声。
刚至穿堂那边,只见玉钏儿独坐在廊檐下垂泪,\geng{总是千奇百怪的文字。
}一见他来,便收泪说道:“凤凰来了,快进去罢。
再一会子不来,都反了。
”\geng{是平常言语,却是无限文章,无限情理。
看至后文,再细思此言,则可知矣。
}宝玉陪笑道:“你猜我往那里去了?”玉钏儿不答,只管擦泪。
\geng{无限情理。
}\ping{可能是金钏跳井自杀一周年,在凤姐生日这样热闹喜庆的时候,只有心怀愧疚的宝玉和一母同胞的玉钏还记着那个早逝的生命。
}\ping{刘姥姥说茗玉小姐十七岁早死成精幻化人形,宝玉非要给茗玉小姐盖庙,这可能也让宝玉想起了因为自己而死的金钏,也是一个早死的苦命人,通过给茗玉小姐盖庙,也是传达了宝玉对于金钏的忏悔自责。
}宝玉忙进厅里,见了贾母王夫人等,众人真如得了凤凰一般。
宝玉忙赶着与凤姐儿行礼。
\ping{
与凤姐的“行礼”跟刚才与金钏的“含泪施半礼”形成对比,人世间有很多的礼,有些是你心里的怀念,有些则是排场上的客套。
}
贾母王夫人都说他不知道好歹,“怎么也不说声就私自跑了,这还了得!明儿再这样,等老爷回家来,必告诉他打你。
”说着又骂跟的小厮们都偏听他的话,说那里去就去,也不回一声儿。
一面又问他到底那去了,可吃了什么,可唬着了。
\geng{奇文,毕肖。
}宝玉只回说:“北静王的一个爱妾昨日没了,给他道恼去。
\zhu{道恼:也作“道烦恼”,向遭丧遇祸的人家慰问。
}他哭的那样,不好撇下就回来,所以多等了一会子。
”贾母道:“以后再私自出门,不先告诉我们,一定叫你老子打你。
”宝玉答应着。
因又要打跟的小子们,众人又忙说情,又劝道:“老太太也不必过虑了,他已经回来,大家该放心乐一回了。
”贾母先不放心,自然发狠,如今见他来了,喜且有馀,那里还恨,也就不提了;还怕他不受用,或者别处没吃饱,路上着了惊怕,反百般的哄他。
袭人早过来伏侍。
大家仍旧看戏。
当日演的是《荆钗记》。
\zhu{《荆钗记》:南戏剧本,描写王十朋和钱玉莲悲欢离合的故事。
}贾母薛姨妈等都看的心酸落泪,也有叹的,也有骂的。
要知端的,下回分解。
\par
\qi{总评:攒金办寿家常乐,素服焚香无限情。
\hang
写办事不独熙凤,写多情不漏亡人,情之所钟必让若辈。
\zhu{若:你,你们。
}此所谓“情情”者也。
\zhu{“情情”:指宝玉。}
}
\dai{085}{凑份子庆凤姐寿辰,尤氏退还平儿银子}
\dai{086}{宝玉祭奠金钏,茗烟代为祝祷}
\sun{p43-1}{不了情暂撮土为香}{宝玉带着茗烟骑马一气跑了七八里路到水仙庵,宝玉进去,也不拜洛神之像,却只管赏鉴。
虽是泥塑的,却真有“翩若惊鸿,婉若游龙”之态,“荷出绿波,日映朝霞”之姿。
然后含泪祭奠金钏。
}