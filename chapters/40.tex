\chapter{史太君两宴大观园 \quad 金鸳鸯三宣牙牌令}
\zhu{牙牌令:用牙牌作为酒令。
酒令:古代宴会中,佐饮助兴的游戏。
推一人为令官,其余的人听其号令,轮流说诗词或做其他游戏,违令或输的人饮酒。
牙牌:也作“骨牌”,一种用象牙或骨、角等制成的玩具。
共三十二张,作长方形,面雕二到十二个圆点。
}
\par
\qi{两宴不觉已深秋,惜春只如画春游。
\zhu{
惜春画的“大观园‘行乐’图”中的背景时间是秋季,可脂批却说“如春游”,有意在比附春季的游乐。
康熙的第二次“南巡”从康熙二十八年一月出发,三月到南京,从时间上来说,是一次“春游”。
一说“惜春只如画春游”应为“惜春只知画春游”,贾府众人“只知”享乐宴游,“不觉”家道也随秋风萧瑟,渐渐凋零。
}
可怜富贵谁能保,只有恩情得到头。
\zhu{恩情:刘姥姥在贾家败落后不忘贾府旧恩,救助落难的巧姐。}
}\par
话说宝玉听了,忙进来看时,只见琥珀站在屏风跟前说:“快去吧,立等你说话呢。
”宝玉来至上房,只见贾母正和王夫人、众姊妹商议给史湘云还席。
宝玉因说道:“我有个主意。
既没有外客,吃的东西也别定了样数,谁素日爱吃的拣样儿做几样。
也不要按桌席,每人跟前摆一张高几,各人爱吃的东西一两样,再一个什锦攒心盒子,\zhu{
什锦:由多种原料制成或多种花样拼成的。
攒心,即向心。
什锦攒心盒子:一种盛果、菜的食盒,中心一圆格,四周以扇形盘格向中心聚拢。
又称“攒盒”。
}自斟壶,岂不别致。
”贾母听了,说“很是”,忙命传与厨房:“明日就拣我们爱吃的东西作了,按着人数,再装了盒子来。
早饭也摆在园里吃。
”商议之间早又掌灯,一夕无话。
\par
次日清早起来,可喜这日天气清朗。
李纨侵晨先起,\zhu{侵晨:天色渐亮时。
}看着老婆子丫头们扫那些落叶,\ji{是八月尽。
}并擦抹桌椅,预备茶酒器皿。
只见丰儿带了刘姥姥板儿进来,说“大奶奶倒忙的紧。
”李纨笑道:“我说你昨儿去不成,只忙着要去。
”刘姥姥笑道:“老太太留下我,叫我也热闹一天去。
”丰儿拿了几把大小钥匙,说道:“我们奶奶说了,外头的高几恐不够使,不如开了楼把那收着的拿下来使一天罢。
奶奶原该亲自来的,因和太太说话呢,请大奶奶开了,带着人搬罢。
”李氏便令素云接了钥匙,又令婆子出去把二门上的小厮叫几个来。
李氏站在大观楼下往上看,令人上去开了缀锦阁,一张一张往下抬。
小厮、老婆子、丫头一齐动手,抬了二十多张下来。
李纨道:“好生着,别慌慌张张鬼赶来似的,仔细碰了牙子。
”\zhu{牙子:这里指镶在几面或凳面边沿的雕花装饰。
}又回头向刘姥姥笑道:“姥姥,你也上去瞧瞧。
”刘姥姥听说,巴不得一声儿,便拉了板儿登梯上去进里面,只见乌压压的堆着些围屏、桌椅、大小花灯之类,虽不大认得,只见五彩炫耀,各有奇妙。
念了几声佛,便下来了。
然后锁上门,一齐才下来。
李纨道:“恐怕老太太高兴,越性把舡上划子、\zhu{舡:同“船”。
}篙桨、\zhu{篙:音“高”,撑船的竹竿或木杆。
}遮阳幔子都搬了下来预备着。
”众人答应,复又开了,色色的搬了下来。
\zhu{色色:样样。
}令小厮传驾娘们到舡坞里撑出两只船来。
\par
正乱着安排,只见贾母已带了一群人进来了。
李纨忙迎上去,笑道:“老太太高兴,倒进来了。
我只当还没梳头呢,才撷了菊花要送去。
”一面说,一面碧月早捧过一个大荷叶式的翡翠盘子来,里面盛着各色的折枝菊花。
贾母便拣了一朵大红的簪于鬓上。
因回头看见了刘姥姥,忙笑道:“过来带花儿。
”一语未完,凤姐便拉过刘姥姥,笑道:“让我打扮你。
”说着,将一盘子花横三竖四的插了一头。
贾母和众人笑的了不得。
刘姥姥笑道:“我这头也不知修了什么福,今儿这样体面起来。
”众人笑道:“你还不拔下来摔到他脸上呢,把你打扮的成了个老妖精了。
”刘姥姥笑道:“我虽老了,年轻时也风流,爱个花儿粉儿的,今儿老风流才好。
”\par
说笑之间,已来至沁芳亭子上。
丫鬟们抱了一个大锦褥子来,铺在栏杆榻板上。
\zhu{
榻:狭长的矮床;泛指床。
榻板:固定在回廊栏杆下部供人小憩的榻形长板。
}贾母倚柱坐下,命刘姥姥也坐在旁边,因问他:“这园子好不好?”刘姥姥念佛说道:“我们乡下人到了年下,都上城来买画儿贴。
时常闲了,大家都说,怎么得也到画儿上去逛逛。
想着那个画儿也不过是假的,那里有这个真地方呢。
谁知我今儿进这园里一瞧,竟比那画儿还强十倍。
怎么得有人也照着这个园子画一张,我带了家去,给他们见见,死了也得好处。
”贾母听说,便指着惜春笑道:“你瞧我这个小孙女儿,他就会画。
等明儿叫他画一张如何?”刘姥姥听了,喜的忙跑过来,拉着惜春说道:“我的姑娘,你这么大年纪儿,又这么个好模样,还有这个能干,别是神仙托生的罢。
”\par
贾母少歇一回,自然领着刘姥姥都见识见识。
先到了潇湘馆。
一进门,只见两边翠竹夹路,土地下苍苔布满,中间羊肠一条石子漫的路。
刘姥姥让出路来与贾母众人走,自己却赾走土地。
\zhu{赾(音“寝”)走:小心地行走。
}琥珀拉着他说道:“姥姥,你上来走,仔细苍苔滑了。
”刘姥姥道:“不相干的,我们走熟了的,姑娘们只管走罢。
可惜你们的那绣鞋,别沾脏了。
”他只顾上头和人说话,不防底下果踩滑了,咕咚一跤跌倒。
众人拍手都哈哈的笑起来。
贾母笑骂道:“小蹄子们,还不搀起来,只站着笑。
”说话时,刘姥姥已爬了起来,自己也笑了,说道:“才说嘴就打了嘴。
”贾母问他:“可扭了腰了不曾?叫丫头们捶一捶。
”刘姥姥道:“那里说的我这么娇嫩了。
那一天不跌两下子,都要捶起来,还了得呢。
”\ping{乡村民妇虽然生活条件艰苦,确是生机勃勃。
}\par
紫鹃早打起湘帘,贾母等进来坐下。
林黛玉亲自用小茶盘捧了一盖碗茶来奉与贾母。
\zhu{
盖碗:一种上有盖、下有托,中有碗的汉族茶具。
又称“三才碗”、“三才杯”,盖为天、托为地、碗为人,暗含天地人和之意。
}
王夫人道:“我们不吃茶,姑娘不用倒了。
”\ping{女子受聘,俗谓“吃茶”。
《七修类稿》:种茶下子,不可移植,移植则不复生,故以喻女子受聘。
第二十五回,凤姐笑道:“倒求你,你倒说这些闲话。
你既吃了我们家的茶,怎么还不给我们家作媳妇?”王夫人通过不吃茶,可能含蓄地表明了自己对于宝黛婚姻的态度,王夫人为首的“我们”和贾母意见相左。
也有可能这里的“我们”包括贾母,贾母要吃茶,而王夫人不让吃茶,婆媳在吃茶这个问题上意见相左,也暗示了在对待林黛玉的问题上意见相左。
值得注意的是,本回后文众人来到秋爽斋,“丫鬟端过两盘茶来,大家吃毕”。由此可见王夫人在这里的不吃茶,是有意为之。
}林黛玉听说,便命丫头把自己窗下常坐的一张椅子挪到下首,\zhu{下首:同“下手”,位置较卑的一侧,就室内说,一般指靠外的或者靠右的(左右以人在室内而脸朝外时为准)。
}请王夫人坐了。
刘姥姥因见窗下案上设着笔砚,又见书架上磊着满满的书,\zhu{磊:音“洛”,叠,摞。
}刘姥姥道:“这必定是那位哥儿的书房了。
”贾母笑指黛玉道:“这是我这外孙女儿的屋子。
”刘姥姥留神打量了黛玉一番,方笑道:“这那像个小姐的绣房,竟比那上等的书房还好。
”贾母因问:“宝玉怎么不见?”众丫头们答说:“在池子里舡上呢。
”贾母道:“谁又预备下舡了?”李纨忙回说:“才开楼拿几,我恐怕老太太高兴,就预备下了。
”贾母听了方欲说话时,有人回说:“姨太太来了。
”贾母等刚站起来,只见薛姨妈早进来了,一面归坐,笑道:“今儿老太太高兴,这早晚就来了。
”贾母笑道:“我才说来迟了的要罚他,不想姨太太就来迟了。
”\par
说笑一会,贾母因见窗上纱的颜色旧了,便和王夫人说道:“这个纱新糊上好看,过了后来就不翠了。
这个院子里头又没有个桃杏树,这竹子已是绿的,再拿这绿纱糊上反不配。
我记得咱们先有四五样颜色糊窗的纱呢,明儿给他把这窗上的换了。
”凤姐儿忙道:“昨儿我开库房,看见大板箱里还有好些匹银红蝉翼纱,
\zhu{
银红:把银朱加在粉红色的颜料里调和而成的颜色。
银朱:硫化汞的通称。无机化合物,鲜红色,粉末状,有毒。用作颜料和药品等。
}
也有各样折枝花样的,
\zhu{
折枝:花卉画的一种。画花卉不写全株,只选择其中一枝或若干小枝入画,故名。
}
也有流云万福花样的,\zhu{福:也作“蝠”,蝙蝠图案。
“蝠”与“福”谐音,取吉祥多福之意。
}也有百蝶穿花花样的,颜色又鲜,纱又轻软,我竟没见过这样的。
拿了两匹出来,作两床绵纱被,想来一定是好的。
”贾母听了笑道:“呸,人人都说你没有不经过不见过,连这个纱还不认得呢,明儿还说嘴。
”薛姨妈等都笑说:“凭他怎么经过见过,如何敢比老太太呢。
老太太何不教导了他,我们也听听。
”凤姐儿也笑说:“好祖宗,教给我罢。
”\par
贾母笑向薛姨妈众人道:“那个纱,比你们的年纪还大呢。
怪不得他认作蝉翼纱,原也有些像,不知道的,都认作蝉翼纱。
正经名字叫作‘软烟罗’。
”凤姐儿道:“这个名儿也好听。
只是我这么大了,纱罗也见过几百样,从没听见过这个名色。
”贾母笑道:“你能够活了多大,见过几样没处放的东西,就说嘴来了。
那个软烟罗只有四样颜色:一样雨过天晴,一样秋香色,一样松绿的,一样就是银红的。
若是做了帐子,糊了窗屉,\zhu{窗屉:装置于窗上,可支起或放落的木架,上糊以纱或纸。
}远远的看着,就似烟雾一样,所以叫作‘软烟罗’,那银红的又叫作‘霞影纱’。
如今上用的府纱也没有这样软厚轻密的了。
”
\zhu{府纱:指江宁府所织之纱。江宁即今南京,为清时江南织造所在地。}
\par
薛姨妈笑道:“别说凤丫头没见,连我也没听见过。
”凤姐儿一面说,早命人取了一匹来了。
贾母说:“可不是这个!先时原不过是糊窗屉,后来我们拿这个作被作帐子,试试也竟好。
明儿就找出几匹来,拿银红的替他糊窗子。
”凤姐答应着。
众人都看了,称赞不已。
刘姥姥也觑着眼看个不了,\zhu{觑:音“去”,眯着眼注视。
}念佛说道:“我们想他作衣裳也不能,拿着糊窗子,岂不可惜?”贾母道:“倒是做衣裳不好看。
”凤姐忙把自己身上穿的一件大红绵纱袄子襟儿拉了出来,向贾母薛姨妈道:“看我的这袄儿。
”贾母薛姨妈都说:“这也是上好的了,这是如今的上用内造的,竟比不上这个。
”凤姐儿道:“这个薄片子,还说是上用内造呢,竟连官用的也比不上了。
”\ping{昔盛今衰,追忆往昔。
}\ping{贾母拿出箱底珍藏的、比皇宫中用的还好的纱给黛玉做窗纱,可见贾母心中黛玉的地位之高。
}贾母道:“再找一找,只怕还有青的。
若有时都拿出来,送这刘亲家两匹,做一个帐子我挂,下剩的添上里子,做些夹背心子给丫头们穿,白收着霉坏了。
”凤姐忙答应了,仍令人送去。
\ping{换窗纱事件中,一开始,贾母明明知道实际管家的不是王夫人,而是王熙凤,本来应该直接吩咐王熙凤,但是却让王夫人给林黛玉换窗纱,可能暗含对王夫人不关心黛玉的不满。
但是贾母后面再也没和王夫人说话,王夫人也没有再说话和行动。
后面的主角是薛姨妈和王熙凤,王熙凤作为管家人积极执行贾母的吩咐。
在宝玉婚姻问题上的两派,在这里也显露端倪:王夫人自成一派,贾母心有不满,凤姐积极支持贾母。
}\par
贾母起身笑道:“这屋里窄,再往别处逛去。
”刘姥姥念佛道:“人人都说大家子住大房。
昨儿见了老太太正房,配上大箱大柜大桌子大床,果然威武。
那柜子比我们那一间房子还大还高。
怪道后院子里有个梯子。
我想并不上房晒东西,预备个梯子作什么?后来我想起来,定是为开顶柜收放东西,非离了那梯子,怎么得上去呢。
如今又见了这小屋子,更比大的越发齐整了。
满屋里的东西都只好看,都不知叫什么,我越看越舍不得离了这里。
”凤姐道:“还有好的呢,我都带你去瞧瞧。
”\par
说着,一径离了潇湘馆,远远望见池中一群人在那里撑舡。
贾母道:“他们既预备下船,咱们就坐。
”一面说着,便向紫菱洲、蓼溆一带走来。
未至池前,只见几个婆子手里都捧着一色捏丝戗金五彩大盒子走来。
\zhu{
捏丝:把金丝捏成花样或图案,然后镶嵌在器物上作为装饰的一种工艺。
戗金[qiàng]:漆器工艺之一,在深色漆地上,缕划出纤细的花纹沟槽,槽内涂胶,上粘金箔,呈现金色花纹。
}
凤姐忙问王夫人早饭在那里摆。
王夫人道:“问老太太在那里,就在那里罢了。
”贾母听说,便回头说:“你三妹妹那里就好。
你就带了人摆去,我们从这里坐了舡去。
”\par
凤姐听说,便回身同了探春、李纨、鸳鸯、琥珀带着端饭的人等,抄着近路到了秋爽斋,就在晓翠堂上调开桌案。
鸳鸯笑道:“天天咱们说外头老爷们吃酒吃饭都有一个篾片相公,\zhu{篾(音“灭”)片:旧时依附于富贵人家,为主子帮闲凑趣的人叫“篾片”。
}拿他取笑儿。
咱们今儿也得了一个女篾片了。
”李纨是个厚道人,听了不解。
凤姐儿却知是说的是刘姥姥了,也笑说道:“咱们今儿就拿他取个笑儿。
”二人便如此这般的商议。
李纨笑劝道:“你们一点好事也不做,又不是个小孩儿,还这么淘气,仔细老太太说。
”鸳鸯笑道:“很不与你相干,有我呢。
”\par
正说着,只见贾母等来了,各自随便坐下。
先着丫鬟端过两盘茶来,大家吃毕。
凤姐手里拿着西洋布手巾,裹着一把乌木三镶银箸,\zhu{乌木三镶银箸:在乌木筷子的下截、中腰及顶端包镶银箔,故称“三镶”。
银能试毒,银箸是贵重餐具。
}敁敠人位,\zhu{敁敠:音“颠多”,也写作“掂掇”,估量、盘算、斟酌的意思。
}按席摆下。
贾母因说:“把那一张小楠木桌子抬过来,让刘亲家近我这边坐着。
”众人听说,忙抬了过来。
凤姐一面递眼色与鸳鸯,鸳鸯便拉了刘姥姥出去,悄悄的嘱咐了刘姥姥一席话,又说:“这是我们家的规矩,若错了我们就笑话呢。
”调停已毕,然后归坐。
\zhu{调停:照料,安排(多见于早期白话)。}
薛姨妈是吃过饭来的,不吃,只坐在一边吃茶。
\ji{妙!若只管写薛姨妈来则吃饭,则成何文理?}贾母带着宝玉、湘云、黛玉、宝钗一桌,王夫人带着迎春姊妹三个人一桌,刘姥姥傍着贾母一桌。
贾母素日吃饭,皆有小丫鬟在旁边,拿着漱盂、麈尾、巾帕之物。
\zhu{麈尾:即拂尘:形如马尾,后有持柄,用以拂拭尘土,或驱赶蝇蚊,俗称“蝇甩子”。
古时多用麈(麈:音“主”,一种似骆驼的鹿类动物)兽之尾制成,故又称麈尾。
}如今鸳鸯是不当这差的了,今日鸳鸯偏接过麈尾来拂着。
丫鬟们知道他要撮弄刘姥姥,便躲开让他。
鸳鸯一面侍立,一面悄向刘姥姥说道:“别忘了。
”刘姥姥道:“姑娘放心。
”那刘姥姥入了坐,拿起箸来,沉甸甸的不伏手。
\zhu{伏手:顺手。
}原是凤姐和鸳鸯商议定了,单拿一双老年四楞象牙镶金的筷子与刘姥姥。
\zhu{老年:指象牙筷子上刻有寿星等图案,对老年人含有祝颂长寿之意。
四楞:指筷子顶端是方形的,有四个棱角。
象牙本已非常名贵,且沉重,又镶上金子,可见其贵重,真的是又“贵”又“重”。
}刘姥姥见了,说道:“这叉爬子比俺那里铁锨还沉,那里犟的过他。
”说的众人都笑起来。
\par
只见一个媳妇端了一个盒子站在当地,一个丫鬟上来揭去盒盖,里面盛着两碗菜。
李纨端了一碗放在贾母桌上。
凤姐儿偏拣了一碗鸽子蛋放在刘姥姥桌上。
贾母这边说声“请”,刘姥姥便站起身来,高声说道:“老刘,老刘,食量大似牛,吃一个老母猪不抬头。
”自己却鼓着腮不语。
\par
众人先是发怔,后来一听,上上下下都哈哈的大笑起来。
史湘云撑不住,一口饭都喷了出来;林黛玉笑岔了气,伏着桌子“嗳哟”;宝玉早滚到贾母怀里,贾母笑的搂着宝玉叫“心肝”;王夫人笑的用手指着凤姐儿,只说不出话来;薛姨妈也撑不住,口里茶喷了探春一裙子;探春手里的饭碗都合在迎春身上;惜春离了坐位,拉着他奶母叫揉一揉肠子。
地下的无一个不弯腰屈背,也有躲出去蹲着笑去的,也有忍着笑上来替他姊妹换衣裳的,独有凤姐鸳鸯二人撑着,还只管让刘姥姥。
\par
刘姥姥拿起箸来,只觉不听使,又说道:“这里的鸡儿也俊,下的这蛋也小巧,怪俊的。
我且肏攮一个。
”\zhu{肏攮:肏:有贬义色彩,本义为“性交”,一般用于骂人或戏谑。
肏攮;粗话,指吃喝,含贬义或戏谑义。
}众人方住了笑,听见这话又笑起来。
贾母笑的眼泪出来,琥珀在后捶着。
贾母笑道:“这定是凤丫头促狭鬼儿闹的,\zhu{促狭:刁钻机灵,爱捉弄人。
}快别信他的话了。
”那刘姥姥正夸鸡蛋小巧,要肏攮一个,凤姐儿笑道:“一两银子一个呢,你快尝尝罢,那冷了就不好吃了。
”刘姥姥便伸箸子要夹,那里夹的起来,满碗里闹了一阵,好容易撮起一个来,才伸着脖子要吃,偏又滑下来滚在地下,忙放下箸子要亲自去捡,早有地下的人捡了出去了。
刘姥姥叹道:“一两银子,也没听见响声儿就没了。
”众人已没心吃饭,都看着他笑。
\par
贾母又说:“这会子又把那个筷子拿了出来,又不请客摆大筵席。
都是凤丫头支使的,还不换了呢。
”地下的人原不曾预备这牙箸,本是凤姐和鸳鸯拿了来的,听如此说,忙收了过去,也照样换上一双乌木镶银的。
刘姥姥道:“去了金的,又是银的,到底不及俺们那个伏手。
”凤姐儿道:“菜里若有毒,这银子下去了就试的出来。
”刘姥姥道:“这个菜里若有毒,俺们那菜都成了砒霜了。
那怕毒死了也要吃尽了。
”贾母见他如此有趣,吃的又香甜,把自己的也都端过来与他吃。
又命一个老嬷嬷来,将各样的菜给板儿夹在碗上。
\par
一时吃毕,贾母等都往探春卧室中去说闲话。
这里收拾过残桌,又放了一桌。
刘姥姥看着李纨与凤姐儿对坐着吃饭,叹道:“别的罢了,我只爱你们家这行事。
怪道说‘礼出大家’。
”凤姐儿忙笑道:“你可别多心,才刚不过大家取笑儿。
”一言未了,鸳鸯也进来笑道:“姥姥别恼,我给你老人家赔个不是。
”刘姥姥笑道:“姑娘说那里话,咱们哄着老太太开个心儿,可有什么恼的!你先嘱咐我,我就明白了,不过大家取个笑儿。
我要心里恼,也就不说了。
”\ping{刘姥姥活得如此明白爽利。
}鸳鸯便骂人“为什么不倒茶给姥姥吃?”刘姥姥忙道:“刚才那个嫂子倒了茶来,我吃过了。
姑娘也该用饭了。
”凤姐儿便拉鸳鸯:“你坐下和我们吃了罢,省的回来又闹。
”鸳鸯便坐下了。
婆子们添上碗箸来,三人吃毕。
\par
刘姥姥笑道:“我看你们这些人都只吃这一点儿就完了,亏你们也不饿。
怪只道风儿都吹的倒。
”鸳鸯便问:“今儿剩的菜不少,都那去了?”婆子们道:“都还没散呢,在这里等着一齐散与他们吃。
”鸳鸯道:“他们吃不了这些,挑两碗给二奶奶屋里平丫头送去。
”凤姐儿道:“他早吃了饭了,不用给他。
”鸳鸯道:“他不吃了,喂你们的猫。
”婆子听了,忙拣了两样拿盒子送去。
鸳鸯道:“素云那去了?”李纨道:“他们都在这里一处吃,又找他作什么。
”鸳鸯道:“这就罢了。
”凤姐儿道:“袭人不在这里,你倒是叫人送两样给他去。
”鸳鸯听说,便命人也送两样去后,鸳鸯又问婆子们:“回来吃酒的攒盒可装上了?”婆子道:“想必还得一会子。
”鸳鸯道:“催着些儿。
”婆子应喏了。
\par
凤姐儿等来至探春房中,只见他娘儿们正说笑。
探春素喜阔朗,这三间屋子并不曾隔断。
当地放着一张花梨大理石大案,
\zhu{花梨大理石大案:花梨木制作、配以大理石案芯的大型书画案。}
案上磊着各种名人法帖,\zhu{法帖:这里指供人临摹的前人书法范本。
}并数十方宝砚,各色笔筒,笔海内插的笔如树林一般。
\zhu{笔海:插笔的器具。
}那一边设着斗大的一个汝窑花囊,\zhu{
汝窑:北宋时建于汝州(今河南临汝)的为宫廷烧制瓷器的官窑。
花囊:贮水插花的用具,上有圆孔多个,便于插枝软而朵大的花。
}插着满满的一囊水晶球儿的白菊。
西墙上当中挂着一大幅米襄阳《烟雨图》,\zhu{米襄阳:即米南宫,本名米芾.因是襄阳人,故称米襄阳。
宋代书画家,善画烟雨山水。
}左右挂着一副对联,乃是颜鲁公墨迹,
\zhu{颜鲁公:唐著名书法家颜真卿,因官平原太守,封鲁国公,故亦称“颜平原”、“颜鲁公”。}
其词云:\par
\hop
烟霞闲骨格,泉石野生涯。
\par
\zhu{这一联是对古代隐士浪迹山林优闲情趣的写照。
唐代田游岩曾对唐高宗李治说:“臣所谓泉石膏肓,烟霞痼疾者”(《新唐书·田游岩传》),或为此联所本。
烟霞:代指山水,山林。
骨格:这里是性情、志趣、格调的意思。
}\par
\hop
案上设着大鼎。
左边紫檀架上放着一个大观窑的大盘,\zhu{大观窑:大观是宋徽宗赵佶年号,可能指的是这个年代生产的。
}盘内盛着数十个娇黄玲珑大佛手。
\zhu{佛手:即佛手柑,形状像半握着的拳,秋天成熟,皮色鲜黄,有芳香。
}右边洋漆架上悬着一个白玉比目磬,\zhu{
洋漆:也称洋倭漆。指明朝宣德年间从日本传入的彩漆工艺。
比目:比目鱼,因两目比连于头部的一侧,故用以比喻恩爱夫妻。
磬:音“庆”,古代的一种打击乐器。
}旁边挂着小锤。
\par
那板儿略熟了些,便要摘那锤子要击,丫鬟们忙拦住他。
他又要那佛手吃,探春拣了一个与他说:“顽罢,吃不得的。
”东边便设着卧榻,拔步床上悬着葱绿双绣花卉草虫的纱帐。
\zhu{拔步床:我国型制最大的传统卧床,床前留空二至三尺,与床门围子形成小廊屋,前有镂雕花罩,两边有小柜,床屉下有抽斗,柜上设奁盒、灯台,并置有衣笼、便桶等,成为一个有照明条件的生活空间。
双绣:两面刺绣,无正反面的分别。
}板儿又跑过来看,说:“这是蝈蝈,这是蚂蚱。
”刘姥姥忙打了他一巴掌,骂道:“下作黄子,\zhu{下作黄子:
隋代称三岁以下的小孩为“黄”,唐代称初生的婴儿为“黄”。
}没干没净的乱闹。
倒叫你进来瞧瞧,就上脸了。
”\zhu{上脸:因受宠而放肆的意思。
俗有“蹬着鼻子上脸”的话。
}打的板儿哭起来,众人忙劝解方罢。
\par
贾母因隔着纱窗往后院内看了一回,说道:“后廊檐下的梧桐也好了,就只细些。
”正说话,忽一阵风过,隐隐听得鼓乐之声。
贾母问“是谁家娶亲呢?这里临街倒近。
”王夫人等笑回道:“街上的那里听的见,这是咱们的那十几个女孩子们演习吹打呢。
”贾母便笑道:“既是他们演,何不叫他们进来演习。
他们也逛一逛,咱们可又乐了。
”凤姐听说,忙命人出去叫来,又一面吩咐摆下条桌,铺上红毡子。
\par
贾母道:“就铺排在藕香榭的水亭子上,借着水音更好听。
回来咱们就在缀锦阁底下吃酒,又宽阔,又听的近。
”众人都说那里好。
贾母向薛姨妈笑道:“咱们走罢。
他们姊妹们都不大喜欢人来坐着,怕脏了屋子。
咱们别没眼色,正经坐一回子船喝酒去。
”说着大家起身便走。
探春笑道:“这是那里的话,求着老太太姨太太来坐坐还不能呢。
”贾母笑道:“我的这三丫头却好,只有两个玉儿可恶。
回来吃醉了,咱们偏往他们屋里闹去。
”\ping{伏刘姥姥醉卧绛芸轩。
}\par
说着,众人都笑了,一齐出来。
走不多远,已到了荇叶渚。
姑苏选来的几个驾娘早把两只棠木舫撑来,\zhu{舫:音“仿”,船的通称。
}众人扶了贾母、王夫人、薛姨妈、刘姥姥、鸳鸯、玉钏儿上了这一只,落后李纨也跟上去。
凤姐儿也上去,立在舡头上,也要撑舡。
贾母在舱内道:“这不是顽的,虽不是河里,也有好深的。
你快不给我进来。
”凤姐儿笑道:“怕什么!老祖宗只管放心。
”说着便一篙点开。
到了池当中,舡小人多,凤姐只觉乱晃,忙把篙子递与驾娘,方蹲下了。
然后迎春姊妹等并宝玉上了那只,随后跟来。
其馀老嬷嬷、散众丫鬟俱沿河随行。
宝玉道:“这些破荷叶可恨,怎么还不叫人来拔去。
”宝钗笑道:“今年这几日,何曾饶了这园子闲了,天天逛,那里还有叫人来收拾的工夫。
”林黛玉道:“我最不喜欢李义山的诗,只喜他这一句‘留得残荷听雨声’。
偏你们又不留着残荷了。
”宝玉道:“果然好句,以后咱们就别叫人拔去了。
”说着已到了花溆的萝港之下,\zhu{港:这里读作“哄”四声,指桥下涵洞。
}觉得阴森透骨,两滩上衰草残菱,更助秋情。
\par
贾母因见岸上的清厦旷朗,便问“这是你薛姑娘的屋子不是?”众人道:“是。
”贾母忙命拢岸,\zhu{拢岸:移船靠岸。
}顺着云步石梯上去,\zhu{云步石梯:回转曲折的石砌阶级。
}一同进了蘅芜苑,只觉异香扑鼻。
那些奇草仙藤愈冷愈苍翠,都结了实,似珊瑚豆子一般,
\zhu{珊瑚豆子:珊瑚作的像豆子形的饰物。}
累垂可爱。
及进了房屋,雪洞一般,一色玩器全无,案上只有一个土定瓶中供着数枝菊花,\zhu{土定瓶:定窑(北宋时建于定州,即今河北曲阳)烧制的一种质地较粗的瓶子。
土定:定窑瓷的品种之一。
}并两部书,茶奁茶杯而已。
\zhu{奁:音“连”,古代妇女梳妆用的镜匣和盛其他化妆品的器皿。
这里就是指小匣子。
}床上只吊着青纱帐幔,衾褥也十分朴素。
\par
贾母叹道:“这孩子太老实了。
你没有陈设,何妨和你姨娘要些。
我也不理论,也没想到,你们的东西自然在家里没带了来。
”说着,命鸳鸯去取些古董来,又嗔着凤姐儿:“不送些玩器来与你妹妹,这样小器。
”王夫人凤姐儿等都笑回说:“他自己不要的。
我们原送了来,他都退回去了。
”薛姨妈也笑说:“他在家里也不大弄这些东西的。
”贾母摇头道:“使不得。
虽然他省事,倘或来一个亲戚,看着不像;二则年轻的姑娘们,房里这样素净,也忌讳。
我们这老婆子,越发该住马圈去了。
你们听那些书上戏上说的小姐们的绣房,精致的还了得呢。
他们姊妹们虽不敢比那些小姐们,也不要很离了格儿。
有现成的东西,为什么不摆?若很爱素净,少几样倒使得。
我最会收拾屋子的,如今老了,没有这些闲心了。
他们姊妹们也还学着收拾的好,只怕俗气,有好东西也摆坏了。
我看他们还不俗。
如今让我替你收拾,包管又大方又素净。
我的梯己两件,收到如今,没给宝玉看见过,若经了他的眼,也没了。
”说着叫过鸳鸯来,亲吩咐道:“你把那石头盆景儿和那架纱桌屏,
\zhu{纱桌屏:摆放在桌案上的小型座屏,是清代以来使用比较普遍的室内陈设品。
座屏:带有底座而不能折叠的屏风。
}
还有个墨烟冻石鼎,\zhu{
墨烟:形容石头黑白相间、纹理略似云烟。
冻石:是一种半透明的名贵石头。
}这三样摆在这案上就够了。
再把那水墨字画白绫帐子拿来,把这帐子也换了。
”鸳鸯答应着,笑道:“这些东西都搁在东楼上的不知那个箱子里,还得慢慢找去,明儿再拿去也罢了。
”贾母道:“明日后日都使得,只别忘了。
”\ping{贾母不喜欢黛玉屋子的窗纱,凤姐儿一面说,早命人取了一匹软烟罗来了,是真心疼爱黛玉。
贾母不喜欢宝钗屋子的素净,但是并没有立刻把许诺的东西送过来,很可能只是客套而已。
}说着,坐了一回方出来,一径来至缀锦阁下。
文官等上来请过安,因问“演习何曲”。
贾母道:“只拣你们生的演习几套罢。
”文官等下来,往藕香榭去不提。
\par
这里凤姐儿已带着人摆设整齐,上面左右两张榻,榻上都铺着锦裀蓉簟,\zhu{
裀:音“音”,褥子、垫子、毯子等的通称。
锦裀:这里指华美的褥子。
蓉:芙蓉,荷花。
簟:音“电”,竹席。
蓉簟:有荷花图案的竹席。
}每一榻前有两张雕漆几,\zhu{雕漆:将涂上许多层漆的铜胎或木胎烘干、磨光后,再雕出立体花纹的技术。
几:低矮的桌子。
}也有海棠式的,也有梅花式的,也有荷叶式的,也有葵花式的,也有方的,也有圆的,其式不一。
一个上面放着炉瓶,\zhu{炉瓶:焚香用具。
即本书第五十三回所说的“炉瓶三事”,指一个香炉、一个香盒和一个放香铲等用的瓶子。
}一分攒盒,一个上面空设着,预备放人所喜食物。
上面二榻四几,是贾母薛姨妈;下面一椅两几,是王夫人的,馀者都是一椅一几。
东边是刘姥姥,刘姥姥之下便是王夫人。
西边便是史湘云,第二便是宝钗,第三便是黛玉,第四迎春、探春、惜春挨次下去,宝玉在末。
李纨凤姐二人之几设于三层槛内,二层纱厨之外。
\zhu{
纱厨:即碧纱橱,装在房内起隔开作用的一扇一扇的木板墙,也称“隔扇”、“槅扇”。中间两扇平日可以开关,或加挂帘子帷帐,又叫“纱橱”、“纱厨”。
槅心部分常糊以绿纱,故称碧纱橱。
}攒盒式样,亦随几之式样。
每人一把乌银洋錾自斟壶,\zhu{乌银:一种夹用硫磺、特殊方法熔铸的黑色银质。
錾(音“赞”):雕刻。
洋錾:西洋雕刻。
}一个十锦珐琅杯。
\zhu{十锦:即“什锦”,由多种原料制成或多种花样拼成的。
珐琅:音“发廊”,用石英、长石、硝石和碳酸钠等加上铅和锡的氧化物烧制成的像釉子的物质。
用它涂在铜质或银质器物上,经过烧制,形成不同颜色的釉质表面,既可防锈,又可作为装饰。
如搪瓷、景泰蓝等均为珐琅制品。
釉子:釉音“又”,以石英、长石、硼砂、黏土等为原料,磨成粉末,加水调制而成的物质,用来涂在陶瓷半成品的表面,烧制后发出玻璃光泽,并能增加陶瓷的机械强度和绝缘性能。
}\par
大家坐定,贾母先笑道:“咱们先吃两杯,今日也行一令才有意思。
”薛姨妈等笑道:“老太太自然有好酒令,我们如何会呢,安心要我们醉了。
我们都多吃两杯就有了。
”贾母笑道:“姨太太今儿也过谦起来,想是厌我老了。
”薛姨妈笑道:“不是谦,只怕行不上来倒是笑话了。
”王夫人忙笑道:“便说不上来,就便多吃一杯酒,醉了睡觉去,还有谁笑话咱们不成。
”薛姨妈点头笑道:“依令。
老太太到底吃一杯令酒才是。
”贾母笑道:“这个自然。
”说着便吃了一杯。
\par
凤姐儿忙走至当地,笑道:“既行令,还叫鸳鸯姐姐来行更好。
”众人都知贾母所行之令必得鸳鸯提着,故听了这话,都说:“很是。
”凤姐儿便拉了鸳鸯过来。
王夫人笑道:“既在令内,没有站着的理。
”回头命小丫头子:“端一张椅子,放在你二位奶奶的席上。
”鸳鸯也半推半就,谢了坐,便坐下,也吃了一钟酒,笑道:“酒令大如军令,不论尊卑,惟我是主。
违了我的话,是要受罚的。
”王夫人等都笑道:“一定如此,快些说来。
”鸳鸯未开口,刘姥姥便下了席,摆手道:“别这样捉弄人家,我家去了。
”众人都笑道:“这却使不得。
”鸳鸯喝令小丫头子们:“拉上席去!”小丫头子们也笑着,果然拉入席中。
刘姥姥只叫:“饶了我罢!”鸳鸯道:“再多言的罚一壶。
”刘姥姥方住了声。
\par
鸳鸯道:“如今我说骨牌副儿,\zhu{骨牌副儿:用两张以上骨牌的色点配成一套,叫做“一副儿”。
这里是三张牌为一副儿。
}从老太太起,顺领说下去,至刘姥姥止。
比如我说一副儿,将这三张牌拆开,先说头一张,次说第二张,再说第三张,说完了,合成这一副儿的名字。
无论诗词歌赋,成语俗话,比上一句,\zhu{比:类比,比喻。
从后文可见,玩法是需要从牌上图案发挥联想。
}都要叶韵。
\zhu{叶韵,叶音“协”,一作“谐韵”、“协韵”。
诗韵术语。
谓有些韵字如读本音,便与同诗其他韵脚不和,须改读某音,以协调声韵,故称。
}错了的罚一杯。
”众人笑道:“这个令好,就说出来。
”鸳鸯道:“有了一副了。
左边是张‘天’。
”贾母道:“头上有青天。
”\zhu{天:指天牌。
此牌上下两个六点,点色红绿各半,互相岔开。
头上有青天:这是一句旧谚语的下半部分,全句为“眼子头上有青天”(眼子:指外行,容易受欺的人),意思是容易上当受骗常常吃亏的人,自有头上的“青天”来保佑。
}众人道:“好。
”鸳鸯道:“当中是个‘五与六’。
”贾母道:“六桥梅花香彻骨。
”\zhu{六桥梅花香彻骨:这是一张五六共十一点的牌,各点皆绿色。
这里用“六桥”代指六点。
“梅花”代指五点。
六桥:即跨虹、东浦、压堤、望山、锁澜、映波等六座桥,北宋苏轼修建于杭州西湖苏堤上,堤上多种梅花。
见《浙江通志·杭州府》。
香彻骨:香气沁透肌骨。
}鸳鸯道:“剩得一张‘六与幺’。
”贾母道:“一轮红日出云霄。
”\zhu{一轮红日出云霄:这是一张幺六共七点的牌,“一轮红日” 代指红色的幺点,“云霄”代指绿色的六点。
}\ping{这话有种乐观在里面,红楼梦一部书就是个“散”字,从热热闹闹零落到白茫茫真干净,可是在干净的地上又会重新聚集起新的热闹,所以其实聚散有时,纯粹的乐观悲观都是不必要的。
}鸳鸯道:“凑成便是个‘蓬头鬼’。
”贾母道:“这鬼抱住钟馗腿。
”\zhu{蓬头鬼:是长六、五六、幺六这副牌的名称。
这鬼抱住钟馗腿:钟馗(馗音“葵”),传说是唐代开元间人,应举未中,死后托梦给唐玄宗,立誓要除尽天下妖孽。
玄宗醒后,命画工吴道子画成图像,告示天下,岁暮时家家画其像“以祛邪魅”。
见宋代沈括《梦溪笔谈》。
昆曲《嫁妹》中有五个小鬼扯衣抱腿同钟馗玩闹的情节。
}说完,大家笑说:“极妙。
”贾母饮了一杯。
鸳鸯又道:“有了一副。
左边是个‘大长五’。
”薛姨妈道:“梅花朵朵风前舞。
”\zhu{大长五:牌名,上下各五点,皆绿色。
梅花朵朵:因为“大长五”是由两个梅花形的五点组成,所以说“梅花朵朵”。
}鸳鸯道:“右边还是个‘大五长’。
”薛姨妈道:“十月梅花岭上香。
”\zhu{大五长:是为了押韵而把“大长五”倒过来的说法。
十月梅花:因为“大五长”是由两个梅花形的五点共十点组成,所以说“十月梅花”。
}鸳鸯道:“当中‘二五’是杂七。
”薛姨妈道:“织女牛郎会七夕。
”\zhu{杂七:是一张上二下五共七点皆为绿色的牌。
}鸳鸯道:“凑成‘二郎游五岳’。
”薛姨妈道:“世人不及神仙乐。
”\zhu{
二郎游五岳:是长五、二五、长五这副牌的名称。
这里用“二郎”代指其中的一个“二”,“五岳”代指其中的五个“五”。
二郎:神话传说人物。
五岳:中岳嵩山、东岳泰山、南岳衡山、西岳华山、北岳恒山。
}说完,大家称赏,饮了酒。
鸳鸯又道:“有了一副。
左边‘长幺’两点明。
”湘云道:“双悬日月照乾坤。
”\zhu{长幺:牌名,上下各一点,皆红色。
“双悬日月”代指“长幺”的两个红点。
乾坤:本为《周易》的两个卦名,这里引申为天地的代称,亦即指牌面的上下两点。
}鸳鸯道:“右边‘长幺’两点明。
”湘云道:“闲花落地听无声。
”\zhu{闲花落地听无声:这里用“闲花”代指“长幺”的两个红点,而“长幺”又名地牌,所以称作“落地”。
}鸳鸯道:“中间还得‘幺四’来。
”湘云道:“日边红杏倚云栽。
”\zhu{日边红杏倚云栽:这是一张“幺四”牌,“日”代指红色的一点,“红杏”代指红色的四点。
}鸳鸯道:“凑成‘樱桃九熟’。
”湘云道:“御园却被鸟衔出。
”\zhu{樱桃九熟:是长幺、幺四、长幺这副牌的名称,全副九点皆红色,所以用九颗熟透的樱桃作比。
御园却被鸟衔出:这句和上句紧紧相连,意思是樱桃被鸟从皇帝的花园里衔出去了。
御园:皇帝的花园。
衔,同“含”。
}说完饮了一杯。
鸳鸯道:“有了一副。
左边是‘长三’。
”宝钗道:“双双燕子语梁间。
”\zhu{长三:牌名,上下各三点,皆绿色,双行斜排。
双双燕子语梁间:用“双双燕子”代指“长三”分两行平行斜排的绿色六点。
}鸳鸯道:“右边是‘三长’。
”宝钗道:“水荇牵风翠带长。
”\zhu{三长:是为押韵而把“长三”倒过来的说法。
荇:音“幸”,多年生水草。
水荇牵风翠带长:荇菜根在水底,叶浮水上,顺风逐波如翠带飘动,代指“长三”分两行斜排的绿色六点。
}鸳鸯道:“当中‘三六’九点在。
”宝钗道:“三山半落青天外。
”\zhu{三六:牌名,上三下六共九点,皆绿色。
三山半落青天外:用“三山”代指“三六”上面斜排的三点,用“青天”代指“三六”下面六点,六点是“天牌”的一半,所以说“半落青天”。
}鸳鸯道:“凑成‘铁锁练孤舟’。
”宝钗道:“处处风波处处愁。
”\zhu{
练:通“链”,链子。
铁锁练孤舟:这里是一副骨牌的名字,代指长三、三六、长三这副牌。
其中两个“长三”和“三六”中左右上下的“三”,很像是一条条的“铁锁链”,以孤“六”代指“孤舟”。
}说完饮毕。
鸳鸯又道:“左边一个‘天’。
”黛玉道:“良辰美景奈何天。
”宝钗听了,回头看着他。
黛玉只顾怕罚,也不理论。
鸳鸯道:“中间‘锦屏’颜色俏。
”黛玉道:“纱窗也没有红娘报。
”\zhu{锦屏:牌名.上四下六共十点,上红下绿形似彩色屏风,故称。
纱窗也没有红娘报:元代王实甫《西厢记》第一本第四折中张生的唱词:“纱窗外定有红娘报”而成,这句话的意思是,红娘肯定会站在纱窗外给张生和崔莺莺通风报信。
《西厢记》金圣叹评本改“定有”为“没有”,此依金批本。
纱窗:代指牌上的六个绿点。
红娘:代指牌上的四个红点。
}鸳鸯道:“剩了‘二六’八点齐。
”黛玉道:“双瞻玉座引朝仪。
”\zhu{二六:牌名,上二下六共八点,各点皆绿色。
双瞻玉座引朝仪:双:指站在殿门外的两个女官(昭容),这里代指牌上的两点。
瞻:仰望。
玉座:即御座,皇帝的座位。
朝仪:封建时代臣子朝见皇帝时按礼仪规定所站的行列,这里代指牌上的六点。
}鸳鸯道:“凑成‘篮子’好采花。
”黛玉道:“仙杖香挑芍药花。
”\zhu{凑成篮子好采花:这副牌由长六、四六、二六凑成,名叫“篮子”,其中红色的四点像花朵,所以说“好采花”。
}说完,饮了一口。
鸳鸯道:“左边‘四五’成花九。
”\zhu{花九:牌名,也叫“四五”。
这张牌上四下五共九点,点色上红下绿,故称“花九”。
}迎春道:“桃花带雨浓。
”众人道:“该罚!错了韵,而且又不像。
”迎春笑着饮了一口。
原是凤姐儿和鸳鸯都要听刘姥姥的笑话,故意都令说错,都罚了。
至王夫人,鸳鸯代说了个,下便该刘姥姥。
\par
刘姥姥道:“我们庄家人闲了,也常会几个人弄这个,但不如说的这么好听。
少不得我也试一试。
”众人都笑道:“容易说的。
你只管说,不相干。
”鸳鸯笑道:“左边‘四四’是个人。
”刘姥姥听了,想了半日,说道:“是个庄家人罢。
”\zhu{四四:牌名,上下各四共八点,全红色,也叫“长四”,又叫“人牌”,所以说“是个人”。
}众人哄堂笑了。
贾母笑道:“说的好,就是这样说。
”刘姥姥也笑道:“我们庄家人,不过是现成的本色,众位别笑。
”鸳鸯道:“中间‘三四’绿配红。
”刘姥姥道:“大火烧了毛毛虫。
”\zhu{三四:牌名,上三下四共七点,点色上绿下红。
大火烧了毛毛虫:这里用“大火”代指“三四”下边红色的四点,用“毛毛虫”代指“三四”上边绿色斜排的三点。
}众人笑道:“这是有的,还说你的本色。
”鸳鸯道:“右边‘幺四’真好看。
”刘姥姥道:“一个萝蔔一头蒜。
”\zhu{幺四:牌名,上一下四共五点,皆红色。
一个萝蔔一头蒜:这里用“一个萝蔔”(萝蔔即萝卜)代指“么四”上边红色的一点,用紫皮多瓣的“一头蒜”代指“幺四”下边红色的四点。
}众人又笑了。
鸳鸯笑道:“凑成便是一枝花。
”刘姥姥两只手比着,说道:“花儿落了结个大倭瓜。
”\zhu{一枝花:是长四、三四、幺四这副牌的名称,其中斜排的三个绿点像花枝,其馀红色的各点像花朵。
倭瓜:南瓜。
}众人大笑起来。
只听外面乱嚷——\par
\foot{
\footPic{牙牌示意图(黑色代表红色,灰色代表绿色)}{yapailing.png}{1.0}
}
\qi{总评:写贫贱辈低首豪门,凌辱不计,诚可悲乎!此故作者以警贫贱。
而富室贵豪亦当于其间着意。
}
\dai{079}{刘姥姥吃饭逗趣}
\dai{080}{金鸳鸯三宣牙牌令}
\sun{p40-1}{沁芳亭坐赏大观园}{沁芳亭上,大锦褥子铺在栏杆榻板上,贾母和刘姥姥坐下。
刘姥姥感叹,这个园子比画上的还要好十倍。
贾母于是让惜春给刘姥姥画一张画带回家去。
刘姥姥听了夸赞惜春既漂亮还能干,是神仙托生的。
}
\sun{p40-2}{刘姥姥潇湘馆滑倒}{刘姥姥到了潇湘馆,一进门,只见两边翠竹夹路,土地下苍苔布满,中间羊肠一条石子漫的甬路。
刘姥姥让出来与贾母众人走,自己却走土地。
琥珀拉她道:“姥姥,你上来走,仔细苍苔滑了。
”刘姥姥道:“不相干,我们走熟了。
”她只顾说话,不防脚下果踩滑了,“咕咚”一跤跌倒,众人都拍手呵呵大笑。
贾母笑骂道:“小蹄子们!还不搀起来,只站着笑!”说话时,刘姥姥已爬起来,自己也笑了,说道:“才说嘴,就打了嘴了。
”}
\sun{p40-3}{刘姥姥吃饭闹笑话}{贾母等人乘船便向紫菱洲、蓼溆一带走来。
未至池前,只见几个婆子手里都捧着一色捏丝戗金五彩大盒子走来。
众人在秋爽斋吃早饭,鸳鸯凤姐有意拿刘姥姥取乐,鸳鸯事前与刘姥姥耳语了几句。
席间,拿了一双老年四楞象牙镶金的筷子,又端来一碗鸽子蛋放在刘姥姥桌上。
刘姥姥高声道:“老刘,老刘,食量大似牛,吃一个老母猪不抬头!”众人哈哈大笑起来。
}
\sun{p40-4}{金鸳鸯三宣牙牌令}{图左侧:众人隔着纱帘隐隐听到鼓乐之声,得知是家里十来个女孩子演习吹打,便命她们铺排在藕香榭的水亭子上,借着水音更好听,图右侧:
众人一径来到缀锦阁。
凤姐早已带着人摆设整齐。
上面是贾母和薛姨妈,东边是刘姥姥、王夫人。
西边依次是湘云、宝钗、黛玉、迎春、探春、惜春,宝玉在末。
贾母提议行令助兴,便由鸳鸯执令,欢笑中众人依次行令畅饮。
}