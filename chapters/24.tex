\chapter{醉金刚轻财尚义侠\quad 痴女儿遗帕惹相思}
\geng{夹写“醉金刚”一回是书中之大净场,
\zhu{
大净场:谓此回醉金刚倪二在旧剧中应饰净角(大花脸),故其戏为“大净场”。
净角即花面,以粉墨涂面而演出者,多扮演直臣名将,兼扮奸雄暴君。
}
聊醒看官倦眼耳。
然亦书中必不可少之文,必不可少之人。
今写在市井俗人身上,又加一“侠”字,则大有深意存焉。
}\par
话说林黛玉正自情思萦逗,缠绵固结之时,忽有人从背后击了一掌,说道:“你作什么一个人在这里?”林黛玉倒唬了一跳,回头看时,不是别人,却是香菱。
林黛玉道:“你这个傻\geng{此“傻”字加于香菱,则有多少丰神跳于纸上,其娇憨之态可想而知。
}丫头,唬我这么一跳好的。
你这会子打那里来?”香菱嘻嘻的笑道:“我来寻我们的姑娘的,找他总找不着。
你们紫鹃也找你呢,\geng{一丝不漏。
}说琏二奶奶送了什么茶叶来给你的。
走罢,回家去坐着。
”\geng{“回家去坐着”之言,是恐石上冷意。
}一面说着,一面拉着黛玉的手回潇湘馆来了。
果然凤姐儿送了两小瓶上用新茶来。
林黛玉和香菱坐了。
况他们有甚正事谈讲。
\geng{为学诗伏线。
}不过说些这一个绣的好,那一个刺的精,又下一回棋,看两句书,\geng{棋不论盘,书不论章,皆是娇憨女儿神理,写得不即不离,似有似无,妙极!}香菱便走了。
不在话下。
\geng{是书最好看如此等处,系画家山水、树头、丘壑俱备,末用浓淡墨点苔法也。
丁亥夏。
畸笏叟。
}\par
如今且说宝玉因被袭人找回房去,果见鸳鸯歪在床上看袭人的针线呢,见宝玉来了,便说道:“你往那里去了?老太太等着你呢,叫你过那边请大老爷的安去。
还不快换了衣服走呢。
”袭人便进房去取衣服。
宝玉坐在床沿上,褪了鞋等靴子穿的工夫,回头见鸳鸯穿着水红绫子袄儿,\zhu{水红:比粉红略深而鲜艳。
}青缎子背心,束着白绉绸汗巾儿,\zhu{绉:音“宙”,皱纹。
}脸向那边低着头看针线,脖子上戴着花领子。
\zhu{领子:亦称领衣,即俗所谓“假领”。
清代衣、领分用。
}宝玉便把脸凑在他脖项上,闻那粉香油气,不住用手摩挲,其白腻不在袭人之下,便猴上身去涎皮笑道:
\zhu{涎[xián]皮:即“涎脸”,厚着脸皮与人纠缠。}
“好姐姐,把你嘴上的胭脂赏我吃了罢。
”\geng{胭脂是这样吃法。
看官可经过否?}一面说着,一面扭股糖似的粘在身上。
\par
鸳鸯便叫道:“袭人,你出来瞧瞧。
\geng{不向宝玉说话,又叫袭人,鸳鸯亦是幻情洞天也。
\zhu{
洞天:道家认为神仙居处多在名山洞府中,因洞中别有天地,故称为「洞天」。
借指地形隐蔽,风景优美的地方。
}
}你跟他一辈子,也不劝劝,还是这么着。
”袭人抱了衣服出来,向宝玉道:“左劝也不改,右劝也不改,你到底是怎么样?你再这么着,\geng{此五字内有深意深心。
}这个地方可就难住了。
”一边说,一边催他穿了衣服,同鸳鸯往前面来见贾母。
\par
见过贾母,出至外面,人马俱已齐备。
刚欲上马,只见贾琏请安回来了,\geng{一丝不漏。
}正下马,二人对面,彼此问了两句话。
只见旁边转出一个人来,\geng{芸哥此处一现,后文不见突然。
}“请宝叔安”。
宝玉看时,只见这人容长脸,\zhu{容长脸:犹言长方脸。
}长挑身材,年纪只好十八九岁,生得着实斯文清秀,倒也十分面善,只是想不起是那一房的,\geng{大族人众,毕真,有是理。
}叫什么名字。
贾琏笑道:“你怎么发呆,连他也不认得?他是后廊上住的五嫂子的儿子芸儿。
”\zhu{后廊上:这里不是指房屋的后廊,是指贾府附近的某一居处。
如说“西廊下”,也是指某一居处而言。
}宝玉笑道:“是了,是了,我怎么就忘了。
”因问他母亲好,这会子什么勾当。
贾芸指贾琏道:“找二叔说句话。
”宝玉笑道:“你倒比先越发出挑了,\geng{何尝是十二三岁小孩语。
}倒像我的儿子。
”贾琏笑道:“好不害臊!人家比你大四五岁呢,就替你作儿子了?”宝玉笑道:“你今年十几岁了?”贾芸道:“十八岁。
”\ping{贾芸找关系之一:贾琏。
}\par
原来这贾芸最伶俐乖觉,听宝玉这样说,便笑道:“俗语说的,‘摇车里的爷爷,拄拐的孙孙’。
虽然岁数大,山高高不过太阳。
只从我父亲没了,
\zhu{只从:自从。}
这几年也无人照管教导。
\geng{虽是随机而应,伶俐人之语,余却伤心。
}
如若宝叔不嫌侄儿蠢笨,认作儿子,就是我的造化了。
”\ping{认比自己小的孩子做爸爸,贾芸实在能屈能伸。
}贾琏笑道:“你听见了?认儿子不是好开交的呢。
”\geng{是兄凑弟趣,可叹!}说着就进去了。
宝玉笑道:“明儿你闲了,只管来找我,别和他们鬼鬼祟祟的。
\geng{何其堂皇正大之语。
}这会子我不得闲儿。
明儿你到书房里来,和你说天话儿,我带你园里顽耍去。
”说着扳鞍上马,\zhu{扳:通“攀”。
}众小厮围随往贾赦这边来。
\ping{贾芸找关系之二:贾宝玉。
}\par
见了贾赦,不过是偶感些风寒,先述了贾母问的话,然后自己请了安。
贾赦先站起来回了贾母话,\geng{一丝不乱。
}次后便唤人来:“带哥儿进去太太屋里坐着。
”宝玉退出,来至后面,进入上房。
邢夫人见了他来,先倒站了起来请过贾母安,\geng{一丝不乱。
}宝玉方请安。
\chen{好规矩。
}
邢夫人拉他上炕坐了,方问别人好,又命人倒茶来。
\geng{好层次,好礼法,谁家故事?}一钟茶未吃完,只见那贾琮来问宝玉好。
邢夫人道:“那里找活猴儿去!你那奶妈子死绝了,也不收拾收拾你,弄的黑眉乌嘴的,那里像大家子念书的孩子!”\par
正说着,只见贾环、贾兰小叔侄两个也来了,请过安,邢夫人便叫他两个椅子上坐了。
贾环见宝玉同邢夫人坐在一个坐褥上,邢夫人又百般摩挲抚弄他,早已心中不自在了,\geng{千里伏线。
}坐不多时,便和贾兰使眼色儿要走。
贾兰只得依他,一同起身告辞。
宝玉见他们要走,自己也就起身,要一同回去。
邢夫人笑道:“你且坐着,我还和你说话呢。
”宝玉只得坐了。
邢夫人向他两个道:“你们回去,各人替我问你们各人母亲好。
你们姑娘、姐姐妹妹都在这里呢,闹的我头晕,今儿不留你们吃饭了。
”\geng{明显薄情之至。
}贾环等答应着,便出来回家去了。
\par
宝玉笑道:“可是姐姐们都过来了,怎么不见?”邢夫人道:“他们坐了一会子,都往后头不知那屋里去了。
”宝玉道:“大娘方才说有话说,不知是什么话?”邢夫人笑道:“那里有什么话,不过是叫你等着,同你姊妹们吃了饭去。
还有一个好玩的东西给你带回去玩。
”娘儿两个说话,不觉早又晚饭时节。
调开桌椅,罗列杯盘,母女姊妹们吃毕了饭。
宝玉去辞贾赦,同姊妹们一同回家,见过贾母、王夫人等,各自回房安息。
不在话下。
\geng{一段为五鬼魇魔法作引。
\zhu{魇:音“眼”。
魇魔法:迷信中,一种可致人于死的妖术。
}脂砚。
}\ping{邢夫人对待宝玉和贾环差异很大,可能是故意挑拨两人的关系,激怒贾环,引出后文赵姨娘和贾环的报复。
}\par
且说贾芸进去见了贾琏,因打听可有什么事情。
贾琏告诉他:“前儿倒有一件事情出来,偏生你婶子再三求了我,\geng{反说体面话,惧内人累累如是。
}给了贾芹了。
他许了我,说明儿园里还有几处要栽花木的地方,等这个工程出来,一定给你就是了。
”贾芸听了,半晌说道:\ping{失望,也了解自己求错了人。
}“既是这样,我就等着罢。
叔叔也不必先在婶子跟前提我今儿来打听的话,\geng{已得了主意了。
}到跟前再说也不迟。
”贾琏道:“提他作什么,\geng{已被芸哥瞒过了。
}我那里有这些工夫说闲话儿呢。
明儿一个五更,还要到兴邑去走一趟,须得当日赶回来才好。
你先去等着,后日起更以后你来讨信儿,
\zhu{起更[gēng]:旧指入夜第一次打更(约在晚七点)。}
来早了我不得闲。
”说着便回后面换衣服去了。
\ping{贾芸找关系之三:凤姐。
此时贾芸已经看出来贾琏并不能做主。
}\par
贾芸出了荣国府回家,一路思量,想出一个主意来,便一径往他母舅卜世仁\chen{名义可思。
}家来。
\geng{既云“不是人”,如何肯共事?想芸哥此来空了。
}原来卜世仁现开香料铺,方才从铺子里来,忽见贾芸进来,彼此见过了,因问他这早晚什么事跑了来。
贾芸道:“有件事求舅舅帮衬帮衬。
\zhu{帮衬:意为帮助。
}我有一件事,用些冰片麝香使用,好歹舅舅每样赊四两给我,八月里按数送了银子来。
”\geng{甥舅之谈如此,叹叹!}卜世仁冷笑道:“再休提赊欠一事。
\geng{何如,何如?余言不谬。
}前儿也是我们铺子里一个伙计,替他的亲戚赊了几两银子的货,至今总未还上。
因此我们大家赔上,立了合同,再不许替亲友赊欠。
谁要赊欠,就要罚他二十两银子的东道。
\zhu{东道:请客的事或义务。
}况且如今这个货也短,你就拿现银子到我们这不三不四的铺子里来买,\geng{推脱之辞。
}也还没有这些,只好倒扁儿去。
\zhu{倒扁儿:即倒揣。
这里指无货可卖,须到别家铺子去套购货物来应付门面。
}这是一。
二则你那里有正经事,不过赊了去又是胡闹。
你只说舅舅见你一遭儿就派你一遭儿不是。
你小人儿家很不知好歹,也到底立个主见,赚几个钱,弄得穿是穿吃是吃的,我看着也喜欢。
”\par
贾芸笑道:“舅舅说的倒干净。
我父亲没的时候,我年纪又小,不知事。
后来听见我母亲说,都还亏舅舅们在我们家出主意,料理的丧事。
难道舅舅就不知道的,还是有一亩地两间房子,如今在我手里花了不成?巧媳妇做不出没米的粥来,叫我怎么样呢?\ping{可能暗指舅舅侵吞家产。
}还亏是我呢,要是别个,死皮赖脸三日两头儿来缠着舅舅,\geng{芸哥亦善谈,井井有理。
}要三升米二升豆子的,\geng{余二人亦不曾有是气。
\ping{二人可能是指作者和批书人,可见两人关系密切。
亦不曾有:可能是“亦曾有”。
}}舅舅也就没有法呢。
”\par
卜世仁道:“我的儿,舅舅要有,还不是该的。
我天天和你舅母说,只愁你没算计儿。
你但凡立的起来,到你大房里,就是他们爷儿们见不着,便下个气,和他们的管家或者管事的人们嬉和嬉和,\zhu{嬉和嬉和:主动亲近、讨好之意。
}\geng{可怜可叹,余竟为之一哭。
}也弄个事儿管管。
前日我出城去,撞见了你们三房里的老四,骑着大叫驴,带着五辆车,有四五十和尚道士,\geng{妙极!写小人口角,羡慕之言加一倍,\zhu{二十三回:“(贾芸)登时雇了大叫驴,自己骑上,又雇了几辆车,至荣国府角门,唤出二十四个人来,坐上车,一径往城外铁槛寺去了。
”从二十四夸张到四五十,即为“加一倍”。
}毕肖。
却又是背面傅粉法。
\zhu{背面傅粉:小说艺术技法之一。
我国古代画家用绢作画时,在绢面涂上铅粉,以衬托画面色彩鲜明。
古代评论家借来说明小说的一种艺术技法,是指从反面、从侧面,或者通过间接描写别的人物,去描写某个人物,是一种间接描写。
}}往家庙去了。
他那不亏能干,这事就到他了!”贾芸听他韶刀的不堪,\zhu{韶刀:较“唠叨”稍重,不仅语言啰嗦,而且分寸失当。
}便起身告辞。
\geng{有志气,有果断。
}卜世仁道:“怎么急的这样,吃了饭再去罢。
”一句未完,只见他娘子说道:“你又糊涂了。
\geng{虽写小人家涩细,一吹一唱,酷肖之至,却是一气逼出,后文方不突然。
《石头记》笔仗全在如此样者。
}说着没有米,这里买了半斤面来下给你吃,这会子还装胖呢。
留下外甥挨饿不成?”卜世仁说:“再买半斤来添上就是了。
”他娘子便叫女孩儿:“银姐,往对门王奶奶家去问,有钱借二三十个,明儿就送过来。
”夫妻两个说话,那贾芸早说了几个“不用费事”,去的无影无踪了。
\geng{有知识有果断人,自是不同。
}\chen{世情写透。
}\par
不言卜家夫妇,且说贾芸赌气离了母舅家门,一径回归旧路,心下正自烦恼,一边想,一边低头只管走,不想一头就碰在一个醉汉身上,把贾芸唬了一跳。
\geng{自上看来,可是一口气否?\zhu{一口气:应该是指从离开舅舅家到撞到倪二间隔时间很短。
}}听那醉汉骂道:“臊你娘的!瞎了眼睛,碰起我来了。
”贾芸忙要躲身,早被那醉汉一把抓住,对面一看,不是别人,却是紧邻倪二。
原来这倪二是个泼皮,专放重利债,在赌博场吃闲钱,
\zhu{吃闲钱:指抽头(抽头:向赢钱的赌徒抽取一部分的利益给提供赌博场所的人)、放赌帐等不用费力就获取银钱。}
专管打降吃酒。
\zhu{打降[jiàng]:即打架。
清代郝懿行《证俗文》:“俗谓手械斗为打降。
降,下也,打之使降服也。
……或读为打架,盖降声之转也。
”一说即为“打杠”。
赌场开局,下注后,某一局外人将局中某一方押的注提走,意为吃赔都归他负责,谓之“打降[gàng]”。
能在赌场打降者,一般都有一定的势力。
}如今正从欠钱人家索了利钱,吃醉回来,不想被贾芸碰了一头,正没好气,抡拳就要打。
\geng{这一节对《水浒》杨志卖大刀遇没毛大虫一回看,觉好看多矣。
\zhu{杨志卖大刀遇没毛大虫:
《水浒传》第十二回写杨志失陷花石纲后,盘缠使尽,到桥头卖祖传宝刀,遇到“没毛大虫”牛二纠缠。
倪二是泼皮,牛二也是泼皮,故脂批将倪二与牛二相比。但牛二无赖撒泼,被杨志一时性起杀死;
倪二虽也是泼皮,但这一回中“轻财尚义侠”,二书虽均写泼皮,写法及人物个性却各有不同。
}
己卯冬夜。
脂砚。
}只听那人叫道:“老二住手!是我冲撞了你。
”倪二听见是熟人的语音,将醉眼睁开看时,见是贾芸,忙把手松了,趔趄着笑道:\zhu{趔趄[lièqie]:脚步歪斜不稳。
}\geng{写生之笔。
}“原来是贾二爷,\geng{如此称呼,可知芸哥素日行止,是“金盆虽破分量在”也。
}我该死,我该死。
这会子往那里去?”贾芸道:“告诉不得你,平白的又讨了个没趣儿。
”\geng{本无心之谈也。
}倪二道:“不妨不妨,\geng{如闻。
}有什么不平的事,告诉我,替你出气。
\geng{写得酷肖,总是渐次逼出,不见一丝勉强。
}这三街六巷,凭他是谁,有人得罪了我醉金刚倪二的街坊,管叫他人离家散!”贾芸道:“老二,你且别气,听我告诉你这原故。
”\geng{可是一顺而来?}说着,便把卜世仁一段事告诉了倪二。
倪二听了大怒,“要不是令舅,我便骂不出好话来,\geng{仗义人岂有不知礼者乎?何尝是破落户?冤杀金刚了。
}真真气死我倪二。
也罢,你也不用愁烦,我这里现有几两银子,你若用什么,只管拿去买办。
但只一件,你我作了这些年的街坊,我在外头有名放帐,你却从没有和我张过口。
也不知你厌恶我是个泼皮,\geng{知己知彼之话。
}怕低了你的身分,也不知是你怕我难缠,利钱重?若说怕利钱重,这银子我是不要利钱的,也不用写文约,若说怕低了你的身分,\geng{知己知彼之话。
}我就不敢借给你了,各自走开。
”一面说,一面果然从搭包里掏出一卷银子来。
\zhu{搭包:也作搭膊。
有两种:一种是长条形,两端有口袋,搭在肩上,前后盛放钱物。
一种是用长条布捆叠成腰带状,扎在腰间,也可裹系钱物。
}\par
贾芸心下自思:“素日倪二虽然是泼皮无赖,却因人而使,\zhu{因人而使:意思是指按照各人不同的情况采取不同的方法区别对待。指倪二看人下菜、世故变通,倘若贾芸没有贾府背景,碰到倪二的下场会很惨。
}\geng{四字定评,难得难得,非豪杰不可当。
}颇颇的有义侠之名。
若今日不领他这情,怕他臊了,倒恐生事。
\ping{贾芸害怕倪二逼自己借钱,然后收取高利贷。
倪二虽然这时候说不受利钱,但是保不准酒醒之后依旧索要高额利息。
}不如借了他的,改日加倍还他也倒罢了。
”想毕笑道:“老二,你果然是个好汉,我何曾不想着你,和你张口。
但只是我见你所相与交结的,都是些有胆量的有作为的人,似我们这等无能无为的你倒不理。
\geng{芸哥亦善谈,好口齿。
}我若和你张口,你岂肯借给我。
今日既蒙高情,我怎敢不领,回家按例写了文约过来便是了。
”倪二大笑道:“好会说话的人。
我却听不上这话。
\geng{“光棍眼内揉不下沙子”是也。
\zhu{
光棍:精明人,好汉。光棍眼内揉不下沙子:比喻精明人不能容忍糊弄人的不平的事,也指骗不住精明人的眼睛,又指好汉容不得不平的事。
}
}既说‘相与交结’四个字,如何放帐给他,使他的利钱!\geng{如今不单是亲友言利,不但亲友,即闺阁中亦然,不但生意新发户,即大户旧族颇颇有之。
}
既把银子借与他,图他的利钱,便不是相与交结了。
闲话也不必讲。
既肯青目,\zhu{(垂)青目:也作“垂青”,意思是用青眼(即正眼)看人,表示尊重、看得起。
《晋书·阮籍传》:“籍又能为青白眼。
见礼俗之士,以白眼对之。
及嵇喜来吊,籍作白眼,喜不怿而退。
喜弟康闻之,乃赍酒扶琴造焉,籍大悦,乃见青眼。
”}这是十五两三钱有零的银子,便拿去治买东西。
你要写什么文契,趁早把银子还我,让我放给那些有指望的人使去。
”\geng{爽快人,爽快语。
}贾芸听了,一面接了银子,一面笑道:“我便不写罢了,有何着急的。
”倪二笑道:“这不是话。
天气黑了,也不让茶让酒,我还到那边有点事情去,你竟请回去。
我还求你带个信儿与舍下,叫他们早些关门睡罢,我不回家去了,倘或有要紧事儿,叫我们女儿明儿一早到马贩子王短腿家\geng{常起坐处人,毕真。
}来找我。
”一面说,一面趔趄着脚儿去了,\geng{仍应前。
}不在话下。
\geng{读阅“醉金刚”一回,\sout{务}[如]吃刘铉丹家山查丸一付,一笑。
\zhu{山查:即“山楂”。
《帝京岁时记胜》「皇都品汇」云:刘铉丹山楂丸子,能补能消。
舅舅不疼叔叔不爱的贾芸的遭遇令读者生气,遇到醉金刚倪二“轻财尚义侠”,读至此方才消气也。}
余卅年来得遇金刚之样人不少,不及金刚者亦不少,惜书上不便历历注上芳讳,是余不\sout{是}[足]心事也。
壬午孟夏。
}\par
且说贾芸偶然碰了这件事,心中也十分罕希,想那倪二倒果然有些意思,只是还怕他一时醉中慷慨,到明日加倍的要起来,便怎处,心内犹豫不决。
\geng{芸哥实怕倪二,并非以小人之心度君子也。
}忽又想道:“不妨,等那件事成了,也可加倍还他。
\ping{贾府当差油水大,难怪贾芸贾芹都上赶着托关系谋个差事。}
”想毕,一直走到个钱铺里,将那银子称一称,十五两三钱四分二厘。
贾芸见倪二不撒谎,心下越发欢喜,\ping{贾芸害怕倪二口头上说借给自己十五两,但是实际上只交给十两,这样的话,即使倪二不要明面上的利息,等到贾芸还钱的时候,也是相当于收了暗地里五两利息。
}收了银子,来至家门,先到隔壁将倪二的信捎了与他娘子知道,方回家来。
见他母亲自在炕上拈线,见他进来,便问那去了一日。
贾芸恐他母亲生气,便不说起卜世仁的事来,\geng{孝子可敬。
此人后来荣府事败,必有一番作为。
}只说在西府里等琏二叔的,问他母亲吃了饭不曾。
他母亲已吃过了,说留的饭在那里。
小丫头子拿过来与他吃。
那天已是掌灯时候,贾芸吃了饭收拾歇息,一宿无话。
\par
次日一早起来,洗了脸,便出南门,大香铺里买了冰麝,便往荣国府来。
打听贾琏出了门,贾芸便往后面来。
\par
到贾琏院门前,只见几个小厮拿着大高笤帚在那里扫院子呢。
忽见周瑞家的从门里出来叫小厮们:“先别扫,奶奶出来了。
”贾芸忙上前笑问:“二婶婶那去?”周瑞家的道:“老太太叫,想必是裁什么尺头。
”正说着,只见一群人簇着凤姐出来了。
\geng{当家人有是派头。
}贾芸深知凤姐是喜奉承尚排场的,\geng{那一个不喜奉承。
}忙把手逼着,\zhu{把手逼着:逼:音“闭” 。
逼着:即并着。
两臂下垂,两手紧贴身体的两侧,以示敬畏的样子。
}恭恭敬敬抢上来请安。
凤姐连正眼也不看,仍往前走着,只问他母亲好,“怎么不来我们这里逛逛?”贾芸道:“只是身上不大好,倒时常记挂着婶子,要来瞧瞧,又不能来。
”凤姐笑道:“可是会撒谎,不是我提起他来,你就不说他想我了。
”贾芸笑道:“侄儿不怕雷打了,就敢在长辈前撒谎。
昨儿晚上还提起婶子来,说婶子身子生的单弱,事情又多,亏婶子好大精神,竟料理的周周全全,要是差一点儿的,早累的不知怎么样呢。
”\geng{自往卜世仁处去已安排下的。
芸哥可用。
己卯冬夜。
}\par
凤姐听了满脸是笑,不由的便止了步,问道:“怎么好好的你娘儿们在背地里嚼起我来?”\geng{过下无痕,天然而来文字。
}贾芸道:“有个原故,\geng{接得如何?}只因我有个朋友,家里有几个钱,现开香铺。
只因他身上捐着个通判,\zhu{通判:明、清时为协助知府处理政务的官。
}前儿选了云南不知那一处,\geng{随口语,极妙!}连家眷一齐去,把这香铺也不在这里开了。
便把帐物攒了一攒,\zhu{攒:凑聚。
这里指把现有的家财凑一凑。
}该给人的给人,该贱发的贱发了,
\zhu{贱发:低价发卖。}
\meng{世法人情,随便招来,皆是奇妙文章。
}像这细贵的货,都分着送与亲朋。
他就一共送了我些冰片、麝香。
我就和我母亲商量,\geng{像得紧,何尝撒谎?}若要转卖,不但卖不出原价来,而且谁家拿这些银子买这个作什么,便是很有钱的大家子,也不过使个几分几钱就挺折腰了,\zhu{挺折腰:这里是到顶的意思。
}若说送人,也没个人配使这些,\meng{作者是何神圣,具此等大光明眼,无微不照?}倒叫他一文不值半文转卖了。
因此我就想起婶子来。
\meng{为大千世界一哭。
}往年间我还见婶子大包的银子买这些东西呢,别说今年贵妃宫中,就是这个端阳节下,
\zhu{端阳节:端午节。}
不用说这些香料自然是比往常加上十倍去的。
\zhu{十倍:指需求量增加十倍。
}
因此想来想去,只孝顺婶子一个人才合式,\zhu{合式:同“合适”。
}\meng{有此一番必当孝顺、必当收下、必得备用之情景,行文\sout{妙}[好]看杀人,立意\sout{稀}[奚]落杀人,\zhu{杀人:煞人。
}看至此不知当哭当笑。
}方不算遭塌这东西。
”一边说,一边将一个锦匣举起来。
\par
凤姐正是要办端阳的节礼,采买香料药饵的时节,忽见贾芸如此一来,听这一篇话,心下又是得意又是欢喜,\meng{逼真。
}便命丰儿:“接过芸哥儿的来,\geng{像个婶子口气,好看杀!}送了家去,交给平儿。
”因又说道:“看着你这样知好歹,怪道你叔叔常提你,说你说话儿也明白,心里有见识。
”\geng{看官须记,凤姐所喜是奉承之言,打动了心,不是见物而欢喜,若说是见物而喜,便不是阿凤矣。
}贾芸听这话入了港,\zhu{入港:说话投机。
}便打进一步来,故意问道:“原来叔叔也曾提我的?”\ping{贾芸之前已经和贾琏说了,不要让贾琏告诉凤姐,方便自己绕开不中用的贾琏,直接找凤姐的门路走关系。
如果让凤姐知道自己是求贾琏无果才想起求她办事,会给凤姐留下一开始看不上自己的办事能力的印象,可能惹恼要强的凤姐。
}凤姐见问,才要告诉他与他管事情的那话,便忙又止住,心下想道:\geng{的是阿凤行事心机笔意。
}“我如今要告诉他那话,倒叫他看着我见不得东西似的,为得了这点子香,就混许他管事了。
今儿先别提起这事。
”想毕,便把派他监种花木工程的事都隐瞒的一字不提,随口说了两句淡话,\zhu{淡:没有意味的,无关紧要的。
“淡话”、“淡事”、“扯淡”中的“淡”字都是这个意思。
}便往贾母那里去了。
贾芸也不好提的,只得回来。
\par
因昨日见了宝玉,叫他到外书房等着,\meng{一样叔婶,两般侍奉。
}贾芸吃了饭便又进来,到贾母那边仪门外绮霰斋书房里来。
只见茗烟\foot{原作“焙茗”。
宝玉的小厮“茗烟”,从本回至第三十四回被改称“焙茗”,第三十九回以后又恢复“茗烟”。
现统一为“茗烟”。
}、锄药两个小厮下象棋,为夺“车”正拌嘴,还有引泉、扫花、\geng{好名色。
}
挑云、伴鹤四五个,又在房檐上掏小雀儿玩。
\meng{行云流[水],一字不空。
真是空灵活跳。
}\par
贾芸进入院内,把脚一跺,说道:“猴头们淘气,我来了。
”众小厮看见贾芸进来,都才散了。
贾芸进入房内,便坐在椅子上问:“宝二爷没下来?”茗烟道:“今儿总没下来。
二爷说什么,我替你哨探哨探去。
”\geng{五遁之外,名曰“哨探遁”法。
\zhu{哨探遁:以打探消息为借口,跑路溜掉。
}}说着,便出去了。
这里贾芸便看字画古玩,有一顿饭工夫还不见来,再看看别的小厮,都顽去了。
正是烦闷,只听门前娇声嫩语的叫了一声“哥哥”。
\ping{这声“哥哥”肯定不是叫宝玉,因为仆人叫宝玉一般都是“二爷”,也不是叫贾芸,因为下文小红看到贾芸还要抽身躲了过去,可能是叫宝玉屋里的小厮。
}\par
贾芸往外瞧时,看是一个十六七岁的丫头,生的倒也细巧干净。
那丫头见了贾芸,便抽身躲了过去。
\meng{是必然之理。
}恰值茗烟走来,见那丫头在门前,便说道:“好,好,\geng{二“好”字是遮饰半句来不到语。
}正抓不着个信儿。
”贾芸见了茗烟,也就赶了出来,问怎么样。
茗烟道:“等了这一日,也没个人儿过来。
\ping{茗烟嘴上说自己是去传话,其实并没有,传话是一个溜掉的借口,所以脂评有“哨探遁”的说法。
}这就是宝二爷房里的。
好姑娘,\geng{口气极像。
}你进去带个信儿,就说廊上的二爷来了。
”那丫头听说,方知是本家的爷们,便不似先前那等回避,\geng{一句,礼当。
}下死眼把贾芸钉了两眼。
\geng{这句是情孽上生。
}\meng{五百年风流孽冤。
}\zhu{钉:可能是“盯”的错讹,也可能是用这个字反映小红盯得程度之深。
}听那贾芸说道:“什么是廊上廊下的,你只说是芸儿就是了。
”半晌,那丫头冷笑了一笑:\geng{神情是深知房中事的。
}“依我说,二爷竟请回家去,有什么话明儿再来。
今儿晚上得空儿我回了他。
”茗烟道:“这是怎么说?”那丫头道:“他\geng{一连两个“他”字,怡红院中使得,否则有假矣。
\ping{怡红院内,宝玉和仆人平等相待。
}}今儿也没睡中觉,自然吃的晚饭早。
晚上他又不下来。
难道只是耍的二爷在这里等着挨饿不成!\meng{业已种下爱根,俟后无计可拔。
}不如家去,明儿来是正经。
便是回来有人带信,那都是不中用的。
他不过口里应着,他倒给带呢!”\ping{小红讽刺茗烟嘴上说去带话,其实自己溜掉了,根本没去,“哨探遁”。
}贾芸听这丫头说话简便俏丽,待要问他的名字,因是宝玉房里的,又不便问,只得说道:“这话倒是,我明儿再来。
”说着便往外走。
茗烟道:“我倒茶去,\geng{滑贼。
}二爷吃了茶再去。
”贾芸一面走,一面回头说:“不吃茶,我还有事呢。
”口里说话,眼睛瞧那丫头还站在那里呢。
\par
那贾芸一径回家。
至次日来至大门前,可巧遇见凤姐往那边去请安,才上了车,见贾芸来,便命人唤住,隔窗子笑道:“芸儿,你竟有胆子在我的跟前弄鬼。
\geng{也作的不像撒谎,用心机人可怕是此等处。
\zhu{凤姐装作刚知道贾芸昨天送礼的真实目的,令人看不出在撒谎。}
}怪道你送东西给我,原来你有事求我。
昨儿你叔叔才告诉我说你求他。
”\meng{非此等话法,则是因昨日之物起见了。
\zhu{起见:常与介词“为”构成“为……起见”格式,表示出于某种原因或愿望。
}锦心绣口,真真拜服。
}贾芸笑道:“求叔叔这事,婶子休提,我昨儿正后悔呢。
早知这样,我竟一起头求婶子,这会子也早完了。
谁承望叔叔竟不能的。
”\meng{这样话实是以非理加之,
\zhu{非理:不合常理,违背情理;不讲道理。}
而世人大都乐爱喜闻,吾深怪之。
}凤姐笑道:“怪道你那里没成儿,\zhu{没成儿:即没指望。
}昨儿又来寻我。
”贾芸道:“婶子辜负了我的孝心,我并没有这个意思。
若有这个意思,昨儿还不求婶子?如今婶子既知道了,我倒要把叔叔丢下,少不得求婶子好歹疼我一点儿。
”凤姐冷笑道:“你们要拣远路儿走,叫我也难说。
\geng{曹操语。
}早告诉我一声儿,有什么不成的,多大点子事,耽误到这会子。
那园子里还要种花,我只想不出一个人来,你早来不早完了。
”贾芸笑道:“既这样,婶子明儿就派我罢。
”凤姐半晌道:“这个我看着不大好。
\geng{又一折。
}等明年正月里烟火灯烛那个大宗儿下来,再派你罢。
”贾芸道:“好婶子,先把这个派了我罢。
果然这个办的好,再派我那个。
”凤姐笑道:“你倒会拉长线儿。
罢了,要不是你叔叔说,我不管你的事。
\geng{总不认受冰麝贿。
}我也不过吃了饭就过来,你到午错的时候来领银子,\zhu{午错:中午已过。
}后儿就进去种树。
”说毕,令人驾起香车,一径去了。
\par
贾芸喜不自禁,来至绮霰斋打听宝玉,谁知宝玉一早便往北静王府里去了。
贾芸便呆呆的坐到晌午,打听凤姐回来,便写个领票来领对牌。
至院外,命人通报了,彩明走了出来,单要了领票进去,批了银数年月,一并连对牌交与了贾芸。
贾芸接了,看那批上银数批了二百两,心中喜不自禁,翻身走到银库上,交与收牌票的,领了银子。
回家告诉母亲,自是母子俱各欢喜。
次日一个五鼓,贾芸先找了倪二,将前银按数还他。
那倪二见贾芸有了银子,他便按数收回,不在话下。
这里贾芸又拿了五十两,出西门找到花儿匠方椿家里去买树,不在话下。
\geng{至此便完种树工程。
}\geng{一者见得趱赶工程原非正文,\zhu{趱[zǎn]:赶。
}不过虚描盛时光景,借此以出情文。
二者又为避难法。
若不如此了,必曰其树其价、怎么买、定几株,岂不烦絮矣?}\ping{贾府财务管理漏洞很大,如果能够成为贾府买办,可以捞到很大油水,难怪贾芹贾芸这些贾府子弟都想要谋个职务。二百两银子的预算,只用掉五十两,剩下的一百五十两中饱私囊。
}\par
如今且说宝玉,自那日见了贾芸,曾说明日着他进来说话儿。
如此说了之后,他原是富贵公子的口角,那里还把这个放在心上,因而便忘怀了。
\geng{若是一个女孩子,可保不忘的。
}这日晚上,从北静王府里回来,见过贾母,王夫人等,回至园内,换了衣服,正要洗澡。
袭人因被薛宝钗烦了去打结子,\zhu{结子:用丝绳或绦带编结成各种花样,用以系挂珠玉等饰物,下有长穗。
}秋纹、碧痕两个去催水,檀云又因他母亲的生日接了出去,麝月又现在家中养病,虽还有几个作粗活听唤的丫头,估着叫不着他们,都出去寻伙觅伴的玩去了。
不想这一刻的工夫,\geng{妙!必用“一刻”二字方是宝玉的房中,见得时时原有人的,又有今一刻无人,所谓凑巧其一也。
}
只剩了宝玉在房内。
偏生的\geng{三字不可少。
}宝玉要吃茶,一连叫了两三声,方见两三个老嬷嬷走进来。
\geng{妙!文字细密,一丝不落,非批得出者。
}
宝玉见了他们,连忙摇手儿说:“罢,罢,不用你们了。
”\geng{是宝玉口气。
}老婆子们只得退出。
\par
宝玉见没丫头们,只得自己下来,拿了碗向茶壶去倒茶。
只听背后说道:“二爷仔细烫了手,让我们来倒。
”\geng{神龙变化之文,人岂能测?}
一面说,一面走上来,早接了碗过去。
宝玉倒唬了一跳,问:“你在那里的?忽然来了,唬我一跳。
”那丫头一面递茶,一面回说:“我在后院子里,才从里间的后门进来,难道二爷就没听见脚步响?”宝玉一面吃茶,一面\geng{六个“一面”,是神情,并不觉厌。
}仔细打量那丫头:穿着几件半新不旧的衣裳,倒是一头黑鬒鬒的头发,\zhu{
鬒:音“诊”,黑发。
黑鬒鬒:这里形容头发乌黑。
}挽着个纂,\zhu{纂[zuǎn]:妇女的发髻。
}容长脸面,
\zhu{容长脸:犹言长方脸。}
细巧身材,却十分俏丽干净。
\geng{与贾芸目中所见不差。
}宝玉看了,便笑问道:\geng{神情写得出。
}“你也是我这屋里的人么?”\geng{妙问。
必如此问方是笼络前文。
}那丫头道:“是的。
”宝玉道:“既是这屋里的,我怎么不认得?”那丫头听说,便冷笑了一声道:\geng{神情如画。
}“认不得的也多,岂只我一个。
从来我又不递茶递水,拿东拿西,眼见的事一点儿不作,那里认得呢。
”宝玉道:“你为什么不作那眼见的事?”\geng{这是下情不能上达意语也。
}那丫头道:“这话我也难说。
\geng{不伏气语,况非尔可完,\zhu{非尔可完:不这样说,没办法完结这个话题。
}故云“难说”。
}只是有一句话回二爷:昨儿有个什么芸儿来找二爷。
我想二爷不得空儿,便叫茗烟回他,叫他今日早起来,不想二爷又往北府里去了。
”刚说到这句话,只见秋纹、碧痕嘻嘻哈哈的说笑着进来,两个人共提着一桶水,一手撩着衣裳,趔趔趄趄、泼泼撒撒的。
那丫头便忙迎去接。
\geng{好!有眼色。
}那秋纹、碧痕正对着抱怨,“你湿了我的裙子”,那个又说“你踹了我的鞋”。
忽见走出一个人来接水,二人看时,不是别人,原来是小红。
二人便都诧异,将水放下,忙进房来东瞧西望,\geng{四字渐露大丫头素日怡红细事也。
}\geng{怡红细事俱用带笔白描,\zhu{
带笔白描是指绘画作点作线时,顺笔把画中局部形体的线条勾勒出来,这是中国画创作特有的技法风格和用笔方法。
评点者将绘画中的带笔白描技法直接运用于小说评点文本中的人物形象塑造,在精心描绘宝玉的时候,顺笔通过小红的谈话勾勒一下怡红院丫鬟之间的关系及宝玉和丫鬟之间的关系。
这种手法的运用不仅省去了不必要的笔墨,而且对人物形象典型性格的塑造及其背景布局的描画也有着重要的价值。
}是大章法也。
丁亥夏。
畸笏叟。
}并没个别人,只有宝玉,便心中大不自在。
只得预备下洗澡之物,待宝玉脱了衣裳,二人便带上门出来,\geng{清楚之至。
}走到那边房内便找小红,问他方才在屋里说什么。
小红道:“我何曾在屋里的?只因我的手帕子不见了,往后头找手帕子去。
不想二爷要茶吃,叫姐姐们一个没有,是我进去了,才倒了茶,姐姐们便来了。
”秋纹听了,兜脸啐了一口,骂道:“没脸的下流东西!正经叫你去催水去,你说有事故,倒叫我们去,你可等着做这个巧宗儿。
\geng{难说小红无心,白描。
}\ping{前文小红进屋,宝玉没有听到,这是小红刻意为之,意图即为趁机接近宝玉。
}一里一里的,\zhu{一里一里:一步一步,得寸进尺。
}这不上来了。
难道我们倒跟不上你了?你也拿镜子照照,配递茶递水不配!”\geng{“难说”二字全在此句来。
}\ping{丫头本身是贾府的二等公民,但是内部也是层级分明。
}\ping{单写都是可爱女孩子,凑一起还是踩来踩去,争斗不止,仰仗主子嘴边掉下来那么点渣滓,一个个就上赶着要。
}碧痕道:“明儿我说给他们,凡要茶要水送东送西的事,咱们都别动,只叫他去便是了。
”秋纹道:“这么说,不如我们散了,单让他在这屋里呢。
”二人你一句我一句,正闹着,只见有个老嬷嬷进来传凤姐的话说:“明日有人带花儿匠来种树,叫你们严禁些,衣服裙子别混晒混晾的。
那土山上一溜都拦着帏幕呢,可别混跑。
”秋纹便问:\geng{用秋纹问,是暗透之法。
}“明儿不知是谁带进匠人来监工?”那婆子道:“说什么后廊上的芸哥儿。
”秋纹、碧痕听了都不知道,只管混问别的话。
那小红听见了,\geng{可是暗透法?\zhu{暗透法:指一个人物内心所想的事通过另外的人物不经意地说出来。秋纹和婆子的问答,正是说出了小红心中的芸哥儿。}}心内却明白,就知是昨儿外书房所见那人了。
\par
原来这小红本姓林,\geng{又是个林。
}小名红玉,\geng{“红”字切“绛珠”,“玉”字则直通矣。
\ping{暗指林黛玉。
}}只因“玉”字犯了林黛玉、宝玉,\geng{妙文。
}便都把这个字隐起来,便都叫他“小红”。
原是荣国府中世代的旧仆,他父母现在收管各处房田事务。
这红玉年方十六岁,因分人在大观园的时节,把他便分在怡红院中,倒也清幽雅静。
不想后来命人进来居住,偏生这一所儿又被宝玉占了。
这红玉虽然是个不谙事的丫头,却因他有三分容貌,\geng{有三分容貌尚且不肯受屈,况黛玉等一干才貌者乎?}心内着实妄想痴心的往上攀高,\geng{争夺者同来一看。
}每每的要在宝玉面前现弄现弄。
只是宝玉身边一干人,都是伶牙利爪的,\geng{“难说”的原故在此。
}那里插的下手去。
不想今儿才有些消息,\geng{余前批不谬。
}又遭秋纹等一场恶意,心内早灰了一半。
\geng{争名夺利者齐来一哭。
}正闷闷的,忽然听见老嬷嬷说起贾芸来,不觉心中一动,便闷闷的回至房中,睡在床上暗暗盘算,翻来掉去,正没个抓寻。
忽听窗外低低的叫道:“红玉,你的手帕子我拾在这里呢。
”红玉听了忙走出来看,不是别人,正是贾芸。
红玉不觉的粉面含羞,问道:“二爷在那里拾着的?”贾芸笑道:“你过来,我告诉你。
”一面说,一面就上来拉他。
那红玉急回身一跑,却被门槛绊倒。
\geng{睡梦中当然一跑,这方是怡红之鬟。
\ping{小红知礼。
}}要知端的,下回分解。
\par
\geng{《红楼梦》写梦章法总不雷同。
此梦更写的新奇,不见后文,不知是梦。
\hang
红玉在怡红院为诸环所掩,亦可谓生不遇时,但看后四章供阿凤驱使可知。
}\par
\qi{总评:冷暖时,只自知,金刚、卜氏浑闲事。
眼中心,言中意,三生旧债原无底。
\zhu{
三生旧债:用唐代李源和圆观的典故,圆观转世后成牧童,对前生好友李源唱歌,首句即“三生石上旧精魂”。
用在这里感叹贾芸和小红的一见钟情。
}
任你贵比王侯,任你富似郭、石,\zhu{
郭:汉朝大侠郭解,“铸钱掘冢,固不可胜数”。石:西晋的石崇,与王恺斗富,富甲天下。
}一时间,风流愿,不怕死!}
\dai{047}{醉金刚轻财尚义侠}
\dai{048}{秋纹碧痕骂小红}
\sun{p24-1}{牡丹亭艳曲警芳心,宝玉出门遇见贾芸,醉金刚轻财尚义侠,初见贾芸小红动心}{图右上:黛玉听到从梨香院传来的《牡丹亭》戏词,情思萦逗,缠绵不已,忽有人从背后拍了一下,回头看时,却是香菱。
图右下:宝玉听贾母说贾赦病了,正备马前去探望,在门前遇到了贾琏和找贾琏谋差使的贾芸。
宝玉见贾芸斯文清秀,心里喜欢,笑道:“明儿你闲了,只管来找我。
”图上侧中部:贾芸从醉金刚倪二那里借了钱买了香料谋到了差事,图左侧:来看宝玉时,虽没有见到宝玉,却意外遇到了小红。
那小红长得细挑身材,俏丽甜净,自见了贾芸后,不觉心中一动。
}