\chapter{林黛玉重建桃花社 \quad 史湘云偶填柳絮词}
\qi{空将佛事图相报,\zhu{佛事:宝玉“悬崖撒手”、“弃而为僧”出家之事。
图相报:第三十回:林黛玉道:“我死了。
”宝玉道:“你死了,我做和尚!”第三十一回:林黛玉笑道:“……我先就哭死了。
”宝玉笑道:“你死了,我作和尚去。
”……林黛玉将两个指头一伸,抿嘴笑道:“作了两个和尚了。
我从今以后都记着你作和尚的遭数儿。
”“图相报”的意思是,宝玉在黛玉死后出家为僧以实现自己对黛玉爱情的承诺。
}已触飘风散艳花。
一片精神传好句,题成谶语任吁嗟!}\par
话说贾琏自在梨香院伴宿七日夜,天天僧道不断做佛事。
贾母唤了他去,吩咐不许送往家庙中。
贾琏无法,只得又和时觉说了,\zhu{时觉:给尤二姐亡灵做佛事的和尚。
}就在尤三姐之上点了一个穴,\zhu{点穴:阴阳生或僧道为死者选定墓穴的位置。
上:位置在高处的。
}破土埋葬。
那日送殡,只不过族中人与王信夫妇、尤氏婆媳而已。
凤姐一应不管,只凭他自去办理。
\par
因又年近岁逼,诸务猬集不算外,\zhu{猬集:比喻众多,如猬毛丛聚。
}又有林之孝开了一个人名单子来,共有八个二十五岁的单身小厮应该娶妻成房,等里面有该放的丫头们好求指配。
凤姐看了,先来问贾母和王夫人。
大家商议,虽有几个应该发配的,奈各人皆有原故:第一个鸳鸯发誓不去。
自那日之后,一向未和宝玉说话,也不盛妆浓饰。
众人见他志坚,也不好相强。
第二个琥珀,又有病,这次不能了。
彩云因近日和贾环分崩,也染了无医之症。
\ping{彩云退场,简写省笔。
}只有凤姐儿和李纨房中粗使的大丫鬟出去了,其馀年纪未足。
令他们外头自娶去了。
\par
原来这一向因凤姐病了,李纨探春料理家务不得闲暇,接着过年过节,出来许多杂事,竟将诗社搁起。
如今仲春天气,虽得了工夫,争奈宝玉因冷遁了柳湘莲,\zhu{争奈:怎奈。
}剑刎了尤小妹,金逝了尤二姐,气病了柳五儿,连连接接,闲愁胡恨,一重不了一重添。
弄得情色若痴,语言常乱,似染怔忡之疾。
\zhu{怔忡:中医指心悸,表现为心脏剧烈跳动,感到很不舒服。
}慌的袭人等又不敢回贾母,只百般逗他顽笑。
\par
这日清晨方醒,只听外间房内咭咭呱呱笑声不断。
袭人因笑说:“你快出去解救,晴雯和麝月两个人按住温都里那膈肢呢。
”宝玉听了,忙披上灰鼠袄子出来一瞧,只见他三人被褥尚未叠起,大衣也未穿。
那晴雯只穿葱绿院绸小袄,\zhu{院绸:濮院绸的简称。
濮院(今作卜院)在浙江嘉兴县西南,以产素绸、花绸等著称。
见《嘉庆一统志》二八八《嘉兴府》。
}红小衣红睡鞋,\zhu{小衣:内裤。
睡鞋:缠足女子睡觉时所穿之鞋。
徐珂《清稗类钞·服饰类》:“睡鞋,缠足妇女所著以就寝者。
盖非此,则行缠必弛,且借以使恶臭不外泄也。
”}披着头发,骑在雄奴身上。
麝月是红绫抹胸,\zhu{抹胸:挂束在胸腹间的贴身小衣,只盖住胸、肚,无袖,无后背。
徐珂《清稗类钞·服饰类》:“抹胸,胸间小衣也。
一名抹腹,又名抹肚。
以尺方之布为之,紧束前胸,以防风之内侵者。
俗谓之兜肚,男女皆有之。
”}披着一身旧衣,在那里抓雄奴的肋肢。
雄奴却仰在炕上,穿着撒花紧身儿,\zhu{撒花:衣面上状如撒花之纹饰。
紧身儿:背心。
}红裤绿袜,两脚乱蹬,笑的喘不过气来。
宝玉忙上前笑说:“两个大的欺负一个小的,等我助力。
”说着,也上床来膈肢晴雯。
晴雯触痒,笑的忙丢下雄奴,和宝玉对抓。
雄奴趁势又将晴雯按倒,向他肋下抓动。
袭人笑说:“仔细冻着了。
”看他四人裹在一处倒好笑。
\par
忽有李纨打发碧月来说:“昨儿晚上奶奶在这里把块手帕子忘了,不知可在这里?”小燕说:“有,有,有,我在地下拾了起来,不知是那一位的,才洗了出来晾着,还未干呢。
”碧月见他四人乱滚,因笑道:“倒是这里热闹,大清早起就咭咭呱呱的顽到一处。
”宝玉笑道:“你们那里人也不少,怎么不顽?”碧月道:“我们奶奶不顽,把两个姨娘和琴姑娘也宾住了。
\zhu{两个姨娘:李纹李绮。
宾住:拘束住的意思。
}如今琴姑娘又跟了老太太前头去了,更寂寞了。
两个姨娘今年过了,到明年冬天都去了,又更寂寞呢。
你瞧宝姑娘那里,出去了一个香菱,就冷清了多少,把个云姑娘落了单。
”\par
正说着,只见湘云又打发了翠缕来说:“请二爷快出去瞧好诗。
”宝玉听了,忙问:“那里的好诗?”翠缕笑道:“姑娘们都在沁芳亭上,你去了便知。
”宝玉听了,忙梳洗了出来,果见黛玉、宝钗、湘云、宝琴、探春都在那里,手里拿着一篇诗看。
见他来时,都笑说:“这会子还不起来,咱们的诗社散了一年,也没有人作兴。
\zhu{作兴:举办,使之兴盛。
}如今正是初春时节,万物更新,正该鼓舞另立起来才好。
”湘云笑道:“一起诗社时是秋天,就不应发达。
如今恰好万物逢春,皆主生盛。
况这首桃花诗又好,就把海棠社改作桃花社。
”\ji{起时是后有名,此是先有名。
}宝玉听着,点头说:“很好。
”且忙着要诗看。
众人都又说:“咱们此时就访稻香老农去,大家议定好起的。
”说着,一齐起来,都往稻香村来。
宝玉一壁走,一壁看那纸上写着《桃花行》一篇,曰:\par
\hop
桃花帘外东风软,桃花帘内晨妆懒。
\par
帘外桃花帘内人,人与桃花隔不远。
\par
东风有意揭帘栊,花欲窥人帘不卷。
\par
桃花帘外开仍旧,帘中人比桃花瘦。
\par
花解怜人花也愁,隔帘消息风吹透。
\par
风透湘帘花满庭,庭前春色倍伤情。
\par
闲苔院落门空掩,斜日栏杆人自凭。
\par
凭栏人向东风泣,茜裙偷傍桃花立。
\zhu{茜裙:茜纱裙。
指大红色的裙子。
茜:一种草根可作大红色染料的植物。
诗文中故多用以代称红色。
}\par
桃花桃叶乱纷纷,花绽新红叶凝碧。
\par
雾裹烟封一万株,烘楼照壁红模糊。
\par
天机烧破鸳鸯锦,\zhu{天机烧破鸳鸯锦:即烧破天机鸳鸯锦。
这里是形容盛开的桃花犹如天上的纹锦烧成碎片落到了人间一样。
天机:传说中天上仙女用的织机。
烧:极喻其红。
}春酣欲醒移珊枕。
\zhu{珊枕:珊瑚枕。}
\par
侍女金盆进水来,香泉影蘸胭脂冷。
\zhu{香泉影蘸胭脂冷:面容的倒影蘸在清冷的泉水中。
胭脂:代指少女的面容。
}\par
胭脂鲜艳何相类,花之颜色人之泪;\par
若将人泪比桃花,泪自长流花自媚。
\ping{滴不尽相思血泪抛红豆。}
\par
泪眼观花泪易干,泪干春尽花憔悴。
\par
憔悴花遮憔悴人,花飞人倦易黄昏。
\par
一声杜宇春归尽,\zhu{杜宇:即杜鹃,又名子规,传说是由古代蜀国国王望帝所化,至春则啼,其声凄切悲苦。
诗文中常用以描写哀怨、凄凉或思归的心情。
}寂寞帘栊空月痕!\par
\hop
宝玉看了并不称赞,却滚下泪来。
便知出自黛玉,因此落下泪来,又怕众人看见,又忙自己擦了。
\ping{诗为谶语,专为命薄如桃花的林黛玉的夭亡预作象征性的写照。}
因问:“你们怎么得来?”宝琴笑道:“你猜是谁做的?”宝玉笑道:“自然是潇湘子稿。
”宝琴笑道:“现是我作的呢。
”宝玉笑道:“我不信。
这声调口气,迥乎不像蘅芜之体,所以不信。
”宝钗笑道:“所以你不通。
难道杜工部首首只作‘丛菊两开他日泪’之句不成!一般的也有‘红绽雨肥梅’、‘水荇牵风翠带长’之媚语。
”\zhu{荇:音“幸”,多年生水草。
难道杜工部……之媚语:杜工部,即杜甫。
这里薛宝钗是说杜诗风格沉郁,但非首首如此,也有清灵明媚的句子。
}宝玉笑道:“固然如此说。
但我知道姐姐断不许妹妹有此伤悼语句,妹妹虽有此才,是断不肯作的。
比不得林妹妹曾经离丧,作此哀音。
”众人听说,都笑了。
\par


已至稻香村中,将诗与李纨看了,自不必说称赏不已。
说起诗社,大家议定:明日乃三月初二日,就起社,便改“海棠社”为“桃花社”,林黛玉就为社主。
明日饭后,齐集潇湘馆。
因又大家拟题。
黛玉便说:“大家就要桃花诗一百韵。
”宝钗道:“使不得。
从来桃花诗最多,纵作了必落套,比不得你这一首古风。
须得再拟。
”正说着,人回:“舅太太来了。
姑娘出去请安。
”因此大家都往前头来见王子腾的夫人,陪着说话。
吃饭毕,又陪入园中来,各处游顽一遍。
至晚饭后掌灯方去。
\par
次日乃是探春的寿日,元春早打发了两个小太监送了几件玩器。
合家皆有寿仪,自不必说。
饭后,探春换了礼服,各处去行礼。
黛玉笑向众人道:“我这一社开的又不巧了,偏忘了这两日是他的生日。
虽不摆酒唱戏的,少不得都要陪他在老太太、太太跟前顽笑一日,如何能得闲空儿。
”因此改至初五。
\par
这日众姊妹皆在房中侍早膳毕,便有贾政书信到了。
宝玉请安,将请贾母的安禀拆开念与贾母听,\zhu{禀:旧时指下对上报告。
安禀:写给尊长的平安书信。
}上面不过是请安的话,说六月中准进京等语。
其馀家信事务之帖,自有贾琏和王夫人开读。
众人听说六七月回京,都喜之不尽。
偏生近日王子腾之女许与保宁侯之子为妻,择日于五月初十日过门,凤姐儿又忙着张罗,常三五日不在家。
这日王子腾的夫人又来接凤姐儿,一并请众甥男甥女闲乐一日。
贾母和王夫人命宝玉、探春、林黛玉、宝钗四人同凤姐去。
众人不敢违拗,只得回房去另妆饰了起来。
五人作辞,去了一日,掌灯方回。
宝玉进入怡红院,歇了半刻,袭人便乘机见景劝他收一收心,闲时把书理一理预备着。
宝玉屈指算一算说:“还早呢。
”袭人道:“书是第一件,字是第二件。
到那时你纵有了书,你的字写的在那里呢?”宝玉笑道:“我时常也有写的好些,难道都没收着?”袭人道:“何曾没收着。
你昨儿不在家,我就拿出来,共总数了一数,才有五六十篇。
这三四年的工夫,难道只有这几张字不成。
依我说,从明日起,把别的心全收了起来,天天快临几张字补上。
虽不能按日都有,也要大概看得过去。
”宝玉听了,忙的自己又亲检了一遍,实在搪塞不去,便说:“明日为始,一天写一百字才好。
”说话时大家安下。
至次日起来梳洗了,便在窗下研墨,恭楷临帖。
贾母因不见他,只当病了,忙使人来问。
宝玉方去请安,便说写字之故,先将早起清晨的工夫尽了出来,再作别的,因此出来迟了。
贾母听了,便十分欢喜,吩咐他:“以后只管写字念书,不用出来也使得。
你去回你太太知道。
”宝玉听说,便往王夫人房中来说明。
王夫人便说:“临阵磨枪,也不中用。
有这会子着急,天天写写念念,有多少完不了的。
这一赶,又赶出病来才罢。
”宝玉回说不妨事。
这里贾母也说怕急出病来。
探春宝钗等都笑说:“老太太不用急。
书虽替他不得,字却替得的。
我们每人每日临一篇给他,搪塞过这一步就完了。
一则老爷到家不生气,二则他也急不出病来。
”贾母听说,喜之不尽。
\par
原来林黛玉闻得贾政回家,必问宝玉的功课,宝玉肯分心,\zhu{肯:表示时常、易于。
}恐临期吃了亏。
因此自己只装作不耐烦,把诗社便不起,也不以外事去勾引他。
探春宝钗二人每日也临一篇楷书字与宝玉,宝玉自己每日也加工,或写二百三百不拘。
至三月下旬,便将字又集凑出许多来。
这日正算,再得五十篇,也就混的过了。
谁知紫鹃走来,送了一卷东西与宝玉,拆开看时,却是一色老油竹纸上临的钟王蝇头小楷,\zhu{
老油竹纸:竹纸乃以竹材为原料所制成的纸张,纸薄而韧,透明度高,蒙在字画上临摹用。“老油”疑即带淡黄色之谓。
钟王:指三国时魏的钟繇和晋代的王羲之,都是大书法家,被历代推尊为楷、行书法之祖。
蝇头:比喻小字。
}字迹且与自己十分相似。
喜的宝玉和紫鹃作了一个揖,又亲自来道谢。
史湘云宝琴二人亦皆临了几篇相送。
凑成虽不足功课,亦足搪塞了。
宝玉放了心,于是将所应读之书,又温理过几遍,正是天天用功。
可巧近海一带海啸,又遭蹋了几处生民。
地方官题本奏闻,奉旨就着贾政顺路查看赈济回来。
\zhu{着:命令、差使。
}如此算去,至冬底方回。
宝玉听了,便把书字又搁过一边,仍是照旧游荡。
\par
时值暮春之际,史湘云无聊,因见柳花飘舞,便偶成一小令,\zhu{小令:词中体制短小的称“小令”,同“中调”、“长调”相对而言。
一般五十八字以内为小令,五十九至九十字为中调,九十字以上为长调。
下文的《如梦令》、《临江仙》等都是词牌的名称。
词牌本指词的曲调名,相当于歌谱,最初一般据词的内容而定。后来不再配乐歌唱,不一定与内容有关联,只作为文字、音韵结构的定式。每个词牌都有一个名称。也说词调。
}调寄《如梦令》,\zhu{调寄:以某词牌作为填词的曲调。
}其词曰:\par
\par
岂是绣绒残吐,\zhu{绣绒残吐:指吐绒。
古代妇女刺绣,每当换线停针,用齿咬断绣线,口中常沾留线绒,随口吐出,俗谓吐绒。
这里喻柳絮。
}卷起半帘香雾,\zhu{
香雾:亦喻柳絮。
}纤手自拈来,空使鹃啼燕妒。
\zhu{这二句意谓纤手虽然拈得柳絮占得了春光,所以说惹得春鸟产生妒忌。
}
且住,且住!莫使春光别去。
\zhu{别去:“别”、“去”同义复合。
}\par
\ping{
“纤手自拈来”喻指史湘云和卫若兰以金麒麟为信物巧结良姻,美满婚姻使得春鸟都嫉妒。
但是转眼春光别去难留,预示美满婚姻好景不长。
}
\par
\hop
自己作了,心中得意,便用一条纸儿写好,与宝钗看了,又来找黛玉。
黛玉看毕,笑道:“好,也新鲜有趣。
我却不能。
”湘云笑道:“咱们这几社总没有填词。
你明日何不起社填词,改个样儿,岂不新鲜些。
”黛玉听了,偶然兴动,便说:“这话说的极是。
我如今便请他们去。
”说着,一面吩咐预备了几色果点之类,一面就打发人分头去请众人。
这里他二人便拟了柳絮之题,又限出几个调来,写了绾在壁上。
\zhu{绾:音“碗”,系。
}\par
众人来看时,以柳絮为题,限各色小调。
又都看了史湘云的,称赏了一回。
宝玉笑道:“这词上我们平常,少不得也要胡诌起来。
”于是大家拈阄,宝钗便拈得了《临江仙》,宝琴拈得《西江月》,探春拈得了《南柯子》,黛玉拈得了《唐多令》,宝玉拈得了《蝶恋花》。
紫鹃炷了一支梦甜香,\ji{重建,故又写香。
}大家思索起来。
一时黛玉有了,写完。
接着宝琴宝钗都有了。
他三人写完,互相看时,宝钗便笑道:“我先瞧完了你们的,再看我的。
”探春笑道:“嗳呀,今儿这香怎么这样快,已剩了三分了。
我才有了半首。
”因又问宝玉可有了。
宝玉虽作了些,只是自己嫌不好,又都抹了,要另作,回头看香,已将烬了。
李纨笑道:“这算输了。
蕉丫头的半首且写出来。
”探春听说,忙写了出来。
众人看时,\ji{却是先看没作完的,总是又变一格也。
}上面却只半首《南柯子》,写道是:\par
\hop
空挂纤纤缕,徒垂络络丝,也难绾系也难羁,
\zhu{柳条虽然如缕如丝,却难系住柳絮。}
一任东西南北各分离。
\ping{败落流散之谶语。
}\par
\hop
李纨笑道:“这也却好作,何不续上?”宝玉见香没了,情愿认负,不肯勉强塞责,将笔搁下,来瞧这半首。
见没完时,反倒动了兴开了机,乃提笔续道是:\par
\hop
落去君休惜,飞来我自知。
莺愁蝶倦晚芳时,\zhu{晚芳时:指暮春时节。
}纵是明春再见隔年期!\zhu{这一句意谓纵然明春还可再见,但须相隔一年。
期:约;相会。
}\par
\ping{
用柳絮纷飞难系,喻探春远嫁不归;用柳条绾系柳絮不得,喻亲人对探春徒然牵挂悬念。
宝玉续后半首,探春远嫁分离的比喻同样适用于宝黛分离。
“落去”喻黛玉逝去,“晚芳时”喻“红颜老死”之日。
“隔年期”可能指宝玉从避祸出走、流亡在外,到重回物是人非的大观园的时间。
}
\par
\hop
众人笑道:“正经你份内的又不能,这却偏有了。
纵然好,也不算得。
”说着,看黛玉的《唐多令》:\par
\hop
粉堕百花州,香残燕子楼。
\zhu{粉堕、香残:指残花零落,暗喻女子的衰老死亡。
百花州:这里指姑苏城(今苏州市)内的百花州。
传说吴王夫差常携西施泛舟游乐于此。
燕子楼:故址在今江苏徐州市西北。
唐太宗贞观年间,尚书张愔的爱妓关盼盼居住其中,愔死后,盼盼念旧情不嫁,居此楼十馀年。
这二句用“粉堕”、“香残”暗点柳絮飘落的晚春季节,借西施和关盼盼的故事表现作者林黛玉的孤寂悲愁情绪。
}一团团逐对成毬。
\zhu{毬:即“球”,用“毛”字偏旁特别突出了是绒线球,比喻成团的柳絮更贴切。
“毬”谐音“逑”,指配偶。
这句是双关语。
}飘泊亦如人命薄,空缱绻,\zhu{缱绻(音“浅犬”):犹言缠绵。
语出《诗经·大雅·民劳》,本为牢固相结之意。
后多用以形容情投意合、难舍难分的样子。
空:徒然,白白地。
这是指心事成空。
}说风流。
\zhu{风流:因柳絮随风飘流而用此词,这里指才华风度。
“风流”亦言儿女情事。
}\par 草木也知愁,韶华竟白头!叹今生谁舍谁收?
\zhu{戚序、程乙本作“谁舍谁收”,己卯、庚辰本作“谁拾谁收”,即以柳絮飘落,无人收拾自比。}
嫁与东风春不管,凭尔去,忍淹留。
\zhu{嫁与东风:被东风吹落的意思。
淹留:久留。
这三句意谓东风吹落了柳絮,春天竟不闻不问,任凭你随风飘荡,忍心看着你在外久留!}\par
\ping{
黛玉的词不但寄寓着她对自己不幸身世的深切哀愁,而且也有预感到爱情理想行将破灭而发自内心的悲愤呼声。
“草木也知愁,韶华竟白头”,不但以柳絮之色白,比人因悲愁而青春老死,完全切合自称“草木之人“的黛玉。
“凭尔去,忍淹留”暗示佚稿中宝玉出走不归、黛玉泪尽而逝的结局。
}
\par
\hop
众人看了,俱点头感叹,说:“太作悲了,好是固然好的。
”因又看宝琴的是《西江月》:\par
\hop
汉苑零星有限,隋堤点缀无穷。
\zhu{这里汉苑、隋堤皆暗寓“柳”字。
汉苑:指汉代的皇家宫苑,其中的长杨宫等处多植柳树,但规模远不及隋堤。
隋堤:隋炀帝杨广于大业元年强征河南、淮北各郡民上百馀万开通济渠,从洛阳直达江都。
渠宽四十步,渠旁筑“御道”,两岸种垂柳,世称隋堤。
}三春事业付东风,明月梅花一梦。
\par
几处落红庭院,
\zhu{落红:落花,表示春去。}
谁家香雪帘栊?\zhu{香雪:喻柳絮。
栊:窗户;房舍。
香雪帘栊:指沾满柳絮的门窗帘幕。
}江南江北一般同,偏是离人恨重!\zhu{离人恨重:意谓飘零的柳絮犹如漂泊的游人,深怀离愁别恨。
}\par
\ping{
薛宝琴嫁梅翰林之子,所以提到“梅花一梦”。
“三春”可能暗示迎春、探春、惜春三人。也可能是单指探春一人,“付东风”指探春远嫁如风筝随风飘走。
“离人恨重”正是薛宝琴未来的命运。
此外,从宝琴的个人萧索前景中,也反映出整个封建贵族阶级已到了风飘残絮、落红遍地的丧败境地了。
}
\par
\hop
众人都笑说:“到底是他的声调壮。
‘几处’‘谁家’两句最妙。
”宝钗笑道:“终不免过于丧败。
我想,柳絮原是一件轻薄无根无绊的东西,然依我的主意,偏要把他说好了,才不落套。
所以我诌了一首来,未必合你们的意思。
”众人笑道:“不要太谦。
我们且赏鉴,自然是好的。
”因看这一首,《临江仙》道是:\par
\hop
白玉堂前春解舞,\zhu{春解舞:
春能跳舞,这是说柳絮被春风吹散,像在翩翩起舞。
}东风卷得均匀。
\zhu{均匀:指舞姿优美,匀称有度。
}\par
\hop
湘云先笑道:“好一个‘东风卷得均匀’!这一句就出人之上了。
”又看底下道:\par
\hop
蜂团蝶阵乱纷纷。
\zhu{蜂团蝶阵:比喻柳絮纷飞繁乱。
}几曾随逝水,岂必委芳尘。
 \par 万缕千丝终不改,任他随聚随分。
\yang{人事无常,原不必戚戚也。
}韶华休笑本无根,好风频借力,\zhu{频:屡次的、接连的。
}送我上青云!\par
\ping{
“蜂围蝶阵乱纷纷”正是变故来临时,大观园纷乱情景的象征。
宝钗一向以高洁自持,“丑祸”当然不会沾惹到她的身上,何况她颇有处世的本领,能“任他随聚随分”而“终不改”,所以词中以“解舞”、“均匀”自诩。
黛玉就不同了,她不禁聚散的悲痛,就像落絮那样“随逝水”、“委芳尘”了。
“岂必委芳尘”表达了宝钗想违拗没落命运的愿望。
}
\par
\hop
众人拍案叫绝,都说:“果然翻得好。
气力自然,是这首为尊;缠绵悲戚,让潇湘妃子;情致妩媚,却是枕霞;小薛与蕉客今日落第,要受罚的。
”宝琴笑道:“我们自然受罚,但不知付白卷子的又怎么罚?”李纨道:“不要忙,这定要重重罚他。
下次为例。
”\par
\par
一语未了,只听窗外竹子上一声响,恰似窗屉子倒了一般,\zhu{窗屉:装置于窗上,可支起或放落的木架,上糊以纱或纸。
}众人唬了一跳。
丫鬟们出去瞧时,帘外丫鬟嚷道:“一个大蝴蝶风筝挂在竹梢上了。
”众丫鬟笑道:“好一个齐整风筝!不知是谁家放断了绳,拿下他来。
”宝玉等听了,也都出来看时,宝玉笑道:“我认得这风筝。
这是大老爷那院里娇红姑娘放的,拿下来给他送过去罢。
”紫鹃笑道:“难道天下没有一样的风筝,单他有这个不成?我不管,我且拿起来。
”探春道:“紫鹃也学小气了。
你们一般的也有,这会子拾人走了的,也不怕忌讳。
”黛玉笑道:“可是呢,知道是谁放晦气的,\zhu{放晦气:旧时迷信,放风筝时故意剪断扯线,让风筝飞走,认为可以放走坏运气叫“放晦气”。
}快掉出去罢。
\ping{别人的晦气掉到自己家里了。
}把咱们的拿出来,咱们也放晦气。
”紫鹃听了,赶着命小丫头们将这风筝送出与园门上值日的婆子去了,倘有人来找,好与他们去的。
\par
这里小丫头们听见放风筝,巴不得七手八脚都忙着拿出个美人风筝来。
也有搬高凳去的,也有捆剪子股的,\zhu{剪子股:放风筝时,在竹竿头上斜捆一根小木棍,做成剪刀形,以便挑线帮助放风筝。
}也有拨籰子的。
\zhu{籰(音“月”)子:缠丝、纱、线等用的工具,这里是指放风筝用的绕线的轮轴,把风筝线缠在上面,转动起来可放可收。
}宝钗等都立在院门前,命丫头们在院外敞地下放去。
宝琴笑道:“你这个不大好看,不如三姐姐的那一个软翅子大凤凰好。
”宝钗笑道:“果然。
”因回头向翠墨笑道:\zhu{翠墨是探春的丫鬟。
}“你把你们的拿来也放放。
”翠墨笑嘻嘻的果然也取去了。
宝玉又兴头起来,也打发个小丫头子家去,说:“把昨儿赖大娘送我的那个大鱼取来。
”小丫头子去了半天,空手回来,笑道:“晴姑娘昨儿放走了。
”宝玉道:“我还没放一遭儿呢。
”探春笑道:“横竖是给你放晦气罢了。
”宝玉道:“也罢。
再把那个大螃蟹拿来罢。
”丫头去了,同了几个人扛了一个美人并籰子来,说道:“袭姑娘说,昨儿把螃蟹给了三爷了。
这一个是林大娘才送来的,放这一个罢。
”宝玉细看了一回,只见这美人做的十分精致。
心中欢喜,便命叫放起来。
此时探春的也取了来,翠墨带着几个小丫头子们在那边山坡上已放了起来。
宝琴也命人将自己的一个大红蝙蝠也取来。
宝钗也高兴,也取了一个来,却是一连七个大雁的,都放起来。
独有宝玉的美人放不起去。
宝玉说丫头们不会放,自己放了半天,只起房高便落下来了。
急的宝玉头上出汗,众人又笑。
宝玉恨的掷在地下,指着风筝道:“若不是个美人,我一顿脚跺个稀烂。
”黛玉笑道:“那是顶线不好,\zhu{顶线:风筝中间成三角椎柱体拴长线的三根提线,顶线的位置、长度等是风筝能否放得上去的关键。
}拿出去另使人打了顶线就好了。
”宝玉一面使人拿去打顶线,一面又取一个来放。
大家都仰面而看,天上这几个风筝都起在半空中去了。
\par
\ping{
宝钗放的风筝是一连七个大雁,七是奇数,有落单之意,在此暗寓宝钗在婚后过着孤独的生活,如同一只孤雁,最后寡居而终。
宝钗的灯谜有“琴边衾里总无缘”也是暗寓薛宝钗与贾宝玉成婚后孤凄寡居。
此外,雁又能传书,在宝钗的《忆菊》诗中有“念念心随归雁远,寥寥坐听晚砧痴”一句。宝钗虽是位冷美人,但她的心中仍时时牵挂远方的夫君。
}
\par
一时丫鬟们又拿了许多各式各样的送饭的来,\zhu{送饭的:放风筝的一种附加物,俗呼为“送饭的”。
风筝放到空中以后,将它挂在线上,随风鼓起,沿线而上,有的上面系有爆竹在空中鸣响,有的则附有各种绚丽的彩饰。
}顽了一回。
紫鹃笑道:“这一回的劲大,姑娘来放罢。
”黛玉听说,用手帕垫着手,顿了一顿,果然风紧力大,接过籰子来,随着风筝的势将籰子一松,只听一阵豁刺刺响,登时籰子线尽。
黛玉因让众人来放。
众人都笑道:“各人都有,你先请罢。
”黛玉笑道:“这一放虽有趣,只是不忍。
”李纨道:“放风筝图的是这一乐,所以又说放晦气,你更该多放些,把你这病根儿都带了去就好了。
”
\ping{可能暗示黛玉从放风筝后身体好转。}
紫鹃笑道:“我们姑娘越发小气了。
那一年不放几个子,今忽然又心疼了。
姑娘不放,等我放。
”说着便向雪雁手中接过一把西洋小银剪子来,齐籰子根下寸丝不留,咯登一声铰断,笑道:“这一去把病根儿可都带了去了。
”那风筝飘飘摇摇,只管往后退了去,一时只有鸡蛋大小,展眼只剩了一点黑星,再展眼便不见了。
众人皆仰面睃眼说:\zhu{睃[suō]:看或斜着眼看。
}“有趣,有趣。
”宝玉道:“可惜不知落在那里去了。
若落在有人烟处,被小孩子得了还好,若落在荒郊野外无人烟处,我替他寂寞。
想起来把我这个放去,教他两个作伴儿罢。
”于是也用剪子剪断,照先放去。
\zhu{照:向着、对着。
}\par
\ping{
黛玉自丧母之后就离家投奔外祖母,犹如一只断线的风筝,再也难回她朝思暮想的故乡了,命运像断线的风筝一样飘摇不可把握。
她在《葬花吟》中写道:“愿奴胁下生双翼,随花飞到天尽头。”林黛玉的命运不能自己把握,像风筝一样任人摆布。
}
\par
探春正要剪自己的凤凰,见天上也有一个凤凰,因道:“这也不知是谁家的。
”众人皆笑说:“且别剪你的,看他倒像要来绞的样儿。
”说着,只见那凤凰渐逼近来,遂与这凤凰绞在一处。
众人方要往下收线,那一家也要收线,正不开交,又见一个门扇大的玲珑喜字带响鞭,\zhu{鞭:指长条形类似鞭子的物品。
}在半天如钟鸣一般,也逼近来。
众人笑道:“这一个也来绞了。
且别收,让他三个绞在一处倒有趣呢。
”说着,那喜字果然与这两个凤凰绞在一处。
三下齐收乱顿,谁知线都断了,那三个风筝飘飘摇摇都去了。
\ping{探春婚姻巨变并远嫁的伏笔。
}众人拍手哄然一笑,说:“倒有趣,可不知那喜字是谁家的,忒促狭了些。
”\zhu{促狭:刁钻机灵,爱捉弄人。
}
黛玉说:“我的风筝也放去了,我也乏了,我也要歇歇去了。
”宝钗说:“且等我们放了去,大家好散。
”说着,看姊妹都放去了,大家方散。
黛玉回房歪着养乏。
要知端的,下回便见。
\par
\ping{
探春最终含悲随夫远嫁,像断了线的风筝一样一去不归。
在第二十二回中探春打了一个谜说“游丝一断浑无力,莫向东风怨别离”,谜底就是探春命运自画像的风筝。
在第六十三回群芳夜宴时探春抽的签上写的“日边红杏”、“瑶池仙品”也暗示探春远嫁。
风筝的放飞需要借助风的力量,这预示着探春不能主宰自己的命运,只能任人摆布,她的命运像柳絮或断线的风筝一样飘摇不可把握。
}
\par
\qi{总评:文与雪天联诗篇,\zhu{指第五十回《芦雪广争联即景诗 \quad 暖香坞雅制春灯谜》}一样机轴,两样笔墨。
前文以联句起,以灯谜结,以作画为中间横风吹断,此文以填词起,以风筝结,以写字为中间横风吹断,是一样机轴;
\ping{本文的叙事顺序为:桃花行、写字、填词、放风筝,和这条评论里说的顺序不一致。}
前文叙联句详,此文叙填词略,是两样笔墨,前文之叙作画略,此文叙写字详,是两样笔墨。
前文叙灯谜,叙猜灯谜,此文叙风筝,叙放风筝,是一样机轴;前文叙七律在联句后,\zhu{七律指的是第五十回咏红梅的诗。
}
此文叙古歌在填词前,是两样笔墨。
前文叙黛玉替宝玉写诗,此文叙宝玉替探春续词,是一样机轴,前文赋诗后有一首诗,\zhu{一首诗指的是宝玉写的咏梅诗。
}此文填词前有一首词,是两样笔墨。
噫!参伍其变,错综其数,
\zhu{参:“三”的大写。
伍:“五”的大写。
参伍:或三或五,指变化不定的数。
《易·系辞上》:“参伍以变,错综其数,通其变,遂成天下之文,极其数,遂定天下之象。
”
在这里形容行文叙事变化多样。
}
此固难为粗心者道也!}
\dai{139}{紫鹃送来黛玉替宝玉写的楷书字}
\dai{140}{姊妹放风筝}
\sun{p70-1}{众姊妹争看桃花行}{湘云打发了翠缕请宝玉出去瞧好诗。
宝玉出来,果见黛玉、宝钗、湘云、宝琴、探春都在沁芳亭,手里拿着一篇诗看,原来是黛玉写的《桃花行》。
}
\sun{p70-2}{姊妹走访稻香老农}{众姊妹到稻香村中,李纨看了诗,称赏不已。
大家议定重    启诗社,改“海棠社”为“桃花社”。
大家正在拟题,一丫鬟来说:“舅太太来了。
姑娘出去请安。
”}
\sun{p70-3}{填柳絮词放风筝}{图右侧:史湘云因见柳花飘舞,便偶成一小令,湘云建议起社填词。
众人来了,以柳絮之题,分别按调填词。
图左侧:众人兴起,纷纷放起风筝,但是宝玉的美人风筝放不起来。
}