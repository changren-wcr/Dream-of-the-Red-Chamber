\chapter{情中情因情感妹妹 \quad 错里错以错劝哥哥}
\qi{两条素帕,一片真心;三首新诗,万行珠泪。
袭卿高见动夫人,薛家兄妹空争气。
自古道情是苦根苗,慧性灵心的,回头须早。
}\par
话说袭人见贾母王夫人等去后,便走来宝玉身边坐下,含泪问他:“怎么就打到这步田地?”宝玉叹气说道:“不过为那些事,问他做什么!只是下半截疼的很,你瞧瞧打坏了那里。
”袭人听说,便轻轻的伸手进去,将中衣褪下。
宝玉略动一动,便咬着牙叫“嗳哟”,袭人连忙停住手,如此三四次才褪了下来。
袭人看时,只见腿上半段青紫,都有四指宽的僵痕高了起来。
袭人咬着牙说道:“我的娘,怎么下这般的狠手!你但凡听我一句话,也不得到这步地位。
幸而没动筋骨,倘或打出个残疾来,可叫人怎么样呢!”\par
正说着,只听丫鬟们说:“宝姑娘来了。
”袭人听见,知道穿不及中衣,便拿了一床袷纱被替宝玉盖了。
\zhu{
袷,音“夹”,同夹。
袷纱被:表里两层的纱被。
}只见宝钗手里托着一丸药走进来,\meng{请问是关心不是关心?}向袭人说道:“晚上把这药用酒研开,替他敷上,把那淤血的热毒散开,可以就好了。
”说毕,递与袭人,又问道:“这会子可好些?”宝玉一面道谢说:“好了。
”又让坐。
宝钗见他睁开眼说话,不像先时,心中也宽慰了好些,便点头叹道:“早听人一句话,\meng{同袭人语。
}也不至今日。
别说老太太、太太心疼,就是我们看着,心里也……”刚说了半句又忙咽住,自悔说的话急了,不觉的就红了脸,\meng{行云流水语,微露半含时。
}低下头来。
宝玉听得这话如此亲切稠密,大有深意,忽见他又咽住不往下说,红了脸,低下头只管弄衣带,那一种娇羞怯怯,非可形容得出者,不觉心中大畅,将疼痛早丢在九霄云外,心中自思:“我不过捱了几下打,\zhu{捱:同“挨”。
}他们一个个就有这些怜惜悲感之态露出,令人可玩可观,可怜可敬。
假若我一时竟遭殃横死,他们还不知是何等悲感呢!\meng{得遇知己者,多生此等疑思疑喜。
}既是他们这样,我便一时死了,得他们如此,一生事业纵然尽付东流,亦无足叹惜,冥冥之中若不怡然自得,亦可谓糊涂鬼祟矣。
”想着,只听宝钗问袭人道:“怎么好好的动了气,就打起来了?”袭人便把茗烟的话说了出来。
\par
宝玉原来还不知道贾环的话,见袭人说出方才知道。
因又拉上薛蟠,惟恐宝钗沉心,\zhu{沉心:多指言者无意而听者有心,陡生不快。
也叫“吃心”或“嗔心”。
}忙又止住袭人道:“薛大哥哥从来不这样的,你们不可混猜度。
”宝钗听说,便知道是怕他多心,用话相拦袭人,\zhu{相:原意是互相,但是这里表示动作偏向一方。
}因心中暗暗想道:“打的这个形像,疼还顾不过来,还是这样细心,怕得罪了人,可见在我们身上也算是用心了。
\meng{天下古今英雄同一感慨。
}你既这样用心,何不在外头大事上做工夫,老爷也欢喜了,也不能吃这样亏。
但你固然怕我沉心,所以拦袭人的话,难道我就不知我的哥哥素日恣心纵欲,毫无防范的那种心性。
当日为一个秦钟,还闹的天翻地覆,\ping{书中并未见秦钟和薛蟠之间的联系,可能是漏写情节,也可能是补充往事。
舒本第九回结尾这样写:贾瑞只要暂息此事,又悄悄的劝金荣说:“俗语说的‘光棍不吃眼前亏’。
咱们如今少不得委曲着陪个不是,然后再寻主意报仇。
不然,弄出事来,道是你起端,也不得干净。
”金荣听了有理,方忍气含愧的来与秦钟磕了一个头,方罢了。
贾瑞遂立意要去调拨薛蟠来报仇,与金荣计议已定,一时散学,各自回家。
不知他怎么去调拨薛蟠,且看下回分解。
舒本的文字比较特别,它跟第十回开头衔接不起来,因为后面再没有提到贾瑞是怎样去调拨薛蟠来报仇的。
但是这里宝钗联想到,她哥“当日为一个秦钟还闹的天翻地覆”。
书中并没有其他地方有薛蟠和秦钟同时登场的,所以“闹的天翻地覆”应该就是指第九回结尾提到的这件事。
}自然如今比先又更利害了。
”想毕,因笑道:“你们也不必怨这个,怨那个。
据我想,到底宝兄弟素日不正,肯和那些人来往,老爷才生气。
就是我哥哥说话不防头,\zhu{不防头:冒失,不留神、不经意。
}一时说出宝兄弟来,也不是有心调唆:一则也是本来的实话,二则他原不理论这些防嫌小事。
\zhu{防嫌:避嫌,避开嫌疑。
}袭姑娘从小儿只见宝兄弟这么样细心的人,\meng{心头口头不觉透漏。
}你何尝见过天不怕地不怕、心里有什么口里就说什么的人。
”\ping{宝钗虽然心里埋怨自己的哥哥,但是嘴上不能说。
}\par
袭人因说出薛蟠来,见宝玉拦他的话,早已明白自己说造次了,恐宝钗没意思,听宝钗如此说,更觉羞愧无言。
宝玉又听宝钗这番话,一半是堂皇正大,一半是去己疑心,更觉比先畅快了。
方欲说话时,只见宝钗起身说道:“明儿再来看你,你好生养着罢。
方才我拿了药来交给袭人,晚上敷上管就好了。
\meng{何等关心!}”说着便走出门去。
袭人赶着送出院外,说:“姑娘倒费心了。
改日宝二爷好了,亲自来谢。
”宝钗回头笑道:“有什么谢处。
你只劝他好生静养,别胡思乱想的就好了。
\meng{的确真心。
}要想什么吃的玩的,悄悄的往我那里去取了,不必惊动老太太、太太众人,倘或吹到老爷耳朵里,虽然彼时不怎么样,将来对景,终是要吃亏的。
\meng{要紧。
}”说着,一面去了。
\par
袭人抽身回来,心内着实感激宝钗。
进来见宝玉沉思默默似睡非睡的模样,因而退出房外,自去栉沐。
\zhu{栉:音“志”,梳子、篦子的通称,引申为梳头。
栉沐:梳洗。
}宝玉默默的躺在床上,无奈臀上作痛,如针挑刀挖一般,更又热如火炙,略展转时,禁不住“嗳哟”之声。
那时天色将晚,因见袭人去了,却有两三个丫鬟伺候,此时并无呼唤之事,因说道:“你们且去梳洗,等我叫时再来。
”众人听了,也都退出。
\par
这里宝玉昏昏默默,只见蒋玉菡走了进来,诉说忠顺府拿他之事;又见金钏儿进来哭说为他投井之情。
宝玉半梦半醒,都不在意。
忽又觉有人推他,恍恍惚惚听得有人悲戚之声。
宝玉从梦中惊醒,睁眼一看,不是别人,却是林黛玉。
宝玉犹恐是梦,忙又将身子欠起来,向脸上细细一认,只见两个眼睛肿的桃儿一般,满面泪光,不是黛玉,却是那个?宝玉还欲看时,怎奈下半截疼痛难忍,支持不住,便“嗳哟”一声,仍就倒下,叹了一声,说道:“你又做什么跑来!虽说太阳落下去,那地上的馀热未散,走两趟又要受了暑。
我虽然捱了打,并不觉疼痛。
我这个样儿,只装出来哄他们,好在外头布散与老爷听,其实是假的。
你不可认真。
”\meng{有这样一段\sout{语}[话],方不没灭颦儿之痛哭眼肿。
英雄失足,每每至死不改,皆犹此耳。
\zhu{至死不改:后文黛玉劝宝玉改,但是宝玉回答说“就便为这些人死了,也是情愿的!”拼着一死也不改悔。}
}\ping{这里宝玉可能是为了安慰黛玉说的谎话,故意说自己是装出来的疼不是真疼。
}此时林黛玉虽不是嚎啕大哭,然越是这等无声之泣,气噎喉堵,更觉得利害。
听了宝玉这番话,心中虽然有万句言词,只是不能说得,半日,方抽抽噎噎的说道:“你从此可都改了罢!\meng{心血淋漓,酿成此数字。
}”宝玉听说,便长叹一声,道:“你放心,别说这样话。
就便为这些人死了,\meng{文气斩截。
}也是情愿的!”一句话未了,只见院外人说:“二奶奶来了。
”林黛玉便知是凤姐来了,连忙立起身说道:“我从后院子去罢,回来再来。
”宝玉一把拉住道:“这可奇了,好好的怎么怕起他来。
”林黛玉急的跺脚,悄悄的说道:“你瞧瞧我的眼睛,又该他取笑开心呢。
\meng{不避嫌疑,不惜声名,破格牵连,诚为可叹,着实可怜。
}”宝玉听说,赶忙的放手。
黛玉三步两步转过床后,出后院而去。
凤姐从前头已进来了,问宝玉:“可好些了?想什么吃,叫人往我那里取去。
”接着,薛姨妈又来了。
一时贾母又打发了人来。
\par
至掌灯时分,宝玉只喝了两口汤,便昏昏沉沉的睡去。
接着,周瑞媳妇、吴新登媳妇、郑好时媳妇这几个有年纪常往来的,听见宝玉捱了打,也都进来。
袭人忙迎出来,悄悄的笑道:“婶婶们来迟了一步,\meng{袭卿善词令,会周旋。
}二爷才睡着了。
”说着,一面带他们到那边房里坐了,倒茶与他们吃。
那几个媳妇子都悄悄的坐了一回,向袭人说:“等二爷醒了,你替我们说罢。
”
\ping{礼节性的探视病人甚至不需要见到病人。}
\par
袭人答应了,送他们出去。
刚要回来,只见王夫人使个婆子来,口称“太太叫一个跟二爷的人呢”。
袭人见说,想了一想,便回身悄悄告诉晴雯、麝月、檀云、秋纹等说:“太太叫人,你们好生在房里,我去了就来。
”\meng{身任其责,不惮劳烦。
}说毕,同那婆子一径出了园子,来至上房。
王夫人正坐在凉榻上摇着芭蕉扇子,见他来了,说:“不管叫个谁来也罢了。
你又丢下他来了,谁伏侍他呢?”
\ping{凸显了在王夫人心目中,袭人是最可靠的人。}
袭人见说,连忙陪笑回道:“二爷才睡安稳了,那四五个丫头如今也好了,会伏侍二爷了,太太请放心。
恐怕太太有什么话吩咐,打发他们来,一时听不明白,倒耽误了。
\meng{能事解事,能了事。
}”王夫人道:“也没甚话,白问问他这会子疼的怎么样。
”\zhu{白:单单,只是。
}袭人道:“宝姑娘送去的药,我给二爷敷上了,\meng{补足。
}\ping{宝钗在王夫人面前再次加分。
}比先好些了。
先疼的躺不稳,这会子都睡沉了,可见好些了。
”王夫人又问:“吃了什么没有?”袭人道:“老太太给的一碗汤,喝了两口,只嚷干渴,要吃酸梅汤。
我想着酸梅是个收敛的东西,才刚捱了打,又不许叫喊,自然急的那热毒热血未免不存在心里,倘或吃下这个去激在心里,再弄出大病来,可怎么样呢。
因此我劝了半天才没吃,\meng{能事处。
}只拿那糖腌的玫瑰卤子和了吃,
\zhu{
玫瑰卤:玫瑰卤应该是较稀薄的玫瑰酱汁。清代《调鼎集》记录的造酱法,玫瑰酱是将玫瑰花蕊置入甜酱中,而制甜酱的基本原料是面粉和黄豆。
就原料特性推想,宝玉觉得玫瑰卤子腻味,是不难理解的。卤子终究是酱料,虽说添加了玫瑰,也不离甜酱的底子,拿来调和汤水,却只是一碗略带玫瑰花香的酱汤,怎比得香露那般轻盈绝尘。
}
吃了半碗,又嫌吃絮了,\zhu{絮:连续重复,惹人厌烦,腻烦。
类似的词有“絮叨”。
这里指对某种食物因吃得过于频繁或过多而生厌。
}不香甜。
”王夫人道:“嗳哟,你不该早来和我说。
前儿有人送了两瓶子香露来,原要给他点子的,我怕他胡糟踏了,就没给。
既是他嫌那些玫瑰膏子絮烦,把这个拿两瓶子去。
一碗水里只用挑一茶匙儿,就香的了不得呢。
”说着就唤彩云来,“把前儿的那几瓶香露拿了来。
”袭人道:“只拿两瓶来罢,多了也白糟踏。
等不够再要,再来取也是一样。
”彩云听说,去了半日,果然拿了两瓶来,付与袭人。
袭人看时,只见两个玻璃小瓶,却有三寸大小,上面螺丝银盖,鹅黄笺上写着“木樨清露”,
\zhu{木樨:即桂花。木樨清露:桂花蒸馏所得香液。}
那一个写着“玫瑰清露”。
袭人笑道:“好金贵东西!这么个小瓶儿,能有多少?”王夫人道:“那是进上的,你没看见鹅黄笺子?你好生替他收着,别糟踏了。
”\par
袭人答应着,方要走时,王夫人又叫:“站着,我想起一句话来问你。
”袭人忙又回来。
王夫人见房内无人,便问道:“我恍惚听见宝玉今儿捱打,是环儿在老爷跟前说了什么话。
你可听见这个了?你要听见,告诉我听听,我也不吵出来教人知道是你说的。
”袭人道:“我倒没听见这话,为二爷霸占着戏子,人家来和老爷要,为这个打的。
”王夫人摇头说道:“也为这个,还有别的原故。
”袭人道:“别的原故实在不知道了。
\ping{袭人向宝钗转述了茗烟的猜测:“那琪官的事,多半是薛大爷素日吃醋,没法儿出气,不知在外头唆挑了谁来,在老爷跟前下的火。
那金钏儿的事是三爷说的,我也是听见老爷的人说的。
”此时却不敢向王夫人提及,可能是由于宝钗临走时说的一句话:“倘或吹到老爷耳朵里,虽然彼时不怎么样,将来对景,终是要吃亏的”。
袭人抽身回来,心内着实感激宝钗。
感激的原因,一方面是感激宝钗没有计较自己口无遮拦说出了可能是宝钗的哥哥薛蟠造成的宝玉的挨打,另一方面也是感激宝钗提醒不要声张,如果闹大了,传到贾政耳朵里,细究金钏之死的的真相,那么草菅人命的王夫人肯定是脱不了干系。
}我今儿在太太跟前大胆说句不知好歹的话。
论理……”说了半截忙又咽住。
王夫人道:“你只管说。
”袭人笑道:“太太别生气,我就说了。
”王夫人道:“我有什么生气的,你只管说来。
”\par
袭人道:“论理,我们二爷也须得老爷教训两顿。
若老爷再不管,将来不知做出什么事来呢。
”王夫人一闻此言,便合掌念声“阿弥陀佛”,\meng{能了事处。
}由不得赶着袭人叫了一声“我的儿,亏了你也明白,这话和我的心一样。
\meng{袭卿之心,所谓“良人所仰望而终身也”。
\zhu{良人:古代女子对丈夫的称呼。
}
\zhu{
语出《孟子·离娄下》:“齐人有一妻一妾而处室者,其良人出,则必餍酒肉而后反。其妻问所与饮食者,则尽富贵也。
其妻告其妾曰:「良人出,则必餍酒肉而后反;问其与饮食者,尽富贵也,而未尝有显者来,吾将瞷良人之所之也。」		
蚤起,施从良人之所之,遍国中无与立谈者。卒之东郭墦闲,之祭者,乞其余;不足,又顾而之他,此其为餍足之道也。
其妻归,告其妾曰:「良人者,所仰望而终身也。今若此。」与其妾讪其良人,而相泣于中庭。而良人未之知也,施施从外来,骄其妻妾。		
由君子观之,则人之所以求富贵利达者,其妻妾不羞也,而不相泣者,几希矣。”
这是一个生动的寓言故事,辛辣地讽刺了那种不顾礼义廉耻,以卑鄙的手段追求富贵利达的人。
}
\ping{袭人把宝玉当作了终身的依靠。
}今若此,能不痛哭流\sout{泣}[涕],以成此语?}我何曾不知道管儿子,先时你珠大爷在,我是怎么样管他,难道我如今倒不知管儿子了?只是有个原故:如今我想,我已经快五十岁的人,通共剩了他一个,他又长的单弱,况且老太太宝贝似的,若管紧了他,倘或再有个好歹,或是老太太气坏了,那时上下不安,岂不倒坏了,所以就纵坏了他。
\ping{王夫人对于贾珠可能过于严格,贾珠的早逝,使得王夫人不敢管仅剩下的唯一的儿子贾宝玉,又走到了另一个极端,管的过于宽松。
}我常常掰着口儿劝一阵,说一阵,气的骂一阵,哭一阵,彼时他好,过后儿还是不相干,端的吃了亏才罢了。
若打坏了,将来我靠谁呢!\meng{变转之句,勉强之言,真体贴尽溺爱之心。
}”说着,由不得滚下泪来。
\par
袭人见王夫人这般悲感,自己也不觉伤了心,陪着落泪。
又道:“二爷是太太养的,岂不心疼。
便是我们做下人的伏侍一场,大家落个平安,也算是造化了。
要这样起来,连平安都不能了。
那一日那一时我不劝二爷,只是再劝不醒。
偏生那些人又肯亲近他,也怨不得他这样,总是我们劝的倒不好了。
今儿太太提起这话来,我还记挂着一件事,每要来回太太,讨太太个主意。
只是我怕太太疑心,不但我的话白说了,且连葬身之地都没了。
\meng{打进一层。
非有前项如许讲究,这一层即为唐突了。
}”王夫人听了这话内有因,忙问道:“我的儿,你有话只管说。
近来我因听见众人背前背后都夸你,我只说你不过是在宝玉身上留心,或是诸人跟前和气,这些小意思好,所以将你和老姨娘一体行事。
谁知你方才和我说的话全是大道理,正和我的想头一样。
你有什么只管说什么,只别教别人知道就是了。
”\par
袭人道:“我也没什么别的说。
我只想着讨太太一个示下,怎么变个法儿,以后竟还教二爷搬出园外来就好了。
”王夫人听了,吃一大惊,忙拉了袭人的手问道:“宝玉难道和谁作怪了不成?”\ping{宝玉就和王夫人面前这个袭人作怪了,贼喊捉贼。
}袭人忙回道:“太太别多心,并没有这话。
这不过是我的小见识。
如今二爷也大了,里头姑娘们也大了,况且林姑娘宝姑娘又是两姨姑表姊妹,虽说是姊妹们,到底是男女之分,日夜一处起坐不方便,由不得叫人悬心,\meng{远忧近虑,言言字字,真是可人。
}便是外人看着也不像。
\zhu{不像:指言行超越常轨。}
一家子的事,俗语说的‘没事常思有事’,世上多少无头脑的事,多半因为无心中做出,有心人看见,当做有心事,反说坏了。
只是预先不防着,断然不好。
二爷素日性格,太太是知道的。
他又偏好在我们队里闹,倘或不防,前后错了一点半点,不论真假,人多口杂,那起小人的嘴有什么避讳,心顺了,说的比菩萨还好,心不顺,就贬的连畜牲不如。
二爷将来倘或有人说好,不过大家直过没事;
\zhu{直过:平稳,无过错。}
若叫人说出一个不好字来,我们不用说,粉身碎骨,罪有万重,都是平常小事,但后来二爷一生的声名品行岂不完了,\meng{袭卿爱人以德,竟至如此。
字字逼来,不觉令人敬听。
看官自省,切[不]可阔略,\zhu{阔略:忽略,疏漏。
}戒之。
}二则太太也难见老爷。
俗语又说‘君子防不然’,\zhu{君子防不然:亦作“君子防患于未然”。
意谓君子防备祸患于未发生之时。
见宋代郭茂倩编《乐府诗集·君子行》:“君子防未然,不处嫌疑间。
瓜田不纳履,李下不正冠。
”}不如这会子防避的为是。
太太事情多,一时固然想不到。
我们想不到则可,既想到了,若不回明太太,罪越重了。
近来我为这事日夜悬心,又不好说与人,惟有灯知道罢了。
”
\ping{一个丫头去评判主人的是非风险很大,如果王夫人不能了解她的真正用意,她将死无葬身之地。}
\par
王夫人听了这话,如雷轰电掣一般,正触了金钏儿之事,心内越发感爱袭人不尽,忙笑道:“我的儿,你竟有这个心胸,想的这样周全!我何曾又不想到这里,只是这几次有事就忘了。
你今儿这一番话提醒了我。
难为你成全我娘儿两个声名体面,真真我竟不知道你这样好。
罢了,你且去罢,我自有道理。
\meng{溺爱者偏会如此说。
}只是还有一句话:你如今既说了这样的话,我就把他交给你了,好歹留心,保全了他,就是保全了我。
我自然不辜负你。
”\par
袭人连连答应着去了。
回来正值宝玉睡醒,袭人回明香露之事。
宝玉喜不自禁,即令调来尝试,果然香妙非常。
因心下记挂着黛玉,满心里要打发人去,只是怕袭人,\ping{第三十二回,诉肺腑心迷活宝玉,袭人撞见宝玉对黛玉的表白,所以这里说“怕袭人”。
}便设一法,先使袭人往宝钗那里去借书。
袭人去了,宝玉便悄命晴雯\ji{前文晴雯放肆,原有把柄所持也。
\zhu{宝玉拿住了晴雯之前放肆的把柄,所以交办见不得人的事情给晴雯做。}
}吩咐道:“你到林姑娘那里看看他做什么呢。
他要问我,只说我好了。
”\ping{宝钗与袭人,晴雯和黛玉,确实是成双成对。
晴为黛影,袭为钗副。
袭人可以得到王夫人的赞赏,可是宝玉却在这个时候觉得给黛玉带话她不合适。
一个是深情,一个是礼教,在礼教当中袭人虽然成功了,在深情里宝玉却不用她。    
}晴雯道:“白眉赤眼,\zhu{白眉赤眼:平白无故、没来由的意思。
}做什么去呢?到底说句话儿,也像一件事。
”宝玉道:“没有什么可说的。
”晴雯道:“若不然,或是送件东西,或是取件东西,不然我去了怎么搭讪呢?”宝玉想了一想,便伸手拿了两条手帕子撂与晴雯,笑道:“也罢,就说我叫你送这个给他去了。
”晴雯道:“这又奇了。
他要这半新不旧的两条手帕子?他又要恼了,说你打趣他。
”宝玉笑道:“你放心,他自然知道。
”\par
晴雯听了,只得拿了帕子往潇湘馆来。
只见春纤正在栏杆上晾手帕子,\meng{送的是手帕,晾的是手帕,妙文。
}见他进来,忙摆手儿,说:“睡下了。
”晴雯走进来,满屋魆黑,\zhu{魆:音“虚”。
黑魆魆:形容黑暗。
}并未点灯。
黛玉已睡在床上,问是谁。
晴雯忙答道:“晴雯。
”黛玉道:“做什么?”晴雯道:“二爷送手帕子来给姑娘。
”黛玉听了,心中发闷:“做什么送手帕子来给我?”因问:“这帕子是谁送他的?必是上好的,叫他留着送别人罢,我这会子不用这个。
”晴雯笑道:“不是新的,就是家常旧的。
”林黛玉听见,越发闷住,着实细心搜求,思忖一时,方大悟过来,连忙说:“放下,去罢。
”晴雯听了,只得放下,抽身回去,一路盘算,不解何意。
\ping{宝钗的影子袭人自信地去搞定了王夫人,黛玉的影子晴雯糊里糊涂地送了个帕子,虽然情感上黛玉更胜一筹,但是周边布局上看,确实危险,所以黛玉看到手帕后发出了“我这番苦意,不知将来如何,又令我可悲”这样的感慨。
}\par
这里林黛玉体贴出手帕子的意思来,不觉神魂驰荡:宝玉这番苦心,能领会我这番苦意,又令我可喜;我这番苦意,不知将来如何,又令我可悲;忽然好好的送两块旧帕子来,若不是领我深意,
\zhu{“领我深意”应该意为“我领深意”。}
单看了这帕子,又令我可笑;再想令人私相传递与我,又可惧;我自己每每好哭,想来也无味,又令我可愧。
如此左思右想,一时五内沸然炙起。
黛玉由不得馀意绵缠,令掌灯,也想不起嫌疑避讳等事,便向案上研墨蘸笔,便向那两块旧帕上走笔写道:\ping{第三十回:“林黛玉虽然哭着,却一眼看见了,见他穿着簇新藕合纱衫,竟去拭泪,便一面自己拭着泪,一面回身将枕边搭的一方绡帕子拿起来,向宝玉怀里一摔,一语不发,仍掩面自泣。
宝玉见他摔了帕子来,忙接住拭了泪。
”虽是旧手帕,但是饱含了宝玉自己为黛玉流的眼泪。
前文有小红和贾芸借手帕传情,这里类似。
}\par
\hop
其一\par
眼空蓄泪泪空垂,暗洒闲抛却为谁?\par
尺幅鲛鮹劳解赠,叫人焉得不伤悲!\zhu{鲛鮹:音“交消”,传说南海中有鲛人,即人鱼,能织绡,后用以泛称薄纱。
}\par
\hop
其二\par
抛珠滚玉只偷潸,镇日无心镇日闲;\zhu{潸:音“山”,流泪的样子。
镇日:整天,“镇”“整”通。
}\par
枕上袖边难拂拭,任他点点与斑斑。
\par
\hop
其三\par
彩线难收面上珠,湘江旧迹已模糊;\zhu{湘江旧迹:代指泪痕。
用湘妃哭舜,泪染斑竹的典故。
}\par
窗前亦有千竿竹,不识香痕渍也无?\zhu{最后一句的意思是,如果不是有心有情之人,不能认出竹子上洒的泪痕,那么也就不知道斑竹上的斑点其实是思念的泪水染成的。
}\par
\hop
林黛玉还要往下写时,觉得浑身火热,面上作烧,走至镜台揭起锦袱一照,只见腮上通红,自羡压倒桃花,却不知病由此萌。
一时方上床睡去,犹拿着那帕子思索,不在话下。
\par
却说袭人来见宝钗,谁知宝钗不在园内,往他母亲那里去了,袭人便空手回来。
等至二更,宝钗方回来。
原来宝钗素知薛蟠情性,心中已有一半疑是薛蟠调唆了人来告宝玉的,谁知又听袭人说出来,越发信了。
究竟袭人是听茗烟说的,那茗烟也是私心窥度,并未据实,竟认准是他说的。
那薛蟠都因素日有这个名声,其实这一次却不是他干的,被人生生的一口咬死是他,有口难分。
\zhu{分:分辨,辩解。
}
\ping{《红楼梦》里作者很少直接出来说话,在这里作者蛮维护薛蟠的。}
这日正从外头吃了酒回来,见过母亲,只见宝钗在这里,说了几句闲话,因问:“听见宝兄弟吃了亏,是为什么?”薛姨妈正为这个不自在,见他问时,便咬着牙道:“不知好歹的东西,都是你闹的,你还有脸来问!”薛蟠见说,便怔了,忙问道:“我何尝闹什么?”薛姨妈道:“你还装憨呢!人人都知道是你说的,还赖呢。
”薛蟠道:“人人说我杀了人,也就信了罢?”薛姨妈道:“连你妹妹都知道是你说的,难道他也赖你不成?”宝钗忙劝道:“妈和哥哥且别叫喊,消消停停的,就有个青红皂白了。
”因向薛蟠道:“是你说的也罢,不是你说的也罢,事情也过去了,不必较证,倒把小事儿弄大了。
我只劝你从此以后在外头少去胡闹,少管别人的事。
天天一处大家胡逛,你是个不防头的人,
\zhu{不防头:冒失,不留神、不经意。}
过后儿没事就罢了,倘或有事,不是你干的,人人都也疑惑是你干的,不用说别人,我就先疑惑。
”\par
薛蟠本是个心直口快的人,一生见不得这样藏头露尾的事,又见宝钗劝他不要逛去,他母亲又说他犯舌,\zhu{犯舌:犹多嘴。
}宝玉之打是他治的,早已急的乱跳,赌身发誓的分辩。
又骂众人:“谁这样赃派我?我把那囚攮的牙敲了才罢!
\zhu{
攮:骂人糊涂愚笨。例如“狗攮的”。 
囚攮的:骂人的话。意指囚犯的子女。
}
分明是为打了宝玉,没的献勤儿,拿我来作幌子。
\zhu{
幌子:店铺门前的招牌谓之幌子。作幌子:比喻假借某种名义进行活动,
这里的意思是假借教训薛蟠来献殷勤。
}
难道宝玉是天王?他父亲打他一顿,一家子定要闹几天。
那一回为他不好,姨爹打了他两下子,过后老太太不知怎么知道了,说是珍大哥哥治的,好好的叫了去骂了一顿。
今儿越发拉上我了!既拉上,我也不怕,越性进去把宝玉打死了,我替他偿了命,大家干净。
”一面嚷,一面抓起一根门闩来就跑。
慌的薛姨妈一把抓住,骂道:“作死的孽障,你打谁去?你先打我来!”\par
薛蟠急的眼似铜铃一般,嚷道:“何苦来!又不叫我去,又好好的赖我。
将来宝玉活一日,我担一日的口舌,不如大家死了清净。
”宝钗忙也上前劝道:“你忍耐些儿罢。
妈急的这个样儿,你不说来劝妈,你还反闹的这样。
别说是妈,便是旁人来劝你,也为你好,倒把你的性子劝上来了。
”薛蟠道:“这会子又说这话。
都是你说的!”宝钗道:“你只怨我说,再不怨你顾前不顾后的形景。
”薛蟠道:“你只会怨我顾前不顾后,你怎么不怨宝玉外头招风惹草的那个样子!别说多的,只拿前儿琪官的事比给你们听:那琪官,我们见过十来次的,我并未和他说一句亲热话;怎么前儿他见了,连姓名还不知道,就把汗巾子给他了?难道这也是我说的不成?”薛姨妈和宝钗急的说道:“还提这个!可不是为这个打他呢。
可见是你说的了。
”薛蟠道:“真真的气死了人了!赖我说的我不恼,我只为一个宝玉闹的这么天翻地覆的。
”宝钗道:“谁闹了?你先持刀动杖的闹起来,倒说别人闹。
”\par
薛蟠见宝钗说的句句有理,难以驳正,比母亲的话反难回答,因此便要设法拿话堵回他去,就无人敢拦自己的话了;也因正在气头儿上,未曾想话之轻重,便说道:“好妹妹,你不用和我闹,我早知道你的心了。
从先妈和我说,你这金要拣有玉的才可正配,你留了心,见宝玉有那劳什骨子,\zhu{劳什骨子:通灵宝玉。
}你自然如今行动护着他。
”
\ping{作者借薛蟠这个没有头脑的人,把真相讲出来。}
话未说了,把个宝钗气怔了,拉着薛姨妈哭道:“妈妈你听,哥哥说的是什么话!”\meng{插写薛蟠,不过要补足宝钗告袭人前项之言。
\ping{宝钗听了袭人转述宝玉挨打的原因,告诫袭人不要再继续追责,否则把事情闹大没有好处,袭人果然照做,面对王夫人的询问,袭人装作不知道。
宝钗自己却没有做到不继续追责,虽然在袭人说出宝玉挨打和薛蟠有关时,宝钗心里埋怨嘴上还是为自己的哥哥开脱,但是回到家之后,想到宝玉挨打的惨状,还是忍住不要埋怨自己的哥哥,宝钗对宝玉也是很上心了。
宝钗的冷静理性,体现在给袭人不要追责的告诫,而宝钗的激动感性,体现在自己却继续追责。
}}薛蟠见妹妹哭了,便知自己冒撞了,便赌气走到自己房里安歇不提。
\ping{若宝钗之前还是懵懂,从这刻起也是明了自己的心意了吧。
这一回宝钗发现自己的感情,探视宝玉,送药,为了宝玉追问哥哥,在宝玉床前差点说出了自己也心疼,不觉红了脸;宝玉察觉宝钗的感情,不觉心中大畅,将疼痛早丢在九霄云外。
但是此时宝玉情感尚未成熟,自己虽然深爱黛玉,但是也希望得到其他女孩子的爱,希望她们因为自己的痛苦甚至死亡而“怜惜悲感”,直到第三十六回,贾宝玉才醒悟,自己不可能得到很多女孩子的眼泪和感情,“从此后只是各人各得眼泪罢了”,“深悟人生情缘,各有分定”,从广泛的爱走向专一的爱;宝玉和黛玉心意相通,通过手帕传情,深化了两人的感情。
}\par
这里薛姨妈气的乱战,\zhu{战:通“颤”,发抖。
}一面又劝宝钗道:“你素日知那孽障说话没道理,明儿我叫他给你陪不是。
”宝钗满心委屈气忿,待要怎样,又怕他母亲不安,少不得含泪别了母亲,各自回来,
\zhu{各自:各方自己;个人自己,这里是第二个意思。}
到房里整哭了一夜。
次日早起来,也无心梳洗,胡乱整理整理,便出来瞧母亲。
可巧遇见林黛玉独立在花阴之下,问他那里去。
薛宝钗因说“家去”,口里说着,便只管走。
黛玉见他无精打采的去了,又见眼上有哭泣之状,大非往日可比,便在后面笑道:“姐姐也自保重些儿。
就是哭出两缸眼泪来,也医不好棒疮!”\meng{自己眼肿为谁?偏是以此笑人。
笑人世间人多犯此症。
}不知宝钗如何答对,且听下回分解。
\par
\qi{总评:人有百折不回之真心,方能成旷世稀有之事业。
宝玉意中诸多辐辏,\zhu{
辐:“福”,车轮上连接轮圈和毂的直条。
毂:音“古”,车轮的中心部位。
辏:音“凑”,聚集。
辐辏:形容人物聚集像轮辐集中于毂。
}所谓“求仁得仁,又何怨?”
\zhu{
求仁得仁,又何怨:追求仁,并得到了仁,又有什么后悔呢?
语本《论语·述而》:“曰:‘求仁而得仁,又何怨?’”指按照某人自己的要求满足他,就没有什么怨言了。
}
\zhu{宝玉意中……又何怨:宝玉用真心在姐妹身上,所以自己挨打之后,姐妹都围过来安慰。
宝玉在本回说“就便为这些人死了,也是情愿的!”,由此可见宝玉并非因挨打而有悔改之心。
}凡人作臣作子,出入家庭廊庙,能推此心此志,何患忠孝之不全、事业之不立耶?}
\dai{067}{情中情因情感妹妹}
\dai{068}{王夫人和袭人密谈}