\chapter{痴丫头误拾绣春囊 \quad 懦小姐不问累金凤}
\qi{贾母一席话,隐隐照起全文,便可一直叙去,接笔却置贼不论,转出赌钱,接笔又置赌钱不论,转出奸证,接笔又置奸证不论,转出讨情,一波未平,一波又起,势如怒蛇出穴,蜿蜒不就捕。
}\par
话说那赵姨娘和贾政说话,忽听外面一声响,不知何物。
忙问时,原来是外间窗屉不曾扣好,\zhu{窗屉:装置于窗上,可支起或放落的木架,上糊以纱或纸。
}塌了屈戌了吊下来。
\zhu{屈戌[xū]了吊:门窗上的环钮搭扣。
}赵姨娘骂了丫头几句,自己带领丫鬟上好,方进来打发贾政安歇。
不在话下。
\ping{
在上一回中,贾政对宝玉纳袭人为妾的事情感到惊讶正要追问,被窗屉异响所打断,后面再也没有相关叙述。
两回接续上存在断层,可能是上一回文末和本回开头的底稿破损丢失了相关内容,后人勉强补缀。
或者这里埋下的伏笔将会在遗失的八十回后才会爆发冲突。
}
\par
却说怡红院中宝玉正才睡下,丫鬟们正欲各散安歇,忽听有人击院门。
老婆子开了门,见是赵姨娘房内的丫鬟名唤小鹊的。
问他什么事,小鹊不答,直往房内来找宝玉。
\geng{奇,从未见此婢也。
}只见宝玉才睡下,晴雯等犹在床边坐着,大家顽笑,见他来了,都问:“什么事,这时候又跑了来作什么?”\geng{又是补出前文矣,非只\sout{张}[此]一回也。
}小鹊笑向宝玉道:“我来告诉你一个信儿。
方才我们奶奶这般如此在老爷前说了。
你仔细明儿老爷问你话。
”说着回身就去了。
袭人命留他吃茶,因怕关门,遂一直去了。
\par
这里宝玉听了,便如孙大圣听见了紧箍咒一般,登时四肢五内一齐皆不自在起来。
想来想去,别无他法,且理熟了书预备明儿盘考。
\ping{
在上一回,贾政以读书为由暂缓给宝玉和贾环纳妾,所以宝玉从通风报信的小鹊口中得知后,抓紧温习读书。
但是更严峻的问题是赵姨娘向贾政透露了袭人已经给了宝玉,引起了贾政的惊讶,可以推测小鹊没有听到并通报这件事。
}口内不舛错,\zhu{舛:音“穿”三声,违背,错谬。
}便有他事,也可搪塞一半。
想罢,忙披衣起来要读书。
心中又自后悔,这些日子只说不提了,偏又丢生,早知该天天好歹温习些的。
如今打算打算,肚子内现可背诵的,不过“学”、“庸”、二“论”,\zhu{“学”、“庸”、二“论”:即《大学》、《中庸》和《论语》。
因《论语》分上下两本,故称“二论”。
}
是带注背得出的。
至上本《孟子》,就有一半是夹生的,若凭空提一句,断不能接背的,至“下孟”,就有一大半忘了。
算起“五经”来,\zhu{五经:《易》、《书》、《诗》、《礼》、《春秋》称为五经。
}因近来作诗,常把《诗经》读些,虽不甚精阐,\zhu{精阐:精通明白。
}还可塞责。
\geng{妙!宝玉读书原系从[问]\sout{闺}中\sout{瀶}[滥]而有。
\zhu{问中瀶:应为“闺中滥”即“沉浸于闺阁中”。
此意符合批者在第二十二回评价宝玉所作长批所说“……且宝玉有生以来,此身此心为诸女儿应酬不暇……可知除闺阁之外,并无一事是宝玉立意作出来的……”}}别的虽不记得,素日贾政也幸未吩咐过读的,纵不知,也还不妨。
至于古文,这是那几年所读过的几篇,连《左传》、《国策》、《公羊》、《穀粱》、汉唐等文,\zhu{《左传》亦称《春秋左氏传》,相传为春秋时左丘明所撰,是以《春秋》为纲依据各国史籍编写而成。
《公羊》亦称《春秋公羊传》,《穀粱》即《谷粱》亦称《春秋谷粱传》,相传为战国时公羊高、谷梁赤所撰,多用义理阐释《春秋》。
《国策》即《战国策》,分别叙写战国时期各诸侯国的历史,多记述当时那些谋臣说客的论辩说辞。
}不过几十篇,这几年竟未曾温得半篇片语,虽闲时也曾遍阅,不过一时之兴,随看随忘,未下苦工夫,如何记得。
这是断难塞责的。
更有时文八股一道,\zhu{八股:明清科举考试的一种文体,也称制艺、制义、时艺、时文、八比文。
其体源于宋元的经义,而成于明成化以后,至清光绪末年始废。
文章就四书取题。
开始先揭示题旨,为“破题”。
接着承上文而加以阐发,叫“承题”。
然后开始议论,称“起讲”。
再后为“入手”,为起讲后的入手之处。
以下再分“起股”、“中股”、“后股”和“束股”四个段落,而每个段落中,都有两股排比对偶的文字,合共八股,故称八股文。
其所论内容,都要根据宋朱熹《四书集注》等书“代圣人立説”,不许作者自由发挥。
它是封建统治者束缚人民思想,维护封建统治的工具。
}因平素深恶此道,原非圣贤之制撰,焉能阐发圣贤之微奥,不过作后人饵名钓禄之阶。
虽贾政当日起身时选了百十篇命他读的,不过偶因见其中或一二股内,或承起之中,有作的或精致,或流荡,或游戏,或悲感,稍能动性者,偶一读之,不过供一时之兴趣,究竟何曾成篇潜心玩索。
\geng{妙!写宝玉读书非为功名也。
}如今若温习这个,又恐明日盘诘那个,\zhu{诘:音“洁”,责问,追问。
}若温习那个,又恐盘驳这个。
况一夜之功,亦不能全然温习,因此越添了焦燥。
自己读书不致紧要,却带累着一房丫鬟们皆不能睡。
袭人、麝月、晴雯等几个大的自不用说,在旁剪烛斟茶,那些小的,都困眼朦胧,前仰后合起来。
晴雯因骂道:“什么蹄子们,一个个黑日白夜挺尸挺不够,
\zhu{黑日白夜:白天黑夜。照字面应是“黑夜白日”。这种四字格第二字说成轻音,而“日”音最适合这个轻音位置,所以说快了字面上就颠倒了。}
偶然一次睡迟了些,就装出这腔调来了。
再这样,我拿针戳给你们两下子!”\par
话犹未了,只听外间咕咚一声,急忙看时,原来是一个小丫头子坐着打盹,一头撞到壁上了,从梦中惊醒,恰正是晴雯说这话之时,他怔怔的只当是晴雯打了他一下,遂哭央说:“好姐姐,我再不敢了。
”众人都发起笑来。
\par
宝玉忙劝道:“饶他去罢,原该叫他们都睡去才是。
你们也该替换着睡去。
”袭人忙道:“小祖宗,你只顾你的罢。
通共这一夜的工夫,你把心暂且用在这几本书上,等过了这一关,由你再张罗别的去,也不算误了什么。
”宝玉听他说的恳切,只得又读。
读了没有几句,麝月又斟了一杯茶来润舌,宝玉接茶吃了。
因见麝月只穿着短袄,解了裙子,宝玉道:“夜静了,冷,到底穿一件大衣裳才是。
”麝月笑指着书道:“你暂且把我们忘了,心且略对着他些罢。
”\geng{此处岂是读书之处,又岂是伴读之人?古今天下误尽多少纨绔!何况又是此等时之怡红院,此等之鬟婢,又是此等一个宝玉哉!}\par
话犹未了,只听金星玻璃从后房门跑进来,\zhu{金星玻璃:芳官在第六十三回由宝玉改名为金星玻璃,简称玻璃。
}口内喊说:“不好了,一个人从墙上跳下来了!”众人听说,忙问在那里,即喝起人来,各处寻找。
晴雯因见宝玉读书苦恼,劳费一夜神思,明日也未必妥当,心下正要替宝玉想出一个主意来脱此难,正好忽然逢此一惊,即便生计,向宝玉道:“趁这个机会快装病,只说唬着了。
”此话正中宝玉心怀,因而遂传起上夜人等来,打着灯笼,各处搜寻,并无踪迹,都说:“小姑娘们想是睡花了眼出去,风摇的树枝儿,错认作人了。
”晴雯便道:“别放诌屁!\zhu{诌:音“周”,信口胡说,编瞎话。
}你们查的不严,怕得不是,还拿这话来支吾。
才刚并不是一个人见的,宝玉和我们出去有事,大家亲见的。
如今宝玉唬的颜色都变了,满身发热,我如今还要上房里取安魂丸药去。
太太问起来,是要回明白的,难道依你说就罢了不成。
”\par
众人听了,吓的不敢则声,\zhu{则:做。
则声:开口发言、出声。
}只得又各处去找。
晴雯和玻璃二人果出去要药,故意闹的众人皆知宝玉吓着了。
王夫人听了,忙命人来看视给药,又吩咐各上夜人仔细搜查,又一面叫查二门外邻园墙上夜的小厮们。
于是园内灯笼火把,直闹了一夜。
至五更天,就传管家男女,命仔细查一查,拷问内外上夜男女等人。
\par
贾母闻知宝玉被吓,细问原由。
不敢再隐,只得回明。
贾母道:“我必料到有此事。
如今各处上夜都不小心,还是小事,只怕他们就是贼也未可知。
”当下邢夫人并尤氏等都过来请安,凤姐及李纨姊妹等皆陪侍,听贾母如此说,都默无所答。
独探春出位笑道:“近因凤姐姐身子不好,几日园内的人比先放肆了许多。
先前不过是大家偷着一时半刻,或夜里坐更时,三四个人聚在一处,或掷骰或斗牌,小小的顽意,不过为熬困。
近来渐次放诞,竟开了赌局,甚至有头家局主,\zhu{头家:聚赌抽头的人。
抽头:向赢钱的赌徒抽取一部分的利益给提供赌博场所的人。
也称为“拈头”。
聚赌抽头所得的钱叫头儿钱。
}或三十吊五十吊三百吊的大输赢。
半月前竟有争斗相打之事。
”贾母听了,忙说:“你既知道,为何不早回我们来?”探春道:“我因想着太太事多,且连日不自在,所以没回。
只告诉了大嫂子和管事的人们,戒饬过几次,\zhu{饬:音“赤”,整顿;告诫;命令。
}近日好些。
”贾母忙道:“你姑娘家,如何知道这里头的利害。
你自为耍钱常事,\zhu{为:认为。
}不过怕起争端。
殊不知夜间既耍钱,就保不住不吃酒,既吃酒,就免不得门户任意开锁。
或买东西,寻张觅李,其中夜静人稀,趋便藏贼引奸引盗,何等事作不出来。
况且园内的姊妹们起居所伴者皆系丫头媳妇们,贤愚混杂,贼盗事小,再有别事,\ping{暗指司棋和表哥园内偷情之事。
}倘略沾带些,关系不小。
这事岂可轻恕。
”探春听说,便默然归坐。
凤姐虽未大愈,精神因此比常稍减,\geng{看他渐次写来,从不作一笔安逸之笔,况阿凤之文哉。
}今见贾母如此说,便忙道:“偏生我又病了。
”遂回头命人速传林之孝家的等总理家事四个媳妇到来,当着贾母申饬了一顿。
\zhu{饬:音“赤”,整顿;告诫;命令。}贾母命即刻查了头家赌家来,
\zhu{
头家:聚赌抽头的人。
抽头:向赢钱的赌徒抽取一部分的利益给提供赌博场所的人。
也称为“拈头”。
聚赌抽头所得的钱叫头儿钱。
}
有人出首者赏,\zhu{出首:告发别人的犯罪行为。
}隐情不告者罚。
\ping{搬出贾母才能压得住,这说明贾府已经大乱了。
}\par
林之孝家的等见贾母动怒,谁敢徇私,忙至园内传齐人,一一盘查。
虽不免大家赖一回,终不免水落石出。
查得大头家三人,小头家八人,聚赌者通共二十多人,都带来见贾母,跪在院内磕响头求饶。
贾母先问大头家名姓和钱之多少。
原来这三个大头家,一个就是林之孝家的两姨亲家,\zhu{两姨亲:姐妹的子女间的亲属关系。
亲家:两家儿女相婚配的亲戚关系,这里应该是泛称亲戚之家。
}一个就是园内厨房内柳家媳妇之妹,一个就是迎春之乳母。
这是三个为首的,馀者不能多记。
\par
贾母便命将骰子牌一并烧毁,所有的钱入官分散与众人,将为首者每人四十大板,撵出,总不许再入;从者每人二十大板,革去三月月钱,拨入圊厕行内。
\zhu{圊厕行:打扫管理厕所的行当。
圊:音“青”,厕所。
}又将林之孝家的申饬了一番。
林之孝家的见他的亲戚又与他打嘴,\zhu{打嘴:打嘴巴。
喻指出丑,丢脸。
}自己也觉没趣。
\par
迎春在坐,也觉没意思。
黛玉、宝钗、探春等见迎春的乳母如此,也是物伤其类的意思,遂都起身笑向贾母讨情说:“这个妈妈素日原不顽的,不知怎么也偶然高兴。
求看二姐姐面上,饶他这次罢。
”贾母道:“你们不知。
大约这些奶子们,一个个仗着奶过哥儿姐儿,原比别人有些体面,他们就生事,比别人更可恶,专管调唆主子护短偏向。
我都是经过的。
况且要拿一个作法,\zhu{作法:就是树立某种标准,给别人立规矩,通过责骂、惩罚等手段处理某人立威,杀鸡儆猴,以儆其馀。
}恰好果然就遇见了一个。
你们别管,我自有道理。
”宝钗等听说,只得罢了。
\par
一时贾母歇晌,大家散出,都知贾母今日生气,皆不敢各散回家,只得在此暂候。
尤氏便往凤姐处来闲话了一回,因他也不自在,只得往园内寻众姑嫂闲谈。
邢夫人在王夫人处坐了一回,也就往园内散散心来。
刚至园门前,只见贾母房内的小丫头子名唤傻大姐的笑嘻嘻走来,手内拿着个花红柳绿的东西,低头一壁瞧着,一壁只管走,不防迎头撞见邢夫人,抬头看见,方才站住。
邢夫人因说:“这痴丫头,又得了个什么狗不识儿这么欢喜?拿来我瞧瞧。
”\zhu{狗不识:戏谑语,戏称对方为“狗”,所以才见识短浅,不认识这个东西。
另一种说法,“狗不识”也称“狗不食”:一指令人憎恶的人,做尽伤天害理之事,连狗都懒得去吃他;二指极渺小、极不值钱的东西, 廉价得狗都不食。
}\par
原来这傻大姐年方十四五岁,是新挑上来的与贾母这边提水桶扫院子专作粗活的一个丫头。
只因他生得体肥面阔,两只大脚,作粗活简捷爽利,且心性愚顽,一无知识,行事出言,常在规矩之外。
贾母因喜欢他爽利便捷,又喜他出言可以发笑,便起名为“呆大姐”,常闷来便引他取笑一回,毫无避忌,因此又叫他作“痴丫头”。
他纵有失礼之处,见贾母喜欢他,众人也就不去苛责。
\ping{《红楼梦》里面安排了这个傻丫头的最大目的,是她发现了那个“绣春囊”,凡是有心机的人捡到这个东西都会尽量掩藏,可是就因为她傻,根本不知道那是什么东西。}
\par
这丫头也得了这个力,若贾母不唤他时,便入园内来顽耍。
今日正在园内掏促织,\zhu{促织:蟋蟀的别名。
}忽在山石背后得了一个五彩绣香囊,
\ping{这个香囊可能是第七十一回司棋和潘又安偷情时遗落的。}
其华丽精致,固是可爱,但上面绣的并非花鸟等物,一面却是两个人赤条条的盘踞相抱,一面是几个字。
这痴丫头原不认得是春意,
\zhu{春意:两性爱恋的情意。}
便心下盘算:“敢是两个妖精打架?不然必是两口子相打。
”左右猜解不来,正要拿去与贾母看,\geng{险极妙极!荣府堂堂诗礼之家,且大观园又何等严肃清幽之地,金闺玉阁尚有此等秽物,天下浅阁薄幕之家宁不慎乎!虽然,但此等偏出大官世族之中者,盖因其房室香宵、\zhu{宵:夜。
}鬟婢混杂,焉保其个个守礼持节哉?此正为大官世族而告诫。
其浅阁薄幕之处,母女主婢日夕耳鬓交磨,一止一动悉在耳目之中,又何必谆谆再四焉!}是以笑嘻嘻的一壁看,一壁走,忽见了邢夫人如此说,便笑道:“太太真个说的巧,真个是狗不识呢。
\geng{妙!寓言也,大凡知此交媾之情者,真狗畜之识耳,非肆言恶詈凡识此事者即狗矣。
\zhu{詈:音“立”,骂。
}然则云与贾母看,则先骂贾母矣。
此处邢夫人亦看,然则又骂邢夫人乎?故作者又难。
\zhu{
交媾之情连狗畜都知道,人亦知道。知此事并非就是狗畜。
批书人采用了归谬法论证这个道理,假设“狗不识”真的是骂认识交媾的人是狗,那么贾母和邢夫人知道交媾之事,难道就是狗吗?推导出了荒谬的结论,所以证明前提“狗不识”是骂人的为假命题。
}
}太太请瞧一瞧。
”说着,便送过去。
\par
邢夫人接来一看,吓得连忙死紧攥住,\geng{妙!这一“吓”字方是写世家夫人之笔。
虽前文明书邢夫人之为人稍劣,然\sout{不}[亦]在情理之中,若不用慎重之笔,则邢夫人直系一小家卑污极轻贱之人矣,岂得与荣府联房哉?所谓此书针线慎密处,全在无意中一字一句之间耳,看者细心方得。
}忙问:“你是那里得的?”傻大姐道:“我掏促织儿在山石上拣的。
”邢夫人道:“快休告诉一人。
这不是好东西,连你也要打死。
皆因你素日是傻子,以后再别提起了。
”这傻大姐听了,反吓的黄了脸,说:“再不敢了。
”磕了个头,呆呆而去。
邢夫人回头看时,都是些女孩儿,不便递与,自己便塞在袖内,心内十分罕异,揣摩此物从何而至,且不形于声色,且来至迎春室中。
\par
迎春正因他乳母获罪,自觉无趣,心中不自在,忽报母亲来了,遂接入内室。
奉茶毕,邢夫人因说道:“你这么大了,你那奶妈子行此事,你也不说说他。
如今别人都好好的,偏咱们的人做出这事来,什么意思。
”\geng{“咱们”二字便见自怀异心,从上文生离异发沥而来,\zhu{离异:指的是邢夫人和王熙凤王夫人矛盾嫌隙日深。
沥:液体一滴一滴地落下。
}谨密之至。
更有人[甚]于此者,君未知也,一笑。
}迎春低着头弄衣带,半晌答道:“我说他两次,他不听也无法。
况且他是妈妈,只有他说我的,没有我说他的。
”\geng{妙极!直画出一个懦弱小姐来。
}邢夫人道:“胡说!你不好了他原该说,如今他犯了法,你就该拿出小姐的身分来。
他敢不从,你就回我去才是。
如今直等外人共知,是什么意思。
\geng{我敬问:外人为谁?}再者,只他去放头儿,\zhu{放头儿:这里是聚赌、作头家的意思。
抽头:向赢钱的赌徒抽取一部分的利益给提供赌博场所的人。
也称为“拈头”。
头家:聚赌抽头的人。
聚赌抽头所得的钱叫头儿钱。
}还恐怕他巧言花语的和你借贷些簪环衣履作本钱,你这心活面软,未必不周接他些。
\zhu{周接:周济、接济。}
若被他骗去,我是一个钱没有的,看你明日怎么过节。
”迎春不语,只低头弄衣带。
\par
邢夫人见他这般,因冷笑道:“总是你那好哥哥好嫂子,一对儿赫赫扬扬,琏二爷凤奶奶,两口子遮天盖日,百事周到,竟通共这一个妹子,全不在意。
\geng{加\sout{在}[罪]于琏凤,的是父母常情,极是。
何必又如此说来,便见又有私意。
}但凡是我身上掉下来的,又有一话说——只好凭他们罢了。
\geng{如何?此皆妇女私假之意,\zhu{私假:秘密私心,虚情假意。
}大不可者。
}况且你又不是我养的,\zhu{养:这里指生孩子。
}\geng{更不好。
}
你虽然不是同他一娘所生,到底是同出一父,也该彼此瞻顾些,也免别人笑话。
\geng{又问:别人为谁?又问:彼二人虽不同母,终是同父。
彼二人既系同父,其父又系君之何人?吁!妇人私心,今古有之。
}我想天下的事也难较定,\zhu{较定:考核断定。
}
你是大老爷跟前人养的,这里探丫头也是二老爷跟前人养的,出身一样。
如今你娘死了,从前看来,你两个的娘,只有你娘比如今赵姨娘强十倍的,你该比探丫头强才是。
怎么反不及他一半?谁知竟不然,这可不是异事\foot{“怎么反不及他一半”、“谁知竟不然,这可不是异事”,此两句意思重复,而且叠加在一起,造成语气不连贯。
这也许是在传抄过程把母本中的初稿和改稿一并抄录的结果。
除甲辰本删去后句外,馀本均同底本。
}!倒是我一生无儿无女的,一生干净,也不能惹人笑话议论为高。
”\geng{最可恨妇人无嗣者引此话是说。
}旁边伺候的媳妇们便趁机道:“我们的姑娘老实仁德,那里像他们三姑娘伶牙俐齿,会要姊妹们的强。
\zhu{要强:争强好胜,不肯认输。
}他们明知姐姐这样,他竟不顾恤一点儿。
”\geng{杀杀杀!此辈专生离异。
余因实受其蛊,今读此文,直欲拔剑劈纸。
又不知作者多少眼泪洒出此回也。
又问:不知如何顾恤些?又不知有何可顾恤之处?直令人不解愚奴贱婢之言。
酷肖之至。
}邢夫人道:“连他哥哥嫂子还如是,别人又作什么呢。
”一言未了,人回:“琏二奶奶来了。
”邢夫人听了,冷笑两声,命人出去说:“请他自去养病,我这里不用他伺候。
”接着又有探春的小丫头来报说:“老太太醒了。
”邢夫人方起身前边来。
迎春送至院外方回。
\par
绣橘因说道:“如何,前儿我回姑娘,那一个攒珠累丝金凤竟不知那里去了。
\zhu{攒珠累丝金凤:以金丝穿聚珍珠堆叠连缀成凤形的发饰。
}回了姑娘,姑娘竟不问一声儿。
我说必是老奶奶拿去典了银子放头儿的,姑娘不信,只说司棋收着呢。
问司棋,司棋虽病着,心里却明白。
我去问他,他说没有收起来,还在书架上匣内暂放着,预备八月十五日恐怕要戴呢。
姑娘就该问老奶奶一声,只是脸软怕人恼。
如今竟怕无着,\zhu{无着:没有着落。
}明儿要都戴时,独咱们不戴,是何意思呢。
”\geng{这个“咱们”使得恰,是女儿喁喁私语,\zhu{喁:音“鱼”,模拟小声说话的声音。
}非前文之一例可比者。
写得出,批得出。
}\par
迎春道:“何用问,自然是他拿去暂时借一肩了。
\zhu{借一肩:挑担时让别人挑一会自己歇一会叫借力歇肩或借一肩,这里是借人之物典押得钱以应急用的意思。
}我只说他悄悄的拿了去,不过一日半晌,仍旧悄悄的送了来,谁知他就忘了。
今日偏又闹出来,问他想也无益。
”绣橘道:“何曾是忘记!他是试准了姑娘的性格,所以才这样。
如今我有个主意:我竟走到二奶奶房里将此事回了他,或他着人去要,或他省事拿几吊钱来替他赔补。
如何?”\geng{写女儿各有机变,个个不同。
}迎春忙道:“罢,罢,罢,省些事罢。
宁可没有了,又何必生事。
”\geng{总是懦语。
}绣橘道:“姑娘怎么这样软弱。
都要省起事来,将来连姑娘还骗了去呢,
\ping{后来迎春果真被孙绍祖骗走成婚受尽折磨。}
我竟去的是。
”说着便走。
迎春便不言语,只好由他。
\par
谁知迎春乳母子媳王住儿媳妇正因他婆婆得了罪,来求迎春去讨情,听他们正说金凤一事,且不进去。
也因素日迎春懦弱,他们都不放在心上。
如今见绣橘立意去回凤姐,估着这事脱不去的,且又有求迎春之事,只得进来,陪笑先向绣橘说:“姑娘,你别去生事。
姑娘的金丝凤,原是我们老奶奶老糊涂了,输了几个钱,没的捞梢,\zhu{捞梢:赌博中称翻本为“捞梢”,即把输了的钱赢回来。
}所以暂借了去。
原说一日半晌就赎的,因总未捞过本儿来,就迟住了。
可巧今儿又不知是谁走了风声,弄出事来。
虽然这样,到底主子的东西,我们不敢迟误下,终久是要赎的。
如今还要求姑娘看从小儿吃奶的情常,\zhu{情常:情分。
}往老太太那边去讨个情面,救出他老人家来才好。
”迎春先便说道:“好嫂子,你趁早儿打了这妄想,要等我去说情儿,等到明年也不中用的。
方才连宝姐姐林妹妹大伙儿说情,老太太还不依,何况是我一个人。
我自己愧还愧不来,反去讨臊去。
”绣橘便说:“赎金凤是一件事,说情是一件事,别绞在一处说。
难道姑娘不去说情,你就不赎了不成?嫂子且取了金凤来再说。
”\par
王住儿家的听见迎春如此拒绝他,绣橘的话又锋利无可回答,一时脸上过不去,也明欺迎春素日好性儿,乃向绣橘发话道:“姑娘,你别太仗势了。
你满家子算一算,谁的妈妈奶子不仗着主子哥儿多得些益,偏咱们就这样丁是丁卯是卯的,\zhu{丁是丁卯是卯:某个钉子一定要安在相应的铆处,不能有差错。
形容对事认真,毫不含糊。
另一种说法,丁为天干之一,卯为地支之一,有错就会影响农历推算。
又丁为物之凸出者,即榫头;卯为物之凹入者,即卯眼,二者若错就会安装不上。
表示做事认真、不马虎,含有不肯通融之意。
}只许你们偷偷摸摸的哄骗了去。
自从邢姑娘来了,太太吩咐一个月俭省出一两银子来与舅太太去,这里饶添了邢姑娘的使费,\zhu{饶:额外增添。
}反少了一两银子。
常时短了这个,少了那个,那不是我们供给?谁又要去?不过大家将就些罢了。
算到今日,少说些也有三十两了。
我们这一向的钱,岂不白填了限呢。
”\zhu{填限:也作“填馅”,代人受过、白白充当牺牲品的意思。
}绣橘不待说完,便啐了一口,道:“作什么的白填了三十两,我且和你算算帐,姑娘要了些什么东西?”迎春听见这媳妇发邢夫人之私意,\zhu{邢夫人之私意:第五十七回,邢岫烟说邢夫人要她每月省出一两月钱给自己的爸妈,要用什么就和迎春伙着用就行了。
}\geng{大书此句,诛心之笔。
}忙止道:“罢,罢,罢。
你不能拿了金凤来,不必牵三扯四乱嚷。
我也不要那凤了。
便是太太们问时,我只说丢了,也妨碍不着你什么的,出去歇息歇息倒好。
”一面叫绣橘倒茶来。
\par
绣橘又气又急,因说道:“姑娘虽不怕,我们是作什么的,把姑娘的东西丢了。
他倒赖说姑娘使了他们的钱,这如今竟要准折起来。
\zhu{准折:抵销,弥补,抵偿。
}倘或太太问姑娘为什么使了这些钱,敢是我们就中取势了?\zhu{取势:得到利益。
}这还了得!”一行说,
\zhu{一行[xíng]:一边。}
一行就哭了。
司棋听不过,只得勉强过来,帮着绣橘问着那媳妇。
迎春劝止不住,自拿了一本《太上感应篇》来看。
\zhu{《太上感应篇》:书名,晋代葛洪(自号抱朴子)托名道家始祖太上老君之名所作,旨在劝善惩恶,宣扬因果报应。
}\geng{神妙之至!从纸上跳出一位懦弱小姐,且书又有奇,大妙!}\ping{要是懦弱就显出畏惧也还好,无端宽容更令人生气,因为这似乎显出追究的人反而不宽容是个坏人。
}\par
三人正没开交,可巧宝钗、黛玉、宝琴、探春等因恐迎春今日不自在,都约来安慰他。
走至院中,听得两三个人较口。
\zhu{较口:争吵。
}探春从纱窗内一看,只见迎春倚在床上看书,若有不闻之状。
\geng{看他写迎春,虽稍劣,然亦大家千金之格也。
}探春也笑了。
小丫鬟们忙打起帘子,报道:“姑娘们来了。
”迎春方放下书起身。
那媳妇见有人来,且又有探春在内,不劝而自止了,遂趁便要去。
\par
探春坐下,便问:“才刚谁在这里说话?倒像拌嘴似的。
”\geng{瞧他写探春气宇。
}迎春笑道:“没有说什么,左不过是他们小题大作罢了。
\zhu{左不过:反正,只不过,无非。
}
何必问他。
”探春笑道:“我才听见什么‘金凤’,又是什么‘没有钱只和我们奴才要’,谁和奴才要钱了?难道姐姐和奴才要钱了不成?难道姐姐不是和我们一样有月钱的,一样有用度不成?”司棋绣橘道:“姑娘说的是了。
姑娘们都是一样的,那一位姑娘的钱不是由着奶奶妈妈们使,连我们也不知道怎么是算帐,不过要东西只说得一声儿。
如今他偏要说姑娘使过了头儿,他赔出许多来了。
究竟姑娘何曾和他要什么了。
”
\ping{下人贪了迎春的钱却反咬一口说我们一直在帮你贴钱。}
\par
探春笑道:“姐姐既没有和他要,必定是我们或者和他们要了不成!你叫他进来,我倒要问问他。
”迎春笑道:“这话又可笑。
你们又无沾碍,何得带累于他。
”探春笑道:“这倒不然。
我和姐姐一样,姐姐的事和我的也是一般,他说姐姐就是说我。
我那边的人有怨我的,姐姐听见也即同怨姐姐是一理。
咱们是主子,自然不理论那些钱财小事,只知想起什么要什么,也是有的事。
但不知金累丝凤因何又夹在里头?”那王住儿媳妇生恐绣橘等告出他来,遂忙进来用话掩饰。
探春深知其意,因笑道:“你们所以糊涂。
如今你奶奶已得了不是,趁此求求二奶奶,把方才的钱尚未散人的拿出些来赎取了就完了。
比不得没闹出来,大家都藏着留脸面,如今既是没了脸,趁此时纵有十个罪,也只一人受罚,没有砍两颗头的理。
你依我,竟是和二奶奶说说。
在这里大声小气,如何使得。
”\par
这媳妇被探春说出真病,也无可赖了,只不敢往凤姐处自首。
探春笑道:“我不听见便罢,既听见,少不得替你们分解分解。
”谁知探春早使个眼色与待书出去了。
\par
这里正说话,忽见平儿进来。
宝琴拍手笑说道:“三姐姐敢是有驱神召将的符术?”黛玉笑道:“这倒不是道家玄术,倒是用兵最精的,所谓‘守如处女,脱如狡兔’,\zhu{守如处女,脱如狡兔:喻出其不意的举动。
《孙子·九地》:“始如处女,敌人开户,后如脱兔,敌不及拒。
”这是说作战时开始要像处女一样沉静,使敌人放松戒备,然后像脱兔一样迅捷,使敌人不及抗拒。
}出其不备之妙策也。
”二人取笑。
宝钗便使眼色与二人,令其不可,遂以别话岔开。
探春见平儿来了,遂问:“你奶奶可好些了?真是病糊涂了,事事都不在心上,叫我们受这样的委曲。
”平儿忙道:“姑娘怎么委曲?谁敢给姑娘气受,姑娘快吩咐我。
”\par
当时住儿媳妇儿方慌了手脚,遂上来赶着平儿叫:“姑娘坐下,让我说原故请听。
”平儿正色道:“姑娘这里说话,也有你我混插口的礼!你但凡知礼,只该在外头伺候。
不叫你,进不来的地方,几时有外头的媳妇子们无故到姑娘们房里来的?”绣橘道:“你不知我们这屋里是没礼的,谁爱来就来。
”平儿道:“都是你们的不是。
姑娘好性儿,你们就该打出去,然后再回太太去才是。
”王住儿媳妇见平儿出了言,红了脸方退出去。
\par
探春接着道:“我且告诉你,若是别人得罪了我,倒还罢了。
如今那住儿媳妇和他婆婆仗着是妈妈,又瞅着二姐姐好性儿,如此这般私自拿了首饰去赌钱,而且还捏造假帐妙算,威逼着还要去讨情,和这两个丫头在卧房里大嚷大叫,二姐姐竟不能辖治,所以我看不过,才请你来问一声:还是他原是天外的人,不知道理?还是谁主使他如此,先把二姐姐制伏,然后就要治我和四姑娘了?”平儿忙陪笑道:“姑娘怎么今日说这话出来?我们奶奶如何当得起!”\par
探春冷笑道:“俗语说的,‘物伤其类’,‘齿竭唇亡’,\zhu{齿竭唇亡:亦作“唇亡齿寒”,比喻两者相互依存,利害关系十分密切。
}我自然有些惊心。
”平儿道:“若论此事,还不是大事,极好处置。
但他现是姑娘的奶嫂,
\zhu{奶嫂:奶哥哥的妻子。奶哥哥:乳母的比自己年长的儿子;又称奶兄。}
据姑娘怎么样为是?”当下迎春只和宝钗阅《感应篇》故事,究竟连探春之语亦不曾闻得,忽见平儿如此说,乃笑道:“问我,我也没什么法子。
他们的不是,自作自受,我也不能讨情,我也不去苛责就是了。
至于私自拿去的东西,送来我收下,不送来我也不要了。
太太们要问,我可以隐瞒遮饰过去,是他的造化,若瞒不住,我也没法,没有个为他们反欺枉太太们的理,少不得直说。
你们若说我好性儿,没个决断,竟有好主意可以八面周全,不使太太们生气,任凭你们处治,我总不知道。
”\par
众人听了,都好笑起来。
黛玉笑道:“真是‘虎狼屯于阶陛,尚谈因果’。
\zhu{虎狼屯于阶陛,尚谈因果:屯:聚集;驻扎。
阶陛:宫殿的台阶。
此语原意在讽喻封建帝王佞信佛道以致祸国殃民。
如南朝梁武帝萧衍,当叛将侯景的军队己打到京师围困台城,他还一心皈依佛教奢谈因果。
这里是指对与己生死攸关的事,采取不闻不问的态度。
}若使二姐姐是个男人,这一家上下若许人,又如何裁治他们。
”迎春笑道:“正是。
多少男人尚如此,何况我哉?”一语未了,只见又有一人进来。
正不知道是那个,且听下回分解。
\par
\qi{总评:一篇奸盗淫邪文字,反以四子书五经、\zhu{四子书五经:四书五经,四书指《大学》《中庸》《论语》《孟子》;五经指《诗经》《书经》《礼记》《易经》《春秋》。
它们是儒家的主要经典。
}《公羊》、《谷梁》、秦汉诸作起,以《太上感应篇》结,彼何心哉!他深见“书中自有黄金屋”、“书中有女美如玉”等语误尽天下苍生,而大奸大盗皆从此出。
故特作此一起结,为五阴浊世顶门一声棒喝也。
\zhu{五阴:又作“五蕴”,“蕴”是集聚、类别的意思,佛教将包括个人身心与身心环境的一切物质与精神分成五种“聚集”,分别是色、受、想、行、识,故称为五蕴,是对世间所有的物质和精神现象的总结和归纳。
棒喝:为禅门启悟的一种方式,禅宗以为心外无佛,反对通过语言、文字、概念等外在方式求悟,主张内心自悟,因此在师受时不采用言教,而是采用棒打和声喝之类方式以使学禅者开悟。
后来,人们借用这一概念来表示使执迷不悟者警醒。
}眼空似箕,\zhu{
箕[jī]:星宿名,二十八宿之一。 
眼空似箕:这里应该是形容作者眼界高。
}笔大如椽,\zhu{笔大如椽:比喻大作家、大书法家的大手笔。
《晋书·王珣传》:珣梦人以大笔如掾与之,既觉,语人云:“此当有大手笔事。
”俄而,帝崩,哀册谥议,皆珣所草。
}
何得以寻行数墨绳之。
\zhu{寻行数墨:
这里应该是形容文字之少。
}\hang
探春处处出头,人谓其能,吾谓其苦;迎春处处藏舌,人谓其怯,吾谓其超。
探春运符咒,固足役鬼驱神;迎春说因果,更可降狼伏虎。
}
\dai{145}{痴丫头误拾绣春囊}
\dai{146}{懦小姐不问累金凤}
\sun{p73-1}{懦小姐不问累金凤}{迎春奶妈拿迎春的金凤质押换钱来赌博,且赖着不赎。
迎春向来懦弱,欲息事宁人。
绣橘、司棋出来争执,正没开交时,可巧探春几个来看迎春。
探春替迎春出头,教训了欺软怕硬的奴才。
}