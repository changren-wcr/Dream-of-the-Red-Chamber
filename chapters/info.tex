\chapter{制 作 说 明}

本书主要从两本书为底本进行整理汇编,一是《红楼梦脂评汇校本》,二是人民文学出版社1996年12月北京第2版《红楼梦》。
前者提供原文和脂批,后者提供注释。
\par
本书采取的新型注释形式是:注释插入到正文中被注释文字的紧后方,并和正文有效隔离。
以往的书中的注释,都是置于每一章节末尾,或者置于每一页下方脚注。
置于章节末尾的注释带来了查阅的困难,需要读者不停的翻书。
置于每页下方脚注的注释带来了查阅的便利,读者不需要翻书即可在本页找到注释,但是仍然存在需要读者将目光从正文中跳转出来到本页下方的脚注区域,再根据编号找到对应的注释,读完注释后再跳转出脚注区域回到正文,当注释较多时这将会带来频繁的视线跳转,不利于读者流畅阅读,降低了阅读的舒适性。
\par
诚然,必须要承认过去的出版形式有其合理之处,因为一本书要面向各种类型的读者。
对于初读者,需要看注释才能理解,注释插入到正文之中更合适,这样就避免了频繁的视线跳转,增强了阅读体验;对于再读者,基本不需要注释的帮助就能理解,冗余的注释插入到正文打断正文行文,削弱了阅读体验,注释排列于章节末尾或者页末尾更合适,这样在需要的时候能够起到参考的作用而不打断正文。
\par
由上述分析可见,传统注释形式更适合再读者,新型注释形式更适合初读者。
本书作为一个可以在线根据需求自我定制的文档,不仅可以将注释的位置从文中移动到页脚,而且可以在文本中加入自己的理解,写出自己的“脂批”,还可以改变文档字体页面大小,适合多种尺寸的电子设备。
\par
本书正文以甲戌本及庚辰本为底本(第六十四、六十七回以列藏本为底本),以其他各脂本参校。
甲戌本所存十六回,文字大大优于他本,基本原文照录,只在确有必要时改动少量字词;庚辰本内容比较完整,惜抄写草率,错误甚多,不得不参照己卯本等本子校改大量字句。
第六十七回两种版本文字差异过大,无法互校,一并附录。
\par
为节省篇幅,本书辑录批语使用略字,意义如下:\jia{}(甲戌本)、\ji{}(己卯本)、\geng{}(庚辰本)、\qi{}(戚序本)、\meng{}(蒙府本)、\lie{}(列藏本)、\yang{}(杨藏本)、\chen{}(甲辰本)。
\par
本书校改文字记号:\sout{某}表示删字,[某]表示补字,\sout{甲}[乙]表示改字。
