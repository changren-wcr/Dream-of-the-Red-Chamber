\chapter{林潇湘魁夺菊花诗 \quad 薛蘅芜讽和螃蟹咏}
\ji{题曰“菊花诗”、“螃蟹咏”,偏自太君前阿凤若许诙谐中不失体、\zhu{体:礼体、礼数。
}鸳鸯平儿宠婢中多少放肆之迎合取乐写来,似难入题,却轻轻用弄水戏鱼看花等游玩事及王夫人云“这里风大”一句收住入题,并无纤毫牵强,此重作轻抹法也。
\zhu{
重作:重要的创作需要重笔着力描写。轻抹:这里指用轻描淡写。
重作轻抹表现了作者写作功力之深,举重若轻。
}
妙极!好看煞!}\par
话说宝钗湘云二人计议已妥,一宿无话。
湘云次日便请贾母等赏桂花。
贾母等都说道:“是他有兴头,须要扰他这雅兴。
”\ji{若在世俗小家,则云:“你是客,在我们舍下,怎么反扰你的呢?”一何可笑。
}至午,果然贾母带了王夫人凤姐兼请薛姨妈等进园来。
贾母因问:“那一处好?”\ji{必如此问方好。
}王夫人道:“凭老太太爱在那一处,就在那一处。
”\ji{必是王夫人如此答方妙。
}凤姐道:“藕香榭已经摆下了,那山坡下两颗桂花开的又好,河里的水又碧清,坐在河当中亭子上岂不敞亮,看着水眼也清亮。
”
\zhu{水眼:泉眼。}
\ji{智者乐水,岂其然乎?}贾母听了,说:“这话很是。
”说着,就引了众人往藕香榭来。
原来这藕香榭盖在池中,四面有窗,左右有曲廊可通,亦是跨水接岸,后面又有曲折竹桥暗接。
众人上了竹桥,凤姐忙上来搀着贾母,口里说:“老祖宗只管迈大步走,不相干的,这竹子桥规矩是咯吱咯喳的。
”\ji{如见其势,如临其上,非走过者形容不到。
}\par
一时进入榭中,\zhu{榭:音“谢”,建在高土台或水面(或临水)上的木屋。
}只见栏杆外另放着两张竹案,一个上面设着杯箸酒具,一个上头设着茶筅茶盂各色茶具。
\zhu{
茶筅(筅音“险”):用竹子做的洗涤茶具的刷箒。
茶盂[yú]:茶具,用于盛接废弃茶水的器具。
}
那边有两三个丫头煽风炉煮茶,这一边另外几个丫头也煽风炉烫酒呢。
贾母喜的忙问:“这茶想的到,且是地方,东西都干净。
”湘云笑道:“这是宝姐姐帮着我预备的。
”贾母道:“我说这个孩子细致,凡事想的妥当。
”一面说,一面又看见柱上挂的黑漆嵌蚌的对子,\zhu{嵌蚌:即嵌螺钿,将蚌壳有光泽的部分加工嵌入器物表面的装饰工艺。
}命人念。
湘云念道:\par
\hop
芙蓉影破归兰桨,菱藕香深写竹桥。
\zhu{
芙蓉:荷花。上句的意境是从王维《山居秋暝》“竹喧归浣女,莲动下渔舟”得到启发。
在技巧上同样借鉴了先果后因的手法,先闻声见动,后方出人。上句若写成“兰桨归时莲影破”就平淡无奇了,
写:画。下句说竹桥架于菱藕水面,恰如画出。
另一种解释,程高本“写”作“泻”,流溢。桥下有水,荷香如水一般泛滥,也可以理解为菱藕香的深处,流水泻过竹桥。
}\ji{妙极!此处忽又补出一处不入贾政“试才”一回,皆错综其事,不作一直笔也。
}\par
\hop
贾母听了,又抬头看匾,因回头向薛姨妈道:“我先小时,家里也有这么一个亭子,叫做什么‘枕霞阁’。
我那时也只像他们这么大年纪,同姊妹们天天顽去。
那日谁知我失了脚掉下去,几乎没淹死,好容易救了上来,\zhu{好容易:好不容易。
}到底被那木钉把头碰破了。
如今这鬓角上那指头顶大一块窝儿就是那残破了。
众人都怕经了水,又怕冒了风,都说活不得了,谁知竟好了。
”\par
凤姐不等人说,先笑道:“那时要活不得,如今这大福可叫谁享呢!可知老祖宗从小儿的福寿就不小,神差鬼使碰出那个窝儿来,好盛福寿的。
寿星老儿头上原是一个窝儿,因为万福万寿盛满了,所以倒凸高出些来了。
”未及说完,贾母与众人都笑软了。
\ji{看他忽用贾母数语,闲闲又补出此书之前似已有一部《十二钗》的一般,令人遥忆不能一见,余则将欲补出枕霞阁中十二钗来,岂不又添一部新书?}贾母笑道:“这猴儿惯的了不得了,只管拿我取笑起来,恨的我撕你那油嘴。
”凤姐笑道:“回来吃螃蟹,恐积了冷在心里,讨老祖宗笑一笑开开心,一高兴多吃两个就无妨了。
”贾母笑道:“明儿叫你日夜跟着我,我倒常笑笑觉的开心,不许回家去。
”王夫人笑道:“老太太因为喜欢他,才惯的他这样,还这样说,他明儿越发无礼了。
”贾母笑道:“我喜欢他这样,况且他又不是那不知高低的孩子。
家常没人,娘儿们原该这样。
横竖礼体不错就罢,没的倒叫他从神儿似的作什么。
”\zhu{从神儿:跟从神仙,形容拘谨严肃的样子。
}\ji{近之暴发专讲礼法,竟不知礼法,此似无礼而礼法井井,所谓“整瓶不动半瓶摇”,又曰“习惯成自然”,真不谬也。
}
\ping{王夫人在贾母面前就是恭顺有余灵活不足“从神儿似的”,这里贾母在暗刺王夫人。}
\par
说着,一齐进入亭子,献过茶,凤姐忙着搭桌子,要杯箸。
上面一桌,贾母、薛姨妈、宝钗、黛玉、宝玉;东边一桌,史湘云、王夫人、迎、探、惜;西边靠门一桌,李纨和凤姐的,虚设坐位,二人皆不敢坐,只在贾母王夫人两桌上伺候。
凤姐吩咐:“螃蟹不可多拿来,仍旧放在蒸笼里,拿十个来,吃了再拿。
”一面又要水洗了手,站在贾母跟前剥蟹肉,头次让薛姨妈。
薛姨妈道:“我自己掰着吃香甜,不用人让。
”凤姐便奉与贾母。
二次的便与宝玉,又说:“把酒烫的滚热的拿来。
”又命小丫头们去取了菊花叶儿、桂花蕊熏的绿豆面子来,\zhu{绿豆面子:即绿豆粉,经桂花和菊花叶之类熏过,供洗去手上螃蟹的油腥用的。
}预备着洗手。
史湘云陪着吃了一个,就下座来让人,又出至外头,令人盛两盘子与赵姨娘周姨娘送去。
又见凤姐走来道:“你不惯张罗,你吃你的去。
我先替你张罗,等散了我再吃。
”湘云不肯,又令人在那边廊上摆了两桌,让鸳鸯、琥珀、彩霞、彩云、平儿去坐。
鸳鸯因向凤姐笑道:“二奶奶在这里伺候,我们可吃去了。
”凤姐儿道:“你们只管去,都交给我就是了。
”说着,史湘云仍入了席。
凤姐和李纨也胡乱应个景儿。
\par
凤姐仍是下来张罗,一时出至廊上,鸳鸯等正吃的高兴,见他来了,鸳鸯等站起来道:“奶奶又出来作什么?让我们也受用一会子。
”凤姐笑道:“鸳鸯小蹄子越发坏了,我替你当差,倒不领情,还抱怨我。
还不快斟一钟酒来我喝呢。
”鸳鸯笑着忙斟了一杯酒,送至凤姐唇边,凤姐一扬脖子吃了。
平儿早剔了一壳黄子送来,凤姐道:“多倒些姜醋。
”一面也吃了,笑道:“你们坐着吃罢,我可去了。
”\par
鸳鸯笑道:“好没脸,吃我们的东西。
”凤姐儿笑道:“你和我少作怪。
你知道你琏二爷爱上了你,要和老太太讨了你做小老婆呢。
”鸳鸯道:“啐,这也是作奶奶说出来的话!我不拿腥手抹你一脸算不得。
”说着赶来就要抹。
凤姐儿央道:“好姐姐,饶我这一遭儿罢。
”琥珀笑道:“鸳丫头要去了,平丫头还饶他?你们看看他,没有吃了两个螃蟹,倒喝了一碟子醋,他也算不会揽酸了。
”\zhu{揽酸:吃醋。
}\ping{平儿“算不会揽酸”,暗指凤姐更容易吃醋。
这里的吃醋,既指蘸醋吃,也指嫉妒。
}平儿手里正掰了个满黄的螃蟹,听如此奚落他,便拿着螃蟹照着琥珀脸上抹来,口内笑骂“我把你这嚼舌根的小蹄子!”琥珀也笑着往旁边一躲,平儿使空了,往前一撞,正恰恰的抹在凤姐儿腮上。
凤姐儿正和鸳鸯嘲笑,不防唬了一跳,嗳哟了一声。
众人撑不住都哈哈的大笑起来。
凤姐也禁不住笑骂道:“死娼妇!吃离了眼了,\zhu{离了眼:瞎了眼。
}混抹你娘的。
”平儿忙赶过来替他擦了,亲自去端水。
鸳鸯道:“阿弥陀佛!这是个报应。
”\par
贾母那边听见,一叠声问:“见了什么这样乐,告诉我们也笑笑。
”鸳鸯等忙高声笑回道:“二奶奶来抢螃蟹吃,平儿恼了,抹了他主子一脸的螃蟹黄子。
主子奴才打架呢。
”贾母和王夫人等听了也笑起来。
贾母笑道:“你们看他可怜见的,把那小腿子脐子给他点子吃也就完了。
”鸳鸯等笑着答应了,高声又说道:“这满桌子的腿子,二奶奶只管吃就是了。
”凤姐洗了脸走来,又伏侍贾母等吃了一回。
黛玉独不敢多吃,只吃了一点儿夹子肉就下来了。
\par


贾母一时不吃了,大家方散,都洗了手,也有看花的,也有弄水看鱼的,游玩了一回。
王夫人因回贾母说:“这里风大,才又吃了螃蟹,老太太还是回房去歇歇罢了。
若高兴,明日再来逛逛。
”贾母听了,笑道:“正是呢。
我怕你们高兴,我走了又怕扫了你们的兴。
既这么说,咱们就都去吧。
”回头又嘱咐湘云:“别让你宝哥哥、林姐姐多吃了。
”湘云答应着。
又嘱咐湘云、宝钗二人说:“你两个也别多吃。
那东西虽好吃,不是什么好的,吃多了肚子疼。
”二人忙应着送出园外,仍旧回来,令将残席收拾了另摆。
宝玉道:“也不用摆,咱们且作诗。
把那大团圆桌就放在当中,酒菜都放着。
也不必拘定坐位,有爱吃的大家去吃,散坐岂不便宜。
”宝钗道:“这话极是。
”湘云道:“虽如此说,还有别人。
”因又命另摆一桌,拣了热螃蟹来,请袭人、紫鹃、司棋、待书、入画、莺儿、翠墨等一处共坐。
山坡桂树底下铺下两条花毡,命答应的婆子并小丫头等也都坐了,只管随意吃喝,等使唤再来。
\par
湘云便取了诗题,用针绾在墙上。
\zhu{绾:音“碗”,系。
}众人看了,都说:“新奇固新奇,只怕作不出来。
”湘云又把不限韵的原故说了一番。
宝玉道:“这才是正理,我也最不喜限韵。
”林黛玉因不大吃酒,又不吃螃蟹,自令人掇了一个绣墩倚栏杆坐着,\zhu{掇:音“多”,拾取。
绣墩:有文饰彩绣或彩绘的坐墩。
其上或加绣花套子。
}拿着钓竿钓鱼。
宝钗手里拿着一枝桂花玩了一回,俯在窗槛上掐了桂蕊掷向水面,引的游鱼浮上来唼喋。
\zhu{唼喋:音“霎闸”,鱼或水鸟聚食声。
这里指鱼嘴开合,咂水吞食。
}湘云出一回神,又让一回袭人等,又招呼山坡下的众人只管放量吃。
探春和李纨惜春立在垂柳阴中看鸥鹭。
迎春又独在花阴下拿着花针穿茉莉花。
\zhu{
花针:有针鼻儿可以连着线把花穿起来,做成手链或项链。
}
\ji{看他各人各式,亦如画家有孤耸独出则有攒三聚五,疏疏密密,直是一幅《百美图》。
}
宝玉又看了一回黛玉钓鱼,一回又俯在宝钗旁边说笑两句,一回又看袭人等吃螃蟹,自己也陪他饮两口酒。
袭人又剥一壳肉给他吃。
\par
黛玉放下钓竿,走至座间,拿起那乌银梅花自斟壶来,\zhu{乌银:一种夹用硫磺、特殊方法熔铸的黑色银质。
乌银梅花自斟壶:用乌银所制、上有梅花图案、随手自斟用的小酒壶。
}\ji{写壶非写壶,正写黛玉。
}拣了一个小小的海棠冻石蕉叶杯。
\zhu{
海棠:指酒杯的式样形如四瓣状的海棠花。
冻石:是一种半透明的名贵石头,因晶莹如冰冻而得名。
蕉叶杯:一种较浅的杯子,形似蕉叶。
}\ji{妙杯!非写杯,正写黛玉。
“拣”字有神理,盖黛玉不善饮,此任性也。
}丫鬟看见,知他要饮酒,忙着走上来斟。
黛玉道:“你们只管吃去,让我自斟,这才有趣儿。
”说着便斟了半盏,看时却是黄酒,因说道:“我吃了一点子螃蟹,觉得心口微微的疼,须得热热的喝口烧酒。
”宝玉忙道:“有烧酒。
”便令将那合欢花浸的酒烫一壶来。
\zhu{浸:泡,这里是酿造的意思。
}\ji{伤哉!作者犹记矮䫜舫前以合欢花酿酒乎?屈指二十年矣。
\zhu{矮䫜舫究在何地,尚不清楚。从此句批语看,《红楼梦》作者曹雪芹及作批者二十年前均曾参与此事。}
}黛玉也只吃了一口便放下了。
\par
宝钗也走过来,另拿了一只杯来,也饮了一口,便蘸笔至墙上把头一个《忆菊》勾了,底下又赘了一个“蘅”字。
\ji{妙极韵极!}宝玉忙道:“好姐姐,第二个我已经有了四句了,你让我作罢。
”宝钗笑道:“我好容易有了一首,你就忙的这样。
”黛玉也不说话,接过笔来把第八个《问菊》勾了,接着把第十一个《菊梦》也勾了,也赘一个“潇”字。
\ji{这两个妙题料定黛卿必喜,岂让人作去哉?}宝玉也拿起笔来,将第二个《访菊》也勾了,也赘上一个“绛”字。
探春走来看看道:“竟没有人作《簪菊》,让我作这《簪菊》。
”又指着宝玉笑道:“才宣过总不许带出闺阁字样来,你可要留神。
”\par
说着,只见史湘云走来,将第四第五《对菊》《供菊》一连两个都勾了,也赘上一个“湘”字。
探春道:“你也该起个号。
”湘云笑道:“我们家里如今虽有几处轩馆,我又不住着,借了来也没趣。
”\ji{近之不读书暴发户偏爱起一别号。
一笑。
}宝钗笑道:“方才老太太说,你们家也有这个水亭叫‘枕霞阁’,难道不是你的。
如今虽没了,你到底是旧主人。
”众人都道有理,宝玉不待湘云动手,便代将“湘”字抹了,改了一个“霞”字。
\par
又有顿饭工夫,十二题已全,各自誊出来,都交与迎春,另拿了一张雪浪笺过来,\zhu{雪浪笺:一种白色诗笺,纸中有波浪形暗纹。
此外,还有一种厚的生宣纸,也叫雪浪纸。
}一并誊录出来,某人作的底下赘明某人的号。
李纨等从头看起:\par
\hop
忆菊 \quad 蘅芜君\ji{真用此号,妙极!}\par
怅望西风抱闷思,蓼红苇白断肠时。
\zhu{蓼:这里指红蓼,干叶均呈紫红色,夏秋之际开粉红小花。
苇:芦苇,夏秋之际扬白絮。
断肠:形容极度悲伤的情怀,这里喻忆菊心情之切。
}\par
空篱旧圃秋无迹,\zhu{空篱:菊篱之中空荡无物。
旧圃:去年的花圃。
秋无迹:即菊无迹,没有菊。用“秋”字代指菊花,好像菊花是秋天的灵魂。
}瘦月清霜梦有知。
\zhu{瘦月清霜:极言秋夜寂冷的景色。
梦有知:在梦中才能见到。
知:见。
}\par
念念心随归雁远,寥寥坐听晚砧痴。
\zhu{砧:音“针”,捣衣声,古时妇女为远人作寒衣多于秋夜将衣捣平,故砧声多用以表达妇女秋夜捣衣怀念远人的意境。
}\par
谁怜我为黄花病,\zhu{黄花:菊花。
}慰语重阳会有期。
\zhu{重阳:即阴历九月初九日。
《易经》以九为阳数,因九月九日都逢阳数,故重九亦称重阳。
旧俗多在重阳节赏菊饮酒。
}\par
\hop
访菊 \quad 怡红公子\par
闲趁霜晴试一游,酒杯药盏莫淹留。
\zhu{“酒杯”句:谓不要因为贪杯饮酒或因病吃药而被绊住,不出访菊。
淹留:久留,滞留。
}\par
霜前月下谁家种,槛外篱边何处秋。
\zhu{秋:秋色,在这里指菊花。
}\par
蜡屐远来情得得,\zhu{屐:音“基”,有齿的木底鞋,古时多着以登山。
屐上打蜡,可防湿耐用。
得得:犹“特特”,特地。
这里作情致很高解。
}冷吟不尽兴悠悠。
\zhu{冷吟:即秋吟,冷秋吟诗。
}\par
黄花若解怜诗客,\zhu{解:懂得,理解。
怜:爱怜。
诗客:诗人,作诗者自指。
}休负今朝挂杖头。
\zhu{挂杖头:与首联下句呼应,指诗人杖头挂钱,沽酒访菊。
《晋书·阮籍传》:阮修为籍从子,“常步行,以百钱挂杖头,至酒肆,便独酣畅。
”}\par
\hop
种菊\quad  怡红公子\par
携锄秋圃自移来,篱畔庭前故故栽。
\zhu{故故:故意;特意。
}\par
昨夜不期经雨活,今朝犹喜带霜开。
\par
冷吟秋色诗千首,\zhu{秋色:指菊。
}醉酹寒香酒一杯。
\zhu{酹:音“泪”,洒酒以祭。
寒香:以菊花清冷的幽香代指菊。
}\par
泉溉泥封勤护惜,好知井迳绝尘埃。
\zhu{
好知:须知。
迳:同“径”。
井径:田间小路,此指种菊处。
绝:弃绝。
尘埃:喻指世俗社会。
下句是说,愿同井径亲近,而与尘埃相绝,意含愿与秋菊偕隐,超尘拔俗,息绝世俗的交游。
}\par
\hop
对菊\quad  枕霞旧友\par
别圃移来贵比金,\zhu{别圃:即远圃。
别:远。
}一丛浅淡一丛深。
\par
萧疏篱畔科头坐,\zhu{
萧疏:指秋天萧条疏落的景象。
畔:旁边。
科头:指不戴帽子,表示疏狂不羁。
}清冷香中抱膝吟。
\zhu{清冷香:指菊花。
}\par
数去更无君傲世,看来惟有我知音。
\zhu{数去:数算起来。
知音:知己好友。
本意为闻琴声而解弹奏者的心意。
见《列子·汤问》载钟子期听俞伯牙弹琴事。
}\par
秋光荏苒休辜负,相对原宜惜寸阴。
\zhu{惜寸阴:爱惜时间。
阴:光阴。
}\par
\hop
供菊\quad  枕霞旧友\zhu{供菊:折菊插于瓶中,放置室内供玩赏。
}\par
弹琴酌酒喜堪俦,\zhu{俦:音“愁”,伴侣。
}几案婷婷点缀幽。
\par
隔座香分三径露,\zhu{分:散发。
三径,指栽菊的庭院。
陶渊明《归去来辞》:“三径就荒,松菊犹存。
”}抛书人对一枝秋。
\zhu{上句中的“三径露”与下句“一枝秋”互文,均指菊。
这两句的意思是供菊人抛下看的书,隔着座位看到菊花并闻到供菊散发的香气。
}\par
霜清纸帐来新梦,\zhu{霜清:指秋天,秋日有霜。
上句意谓,因室中供菊,使得清秋季节纸帐中睡觉也出现别具新意的梦境。
}圃冷斜阳忆旧游。
\zhu{下句回忆未折未供之前在夕阳残照中游赏清冷菊圃时的情景,以衬托和加强当前供菊为友的喜悦亲切心情。
这就是下文黛玉说的“背面傅粉”手法。
}\par
傲世也因同气味,\zhu{同气味:情趣相同。
}春风桃李未淹留。
\zhu{春风桃李:喻追求世俗荣利的人。
淹留:久留,滞留。
}\zhu{这两句说自己同秋菊情操一样,故对在春风中摇曳弄姿的桃花李花从不驻足欣赏。
}\par
\hop
咏菊\quad  潇湘妃子\par
无赖诗魔昏晓侵,\zhu{无赖:纠缠不舍。
诗魔:指诗人不可抑止的创作冲动。
佛家以扰乱身心,迷恋外物,妨碍修行的心理活动为魔。
昏晓:犹言早早晚晚。
侵:侵扰。
}绕篱欹石自沉音。
\zhu{
欹:音“七”,倾斜。
欹石:即倚石。
沉音:即沉吟,沉思低诵。
}\par
毫端蕴秀临霜写,\zhu{毫端:笔尖。
蕴秀:谓饱含隽逸的才思和辞藻。
临霜:迎霜。
}口齿噙香对月吟。
\zhu{噙香:口含菊的清香。
“香”也兼喻丽辞佳句。
}\par
满纸自怜题素怨,\zhu{素怨:秋怨。
}片言谁解诉秋心。
\zhu{秋心:感秋而生的情怀。
合即“愁”字。
}\par
一从陶令平章后,\zhu{从:自从。
陶令:即陶渊明,曾做过彭泽县令。
平章:品评;议论。
陶渊明议论菊花作品:《归去来辞》:“三径就荒,松菊犹存。
”《饮酒》诗:“采菊东篱下,悠然见南山。
”}千古高风说到今。
\par
\hop
画菊\quad  蘅芜君\par
诗馀戏笔不知狂,\zhu{诗馀:吟诗之后。
戏笔:随兴挥笔作画。
不知狂:不觉得自己兴态狷狂。
}岂是丹青费较量。
\zhu{丹青:绘画用的红色、青色颜料,代指绘画。
较量:斟酌构思。
下句的意思是,绘画是即兴而作,不必费尽心思斟酌构思。
}\par
聚叶泼成千点墨,\zhu{聚叶:绘画术语,指画叶章法。
要有聚有散,疏密有致。
画谱上有“攒三聚五”之说。
泼墨:中国绘画的画法之一,笔力奔放,宛如水墨泼在纸上。
}攒花染出几痕霜。
\zhu{攒花:指画上一层层花瓣攒簇起来的花朵。
染:在花瓣上涂颜色。
霜:指画面上的菊花瓣。
}\par
淡浓神会风前影,\zhu{淡浓:指用水墨的浓淡表现出菊花的姿态。
神会:充分领会掌握描绘对象的精神,再形之于画,即追求神似。
风前影:指在风中摇曳的菊花。
}跳脱秋生腕底香。
\zhu{跳脱:也作“条脱”“条达”。
手镯的别称。
“跳脱”后来又作灵巧、活脱义用。
这里因写到“腕”,故兼含两义,既巧妙点出画菊者是女子,又形容画的很生动。
所以下句说,腕底画出的秋菊似乎散发出香气来。
}\par
莫认东篱闲采掇,\zhu{东篱:陶渊明《饮酒》诗:“采菊东篱下,悠然见南山。
”采掇(掇音“多”):采摘。
上句意谓不要认为这是东篱边栽的真菊花就随手去采。
}粘屏聊以慰重阳。
\zhu{粘屏:把菊画粘贴在屏风上。
聊:姑且。
慰重阳:重阳佳节借观画代赏菊,聊作慰藉。
}\par
\hop
问菊\quad 潇湘妃子\par
欲讯秋情众莫知,喃喃负手叩东篱。
\zhu{喃喃:低声自语。
负手:倒背手若有所思的样子。
叩:问。
东篱:代指菊。
}\par
孤标傲世偕谁隐,\zhu{孤标:孤高的风操。
标:树梢之最上部,引伸为出众之意。
}一样花开为底迟?\zhu{
一样花开:同样是开花。
为底迟:为什么你开得这样迟?底:何。
}\par
圃露庭霜何寂寞,雁归蛩病可相思?\zhu{雁归蛩病:秋雁南归,蟋蟀悲鸣。
蛩病:喻蟋蟀凄切的叫声。
}\par
休言举世无谈者,解语何妨片语时。
\zhu{解语:会说话,解人意。
下句的意思是,希望菊花哪怕和自己交谈片刻也好。
}\par
\hop
簪菊\quad 蕉下客\zhu{簪菊:采菊插在头上。
古时重阳节民间有簪菊的风俗。
}\par
瓶供篱栽日日忙,折来休认镜中妆。
\zhu{休认镜中妆:不要认为是妇女平常对镜的妆饰,因簪菊是重阳节的风俗。
}\par
长安公子因花癖,\zhu{长安公子:或指晚唐诗人杜牧,因其祖父杜佑在唐德宗、宪宗两朝为相,故称他为长安公子。
杜牧《九月齐山登高》诗:“尘世难逢开口笑,菊花须插满头归。
”}彭泽先生是酒狂。
\zhu{彭泽先生:即陶渊明,他爱菊,亦嗜酒。
南朝梁萧统《陶渊明传》:“江州刺史王弘欲识之(指陶渊明),不能致也。
……尝九月九日无酒,出宅边菊丛中坐,久之,满手把菊,忽值弘送酒至,即便就坐,醉而归。
”又云:“郡将尝候之,值其酿熟,取头上葛巾漉酒,漉毕,还复着之。
”}\par
短鬓冷沾三径露,\zhu{三径:指栽菊的庭院。
陶渊明《归去来辞》:“三径就荒,松菊犹存。
”三径露:指代菊。
}葛巾香染九秋霜。
\zhu{葛巾:东晋士人戴的一种用葛布作的便帽。
九秋霜:代指菊。
九秋:秋季三个月九十天,故称秋天为三秋或九秋。
}\par
高情不入时人眼,\zhu{高情:即簪菊的高尚情趣。
}拍手凭他笑路旁。
\zhu{这两句意谓时俗之人在路旁拍手嘲笑醉酒簪菊之人。
}\ping{一看就是探春的英气。
}\par
\hop
菊影\quad 枕霞旧友\par
秋光叠叠复重重,\zhu{秋光:秋季的风光,代指菊影。
}潜度偷移三径中。
\zhu{潜度偷移:菊影随着日光悄悄移动。
}\par
窗隔疏灯描远近,篱筛破月锁玲珑。
\zhu{这两句是说隔着窗纱透出疏淡的灯光,描绘出远近不同的菊影。
月光透过缝隙如筛的竹篱,投射到地上,被隔成片片碎块,好像将玲珑透剔的菊影幽闭了起来似的。
}\par
寒芳留照魂应驻,\zhu{寒芳:指菊花。
留照:留影。
魂应驻:菊花的精神应当留在菊影里。
驻:留;住。
}霜印传神梦也空。
\zhu{霜印:指菊影。
传神:指菊影表现出菊花的精神。
梦也空:像梦一样空幻不实。
}\par
珍重暗香休踏碎,\zhu{暗香:指月夜菊影。
北宋诗人林逋:“疏影横斜水清浅,暗香浮动月黄昏。
”}凭谁醉眼认朦胧。
\zhu{凭谁:不论是谁。
下句承上句“休踏碎”意,说月光下的菊影朦胧,就像醉眼看花一样。
因而需要仔细辨认,以免踏碎菊影。
}\par
\hop
菊梦\quad 潇湘妃子\par
篱畔秋酣一觉清,\zhu{酣:沉睡。
上句说篱畔秋菊酣睡一场,梦境也清雅不俗。
}和云伴月不分明。
\zhu{和云伴月:菊花在梦中伴随云月飘然高举。
不分明:指菊花梦中迷离恍惚的境界。
}\par
登仙非慕庄生蝶,\zhu{登仙:指菊花梦中宛如登临仙境。
庄生蝶:庄生,即庄周,庄生梦化蝴蝶事,见《庄子·齐物论》。
}忆旧还寻陶令盟。
\zhu{陶令盟:陶渊明和菊花在文化上联系紧密,好像菊花和他有过盟约那样。
}\ping{下句可能暗含木石前盟,绛珠仙子为了报答神瑛侍者的灌溉之恩,怀着对他的旧情下凡,用一生的眼泪还他,履行两人之间的前世盟约即木石前盟。
}\par
睡去依依随雁断,\zhu{依依:留恋依慕的样子。
雁断:飞雁远逝。
“睡去”句意谓菊花依恋地随着鸿雁南归而入梦。
}惊回故故恼蛩鸣。
\zhu{惊回:睡醒梦回。
故故:屡屡。
“惊回”句意谓令人着恼的蟋蟀悲鸣,屡屡惊醒好梦。
}\par
醒时幽怨同谁诉,衰草寒烟无限情。
\par
\hop
残菊\quad 蕉下客\par
露凝霜重渐倾欹,\zhu{露凝霜重:指由秋至冬的气候次第变化。
露凝:秋露因冷而凝。
霜重:指初冬的霜威。
倾欹(欹音“欺”):指菊衰残倾斜。
}宴赏才过小雪时。
\zhu{宴赏:指重阳设宴赏菊。
小雪:二十四节气之一,在阴历十月。
承上句谓自重阳迄于小雪,菊花由盛趋衰。
}\par
蒂有馀香金淡泊,\zhu{蒂有馀香:指菊花蒂上残留的花瓣。
金:黄金色。
淡泊:指花的颜色消褪。
}枝无全叶翠离披。
\zhu{翠:指绿叶。
离披:散乱的样子。
}\par
半床落月蛩声病,\zhu{蛩声病:喻蟋蟀凄切的叫声。
}万里寒云雁阵迟。
\par
明岁秋风知再会,暂时分手莫相思。
\ping{探春面对离别如此乐观,探春即使远嫁也可能过得不错。
}\par
\hop
众人看一首,赞一首,彼此称扬不已。
李纨笑道:“等我从公评来。
通篇看来,各有各人的警句。
今日公评:《咏菊》第一,《问菊》第二,《菊梦》第三,\ping{这三首都是黛玉的诗作,李纨上次诗社力捧薛宝钗问鼎魁首,这次却一反往日,力推林黛玉夺魁,这是为何?薛宝钗自己和史湘云“夜拟菊花题”,没和诗社的领导李纨商量,可能让李纨感觉自己上次被利用了一回之后就再次被冷落,心怀怨气,所以力挺宝钗的情敌林黛玉。
也有可能,上回李纨力挺薛宝钗是因为感激,这回李纨恢复了客观公正,而且黛玉写的三首诗确实都是顶尖水平。
}题目新,诗也新,立意更新,恼不得要推潇湘妃子为魁了;然后《簪菊》《对菊》《供菊》《画菊》《忆菊》次之。
”宝玉听说,喜的拍手叫“极是,极公道。
”黛玉道:“我那首也不好,到底伤于纤巧些。
”李纨道:“巧的却好,不露堆砌生硬。
”\par
黛玉道:“据我看来,头一句好的是‘圃冷斜阳忆旧游’,这句背面傅粉。
‘抛书人对一枝秋’已经妙绝,将供菊说完,没处再说,故翻回来想到未折未供之先,意思深透。
”李纨笑道:“固如此说,你的‘口齿噙香’句也敌的过了。
”探春又道:“到底要算蘅芜君沉着,‘秋无迹’、‘梦有知’,把个忆字竟烘染出来了。
”宝钗笑道:“你的‘短鬓冷沾’、‘葛巾香染’,也就把簪菊形容的一个缝儿也没了。
”湘云道:“‘偕谁隐’、‘为底迟’,真个把个菊花问的无言可对。
”李纨笑道:“你的‘科头坐’、‘抱膝吟’,竟一时也不能别开,\zhu{别开:分开。这里指诗人和菊花不能分开。
}菊花有知,也必腻烦了。
”说的大家都笑了。
\par
宝玉笑道:“我又落第。
难道‘谁家种’、‘何处秋’、‘蜡屐远来’、‘冷吟不尽’都不是访,‘昨夜雨’、‘今朝霜’都不是种不成?但恨敌不上‘口齿噙香对月吟’、‘清冷香中抱膝吟’、‘短鬓’、‘葛巾’、‘金淡泊’、‘翠离披’、‘秋无迹’、‘梦有知’这几句罢了。
”\ji{总写宝玉不及,妙极!}又道:“明儿闲了,我一个人作出十二首来。
”李纨道:“你的也好,只是不及这几句新巧就是了。
”\par
大家又评了一回,复又要了热蟹来,就在大圆桌子上吃了一回。
宝玉笑道:“今日持螯赏桂,亦不可无诗。
\ji{全是他忙,全是他不及。
妙极!}
我已吟成,谁还敢作呢?”说着,便忙洗了手提笔写出。
\ji{且莫看诗,只看他偏于如许一大回诗后又写一回诗,岂世人想得到的?}众人看道:\par
\hop
持螯更喜桂阴凉,\zhu{螯:音“熬”,螃蟹夹子。
}泼醋擂姜兴欲狂。
\par
饕餮王孙应有酒,\zhu{饕餮:音“滔帖”(帖,四声),本为传说中一种贪食的恶兽,后常以喻人贪馋嗜吃,贪婪无厌。
王孙:旧称贵族子弟为王孙。
}横行公子却无肠。
\zhu{横行公子:指蟹。
}\par
脐间积冷馋忘忌,\zhu{
我国传统医药学认为,蟹性寒,不可恣食,其脐(蟹贴腹的长形或团形的浅色甲壳)间积冷尤甚,故食蟹须用辛温发散的生姜、紫苏等来解它。
}指上沾腥洗尚香。
\par
原为世人美口腹,坡仙曾笑一生忙。
\zhu{坡仙:苏东坡的别称。
他在《老饕赋》中用铺张的笔法嘲笑一个贪馋忙吃的老饕,在《初到黄州》诗亦云:“自笑平生为口忙,老来事业转荒唐。
”此联盖融汇苏轼赋意诗境而成,自嘲为饱口腹而忙于吃蟹的狂态。
}\par
\hop
黛玉笑道:“这样的诗,要一百首也有。
”\ji{看他这一说。
}宝玉笑道:“你这会子才力已尽,不说不能作了,还贬人家。
”黛玉听了,并不答言,也不思索,提起笔来一挥,已有了一首。
众人看道:\par
\hop
铁甲长戈死未忘,\zhu{铁甲:喻蟹壳。
长戈:喻蟹螯和蟹脚。
}堆盘色相喜先尝。
\zhu{色相:佛家语,指一切可感触有形质之物的形状。
这里指熟蟹的形状。
}\par
螯封嫩玉双双满,\zhu{嫩玉:喻蟹螯内的白色嫩肉。
}壳凸红脂块块香。
\zhu{红脂:母蟹蒸熟后,腹内的脂状物呈橙红色,俗称蟹黄。
}\par
多肉更怜卿八足,助情谁劝我千觞。
\zhu{觞:音“伤”,酒杯。
}\qi{不脱自己身分。
}\par
对斯佳品酬佳节,桂拂清风菊带霜。
\par
\hop
宝玉看了正喝彩,黛玉便一把撕了,令人烧去,因笑道:“我的不及你的,我烧了他。
你那个很好,比方才的菊花诗还好,你留着他给人看。
”宝钗接着笑道:“我也勉强了一首,未必好,写出来取笑儿罢。
”说着也写了出来。
大家看时,写道是:\par
\hop
桂霭桐阴坐举觞,\zhu{
霭:云气。
桂霭:桂花香气。
}长安涎口盼重阳。
\zhu{长安涎口:代指京都那些好吃馋嘴的人。
}\par
眼前道路无经纬,\zhu{经纬:道路南北为“经”,东西为“纬”。
上句说螃蟹横行,从不管眼前道路的纵横。
这里指横行霸道的人,恣意而行,不管法度。
}皮里春秋空黑黄。
\zhu{皮里春秋:《晋书·褚裒[chǔpóu]传》载,桓彝品评褚裒的为人“有皮里春秋”,意即表面上不露好恶而内心深藏褒贬。
春秋:相传为孔子据鲁史修订而成的史书,叙事极简约,前人以为其字字均藏褒贬。
下句说蟹壳里仅有黑的膏膜和黄的蟹黄而已,讽寓世人心黑意险。
}\par
\hop
看到这里,众人不禁叫绝。
宝玉道:“写得痛快!我的诗也该烧了。
”\par
又看底下道:\par
\hop
酒未敌腥还用菊,性防积冷定须姜。
\par
于今落釜成何益,\zhu{釜:音“府”,锅。
}月浦空馀禾黍香。
\zhu{浦:水边。
下句说蟹食稻伤农,今蟹死,月夜水边只留下禾黍的芳香。
}\zhu{以螃蟹比喻心黑意险,横行霸道的人,螃蟹下场是进锅被吃,类似地,那些人也没有好下场。
}\par
\hop
众人看毕,都说这是食螃蟹绝唱,这些小题目,原要寓大意才算是大才,只是讽刺世人太毒了些。
说着,只见平儿复进园来。
不知作什么,且听下回分解。
\par
\qi{总评:请看此回中,闺中儿女能作此等豪情韵事,且笔下各能自尽其性情,毫不乖舛。
\zhu{乖:背离,不正常。
舛:音“穿”三声,违背,错谬。
}作者之锦心绣口,无庸赘渎。
\zhu{赘渎:啰唆。
}其用意之深,奖劝之勤,读此文者,亦不得轻忽,戒之。
}
\dai{075}{林潇湘魁夺菊花诗}
\dai{076}{吃螃蟹平儿失手抹了凤姐一脸蟹黄}
\sun{p38-1}{藕香榭赏桂花吃螃蟹}{湘云在藕香榭设下螃蟹宴,山坡下两棵桂树开得正旺。
图右侧李纨和凤姐不敢坐,只在贾母王夫人两桌上伺候。
图左侧丫头们忙着煮茶烫酒,凤姐和丫鬟们调笑打闹时,平儿失手抹了凤姐一脸蟹黄。
}
\sun{p38-2}{姊妹斟酌菊花诗}{图中部:湘云令人在桂树下铺下两条花毯,令婆子和小丫头们随意吃。
用针把诗题绾在墙上,众姊妹各自斟酌起来。
图右侧:黛玉倚栏垂钓,宝钗掐了桂蕊,扔在水面,引鱼游来唼喋。
黛玉放下钓竿,走至座间,拿起那乌银梅花自斟壶来,拣了一个小小的海棠冻石蕉叶杯,丫鬟看见,知他要饮酒,忙着走上来斟。
黛玉道:“你们只管吃去,让我自斟,这才有趣儿。
”图左侧:李纨与探春、惜春立于柳阴下看鸥鹭。
}