
%\chapter[脂砚斋重评石头记凡例]{脂砚斋重评石头记凡例\raisebox{.3\baselineskip}{\normalsize\footmark}}
%\foottext{此凡例五条及题诗仅见于甲戌本卷首,退二格抄写。其他各本均无凡例,且均截取第五条“此开卷第一回也”并入第一回作为正文开始。}
\chapter[脂砚斋重评石头记凡例]{脂砚斋重评石头记凡例\foot{此凡例五条及题诗仅见于甲戌本卷首,退二格抄写。
其他各本均无凡例,且均截取第五条“此开卷第一回也”并入第一回作为正文开始。
}}
《红楼梦》旨义\qquad 是书题名极[多,一曰《红楼]\foot{此处原被撕去一角,缺五字。
胡适补“多”“红楼”三字,吴恩裕另校补“一曰”两字。
除“红楼”二字无争议外,前三字所补是否恰当,有不同意见。
}梦》,是总其全部之名也;又曰《风月宝鉴》,是戒妄动风月之情;又曰《石头记》,是自譬石头所记之事也。
此三名皆书中曾已点睛矣。
如宝玉作梦,梦中有曲,名曰《红楼梦十二支》,此则《红楼梦》之点睛。
又如贾瑞病,跛道人持一镜来,上面即錾“风月宝鉴”四字,此则《风月宝鉴》之点睛。
又如道人亲眼见石上大书一篇故事,则系石头所记之往来,此则《石头记》之点睛处。
然此书又名曰《金陵十二钗》,审其名,则必系金陵十二女子也;然通部细搜检去,上中下女子岂止十二人哉!若云其中自有十二个,则又未尝指明白系某某,及至“红楼梦”一回中,亦曾翻出金陵十二钗之簿籍,又有十二支曲可考。
书中凡写长安,在文人笔墨之间,则从古之称;凡愚夫妇、儿女子家常口角,则曰“中京”,是不欲着迹于方向也。
盖天子之邦,亦当以中为尊,特避其东南西北四字样也。
此书只是着意于闺中,故叙闺中之事切,略涉于外事者则简,不得谓其不均也。
此书不敢干涉朝廷,凡有不得不用朝政者,只略用一笔带出,盖实不敢以写儿女之笔墨唐突朝廷之上也,又不得谓其不备。
此书开卷第一回也,作者自云:因曾历过一番梦幻之后,故将真事隐去,而撰此《石头记》一书也。
故曰“甄士隐梦幻识通灵”。
但书中所记何事?又因何而撰是书哉?自云:今风尘碌碌,一事无成,忽念及当日所有之女子,一一细推了去,觉其行止见识皆出于我之上,何堂堂之须眉\zhu{须眉:代指男子。
}诚不若彼一干裙钗?\zhu{裙钗:代指女子。
}实愧则有馀,悔则无益之大无可奈何之日也。
当此时,则自欲将已往所赖——上赖天恩,下承祖德,锦衣纨袴之时,\zhu{锦衣纨绔:富贵者的穿着,引申为富家子弟的代称。
锦:色彩华美的丝织物 。
纨(音“丸”):细绢。
}饫甘餍美之日,\zhu{饫甘餍肥:犹言饱食香甜肥美的食品。
饫(音“玉”)、餍(音“厌”):吃饱吃腻的意思。
}背父母教育之恩,负师兄规训之德,以致今日一事无成、半生潦倒之罪,编述一记,以告普天下人。
虽我之罪固不能免,然闺阁中本自历历有人,万不可因我不肖,\zhu{肖:像。
不肖:子不似父,不能继承父业;不贤,无才能;品性不良。
}则一并使其泯灭也。
虽今日之茅椽蓬牖,\zhu{茅椽蓬牖:代指草房陋室,贫者所居。
茅、蓬都是野草。
椽(音“传”)房椽子(中国古代建筑结构图见页脚\foot{\footPic{中国古代建筑结构图}{house.jpg}{0.8}});牖(音“友”),窗户。
}瓦灶绳床,\zhu{瓦灶绳床:瓦灶为土还烧成的简陋的灶,俗称行灶。
绳床亦名胡床、交床,为一种简易的坐具。
}其风晨月夕,阶柳庭花,亦未有伤于我之襟怀笔墨者。
何为不用假语村言敷演\zhu{敷演:叙述生发。
}出一段故事来,以悦人之耳目哉?故曰“[贾雨村]风尘怀闺秀”,乃是第一回题纲正义也。
开卷即云“风尘怀闺秀”,则知作者本意原为记述当日闺友闺情,并非怨世骂时之书矣。
虽一时有涉于世态,然亦不得不叙者,但非其本旨耳。
阅者切记之。\par
诗曰:\par
\hop
浮生着甚苦奔忙,盛席华筵终散场。\par
悲喜千般同幻渺,古今一梦尽荒唐。\par
谩言红袖啼痕重,更有情痴抱恨长。\par
字字看来皆是血,十年辛苦不寻常。\par
\hop
\zhu{《甲戌本凡例》,其他各本均无,且均从“此开卷第一回也”处作为正文开始。
此凡例的真伪在红学界虽有争议,但其非曹雪芹亲作,却得到公认。
\hang
值得注意的是,红学界曾为本书诸异名先后、哪个是曹雪芹原定名而争执。
从本书凡例看,首先说“红楼梦旨义”,不说“石头记(或其他名)旨义”。
接下去解释书名极多,也是第一个提到《红楼梦》,且说此名“是总其全部之名也”。
看来,此书初稿时是题名极多,但最后甲戌年定稿时作者已经定名《红楼梦》。
而“脂砚斋重评石头记”这个名字则是脂砚选用的。
第一回“至脂砚斋甲戌抄阅再评,仍用《石头记》”一句可证。
\hang
当然,这里只是想弄清哪个书名最符合作者本意,并不是像某些红学家那样,上纲上线,说是脂砚斋恶意把一个充满批判精神的书名《红楼梦》改成无关痛痒的《石头记》,体现脂砚思想的反动本质云云。
}
